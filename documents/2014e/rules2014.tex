\documentclass[a4paper]{article}

\usepackage{dq2014}

\title{Rulebook for Players and Game Masters}

\author{2014 Revision\\Edition 2.0.1\\September 2014\\Alexander et al.}

\date{\thisday}

\begin{document}

Contributors

Ross Alexander • Jim Arona • Jono Bean • Craig Beere • Errol Cavit • Martin Dickson • Daniel Dixon • Jacqui Dunford-Smith • William Dymock • Dean Ellis • Mark Harrison • Michael Haycock • Bryan Holden • Brent Jackson • Gary Jackson • Bart Janssen • Struan Judd • Phil Judd • Rosemary Mansfield • Stephen Martin • Jon McSpadden • George Mitchenson • Michael Parkinson • Carl Reynolds • Lisa Rose • Paul Schmidt • Keith Smith • Terry Spencer • Ben Tabener • Adam Tennant • Sue Turner • Clare West • Andrew Withy • Ian Wood • Michael Young • Kelsie McArthur

Copyright

DragonQuest is copyright to Simulations Publications, Inc. First
Edition, Copyright© 1980. Second Edition, Copyright©1981.  The named
contributors assert their right to this work. This rule book is
copylefted under the GNU Public License, version 2.0.

\begin{multicols}{3}

\section{Character Generation}

1 Character Generation
There are six sections in Character Generation:
1.1 Characteristic Points
1.2 Race
1.3 Description
1.4 Aspect
1.5 Heritage
1.6 Starting Abilities \& Possessions

Sections 1.1 – 1.5 may be done in any order. Each
section is designed so that a player may choose
from a range of options or randomly generate their
character. Section 1.6 should be done last.

\subsection{Characteristic Points}

A character has 6 primary statistics which are
generated by allocating points from a total, and 4
secondary statistics which are either derived from
the primary statistics or are generated randomly.
The higher the number, the better the characteristic.

\subsubsectioon{Generating Characteristic Points}

The player may choose to allocate the primary statistics from a total
of 90 points or may roll 2D10 once against the following table. If
they choose to roll the result must stand.

Die Roll

Points Total

2
3
4
5
6
7
8
9
10
11
12
13
14
15
16
17
18
19
20

81
82
83
84
85
86
87
88
89
90 (default choice)
91
92
93
94
95
96
97
98
99

Assigning Characteristic Points
This total of points needs to be spent on the following characteristics: Physical Strength, Manual
Dexterity, Agility, Magical Aptitude, Willpower \&
Endurance. These characteristics may change
during the game, and may be raised up to 5 points
through training, though not past the character’s
racial maximum.
The human range for each of these characteristics
is 5 – 25; this range is adjusted for non-humans
(see the Characteristic Modifier tables for the nonhuman races). These ranges represent the minimum
and maximum capabilities of the races. The player
should assign the points and then make any adjustment for race.
Prior to assigning the characteristic points, the
player should give some thought to what kind of
character they wish to have and what weapons,
spells, and/or skills are desired for the newly created individual. Some weapons require a great deal
of Physical Strength or Manual Dexterity, and the
player should be sure to assign enough points in
those areas to use the weapons of their choice. All
magical colleges require a minimum Magic Aptitude to join and the player should be aware of these
restrictions. Most skills do not have any special
requirements, but many give bonuses for exceeding
a minimum value in certain characteristics.
When the player has chosen the values for the
character, they must record them on a Character
Sheet. The total value of the six primary characteristics (before racial modifiers) must equal the
amount received in the Generating Characteristic
Points section; thus, a player cannot “save” Characteristic Points and assign them to characteristics

at a later date. The value of each of the six primary
characteristics must be recorded before any secondary characteristics are generated.
Generating Secondary Characteristics
Fatigue, Physical Beauty, Perception and Tactical
Movement Rate are secondary characteristics.
They may be modified if the character is nonhuman (see the Characteristic Modifier tables for
the non-human races).
Fatigue
The value of a character’s Fatigue is a direct function of their Endurance. The player enters the Fatigue value corresponding to the character’s Endurance value after their Endurance has been modified
for race.
Endurance Fatigue
3 or 4
16
5 to 7
17
8 to 10
18
11 to 13
19
14 to 16
20
17 to 19
21
20 to 22
22
23 to 25
23
26 to 27
24

Endurance and Fatigue values in bold type can be
achieved only by members of certain non-human
races.

From this point on, a change in a character’s Endurance value will not
affect their Fatigue value and vice-versa. Fatigue may be raised by up
to 5 points, though not past the character’s racial maximum.

Physical Beauty

The value of the Physical Beauty characteristic is generated randomly
by rolling 4D5 + 3. This characteristic can never be increased by
training.

Perception

A character’s perception value begins at 5. This may be trained up to
racial maximum.

Tactical Movement Rate

A character’s Tactical Movement Rate (TMR) is a direct function of
their Agility. It is based on the character’s Agility value and is
recalculated when Agility is modified by encumbrance and armour
penalties; see the TMR table (§58.2) for values.

\subsection{Race}

A player must choose the race of their character.  The majority of
people in Alusia are human, but the player may choose one of the
common nonhuman races: dwarf, elf, halfling, or orc.

If the player wishes their character to be a giant or shapechanger
they must roll D100. They may roll once per race and if the roll is
lower than the race chance \% they must take that race. If they fail
then the character must be of one of the common races.

If the player is attempting to be a shapechanger they must decide what
type of shapechanger they want prior to rolling (i.e. wolf, tiger,
bear or boar).  Race

Chance (%)

Hill Giant
06
Shapechanger 04

A player may wish to play one of the very rare sentient races. To do
so they must get the agreement of both the generating GM and a member
of the character tribunal. They will decide which of the common races
has the appropriate racial modifiers. For example Erelheine characters
are generated using the Elf option.  Humans learn faster than
non-humans. Learning is represented in game by spending Experience
Points (EP). Divide any experience points a character gains by the
“racial modifier” and then spend the result normally.

5

Race

Modifier

Dwarf
Elf
Halfling
Hill Giant
Human
Orc
Shapechanger

1.1
1.2
1.1
1.5
1.0
1.1
1.4

For every 25,000 Experience Points (EP) the character has spent towards the 'racial modifier' it is
lowered by .1 (but not below 1.0) or after 20 adventures (when PC reaches racial max), the EM
becomes 1 (whichever happens first). E.g. once a
giant has lost 25,000 EP to their race, their 'racial
modifier' is lowered to 1.4. Once they spend an
additional 25,000 EP on their new 'racial modifier'
of 1.4 it would become 1.3.
RM

Amt earned
that cost 25k

Amt spent
that cost 25k

1.5
1.4
1.3
1.2
1.1

75,000
87,500
108,333
150,000
275,000

50,000
62,500
83,333
125,000
250,000

Each race has a description of a stereotypical
member of the race and any special abilities and
characteristic modifiers that apply to a character of that race.
Dwarf
A dwarf is a short, bearded humanoid, usually
taciturn who frequents mountainous areas.
Description: Pride and attention to detail are important to dwarves. They form strong community
ties, and are distrustful of strangers, especially
those of other races. Their strongest antipathies are
towards orcs and elves. Although dwarves are
greedy by nature, they are essentially honest and
stand by their word. Dwarves covet precious stones
and metals, and appreciate fine, detailed workmanship. Dwarven warriors favour the axe as weapon.
Special Abilities
1. Dwarves’ close vision is exceptionally sharp, but
many have poor distance vision. They can see in
the dark as a human does at dusk. Their effective
range of vision in the dark is 50 feet under the open
sky, 100 feet inside manmade structures, and 150
feet inside caves and tunnels.
2. Dwarves can assess the value of and deal in
gems and metals as if they are a Merchant of Rank
5. If a dwarf character progresses in the Merchant
skill, their ability to assess the value of gems and
metals is five greater than their current Rank, to a
maximum of ten.
3. If a dwarf character is a Ranger specialising in
mountains or caverns, they pay half the EP cost
necessary to advance ranks.
4. A dwarf’s capacity for alcohol is twice that of a
human.
Characteristic

Modifier

Physical Strength
Agility
Endurance
Magical Aptitude
Willpower
Perception
Physical Beauty
Tactical Movement Rate
Starting Age:
Average Life Span:

+2
-2
+2
-2
+2
+1
-2
-1
20 +
125 – 150 years

Elf
An elf is a slim agile humanoid, who frequents
wooded areas.
Description: Elves are virtually immortal and generally take the long term view. They are insular,

1 CHARACTER GENERATION
indifferent to others and tend to be traditional.
Elves are great respecters of nature and learning.
Their Elders are repositories of great wisdom while
elvish youth are enthusiastic merry makers. Elven
warriors favour bow weapons and disdain metal
armour. Members of other races generally find
elves attractive.
Special Abilities
1. Elves have superior vision especially over long
distances or in poor lighting. An elf can see in the
dark as a human does on a cloudy day. Their effective range of vision in the dark is 150 feet under
the open sky, and 75 feet elsewhere.
2. If an elf character is a ranger specialising in
woods, they pay one-half the EP to advance ranks.
3. An elf receives a racial Talent which functions
in all respects as the Witchcraft Witchsight Talent.

Hill Giant
A hill giant is a huge, coarse featured humanoid,
who has no patience for laborious learning.
Description: Giants are lusty types, preferring
nothing better than to go through life brawling,
drinking, and wenching. They tend to gather together in a clan arrangement, building huge halls
(or steadings) in out-of-the-way locations. They are
not overly intelligent, and resent humans and elves
particularly. Giants enjoy riddling and bartering.
Giant warriors favour simple weapons scaled to
their size.
Special Abilities
1. A giant has infravision, which allows them to
see faint red shapes where living beings are located
in the dark. Their range of vision is 250 feet.
2. A giant’s magic resistance is increased by 10%.

4. An elf makes little or no noise while walking
and adds 10% to their chance to perform any activity requiring stealth.

3. Whenever a giant attempts minor magic, the GM
should increase the difficulty factor by one, making
it easier.

5. If an elf character takes the healer skill, the elf
pays three-quarters the EP to advance ranks,
though they cannot resurrect the dead.

4. Giants may use the giant weapons listed in the
Weapons Table (§56.1 ).

6. An elf is impervious to the special abilities of
the lesser undead.
7. If an elf character takes the courtier skill, the elf
pays one-half the EP to advance ranks.
Characteristic

Modifier

Physical Strength
Agility
Endurance
Magical Aptitude
Willpower
Perception
Physical Beauty
Tactical Movement Rate
Starting Age
Average Life Span

-1
+1
-1
+1
+1
+1
+2
+1
30 – 300 +
Circa 10,000 years

Halfling
A halfling is a short, cheerful humanoid, who
will be an active participant in village life.
Description: Halflings appreciate the good life
more than most; a successful halfling will arrange a
schedule of much sleep, good food, and relaxed
study or conversation. Halflings are shy around
other races, preferring to merge into the background. Amongst themselves they are a friendly
folk who form into small communities where everyone knows everyone else’s business. While
Halflings take their social responsibilities seriously, they are renowned for their practical jokes
and light fingers. Halflings are noted for their
tough, hairy feet and usually go barefoot. Halflings
avoid the rigours of military life but when forced to
defend themselves they favour small weapons.

Characteristic

Modifier

Physical Strength
Manual Dexterity
Agility
Endurance
Magical Aptitude
Willpower
Fatigue
Physical Beauty
Tactical Movement Rate
Natural Armour
Starting Age
Average Life Span

+7
-1
-2
+8
-1
-1
+1
-1
+3
+1
26 +
500 years

Human
Humans are by far the most common race on
Alusia, frequenting most areas and climes.
Description: Humans have a great diversity of
cultures, languages and sub-racial traits, such as
hair and eye colour or skin tone. Human behaviour
is an odd mix. They can be superstitious and distrustful of the unknown, but they are also insatiably
curious and look for new knowledge. Many also
seek personal fame and fortune as most human
social structures are less rigid than those of nonhumans and a person’s birth need not permanently
define their place in society. This odd combination
of attributes has led them to become great explorers and sailors, and they will venture boldly into
unexplored areas in search of knowledge and
wealth. Humans build great cities and are far more
welcoming of other races than most. Outside of
their own culture they are social chameleons, adept
at adapting their behaviour to match local customs.

Special Abilities
1. A halfling has infravision, which allows them to
see faint red shapes where living beings are located
in the dark. Their range of vision is 100 feet.

Special Abilities
1. Humans can ingratiate themselves with strangers
more readily than other races. A human character
has +10 to any reaction roll in an encounter with
sentient creatures.

2. A halfling adds 20% to their chance to perform
any activity requiring stealth.

Characteristic

Modifier

Starting Age
Average Life Span (varies
widely with wealth and culture)

16 +
40 – 90 years

3. If a halfling takes the thief skill, they pay half
the EP cost to advance ranks.
4. A halfling may drop jewellery down active
volcanoes without anyone thinking the worse of
them.
Characteristic

Modifier

Physical Strength
Manual Dexterity
Agility
Endurance
Magical Aptitude
Willpower
Physical Beauty
Starting Age
Average Life Span

-3
+3
+1
-2
-1
+1
-1
21 +
80 – 90 years

Orc
An Orc is a stoop-shouldered, surly humanoid
and a pack member by nature.
Description: Orcs are a cruel, violent folk, liking
nothing better than to loot and pillage. Individuals
test themselves against their peers, bullying anything weaker but cowering away from anything
stronger. A strong individual will form a pack
around them, and the pack leader’s word is law.
Orcs enjoy the sensual pleasures of life, and reduce
their already short life spans through hard living.
They have a robust digestion and will eat foods
that others turn their nose up at. Orc warriors fa6

vour the great axe and glaive. Orcs are considered
unattractive by other humanoid races.
Special Abilities
1. An orc’s eyes are highly light-sensitive. They
must decrease their chance of hitting a target with
Ranged Combat by 10% in daylight.
2. An orc has infravision, which allows them to see
faint red shapes where living beings are located in
the dark. Their range of vision is 150 feet.
3. Orcs are either back-stabbing scum or brutal
bully-boys. An orc may take one of either Assassin
Skill or Warrior Skill and pay three-quarters the EP
to advance in Ranks.
4. An orc’s seed is highly fertile. The orc and hybrid orc population increase mitigates against the
high orc fatality rate.
Characteristic

Modifier

Physical Strength
Endurance
Magical Aptitude
Willpower
Fatigue
Physical Beauty
Natural Armour
Starting Age
Average Life Span

+2
+1
-2
-2
+2
-4
+1
12 +
40 – 45 years

Shapechanger
Shapechangers are a hidden race amongst humans, with the ability to change into the form of
a particular animal.
Description: Shapechangers are identical in appearance to humans when not in animal form. They
are somewhat bestial in nature, adopting traits one
might expect from an anthropomorphised wolf,
tiger, bear or boar. There exists a love/hate relationship between humans and shapechangers:
shapechangers possess some degree of animal
magnetism, but, if discovered, can expect severe
treatment at the hands of humans. Shapechangers
are, on the whole, bitter towards humans, and are
not above using humans to their advantage. There
are very few ways to tell a shapechanger from a
human (e.g. they will be discomforted by wolfbane) and these vary by shapechanger type.
Shapechangers are a ruthless lot.
Special Abilities
1. A shapechanger can change from human to
animal form (or vice-versa) in 10 seconds during
daytime and 5 seconds during the night-time.
2. A shapechanger possesses a dual nature. While
in animal form, human inhibitions will be muted;
while in human form, animal instincts will be
dulled.
3. A shapechanger cannot be harmed while in
animal form, unless struck by a silvered weapon,
magic or by a being with a Physical Strength
greater than 25. Five Damage Points are automatically absorbed in the latter case.
4. A shapechanger will regenerate 1 Endurance
Point every 60 seconds while in animal form.
5. The player must devise a set of characteristics
for their animal form. Take the difference between
the average for each characteristic in animal and
human form, and modify the human characteristics
appropriately.
6. A shapechanger is automatically lunar aspected.
7. A shapechanger can remain in animal form for a
quarter of the night times the quarters of the moon
showing (i.e. at full moon they may remain in
animal form all night). During the day a
shapechanger can remain in animal form for one
hour times the quarter of the moon. A
shapechanger can make one set of transformations
times the quarter of the of the moon per day (i.e.
dawn to next dawn).
8. If a shapechanger is in animal form during the
day, there is a 1% cumulative chance for each 5

1 CHARACTER GENERATION
minutes they remain in animal form that they will
never be able to change back into human form.
Similarly, if the shapechanger exceeds the time
limits given above, there is a 1% cumulative
chance (per 5 minutes) of their not being able to
return to human form.

Optionally, some characteristics may be adjusted
for a female character. This would also modify her
appropriate racial maximums.

9. A shapechanger will be inconvenienced by those
wards which can be used against were-creatures.

Physical Strength
Manual Dexterity
Endurance

10. A shapechanger’s magic resistance is increased
by 5%.
11. If a shapechanger takes the courtier skill they
pay three-quarters the Experience Points necessary
to advance ranks.
Characteristic

Modifier

Physical Beauty
+1
Starting Age
16 +
Average Life Span 55 – 65 years
A separate set of characteristics must be generated
for the animal form (see Ability 5 above).

1.3 Description
This section covers height, weight, gender, primary
hand, and general description.
Height and Weight
A player should choose their character’s height and
weight. The character’s height and weight should
be chosen according to the player’s idea of the
character, with due regard to the character’s primary characteristics, race and background.
The following charts give a range of heights and
weights within which 90% of adventurers fall, and
the average values within that range. Please modify
your chosen height and weight according to gender
and racial adjustments as below.
Normal Base
Height

Weight

Range

5’3"
5’6"
5’9"
6’0"
6’3"
Adjustments

130
140
150
165
180
Height

100–170
110–185
120–200
130–220
145–240
Weight

Human Male
Human Female
Orc Male
Orc Female
Elf Male
Elf Female
Short Folk Base

+0"
-4"
-4"
-6"
+5"
+2"

100%
80%
110%
100%
80%
65%

Female Characteristic Modifier
Characteristic
Modifier

They may choose either right or left, or roll randomly. If they choose to roll, the result must stand.
The player rolls D5 and D10. If the D10 result is
greater, the character’s right hand is primary. If the
D5 result is higher, their left hand is primary. If the
two results are equal, the character is ambidextrous.
Description
The player will sometimes need to describe their
character and should therefore think about the
character’s physical appearance based on the generated characteristics. They should choose hair, eye
and skin colour (based on race and family background).

1.4 Aspects
The timing of a character’s birth orients them
towards one of several astrological influences, or
aspects. A character will benefit during the time
their aspect is powerful, and will suffer when the
opposite aspect is powerful.
The times of high noon and midnight are extremely
important when applying the effects of aspects.
The GM should allow characters to perform actions at precisely those instants, though the passage
of time must be properly monitored.
Generating an Aspect
The player may choose an Aspect as if they had
rolled any number up to 80, or roll D100 once
against the following table. If they choose to roll
on the table, any roll over 80 may be re-rolled.
If the character is joining one of the elemental
colleges the player may choose any aspect between
1 and 80 that is neutral to their college, or they may
roll.

Height

Weight

Range

3’9"
4’0"
4’3"
4’6"
4’9"
Adjustments

85
95
105
115
125
Height

65-110
75-125
85-140
95-155
105-170
Weight

Dwarf Male
Dwarf Female
Halfling Male
Halfling Female
Hill Giants Base

+0"
-2"
-12"
-13"

100%
90%
65%
60%

01–05
06–10
11–15
16–20
21–25
26–30
31–35
36–40
41–45
46–50
51–55
56–60
61–65
66–70
71–75
76–80
81–85
86–90
91–95
96-00

Weight

Range

8’4"
8’8"
9’0"
9’4"
9’8"
Adjustments

370
420
470
525
580
Height

295–490
335–555
375–625
420–700
465–780
Weight

Giant Male
Giant Female

+0"
-4"

100%
90%

Gender
A player may choose whether their character is
male or female. It is recommended the character be
the same gender as the player, as playing the opposite gender convincingly is difficult.

Aspect
Winter Air
Winter Water
Winter Earth
Winter Fire
Spring Air
Spring Water
Spring Earth
Spring Fire
Summer Air
Summer Water
Summer Earth
Summer Fire
Autumn Air
Autumn Water
Autumn Earth
Autumn Fire
Solar
Lunar
Life
Death

Seasonal Aspects
A character is affected by their seasonal aspect
during their aspect’s season and the opposite season. The following table lists the seasonal aspect
effects and when they apply.
Time

Primary Hand
A player must determine whether their character’s
Primary Hand is their right or their left. This determination affects which hand a weapon is held
during combat, and any penalties assigned for
attacking with a weapon in a secondary hand.

Die

Height

-2
+1
+1

poses Water. Ice and Celestial magic is not affected.

Effect

Midnight, Aspect’s Season
-10
Midnight, Equinox or Solstice of As-25
pect’s Season
Midnight, Opposite Season
+10
Midnight, Equinox or Solstice of Oppo+25
site Season
The effect is applied for 30 seconds before and
after midnight.
Solar and Lunar Aspects
A character of solar or lunar aspect is affected by
their aspect at high noon and midnight. The following table lists the Solar aspect effects, and when to
apply them.
Time

Effect

Noon
-5
Midnight
+5
Noon, Summer Solstice
-25
Midnight, Winter Solstice +25
Lunar aspected characters gain opposite bonuses
and penalties for the same times. The effect is
applied for 10 seconds before and after high noon
or midnight. If the sky is cloudy, the effect may be
reduced to a minimum of +/1 and 5.
Life and Death Aspects
Life and Death aspected people are affected by the
creation and destruction of life force.
The following table lists the Death aspect effects,
and when to apply them.
Event

Range

Aspect

Effect

Birth of mammal
Birth of humanoid
Birth of close relative†
Death of mammal
Death of humanoid
Death of close relative†
†A close relative is no
cousin.

100’
250’
500’

Death
Death
Death

+5
+10
+25

50’
125’
250’

Death
Death
Death

-5
-10
-25

more distant than a second

Life aspected characters gain opposite bonuses and
penalties for the same times. Deaths are noncumulative (only one can be in effect at a given
time), though births are cumulative. A stillbirth
does not affect a life or death aspected character. A
resurrection is treated as a birth.
A death event is applied for as many seconds as the
effect range in feet. A life event is applied for 3
times as long.
A female life aspected character will suffer no pain
after giving birth, and will be as healthy and active
as she was before she became pregnant.
Light and Dark Aspects
All living creatures have an additional celestial
Light or Dark Aspect. This is fully explained in an
addendum to the College of Celestial Magics
(§19.8).

1.5 Heritage
This section is relevant to humans, primarily from
the Western Kingdom and Cazarla, and should be
adapted for other races or regions.

Effects of Aspects
Apart from elemental aspects, all modifiers apply
to percentile rolls, not base chances.
Elemental Aspects
Characters gain a bonus of 1% on the Base Chance
of performing any magic of the same College as
their elemental aspect, and a penalty of -1% on the
opposed College. Air opposes Earth and Fire op7

Social Status
The “social status” is that of the character’s parent(s), usually the father. The table does not represent the population, merely the proportion of backgrounds from which accepted Adventurer’s Guild
applicants originally come. Most social classes are
present in a variety of environments (city,
town/village, rural, court, castle/stronghold, maritime). A player may choose any social category in

1 CHARACTER GENERATION
the 01-80 range for the character’s background or
roll; however, any such dice-roll must be accepted.
In general, the higher the number rolled, the higher
the social status within each band. A roll of 40-90
optionally may indicate a respectably retired exadventurer.
Die

Social Status

01–14
15–20
21–29
30–44
45–54
55–70
71–84
85–94
95–98
99–00

Trash / Criminal
Bonded
Skilled retainer
Goodman
Master
Military
Gentry
Lesser Noble
Merchant-prince
Greater Noble

Explanation of Classes
Trash/Criminal No legitimate employment.
Example

Thug, body-snatcher, bandit, pirate, beggar.

Bonded There is no slavery in the Baronies. This is
the next best thing: enforced servitude to one master for a long period [up to life], through birth,
contract, or debt.
Example
Serf, villain, unskilled or semi-skilled
servant, labourer, indentured apprentice or journeyman in
a craft or trade guild, dependent artisan contracted to a
master, lay member of an accepted religious community,
ordinary soldier or sailor.

Skilled Retainer Voluntarily employed person,
physically and legally capable of seeking a position
elsewhere. Owns the tools of the trade and has
other, limited possessions. Usually works under the
direction of a goodman or master. Occasionally an
itinerant artisan of low status.
Example
Clerk, court musician, religious acolyte,
freeborn shepherd or farm hand, merchant’s assistant,
family chaplain, tinker, fisher.

Goodman [Goodwife, Goody] Head of a household: more possessions and commitments than a
mere retainer, comparatively independent. Usually
leases or owns a smallholding (if in the countryside) or a few rooms (if in a town). Much contact
with social peers and superiors. Often employs
skilled retainers. Includes itinerant professionals
and artisans of high status.
Example
Miller, pilot, established artisan, minor
trader, innkeeper, accredited witch, priest in an accepted
temple, shop owner, poor freeman-farmer, forester, gamekeeper, itinerant or privately employed alchemist, healer,
magician or blacksmith.

Master: [Mistress, Mother] Like a goodman, but
with a larger establishment, more employees, more
commitments to subordinates and equals. Tied to
one place as direct contact with, and obligations to,
social superiors and Guilds may make impolitic
any relocation or other changes in social conduct,
despite theoretical liberties and rights.
Example
Guild master of a smaller craft/trading
guild, or councillor in a more powerful one, wealthy
freeman farmer, professional (alchemist, healer etc) trading publicly, with own shop and apprentices, Alphonse the
famous chef, a Ducal Kapellmeister, high-priest of an
accepted temple, captain-owner of a trade ship, mayor of a
medium town.

Military A socially sanctioned, trained fighter or
skilled ancillary. This includes sergeants and lowborn lesser officers (lieutenants, etc); high-ranking
officers are ex officio gentry.
Example
Town guardsman, skilled scout or military
spy, army blacksmith, (legal) mercenary captain.

Gentry By birth or service entitled to a coat of
arms: significant social or military duties. There
are often many social gradations of gentry not
comprehensible to persons outside that class. Often
possesses an estate or “independent means” but is
not of lordly rank; such persons may, technically,
be employed (but usually to a lord, or in service to
their country). May have difficulty ensuring all
children have an acceptable start in society (especially in larger families).

Example
Knight, country squire, beneficed parson,
port-reeve, courtier of significance, respected \& influential
magician, judicial officer of a town or district, tax farmer,
non-noble army or navy officer (generally Captain \& up),
cadet member of a noble family.

Lesser Noble Of lordly rank. Similar to the gentry,
but definitely a cut above. Normally owes feudal
service to, or through, a greater noble.
Example
Non-independent Baron, Lord Admiral of a
small navy, General, ordinary Abbot or other Head of an
established, accepted, religious house, former gentry
ennobled for extraordinary or personal services to a great
noble or royalty.

Merchant-Prince Extremely wealthy city-based
merchant, head of an extended trade/family. Controls a nationally significant trade-empire and / or
monopoly. Has significant power in the local
guilds. Extensive resources (especially in his/her
home city), with contacts and enemies in several
countries. Capable of ordering actions deemed
criminal in less influential personages. On a roll of
98, the family head is the character’s parent; on
95–97, the head is a little more distant (perhaps
uncle or cousin).
Example
Owner of a trade-fleet, trader with a national monopoly on a commodity (e.g. silk, wine), Guild
master of a powerful guild.

Greater Noble Ruler of a minor country, or head
of one of the “Great” families in a larger country.
Will have several estates and titles. Usually has
subordinates of lordly rank. Children may have
courtesy titles.
Example
An independent Baron, Marshall of a
Duke’s or independent Count’s armies, Bishop, Abbot of a
mother-abbey, Marshall or vicar-general of a powerful
order, Count within a duchy, Lord Admiral of a maritime
nation.

Greater Noble and Merchant-Prince families impact seriously on the campaign; the generating GM
may need time to consult with other GMs before
the character’s background is finalised. Characters
who wish to retain an acknowledged, good socialstanding may have to devote time and money to
maintain their position by indulging in appropriate
behaviour - noblesse oblige.

their fellow adventurers to unnecessary risks arising from their backgrounds.
In most cases, achievement begets amnesty. A serf
who has spent a year and a day in a town becomes
a freeman; a now wealthy prodigal is welcomed
back into the family fold.

1.6 Starting Abilities and Possessions
This section covers abilities and possessions a
character has prior to starting life as an adventurer.
None of the experience points awarded in this
section are adjusted by any racial experience modifiers but the player must use their character’s race
and heritage as a guideline to the allocation or
choices they make. Except where noted, the normal
acquisition and ranking rules apply to the spending
of experience points. This section must be started
after all other sections are completed, and each
sub-section must be completed in order.
Language Skills
Every character knows their native language, the
Alusian trade language (Common) and possibly
another language. A Guild member will be literate
in at least one language and literacy is required to
learn magic.
The player should get the GM’s assistance to determine what their character’s native language is
and then choose one of the following options for
their starting language skills:
• Option A Rank 8 and literate in either native
language or common, Rank 6 in the other of native
language or common, Rank 4 in any other common
language.
• Option B Rank 8 and literate in either native
language or common, Rank 7 and literate in the
other of native language or common, Rank 1 in any
other language.
• Option C Rank 9 and literate in either native
language or common, Rank 6 and literate in the
other of native language or common.

Birth Order
Players should now choose their birth order, or roll
on the following table. Note that it is unlikely that
an heir will go adventuring (at least not without
active encouragement from the next-in-line).

Adventuring Skills
A character starts with Rank 0 in the Adventuring
skills of Horsemanship, Climbing, Swimming and
Stealth. The player now receives 1250 experience
points that may be spent on improving these skills.
Any experience points left over are lost. They also
gain Rank 0 Flying, but may not raise it at any
stage during Character Generation.

Die

Birth Order

The possible combinations are:

1
2–3
4
5
6
7–8
9
0

1st or 2nd
3rd
4th
5th
6th
7th
8th or later
bastard

Horsemanship,
Climbing and
Swimming

Disinheritance
Beginning characters never start with an estate,
magic possessions, or other “real” wealth. For
game reasons, characters seldom inherit while
actively adventuring. Most classes will happily
pass over an adventurer in favour of more deserving and capable stay-at-home siblings. If the heir or
heiress cannot be passed over (e.g. a noble estate)
and the player does not wish to retire the character,
a trusted kinsman or tenant must be appointed as
trustee or warder, to administer and enjoy the inheritance until it is reclaimed.
A noble or wealthy parent may disown adventuring
children either through disfavour, or for mutual
protection. A beginning character doesn’t want to
be set upon by family enemies, and no parent
wants the social stigma of refusing to pay a ransom. The guild fully supports such characters
adventuring under an alias, just as it also supports
gifted adventurers who fled legal restraints in order
to join the guild (e.g. a runaway serf turned mage).
Both classes do have the obligation not to expose
8

(in any order)

Stealth

4, 0, 0
3, 0, 0
3, 2, 1
2, 2, 0
2, 2, 2

0
1
0
1
0

Mage or Non Mage?
The player must decide whether the character will
be a magic user or not. (This choice can be made at
any time during character generation).
Mage
If the character is to be a magic user then the
player must choose a college of magic for the
character to belong to. Remember that there is a
minimum Magical Aptitude requirement for each
college.
College

MA

Naming Incantations
Mind
Fire
Air
Ice
Illusion
Celestial
Earth
Bardic

1
11
12
13
13
13
14
15
16

1 CHARACTER GENERATION
E\&E
16
Necromancy
16
Binding \& Animating 17
Water
18
Witchcraft
18
The character now receives all of the general
knowledge abilities of their college including talents, general knowledge spells, general knowledge
rituals, both counterspells, the purification ritual
and ritual spell preparation.
The player should list these on their character
sheet.
Non Mage
If a player decides that their character will not be a
magic user then they receive 6500 experience
points to be spent in the following order:
1. 2500 must be spent on either 1 point of Fatigue
or 3 points of Perception.
2. The character must acquire one new skill at rank
2, and may acquire a second new skill at no more
than rank 1. The Warrior skill may not be chosen at
this time.

3. The character must acquire one weapon at rank
2, and may acquire up to two weapons at no more
than rank 1.
4. The player may save up to 500 points to spend
later. The player must spend any remaining points
on any of:
• 1 rank in any known adventuring skills
• more ranks in any known languages
• more perception.
Any remaining points (other than the permissible
500) are lost.
Background Experience
A character now chooses any one Artisan skill at
Rank 0. This reflects knowledge gained through
childhood and must be appropriate to their family
background.
They also receive 2500 Experience Points which,
together with any left over from the non-mage
generation, can be spent freely.
At this time the character may acquire any one new
skill at Rank 0 for the cost of only 100 EP (rather
than the usual cost).

9

If there is any EP remaining it may be saved for
spending later in the game.
Background Possessions
The character will have two sets of clothing of a
quality appropriate to their family background.
They also have goods up to the value of 500 sp
which may be chosen from the Basic Price List in
the Players Guide. Up to 50 sp may be saved as
cash.
Modified Agility and Manual Dexterity
The player should calculate any agility modifiers
from the weight of their possessions and any armour penalties; see the Encumbrance Table (§58.1)
for values. They should then calculate their modified TMR from this value, see the table (§58.2) for
values. If the character uses a shield, they should
modify their Manual Dexterity as well.
Finishing the Character
The player must choose a name for their character.
They should enter every piece of relevant data onto
their Character Sheet, and calculate base chances
and other variables. The generating GM will check
it, and then sign \& date it as complete.

2 EXPLANATION OF CHARACTERISTICS

2 Explanation of Characteristics
This section is an explanation of a character’s
characteristics and how they are used in the game.
All characteristics are calculated when the character is generated but Adventurers in a world of
magic can expect them to change from time to
time. A “temporary” change indicates an increase
or decrease of limited duration to the value of a
characteristic; a “permanent” change indicates an
increase or decrease of indefinite duration to the
value of a characteristic.
The first six characteristics are the primary characteristics. These can be increased temporarily by
magic or permanently by training (expenditure of
experience points), and can be decreased temporarily by magic or injury, or permanently by injury to
the character. These primary characteristics can
never be trained more than 5 above their starting
value, and never above racial maximum, except by
unusual magical means.
All other characteristics are secondary characteristics. The manner in which a secondary characteristic can be changed will be covered in the appropriate explanation.
Generally, a high characteristic value indicates a
character’s ability to perform a certain task well,
while a low value indicates a relative lack of such
ability. A characteristic’s effect is almost always
translated into numerical terms for the purposes of
resolving action during play. Adventurers generally
have higher characteristics than normal people in
the world, that is what makes them heroes after all.
Effects of Characteristics
A character develops specific skills during the
game, and their characteristics influence their base
chances with these skills. However there are also
many feasible tasks that a character may wish to
perform without having a specific skill to do so.
The GM then uses the most appropriate characteristic to generate a base chance to perform that task.
Difficulty Factors (Characteristic Multipliers)
When a player declares that their character will
attempt a task which the GM acknowledges as
dependent upon a particular characteristic, the GM
assigns the task a difficulty factor. This difficulty
factor will be a number from 1/2 through to 5.
The greater the difficulty factor value, the easier a
task will be to perform.
The player multiplies the difficulty factor by the
appropriate characteristic, arriving at the percentage chance of the character performing the task.
The maximum base chance is (70 + characteristic +
difficulty factor)%. The player then rolls D100,
and if the roll is less than or equal to the percentage
then the character has successfully performed the
task. If the roll is greater than the percentage, the
character has failed. If the roll fails by at least the
value of the characteristic or exceeds the maximum
base chance, the character has failed miserably and
may have injured themselves. The GM may wish
to determine the extent of the injury by how much
the roll exceeds the percentage plus the characteristic.

• Lifting heavy or awkward objects
Example
Consider the sturdiness of the object and
the implement being used to break it for the former, and
consider the weight and bulk of the object plus the purchase afforded the character for the latter.

2.2 Manual Dexterity (MD)
Manual Dexterity is a measure of a character’s
control with their hands. The Manual Dexterity
characteristic represents the character’s hand-toeye co-ordination, the speed at which the character
can perform a complex task with their hands, and
the ability to manipulate their hands.
Specific Influences
• Minimum MD requirement for weapons
• Thievery
• Strike Chance in Combat
Generic Uses
• Handling dangerous substances
• Fine handicrafts \& other delicate tasks.
Example
Consider the delicacy of the task when a
character seeks the careful manipulation or removal of an
object.

2.3 Agility (AG)
Agility is a measure of a character’s ability to
manoeuvre their whole body and their speed of
movement. The Agility characteristic represents
the character’s litheness of body, the speed at
which the character can run, and their ability to
dodge with or contort their body.
Specific Influences
• Tactical Movement Rate
• Defence

2.6 Endurance (EN)
Endurance is a measure of the punishment a character’s body can absorb before the character becomes unconscious, sustains mortal wounds, or
dies. The Endurance characteristic represents the
character’s capacity to withstand wounds, their
resistance to disease and infection and their rate of
recovery from same, and directly affects their
ability to over-exert themselves.
Specific Influences
• Starting Fatigue
• Damage capacity
• Stunning from damage
Generic Uses
• Resisting poison, infection \& disease

2.7 Fatigue (FT)
Fatigue is a measure of a character’s physical and
mental fitness. The Fatigue characteristic represents the degree to which the character can exert
themselves before becoming exhausted, the number of minor cuts and bruises they can take before
their abilities are affected, and the mental energy
that can be used to cast spells. This characteristic
directly reflects a character’s current level of tiredness as it is reduced temporarily with any strenuous
activity and restored to normal with rest. Fatigue
may be permanently increased by training up to 5
points or to racial maximum.

• Minor damage capacity

• Most physical skills

• Spell casting energy

Generic Uses
• Manoeuvring
Example
Consider speed, distance, and complexity
of the manoeuvre, as well as the nature of any obstacles or
features they are using.

2.4 Magical Aptitude (MA)
Magic Aptitude is a measure of a character’s ability to harness and direct magical energies. The
Magic Aptitude characteristic represents the character’s control over the flow of mana (the stuff of
magic), and their ability to remember spells and
rituals.
Specific Influences
• Magic Colleges have a minimum MA requirement
• Cost of training magic
• Base chances of magical skills
Generic Uses
• Noticing arcane mana effects

2.5 Willpower (WP)

Physical Strength is a measure of a character’s
muscle coordination and strength. The Physical
Strength characteristic represents the brute force a
character can exert from the thews of their arms,
the thrusting power of their leg muscles, and their
lift and weight capacity.
Specific Influences
• Effects of weight carried

Specific Influences
• Magic resistance

• Minimum PS requirement for weapons

• Fear resistance

• Damage

• Concentration checks to perform magic

Generic Uses
• Breaking objects

• Persevering with boring or dangerous tasks

Specific Influences
• Sustained activity

• Speed in combat

Willpower is a measure of a character’s self control
of mind and body, especially in stressful situations.
The Willpower characteristic represents a character’s ability to concentrate, their ability to resist the
imposition of another’s will upon their own, and
the degree to which their will can be used to
counter their instincts (when, for instance, the
character might be attempting an action which
could be suicidal).

2.1 Strength (PS)

Generic Uses
• Resisting suffering

• Recovering from being stunned

10

Generic Uses
• Ignoring cold
• Coping with missing meals or sleep

2.8 Physical Beauty (PB)
Physical Beauty is a measure of a character’s exterior attractiveness (or repulsiveness) as perceived
by the humanoid races. Physical Beauty is a characteristic representing a character’s appearance
compared to the aesthetic standards of the main
sentient races. It is in no way a reflection of a
character’s personality. Specific reactions to PB
are also influenced by the observer’s race and
gender. The Physical Beauty values for monsters
describe how that monster appears to a character,
and not to another monster of the same race. Physical Beauty can be increased or decreased temporarily by magic, and decreased permanently by disfigurement. It cannot be increased by training.
Specific Influences
• Reaction rolls
Generic Uses
• Influencing NPCs

2.9 Perception (PC)
Perception is a measure of a character’s intuition
developed as a result of their experience. The Perception characteristic represents the character’s
ability to note peculiarities in a given situation,
their ability to deduce a person’s habits or customs
from scant information, and their general knowledge of the world.
The Perception value can be increased or decreased
temporarily, and can be increased permanently
through training up to racial maximum. Magic,
certain natural or alchemical preparations, and the
character’s condition can cause a temporary increase or decrease in the Perception value.

2 EXPLANATION OF CHARACTERISTICS
Specific Influences
• Detecting ambushes or traps

• Remembering vague information

• Detecting hidden things

• Making connections between new clues and
previous knowledge

• Initiative

2.10 Tactical Movement Rate (TMR)

Generic Uses
• Picking up information from conversation or
observation

The Tactical Movement Rate is the fastest speed a
character can move in combat. A character’s Tactical Movement Rate (TMR) characteristic is based
on their Agility and influenced by any weight
carried or restricting clothing. It may be temporar-

• Peripheral vision
• Noticing things out of the ordinary

11

ily modified by magic or injury, but cannot be
trained.
Specific Influences
• Distance moved in combat
Generic Uses
• comparative speeds

3 RANKING

3 Ranking
Experience points are required to advance in anything. Time spent training is required to increase
proficiency in spells, skills and weapons. Adventuring time is required to advance in characteristics
and talents.
EP is spent as per below but note the following.
• Talents may be ranked only once per game adventuring week.
• Weapon skills take 1 week of training to reach
Rank 0, and 2 weeks × (rank to be achieved) to
improve.
• Skill ranks 8, 9, and 10 must be ratified by a GM
and in general required significant use of the skill.
• The character may rank any combination of two
things at the same time, providing the character
does not rank magic (i.e. spells or rituals) at the
same time as non-magic (i.e. weapons or skills).
• Time spent training for a particular rank of an
ability may be interrupted by other activity (including being on adventure). However training for a
given rank must be resumed within 6 months of
starting training for that rank.
• Namers may rank 1 name in addition to other
forms of ranking. They may also substitute ranking
names for ranking any single magical or nonmagical ability; ie Namers may rank 1 name while
learning two related abilities, 2 names with one
ability or, if doing no other ranking, 3 names at
once.
All adventurers can learn any skill or weapon, and
any magic within their college. Learning costs EP,
time and money. The following can be ranked:
Skills, Weapons, Spells, and Languages, Names,
Rituals, Talents, Adventuring Skills and Characteristics.

3.1 Characteristics
A Characteristic may only be raised by five points
over its starting value to a maximum of 25 (modified by racial bonuses / penalties), except Fatigue,
and Physical Beauty (which cannot be raised). The
exception to this is Perception, which can be raised
to racial maximum.
To calculate the maximum for Fatigue, take the
racial maximum for Endurance and find the maximum Fatigue (see table §1.1). For example, orcs
have +1 to EN so their maximum EN is 26, hence
their maximum FT is 24 from the chart. Then apply
any additional racial modifiers to FT. For example,
orcs gain +2 to FT so their maximum is 24 + 2 =
26.
A characteristic may only be increased once per
adventuring session (if a session took more than
the normal session length, then this rule should be
applied appropriately). If a character did not adventure during a session then they cannot raise any
characteristics, and if they participated on more
than one adventure during the session, then they
can still only raise a characteristic by one. Any or
all of the characteristics may be raised simultaneously if permissible.

If a character has lost characteristic points for any
reason, they may buy back as many points as they
wish in addition to any
normal increase. The cost of buying a characteristic
point back is the same as buying an extra point (see
table §55.3).

Achieving Ranks 8, 9 and 10 is difficult. You must
find and complete a special task relating to your
skill (although of less stature in the case of Artisan
skills), with the assistance of a GM, for each of
these Ranks. Rank 10 is the maximum achievable
Rank in all Skills.

3.6 Adventuring Skills

3.2 Talents
For each week of actual, out in the field, adventuring, you can rank each of your talents once. No
training time is required to rank Talents. Like
Spells and Rituals, each Talent has an EM. No MA
discounts apply to any Talents.

3.3 Spells
If you are an Adept (i.e. cast magic), you can rank
your spells. Each spell has an EM, or Experience
Multiplier. This is multiplied by the Rank that you
wish to achieve, to give a total EP cost. If the
Adept has MA > 15, (MA - 15) × 5% of the EP
cost of General Spells may be discounted. Training
time for spells is (Rank to be achieved) days.
Learning a new spell to Rank 0 takes (EM / 100
rounded up + 1) weeks, but no experience points.
See the Players Guide for availability of special
knowledge spells. You cannot have more spells
and rituals below rank 6 than your MA characteristic. Rank 20 is the Maximum Rank achievable with
any Spell.

Adventuring Skills are skills used every day by
adventurers to survive, and thus are continually
honed. These skills include Horsemanship, Swimming, Flying, Stealth and Climbing.
If you have extensively used an adventuring skill
while on adventure, you may rank this skill without
any time requirements. Otherwise, Ranking time is
as per normal skills.
Adventurers are assumed to start off with Rank 0
or more in all these skills, unless specifically told
otherwise. The EP cost for ranking each Adventuring Skill is listed in table §55.2. The maximum
rank in each of these Skills is 10. No special task is
required for Rank 8 and above.

3.7 Languages
Languages have the same time requirement as
normal skills, except that the time for Rank 0 in a
language is only 1 week, and no special task is
required for Rank 8 and above.

Rituals are learnt and ranked just like spells, except
that Ranking time is (Rank to be achieved) weeks,
rather than days. MA discount applies to General
Knowledge Rituals.

The undiscounted EP costs are set out in Table
§55.2. Note that knowing related languages or the
Philosopher Skill may grant an EP discount. The
maximum total EP discount applicable is 50%,
regardless of how many individual discounts are
available to the character.

3.5 Skills

3.8 Weapons

All skills are assumed to be unranked (i.e. unknown) initially. The first level of competence is
Rank 0, and will take eight weeks to learn. Each
subsequent rank will take that number of weeks to
reach (eg. to get to Rank 7 from Rank 6 will take 7
weeks). The EP cost for ranking each Skill is listed
in table §55.2. Some skills require minimum Characteristic requirements to Rank, or impose EP
penalties (or discounts) for exceptional Characteristics.

All weapons are assumed to be unranked initially.
Rank 0 in a weapon takes 1 week. All higher ranks
take 2 × Rank weeks. Weapons have individual
maximum Ranks. EP costs are detailed in 55.1
Weapons. All Weapons require minimum PS and
MD Characteristics. If you do not fulfil both requirements, you may not rank a weapon. You may
not get an EP discount for training, but if no trainer
is available, you may not increase in Rank. The
cost of a trainer is 10 × Rank squared (minimum

If the character is taught by someone of greater
Rank in the skill, decrease any Experience Point
cost by 10%. If the character learns from a book
(the availability of which is up to the GM), verbal
descriptions or practices with someone of equal or
lesser Rank in the skill, any Experience Point cost
is unmodified. If the character practices with no
useful outside assistance, any Experience Point
cost is increased by 25%. If training is done at the
Guild, it costs 150sp × (Rank to be achieved,
minimum 1).

1) silver pennies.

3.4 Rituals

Some skills include specific abilities (subskills)
which are learnt when increasing your rank in the
skill. It is possible in some skills to learn subskills
by spending time and EP without increasing your
rank. Knowledge has a one-off cost. In both cases,
see table §55.2 and each skill concerned.

12

3.9 Names
Anyone can learn Names, but only Namers can
Rank them beyond Rank 0. Once acquired, an
Individual or Generic Name may be studied and
fully learnt. For Ranking Names beyond Rank 0
see the College of Naming Incantations (§17.3).
• Learning a Name replaces ranking any other
single magical or non-magical ability.
• Generic Names take one day of study to be
learned (i.e. Rank 0).
• Individual Names take one week of study to be
learned.

4 HEALTH AND FITNESS

4 Health and Fitness
A character’s Fatigue will vary depending upon the
amount of food and rest they get compared to their
activities.
A character’s Endurance may be temporarily reduced by lack of sustenance, extreme activities,
damage, or illness.

4.1 Eating and Drinking
The amount of food and water required per day is
dependent on many factors. These include the
person (endurance, weight, build, metabolic rate
and race) and the level of activity they are involved
in (light, medium, hard or strenuous).
On average 1 lb of food and 2 pints of water per
day is required.

4.2 Starvation
Starvation occurs when a character does not have
at least 1 nourishing meal a day.
If a character is starved they will have their Fatigue
maximum and Endurance temporarily reduced by 1
each day. This decrease will last until the character
starts receiving proper nourishment.
A starved character’s Fatigue maximum and Endurance will recover by 1 point each day, after the
first, that they receive proper nourishment

4.3 Dehydration
Dehydration occurs when a character does not have
at least 2 pints of water a day. This amount will
increase in high temperatures by 1 pint per 10
degrees above 20.
If a character is dehydrated they will have their
Fatigue maximum and Endurance temporarily
reduced by 5 each day. If the character receives
part of their water requirement, the penalty is reduced. For every 20% (or fraction) less than the
daily requirement they lose 1 from FT max and
EN. This decrease will last until the character starts
receiving adequate quantities of water.
A dehydrated character’s Fatigue and Endurance
maximums will recover by 5 points each day, after
the first, that they receive adequate quantities of
water.

4.4 Tiredness and Rest
Characters have a tendency to lose Fatigue points
on adventure. A fatigued character must rest to
recover Fatigue points. Sleep, as might be expected, is the best way to become refreshed, but
food and rest will also help.
The Fatigue point loss and recovery rates given in
these rules assume that the character is in good
health and is well fed. If the character is not in
good condition, the GM may adjust the effects of
activity, the effects of weight carried and the rate
of recovery.
Fatigue Loss
A character can lose Fatigue points when they
engage in any activity more stressful than a leisurely walk.
There are four classes of activity which can fatigue
a character:
1. Light Exercise includes moderate to brisk walking, riding slowly or at a moderate pace on a docile
mount, etc.
2. Medium Exercise includes jogging, riding on a
cantering mount, light construction or precision
work, etc.
3. Hard Exercise includes paced running, riding at
a gallop, hard manual labour, etc.
4. Strenuous Exercise includes constant sprinting,
breakneck riding, and generally those actions with
which the character pushes their body to its practical limits.
It is possible for a character’s actions to be more
taxing than Strenuous Exercise, which requires

superhuman exertion. This Fatigue loss from this
activity will be determined by the GM.
A character’s degree of exertion is judged each
hour.
The GM should indicate to players the level of
exertion of their activities (averaging where necessary). If the GM gives consistent guidelines the
players will be able to keep an ongoing track of
fatigue loss.
Encumbrance
A character is limited in the weight they can bear,
and may become fatigued if they engage in exercise.
The Fatigue and Encumbrance Table (§58.1) lists
the maximum weight a character may carry.
A player must determine the total weight their
character is carrying if the character is to engage in
light or more stressful exercise for a significant
length of time during a day.
When an entity has a Physical Strength value
greater than 40, the GM divides that value by 40.
Multiply the quotient by the entry for 40, and add
the entry corresponding to the remainder to determine that entity’s capabilities.
Damage
A character may lose Fatigue by being damaged.
This may be recovered naturally or by being
healed.
Spell Casting
A character may lose Fatigue by using magical
abilities. This may be recovered naturally but may
not be healed.
Calculating Current Fatigue
The Fatigue status of a character only needs to be
calculated before they enter into combat, wish to
perform magic or if they perform fatiguing activities for long periods. To calculate current Fatigue
use the Fatigue and Encumbrance Table (§58.1):
1. Cross-reference the character’s Physical
Strength and the weight they are carrying.
2. Read down this column until it intersects with
the row corresponding to the character’s rate of
exercise.
3. Multiply the resulting number (Fatigue points
lost per hour) by the number of hours at this exercise level.
4. Perform this calculation once for each time one
(or more) of the three factors changes.
5. Add each separate subtotal to determine the total
Fatigue points expended by the character so far.
Exhaustion
If a character’s Fatigue point total is reduced below
zero, they are exhausted. An exhausted character is
limited in the activities they may choose to do and
the performance of their abilities is adversely effected. Their Fatigue is considered zero for the
purposes of combat or magic use.
A character may choose to exert themselves after
their Fatigue points are reduced to zero until they
have expended a nominal one-half their initial
Fatigue points (round down). When they reach this
limit they will collapse unless they succeed a 1 ×
WP check every (2 × Endurance) minutes.
An exhausted character must sleep for as much
time as they were performing any exercise while
exhausted before they may recover any Fatigue
points.
If an exhausted character wishes to engage in
Strenuous Exercise, they must succeed a separate 1
× WP check.
Exhaustion Modifier
The character must subtract 1 / half hour (or fraction) of exhaustion to any base chance.
13

4.5 Fatigue Recovery
A character may regain Fatigue points naturally by
eating a hot meal or resting.
A character may never have a Fatigue total greater
than their Fatigue Characteristic.
A character naturally recovers Fatigue points as
follows:
Activity

Fatigue points / Hour

Eat Hot Meal 2
Relaxation
1
Nap
2
Sleep
3
1. A character may benefit from a hot meal no
more than three times during a 24 hour period, and
each meal must be separated by at least 4 hours.
2. A character that does not get at least 6 hours of
rest and/or sleep per day will have their Fatigue
maximum temporarily reduced by 1 FT / hour (or
fraction) of sleep under 6 hours. This may be recovered at the rate of 1 FT / 4 hours sleep.
3. If a character’s Endurance is less than 10, they
recover one-half of a FT point less per hour or
meal, and if their Endurance is less than 5, they
recover one less FT point. However, a character
always recovers a minimum of one-half a FT point
when resting.
4. If a character’s Endurance is from 21 to 30, they
recover an additional one-half of a FT point per
hour or meal. Each succeeding ten point Endurance
bracket carries an additional one-half FT point
bonus.
5. Fatigue loss from damage may also be recovered
by magical healing (but not the Healer skill Heal
Endurance).

4.6 Damage and Illness
Effects of Low Endurance
Unconsciousness When an entity’s Endurance
reaches 3 or less, they must make a (current EN) ×
WP check or fall unconscious; this WP check is
repeated every minute or if their EN changes.
An entity on 0 Endurance is unconscious, but stable. An entity with a full Endurance of 5 or less
does not make consciousness checks. They remain
conscious until they fall to 0 or less Endurance.
Below Zero Endurance An entity on negative
Endurance will lose one point of Fatigue (Endurance when no Fatigue remains) until the bleeding is
stanched by a Healer, or until dead. They will
continue to take damage from any further blows,
spells, grievous wounds which are bleeders, burning, etc.
When an entity is below zero Endurance they are
on the very brink of death. It takes time and skill to
tell the difference between this state and death (e.g.
empathy, DA). GMs should not let players take
advantage of out of character information when
another player’s character is below 0 Endurance.
Death When an entity’s Endurance falls below
negative one-half their full Endurance, they are
dead. Once dead, ongoing damage (e.g. poison or
bleeding) ceases but further damage may be inflicted on the body.
Endurance Recovery
There are many causes of a character losing Endurance points. However, once lost there are two
primary methods of recovering them.
Healers and Magical Healing
Healers, herbalists, potions, medicines and some
magics may aid the recovery of Endurance. The
exact effects can be found under the appropriate
skill or magic.

4 HEALTH AND FITNESS/ 5 EQUIPMENT AND MONEY
Natural Healing
The rate at which Endurance Points recover naturally primarily depends on how active the injured
being is.
If an entity is resting they regain 1 Endurance point
every three full days.
This rate is reduced to 1 / 4 days if the entity:
• takes any further EN damage
• uses more than half their FT
• does not receive adequate nourishment
If an entity is given ministrations from a
physicker’s kit, their body requires one less day to
regain an Endurance Point.
Injuries which are not quantified as Endurance
point losses or grievous injuries (e.g. hamstrung
muscles) heal at the same rate as they do in this
world.
These healing rates are based on average Endurance value of 15. The GM may chose to increase
the healing rate if an entity’s full Endurance is very
high or decrease it for a low Endurance entity.
Potions \& Unconscious Patients
An entity cannot drink a healing potion when they
are unconscious or below zero endurance but one
can be massaged down their throat. The chance of
doing this is equal to the Manual Dexterity + Perception of the person administering the potion, or if
a healer, 90 + Healer Rank. If successful then D10
per 10 points of the healing potion’s curing (round
down) will be received. If the person fails the roll,
the potion is wasted, but no harmful effects occur
to the patient.
Grievous Injuries
Endurance loss resulting from specific grievous
injuries may not be healed separately from the
underlying specific injury. When the specific injury is fully cured the related endurance is recovered automatically.

Natural Healing of Grievous Injuries
Major injuries take a long time to heal and some
will not heal naturally but require a healer. Here
are guidelines for the healing requirements of some
common major injuries.
Broken bones will knit in 4 weeks for a simple
fracture, or up to 10 weeks for a compound fracture. A bone must be properly set before the bone
may knit together.
Internal injuries an entity will usually die from
internal injuries. If the patient is comfortable, unmoving, and kept alive by a healer or physician,
internal injuries will heal 1 Endurance point per
week
Open wounds will heal at half the normal rate,
provided that they are kept free of infection. Open
wounds will leave scars.
Removed body parts will not regrow naturally.
However, the remaining wound will heal over at
quarter the normal rate, provided it is kept free
from infection.

Dry
-5%
Humid
+20%
The average temperature is ...
Below 0
+20%
1–5
+10%
30 – 40
+10%
Above 40
+20%
Some specific grievous injuries also increase the
chance of infection.
Effects of Infection
An entity with an infection will be slowly poisoned
by the infection. The damage is [D - 5] Endurance
per day, until the infection is cured. An infected
wound will not heal until the infection is cured.
Curing Infection
There are two ways to recover from infection. The
first is to tough it out. The second is to be healed
by a healer.
Toughing it out An infected character may make a
1 × Endurance check every day to recover naturally.

Magical healing of Specific Injuries Healers and
certain magics may heal specific injuries. The time
taken and effects of these magics may be found
under the appropriate skill or magic.

Healing An infected character may be cured by the
arts of a Healer or by magic. The rank at which this
is possible, and the chance of success can be found
under the appropriate skill or spell.

4.7 Infection

4.8 Conception

If a character is wounded there is the possibility
that they have become infected as a result of their
wounds.

The natural conception chances for character races
are:

An Infection Check must be performed to determine whether they are infected or not.
Becoming Infected
The chance of becoming infected depends on the
entity’s health, the type of injury, and the environment the entity is in. Modifiers are cumulative one
is applied from each category:
There is a wound which is ...
Dirty
+20%
Heavily contaminated +50%
The environment is ...

Dwarf
3%
Elf
1%
Halfling
4%
Hill Giant
2%
Human
6%
Orc
10%
Shapechanger 5%
Checks against the relevant chance should be made
no more often than once per 48 hours of appropriate activity.

5 Equipment and Money
5.1 Purchase of goods and items
The GM will be guided in determining the price (in
Silver Pennies) of the various goods produced by
craftsmen by the Price List (see Players Handbook
and Tables §56.1, §56.2 and §56.3). The three
factors which determine the price of finished goods
are the quality of the material used, the hours spent
in construction, and the estimated Rank of the
artisan (if one person produces the goods) or of the
overseer (if the effort is a team project). However,
if a character wishes to purchase a custom-made or
rare item, then they will have to negotiate with the
artisan (represented by the GM), and may defray
costs by providing some of the scarcer components
themselves. The barter system is acceptable when
dealing in costly or rare items.

The value of a coin is determined by its weight and
the metal of which it is made.
Name

Weight Value

Copper farthing (cf) 1/5 oz
Silver penny (sp)
1/20 oz 4 cf
Gold shilling (gs)
1/20 oz 12 sp
Truesilver guinea (tg) 1/10 oz 21 gs
Other common coins include the halfpenny, threepence, and sixpence. The values and weights of
these coins correspond to those of the Silver
Penny.

5.2 Encumbrance Modifies Agility
The weight borne by a character may temporarily
reduce the character’s Agility.
To calculate modified Agility use the Fatigue and
Encumbrance Table (§58.1) and:

14

1. Cross-reference the character’s Physical
Strength and the weight they are carrying. Clothing
(other than armour) the character is wearing does
not count towards this weight.
2. Read down this column until it intersects with
the row which reads “Agility Loss.”
3. Deduct the resulting number from the character’s Agility to give Modified Agility.
4. Re-calculate this number if there is a change in
the weight they bear.
The character’s Modified Agility is used as a basis
for determining their current TMR. A character is
considered to have a minimum Agility of 1 for all
other game functions.

\section{Combat}

6 Combat (Ver 1.1)
There are nineteen sections in Combat:
6.1
Definitions
6.2
Preparation for Combat
6.3
Combat Sequence
6.4
Engaged Actions
6.5
Close Combat Actions
6.6
Unengaged Actions
6.7
Free Acts
6.8
Action Restrictions
6.9
Attacking
6.10 Resolving Attempted Attacks
6.11 Damage
6.12 Effects of Damage
6.13 Weapons
6.14 Unarmed Combat
6.15 Multi-hex Figures
6.16 Mounted Combat
6.17 Aerial Combat
6.18 Aquatic Combat
6.19 Magical Combat
Individual Combat is an inevitable and sometimes
necessary occurrence, and characters should be
aware of its dangers. Fighting is a deadly process
and should be avoided if at all possible. Heroes are
made by defeating the dragon, but more graves are
dug than heroes made. The DragonQuest combat
system reflects these dangers and emphasises skills
and smarts over brawn and brutality.
When combat has begun, the players should place
the figures representing their characters on the
Tactical Display, with the GM determining their
final position. The hostile figures are placed by the
GM, and the Combat Sequence begins.
Combat time on the Tactical Display is divided
into five second Pulses during which all figures
may attempt to take an action, depending on their
position relative to hostile figures. The order in
which these actions take place is determined by the
figures’ engaged or unengaged Initiative values.
Attacks are resolved by comparing the attacker’s
Modified Strike Chance with a percentile roll. A
successful hit does D10 damage, plus any bonus
for weapon type and skill.
When all figures on the Tactical Display are dead,
unconscious, incapacitated, or friendly to each
other, the combat is finished. Combat should never
last longer than necessary to resolve the situation;
returning to normal interactive roleplaying will
speed overall play.

6.1 Definitions
Action The movement and combat activity a figure
may attempt during one pulse.
Attacker The figure performing the Action currently being resolved.
Attack Zone Any hex into which a figure may
attack in Melee or at Range.
Base Chance The base percentage chance of hitting with a weapon, as listed in the Weapon Chart
(§56.1).
Blocked Hex A hex which contains enough solid
material to block any attack. A Blocked Hex is
never part of an Attack Zone.
Cast A Magical Action, used to Cast magic.
in Close A figure in the same hex as a hostile
figure is in Close with the opponent.
Damage Check A roll on a D10 to determine the
amount of damage done after a successful hit. This
roll is modified by the weapon, the Rank or Physical Strength of the wielder, circumstances and
magic.
Damage Points The number of points of damage
done as a result of a damage check.
Defence The amount that a target may subtract
from an Attacker’s Strike Chance, determined by
Modified Agility, a shield, magic and conditions.

Effective Damage Any Damage Points (to either
Fatigue or Endurance) that are actually inflicted on
the figure hit after allowing for absorption due to
armour or magic.

Tactical Display The area to which a combat is
confined, assumed to be covered with a grid of
hexes.

Engaged A figure who is in the Melee Zone of an
opponent, or is in Close, is engaged.

Tactical Movement Rate (TMR) The maximum
number of hexes that a figure may move in a single
action, determined by Modified Agility and Race.

Engagement A group of adjacent figures, all of
who are engaged with each other.

Target The figure on the receiving end of any
Attacker’s action.

Facing A figure must be unambiguously oriented
towards one hex side. This determines their Front
and Rear Hexes, and Attack Zone. They may
change facing during any Action unless specifically prohibited.

Unengaged A figure who is not adjacent to an
opponent, or who is not in an opponent’s melee
zone and chooses to be unengaged.

Figure Any entity or combatant occupying the
Tactical Display.
Grapple An attack in Close Combat.
Grievous Injury An injury that results in specific
damage to a body part.
Hex A 5 foot diameter hexagonal area, with sufficient room for a figure to fight in Melee.
Initiative Engaged Initiative determines the order
of individual actions within an Engagement. Unengaged Initiative determines the order in which
entire sides of Unengaged figures act.
Line of Fire A straight line drawn from the centre
of an Attacker’s hex to a target’s hex that is in the
Attacker’s Ranged Zone.
Melee Zone The front hexes of any conscious,
unstunned, standing or kneeling figure armed with
a prepared weapon rated for Melee combat.
Modified Agility A figure’s Agility after it has
been modified due to weight carried (see §58.1),
armour worn (see §56.3) and circumstance.
Modified Manual Dexterity A figure’s Manual
Dexterity after it has been modified due to the type
of shield carried (see §56.2) and circumstance.
Modified Strike Chance The percentile chance to
successfully hit a target after the target’s Defence
and all Strike Chance Modifiers (see §57.1) have
been taken into account.
Obscured Hex A hex which a figure cannot see
into, but through which objects may pass.
Pass Action Any preparatory Action that does not
directly affect another figure and is not otherwise
covered by specific Actions.
Preparing a Spell A type of Magical Pass Action
Prepared Item Any item (weapon, shield, flask,
etc.) that a figure has in their hands and may immediately use.
Pulse A five second period of game time that regulates Actions on the Tactical Display.
Ranged Zone The hexes radiating out from a
figure’s front hexes into which that figure may see
and fire a missile weapon.
Sheltered Hex A hex which contains enough solid
material for a target to shelter behind such that
approximately half of their body is protected from
any attack.
Strike Any Action that attempts to hit a figure.
Strike Chance The standard percentage chance of
hitting with a weapon; it is a combination of Base
Chance, Manual Dexterity, Rank and magic.
Strike Check The percentile roll against an Attacker’s Modified Strike Chance to see if a Strike
was successful.
Stun A figure who takes sufficient Effective Damage in a single blow is Stunned, and may not attempt to perform any action except Recover from
Stun (see §6.8).

15

Weapon Any item used to Strike a figure.

6.2 Preparation for Combat
Paperwork
Character Sheets for all combatants should be
prepared before the combat. These contain information that will be used continuously during combat, such as Strike Chances, Initiatives, Movement
Rates, etc. A player is responsible for ensuring the
completeness and accuracy of the information on
their Character Sheet, while the GM should prepare
this information for all NPCs in advance. Any
damage or other losses in combat should be recorded as the combat proceeds. Percentile dice and
copies of all charts and tables should be available
for ready usage. Lead figures or counters for all
combatants should also be available, as these add
to the clarity and excitement of a combat.
Surprise
One side in a combat may gain a free pulse of
activity if it surprises the other. If one side in a
combat is unaware of either the opposition’s intent
or their location, they are surprised unless the
figure with the highest effective Perception succeeds in making a Perception Check. This Perception check is modified by both Ranger Detect
Ambush and the Sense Danger Talent.
Prior to placing any figures on the Tactical Display, the GM should determine whether surprise
exists. If one side is surprised, they should place
their figures in a way that represents their lack of
readiness. If no surprise exists, the players may
place their figures as they choose, then the GM
places the opposing figures, with the GM having
the final say on all placements. As a guideline, if
there is surprise, the distance between the two
parties should not exceed 8 hexes, while if there is
no surprise, the distance should not be less than 8
hexes.
If surprise exists, the party with the advantage may
have a free pulse of activity without the surprised
party being able to respond; otherwise, the normal
Combat Sequence starts.
Fatigue
After placing all the figures on the Tactical Display
the GM must assign any Fatigue losses the figures
may have incurred as a result of their actions prior
to combat. For player characters, this should have
been recorded as they slept, rested, travelled, cast
magic or attempted other fatiguing actions. For all
NPCs, the GM should make a quick estimate relating all presumed activity they may have undertaken up to the start of combat.

6.3 Combat Sequence
The order in which all actions are attempted in a
pulse is called the Combat Sequence. This sequence should be adhered to at all times as this will
greatly speed play. Each pulse, events occur in the
following order:
1. Unengaged Initiative is determined for each
side.
2. If any figures are engaged or in Close, these
figures are grouped into Engagements, and each
Engagement is dealt with separately.

6 COMBAT
3. In each Engagement, engaged Initiative is determined, and then the figures act in order of Initiative (highest to lowest), each performing one Action from the Engaged Actions list.
4. The winner of the unengaged Initiative now
resolves the Actions (selected from the Unengaged
Actions List) of all their unengaged figures, in any
order they choose. All their actions must be resolved before any figures on the opposing side may
act.
5. Remaining unengaged figures may act as in Step
4.
6. End of Pulse activity occurs. This may include
an additional stun recovery attempt for figures that
were Stunned during the Pulse and any housekeeping.
Exceptions to the Combat Sequence
Engaged Figures becoming Unengaged If a
figure becomes unengaged before their engaged
Action is resolved, they must act on their unengaged Initiative. If they become unengaged after
they have had their engaged Action, they do not
gain an extra Action.
Unengaged Figures becoming Engaged If a
figure becomes engaged before their unengaged
Action is resolved, they must select an engaged
Action on their unengaged Initiative.
Optionally Engaged Figures A figure who is
adjacent to a hostile figure, but is not in any opponent’s Melee Zone, may choose whether to be
treated as engaged or unengaged, and acts on the
appropriate Initiative.
Extraordinarily Agile Figures A figure who has a
modified Agility of 26 or more may perform two
Actions on their initiative. Their choice of Actions
is restricted. See §6.8.
Stunned Figures If a figure is Stunned before their
Action, they may attempt to Recover from Stun as
their Action. This takes place when specified for
Stunned figures in the Initiative Section below. If a
figure is Stunned during the Pulse they may attempt to Recover from Stun at the end of the Pulse
in which they were Stunned, regardless of whether
they acted or not.
Initiative
Engaged Initiative A figure’s engaged Initiative
Value is their modified Agility + Perception + their
Rank with prepared weapon + Warrior bonus. If
the figure has no prepared weapon, they may use
their Unarmed Rank. Any ties indicate simultaneous combat. In each Engagement, figures with
faster initiative may choose to act after figures with
lower initiative, but all engaged Actions must be
resolved before any unengaged Actions. If an
engaged figure is not in any opponent’s Melee
Zone, they may act first in the engagement. If a
figure is Stunned they act last in their Engagement.
Unengaged Initiative A side’s unengaged Initiative is their leader’s Perception + D10. If the leader
is a Military Scientist, they gain a bonus to this
roll. Any ties are re-rolled. The leader may choose
to have their entire party act after a slower side.
This decision is made before any engaged actions
are resolved. If a side’s leader is engaged, Stunned
or otherwise incapacitated at this point, another
character may assume this role. All figures who are
under a single Leader take their Actions in the
same Initiative, in any order that they find convenient. A Stunned figure always acts last in their side.
Action Timing
When a figure chooses an Action, they are assumed to be performing that Action until they start
a new action or are stunned. However, any Action
which requires a dice roll to resolve is completed
when that dice roll is made; the figure is assumed
to be engaged in follow-up manoeuvres until their
next action. After successfully recovering from
Stun, a figure is assumed to have just finished a
Pass Action until their next Action, for all pur-

poses, and any previously prepared shield or weapons are once again prepared.
GM Conventions
These are only conventions; the GM may modify
these conventions to suit their own style.
Announcements of Intent When combat occurs
on the Tactical Display, there should be no lapses
of time between player announcements of character
intentions and resolution of them. When it is a
character’s turn to take action, the player must
announce within 5 seconds what their character
will do, or the character will take a Pass Action.
The GM should restrict themselves to a similar
guideline for their NPCs. A player may change the
action they announced for their character to a Pass
Action (only) up to five seconds after they announce it.
Discussions during Combat If the players wish to
discuss tactics amongst themselves, they must do
so quietly while the GM is determining the result
of a particular action. Anything said by one character to another during combat may be overheard. A
Military Scientist character may allow a party a
Time-out during which they will not be overheard.
Rule Clarification Any player may, at the GM’s
discretion, suspend the passage of time by requesting a clarification of a relevant point by the GM.
They may also appeal a decision made by the GM
that they feel to be arbitrarily or improperly resolved. The player has as much time as the GM
may grant. The GM may modify or reverse their
decision, or stand behind it. The GM’s word is
always law.

6.4 Engaged Actions
Being engaged imposes certain limitations on the
actions that a figure may attempt. The primary
restriction is that an engaged figure may not move
out of the Melee Zone of an opponent except in
special circumstances. If an engaged figure is in
Close Combat, their range of Actions is further
restricted.
The order in which the Actions of engaged figures
is resolved is determined by each figure’s engaged
Initiative.
Melee Attack
An engaged figure may move one hex and change
facing, and then attempt a Melee Attack. They may
not move after they have attacked.
Close \& Grapple
An engaged figure may attempt to Close & Grapple. If the Attacker is within the Melee Zone of the
target, they may try to Repulse the Attacker’s
attempt to Close by rolling less than or equal to
their prepared Melee weapon Rank on a D10.
Multihex figures may not be Repulsed, but the
defending figure may avoid the attack by succeeding in a 1 × AG check.
A successful Repulse means that the target has
interposed their weapon between themselves and
the Attacker, and the action has failed. If the Repulse is unsuccessful, the Attacker may go into
Close and make a normal Grapple attack.
Evade
An engaged figure may move one hex and change
facing while executing an Evade. If a Melee Attack
is attempted on a figure who is Evading with a
Ranked weapon, they may be able to Parry the
attack. An Evading figure receives a bonus to
Defence versus Melee and Ranged Attacks.

and change facing. They may freely leave the
Melee Zone of any opponent, but may not move
into that opponent’s rear hex.
Flee
An engaged figure who does not have an opponent
in their Melee Zone may Flee. This allows them
the same options as
an unengaged Move. Any opponent able to Melee
attack the figure automatically receives Initiative.
Pass
An engaged figure may move one hex and change
facing while performing a Pass Action.
Cast
An engaged figure may change facing but not
move while attempting to Cast a Spell. Casting is a
Magical Action. Like all other actions, Casting is
resolved on the figure’s initiative.
Throw
An engaged figure may change facing but not
move while attempting to Throw a weapon. They
may only Throw into their Ranged Zone where
they have a Line of Fire.
Recover from Stun
An engaged figure who is stunned may attempt to
recover from Stun. They may not make any
movement or change facing.
Leaving Melee Combat
A figure engaged in Melee Combat may only leave
Melee Combat by (i) executing a Close & Grapple,
(ii) executing a Withdraw or Flee, or (iii) by stunning or otherwise incapacitating all opponents who
have the figure in their Melee Zones. Other Actions may never take the figure out of any hostile
Melee Zone.

6.5 Close Combat Actions
All figures in Close Combat are treated as engaged.
However, while in Close, only close-rated weapons
may be employed. All other weapons or items must
be dropped immediately. Figures in Close Combat
are treated as prone, and thus have no Melee Zone
or facing.
If an engaged figure is in Close Combat, their
Action is limited to one of the following:
Grapple
A figure engaged in Close Combat may neither
move nor change facing while attempting a Grapple. A Grapple is an attack with any close-rated
weapon (including Unarmed).
Withdraw from Close
A figure may attempt to Withdraw from Close
Combat. If they are successful, they may move one
hex, but are still treated as prone. A figure may
Withdraw from Close if a D10 roll plus any positive difference in total Physical Strength between
the friendly and hostile figures in the hex is at least
10.
Pass
A figure engaged in Close Combat may neither
move nor change facing while taking a Pass Action. They may not attempt a Magical Pass Action
or Multi-Pulse Action. Some other Pass Actions
will be impossible, as they are effectively prone.
Recover from Stun
A figure engaged in Close Combat who is stunned
may attempt to recover from Stun as their Close
Combat Action.

Offensive Withdraw
An engaged figure may make a Melee Attack with
a -20 penalty to their Strike Chance. They may
then move one hex and change facing. They may
freely leave the Melee Zone of any opponent, but
may not move into that opponent’s rear hex.

Leaving Close Combat
A figure engaged in Close Combat may leave
Close Combat by (i) executing a Withdraw from
Close, or (ii) by stunning or otherwise incapacitating all opponents who are in Close with the figure.

Defensive Withdraw
An engaged figure may solely defend, increasing
their defence by 20. They may then move one hex

An unengaged figure is one that is not engaged.

16

6.6 Unengaged Actions

6 COMBAT
Move
An unengaged figure may move any number of
hexes up to their TMR. During movement, a figure
may change facing as desired.

Recover from Stun
An unengaged figure who is stunned may attempt
to recover from Stun. They may not make any
movement or change facing.

Step & Melee Attack
An unengaged figure may move one hex and
change facing, and then attempt to Melee Attack.
They may not move after they have attacked.

6.7 Free Acts

Charge
An unengaged figure may move up to 1/2 TMR
and attempt to Melee Attack with a non-pole
weapon. At the end of the figure’s movement, if
there is a hostile figure in the Attacker’s Melee
Zone, they may make a Melee Attack with a -15
penalty to Strike Chance. The figure may not
change facing after the Melee Attack.
Charge with Pole Weapon or Shield
An unengaged figure may move up to TMR and
attempt to Melee Attack with a pole weapon or
Shield. At the end of the figure’s movement, if
there is a hostile figure in the Attacker’s Melee
Zone, they may make a Melee Attack with a +20
bonus to Strike Chance. The figure must move at
least 2 hexes, and may not change facing after the
Melee Attack. This action may not be attempted
with a Tower Shield or a Main Gauche.
Charge & Close
An unengaged figure may move up to 1/2 TMR
and attempt to Close. If the figure passes through
the Melee Zone of the target, the target may try to
Repulse the figure in the same way as for a Close
& Grapple.
If the Repulse is successful, the target has interposed their weapon between themselves and the
Attacker. If the Attacker cannot or will not stop
entering Close, the target automatically inflicts a
potential Specific Grievous Injury on the Attacker.
If the Repulse is unsuccessful, the Attacker may go
into Close, and may attempt a normal Grapple
action or a Trample attack.
Evade
An unengaged figure may move up to 1/2 TMR
and change facing while executing an Evade. If a
Melee attack is attempted on a figure who is Evading with a Ranked weapon, they may be able to
Parry the attack. An Evading figure receives a
bonus to Defence versus Melee and Ranged Attacks.
Retreat
An unengaged figure may Retreat, increasing their
defence by 20. They may move up to 2 hexes
backwards and change facing.
Pass
An unengaged figure may move two hexes and
change facing while performing a Pass Action.
Cast
An unengaged figure may not move while attempting to Cast a Spell, but may change facing. Like all
other actions, Casting is resolved on the figure’s
initiative. Casting is a Magical Action.
Throw
An unengaged figure may move up to 2 hexes and
change facing while attempting to Throw a
weapon. They may only Throw into their Ranged
Zone where they have a Line of Fire.
Fire
An unengaged figure may not move while attempting to Fire a missile weapon, but may change facing. Once a Crossbow is prepared and loaded, a
figure may carry it around and fire whenever they
wish. In this instance, the figure may move up to 2
hexes either before or after firing. All missile
weapons need to be Loaded before they may be
Fired. The figure may only Fire into their Ranged
Zone where they have a Line of Fire.

In addition to their normal Action, a figure may
always drop whatever they have in their hands and,
if not performing a Magical Action, they may say a
short phrase during their Action.

6.8 Action Restrictions
Movement may be restricted by terrain or other
conditions. Figures with a modified Agility of 8 or
less may have their movement reduced when performing other Actions, while those with a modified
Agility of 22 or more may gain extra movement or
Actions. Figures who become stunned or otherwise
incapacitated may not attempt normal Actions,
except that stunned figures may attempt to recover
from Stun as their Action. The type of Action a
figure may attempt is restricted by their position on
the Tactical Display, and their visibility. The use of
magic in combat is subject to restrictions, and may
in some cases be impossible.
Movement
Any complicated turning manoeuvre may result in
a reduction in the figure’s movement allowance for
that pulse. A reduction of 1 hex is suggested for
each 180◦ turn. At the end of the Action, the figure
must be unambiguously oriented towards one hex
side.
A figure’s movement allowance assumes a flat
surface with little or no hindrance to movement.
Some terrain is not conductive to quick traversal,
and the figure should suffer a reduction to movement in such conditions. A figure should normally
be able to move at least 1 hex per pulse, no matter
what the terrain.
If a figure enters the Melee Zone of any hostile
figure, they become engaged, and must stop
movement, though they may change facing. If the
figure is performing a Charge & Close, they may
attempt to enter Close, but the opponent’s hex
counts as a hex of movement.
If a figure wishes to jump during their movement,
they should have their movement allowance reduced, and the figure must make an Agility Check
to land cleanly.
Often two or more friendly figures will wish to
pass through a hex at the same time, or need to
squeeze past each other in the same hex. The GM
should judge whether circumstances permit this,
and if so, whether the figures are hindered. If neither figure is endeavouring to do more than move
through the hex, there will usually be little problem, but more dangerous manoeuvres may reduce
movement or require Agility Checks.
A figure may move backwards at half their movement rate, and crawl at 1/4 rate.
Pass Actions
A Pass Action is any generic non-attacking action a
figure may attempt which is not otherwise covered
by specific Actions. Typical Pass Actions include:
preparing an item or weapon, putting an item away,
picking up a dropped item, mounting or dismounting a steed, loading a missile weapon, drinking
from a flask, dropping to one knee or prone, rising
up, etc.
Pass Actions also include Multi-Pulse Actions and
the following Magical Pass Actions:
• Prepare Spell (see §7.3).
• Actively Resist (see §7.8).
• Concentrate (see §7.6).
Typical
Actions: The following list is intended as a guide
for the GM to be able to judge how many pulses an
attempted Action will take to perform. Note that
some Actions that figures in combat wish to attempt will take far more than one pulse.
17

Search for trap in specific place
Attempt to remove trap
Quick search of 10’ × 10’ for
disguised objects
Sound Wall
Pick Lock
Force Lock
Spike Door
Light Torch
Light Lantern
Putting on / Removing Armour:
Helm
Leather
Scale
Chain
Plate
Using a backpack:
Put on / Take off
Remove item
Store item
Dismount / Mount Horse
Drink 1/2 pint flask
Load missile weapon:
Crossbow
Crossbow using cranequin
Other

2
(see §47.2)
3
1
(see §47.2)
3
2
3
5
1
6
24
12
60
1
2
1
1
1
2
3
(see note L
§56.1)

Figures with Low Agility
Figures with modified Agility of 8 or less are allowed one less hex of movement when executing
any of the following Actions: engaged Melee Attack, engaged Evade, Retreat, Pass, Fire Crossbow,
Throw.
Figures with High Agility
Figures with modified Agility of 22 through 25 are
allowed one extra hex of movement when executing any of the following Actions: Melee Attack,
engaged Evade, Withdraw, Retreat, Pass, Fire
Crossbow, Throw.
Figures with Extraordinary Agility
Figures with a modified Agility of 26 or more may
perform an additional defensive withdraw, retreat
or non-magical pass action except when Stunned.
The actions are resolved consecutively, in either
order. The figure’s total movement may not exceed
their TMR.
Stunned Figures
A figure who becomes Stunned may only take
Recover from Stun as their Action. A figure who
was Stunned during the Pulse gets an additional
attempt to Recover from Stun at the end of that
Pulse. They may still take free Acts. The Base
Chance of Recovering from Stun is 2 × WP +
current FT. A stunned figure has no Melee Zone.
Position of Opposing Figures
The type of Action a figure may attempt is restricted by the position of the nearest opposing
figure. If a figure is in the same hex as a hostile
figure, they are in Close, and may only select an
Action from §6.5. If a figure is in a hostile figure’s
Melee Zone and is not in Close, they are engaged,
and may only select an Action from §6.4. If a
figure is not in a hostile figure’s Melee Zone, but
has a target in their Melee Zone, they may select an
Action from either §6.4 or §6.6, depending on how
they wish to be treated. Otherwise, a figure is unengaged, and must select an Action from §6.6.
Visibility
If a figure is attempting to perform a Melee or
Ranged Attack on a hostile figure who occupies a
hex that is obscured (due to smoke, magic, etc.),
they may be affected by visibility modifiers (see
§57.1). If they are attempting to cast a spell that
requires targeting, they must make a Perception
Check. If the figure is totally obscured, they are
treated as invisible for Strike Chances, and may not
normally be targeted by a target: Entity spell. The
GM must determine if a figure is affected by an
Area or Line of Fire spell, or Ranged Attack.

6 COMBAT
Disturbing Magical Actions
If an Adept is performing a Magical Action, and is
attacked, they must make a Concentration Check
(see §7.6) or their Action will fail. If the Adept is
stunned or has sufficient cold iron lodged within
them, their Action will automatically fail. An
Adept may not cast while prone.
Action Summary
The Action Summary (§57.3) lists all valid Actions
and their Restrictions.

6.9 Attacking
The order of all attacking Actions is determined by
the Initiative procedure as detailed in §6.3. Combat
involving engaged figures is always resolved before any combat involving unengaged figures. An
attacker’s weapon is always assumed to be held in
their primary hand unless stated otherwise. Empty
bare hands are always considered a prepared
weapon.
A hostile figure may be attacked by Ranged, Melee
or Close combat while on the Tactical Display.
Special types of attacks are allowed, and these
include Multi-hex Strikes, Multiple Weapon
Strikes, and attempting to Trip, Entangle, Restrain,
Knockout, Shield Rush or Disarm.
Ranged Attacks
A figure may attempt to attack a hostile figure in
their Ranged Zone via ranged combat by executing
a Fire or Throw Action. The figure declares their
target, determines and applies any Ranged Combat
modifiers (see §57.1), and executes a Strike Check.
To Fire a missile weapon, the figure must be armed
with a prepared and loaded missile weapon. To
Throw a weapon, the figure must be armed with a
prepared weapon rated for ranged combat. The
figure must have a Line of Fire to the target. If the
Line of Fire contains an obscured hex, the figure
may not Aim, and treats the target as if invisible.
Whenever the weapon enters a hex occupies by a
figure or object (other than a solid obstacle that the
missile must hit), there is a chance (as determined
by the GM) that the weapon will hit the figure or
object instead of continuing its flight. This must be
resolved for each figure occupying any hex along
the Line of Fire until the weapon hits something or
loses momentum and falls to the ground.
A figure cannot check a Line of Fire without executing an Aim, Fire or Throw action, whether or
not the weapon is actually loosed.
Snapshooting A figure with a prepared Short Bow,
Long Bow, Composite Bow, Giant Bow or Sling,
with which they are at least Rank 3, may prepare
an arrow or bullet and Fire in the same Action. The
Strike Chance is reduced by -15. Snapshooting is a
Fire Action.
Aiming A figure with a prepared and loaded missile weapon may choose to take a Pass Action to
Aim the missile weapon at a particular target. If the
figure then Fires at that target in their next Action,
their Strike Chance is increased by +20, and in
addition, the chances of causing Endurance or
Specific Grievous damage are increased to 20%
and 10% of the modified Strike Chance, respectively.
Melee Attacks
A figure may attempt to Melee Attack any hostile
figure who occupies at least one hex of their Melee
Zone. The figure declares their target, determines
and applies any Melee Combat modifiers (see
§57.1), and executes a Strike Check with a prepared melee-rated weapon. The attacker may move
adjacent to the target during that pulse.

cuting a Grapple Action. The figure declares their
target, determines and applies any Close Combat
modifiers (see §57.1), and executes a Strike Check
with a prepared close-rated weapon. The attacker
may move into the target’s hex during the pulse;
this is known as closing.

by -20. If the attack is successful, one point of EN
is inflicted, and the target must roll under (MD +
Rank) or drop a weapon or item of the attacker’s
choice. If the item is being held in two hands, the
check is (2 × MD + Rank). A figure may not move
while attempting a Disarm.

Attacking into Combat
A Ranged attack on a figure in Melee combat is
resolved normally, bearing in mind the Line of
Sight restrictions. A Ranged or Melee attack on a
figure in Close combat suffers a penalty of -10. If
the attack misses, an additional attack with the
same penalty must be resolved against each remaining figure in that hex (in a random order). If a
multi-hex creature is in close with single-hex creatures, it may be targeted normally.

6.10 Resolving Attempted Attacks

Special Attacks
A figure may attempt to attack using any one of the
following special attacks.

If the target is Evading, the attacker has a reduced
strike chance and, if they miss, they may be Disarmed or Riposted.

Multiple Strike A figure who is armed with two
prepared weapons (one in each hand) may attempt
a Multiple Strike. The two weapons need not be
targeted against the same opponent, but must be of
the same type (Ranged, Melee or Close). The
Strike Chance of the Primary weapon is reduced by
-10, while the Strike Chance of the Secondary
weapon is reduced by -30. Ambidextrous figures
suffer a -10 penalty with each attack. A figure may
not move while making a Multiple Strike.

Strike Chance
When attacking with any Ranked weapon, the
Strike Chance is (Weapon Base Chance) + (Mod.
Manual Dexterity) + (4 × Rank). When attacking
with an unranked weapon, the Strike Chance is
equal to the Base Chance. Wild creatures using
natural attack forms such as teeth, claws, etc.,
always add their Manual Dexterity + (4 × Rank).

Multi-hex Strike A figure who has a prepared
two-handed B-class weapon, with which they are at
least Rank 4, may strike up to three figures in
adjacent hexes in their Melee Zone. Their Strike
Chance is reduced by -20 on each attack. A figure
may not move while making a Multi-hex Strike.
Trip A figure with a prepared Quarterstaff, Spear,
Halberd, Poleaxe or Glaive may attempt to trip an
opponent in their Melee Zone. The Base Chance is
reduced to 40%, and the damage to D10. If the
attack is successful, the target must make a 3 × AG
Check or fall prone. This attack may not be attempted on a target significantly larger than the
attacker. A figure may not move before attempting
a Trip.
Entangle A figure with a prepared Net, Whip,
Lasso or Bola may attempt to Entangle their opponent during any attack. If the attack is successful,
the target must make a 3 × AG Check or fall prone.
The target must disentangle themselves before
rising, requiring 2 Pass Actions.
Restrain A figure may attempt to restrain an opponent by pinning them to the ground. The Base
Chance is three times the difference in total PS &
AG between the attacker(s) and their opponent. No
damage is done. A restrained figure is treated as
incapacitated, and remains restrained until the
restraint is broken by an attack from outside the
hex that does effective damage to a restrainer. A
Restrain may only be attempted in Close Combat.
Knockout A figure with any prepared Melee rated
weapon excluding entangling weapons, Lances and
Pikes, may attempt to knock out their opponent.
The attack is successful if the Strike Check would
normally result in an Endurance blow (see §6.11).
No damage is done, but the target is unconscious
for [D + 5] minutes. This attack may not be attempted on a target significantly larger than the
attacker. A figure may not move while attempting
a Knockout.

The normal Melee attack is intended to do as much
damage to the target as possible, but other forms of
specialised attack exist.

Shield Rush A figure with a prepared shield (other
than a Main Gauche or Tower Shield) may attempt
to Shield Rush their opponent. If the attack is successful, the target must make a 3 × AG Check or
fall prone. This attack may not be attempted on a
target significantly larger than the attacker. A
figure must move at least one hex before attempting a Shield Rush.

Close Combat Attacks
A figure may attempt to attack any figure who
occupies the same hex via Close Combat by exe-

Disarm A figure may attempt to Disarm an opponent with any prepared Melee or Close rated
weapon. The Strike Chance of the attack is reduced
18

Every weapon and attack form has a Base Chance.
The Base Chance with all modifiers applied is the
Modified Strike Chance. The attacker performs a
Strike Check by rolling D100; if the result is less
than or equal to the Modified Strike Chance, the
attack has been successful; above and the attack
has missed. Particularly poor rolls may result in the
weapon being broken or dropped. Once a successful hit has been made, a Damage Check occurs.

Modified Strike Chance
An attacker’s Modified Strike Chance is equal to
their Strike Chance plus any modifications for
attack type and conditions, minus the target’s current defence. If the attacker rolls less than or equal
to the Modified Strike Chance, a successful hit has
occurred, and a Damage Check is made (see
§6.11).
Attack condition modifiers are detailed in (§57.1
Strike Chance Modifiers).
Evading
If a figure evades, their Defence against Melee
attacks increases by 10 + 4 / Rank of their prepared
Melee weapon, and their Defence against Missile
attacks increases by 20.
If a figure is Evading, and an opponent in their
Melee Zone misses an attack at them by 30 or
more, they may choose to try to Parry the attack.
The target rolls D10, adds the Rank of the prepared
weapon they are Evading with, and subtracts the
Rank of the attacker’s weapon. If this result is 3 or
less, the attack has been successfully Parried, but
the target has been thrown off balance, and their
next action must be a pass action. If the modified
result is 4 through 7, the target may Disarm the
attacker (see Disarm). If the modified result is 8 or
above, the attack has been Parried and the target
may execute a free Melee Attack on their attacker
as well as a Disarm. This is called a Riposte.
A Riposte cannot itself be Parried, and may occur
as many times in the pulse as the evading target
was Melee Attacked. An unarmed figure may Parry
if they are ranked in Unarmed Combat.
Defence
A figure’s defence is subtracted from an attacker’s
Strike Chance. Defence is equal to modified Agility, plus defence afforded by a prepared shield,
defensive manoeuvres and magic.
Defensive advantages due to terrain conditions and
visibility modifiers are covered in §57.1 Strike
Chance Modifiers. A figure has no defence except
for that provided by magic if they are stunned or
incapacitated.
A prepared shield provides defence against all
Melee and Ranged attacks that pass through a
figure’s front hexes, if they have the Shield skill.
At Rank 0 and each additional Rank, the defence
bonus (2% to 6%, see §56.2 Shields) is added to
defence. No bonus is given for an unranked shield.
A figure may not attack with their shield or count
their shield as a prepared weapon for Evading

6 COMBAT
while retaining the shield defence bonus. A prepared Main Gauche also provides some defence;
however defence is only applied against Melee
attacks, and no defence is gained at Rank 0.
Fumbles
An unmodified Strike Check of 00 indicates that
the attacker has fumbled; they lose 10 from their
Initiative Value until the end of the next pulse. This
chance of fumbling is increased if the weapon is
made of a material other than cold iron, as listed
below, unless it is magical, or a Bow or Crossbow.
any silver or truesilver alloy of
iron
any other hard metal alloy (e.g.
bronze).
viable weapons made of other
materials

1%
2%
3% (or
more)*

* the actual figure should be specified by the GM at the
time of the weapon’s creation.

When an attacker fumbles, they make a totally
unmodified D100 roll. If that roll is under their
current Initiative Value, they suffer no further
penalty for their slight fumble; if it not under their
current Initiative Value, apply the corresponding
result from §52.3 or §52.4 (the Fumble tables).

6.11 Damage
A successful Strike Check usually results in a
Damage Check being performed. Each attack has a
damage modifier that is applied to a D10 roll, and
the result is the number of damage points inflicted
by the attack (minimum damage 1). There are three
types of physical damage possible from a successful strike, depending on how successful the Strike
Check was: Fatigue Damage, Endurance Damage,
and Specific Grievous Injuries.
Fatigue Damage
Physical Damage affecting Fatigue may be absorbed by armour. Each type of armour has a Protection Rating (as listed in §56.3 Armour Chart),
which is subtracted from the Fatigue damage inflicted. When a figure’s Fatigue reaches 0, any
subsequent attacks affecting Fatigue are subtracted
from Endurance instead. A figure normally cannot
lose both Fatigue and Endurance from one Strike
Check.

or Rank. Only one of these two modifiers may be
applied at any time.
If a figure chooses to over-strength a weapon, they
may inflict an additional point of damage for every
5 full points of Physical Strength they have over
the minimum required to use the weapon. Thrown
or Missile weapons may not be over-strengthed.
See §6.14 for Unarmed Combat.
If a figure chooses to apply skill to inflict extra
damage, they may inflict an additional point of
damage for every full 4
Ranks they have in the weapon. This affects Close,
Melee Thrown, and Missile weapons.

6.12 Effects of Damage
Missile Lodgement
When a figure takes effective Endurance Damage
from an A-class Missile or Thrown weapon, the
weapon has lodged itself in their body, and reduces
the figure’s Agility by 3 (5 if a pole weapon). The
Agility loss for multi-hex creatures will be reduced
in proportion to their size. The weapon remains
lodged until a Pass Action is taken to remove it. A
barbed arrow lodges if it inflicts any effective
damage, and the figure will take D-4 Fatigue damage when the barbed arrow is removed unless it is
removed by a Healer. Barbed arrows have a Strike
Chance penalty of -25.
Stunning
Whenever a figure suffers effective damage greater
than one-third their full Endurance, they become
stunned.
• They stop performing any existing Action.
• They have no Melee Zone, but remain Engaged
as long as they are in the Melee Zone of an opponent.
• Their Initiative changes (see §6.3)
• They have no defence except that provided by
magic.
• Any shield or weapon (including unarmed) becomes unprepared.
• Their only Action which they may attempt is
recover from Stun.

Endurance Damage
A Strike Check of 15% or less of the Modified
Strike Chance results in damage directly affecting
Endurance, and which is never absorbed by armour.

• At the end of the Pulse in which they were
stunned, a figure may attempt to Recover from
Stun.

Specific Grievous Injuries
In addition to Endurance damage, a Specific
Grievous Injury may occur if the Strike Check is
5% or less of the Modified Strike Chance. If a
potential Specific Grievous Injury occurs, the
attacker rolls D100 and consults the Grievous
Injury Table (§51). If the roll falls within the range
specified for the weapon class, a Specific Grievous
Injury has occurred, and the effects of the resulting
injury are applied in combination with any Endurance damage inflicted.

• They may not move on the Tactical Display, or
change facing. They may still perform Free
Acts(see §6.7).

A figure who suffers a Grievous Injury while wearing armour has the Protection Rating of their armour reduced by two until repaired. Optionally, a
figure who is also carrying a shield may choose to
have the shield cloven instead. A cloven shield is
useless.

Unconsciousness
When a figure’s Endurance reaches 3 or less, they
must make a (current EN) × WP check or fall
unconscious; this WP check is repeated every
minute. A figure on 0 Endurance is unconscious,
but stable. A figure on negative Endurance will
lose one point of Fatigue (Endurance when no
Fatigue remains) until the bleeding is stanched by a
Healer, or until dead. A creature with a full Endurance of 5 or less does not make consciousness
checks. They remain conscious until they fall to 0
or less Endurance.

Magical Damage
All magical damage affects Fatigue unless otherwise states in the spell description. Spell damage is
assumed to be nonphysical, and thus unaffected by
armour, unless the spell explicitly states that it is
affected by armour. Magical damage that is not
affected by armour never stuns. Breath weapons
are treated as magical damage, but are Passively
Resistible for half damage.
Additional Damage
The damage inflicted with a particular weapon may
be increased due to exceptional Physical Strength

• The Base Chance to recover from Stun in 2 × WP
+ current Fatigue.

Massive Damage
If a figure with positive Fatigue suffers effective
Fatigue damage greater than their combined full
Fatigue and Endurance, they lose all their Fatigue
and are reduced to -1 Endurance. If they suffer
more than their combined full Fatigue and 1.5 ×
Endurance, they are dead.

Death
When a figure’s Endurance falls below negative
one-half their full Endurance, the figure is dead.
Once dead, further damage may be inflicted, but no
more damage will be inflicted from poison or
bleeding.

19

Infection
Whenever a figure has had Physical Damage inflicted (or some particularly nasty form of magical
attack), they may have become Infected. There is
normally a 10% chance of any wound becoming
infected. This is increased by (20 + Endurance
Damage)% if any Endurance damage was inflicted.
Bite, claw and talon attacks, hostile environmental
conditions and poor treatment may further increase
the chance. See §4.7 for more information.

6.13 Weapons
Any instrument used to inflict damage on a figure
is called a weapon. Weapons may include the
figure’s hands, feet, teeth, etc. All normal weapons
are listed on the Weapons Chart along with their
characteristics. The only limits to the number of
weapons a character may have in their possession
are the weight and bulk of those weapons; the GM
should disallow any odd or unlikely method of
carrying weapons.
Normal Weapons
The Weapons Chart (§56.1) lists all the normal
weapons and their characteristics.
Weight The weight of the weapon in pounds (excluding scabbards, etc.).
Physical Strength The minimum Physical
Strength a figure needs to wield the weapon properly; a figure without the required PS does 1 less
point of damage for each point of PS they are
below the minimum. A figure may never achieve
Rank in a weapon they do not have the PS to wield.
Manual Dexterity The minimum modified Manual Dexterity a figure needs to manipulate the
weapon properly; a figure without the required MD
has the Base Chance of the weapon lowered by 5
for every point they are below the minimum. A
figure may never achieve Rank in a weapon they
do not have the MD to manipulate.
Range The distance (in hexes) which the weapon
may be Fired or Thrown.
Class The type of damage done by the weapon - Aclass for thrusting damage, B-class for slashing
damage, and C-class for crushing damage. This is
used for determining Specific Grievous Injuries.
Use The range(s) of attack the weapon may be
used at: R for ranged combat, M for Melee combat,
C for Close combat. A weapon may not be used at
an inappropriate range.
Cost The standard cost (in Silver Pennies) to buy a
typical example of the weapon.
Maximum Rank The highest Rank attainable with
the weapon.
Unusual Weapons
A figure may attempt to strike bare-handed (see
Unarmed Combat), but only if one hand is free. A
figure may attempt to use an item not normally
used as a weapon at the GM’s discretion, who
assigns Base Chances, damage modifiers, and so
forth. Makeshift weapons will generally be no
better than a Crude Club.
Envenomed Weapons
If the GM permits, figures may carry and use Aclass & B-class weapons coated with poison. At
least one point of effective damage must be done
for the poison to affect the target.
When anyone except an Assassin handles an envenomed weapon (§33.2), they must make a 3 ×
MD check every time they handle the weapon.
This includes coating the weapon, preparing or
unpreparing the weapon, and attacking. An envenomed weapon will usually remain effective for 6
hours or until at least one point of effective damage
has been inflicted.

6.14 Unarmed Combat
Any figure may attempt to attack a hostile figure
by using their natural weapons. For many creatures, this is the only way they may attack. Unless

6 COMBAT
otherwise specified, all figures receive one Unarmed attack per pulse without penalty. Some
creatures may be able to attack more than once (see
Bestiary). A figure may achieve Rank with natural
weapons just as they may with any weapon.
The Base Chance for a humanoid to strike with
their primary hand is their modified Agility × 2
plus Physical Strength over 15. The damage modifier is -4 (+ 1 for every 3 full points of Physical
Strength over 15).
Figures with Rank 3 or more Unarmed may kick
rather than striking with their hands, enabling them
to attack with their hands full. They may attempt to
Trip with their feet; the normal Unarmed Base
Chance and damage apply. They may also use a
kick as their secondary weapon for a Multiple
Strike Attack.

6.15 Multi-hex Figures
Many figures will occupy more than one hex on
the tactical display. Their size necessitates alterations in the resolution of movement and combat.
Multi-hex figures have three types of hexes surrounding them: Front, Rear and Flank. The exact
configuration of Front, Rear and Flank hexes varies
with the size of the figure. Front and Rear hexes
function in the same way for them as for any other
figure. Figures in Flank hexes gain a bonus to
strike (see §57.1), and are not in the Multi-hex
figure’s Melee Zone, but do not gain the advantages of a Rear attack.
A multi-hex figure may move in any way so that its
head enters any Front hex, and may move up to its
full TMR in this fashion. At the end of its move,
the figure must be unambiguously oriented towards
one hex vertex. A reduction of 1 hex is suggested
for each 120◦ turn.
A multi-hex figure may freely pivot or move into
any hex occupied by a 1-hex figure. The smaller
figure is knocked prone automatically and the
figure may then attempt to trample with a Base
Chance of 40%, doing (D10 + size of the monster
in hexes) damage. Trampling is C-class damage.
Subsequent attacks on the prone figure use the
Trample Base Chance and damage in the Bestiary.
A multi-hex figure in close with smaller figures
does not automatically fall prone.

6.16 Mounted Combat
In mounted combat, the TMR of the figure (mount
and rider combined) is that of the mount; the rider
may not move at all. A rider and mount will occupy the hexes that the mount would normally
occupy (as specified in the Bestiary).
Controlling a mount during combat is dependent
on the rider’s Horsemanship skill. An inexperienced horseman will have an incredibly difficult
time even controlling their mount in a chaotic
melee; it would be better for them to dismount and
fight on foot.
Action Restrictions
Almost any action the figure is capable of while
standing on the ground may be performed while

mounted. They may not (1) use a two-handed
weapon, (2) fire a missile weapon or throw a
weapon while moving, (3) use more than one
weapon at a time. These restrictions are lifted
depending on the Horsemanship Rank on the Rider
(see §29.2 Horsemanship). A figure may always
use a shield and a one-handed weapon while
mounted.

Close Combat
An airborne figure will be pulled from the air and
become prone if their combined PS + AG is less
than that of their ground-based opponent. Otherwise the airborne figure will remain in flight. The
ground-based figure may be lifted from the ground
if the airborne figure has sufficient Physical
Strength and leverage.

On a normal mount, the rider will not be able to
attack figures directly in front of them except with
a spear (or similar long hafted weapon) or any
Ranged weapon. A mounted figure may not attempt a Shield Bash (except against other mounted
figures). However, they may attempt a Mounted
Charge. A rider may freely mount or dismount
when the mount is stationary, by taking a Pass
Action; the difficulty of dismounting when moving
is determined by the GM.

An airborne figure may benefit from making a
charge attack by diving on the target.

Charge
A Charge on a mount is executed in the same manner as a Charge on foot except the amount of
movement prior to the attack may be greater and
the Charge must be in a straight line (no facing
changes allowed).
In addition to the normal charging options, an
unengaged mounted figure may attempt a Mounted
Charge. This requires the mount to move at least
1/2 TMR without changing facing. At the end of
the figure’s movement, they may make a Melee
Attack with a +20 bonus to Strike Chance. If the
figure overstrengths the weapon, the Mount’s TMR
may be added to the rider’s Physical Strength. If
using a Lance, the Mount’s Physical Strength may
be used for the purposes of over-strengthing (§6.11
Additional Damage).
Any act of turning the mount or stopping it after
the Charge will require a Horsemanship Check (see
§29.2). The pulse following any mounted Charge,
the momentum will take the mount past the target
to its full TMR. Any attempt to turn or stop the
mount will occur after that movement is terminated. A failed check will result in the mount continuing on its way.

Casting
If an Adept is flying and the Adept is in all other
ways eligible to cast a spell (has their hands free, is
not out of Fatigue, etc.) they may move up to 1/2
(rounded down) of their TMR and attempt to cast
the spell prior to, during or after their movement.
This also applies to all flying magic-using monsters and Adepts with flying mounts.

6.18 Aquatic Combat
Aquatic Combat may take place between figures at
different depths. Refer to the aerial combat section
for guidelines.
Defence
• Defence caused by natural agility is halved for
non-aquatics.
• Non-magical defence is always halved.
• Magical defences are unaffected.
Weapons
If the character is on a solid surface then the
following applies:
• A class weapons are unaffected
• B & C class weapons have their non-magical base
chances and damage halved.
If the character is floating:
• A class weapons have their non-magical base
chances and damage halved. Exceptions are tridents, javelins, spears.
• B & C class weapons cannot be used. Exceptions
are nets and garottes.

6.17 Aerial Combat

Magical bonuses are unaffected.

Whenever an avian (or any other flying entity) is
airborne, the figure’s height above the ground may
have to be noted.

Close combat is unaffected but the GM can rule
that certain actions are impossible.

Combat Ranges
Hostile figures are regarded as being in adjacent
hexes if the Range between them is less than 10
feet. Hostile figures are in Close Combat if they are
in the same hex and the height difference is 3 feet
or less. For Ranged & Magical Combat, the range
of weapons & spells may be calculated using the
following formula:

No shield rushes are possible with a standard shield
because of water resistance.

A2 + B2 = C2 (Pythagorean) where A is the horizontal distance between the two characters, B is the
difference in their altitude, and C is the range between the figures.

20

Bows and crossbows must be waterproofed. The
effective range of a thrown or missile weapon is
divided by 10.

Evading defence bonuses are 10% + 2% / Rank for
prepared
B & C class weapons. A class weapons are unaffected.

6.19 Magical Combat
See §7.9 Incorporating Magic into Combat for a
summary.

7 MAGIC

7 Magic
There are thirteen sections in Magic:
7.1
7.2
7.3
7.4
7.5
7.6
7.7
7.8
7.9
7.10
7.11
7.12
7.13

Introduction to Magic
How Magic Works
How to Cast Spells
Cast Check Modifiers
Spell Effects
Restrictions on Magic
Backfires
Magic Resistance
Incorporating Magic into Combat
The Colleges of Magic
Magic Descriptions
Spell Descriptions
Storage and Entrapment of Magic

7.1 Introduction to Magic
Magic represents the effects of the unknown forces
that shape and control the worlds. Those who have
talent and knowledge can tap these energies
(known as mana) and shape them to their own
ends. These people are known as Mages. They are
usually either revered or reviled by the normal
population.
There are three types of Magic: Talent, Spell, and
Ritual. Talent Magic operates more or less immediately, while Spell and Ritual Magic require
preparation before taking effect. Spells may be
prepared in seconds or minutes, but Rituals take
hours (and sometimes many weeks) to perform.
There are a number of separate Colleges of Magic.
Each represents a specific type of magic, and each
has a list of Spells, Rituals and Talents available
only to Adepts of that College. Most of the magic
detailed within these rules is College magic.
Definitions:
Active Resistance A special type of Magic Resistance, where the entity can choose to concentrate
their attention on resisting a magical effect, and
thus reduce its Cast Chance by their Magic Resistance. Only some magic is actively resistible.
Adept A member of a College of Magic is known
as an Adept.
Backfire If a spell or ritual is particularly incompetently cast, unpredictable and often dangerous
effects can occur. This is colloquially known as a
backfire.
Branches of Magic There are 3 branches of
Magic: the Thaumaturgies, the theoretical branch
of Magic including the Bardic, E & E, Mind, Naming, Illusion and Binding Colleges; the Elementals,
the naturalistic branch of Magic that includes the
Earth, Water, Fire, Air, Ice and Celestial Colleges;
and the Entities, the old, “dark” branch of Magic
that includes the Necromantic, Rune, Summoning
and Witchcraft Colleges. The Thaumaturgies and
Entities are opposed to each other.
Cast Chance The modified Base Chance of effectively casting a spell or performing a ritual.
Cast Check The game mechanic whereby a
Mage’s player determines the result of an attempted spell or ritual.
Cold Iron All solid metals that are primarily composed of iron ore are termed Cold Iron. This includes both Iron and Steel. Such metals in a liquid
state are not “cold”. Cold Iron inhibits the ability of
Mages to use mana.
College Most magic is divided up into numerous
Colleges, each of which specialise in a type of
magic (e.g. Fire, Necromancy). A Mage who has
joined a particular College is known as an Adept of
that College, and may not belong to another College without first forsaking all knowledge of their
previous College.
Concentration If a spell has a concentration component in its duration, then the Adept must concentrate in order to maintain the spell. A Mage may

only have one concentration spell in effect at any
time.
Consecrated Ground Any ground that has been
consecrated to the “Powers of Light” affects the
Magic Resistance of all within it. There is no College specifically dedicated to the Powers of Light,
because they are, in effect, opposed to the use of
magic. Most temples and monasteries and some
graveyards will be consecrated ground. Barrows,
pagan temples and the dwellings of magical beings
can never be consecrated ground. Undead and
Necromancers suffer special penalties on consecrated ground.
Counterspell A type of spell which helps to protect individuals and areas against the effects of a
particular College of Magic.
Fatigue Cost The amount of energy, in the form of
Fatigue, that a Mage must expend in order to cast a
spell.
General Knowledge All Colleges of Magic have a
body of Spells, Talents and Rituals which are classified as General Knowledge. These magics are
taught to all Adepts of the College during their
initial training.
High Mana An area that is rich in mana is referred
to as a high mana area. Such areas are rare, and
include locations where human sacrifice is practised or where the inter-planar boundaries are
weak, and mana leaks through. Often mountain
tops or clearings in jungles will contain such areas.
They are likely to be well guarded by beasts and
individuals attracted by the mana, including a
larger than usual proportion of fantastical beasts.
Magic is easier to perform in these areas.
Low Mana An area with depleted mana is known
as a low mana area. Most densely populated or
civilised parts of the world are Low Mana, as are
some battle-scarred areas. Magic is harder to cast
in low mana areas.
Mage Any sentient being who can manipulate
mana to produce (often fantastic) results (excluding
racial Talents). A Mage must have a Magical Aptitude characteristic.
Magic Resistance All targets with a Willpower
value have the capacity to resist some magics
directed against them. This ability is their Magic
Resistance, and is a function of their Willpower.
Not all magic is resistible.
Magical Animates Anything that has been animated, except undead, is a magical animate. Some
Magical animates gain a magic resistance. Only
those animates that have a Magical Aptitude or
Willpower gain a Magic Resistance. Those animates that have neither MA nor WP have no resistance to magic, and in addition, may be affected by
spells that affect Entities and those that affect
objects.
Mana The type of energy that is used in all magic.
A Mage must draw upon mana to perform any
magic. If there is no mana present, a Mage cannot
perform any magic.
Object An item wholly composed of never living
or formerly living matter, or some combination
thereof. Objects do not have a Magic Resistance
except when they are Possessions or Magical Animates.
Passive Resistance This is the default Magic Resistance made by all targets with willpower and
operates automatically against all spells that may
be passively resisted. It is possible to stop passively resisting temporarily.
Place of Power Certain places aid the practise of
magic. The most well known places are Earth
places of power, but they exist for all the Elemental
and Entity Colleges (excluding Rune). Such places
are rare, and often co-exist with High Mana areas.
21

Possessions Possessions are those objects held,
carried or otherwise within the personal area of an
Entity. They are affected by those spells that affect
the Entity, and are entitled to the Entity’s Magic
Resistance.
Resistance Check The game mechanic which
determines whether a resisting entity is fully affected by a magical effect.
Ritual Magic Complex procedures and techniques
that require the Mage to spend large amounts of
preparation time (and often ingredients) to complete successfully.
Special Knowledge All Colleges of Magic have a
body of complex or specialised spells and rituals
which are not taught to mere apprentices, but
which are gained with time and effort after the
Adepts prove themselves worthy. These magics are
termed Special Knowledge.
Spell Magic Codified magical formulae that take
anywhere from a few seconds to a minute to perform, require energy from the Mage, and which
result in specific alterations to Natural Law.
Talent Magic Magical abilities that require mana,
but no energy and minimal time from the user.
Many species have racial Talents.

7.2 How Magic Works
There are three types of Magic: Talent, Spell and
Ritual Magic.
Talent Magic is broken into Racial and College
Talents. Talents are common to all members of a
Race or College of which they are a characteristic
part and may never be learned or forgotten, though
they often may be “ranked”. Talents require no
preparation, take a maximum of 5 seconds to utilise, and require no expenditure of energy. All
Talents can be classified as either active or passive.
Passive Talents are always in effect. Active Talents
require a Pass Action to utilise, and often require
rolls to see if they succeed. Racial Talents are
described in Character Generation. College Talents
are discussed in the individual Colleges.
Spell Magic constitutes the great majority of the
magic utilised by Mages. Unless otherwise stated,
all magic mentioned in these rules is Spell Magic.
All Spell Magic has the following characteristics in
common:
Each individual Spell has a defined range, duration, base chance and effect. Spells must usually be
prepared by the Mage through a process of incantation to draw mana to activate the Spell. Spells are
unstable in their workings, and if cast ineptly, may
fail entirely or have unexpected effects on the
vicinity. The casting of a Spell drains energy from
the caster in the form of tiredness Fatigue.
Ritual Magic requires the expenditure of large
blocks of time (usually hours) and usually certain
conditions must be fulfilled while performing the
Ritual. Ritual Magic occasionally requires a large
number of special tools and substances and may be
restricted to particular times or places. Magical
effects from Ritual Magic tend to be more powerful, prolonged or delayed than those of Spells.
Most rituals require a Cast Check to determine
whether the ritual was successful. If not otherwise
stated in the specific ritual description, a ritual may
backfire (roll greater than Base Change + 30) with
similar consequences to a spell. Rituals may also
cause a multiple effect similar to spells.
Material
Some spells and rituals require material components. These materials must be present to perform
the magic. If the spell or ritual also has a Material
Cost then unless stated otherwise in the description, these materials are consumed during the casting of the magic regardless of the success or failure
of the casting.

\\end{multicols}

\section{Magic}

Extended Rituals
Some rituals require a far greater time to perform
than the standard one hour, possibly requiring
weeks or even months. During these rituals the
Adept is not involved in constant concentration.
The Adept may eat, sleep (8 hours a day) and perform other activities requiring less than 2 hours a
day while engaged in a lengthy ritual. During the
extended ritual the Adept can utilise only stored
magic, and that inherent in the ritual.

7.3 How to Cast Spells
Casting a Spell is a complex process.
Preparing Firstly, the spell must be prepared and
mana gathered for the spell. This does not require
any fatigue, and normally carries no risk. However,
it does involve gesticulations and conversationlevel speech, which will be obvious to observers.
The spell may be prepared in 5 seconds, 1 minute,
or multiples of an hour (using Ritual Spell preparation). The length of time taken to prepare the spell
is proportional to the resulting safety of the Mage.
The length of time spent preparing a spell must be
decided upon in advance. Preparing a spell is subject to the restrictions mentioned in §7.6.
Casting Once prepared, the spell is Cast by an
expenditure of Energy in the form of tiredness
Fatigue, used to shape and direct the magic. This
takes 5 seconds. Once cast, a spell will either impact upon its target or fail. If the spell impacts on
its target, it may be partially or wholly resisted. If
the spell fails, it may backfire (see §7.7). If the
spell is cast particularly competently, it may be
especially effective.
Casting Mechanics
Preparation The Mage’s player announces the
spell and length of preparation (either 5 seconds, 1
minute, or a number of hours). They may break off
their preparation at any time and abort the casting
process. A spell must be used immediately upon
being prepared or it is dissipated and the preparation must be restarted before it can be Cast. Only
one spell can be prepared at any one time. At the
end of the preparation, the Mage is aware of the
state of the surrounding mana. During combat,
Preparing is a Pass Action.
Casting The Mage’s player announces the spell, its
target, and any additional options desired (such as
lowering the Rank of some attributes). During
combat, Casting is a Fire Action.
Cast Check The player then modifies the Base
Chance of the spell due to current conditions to
produce a Cast Chance as a percentage. This Cast
Chance is then compared with a D100 roll. If the
die roll is less than or equal to the Cast Chance, the
spell works. If the die roll is less than 5% of the
Cast Chance, the spell succeeds with a “triple
effect”. If the die roll is between 6% and 15% of
the Cast Chance, the spell succeeds with a “double
effect”. If the die roll is more than 30 higher than
the Cast Chance (with 5 second preparation), or 40
higher than the Cast Chance (with longer preparation), the spell has not only failed, but Backfired,
as per §7.7.
Fatigue A Mage may not cast a spell unless they
have sufficient Fatigue to pay the expected Fatigue
cost. At the end of the preparation, the state of the
Mana (none, Low, normal or High) is known, and
the Mage may abandon the attempt at that point,
before losing the Fatigue. Whether the spell succeeds, fails or backfires, the Mage must pay the
Fatigue cost. It usually costs 1 Fatigue point to cast
a General Knowledge Spell or 2 Fatigue points to
cast a Special Knowledge Spell. In a High Mana
area, these Fatigue costs are reduced by one. In a
Low Mana area, the Fatigue Cost is doubled.
Success If the Cast Check is a double or triple
effect, the element to be doubled or tripled must be
decided before anything else is resolved. If the
Cast Check is a success or better, the target(s) may
resist the spell, if it is passively resistible (see §7.8
for details). This will reduce or nullify the effects

of the spell, as defined in the individual spell description. See §7.5 to resolve the spell effects. Note
that some backfires will result in partial success.
Failure Nothing occurs (except Fatigue loss). If
the result is a backfire, consult the Backfire Table
§53.

7.4 Cast Check Modifiers
In addition to the individual College modifiers, all
Mages receive the following modifiers whenever
engaging in Spell casting:
Each point the Mage’s MA is greater than 15
Each point the Mage’s MA is less than 15
Each Rank the Mage has with the spell they
are casting
Each hour the Mage engages in Ritual Spell
preparation

+1
-1
+3
+3

7.5 Spell Effects
Spells which are successfully cast on valid targets
immediately take effect on those targets (unless
explicitly stated within the spell description). If the
spell is Passively Resistible, as stated in the individual spell, all targets with a Willpower value
may resist at their current Magic Resistance. A
successful resistance may reduce the spell effect
for the target, or even nullify the effects altogether.
In some cases, the duration, damage, or other aspect, may be random, and will need to be determined. If the Cast Check is a double or triple effect, the element to be doubled or tripled must be
decided before anything else is resolved.
All magic works in quanta. Any attribute of any
magic may be performed at any Rank up to the
Mage’s maximum level of skill, but only at a
whole number of Ranks. It is assumed that all
attributes are being cast at maximum, unless it is
otherwise stated before the Cast Check is made.
Damage
Damage due to magic ignores armour and does not
cause stunning, unless a strike check against the
target’s defence is required as part of the spell.
Damage is done to Fatigue, and to Endurance once
Fatigue is exhausted. A single Damage Check will
not “wrap” from Fatigue to Endurance, unless the
total damage exceeds the target’s combined full
Fatigue and Endurance.
Doubles and Triples
There are three characteristics of a spell which can
be increased by the Mage as a result of a spell
causing a double or triple effect: Range, Duration
and Damage.
Whenever a spell is cast for double effect, the
Mage has the option to double one of Range, Duration or Damage. Some spells may not have one or
more of these attributes. Such attributes may not be
affected.
Whenever a spell is cast for triple effect, the Mage
has the option of either tripling one of Range,
Duration or Damage; doubling any two of these
attributes; or decreasing the target’s Magic Resistance by 20.
A Mage may attempt to cast a spell at a target
which is not within range in the hope of achieving
a double or triple effect.

7.6 Restrictions on Magic
Mages are restricted as to where and when they can
employ magic. General restrictions that apply to all
Mages are covered in this section. Specific restrictions applying to Adepts of particular Colleges are
covered in the opening sections of those Colleges.
Cold Iron
A Mage may never prepare or cast a spell or engage in Ritual Magic while in physical contact with
Cold Iron. They may exercise any Racial Talent
Magic, but no learnt Talent Magic. Wearing armour made of Cold Iron, or holding weapons or
tools made of Cold Iron is regarded as being in
physical contact. GM discretion covers all other
22

cases. Several ounces of Cold Iron is required to
cut off the mana flow.
There are several possible means of circumventing
the effects of cold iron:
• The Mage may drop all iron items prior to performing any magic. Note that donning and doffing
armour is very time consuming.
• The Mage may employ weapons, tools and armour that are not metallic (e.g. quarterstaff, leather
armour). Weapons that are normally metallic can
be made out of wood, bone or stone, but their Base
Chance is reduced by 10, their Damage reduced by
2 and their Fumble chance increases (see §6.10). A
similar loss of effectiveness will be experienced
with other tools that are normally iron.
• The Mage may use metallic items that have no
iron content, such as copper, tin and bronze. Such
items cost the same as equivalent iron items, but
they are less effective: weapons do one less point
of damage, and fumble on a 98 to 00 roll (instead
of just a 00); and armour provides two less points
of protection.
• The effects of cold iron can be neutralised by
combining it with precious metal (silver or truesilver). Such items are as effective as Iron items.
Silvered items costs at least 10 times the standard
price and truesilvered items 180 times the cost. A
Mage’s Cast Chances are reduced by 10 if carrying
silvered items. Wearing Cold Iron does not protect
from the effects of magic.
Confinement
A Mage must have the freedom to make the necessary gestures and sounds in order to cast a spell or
perform a ritual. Mute, bound, paralysed, unconscious, stunned, prone or restrained Mages may not
use Spell or Ritual Magic, though Talent Magic is
usually possible. In addition, an Adept must have
at least one hand free and be able to speak clearly
to prepare and cast a spell.
Proper Procedure
A Mage may never employ a type of Magic,
whether a Spell, Ritual or Talent, which they have
not learnt. The Mage also must have whatever
equipment or working materials are specified in the
Spell or Ritual description.
Concentration
A Mage may not cast a spell or perform a ritual if
their concentration is broken. This usually occurs
by being engaged in Melee or Close Combat. Other
types of attack or distraction may also suffice. If an
event is deemed distracting, the Mage’s player
must roll a 4 times Willpower check to maintain
concentration, or the spell or ritual is disrupted.
The concentration required to control spells already
cast, or the concentration required to control an
entity, will not be broken by entering combat or
being attacked. It will only be broken if they are
stunned, knocked out or killed.
Queuing
Spells that have the same effects are not cumulative. If a spell is cast on a target that is already
under the effect of a spell which has the same
effect, then the spell “queues”. This means that,
although the spell is in effect on the target, it has
no effect until the earlier spell is gone.
Spells with any overlapping effects are affected by
this rule. Note that it is the spells’ effects that are
important, not the spell itself. If the spells both
affect one attribute (e.g. defence), they queue. If
two spells affect different attributes (e.g. PS & FT),
they stack. This rule also applies to items. Unless
stated otherwise, any item that contains a magical
effect that can be caused by a spell, cannot be
affected by that spell (e.g. a weapon with a magical
bonus to hit or damage may not stack with Weapon
of Flames).
The duration of the second, queuing, spell is measured from casting, but it only takes effect when the
first spell wears off.

7 MAGIC

7.7 Backfires
Particularly inept Spell casts or Ritual performances may cause backfires. If the Mage’s Cast
Check fails by more than 30 for a 5 second Spell
preparation or Ritual performance, or 40 for a
longer Spell preparation, the Magic backfires. The
Magic will always backfire on a natural roll of 100,
unless the Cast Chance is over 100%, in which
case the Magic fails. The GM rolls on the Backfire
Table (§53) and applies the result. Effects include
extra Fatigue loss, partial or awry magical effects,
and curses and afflictions on the caster.
Backfire Interpretation
The effects are to be interpreted as widely as desired by the GM. All curses and afflictions are
resistible, and partial or awry effects may be also,
depending on the magic. All backfire effects are
cumulative. It may be impossible to apply a specific backfire effect in certain situations. This is
generally described by “No apparent effect”, along
with most subtle afflictions. The effects of a backfire should be kept secret for as long as possible.
In most cases, specific reductions in numerical
ratings are given when a Mage is cursed as a result
of a backfire. However, ancillary effects of the
curse must be determined within the guidelines of
the curse description.

7.8 Magic Resistance
An entity who is the target of a spell may resist the
spell if it is resistible. There are two types of resistance: Active Resistance reduces the Cast Chance
of a spell; Passive Resistance avoids or reduces the
effects of a spell. Magic Resistance is used for both
Active and Passive Resistance. It is equal to an
entity’s Willpower, modified as follows:
Target and caster are of the same
+5
Branch of Magic
Target and caster are of opposed
-5
Branches of Magic
Target is sentient but not a Mage
+20
Target is in a consecrated area
+50
Target has on, or is in the area of, the +30 + 3
appropriate counterspell
per Rank
Spell had triple effect and the caster
-20
chose to affect Magic Resistance
Branches of Magic are covered in §7.10. Sentient
entities are those with an MA of 0 or greater.
Counterspells are covered in §10.2. Triple Effects
are covered in §7.5. Consecration is not covered in
these rules.
Purification, certain other spells, and many items
also affect Magic Resistance.
If an entity successfully resists a targeted spell,
then they will be aware that they have resisted
some Magic. While the spell did not directly affect
the entity, their aura is marked sufficiently that the
last spell to impact is determinable. Note that only
spells which target possessions or entities are noticeable in this way. Spells which affect an area
cannot be detected when a character resists them.
Objects cannot normally resist Magic. However, if
the object is an entity’s possession, and the spell
can be passively resisted, then the entity may apply
their normal resistance.
Passive Resistance
When a spell that is passively resistible impacts on
an entity, the entity may attempt to resist the effects of the spell. This is known as Passive Resistance. The player must roll D100. If the die roll is
less than or equal to the entity’s Magic Resistance,
the spell’s effects are reduced or nullified. Passive
Resistance is an automatic bodily function which
occurs regardless of whether an entity is conscious
or not. If an entity re-encounters an area effect
magic, they must re-resist, whether they successfully resisted last time or not. At the start of a
pulse, an entity may choose to not resist. For the
remainder of the pulse, the entity may not passively resist any spells, unless they become stunned

or unconscious. While choosing to not resist, an
entity may only perform a pass action.
Active Resistance
An entity can choose to actively resist another
entity. When an entity attempts to cast or trigger a
spell which is actively resistible, then the highest
Magic Resistance of any target who is actively
resisting the entity is subtracted from the Cast
Chance. Note that Active Resistance is only effective if the entity who is actively resisting is a target, or in the area of effect of the spell. In combat,
Active Resistance is a Pass Action. An entity must
concentrate on the caster in order to actively resist
them. Anything that can disrupt Spell preparation
can also disrupt Active Resistance. Active Resistance against a spell that is not able to be actively
resisted, or against a spell which is not targeted at
the entity, has no effect.

7.9 Incorporating Magic into Combat
Spell Magic may be employed, usually with 5
second preparation. A Mage must perform a magical pass action to prepare and a cast action to cast a
spell. See §6.4 and §6.6 for what can be done during pass and cast actions in combat.
Talent Magic may be used during combat. Passive
talents operate normally and do not require any
actions to employ. Active talents (e.g. Detect Aura)
require a pulse to implement. An entity may actively resist a spell during combat by implementing
a pass action. They may lower passive resistance
during any pass action. Triggering an invested item
takes one full pulse irrespective of the number of
actions that can be performed.
Ritual Magic is difficult to employ during combat.
During each pulse that noisy or dangerous events
take place, the Mage may need to make a concentration check (as per §7.6).

7.10 The Colleges of Magic
Most Magic is divided into 16 Colleges representing specific types of magic. A Mage may only
employ the powers and spells of one College. If a
Mage belongs to a College, they are known as an
Adept.
Branches of Magic
The Colleges are divided into three Branches of
Magic, as follows:
The Thaumaturgies:
The College of Bardic Magics
The College of Binding and Animating
The College of Ensorcelments and Enchantments
The College of Illusions
The College of Sorceries of the Mind The College
of Naming Incantations
The Elementals:
The College of Air Magics
The College of Celestial Magics
The College of Earth Magics
The College of Fire Magics
The College of Ice Magics
The College of Water Magics
The Entities:
The College of Greater Summonings
The College of Necromantic Conjurations
The College of Rune Magics
The College of Witchcraft
An Adept’s Magic Resistance is affected by their
Branch of Magic. Their resistance against spells of
the same branch as their own is increased by 5,
while their resistance against magic of the opposed
branch is reduced by 5. Thaumaturgies and Entities
are opposed Branches. The Elemental Branch is
not opposed to any group of Magics.
Thaum
Elemental
Entity

Thaum
Same
Neutral
Opposed

Elemental
Neutral
Same
Neutral

Entity
Opposed
Neutral
Same

Restrictions and Modifications
Each College of Magic has its own individual
minimum Magical Aptitude requirement. This
must be met at the time any entity becomes an
Adept of the college. Many Colleges have restrictions on casting magic further to those in §7.6.
These are specified in the first sub-section of each
College.
Most Colleges are subject to certain modifications
to Cast Chances in addition to §7.4. These are
specified in the second sub-section of each College.
Learning College Magic
An Adept is assumed to have mastered all of the
General Knowledge magic of their College upon
the completion of their training. This mastery is at
Rank 0. Special Knowledge magic is not taught to
novices, and can only be acquired by expending
time (and usually money) to learn it to Rank 0.
Most Special Knowledge magic is available at the
Guild, at fixed prices. All General and Special
Knowledge magic may be ranked to Rank 20 by
the expenditure of time and experience.
An Adept must have ranked a spell or ritual to
Rank 6 before they can teach it. They must have
ranked all talents, and general knowledge spells
and rituals to Rank 6 before they can teach a novice their College.
In general, an Adept may never use spells, talents,
or rituals of more than one College of Magic at one
time (except Counterspells). An Adept may change
College but loses all General and Special Knowledge magics from their old College, and must
spend 6 months (and 6,500 EP) learning the ways
of the new College (including characters learning a
college for the first time). Once an Adept has renounced a College, they may never return to it.
Knowledge Limitations
An Adept may only employ Talents, Spells and
Rituals that they know. They may know any number of talents, but may not know more spells and
rituals below Rank 6 than their Magical Aptitude.
They may know an unlimited number of Spells and
Rituals of Rank 6 or higher. All General and Special Knowledge Spells and Rituals of the Adept’s
College, plus whatever non-Colleged magics
known, apply to this limit (except Ritual Spell
Preparation, and as specified in the Namer College). This includes Counterspells of other Colleges.
An entity may not become an Adept of a College
of Magic unless they have the Magical Aptitude to
account for mastery of the General Knowledge
spells and Rituals of that College. This is enumerated in the restrictions of the College. An Adept
may not learn another spell or ritual if they already
know as many spells and rituals (below Rank 6) as
they have points in MA.
If, as a result of a decrease of Magical Aptitude or
spell ranks, the Adept knows more spells and rituals below Rank 6 than their Magical Aptitude, then
they will permanently lose sufficient knowledge to
satisfy this rule, losing the lowest ranked magics
first. They will still remain Adepts of their College,
while sentient.

7.11 Magic Descriptions
The description of all the College Magics work
under certain conventions. The more important of
them follow:
Sub-Sections
Each College description is broken up into
• An Introduction
• Restrictions
• Cast Chance Modifications
• Talents (coded T-#)
• General Knowledge Spells (coded G-#)
• General Knowledge Rituals (coded Q-#)

23

7 MAGIC
• Special Knowledge Spells (coded S-#)
• Special Knowledge Rituals (coded R-#)

7.12 Spell Descriptions
The description of each spell lists its specific effects, range, duration, and other attributes. Each
spell is fully described under the College to which
it belongs. The following information is included:
Rank Modifications Often range, duration and
other effects will be given as “x + y / Rank”. This
means that the characteristic is equal to x, with an
additional y for each Rank attained in the magic.
Unless otherwise noted, the unit of measurement is
the same for x & y. If an increase of y is noted for
each n Ranks, then partial multiples of n do not
count unless specifically stated.
Range The maximum radius (in feet) within which
the Mage can make the spell take effect. This is
always the distance from the Adept. It can be a
linear measurement between Adept and target, or a
radius of effect. Unless explicitly stated, magical
effects will not occur beyond the range of the
magic. In combat, measurements are taken from
the middle of a hex, and rounded upwards.
Duration The maximum length of time that the
spell remains in effect. Spells with a concentration
component will stop as soon as concentration is
broken. Spells that do not require concentration
will persist regardless of the suffering of the Mage,
even unto death. If a spell is cast in the middle of a
pulse, that pulse counts towards its duration.
Experience Multiple The multiple used in conjunction with the rank to be achieved to determine
the experience cost of increasing a Mage’s Rank
with a particular spell.
Base Chance The base percentage chance of succeeding in casting a spell. This may be equal to
some multiple of a characteristic of the Adept. The
characteristic is taken at its current value, multiplied appropriately, and then other modifiers are
added.
Resistance The Magic Resistances (Passive and
Active) which may be applied against the spell by
its targets.

Storage The valid ways that the spell may be
stored, for example, investment, potion and ward.
Target Spells and Rituals are targeted at either an
Area, Object or Entity. Some spells can be cast at
more than one target type. If a multi-target or area
effect spell is actively resisted, the highest Active
Resistance of those to be affected is applied. Area
effect spells are resisted each time that an entity
encounters them. Animated objects count as objects and entities.
Effects The general purpose and consequences of
the spell. Includes potential damage, effects of
resistance, special effects, and any exceptions to
the normal workings of magic.
Difficulty Factor A difficulty factor will sometimes be given to avoid a spell. This is always a
number by which the stated characteristic of the
target is multiplied, before modifiers are added.
Interpretations
Most of the magic in DQ is designed to be flexible
in application, and up to the interpretation of the
GM within the guidelines laid down by the Gods.
The effects and procedures are meant to apply to
humanoid entities of human size. An Incinerate
Spell that would fry a human would do little more
than discomfort a Dragon. To close every loophole
and explain every application would be impossible.
Therefore, these matters of interpretation have
been left to your GM, in the context of their game
and the atmosphere that they are trying to promote.

7.13 Storage and Entrapment of Magic
There are various methods of storing magical effects. Each has different properties and can store
different types of spells.
Potion Spells, Talents and some quasi-magical
skills (e.g. Healer) can be potioned by an Alchemist. For Talents, the imbiber receives the usage of
the talent for a duration dependent on the Rank
(see §30). For skills, see the skill involved. For
spells, imbibing a potioned spell is equivalent to
being the Adept and casting the spell on oneself.
This is normally the only way for self-only spells
to be stored. Target area and target object spells
cannot be potioned, but spells that affect entities

24

may (possibly) be potionable, for example, Necrosis could be potioned, but drinking the potion
would only cause the imbiber to have their internal
organs ruptured. Potions always work, but they
cannot double or triple effect.
Investment Spells effects can often be stored in
items and at a later stage, be triggered. When a
spell is triggered, it is as if the Adept was there and
had just cast the spell (except the spell characteristics such as base chance, range, etc, are fixed at the
time of investment). The entity triggering the spell
gets to choose the target(s) of the spell at the time
of triggering and maintain concentration spells.
Ward A ward is a way of storing a spell within an
object, area or volume so that when some simple
condition is met, the ward is triggered, and the
entities or objects that fulfilled the condition becomes the target for the spell. When a ward is
triggered, it is always successful, but cannot cause
a multiple effect. For a spell to be wardable, it must
have a range or area component (the range may not
be touch, nor self). Wards cannot maintain concentration spells.
Magical Trap Magical effects can be stored in
mechanician traps. Unlike wards, traps have to be
physically triggered. The spell effect, unless an
area effect, will only be targeted on the triggerer.
Magical traps can only store spells that have a nonself range. The spell in a trap can only target an
area or the triggerer. Spells in traps will always
work, but cannot multiple effect.
Shaped Magic Magical items beyond those containing simple invested spells are known as shaped
items. Shaped items come in two flavours, charged
and permanent. Charged items have a number of
charges which diminishes with use. Certain items
are said to have “bound charges”, which means
they behave as an invested item except they can be
recharged. Charged items which cannot be recharged lose their magical status once all charges
have been expended. Permanent shaped items can
come in any shape or form and can defy many of
the usual “rules” of the magical universe.

8 CANTRIPS

8 Cantrips
Cantrips are minor magic effects, which come in
two types. Glamour cantrips are used to entertain
and Household cantrips make life pleasanter. Most
cantrip effects can also be reversed.

8.1 Restrictions
The effects of cantrips are strictly minor in nature
and cannot be used during combat. Any mage can
cast them and no preparation is needed. It costs 1
Fatigue to cast a cantrip.

8.2 Modifiers & Statistics
All minor magics are non-college magic. College
bonuses and penalties do not apply to them. All
Adepts will start off knowing the following cantrips but they may not be ranked or improved, and
do not affect the number of low-ranked spells that
may be learned.
Base Chance: 3 × MA
Double or triple effects are not possible and should
the roll exceed 4 × MA the cantrip will backfire.
Backfires result only in the effect being distinctly
different from the casters intentions. Actively
resisting a cantrip results in it automatically failing.
Any stronger magic of the same type as a cantrip
will immediately replace it and any area counterspell will dispel all cantrips within range. Where an
Adept is casting a cantrip that has effects similar or
related to their college then the effects will be

stronger, or more convincing than normal although
a cantrip can never be as effective as any spell.

ble the object may catch alight (e.g. wick, kindling). The flame will shed light about 2 feet.

8.3 Glamour Cantrips

Cool/Warm Up to a pint of liquid, or a pound of
food, may be cooled or warmed a few degrees.
This might cool hot food to an edible temperature
or warm milk, but not to boiling or freezing extremes.

Glamour cantrips normally last (Base Chance)
minutes. The Adept may choose to reduce the
duration.
Colour The surface colour of one object may be
changed. If cast on an entity (e.g. skin, hair, or
eyes) the result is flat and not particularly life like.
Perfume A faint pleasant (or unpleasant) scent
permeates one hex. Observers will not notice any
specific odour, just that something smells nice (or
bad).
Polish/Tarnish One entity or object is spruced up
so they look smartly dressed or sparkling clean.
This doesn’t make something look more expensive
merely newer and shinier.
Fireworks The Adept may combine any two of
three effects in a 1 foot sphere which appears
within 10 feet. The effects are coloured smoke,
coloured lights, and a small sharp noise (e.g. bang,
tinkle, fizzle).

8.4 Household Cantrips
Household cantrips normally have an immediate
effect and no duration.
Candle A small natural flame appears on an object
and lasts 5 seconds. If cast on something flamma-

25

Clean/Dirty Surface dirt and grime may be shed
from an entity or object to fall around them. If
reversed all the loose dirt within one hex attaches
to the target. Several casts may be required to
removed thick mud or dirt and it will not remove
old stains.
Dry/Dampen The cantrip will render up to one
item of clothing (or an object up to 3 pounds)
either dry to the touch or damp. Several casts may
be required to dry sodden items or cause the target
to drip water.
Insect Repellent Small insects within one hex, or
on one entity leave as fast as they can. This effect
lasts (Base Chance) minutes.
Tie/Untie causes strands to knit together, or fray
apart. This can be used to knot string, repair minor
rips in fabric, repair fraying rope, or tangle or
untangle hair etc. It will not fill in holes or seamlessly join completely separate things together.

9 NAMES & AURAS

9 Names & Auras
All entities have an aura. Those objects that were
once alive retain traces of their living auras. The
strength and composition of the aura reflects the
amount of life-force and magic that the entity or
object possesses and the other properties that are
intrinsically part of their being.
The base element of any aura is strength. The
strength of an aura is always revealed by any
magic that detects or reads auras. The categories of
strength and their relative strengths are:
Aura

Equivalent

0
1
2
3
4

No Aura
Magic (magical wall, illusion)
Formerly Living Composite (chair, stew)
Formerly Living (dead orc, log)
Non-Sentient Animates (stone golem) and
Non-sentient Undead (skeleton)
5
Living Plants (rose, oak)
6
Living Animals (dog, cat)
8
Sentient Animates (flesh golem) and
Sentient Undead (vampire)
9
Living Sentient (human, sphinx)
10
Long Living Sentient (dragon, titan, elf)
15
Avatar (material form of a Demon, etc.)
The strongest aura will be detected. Thus a human
covered by an illusion (without an aura component) will still be detectable as a Living Sentient.
The rest of an aura consists of information intrinsic
to the possessor. This information will vary depending upon the Generic type of the target, but
will either affect the life-force of the target, or be
magical. Only effects that are still current or continuous in nature will be detected. Information may
be gained in descriptive terms, values or even
proportions as appropriate for the type of information being read.
Information that may be gained from a living being
includes: its Generic True Name, its plane of origin, approximately how far it is through its lifespan (e.g. juvenile, 50%, about 100 years old), its
general state of health (e.g. healthy, diseased, 1/2
Endurance), aptitude with a magical ability (e.g.
low overall, Rank of specific ability) and to which
College of magic (if any) it is attuned. These last
two facts are discernible because the skills that
they represent have an effect on the level and type
of magic in the entity’s aura. Relatively little information can be divined regarding the nonmagical learned abilities of an entity. It will be
possible to learn what is the being’s most intrinsic
skill or ability, but lesser skills may not be sufficiently intrinsic so as to have made an impression
on the being’s aura.
Magical auras will include information such as
College, exact name of a spell or other effect, level
of magical ability (low, medium, high, very high),
approximate length of time that the magic has been
in effect (providing it is still present), and approximately how much duration remains.
Any one part of an object will be representative of
the entire object, for example the aura of a toe
sticking out from underneath a blanket will reveal
the same information as if the entire being were
visible. A detached thumb could reveal some information about its former owner, up to the time it
was detached, providing it is intrinsic to the thumb,
for example Generic True Name, plane of origin,
or age — when it was removed. The thumb would
not reveal a magical college or a skill, as these are
properties of living beings only.

Target: Entity, Object, Area, Volume
Effects: If the talent is successful the Adept learns
which of the aura categories they are seeing (with
the strongest taking precedence), and optionally
learns the answer to one question of the Adept’s
choice about the target. The answer to a DA question will consist of a single concept or "bit" of
information. If the information sought is not intrinsic to the target the Adept will receive no answer.
It is not possible to determine the Individual True
Name of an entity. If the Adept achieves a double
or triple effect, the Adept may ask two or three
questions respectively. The process of reading an
aura and asking a question entails concentration
and requires a magical Pass action. Rereading a
previously seen aura, or learning the category of
the aura without asking a question may be combined with other actions.
Only one attempt at Detect Aura may be made per
object. An individual object will change over time,
however, and a fresh attempt may be made when
the aura has changed sufficiently to class it as a
“new” object. If an aura has been successfully read,
the same information will be available without a
new Cast Check being made, until such time that
the object changes sufficiently to be considered a
“new” object.
Aura require direct line of sight to be read. It is not
possible to use Detect Aura through a mirror, crystal ball, Wizard’s Eye, or by any other indirect
means.

9.2 Interpretation and Examples
1. In general the more specific the question the
more specific the answer — the exact nature is left
to the GM’s discretion. For example:
• “What was the last type of magic to impact on
this person?” might get “A spell”.
• “What is the nature of the most recent magical
affect on this person”?" might get “fortitude”.
• “What was the last spell to impact on this person?” might get “Strength of Stone”.
2. All things change over time, even if outwardly
they look the same. While the times may vary from
object to object depending on circumstances, they
tend to follow a seasonal cycle following the seasons. An object whose aura was read in Spring will
have changed sufficiently by Summer to be able to
be DAed again.
3. The single attempt rule also applies to multi-hex
objects. Only one DA may be performed per item
— one cannot DA a wooden floor three times just
because it covers 3 hexes. Attempts to do so will
get the response (even before rolling the dice) —
“you see an aura, formerly living, and the answer is
oak”, that is the same aura they had previously read
until a change of season. After the change of season a new attempt to detect the aura would have to
be made.
4. Once an aura has been detected, it is available to
the detector for the looking, as is the answer to the
question/s asked, until the aura changes. For example, having detected the Aura of a Ward, the Adept
is able to re-read the same information freely. If the
Ward was then dissipated, the aura would vanish.

9.1 Detect Aura Talent

5. A DQ aura is located very close to the skin
surface. Thus a person in a full suit of plate armour, with the visor down, and no part of the body
visible whatsoever would not be able to have their
aura read. Cloth will hide an aura, but make up will
not.

Detect Aura
Range: Special
Experience Multiple: 75
Base Chance: Perception (× 2 for Namers) + 5% /
Rank - 1% for every foot after the first five feet the
target is from Adept
Resist: Active

6. The size of the object will be determined by its
utility. Thus 100’ of corridor could be a single
object, while the next 5 feet, because it has a
wooden floor, or is magical say, may be an object.
The GM may deliberately break things up so as not
to give away too much information from the groupings chosen.
26

9.3 True Names
Generic True Names
All living things have a Generic True Name. This
name is present in their aura. Formerly living
things retain traces of their aura and the name that
they had when alive. All True Names are in an
ancient language, believed by some to be the language that the gods used when they made the
world, and by others to be the original language of
the first mortal race, the Dragons. A translation of
these names into the common tongue yields such
terms as human, elf, tiger, oak, bee, rattlesnake,
and rose.
A Generic True Name identifies the entity or object
as being of a distinct type. Very similar things with
much the same form and function, will probably
have the same Generic True Name. For example,
many small, harmless, plains-dwelling snakes will
have the Generic True Name “Grass Snake” even if
they look somewhat different. A venomous variety
of similar nature will have a different Generic True
Name. Generic True Names may be learnt:
• By means of a Detect Aura.
• From another being who has studied the name
(e.g. knows the name at Rank 0 or higher).
Individual True Names
All sentient entities (player character races, dragons, merpeople, naga, etc.) have an Individual True
Name. This becomes known to them upon reaching
maturity. All sane sentient entities will know their
own Individual True Name and no force — physical or magical — can coerce the entity to reveal it.
They may choose to reveal it, however. An entity
will be called by a given (or use) name, often given
to them by parents or peers. The entity will know
their Individual True Name in their native tongue
and a Namer would have to spend time translating
the Individual Name into the Namer language
before it could be used. Entities will protect their
Individual True Name vigorously as this knowledge can be used both defensively and offensively.
Indeed, the very mention of an entity’s Individual
True Name would strike great fear into their heart.
Player characters and even members of the College
of Naming Incantations will know only their own
Individual True Name upon completing their education. All other Individual True Names must be
acquired and learnt before they can be used. Four
methods exist for acquiring an entity’s Individual
True Name:
• The entity may choose to reveal it.
• It may be obtained from a Namer Demon —
when an entity is born, their Individual True Name
becomes known to some of the Naming Demons
(see the demon descriptions in the Bestiary).
• It may be obtained from another other being who
has studied the Individual True Name, if they
choose to disclose it.
• It may be found in written form — Adepts of
various Colleges have been known to record important entities names in magic tomes.
The aura of the entity contains both the Generic
and Individual True Names, and so the training of
the College of Namers concentrates greatly on the
study and interpretation of auras, from all living
beings and formerly-living objects. Whilst other
Colleges use abilities to detect auras, only the
Namer is trained to make maximum use of the
information gained from perceiving auras. The
Generic Name is instantly identified when a Namer
perceives an aura, although this Name must still be
studied and Ranked to be of use. The information
is coded into the aura in a form that Namers are
trained to recognise, but other Colleges, through
use of the same Detect Aura spell / talent, would
need to inquire specifically to receive the same
information.

9 NAMES & AURAS
The Individual True Name is also coded into the
aura, but is so complex and varied that they cannot
be deciphered and used by even a Namer. If the
Namer is given the Entity’s True Name, then it is
possible for the Namer to identify the auric characteristics that make up the Individual True Name.
The study of these components takes considerable
time due to their complexity.

27

10 COLLEGE MAGIC

10 College Magic
10.1 Introduction
This section includes those spells and rituals that
are common to all Colleges. These spells and rituals are still specific to a College so an Adept can
only learn them from another Adept of the same
College.

10.2 Counterspells
Counterspells act to increase Magic Resistance and
defeat the workings of other magic. Each college
has two of these spells: a General Knowledge
Counterspell, and a Special Knowledge Counterspell. These are specific to the college — a Fire
College Special Counterspell will not affect the
workings of any Earth College spell, nor would it
affect a General Knowledge spell of the Fire College. Adepts learn both Counterspells of their own
college as part of their General Knowledge.
Counterspell
Range: 25 feet + 25 / Rank
Duration: (D10 + 5) minutes +1 / Rank
Experience Multiple:
100 – General Knowledge Counterspell
200 – Special Knowledge Counterspell
Base Chance: 40%
Resist: Passive
Storage: Investment, Ward, Potion, Magical Trap
Target: Entity, Object, Area
Effects: There are several distinct uses for a Counterspell. They are:
1. If cast upon an entity or object the target adds 30
(+ 3 / Rank) to their Magic Resistance when resisting the type of magic to which the Counterspell
applies.
2. If cast upon an area the Counterspell affects a
space 15 feet in diameter. All targets within the
area gain the magic resistance bonus detailed in #1
above and additionally no one within the area may
cast a spell of the type affected. A double or triple
effect cast may increase the area of effect to 25 feet
or 35 feet respectively.
3. If a Counterspell of the appropriate type is cast
over an area under the effects of a Ward, then that
part of the warded area is temporarily deactivated.
When the duration of the Counterspell ends, the
Ward will become active again.
4. An Adept may use a Counterspell to dissipate a
spell that they have cast. They must direct the
Counterspell at the specific spell effect that they
wish to remove. In the case of area effect spells it
is sufficient to cast one Counterspell within the
area — the entire area need not be covered. One
Counterspell will dissipate one spell.
5. Some spells may be removed by any Adept
casting the appropriate Counterspell at them. Only
spells that specifically state that they may be removed this way can be affected. One Counterspell
will dissipate one spell. The Adept must specify
the name of the spell to be removed at the time of
casting.
A target may only be under the effects of the Counterspells of a single College. The target may also
occupy an area under the effects of another College’s Counterspells. Thus the maximum number
of Counterspells that an entity or object may gain
benefit from is four: the General and Special Counterspells of one college cast upon them, and the
General and Special Counterspell of another college upon the area they occupy. Counterspells of
other colleges cast upon them will obey the normal
rules for queuing. A target may only benefit from
one Counterspell against a particular spell. If there
is more than one appropriate Counterspell protecting a target the highest ranked one will have an
effect.
Characters may learn Counterspells from colleges
other than their own, in which case they are con-

sidered Special Knowledge spells. The Counterspells of other colleges are practised at Rank 0 and
may not be ranked.

10.3 General Knowledge Rituals
Ritual Spell Preparation
For each hour spent in preparation, the Base
Chance of a spell is increased by 3 (up to a maximum of 30 if 10 full hours are spent in preparation). If, at any time during the preparation, the
Adept’s concentration is broken, the entire process
must be restarted from scratch or abandoned and
any time previously spent in preparation is lost. An
Adept’s concentration is always broken if combat
occurs during the ritual. The Adept may engage in
no other activity while preparing the spell. The
spell must be cast immediately upon completing
the Ritual Preparation. The Spell Preparation Ritual is a General Knowledge Ritual. An Adept
cannot achieve Rank with this Ritual.
Purification
Duration: 4 hours + 4 / Rank
Experience Multiple: 200
Base Chance: MA + WP + 3% / Rank
Cast Time: 1 hour
Effects: This purification ritual takes one hour and
confers the following benefits:
• 0 (+ 1 / 5 full Ranks) MA.
• 0 (+ 1 / Rank) Magic Resistance.
The additional MA does not count towards any EP
reduction (e.g. ranking general spells or rituals).
This ritual cannot backfire.

10.4 Special Knowledge Rituals
The Ward Ritual
Duration: Until triggered
Experience Multiple: 400
Base Chance: MA + 3% / Rank
Resist: None
Target: Volume
Cast Time: 1 hour
Material: None
Actions: Concentration
Concentration Check: Standard
Effects: An entity may employ Ritual Magic to set
a Ward over an area which they occupy.
A Ward is a spell which is activated by the entry or
exit of objects or entities into the volume it occupies. Whenever an entity wants to create a Ward,
they engage in one or more hours of Ritual Preparation to create the Ward. At the end of the preparation, they check to see if the Ward is set by making a Cast Check. If the Cast Check is successful,
the Ward is set. If the Check is not successful, no
Ward exists.
It is possible to backfire from an attempt to create a
Ward. In such cases, the spell being incorporated
into the Ward backfires immediately. This is rolled
for on the backfire table as though a normal Cast
Check had resulted in the backfire.
Once the Ward is cast, the entry or exit of any
object or entity in the area occupied by the Ward
(determined by the range of the spell incorporated
into the Ward) may trigger the Ward. The area that
the Ward occupies and the range of the spell incorporated into the Ward are identical. This means
that most spells (Range 15’ + 15 rank) have a
minimum sized ward of
30’ diameter (a sphere, centred on the caster, of
15’radius). Also, note that many spells have an
indefinite range and hence cannot be incorporated
into a Ward. This includes spells with ranges of
unlimited, self, or touch and spells that can only
affect the Adept. Once a Ward has been triggered,
it ceases to exist. It takes full effect on the entity(s)
or object(s) that triggered it, but is dissipated thereafter. It takes full but normal effect on the target(s)
– there is no possibility for a double or triple effect
28

(nor for failure or backfire). Note that a Ward set
up for triggering by an entity or object exiting the
area cannot be a targeted spell, since the target
would no longer be within range. The exiting
method of triggering is still useful for area of effect
spells that don’t have a specific target. All Wards
emanate from the exact spot occupied by the individual who cast the Ward (important for determining range). Note that there are a couple of spells
that can have an effect beyond their range.
Spells that are not suitable for incorporation into a
Ward are those which require concentration, or
some other action by the Adept. A Ward always
consists of only one spell. More than one Ward
may not be set over a specific area. Any attempt to
set a Ward on an area that overlaps another Ward
will fail. The Adept will only become aware of this
if they would otherwise have been successful.
Whenever creating a Ward, the Adept must also
specify under what conditions the Ward will be
triggered. They may decide not to limit its effect,
in which case the Ward will be triggered by anyone
or anything entering the area over which it is set, or
they may limit it to affecting specific individuals or
anything in between. Thus, an Adept could set a
Ward that would only be triggered by the entry or
exit of a troll. If a multi-target spell is required to
hit more than one target from a Ward, then the
trigger must include the number of beings. For
example, a spell which affects three targets could
be set up to be triggered by three trolls. The instant
that the third troll entered the volume the Ward
would be triggered, but prior to that any number of
lone trolls could have freely moved through the
Ward.
A Ward, once it is successfully set, cannot be triggered until the caster leaves the volume of the
Ward. Specifics of the triggering mechanism must
be something intrinsic to the object or entity (similar to Detect Aura). Hence a Ward could be set up
to be triggered by a Rank 4 or higher assassin, but
could not be constructed to trigger on the assassin
known as Mac the Knife. In order to affect, or
exclude, specific entities, those entity’s Individual
True Names may be incorporated into the Ward, or
a sufficiently detailed description so as to identify
the individual. If Individual True Names are incorporated then there is no possible way to determine
what those names are, but a Divination would
reveal the number of entities specifically affected,
or excluded. A Ward cannot recognise a specific
object, but merely an instance of an object. For
example, “my sword” could not be included in the
triggering mechanism, but a “magical sword with a
yak-hide grip” could be. A Ward cannot tell the
time so a Ward cannot include such phrases as
“after × minutes” or “at midnight”. Nor does a
Ward have any memory, so in cannot be set up to
be triggered by the third troll to pass.
Once a Ward is set, any entity or object which
could trigger the Ward and which enters the area
occupied by the Ward is automatically subject to
whatever spell was woven into the Ward. Only
those spells known by the caster of the Ward may
be woven into the Ward and they take effect exactly as if the caster of the Ward were present and
casting at the spot occupied by the entity when
they set the Ward. All entities or objects nearby
which would normally be affected by the spell are
subject to its effects when it is cast as a result of
the Ward being triggered. Note that for entities or
objects to be affected they must be within the volume that the Ward occupied, with the exception of
those spells which can affect beyond their range.
Targeted spells can only affect what is incorporated in the triggering mechanism.
Wards are dispelled in one of two ways: either by a
Namer casting the appropriate Counterspell of the
same College incorporated into the Ward, or by
being triggered by an entity or object. The Adept

10 COLLEGE MAGIC
who set the Ward may always counterspell their
own spells, and hence they can dissipate their own
wards by casting a counterspell into it. Wards exist
in perpetuity until dispelled.
The Investment Ritual (Ver 1.2)
Experience Multiple: 300
Base Chance: MA + 3% / Rank
Target: Object
Cast Time: Special
Effects: This ritual allows an Adept to store a spell
that they know in an object or scroll.
Creation of Invested Items The object to be invested will often be in a form appropriate to the
spell that it is to contain (e.g. Spell of Opening
invested into a set of lock picks, or Spell of Enchanting Armour into a set of armour), and of a
size appropriate to the rank and style of that spell.
Note that a staff engraved with the symbols of the
Adept’s college is always considered to be appropriate. The item must weigh at least one ounce.
An Investment Ritual may not be performed on an
object which still carries charges of invested spells,
or a shaped item, or anything made of cold iron.
The Adept may invest any spell that they know at
any rank up to their rank in the spell. The time
taken to perform the ritual is (Rank of Spell - Rank
of Investment Ritual) days per item, minimum 1
day. In this time the Adept may invest up to
Rank/2 (minimum 1) charges, or they may decrease the charges and save 1 day per reduced
charge (minimum of 1 day still applies). They may

never store more than Rank/2 charges in an invested item.

are applied at the moment of triggering, not investing.

As a ritual which takes an extended period of time,
the rules in §7.2 apply. The cost of materials used
in creating an invested item is [spell Rank (minimum 1) × charges × EM of spell / 2] silver pennies. These ingredients are consumed progressively
during the ritual, with the last snatch of incense
being burned as the success (or otherwise) of the
investment is determined.

The effects of a triggered spell are as if the caster
were standing there casting the spell; the triggering
entity is not considered the caster.

The Adept may elect to spend more than this base
cost to increase the chance of the success of the
ritual of investment. For every 200 extra silver
pennies spent on materials the base chance of
success with this ritual is increased by 1%.
If the Investment ritual backfires, then it is as
though the spell being invested has backfired.
Creation of Invested Scrolls The Adept may
instead opt to prepare a scroll (in a language in
which they have a minimum of Rank 8 literacy).
This takes one day per scroll, and costs only half
the usual sum to create. A scroll may only ever
hold one charge, and weighs only two ounces —
however a scroll case sufficient to protect it from
the elements will weigh much more.
Triggering The Base Chance to successfully trigger an invested spell is the cast chance of the spell
at the moment the adept completes the investment
ritual, including all college bonuses, MA, magic,
and environmental conditions. Dice roll modifiers

29

A spell contained in an invested item functions as
any pulse cast spell, with the usual chance of double and treble effect, and of backfire. Any sentient
entity may trigger an invested item, if it is physically possible for them to do so, and if they have
been taught how to trigger it. The Adept who created the item, and any Adept who has divinated it,
know how to trigger it. Teaching someone how to
trigger an item takes 15 minutes.
Triggering always involves speech or a specific
motion to target the item, which may be perceived
by a sufficiently alert observer. An invested item
always takes a full five seconds to trigger.
Triggering a prepared scroll takes a full ten seconds and may only be done by someone literate in
the language in which the scroll is written. The
scroll must be read aloud without interruption. No
teaching is required to trigger a scroll.
Any item or scroll loses a charge when it is triggered regardless of whether or not the triggering is
successful.
Limitations A Namer casting the appropriate
counterspell may drain an invested item of all
magic; refer to Namer T-2 (§17.3) for details.

30

11 NON-COLLEGE MAGIC

11 Non-College Magic
11.1 Introduction
All spells and rituals listed in this section can be
learnt by Adepts of any college.

11.2 Special Knowledge Spells
Geas
Range: The Adept must be able to see and communicate with the target
Duration: Until removed, fulfilled or target dies
Experience Multiple: 250
Base Chance: Always successful (see below)
Resist: Special
Target: Sentient entity
Effects: A geas is an obligation to complete a
quest, an injunction against the performance of a
particular action, or a requirement to respond in the
same fashion to particular stimuli.
The target must acknowledge their acceptance of
the geas. Furthermore, either the Adept must believe that the target deserves the geas, or the target
must truly wish (not forced by physical or magical
means) to have an unmerited geas placed upon
them. The Adept specifies the nature of the geas in
25 words or less, and the GM will use the most
liberal interpretation of that wording to the benefit
of the target. Rank with the geas spell does not
affect the chance of casting the spell as it is always
automatically successful. The Rank equals the
effectiveness of the geas, expressed in percentage
terms. If a geased entity directly contravenes the
letter of a geas, they have a chance of dying equal
to the Rank of the geas. A geased entity will begin
to feel weak or ill when they first take an action
counter to the restriction of the geas, and will become increasingly afflicted until they once more
comply with the geas.
If a quest geas is fulfilled by the geased entity, they
are no longer subject to that geas. The other two
types of geas (for and against a given action) last
indefinitely. A geas can be removed automatically
by the one who placed it. A geased entity will not
attempt to free themselves from the compulsion.
An Adept may attempt to remove a geas if they
have a higher rank than the geas in effect. The
Adept must inscribe a triangle about the geased
entity, and perform the ritual of geas removal for
12 consecutive hours. If the triangle is silver or
truesilver, the geased entity does not suffer the
penalties for ignoring the geas during the ritual.
The Adept attempting to remove the geas has a
success chance equal to five times the difference
between their Rank with geas and the Rank of the
geas being removed. The GM rolls percentile dice:
if the roll is less than or equal to the success percentage, the geas is removed. If the roll is greater
than the success percentage, the Rank of the geas is
increased by one.
Full Geas An Adept with Rank greater than 15
with the geas spell has the power of full geas. A
full geas can be placed upon an entity without their

consent, though it can be passively resisted. Additionally, one with the power of full geas may
automatically remove (without the support of a
triangle and 12 hours of ritual) a geas which is at
least 5 Ranks less than their Rank with the spell.
Major Curse
Range: 20 feet + 15 / Rank
Duration: Until removed or target dies
Experience Multiple: 750
Base Chance: 15%
Resist: Passive (unless a Death-curse)
Target: Entity or Object
Effects: An Adept’s Endurance value is decreased
by one whenever they inflict a major curse upon a
being. Note that when casting a Death-curse this
Endurance point loss is in addition to any possible
Endurance point loss due to resurrection. There are
several types of major curses:
Affliction The Adept may choose to torment or kill
their target. If the effects of the affliction curse are
intended to be deadly, the target may not die as a
direct result of the curse before (24 - Rank) hours
have passed. The following list of sample affliction
curses is provided to give the GM a guideline as to
what major curses should be allowed in their campaign.
1. Target becomes totally blind, deaf or mute.
2. Target becomes senile.
3. Target suffers from a contagious disease (for
example open running sores).
4. Target is transformed into a frog or other small
creature.
5. Target becomes weakened and enfeebled and
must be helped with any physical action.
6. Target falls into century-long sleep.
Ill Luck Add two times the Rank of the major
curse spell to any percentile roll involving the
target or the use any of their abilities. This may not
be applied favourably.
Doom A doom is a pronouncement, by the Adept,
upon an event that will occur in the target’s future
(e.g. “You will die by the hand of a loved one.”).
The statement which must be indefinite will be true
unless removed. The GM should be careful as to
what to allow for dooms.
Death-curse At the moment of their death, an
Adept may automatically cast a major curse (unless
backfire occurs). The being at which it is cast may
not resist the curse. A Deathcurse must be an affliction, ill luck or doom. If a doom, it will be
gasped out with the Adept’s final breath.
Note Lycanthropy is considered a major curse.

11.3 Special Knowledge Rituals
Remove Curse
Duration: Immediate
Experience Multiple: 500

31

Resist: None
Target: A curse
Concentration Check: Standard
Effects: Every curse is rated by the Magical Aptitude (MA) of the Adept who cast it. If the curse is
natural (such as Lycanthropy) it usually has an MA
of 20.
There are two types of curses, minor ones and
major ones. A minor curse causes its victim to
suffer from a non-fatal malediction. Minor curses
come from various sources, for example the spells
Evil Eye (G-9 of the College of Ensorcelments and
Enchantments), the Damnum Minatum (G-1 of the
Witchcraft College) and certain backfires. Major
curses normally come from the Major Curse spell
(§11.2).
When a ritual of curse removal has been completed, the GM rolls percentile dice. If the roll is
less than or equal to the success percentage the
curse is removed. If the roll is between one and
two times the success percentage, the curse remains in effect. If the roll is equal to or greater than
twice the success percentage, the MA of the curse
is increased by one. This ritual does not backfire in
the normal fashion.
Minor Curse
Base Chance: (15 - MA of curse + 5 × Rank )%
Cast Time: 6 hours
Actions: Inscribe symbol of power
The Adept must inscribe a triangle or symbol of
power about the cursed being, and perform this
ritual for six consecutive hours.
Major Curse
Base Chance: (Adept’s MA - MA of curse + 2 ×
Rank)%
Cast Time: 18 hours
Actions: Inscribe symbol of power
The Adept must have a Magical Aptitude greater
than that of the curse. They must inscribe a triangle
or symbol of power about the cursed being, and
perform this ritual for eighteen consecutive hours.
If the major curse is a death-curse, Base Chance is
(Adept’s MA - MA of curse + Rank)%
Precious Metals The use of triangles or symbols
of power fashioned of varying amounts of precious
metals causes an addition to the success percentage, per the following schedule:
Metal

Add

Cost

Silver
+3
1,000 sp
Gold
+7
10,000 sp
Platinum
+ 10 15,000 sp
Truesilver + 15 20,000 sp
The symbol necessary for this ritual is large
enough for the target to sit in, and is inscribed in
the ground. This symbol may be portable.

32

12 COLLEGE OF BARDIC MAGICS

12 The College of Bardic Magics (Ver 2.1)
The College of Bardic Magic deals primarily with
sound, language, rhythm, and the power in music,
particularly the power to soothe, charm, and otherwise affect emotions. Adepts of this College are
generally known as Bards. Many Bards are employed as court musicians, though a number exercise their art as sole practitioners, wandering the
highways and byways of the land.
Almost all Bards have performance skills, often
highly ranked. Alongside this they often have the
Spy skill which is complemented by certain of the
magics of this College. It is not unknown for Bards
to pass themselves off as simple musicians or
entertainers in order to infiltrate unfriendly courts.
It is this fact, and the consequently secretive nature
of this College, which has lead to its relative obscurity.
The principal difference between this college and
all others is that the spell’s verbal component is
usually sung, instead of spoken. Furthermore,
because a musical instrument may be used to enhance the effect, Bardic spells have little or no
somatic component. It is almost impossible for a
Bard to cast quietly.
Traditional Colours
Bards usually wear an item of a particular shade of
deep blue, sometimes known as Harper Blue,
somewhere on their person. Otherwise they tend to
dress to the occasion.
Traditional Symbols
The symbol of the Bardic College is a golden harp.
This may be worn as a brooch or amulet or embossed on the Bard’s instrument if appropriate.

12.1 Restrictions
Adepts of the College of Bardic Magics can only
practise their art in a region where sound can be
heard. It is not possible to practise Bardic magic in
a location where sound is silenced, magically or
otherwise. Bardic magic may be practised underwater, but the results may vary somewhat from
those expected.
The MA requirement for this College is 16.

12.2 Base Chance Modifiers
The following numbers are added to the Base
Chance of performing any talent, spell or ritual of
the College of Bardic Magics:
For each Rank of Troubadour (Singing)
(This bonus is only applicable when spell is
sung)
Area is acoustically excellent
Area deadens sound
All modifiers are cumulative.

+1

+5
-5

12.3 Talents
Concealed Casting (T-1)
Experience Multiple: 150
Effects: The Adept is able to conceal the casting of
a spell within the words of a mundane song. If an
observer is listening intently the observer may
make a (2 × PC - Rank) check to notice. The bard
may not make any movements which are inappropriate to the song being sung. All aspects of the
spell (including Base Chance) must be cast at the
lower of this talent and the spell’s Rank and the
Base Chance is further reduced by 20%. This Talent only functions with spells of the Bardic College.
Enhanced Hearing (T-2)
Experience Multiple: 75
Effects: The Adept is able to hear sounds too faint
to be heard normally. The Rank of this talent
should be added to the Adept’s perception for the
purposes of detecting sounds only. This talent can
be activated and deactivated at will, and the Adept
may be temporarily deafened by loud noises while
it is activated.

Melodic Memory (T-3)
Experience Multiple: 75
Effects: The Adept can attempt to commit to memory any sounds that they can hear. The Adept must
concentrate to activate this talent, and the Adept’s
player should represent this by taking down key
words and phrases. By means of this talent the
Adept can memorise music, dramas, dialogue etc.
even if they cannot understand them. The Adept’s
chance of success to recall the sounds is 2 × PC (+
5 / Rank) - 5 / week since the sounds were heard. If
the Adept rolls above their Base Chance the higher
the roll the greater the degree of error.
Project Voice (T-4)
Experience Multiple: 75
Effects: The talent allows the Adept to project their
voice so that it may be heard clearly everywhere
within 25 feet (+ 25 / Rank).

12.4 General Knowledge Spells
Clairaudience (G-1)
Range: 15 feet + 15 / Rank
Duration: 5 minutes + 5 / Rank
Experience Multiple: 200
Base Chance: 15%
Resist: None
Storage: Potion
Target: Self
Effects: The Adept creates an invisible, intangible
ear that can be moved about within the spell’s
range. The ear appears in the same hex as the Bard,
and operates as a normal ear except that it is not
physically attached to the Bard. The Adept may
move the ear at a TMR of up to the Rank of the
spell, taking pass actions to do so. The ear may be
detected by Witchsight or similar means of detection. If the ear takes any magical damage (it may
be struck by a magical weapon) it is destroyed and
the Adept is stunned. It cannot be used to target
spells. In the area of a Confusion of Tongues spell
the Adept will only hear meaningless jumbled
words. Clairaudience will not penetrate the volume
of a Shell of Silence spell.
Enchant Instrument (G-2)
Range: Touch
Duration: 10 minutes + 10 / Rank
Experience Multiple: 100
Base Chance: 25%
Resist: None
Storage: Investment
Target: Musical Instrument
Effects: With this spell the Adept draws on the
resonance left in an instrument from it having been
played. The spell enchants one musical instruments
so that a being may play it at an effective Troubadour Rank equal to 0 + (1 / 3 or fraction ranks),
whether or not they know how to play that form of
instrument. This Rank may not exceed the maximum Rank the instrument being targeted has previously been played at.
Ethereal Orchestra (G-3)
Range: 25 feet + 25 / Rank
Duration: 5 minutes + 5 / Rank
Experience Multiple: 100
Base Chance: 35%
Resist: None
Storage: Potion
Target: Self
Effects: Creates magical accompaniment for the
Adept’s performance. The accompaniment consists
of the sounds of one instrument plus one per three
full ranks. The Adept must be familiar with music
made by their chosen instruments, but need not be
able to play the instruments themselves. This accompaniment increases the effective Rank of the
performance by 1 (+ 1 / 5 Ranks) without bestowing new skills, or affecting quasi-magical abilities
or base chances. The maximum volume generated
may not exceed that of loud chanting, and the

33

accompaniment can be heard clearly everywhere
within the range of the spell.
Exhortation (G-4)
Range: 25 feet + 25 / Rank
Duration: Concentration: no maximum
Experience Multiple: 200
Base Chance: 15%
Resist: None
Storage: Potion
Target: Self
Effects: By means of this spell the Adept may
attempt to affect the mood of a crowd, inciting a
riot or calming a mob. The reaction roll of the
crowd is modified by +5% ( + 1 / Rank). Once the
Adept ceases to play and/or sing, the spell ceases to
be in effect, but the effects may continue as determined by the GM.
Quietness (G-5)
Range: 1 foot + 1 / Rank
Duration: 30 minutes + 30 / Rank
Experience Multiple: 100
Base Chance: 30%
Resist: None
Storage: Potion, Investment
Target: Entity
Effects: The sounds of the target’s movement are
partially deadened. The effect of this is to add 5%
(+ 1 / Rank) to Stealth. Any Entity attempting to
cast Bardic magic while under the effect of this
spell suffers a 5 penalty to their Base Chance due
to the deadening of sound.
Mockery (G-6)
Range: 15 feet + 15 / Rank
Duration: 10 seconds + 10 / Rank
Experience Multiple: 200
Base Chance: 20%
Resist: Active, Passive
Storage: Investment
Target: Sentient Entity
Effects: The Adept sings or orates a song or poem,
detailing the target’s shortcomings and inflicting
general abuse, insults and mockery. A target that
fails to resist may be embarrassed, shocked, humiliated, indignant or infuriated as appropriate to
their personality. Whatever the emotional effect,
the character is distracted, and may do nothing
other than attempt to silence the Adept (by whatever means they wish) or take pass actions for the
duration of the spell. A check of 1 × WP may be
made at the end of the pulse following the pulse in
which the spell is cast and every pulse thereafter.
Once the target successfully makes a check the
spell ends. The target must be able to hear and
understand the Adept in order for the spell to have
any effect.
Shatter (G-7)
Range: 5 feet + 5 / Rank
Duration: Immediate
Experience Multiple: 200
Base Chance: 25%
Resist: Special
Storage: Investment, Ward, Magical Trap
Target: Object
Effects: The Adept shrieks in an unnatural and
piercing fashion, creating ethereal dissonance that
can destroy objects. The Adept may affect an object of no more than 0.5 pounds (+ 0.5 / Rank).
Additionally, the Rank of this spell determines
what material may be destroyed:
Rank

Material

0–3
4–6
7–9
10–12
13–14
15–16
17–18
19–20

glass, mirror
ceramics
crystals
stone
gems
bone or ivory
hard metal (iron)
soft metal (bronze)

12 COLLEGE OF BARDIC MAGICS
Note that the base MR of objects is 0. Crafted
objects and those made of precious materials receive bonuses to their resistance rolls, as detailed
below. Shaped magical items are immune to the
effects of this spell. Possessions have their owners
MR. These bonuses are cumulative. Crafting modifier: 2 × Artisan Rank; material modifiers: Silver
+5%, Gold +10%, Truesilver +15%, Precious
gems, etc. +20%. The GM should only apply the
modifier of the material that makes up the majority
of the object.
Silent Sounds (G-8)
Range: Touch
Duration: 10 minutes + 10 minutes / Rank
Experience Multiple: 100
Base Chance: 35%
Resist: Active, Passive
Storage: Investment, Ward, Magical Trap
Target: Object
Effects: The target object generates sounds which
are almost inaudible, but affect living entities
within 5 feet (+ 5 / Rank) rendering them either
edgy and more susceptible to fear and awe, or
tranquil and less prone to fear and awe. Affected
beings either add or subtract 1 (+ 1 / Rank) to rolls
made on the fright or awe tables as appropriate.
Soothe the Savage Beast (G-9)
Range: 5 feet + 5 / Rank
Duration: Concentration: no maximum
Experience Multiple: 200
Base Chance: 25%
Resist: Passive
Storage: Potion
Target: Self
Effects: The Adept plays or sings soothing music
which causes any living non-sentient entities
within or entering the area of effect to resist or be
pacified. Animals that do not resist must make a
check against 2 × WP - Rank of Spell to attack the
Adept, and a check against 2 × WP to initiate any
attack whilst in the area. The creature’s reaction
roll is increased by 10% (+ 2 / Rank), but the reaction roll may not be made to exceed 95 by the
effects of this spell. If any soothed entity is attacked, or the Adept ceases to play and/or sing, the
spell ceases to be in effect.
Speaking Beasts (G-10)
Range: 15 feet + 15 / Rank
Duration: 10 minutes + 10 / Rank
Experience Multiple: 200
Base Chance: 35%
Resist: None
Storage: Investment
Target: Animal
Effects: This spell confers any one language known
to the Adept at Rank 6 or greater, upon an animal
(beast, avian, or aquatic; any non-sentient able to
vocalise sounds) for the duration of the spell, at a
language Rank equal to 1 (+ 1 / 5 Ranks).
Ventriloquism (G-10)
Range: 10 feet + 10 / Rank
Duration: 5 minutes + 5 / Rank
Experience Multiple: 150
Base Chance: 40%
Resist: None
Storage: Potion
Target: Self
Effects: The Adept may project their voice so that
it appears to be emanating from anywhere within
the range of the spell. Also, it may be altered so
that it sounds like any other voice or voices the
Adept has heard and memorised with the Melodic
Memory talent. For every five complete ranks the
Adept can project an extra simultaneous and independent voice.

12.5 General Knowledge Rituals
Implanting Sounds (Q-1)
Range: 10 feet + 10 / Rank
Duration: 1 week + 1 / Rank; Permanent at Rank
20
Experience Multiple: 250

Base Chance: 20% + 4% / Rank
Resist: None
Target: Object or Area
Cast Time: 1 hour
Actions: Perform song or sounds
Concentration Check: None
Effects: This ritual creates a type of ward, implanting a sequence of sounds (which may be words,
spoken or sung) of up to 10 seconds (+ 10 / Rank)
duration into an object or area. The Adept may
determine the triggering conditions, in the same
way as a ward. Additionally, sounds implanted into
an object may have a range of “Touch” thus being
triggered by a tactile cue, usually simply the object
being touched. The sound will be emitted each
time the triggering conditions are met, up to a
maximum of 1 (+ 1 / Rank) times. The volume of
the sound produced may not exceed that of loud
ringing bells. The ritual dissipates at the end of its
duration or if the Bardic General Knowledge counterspell is cast on the target. If the ritual backfires,
the sounds are triggered immediately and will
repeat Rank times, or until counterspelled.
Recitation (Q-2)
Range: Touch
Duration: Special
Experience Multiple: 200
Base Chance: 25% + 4% / Rank
Resist: None
Target: Object, Area or Entity
Cast Time: 1 hour
Actions: None
Concentration Check: Standard
Effects: At the completion of this ritual the Adept
will enter a trance, during which they will recite a
legend or story relating to their current location, to
an object held in their hands, or to an entity they
are touching. If no such story exists, or the ritual
fails, then the Adept will recite some amusing but
trivial song. If the ritual backfires the legend will
be false. The amount of information received is
related to the Bard’s Rank with this ritual. If the
GM prefers this may be played as obtaining an
answer to 3 (+1/3 ranks) short questions related to
the location, object or entity which will be answered in “legendary” terms. This ritual may not
be repeated on a given target more than once per
season.

12.6 Special Knowledge Spells
Charming (S-1)
Range: 50 feet + 5 / Rank
Duration: 1 hour + 1 / Rank
Experience Multiple: 400
Base Chance: 25%
Resist: Active, Passive
Storage: Investment
Target: Entity
Effects: The Adept can influence the actions of an
entity who fails to resist, provided that the target
can hear and understand the Adept. The target will
see the Adept as their true friend and will readily
accept most suggestions emanating from them.
Sworn enemies of the Adept (or of their race) will
not be affected by this spell. Any suggestion that is
not directly and obviously inimical to the target’s
interests (as defined by the GM, but usually limited
to actions that would be injurious or fatal) will be
acted upon 90% of the time. The Adept may only
ensure 100% compliance with a request by making
it an order in which case the target makes an immediate Resistance Check. Any suggestion that
would lead to the target’s injury or death results in
an immediate Resistance Check. If successful the
spell is broken. Otherwise, the target accepts the
order and will carry it out faithfully for the duration of the spell.
Compel Speech (S-2)
Range: Touch
Duration: 1 minute + 1 / Rank
Experience Multiple: 200
Base Chance: 25%
Resist: Active, Passive
34

Storage: Investment, Ward, Magical Trap
Target: Entity
Effects: The target entity must resist, or be compelled to speak continuously on random subjects.
Should anyone put a question to the entity they
must make a 3 × WP - Spell Rank check or speak
to the question. The entity is not compelled to
speak the truth by this spell, or any particular language, and may cast spells.
Comprehending Tongues (S-3)
Range: Self
Duration: 10 minutes + 10 / Rank
Experience Multiple: 300
Base Chance: 15%
Resist: None
Storage: None
Target: Self
Effects: Enables the Adept to speak and understand
one language at a Rank equal to 1 (+ 1 / 4 Ranks).
The Adept must have heard the language that they
wish to speak. The language may be nominated by
name, or by choosing to speak a language that the
Adept has memorised some words and phrases
from using the Talent of Melodic Memory. This
spell allows the Adept to be understood, but confers no other bonuses.
Confusion of Tongues (S-4)
Range: 15 feet + 15 / Rank
Duration: 10 seconds + 10 / Rank
Experience Multiple: 350
Base Chance: 10%
Resist: Active, Passive
Storage: Investment, Ward, Magical Trap
Target: Area
Effects: All entities within the radius of effect of
the spell who fail to resist become unable to communicate in or understand any verbal language
whilst within the affected area. The affected area is
a circle with a radius of 5 feet (+ 5 / Rank). Affected entities may not use the Military Scientist
“time out” ability. They can cast spells and trigger
items, but if casting, the backfire threshold (i.e.
+30 or +40) is decreased by 5% (+ 1 / Rank).
Dance of Swords (S-5)
Range: 5 feet + 5 / Rank
Duration: 30 minutes + 30 / Rank
Experience Multiple: 250
Base Chance: 15%
Resist: None
Storage: Investment, Potion
Target: Entity
Effects: While under the effect of this spell, the
target may cavort and leap with surpassing grace
and extravagance by evoking the magic of the
dance. The target may walk and act in all ways
normally, however, when they enter combat they
may “dance” adding 2 (+2 / Rank) to their Defence
provided they move at least 1 hex each pulse. If the
target is unwillingly confined to a single hex (by
the effect of melee zones for example) the target
must halve the defence bonus due to this spell. If
the target is unable to move freely (in close combat
or stunned) then no defence bonus is awarded. The
target may also subtract the Rank of this spell from
the dice roll for any AG Check solely involving
Agility (e.g. avoid knockdown, leaping pits etc.).
Enthralling Audience (S-6)
Range: 15 feet + 15 / Rank
Duration: 10 minutes + 10 / Rank
Experience Multiple: 200
Base Chance: 25%
Resist: Special
Storage: Invested, Ward
Target: Sentient living entities
Effects: All entities who willingly listen to the song
sung by the Adept will perceive a succession of
evocative images which illustrate the song being
sung. These images are hallucinatory in nature,
having no external reality, but are seen by the
audience. No resistance check needs to be made by
those who do not submit willingly to the effects of
the spell, but they will not perceive the images.

12 COLLEGE OF BARDIC MAGICS
Those who submit to the effects will be enthralled
by the images and may take no other action, unless
and until their concentration is broken by physical
contact with an external entity or force. The Adept
must continue to sing throughout the duration.
Shout of Thunder (S-7)
Range: 15 feet + 15 / Rank
Duration: Immediate
Experience Multiple: 250
Base Chance: 25%
Resist: Active, Passive
Storage: Investment, Ward, Magical Trap
Target: Entity
Effects: The Adept projects a thunderous shout of
rage at the target. If the target fails to resist it suffers [D - 5] (+ 1 / Rank) damage and is automatically stunned. The target will also be deafened for
[D - 5] (+ 1 / 2 Ranks) minutes and will have a
very nasty headache which will make concentration one level more difficult until it wears off or is
cured by a healer. There is a chance equal to 5% +
1 / Rank that this deafness will be permanent until
cured by the arts of a Healer of Rank 4 or greater,
or by the Ritual of Cure Deafness and Muteness.
The sound is clearly audible in the area surrounding the target.
Inspirational Song (S-8)
Range: 15 feet + 5 / Rank
Duration: 20 seconds + 5 / Rank
Experience Multiple: 300
Base Chance: 15%
Resist: None
Storage: None
Target: Sentient Entities
Effects: The Adept sings a song of inspiration
mentioning the name of each target. The spell will
affect 1 target (+ 1/3 or fraction Ranks) who each
gain 1 (+ 1/2 Ranks) to their Strike Chances, and 1
(+ 1 / Rank) to Fear Resistance rolls and Rally
Chances. Targets must be able to hear and understand the Bard. The Adept must sing or orate for
the entire duration of the spell, or the effect ceases.
The Adept may perform other actions as long as
they have no vocal requirement.
Satyr’s Dance (S-9)
Range: 5 feet + 5 / Rank
Duration: Concentration: Maximum 10 seconds +
10 / Rank
Experience Multiple: 450
Base Chance: 10%
Resist: Active, Passive
Storage: Potion
Target: Self
Effects: The Adept sings and dances in a hypnotic
pattern, entrancing all in range who fail to resist,
and forcing them to join in the dance. All dancers
(including the Adept) may move at 1/2 TMR
(round up), but are unable to take any other action.
The area of effect moves with the Adept and all
victims will attempt to stay within range. Any
entity that enters the area of effect must resist or
suffer the same fate. The Adept must be seen and
heard for a target to be effected. Victims of this
song get an additional resistance roll every pulse
that they are attacked or restrained, or somehow
prevented from being within the area of effect.
Shell of Silence (S-10)
Range: 10 feet + 1 / Rank
Duration: 10 minutes + 10 / Rank
Experience Multiple: 150
Base Chance: 15%
Resist: None
Storage: Investment, Ward, Magical Trap
Target: Area
Effects: The Adept creates an invisible spherical
shell with a diameter equal to 15 feet (+ 5 feet for
every 5 Ranks). Sound created within this shell is
totally inaudible to entities outside the shell and
vice versa. A Clairaudience spell projected from
outside the shell cannot penetrate it.
Silver Tongue (S-11)
Range: Self

Duration: 10 minutes + 10 / Rank
Experience Multiple: 200
Base Chance: 25%
Resist: None
Storage: Potion
Target: Self
Effects: Everything that the Adept says becomes
more convincing and believable. This ability does
not force listeners to believe the Adept, and obvious lies where there is evidence to the contrary will
be quickly dismissed. But in the absence of such
evidence, or when the lie is not blatant, listeners
will tend to take the Adept at their word. All natural or magical abilities that are normally able to
determine whether the Adept is lying or telling the
truth suffer a negative modifier to their Base
Chance of 20% (+ 5 / Rank). Those abilities that
always work or have no BC may be presumed to
have a BC of 100% for these purposes. In addition,
no magical ability is able to coerce the Adept to
speak truthfully.
Siren Song (S-12)
Range: 5 feet + 5 / Rank
Duration: 10 minutes + 10 / Rank
Experience Multiple: 250
Base Chance: 15%
Resist: Active, Passive
Storage: None
Target: Entity
Effects: All entities within the range of the spell
and able to hear the Adept, must resist or feel a
sudden affection for the Adept. They will then be
unable to carry out any action that might harm the
Adept. However, if the Adept carries out any hostile action towards an entity or the entity is rendered no longer able to hear, that entity will be
released from the spell. The Adept must continue
to sing for the duration or the spell ceases to be in
effect.
Slumber Song (S-13)
Range: 5 feet + 5 / Rank
Duration: Until waking
Experience Multiple: 250
Base Chance: 30%
Resist: Special
Storage: Investment
Target: Entity
Effects: The Adept plays a song that affects one
willing entity (+ 1 / 3 or fraction ranks) causing
them to drift into a normal sleep. All affected targets fall asleep in 4 minutes (10 seconds / Rank,
minimum of 30 seconds). The targets will remain
asleep until disturbed or they awaken normally.
The resulting sleep is healing and refreshing allowing the targets to regain 1 (+ 1 / 5 full Ranks) more
FT per hour while under its effect.
Whispering World (S-14)
Range: 100 miles + 100 / Rank
Duration: Immediate
Experience Multiple: 250
Base Chance: 20%
Resist: None
Storage: potion
Target: Self
Effects: The Adept whispers a message consisting
of no more than 5 words (+ 5 / Rank). The message
travels to the recipient, who must be known to the
Adept, taking D10 + (distance travelled / 100)
hours. The recipient’s surroundings (such as trees,
waves or wind) whisper the message to the recipient. There is a once times perception chance that
any Bard engaged in concentration will “overhear”
a whisper which passes by their location.

12.7 Special Knowledge Rituals
Cure Deafness and Muteness (R-1)
Range: Touch
Duration: Immediate
Experience Multiple: 200
Base Chance: 25% + 4% / Rank
Resist: None
Target: Entity
Cast Time: 1 hour
35

Material: None
Actions: Singing
Concentration Check: Standard
Effects: The target is cured of deafness or muteness, whether of natural or magical origin. Only
those born deaf or mute cannot be affected by this
ritual.
Resounding Instrument (R-2)
Range: Touch
Duration: 4 weeks + 2 / Rank
Experience Multiple: 250
Base Chance: MA + 4% / Rank
Resist: None
Target: Musical Instrument
Cast Time: 6 Hours
Material: Instrument
Actions: Playing instrument
Concentration Check: None
Effects: This ritual enchants a instrument, which
must be of religious or martial nature, such as
trumpets, horns, bagpipes, drums, bells, or gongs.
The enchanted instrument can be heard at its normal volume (usually loud) throughout an area with
a radius equal to 1 mile (+ 1 / Rank). The effect
may be made permanent if the Adept chooses to
permanently expend a point of Endurance. If the
ritual backfires the instrument will be destroyed in
addition to the normal backfire effect.
Sound of Doom (R-3)
Range: Sight
Duration: Immediate
Experience Multiple: 400
Base Chance: MA + 3% / Rank
Resist: None
Target: Structure
Cast Time: Special
Material: Musical Instrument
Actions: Playing trumpet and walking
Concentration Check: Standard
Effects: The Adept marches around the structure,
within earshot of the walls, playing a musical instrument. The music must be able to be clearly
heard at the structure. The size of structure which
may be encompassed is 50 feet (+ 50 / Rank) in
diameter. The structure begins to shake and vibrate, and at the end of the ritual, if it is successful,
the structure falls apart. The Adept must walk
slowly (1 mph) around the target until they have
completely encircled the target at least once, and
have marched for at least an hour.
The Piper’s Song (R-4)
Range: 30 feet + 30 / Rank
Experience Multiple: 400
Base Chance: MA + 3% / Rank
Resist: None
Target: Self
Cast Time: 1 hour
Material: Instrument (usually pipe)
Actions: Playing instrument and walking
Concentration Check: Standard
Effects: The Adept nominates one type of nonsentient entity which normally forms swarms,
packs or herds (e.g. rats, locusts, wolves, elephants) at the start of this ritual. The size of the
entity which can be affected is dependent on Rank:
Rank 0–5, entities less than 1lb in weight can be
affected; Rank 6–10, entities less than 10lb; Rank
11–15, entities less than 100 lb; Rank 16 and
above, entities greater than 100lb. The Adept then
begins to play a tune which has a compelling effect
on all entities of the target type. The area of effect
moves with the Adept, and as they play and walk,
all of these entities within range will begin to follow the Adept, growing into a horde. At the end of
at least 1 hour the Adept gives a single command
to the horde. The wording of this command may
not exceed 1 word (+ 1/3 or fraction Ranks). The
command will be obeyed for a period of 1 hour (+1
/ Rank).

36

13 COLLEGE OF BINDING AND ANIMATING MAGICS

13 The College of Binding and Animating Magics (Ver 1.2)
Members of the College of Binding and Animating
Magics specialise in the binding, manipulation and
animation of nonliving matter; they are commonly
known as Binders. Binders tend to be less concerned with the theory and philosophy of magic
than members of the other Thaumaturgical Colleges, and are usually more inclined to tinker
around until something works. Binders are often
accused of lacking empathy, because of the amount
of time they spend associating with inanimate
objects. However, many Binders cultivate an impish sense of humour, which their College gives
ample opportunity to develop. The College has
been cloistered until relatively recently, and it is
not well known. Much of the Binders’ knowledge
was lost during the Fall. They are found almost
exclusively in highly developed and civilised areas.

Material Costs This is the cost for magical materials to perform the ritual. The cost of materials for
the golem itself is additional to this.

Most Colleges are scornful of Binders, due to their
reputations as tricksters and their lack of direct
combat magics. They are also mocked for their
strong association with the artisan trades, such as
Carpentry, Sculpting and Smithing. Many Binders
possess at least one such skill, Mechanician, or
Philosopher specialising in engineering, architecture and the like, as all of these abilities complement the College’s mechanistic style of magic

Activate Golem (T-1)
Effects: This talent requires 1 pulse of active concentration. It costs 2 FT to attempt this talent which
activates an inactive golem. If the golem is one
which the Adept has constructed then the talent
automatically works; otherwise the chance of success is the Adept’s chance of creating a golem of
that type. This talent may not be ranked. The adept
who activates a golem is considered its master.

Traditional Colours
Binders do not have traditional colours, but tend to
wear practical work clothes, usually with a leather
apron anchored in place, and bedecked with tools.
Traditional Symbols
The College’s traditional symbol is that of many
cogwheels, each turning others, in an endless
chain. However, the populace generally associates
the College with the Rag and String Golems that
many Binders use as followers and helpers.

13.1 Restrictions
A Magical Aptitude of 17 is required to join the
College. Members of the College of Binding and
Animating Magics may operate without restriction.

13.2 Base Chance Modifiers
The following modifiers are cumulative with all
other modifiers (including those specified in §7.4).
Rituals: Per 10% extra spent on Ritual mate+1
rials
Spells: Per hour of Ritual Spell Preparation
+1
Per Rank in Mechanician or Philosopher
+1
Note: Rank in Mechanician OR Philosopher may
be applied, not both.

13.3 Golem Definitions
Assistants An assistant may be used to provide
skills or abilities which the Binder does not possess. The assistant must be present throughout the
entire ritual. If using an assistant’s skill to craft a
golem, the assistant’s rank may affect the PB of the
golem but not the base chance of performing the
ritual.
Construction Time Each golem type has a base
construction time. This is multiplied by the height
of the golem in feet (round up). This assumes that
the adept has the requisite quantity of materials on
hand and the tools or ability to shape the material.
Crafting Golems To perform a Shaping Golem
ritual the binder must have all of the materials on
hand. They must also have the tools and skill to
shape the material, or an assistant with the appropriate tools and skills.
Magical Materials The following statistics are for
golems made of non-magical materials. Golems
made of magical, enchanted or formerly enchanted
materials are less predictable. A Binder attempting
to use enchanted materials should exercise the
utmost caution. Some golems may turn out fine,
possibly even with beneficial side-effects, others
may be actively inimical.

Upgrades When an Adept increases their rank in a
golem ritual they may upgrade existing golems to
their new rank. This requires the performance of
the ritual. The time required is the base time for the
golem (no matter how big it is).
Weathering / Deterioration When a golem is
crafted or animated, the magic involved does not
provide any sort of protection from normal weathering (i.e. Clay will dry and crumble, cloth will get
torn, wood will rot, iron will rust, etc). Separate
preservation magics may be cast on a golem to
reduce or prevent normal weathering.

13.4 Talents

Detect Enchantment (T-2)
Range: 30 feet (+ 5 / Rank)
Experience Multiple: 50
Base Chance: PC + 3% / Rank
Effects: This talent determines whether an item,
person, or area in line-of-sight and within range is
currently under an enchantment or magical effect.
The Adept can tell whether the enchantment is a
current spell or ritual, a warded, invested, or permanent effect, is contained in a trap or is a curse. A
double effect will reveal the general nature of the
spell (eg. defensive, summoning, damaging) and
the effective Rank or remaining duration of the
enchantment. In addition, a triple effect will tell the
Adept the exact name of the magic (eg. Wall of
Bones, Hellfire), or the College of the spell.
Once the initial, most recent, magical effect has
been successfully detected, older enchantments on
a target with multiple layers of magic may also be
detected. This may be continued while the Adept
continues to succeed in detecting Enchantments.
However, only one attempt per quarter may be
made to detect any given enchantment. If a new
enchantment occurs, the Adept may attempt to
detect it, though this in no way affects the status of
the old layer. If an old, unsuccessfully detected
enchantment expires, the Adept may attempt to
detect any newly revealed magic beneath it.
If the Adept is in contact with the target then the
base chance of this talent is improved to PC + 5% /
Rank.

13.5 General Knowledge Spells
Adhesion (G-1)
Range: 10 feet + 10 / Rank
Duration: 5 minutes + 5 / Rank
Experience Multiple: 175
Base Chance: 35%
Resist: Passive
Storage: Investment, Ward, Magical Trap
Target: Volume
Effects: The surface of a non-living solid, up to 1
cubic foot (+ 2 / Rank), is magically enchanted to
adhere to any objects coming into contact with it.
Once stuck, an object is released when the spell’s
duration expires, or the applied PS + D10 exceeds
the spell’s PS of 10 (+ 2 / Rank), which tears the
object free (the durability of some objects may be
less than the force required to tear them free from
the spell). Several individuals may combine their
PS to free an object. Being broken free of the area
of the Adhesion, or resisting a particular contact, in
no way protects the object from becoming stuck if
37

brought into contact with the affected area again,
nor is the spell in any way broken by having an
object torn away; the area remains as adhesive as
before. Except for the crowding of the area, there is
no limit to the number of objects that may be stuck
with this spell. The chance of a person coming into
contact with an adhesive portion of a hex is 10% /
Rank applied to that hex.
Animating Objects (G-2)
Range: 10 feet + 5 / Rank
Duration: 10 minutes + 10 / Rank
Experience Multiple: 300
Base Chance: 20%
Resist: Passive
Storage: Investment, Ward, Magical Trap
Target: Object
Effects: This spell may be used to animate any one
object, of up to 10 pounds (+ 10 / Rank) in weight.
By taking a pass action the Adept may control the
actions of 1 (+ 1 / Rank) previously controlled
animates within range. Once set in motion, the
animates will attempt to carry out the action until
ordered otherwise. The animates will move about
in a manner applicable to their shape. Their TMR
will not exceed 4.
Animates have a nominal PS value of 5 (+ 1 /
Rank) though this will have limited effect on objects made of flimsy materials. Their strike chance
will be no more than 20% (+2% / Rank), with a
maximum of D+2 damage. The animate will cease
to function if the object is destroyed, or a Binder
General Knowledge Counterspell is cast on it. An
animate is an object, but may also be targeted as an
Entity. If an animate is created through the use of a
Ward or Magical Trap, it will receive one command determined when first cast.
Bound Speech (G-3)
Range: 1 foot + 1 / Rank
Duration: 1 day + 1 day / Rank
Experience Multiple: 200
Base Chance: 40%
Resist: Passive
Storage: None
Target: Object, Area of Object
Effects: This spell allows the Adept to record a
verbal message in an object, and defines the conditions under which the message will be replayed.
This spell operates in most respects similarly to the
Ward ritual, except that the range is the range of
this spell or touch, and a message is stored instead
of a spell. The message is replayed exactly as the
Adept recorded it, and may contain any verbalisations that the Adept is capable of. The message
may not exceed 5 words (+ 3 words / Rank).
Unlike the Ward ritual, the message may be triggered one additional time per two full Ranks.
Durability (G-4)
Range: 10 feet + 5 / Rank
Duration: 10 minutes + 10 / Rank
Experience Multiple: 250
Base Chance: 25%
Resist: None
Storage: Investment, Ward, Magical Trap
Target: Object
Effects: One object weighing up to 2 pounds (+ 2 /
Rank) may be made more resilient and less susceptible to damage. The item becomes almost as
strong as steel of the same thickness, without losing any flexibility. The item cannot be broken
unless exposed to stresses beyond that which steel
could withstand, given the object’s size and shape.
The strength is improved with rank to a maximum
of slightly stronger than steel.
This spell does not protect against soiling, corrosion or fire or any forms of damage other than
physical stress.
A weapon treated with this spell allows the wielder
to add 2% per Rank of the spell to any roll to save
the weapon from breaking. Armour protected by

13 COLLEGE OF BINDING AND ANIMATING MAGICS
this spell will have 2 extra Protection ( + 1 / 10 full
ranks) to a maximum of the equivalent Steel armour. At Rank 20, 1 may be added to this maximum.
Note that the added Protection replaces (rather than
adds to) any other Protection bonuses due to a
material’s strength (eg Armourer bonuses). If the
item is broken, or armour suffers damage from a
Specific Grievous, the magic is dispelled.
Mending (G-5)
Range: Touch
Duration: Immediate
Experience Multiple: 150
Base Chance: 40%
Resist: None
Storage: Investment
Target: Parts of an Object, Golem
Effects: Any single object weighing up to 10
pounds (+ 10 / Rank), or one Golem, can be
mended. A mended object becomes exactly as it
was before it was broken or deformed. Any pieces
missing when the spell is cast will remain missing
when the object is mended. Mending used to fix
objects with an effective Artisan Rank greater than
that the Adept possesses will degrade the object’s
effective rank. Magical items made mundane
through breaking will remain mundane even after
the use of a Mending. The Spell of Mending may
be used on a living creature that has been transformed into stone, and subsequently broken. This
requires that the Binder be a Sculptor of at least
Rank 8. A Spell of Mending may be used to repair
Golems. It may repair either a Specific Grievous
injury, or all general Endurance damage.
Modify Aura (G-6)
Range: 5 feet + 5 / Rank
Duration: 1 hour + 1 / Rank
Experience Multiple: 100
Base Chance: 30%
Resist: None
Storage: Investment, Ward, Magical Trap
Target: Object, Entity
Effects: This spell allows the Adept to able to
modify the aura strength of any one object or entity. The target object may be up to 20 cubic feet (+
20 / Rank).The strength of the aura increases or
decreases by up to 1 + (1 / 4 Ranks) on the table
below. This spell does not alter the target’s aura in
any other way.
Aura

Equivalent

0
1
2
3
4

No Aura
Magic (magical wall, illusion)
Formerly Living Composite (chair, stew)
Formerly Living (dead orc, log)
Non-Sentient Animates (stone golem) and
Non-sentient Undead (skeleton)
Living Plants (rose, oak)
Living Animals (dog, cat)
Sentient Animates (flesh golem) and
Sentient Undead (vampire)
Living Sentient (human, sphinx)
Long Living Sentient (dragon, titan, elf)
Avatar (material form of a Demon, etc.)

5
6
8
9
10
15

Minor Creation (G-7)
Range: Touch
Duration: 15 minutes + 15 / Rank
Experience Multiple: 250
Base Chance: 40%
Resist: None
Storage: Investment
Target: Object
Effects: The Adept may create a simple, common
object from a larger source of its constituent substances by reaching into the source and withdrawing the object. For example, the Adept may reach
into a tree and produce a staff. One (+ 1 / 5 full
Ranks) different substances may be combined into
a single finished object with this spell. The substance sources are in no way damaged or reduced
by the spell. The created object will be a common
example of its type and may not exceed 1 cubic
foot (+ 1 per Rank) in volume, nor 1 lb (+ 1 lb /

Rank) in weight. The object is physically real with
all normal attributes. The object may not be created
enclosed by or enclosing anything. Complicated or
fine quality objects may not be created without the
appropriate artisan skill. No alchemical, herbal or
other quasi-magical objects may be created. Created food provides no sustenance. At the end of its
duration the item vanishes. The object is a magical
construct and will have a magical aura, regardless
of its constituent materials.
Moulding Elements (G-8)
Range: 30 feet + 10 / Rank
Duration: 5 minutes + 5 / Rank
Experience Multiple: 250
Base Chance: 20%
Resist: None
Storage: None
Target: Volume
Effects: A volume of up to 2 cubic feet (+ 2 /
Rank) containing one of the four material Elements
(air, water, earth, fire) can be moulded by the
Adept into any form desired, and will retain that
form for the duration of the spell. The Adept must
immediately mould the element into the correct
shape with their hands. During this time the Adept
is protected from the effects of the element, which
is as malleable as putty. The time taken to mould
the element is dependent on the size and complexity of the object desired, but will be at least a pulse.
After the initial moulding, the element will retain
its shape, unless something disrupts it, when it will
immediately attempt to reform, possibly around or
on top of the impedance. The shaped element will
have a defined boundary and shape, but will not be
rigid (excluding shaped earth). The spell cannot be
cast over any entity. The Adept may use applicable
artisan skills to shape an object of greater than
Rank 0 quality. Once the spell expires the element
will act normally according to its substance, shape,
and natural laws.
Preservation (G-9)
Range: 5 feet + 1 / Rank
Duration: Special
Experience Multiple: 100
Base Chance: 40%
Resist: Passive
Storage: Ward, Investment, Magical Trap
Target: Volume, a dead / undead Entity
Effects: This spell preserves and protects one animate, dead or undead entity, of up to 100 pounds
(+ 100 / Rank) or a collection of small objects of a
total weight not greater than 1 pound (+ 1 / Rank)
against the effects of time, decay, rust, erosion or
wave action. It does not confer any protection
against magical attacks. It will not suspend time
with regard to resurrection, poisons, curses, etc.
Duration is 4 (+ Rank squared) days, but is permanent at Rank 20.
Transparency (G-10)
Range: 10 feet + 5 / Rank
Duration: 10 minutes + 10 / Rank
Experience Multiple: 175
Base Chance: 30%
Resist: None
Storage: Investment, Ward, Magical Trap
Target: Volume
Effects: This spell causes a volume of non-living
solid material to become as transparent as high
quality glass, but to otherwise retain its original
characteristics. One (+ 1 / Rank) adjacent 1 foot
cubes may be affected. The Adept may cause the
volume to be transparent from only a single direction, by reducing the BC by 10.

13.6 General Knowledge Rituals
Linking Lifeforce (Q-1)
Duration: Permanent
Experience Multiple: 300
Base Chance: 40% + 4% / Rank
Resist: Passive
Target: Object, entity
Cast Time: 1 hour
Material: Object
38

Concentration Check: Standard
Effects: This ritual allows the Adept to bind an
object to the life-force of an entity. The object will
reflect the entity’s physical condition. If the entity
is well, the object will be in perfect order; if the
entity is sick or wounded, the object will appear
appropriately damaged; and if the entity dies, the
object will seem ruined. There is no limitation on
the size or type of object, but it must remain the
entity’s possession during the entire ritual. Traditionally, apples, roses, statues, paintings or diamonds are used. The Life-force Link may be broken by destroying the object, or by the object being
beyond 100 miles (+ 100 / Rank) from the entity.
The death of the entity will not break the Link. If
10,000 (-500 / Rank) is spent on ingredients, the
range is unlimited and the entity and object may
occupy different planes without breaking the Link.
Petrifaction (Q-2)
Duration: Permanent
Experience Multiple: 200
Base Chance: 55% + 3% / Rank
Resist: Active & Passive
Target: Entity
Cast Time: 1 hour
Concentration Check: Standard
Effects: This ritual allows the Adept to either
change one entity to marble-like stone, or turn one
magically petrified entity back to flesh. The entity
must be present for the entirety of the Ritual. Any
possessions of an entity are (un)petrified with the
entity. A petrified entity is not aware of their surroundings, and has time stopped for the purposes
of poison, resurrection, curses and ageing. Petrifaction is not fatal, although a dead entity may still be
petrified. If an entity is damaged after petrifaction,
they may be repaired with the Mending Spell if the
Adept is a Rank 8 Sculptor. Any damage will be
applied when the entity becomes flesh. A petrified
entity weighs 3 times their normal weight.
Shaping Clay Golems (Q-3)
Duration: 3 hours + 3 / Rank
Experience Multiple: 300
Base Chance: 10% + 4% / Rank
Resist: None
Target: Object
Base Construction Time: 2 hours
Material: Clay & Rare Earths Material Cost: 100
sp
Actions: Sculpting a statue
Concentration Check: None
Effects: Turns a quantity of earth into a golem.
Clay golems are susceptible to fire and immersion;
exposure to either will cause D10 FT damage to
the golem every minute. Fire and water based
attacks do double damage. A clay golem will
gradually dry out and become immobile if not kept
moist. In a temperate climate a golem will lose 1
AG per day if no action is taken to prevent this. A
golem which has completely dried out cannot be
reactivated. During construction a clay golem may
be designed to resemble a humanoid; this requires
that the Adept or an assistant has the sculptor skill
and is familiar with the humanoid race. The chance
of the golem passing visual inspection is 40%
(+4/Rank sculptor, +2 / Rank spy). Note that the
golem will not pass close inspection — cold, hard
skin, lifeless face.
Base Materials: Clay or any soft earth. However
clay tends to be the material of preference as golems made of more crumbly earth fall apart easily.
Shaping Rag & String Golems (Q-4)
Duration: 2 hours + 2 / Rank
Experience Multiple: 250
Base Chance: 40% + 3% / Rank
Resist: None
Target: Object
Base Construction Time: 3 hours
Material: Cloth, straw & herbs
Material Cost: 50 sp
Actions: Building a golem
Concentration Check: None

13 COLLEGE OF BINDING AND ANIMATING MAGICS
Effects: Turns a small mannequin into a golem.
Rag & string golems are susceptible to fire; exposure to fire will cause [D-4] FT damage to the
golem every second pulse. Fire based attacks do
double damage.
Base Materials: Cloth, string, rope, straw. Flexible,
easily manipulable materials.

13.7 Special Knowledge Spells
Bubble of Force (S-1)
Range: 5 feet + 1 / Rank
Duration: 20 minutes + 20 / Rank
Experience Multiple: 450
Base Chance: 15%
Resist: Passive
Storage: Investment, Ward, Magical Trap
Target: Area
Effects: The spell causes an iridescent bubble of
force to appear. The substance of the bubble prevents the occupant(s) from interacting with the
world outside the bubble other than by sight and
sound. The bubble is transparent to low levels of
light and sound, but it prevents damaging levels of
either passing through it. The bubble is impervious
from both sides. It also absorbs physical damage,
including that caused by falling. The bubble is
impenetrable to magic — it may not be targeted
through, and area effects will not pass through the
bubble. Inside, it provides a stable, warm and dry
environment, with fresh air to breathe. The surface
of the bubble is hard to the touch. The bubble may
be rolled on a solid surface by an entity within it,
by entities outside, or by a very strong wind or
similar physical force. It will float on water or
mud, lava and similar semi-solid substances, and
may be carried by wind or current. At Ranks 0-10,
the bubble is 5’ in diameter (one hex), at Ranks 1115, it is 10’ across (three hexes), and at Ranks 1620, it is 15’ (seven hexes). All the bubble must
appear within range. The bubble may enclose any
entities and objects completely within the target
hexes, if they fail to resist. Successful resistance
will cause the entity to be gently pushed aside by
the bubble. If the spell is cast in a gas, the gas is
displaced outside the bubble. If cast in a liquid, the
liquid is displaced, and the bubble will bob to the
surface. The spell may not be cast inside a solid
substance, or where there is insufficient room for
the bubble to form without having to displace an
immovable solid. Thus the whole sphere will always appear above ground. The bubble may be
dissipated by anyone casting a Binder Special
Knowledge Counterspell of at least equal Rank.
Disintegration (S-2)
Range: 15 feet + 15 / Rank
Duration: Immediate
Experience Multiple: 300
Base Chance: 20%
Resist: Passive
Storage: Investment, Ward, Magical Trap
Target: Object
Effects: The Adept may disintegrate up to 2 cubic
feet (+ 2 / Rank) of a non-magical object. If the
object has any magical attributes, the spell will fail.
Frictionless Floor (S-3)
Range: 40 feet + 15 / Rank
Duration: 30 seconds + 10 / Rank
Experience Multiple: 300
Base Chance: 20%
Resist: None
Storage: Investment, Ward, Magical Trap
Target: Area
Effects: Up to 20 (+ 10 / Rank) square feet, or 1 (+
1 / 2 Ranks) hexes, of contiguous surface becomes
nearly frictionless, somewhat like wet ice. Footing
becomes treacherous, and handholds non-existent.
Any entity attempting to move on a frictionless
surface must make a 1 × AG check each pulse, or
fall prone. If they do fall, a successful AG check is
required to regain their feet, and while prone, they
will continue in the direction of their last movement until they clear the frictionless surface. The
effect is not normally visible.

Instant Golem (S-4)
Range: 10 feet + 5 / Rank
Duration: 5 minutes + 5 / Rank
Experience Multiple: 200
Base Chance: 30%
Resist: None
Storage: Investment, Ward
Target: Object(s)
Effects: This spell is cast to produce a specific type
of golem with which the adept is familiar (must be
greater than Rank 0 in the appropriate ritual) from
materials within their range.
If a sufficient quantity of materials is not within the
Adept’s range the spell will automatically fail.
When cast the materials within range will animate
and draw together at the point where the majority
of the materials are coming from. A golem will
form in 20 seconds (-1 / Rank). In the pulse after it
has finished forming the golem will be ready to
carry out its master’s commands.
The effective rank of the golem is the rank which
the adept has in the appropriate ritual.
The duration of the golem is the duration of this
spell. At the end of the duration the golem will
collapse and its component materials will reanimate and attempt to return to their previous
location and state.
Note for Investment: When Invested, the effective
rank of the golem is the rank that the investing
adept had in the appropriate ritual at the time of
investment. The type of golem is set at the time of
investment. The created golem will consider the
triggerer to be its master.
Note for Ward: The type and effective rank of the
golem are set at the time of warding. The Adept
may also instill a basic instinct into the golem at
the time of warding (e.g. kill, serve, assist, guard,
dig, eat, etc.), the actions of the golem will be
governed by this instinct. The instinct is always
simple and indiscriminate (e.g. kill: the golem will
attempt to kill anyone or anything that it perceives). The golem’s master is the warding adept.
Instant Petrifaction (S-5)
Range: 10 feet + 5 / Rank
Duration: Special
Experience Multiple: 500
Base Chance: 1%
Resist: Active, Passive
Storage: Investment, Ward, Magical Trap
Target: Entity
Effects: This spell immediately starts to petrify the
target. For the next 5 seconds, all the target’s percentage chances and D10 rolls are halved. At the
end of this time, they must make their resistance
check. If they succeed, they may resume normal
activities, otherwise the petrifaction runs its course,
leaving the target as marble-like stone.
As with Ritual Petrifaction, any possessions of the
target are petrified with the target. A petrified
entity is not aware of their surroundings, and has
time stopped for the purposes of poison, resurrection, curses and ageing. Petrifaction is not fatal,
although a dead entity may still be petrified.
Instilling Flight (S-6)
Range: Touch
Duration: Concentration: Maximum 30 minutes +
30 / Rank
Experience Multiple: 350
Base Chance: 20%
Resist: None
Storage: None
Target: Object
Effects: This spell enables the Adept to instil a
possession of up to 5 lbs (+5 / Rank) with the
power of flight. The spell will dissipate if the object stops being a possession of the Adept, the
Adept loses concentration, or if the object is broken. The Adept may cause the object to fly at 20
miles per hour (+ 2 / Rank). It will take off and
accelerate up to full speed, or halt and land, in a
39

single pulse. The object may support 150 lbs (+ 50
/ Rank) in addition to its own weight. Naturally
flexible or fragile items gain sufficient strength and
rigidity to support the load. Any object or entity
that falls from the flying object will move off in a
random direction. If the object is about to crash
into a surface, it will attempt to land, although
some surfaces may be inappropriate for this (lava,
sheer walls, etc.).
Itemisation (S-7)
Range: Touch
Duration: 1 day + 1 / Rank
Experience Multiple: 300
Base Chance: 25%
Resist: None
Storage: None
Target: Object
Effects: One object is transformed into a representative figurine of itself. This spell will fail if the
target is not freestanding or if the target does not fit
within a 5 foot (+ 1 / Rank) cube. The figurine will
look like the original item but have the structural
strength of soft wood and will have no moving
parts. The figurine will be 1/12th of the size (min 1
inch in its longest dimension) and 1/10th the
weight of the original. Any items contained within
the target will not be affected, and may destroy the
item as it shrinks. The figurine will revert to normal when the spell expires or the figurine is broken.
Making (S-8)
Range: Self
Duration: Concentration: maximum 1 hour + 1 /
Rank
Experience Multiple: 200
Base Chance: 20%
Resist: None
Storage: Potion
Target: Self
Effects: This spell enables the Adept to use their
hands as if they were common tools. The hands do
not change appearance. The Adept may freely
change from the mimicry of one tool to another
without re-casting the spell. This spell does not
affect the damage done by Unarmed Combat. This
spell does not enhance the Adept’s craftsmanship,
nor protect them while they work (except for the
protective properties of the tools they are emulating). The tools that may be mimicked must be
reasonably simple, and include: adze, auger, chisel,
crowbar, hammer, hatchet, level, pickaxe, plane,
pliers, plumb-bob, rock drill, saw, shovel, sickle,
spokeshave, square, tongs and mallet.
Matter Transmutation (S-9)
Range: 2 feet + 2 / Rank
Duration: 30 seconds + 30 / Rank
Experience Multiple: 400
Base Chance: 20%
Resist: Passive
Storage: Investment, Ward, Magical Trap
Target: Volume
Effects: The Adept may transmute a volume of 2
cubic feet (+ 2 / Rank) of non-living matter into
any other matter. The matter must retain its elemental state — solid to solid, liquid to liquid, gas
to gas. The transmuted matter will retain its original size and shape, but assumes all other physical
characteristics of the resulting matter. The Adept
must have some familiarity with the resulting matter.
Possess Golem (S-10)
Range: 10 feet
Duration: 10 minutes + 10 / Rank
Experience Multiple: 250
Base Chance: 35%
Resist: Passive
Storage: Potion
Target: Self and Controlled Golem
Effects: The Adept’s body goes into a coma and
the Adept’s mind goes into the body of the target
golem. The Willpower, Magical Aptitude, Perception and Magic Resistance of the golem are re-

13 COLLEGE OF BINDING AND ANIMATING MAGICS
placed by those of the Adept. The Adept may only
use ranks in abilities which they have, and the
golem has been attuned to.
During the possession the Adept uses the golem’s
senses. The Adept may also speak through the
golem’s mouth.
The Adept may cast while possessing any non-iron
golem; however, the FT cost of the magic is applied to the Adept’s body and the host golem. The
Adept cannot cast if the host golem has no FT;
backfires affect the Adept’s body and the host
golem (if applicable).
If the host golem takes Endurance damage then the
adept’s body takes half of that amount as FT damage. If the host golem is knocked unconscious or
killed then the adept must make a Willpower check
or fall unconscious for D10 minutes; the multiple is
× 2 for unconsciousness or × 1 for death.
Unfastening (S-11)
Range: 10 feet + 10 / Rank
Duration: Immediate
Experience Multiple: 250
Base Chance: 30%
Resist: Passive
Storage: Investment, Ward, Magical Trap
Target: Entity or Object
Effects: This spell unfastens, opens or unties all
closures, buckles, straps, ties, knots, locks and
other fastenings on the target (causing most armour, clothes, and packs to fall off). This will not
undo stitching or weaving, and it will not cause any
damage to the target. If targeted on an entity then
all of their possessions will be affected by the spell.
Wall of Dust and Sand (S-12)
Range: 20 feet + 10 / Rank
Duration: 10 minutes + 10 / Rank
Experience Multiple: 150
Base Chance: 20%
Resist: None
Storage: Investment, Ward, Magical Trap
Target: Area
Effects: This spell summons and binds together
particles of dust and sand to create a wall with the
strength and solidity of sandstone. The wall may be
15’ tall 20’ long 1’ thick, or a ring 10’ high with a
10’ diameter, or a pillar 15’ high with a 4’ diameter. The Adept may increase any dimension by 1’/
Rank. The Adept determines the position and orientation of the Wall, but at least one edge must be
affixed to a solid surface. The fixed edge of the
wall will bind fast to the adjacent surface, and can
support the rest of the wall. Any entity that is in the
area of the wall when it is cast will be ejected to
the closest point outside the wall — if this is impossible, the wall will not appear. The wall may be
destroyed by inflicting 100 points of damage, or a
5’ × 5’ × 5’ hole may be caused by inflicting 50
points of damage.
Wall Walking (S-13)
Range: Self
Duration: 10 seconds + 10 / Rank
Experience Multiple: 450
Base Chance: 10%
Resist: None
Storage: Potion
Target: Self
Effects: This spell allows the Adept to pass through
solid matter by becoming insubstantial. However,
they may not pass through cold iron. The Adept
has a TMR of 1 while within solid matter, and may
move in any direction, including up and down. The
Adept gains no ability to see through solid matter,
but may breathe in it. While under the effects of
this spell, the Adept is virtually immune to physical damage except that inflicted by cold iron, but
conversely cannot harm anyone in melee unless
they use cold iron in return. If the Adept is caught
in a solid object when the spell expires, they lose
the ability to breathe in solid matter, and are
trapped.

13.8 Special Knowledge Rituals
Binding Permanency (R-1)
Range: 5 feet
Duration: 1 Day (+1/Rank) or Permanent
Experience Multiple: 400
Base Chance: 20% + 4% / Rank
Resist: None
Target: Spell or Ritual
Cast Time: 1 hour
Material: Special
Concentration Check: Standard
Effects: This ritual enhances the duration of one of
the Adept’s spells or rituals belonging to the College of Binding and Animating Magics. The magic
must have been cast previously, and have sufficient
duration to last throughout the ritual. This ritual
may not be used with any magics with concentration-based or indefinite, condition based duration.
This ritual may be performed without material
components to change the duration of the target
Spell or Ritual to 1 day (+1 per Rank). Or the
Adept may use the material components to change
the duration to permanent. Material costs for this
ritual are 10,000 sp (400 / Rank) for General
Knowledge Spells, 20,000 sp (800 / Rank) for
Special Knowledge Spells and for Rituals. A spell
that has been enhanced by this ritual may only be
removed by a Ritual of Dissipation.
Investment (R-2)
Effects: Except as noted below, this ritual is identical to the ritual of the same name in the College
Magic — Investment section (§10.4). Adepts of
this College learn this Ritual of Investment instead
of the standard version. The Adept may invest
either a spell of their own or that of a willing Mage
who participates in the full ritual. If the spell being
invested is contributed by another Mage, the
maximum number of charges stored is half the
Rank in the ritual. The maximum investable rank
of the spell is Spell Rank - [ (20 -Ritual Rank) / 2 ].
A spell whose maximum investable rank is reduced
below Rank 0 may not be invested. The Adept may
not have any assistance in creating scrolls.
Item Divination (R-3)
Duration: Immediate
Experience Multiple: 150
Base Chance: 2 × MA + 3% / Rank
Resist: None
Target: Object
Cast Time: 1 hour
Material Cost: 500sp (-25 / Rank)
Concentration Check: Standard
Effects: Through this ritual, the Adept discovers
the exact nature of all enchantments, mechanisms,
curses, side-effects, etc. placed on an item. If an
item has been imbued with an Individual True
Name, the name will not be revealed, but its existence will be. This ritual cannot backfire.
Major Creation (R-4)
Duration: 1 day + 1 / Rank
Experience Multiple: 250
Base Chance: 40 + 3% / Rank
Resist: None
Target: Object
Cast Time: 1 hour
Material: None
Concentration Check: Standard
Effects: The Adept may create an object from a
larger source of its constituent substances by drawing forth the materials from the source. All constituent materials must be within 5 feet (+ 5 /
Rank). Any number of different substances may be
combined into a single finished object with this
ritual. The substance sources are in no way damaged or reduced by the ritual. The created object
may be any object which the adept can see or has
studied and may not exceed 10 cubic feet (+ 10 per
Rank) in volume, nor 50 lbs (+ 50 / Rank) in
weight. The object is physically real with all normal attributes. The object may not be created en-

40

closed by or enclosing anything. Complicated or
fine quality objects may not be created without the
appropriate artisan skill. No alchemical, herbal or
other quasi-magical objects may be created. Created food provides no sustenance. At the end of its
duration the item vanishes. The object is a magical
construct and will have a magical aura, regardless
of its constituent materials.
Shaping Iron Golems (R-5)
Duration: 6 hours + 6 / Rank
Experience Multiple: 350
Base Chance: 20% + 4% / Rank
Resist: None
Target: Object
Base Construction Time: 4 days
Material: Forgeable metal & Rare Earths
Material Cost: 1,000 sp
Actions: Forging a golem
Concentration Check: None
Effects: Turns a quantity of metal into a golem.
When activated an iron golem radiates a faint
glow, especially its eyes, and is warm to the touch.
Base Materials: All metals which are normally
solid at room temperature.
Shaping Stone Golems (R-6)
Duration: 5 hours + 5 / Rank
Experience Multiple: 350
Base Chance: 20% + 4% / Rank
Resist: None
Target: Object
Base Construction Time: 4 days
Material: Stone & Rare Earths Material Cost: 500
sp
Actions: Carving a golem
Concentration Check: None
Effects: Turns a quantity of stone into a golem.
Stone golems are the heaviest of all golems. Stone
golems cannot be stealthy, they grind as they
move. A Stone Golem can usually be heard coming
from 100 yards away.
Base Materials: All types of stone or rock which
are normally solid.
Shaping Wood Golems (R-7)
Duration: 4 hours + 4 / Rank
Experience Multiple: 250
Base Chance: 40% + 3% / Rank
Resist: None
Target: Object
Base Construction Time: 2 days
Material Cost: 100 sp
Actions: Carving a Golem
Concentration Check: None
Effects: Turns a quantity of wood into a golem.
Wood golems are susceptible to fire; exposure to
fire will cause D10 FT Damage to the Golem every
30 seconds. Fire based attacks do double damage.
During construction a wood golem may be designed to resemble a tree, this requires that the
Adept or an assistant has the Herbalist skill. The
chance of the golem passing casual inspection is
40% (+ 5 / Rank Herbalist). Note that the disguise
will only work if the golem is not moving.
Base Materials: All wood types, including wicker,
cane and bamboo, but excluding fossilised woods
and grasses. GMs may rule that unusual wood
types have different properties (e.g. An iron wood
golem might have +1 NA, +1 PS, + 2 EN, -2 TMR,
-4 AG).

13.9 Attuning Golems
When constructing a golem, it may be attuned to
certain skills or magics. The adept may attune the
golem with 1 ability (+1/Rank). Each skill, language, weapon, talent (racial or college), spell, or
ritual counts as 1 ability. What a golem has been
attuned to cannot be changed but it can be increased when a golem is upgraded.

13 COLLEGE OF BINDING AND ANIMATING MAGICS

13.10 Golems in Combat

13.13 Golem Statistics

Bleeding Golems don’t bleed.

Golem MA WP PC PB

Damaged Golems Golems cannot be healed, instead they must be repaired. A golem must have
positive Endurance to be activated.

Clay
Iron
R&S
Stone
Wood

none
none
none
none
none

Size

PS MD AG EN FT TMR

Wgt

Clay
3’
4’
5’
6’
7’
8’
9’
Per Rk

8
10
12
14
16
18
20
1

17
16
15
14
13
12
11
1

14
13
12
11
10
9
8
1

8
10
12
14
16
18
20

12
15
18
21
24
27
30

72
128
200
288
392
512
648

Iron
4’
5’
6’
7’
8’
9’
10’
Per Rk

10
12
14
16
18
20
22
1

19
18
17
16
15
14
13
0.5

15
14
13
12
11
10
9
0.5

20
22
24
26
28
30
32
1

20
22
24
26
28
30
32
0.5

Rag & String
6"
3
1’
4
1’6"
5
2’
6
2’6"
7
Per Rk 0.5

22
21
20
19
18
0.5

20
19
18
17
16
0.5

1
2
3
4
5
0.5

3
4
5
6
7

Stone
5’
6’
7’
8’
9’
10’
11’
12’
Rk

14
13
12
11
10
9
8
7
0.5

12
11
10
9
8
7
6
5
0.5

14 12
16 15
18 18
20 21
22 24
24 27
26 30
28 30
0.5 0.5

Dead Golems A golem which has been killed
cannot be reactivated until the damage is repaired
and the appropriate shaping ritual is performed, as
if doing an upgrade.
Fatigue Recovery Golems always have full fatigue
on activation. They also recover 1 FT per hour
while activated.
Sleep and Charm Golems do not sleep and cannot
be charmed. However, if the Binder is possessing a
golem, they can be slept or charmed by targeting
the golem.
Stun Golems do not stun.
Unconsciousness If a golem’s Endurance is reduced below 1 then it deactivates.
Weapon Ranks A golem may wield weapons. If
the golem has been attuned to a weapon then it
may use that weapon at Rank 0. If that weapon is
also built-in to the golem (or unarmed) then it will
wield the weapon at Rank (Lower of: (Rank in
ritual / 2) or maximum Rank). All other weapons
will be wielded as if unranked. Note that a golem
will not evade or attempt any special manoeuvre
(e.g. trip, disarm) unless possessed by an Adept.

13.11 Golem Intelligence
Golems have an animal level intelligence. Each
golem has an animal equivalent to use as a guideline for their intelligence and temperament when
their masters’ instructions are unclear, incomplete
or no longer applicable.
Golems are somewhat different from their animal
equivalents. They have a weak survival instinct,
and their strongest drive is to follow their masters’
instructions.
Golem Animal Equivalents: Clay = Ox, Rag &
String = Monkey, Iron = Tiger, Stone = Elephant,
Wood = Dog.

13.12 Instructing Golems
Golems must be verbally instructed. Instructing a
golem requires active concentration. A golem will
only take instructions from its master. A golem
will attempt to follow the nature and intent of the
Adept’s instructions.

15
18
21
24
27
30
33
36
1.5

8
18
10
20
12

8
15
12
10
15

NA

5 + 2 / Rk (Sculptor)
8 + Rk (Blacksmith)
10 + Rk (Tailor)
6 +3 / 2 Rk (Sculptor)
10 +3 / 2 Rk (Carving)

-1
-1

+1
+1

-1

+1
+1
+1

176
275
396
539
704
891
1100

x 0.25
x 0.33
x 0.50
x 0.66
x 0.75

0.5
1.0
1.5
2.0
2.5

+1
+1
+1
+2
+2

325
468
637
832
1053
1300
1573
1872

Wood
2’
5
16 14
8 12
-1
24
3’
6
15 13 10 15
-1
54
4’
8
14 13 10 15
-1
96
5’
10 13 12 12 18
150
6’
12 12 11 14 21
216
7’
14 11 10 16 24
294
8’
16 10
9
18 27
+1
384
9’
18
9
8
20 30
+1
486
Per Rk 0.5 0.5 0.5 0.5 0.5
* Per Rk — Bonus to statistic per rank in the ritual.
Always truncate fractions.

41

2
8
0
6
4

42

14 COLLEGE OF ENSORCELMENTS & ENCHANTMENTS

14 The College of Ensorcelments & Enchantments (Ver 1.1)
This College is concerned with general magic, but
especially with charming and enchanting individuals and objects. Practitioners of this college are
often knows an Enchanters.

14.1 Restrictions
Adepts of the College of Ensorcelments and Enchantments may practice their arts without restriction.
The MA requirement for this College is 16.

14.2 Base Chance Modifiers
There are no modifiers to the Base Chance of performing any talent, spell, or ritual of the College
except as listed in magic modifiers or under the
descriptions of the specific spells, talents, and
rituals of the College.

14.3 Talents
Wizardsight (T-1)
Experience Multiple: 150
Effects: The Adept may see objects or entities that
are invisible — they appear to have a slight blue
sheen around them. If the invisibility effect (excluding Walking Unseen) is of a higher Rank than
the Wizardsight, the object or entity may not be
clearly identified or directly magically targeted.

14.4 General Knowledge Spells
Charming (G-1)
Range: 15 feet + 15 / Rank
Duration: 1 hour + 1 / Rank
Experience Multiple: 500
Base Chance: 15%
Resist: Active, Passive
Storage: Investment, Ward
Target: Entity
Effects: The Adept may influence the actions of
any one entity by casting a Spell of Charming over
the individual. If the Generic True Name is known,
the Base Chance is increased by 15. If the Individual True Name is known and used, the Base
Chance is increased by 25.
The victim of the spell will then see the caster as
their true friend and will readily accept most suggestions emanating from them. Sworn enemies of
the caster (or of their race) will not be affected by
this spell unless the victim’s Individual True Name
is spoken and, even then, the duration of the spell
is halved.
The caster may either suggest actions to the victim
or may order them to act in a certain way on pain
of losing the caster’s “friendship”. Any suggestion
that is not directly and obviously inimical to the
victim’s interests (as defined by the GM, but usually limited to actions that would be injurious or
fatal to the victim) will be acted upon 90% of the
time. The caster may only ensure 100% compliance with a request by making it an order based
upon the friendship that binds the victim and caster
together. In such instances, however, the victim
immediately makes a Resistance Check. If they
resist, then the spell is broken. Otherwise, the
victim will accept the order and be 100% faithful
to it for the remainder of the spell.

For each Rank of the spell the Adept may move an
additional 5 pounds or increase the speed by an
additional 3 TMR.
Enchanted Sleep (G-3)
Range: 15 feet + 15 / Rank
Duration: 1 hour + 1 / Rank
Experience Multiple: 250
Base Chance: 15%
Resist: Active, Passive
Storage: Investment, Ward, Magical Trap
Target: Entity
Effects: The Adept may send the target, provided it
normally spends any time sleeping, into a deep
enchanted sleep which will last for the duration of
the spell or until the entity is awoken by another
entity (by being shaken, etc.). The target may not
be wakened if the spell is Rank 10 or higher, but
must continue to sleep until the spell wears off. If a
General Knowledge counterspell of this college is
cast upon an affected target by any Adept then the
spell will immediately dissipate and the target can
then be woken normally.
Walking Unseen (G-4)
Range: 1 foot + 1 / Rank
Duration: 1 hour + 1 / Rank
Experience Multiple: 100
Base Chance: 40%
Resist: None
Storage: Investment, Potion, Ward, Magical Trap
Target: Entity
Effects: The target of this spell may move unnoticed, not invisible. This means that it will not
transmit light. As a consequence the target will cast
a shadow (which may or may not be noticed depending on the lighting etc — even if noticed may
not be connected to the target) and have a reflection in a mirror (or any reflective surface). However the target may not be noticed even if another
entity is looking directly at him/her. It should be
noted that a crystal of vision or similar would
count as looking directly at the target, not as a
reflection. An entity will get a perception check if
the target becomes invasive on that entity’s senses
(e.g. standing in a frontal adjacent hex, or standing
behind the entity with the target’s hands over
his/her eyes). Although the target is not invisible, it
may be detected using any magical means for
detecting invisible entities (e.g. witchsight).
If the target of the spell is touched by another
entity, or that entity’s possessions, then the spell is
broken. The target of the spell may not break it
voluntarily (other than by, for example, touching
another entity). Once broken the spell must be
recast.

Whenever any suggestion is made that would lead
to the victim’s injury or death, another Resistance
Check is made. If the check is unsuccessful, the
victim will accept the suggestion, otherwise the
spell will be broken.

Speaking to Enchanted Creatures (G-5)
Range: 15 feet + 15 / Rank
Duration: 10 minutes + 10 / Rank
Experience Multiple: 100
Base Chance: 40%
Resist: None
Storage: Investment
Target: Entity
Effects: The spell gives the Adept the ability to
speak and understand the language of all magical
creatures. One casting will allow communication
with any fantastical creature, but will not allow the
Adept to comprehend fantastical creatures talking
amongst themselves or to other people affected by
this spell.

Telekinesis (G-2)
Range: 15 feet + 15 / Rank
Duration: 10 seconds + 10 / Rank
Experience Multiple: 300
Base Chance: 20%
Resist: None
Storage: Investment, Ward
Target: Entity or object
Effects: Allows the Adept to lift a target of weight
up to 2 pounds and move it at the rate of 2 TMR.

Location (G-6)
Range: 10 miles + 5 / Rank
Duration: 1 hour + 1 / Rank
Experience Multiple: 200
Base Chance: 15%
Resist: None
Storage: Potion
Target: Self
Effects: The Adept may determine the direction in
which they will find any person or object of their
43

desire which they have previously encountered or
studied and which is within range. The direction
will be indicated by a large glowing arrow, only
visible to the Adept. The arrow will not appear if
the target is not within range. If cast, and the target
is not within range, then the spell is dissipated. The
arrow will not appear should the target come
within range during the period the spell would
normally have been in effect.
Mass Charming (G-7)
Range: 15 feet + 15 / Rank
Duration: Concentration: no maximum
Experience Multiple: 850
Base Chance: 5%
Resist: Active, Passive
Storage: Investment
Target: Entity
Effects: Charms 1 entity per Rank as long as the
Adept maintains their concentration. The spell
takes 3 minutes to take effect and the effects linger
3 to 5 minutes after concentration is broken.
The effects of the spell on the individuals are identical to the effects of Spell of Charming [G-1].
Invisibility (G-8)
Range: 15 feet + 15 / Rank
Duration: 5 minutes + 5 / Rank
Experience Multiple: 450
Base Chance: 45%
Resist: None
Storage: Investment, Potion, Ward
Target: Entity or object
Effects: The target becomes invisible. Unless cast
at rank 16 or higher, the spell ceases whenever the
target makes a strike check in melee or close
(whether successful or not). The target may choose
to end the spell at any time.
An invisible thing does not have a shadow or reflection but is still affected by light (e.g. may still
see, be sunburned, be blinded by Flash of Light,
damaged by Solar Flare etc.).
The possessions of the target are also invisible. An
object’s possessions are anything that is totally
enclosed by that object (e.g. coins in an invisible
chest which is closed, but not one which is open).
If a thing ceases to be a possession, then it is no
longer invisible. Similarly if an object becomes a
possession then the spell will affect it (i.e. an object put down by an invisible entity will become
seen; a coin put in the invisible chest will become
invisible). Note that light may not be a possession
— if the target is carrying a lantern it will be invisible, but the light it emits will not.
Evil Eye (G-9)
Range: Self
Duration: 1 day + 1 / Rank
Experience Multiple: 300
Base Chance: 30%
Resist: None
Storage: Potion, Magical Trap, Investment
Target: Entity
Effects: When the adept casts Evil Eye on themself, a third eye appears in their forehead, which is
normally invisible but is able to be seen by Witchsight and similar effects. When the adept is under
the effect of the Evil Eye all spells they cast on
others are resisted at a penalty of Rank unless the
target is wearing an Amulet of Elder Flowers.

14.5 General Knowledge Rituals
Greater Enchantment (Q-1)
Duration: Special
Experience Multiple: 125
Base Chance: 80% + 1% / Rank
Target: Entity
Cast Time: 1 hour
Material: Black Myrrh (optional)
Material Cost: 200 sp per oz used
Effects: The ritual takes 1 hour and requires that
the Adept first draw a Pentacle within which they

14 COLLEGE OF ENSORCELMENTS & ENCHANTMENTS
and the target(s) must remain during the entire
ritual. The Adept may burn black myrrh during the
ritual to enhance the effects. The target of this
ritual will be either blessed or cursed (Adept’s
choice) with an increase or decrease in their Base
Chance of doing anything or suffering any good or
ill fortune by plus or minus 1 + (1 / Rank).
The ritual will affect the target’s fortune in one of
the following areas:
• Resistance — Magic Resistance, Fear, Fright &
Awe Checks.
• Magic — Talent, Spell, and Ritual Base Chances.
• Combat — Weapon Strike Chances and Stun
Recovery.
• Skills and Statistic Checks — Skill Base Chances
and General Statistic Checks (e.g. 1 × PC or 4 ×
AG).
The default duration is until the next end-of-season
High Holiday. This will enhance one of the above
areas.
The Adept may alternatively cast a form of the
ritual that will affect 5 targets for 1 day (plus 1 day
OR 1 target per 4 full ranks). This will enhance
two of the above areas.
To increase the number of areas affected, the
Adept may burn (1 oz × Ritual Rank) Black Myrhh
per additional area.
If the ritual is used to curse, the curse is minor.
Creating Crystal of Vision (Q-2)
Experience Multiple: 200
Base Chance: 75% + 1% / Rank
Cast Time: 1 hour
Effects: The Adept can create a crystal that acts as
a viewing crystal. They perform the ritual over an
available piece of crystal (the bigger the crystal,
the better the image will ultimately be) in their
possession. The Adept must burn 1 ounce of ambergris during the ritual at a cost of 1,000 Silver
Pennies. The resulting crystal may be used once
per day for 10 minutes + 1 minute per Rank (GMs
should carefully time consultations).
They may then view visions (usually precognitive
in nature) concocted by the GM. At Rank 6 and
above, they may use the crystal to spy into an area
to see what is going on there. The maximum distance from the character to the area being spied
into is 5 miles (+ 15 / Rank). To do so, the Adept
must remain in the same place and take no other
action.
If a crystal ball is used to look at a location then the
ball will only look at the location specified. It
cannot then be commanded to move around. It will
enable the observation of events in that location for
its duration and will then clear. Once started, it will
continue to look at that location for its entire duration. If the user cannot form a clear mental image
for the ball to focus on, or command it to focus on
an unambiguously defined point in space, then it is
the GM’s discretion as to what the ball will show.
These are mystical devices, not telescopes or x-ray
machines. It is not possible to use detection talents
(such as Detect Aura) through a crystal of vision.
Creating Sleep Dust (Q-3)
Experience Multiple: 250
Base Chance: 80% + 1% / Rank
Cast Time: 3 hours
Effects: The Adept must spend 3 hours preparing
and implementing this ritual and must expend
10,000 Silver Pennies (500 / Rank) to purchase the
necessary ingredients prior to making the attempt.
If the attempt fails, the ingredients are ruined and
may not be reused or resold. If the ritual succeeds,
one ounce (a single dose) of sleep dust results.
When thrown in the face of one target it has the
effect of an instant Spell of Enchanted Sleep of the
same Rank as the sleep dust. The sleep dust will
only remain fresh for three weeks after manufacture.

The effects of sleep dust can be passively resisted
by the victim, but with a reduction of 20 to their
Magic Resistance.
Manufacturing Poison Dust (Q-4)
Experience Multiple: 250
Base Chance: 80% + 1% / Rank
Cast Time: 3 hours
Effects: This ritual works in the same manner as Q3 and employs the same numbers for all purposes.
However, instead of causing the victim to fall
asleep, it inflicts [D - 5] + 1 for every 2 (or fraction) ranks damage due to poisoning, on individuals who fail to resist. Like sleep dust, poison dust
must be thrown in the face of the target.

14.6 Special Knowledge Spells
Ventriloquism (S-1)
Range: 90 feet
Duration: 5 minutes + 3 / Rank
Experience Multiple: 100
Base Chance: 60%
Resist: None
Storage: Potion
Target: Self
Effects: The spell allows the Adept to project their
voice and alter it so that it sounds like any other
voice the Adept has heard. The voice may be projected so that it appears to be emanating from
anywhere within the radius of the spell.
Bolt of Energy (S-2)
Range: 15 feet + 15 / Rank
Duration: Immediate
Experience Multiple: 200
Base Chance: 50%
Resist: Passive
Storage: Investment, Ward, Magical Trap
Target: Entity or object
Effects: The Adept may cast a bolt of energy at a
target and will, as a result, inflict [D - 5] (+ 1 per
Rank) damage on the first entity or object that the
bolt encounters.
Opening (S-3)
Range: 15 feet + 15 / Ranks
Duration: Immediate
Experience Multiple: 150
Base Chance: 30%
Resist: Passive
Storage: Investment
Target: Object
Effects: Instantly opens any one object or portal,
including those locked by Mage Lock (S-6). A
door or lid will have all locks unlocked and bolts
undone and will become immediately open (ajar).
Enchant Weapon (S-4)
Range: 5 feet + 5 / Rank
Duration: 5 minutes + 1 / Rank
Experience Multiple: 250
Base Chance: 30%
Resist: None
Storage: Investment
Target: Object
Effects: Increases the Base Chance to hit with the
weapon over which it is cast by 1 (+ 1 / Rank) and
increases the damage done by the weapon by 1 for
every 3 (or fraction) Ranks.
Web of Entanglement (S-5)
Range: 15 feet + 15 / Rank
Duration: Concentration: maximum 15 minutes +
15 / Rank
Experience Multiple: 150
Base Chance: 35%
Resist: Passive
Storage: Investment, Ward
Target: Entity or object
Effects: This spell allows the Adept to project a
sticky web, similar to a spider web in construction,
from their fingertips at a target hex, object or entity. Any objects or entities standing between the
Adept and the target are ensnared by the web along
with the target. The web may only ensnare a number of human-sized entities equal to the Adept’s
44

rank and so will stop at the hex at which this limit
is reached (or maximum range).
The web continues until cut or until the Adept
ceases to concentrate on it. In order to cut the web,
a character must successfully strike with a weapon
that does type B damage and must inflict at least 10
points of damage as a result of that single strike.
The web has no defence.
Any ensnared entity must roll 1 × PS (2 × PS if
they successfully resisted) in order to move themselves to an adjacent hex (which may be free of the
web), or to perform an action within the web. A
similar check is required for any entity attempting
to remove an object from the web. If an entity
receives aid in removing themselves from the web,
the PS of the aiding character may be combined
with their own. Any dropped object will become
ensnared by the web, as will any entity who comes
into contact with it (up to the limit of the web).
Mage Lock (S-6)
Range: 15 feet + 15 / Rank
Duration: 1 hour + 1 / Rank
Experience Multiple: 150
Base Chance: 30%
Resist: None
Storage: Investment
Target: Object
Effects: The spell may be cast over any portal
(door, window, etc.) that can normally be opened
or closed. It effectively locks the portal with an
unpickable lock. The portal may still be forced
open by brute strength. The Physical Strength(s) of
all characters attempting to force a portal locked in
this manner are added together and multiplied by
the Difficulty Factor of the task. The Difficulty
Factor is a function of the Rank of the spell:
Rank

Difficulty Factor

1–5
6–10
11–20

3
2
1.5

Enhancing Enchantment (S-7)
Range: 15 feet + 15 / Rank
Duration: 10 seconds + 5 / Rank
Experience Multiple: 300
Base Chance: 25%
Resist: None
Storage: Investment, Ward
Target: Area
Effects: The Rank of this spell is added to the Rank
of one characteristic of any spell being cast within
the area of effect. Note that the actual Rank of the
affected spell is unchanged — only the effect of
one characteristic is enhanced (as per double and
triple effects).
While casting the Spell of Enhancing Enchantment, the adept specifies the spell characteristic to
be affected. Only Range, Duration, Base Chance,
or (where appropriate) Damage, may be affected
by this spell.
A spell may never have a characteristic enhanced
by Ranks greater than its own rank by the use of
this spell (e.g. a Rank 6 spell that is cast within a
Rank 8 Spell of Enhancing Enchantment will only
gain the effects of 6 extra ranks in the affected
characteristic).
Only spells that are cast are affected, not spells
being released from any form of storage. The
caster of the subject spell will be aware that their
spell is being affected by enhancement during the
pulse that they are preparing (as per high or low
Mana). They will not know which characteristic is
being affected nor the amount of enhancement.
They may not restrict the effect of the enhancement, but may reduce the spell effects in the normal manner. If an attempt is made to cast a Spell of
Enhancing Enchantment on an area that has some
part of it under the effects of another Spell of Enhancing Enchantment, then the second spell will
fail.

14 COLLEGE OF ENSORCELMENTS & ENCHANTMENTS
Special Duration, Range, or Damage effects may
be caused by the use of this spell (e.g. Duration of
Lesser Enchantment at Rank 20).
Levitation (S-8)
Range: 15 feet + 15 / Rank
Duration: 10 minutes × [D - 5] × [Rank]
Experience Multiple: 125
Base Chance: 25%
Resist: Active, Passive
Storage: Investment, Ward, Potion
Target: Entity
Effects: Causes the target of the spell to rise into
the air 15 feet (+ 1 / Rank) at the rate of 1 foot /
pulse. The spell is limited to vertical movement
and will in no way propel the target horizontally.
Enchant Armour (S-9)
Range: 15 feet + 15 / Rank
Duration: 30 minutes + 30 / Rank
Experience Multiple: 200
Base Chance: 20%
Resist: None
Storage: Ward, Investment
Target: Entity
Effects: This spell adds 2 (+ 2 / Rank) to the target’s defence versus physical weapons. At Rank 11
and above it also permits the target’s armour to
absorb 1 additional point of damage. The target
must be armoured.

Wizard’s Eye (S-10)
Range: 15 feet + 15 / Rank
Duration: 1 minute + 1 / Rank
Experience Multiple: 200
Base Chance: 20%
Resist: None
Storage: Potion
Target: Self
Effects: The Adept creates an invisible, intangible
eye which they can move around within a radius
equal to the spell’s range. The eye originates in the
same spot as the Adept and operates as would any
normal eye except that it is not attached to the
Adept physically. Wizard’s Eyes have a TMR of
10.
The eye can move through solid objects but will
see only blackness while within an object. The eye
can be seen with witchsight or other means of
detecting invisible objects. It is possible to use the
eye to target spells, but the range is determined
from the Adept, not from the eye. The eye (and
hence the Adept) is susceptible to magical effects if
the effect can affect eyes (e.g. Flash of Light,
blindness).
Slowness (S-11)
Range: 15 feet + 15 / Rank
Duration: 10 seconds × [D - 5] × [Rank]
Experience Multiple: 300

45

Base Chance: 20%
Resist: Active, Passive
Storage: Potion, Investment, Ward, Magical Trap
Target: Entity
Effects: The spell affects 1 (+ 1 / 2, or fraction,
Rank) targets. All entities subject to this spell have
their running, crawling, flying, or swimming speed
halved and have the time it takes them to do anything on the Tactical Display doubled (e.g. they
could only attack once every two pulses).
Quickness (S-12)
Range: 15 feet + 15 / Rank
Duration: 10 seconds × [D - 5] × [Rank]
Experience Multiple: 300
Base Chance: 20%
Resist: Active, Passive
Storage: Potion, Investment, Ward
Target: Entity
Effects: The spell affects 1 (+ 1 / 3, or fraction,
Ranks) targets. The targets’ Initiative Value is
increased by 10, and they can perform Actions
twice as often.

14.7 Special Knowledge Rituals
There are no Special Knowledge Rituals of the
College of Ensorcelments and Enchantments.

46

15 COLLEGE OF ILLUSIONS

15 The College of Illusions (Ver 1.5)
The College of Illusions is concerned with truth,
deception and sensation. Practitioners of the College (or Art) of Illusions are called Illusionists.
This College is one of extremes, of subtlety and
flamboyance, and especially of deception and
honesty. Illusionists are taught that they are privy
to one of the most potent truths known to the sentient races, namely that our senses are the only
window to the existence of reality. Based on this
doctrine, they change the appearance of reality
rather than the actual nature. This often leads to
more impressive but superficial effects than other
Colleges. These effects have meant that Illusionists
have historically been associated with stage magicians and sleight-of-hand artists, which has degraded their status in the eyes of other Adepts.
Most accept their stage heritage, and revel in the
layers of deception involved in revealing a great
truth to their audience. Some, however, become
enmeshed within their own fantasy and fail to
differentiate between this and reality. Others deny
this deception totally, and turn to death as the only
fixed point in a changing world. These Adepts use
people’s dependency on their senses to cause death
and despair.

mute. If an observer is intent on the Adept (e.g. in
conversation with), the observer may make a (2 ×
PC) - Rank check to notice. All aspects of the Spell
(including Base Chance) must be cast at the lower
of the Talent’s and Spell’s ranks, and the Base
Chance is further reduced by 20%. The Talent only
functions with spells of the College of Illusions.

magical darkness. All entities not shielded by a
solid object and within range must either resist or
be blinded for 5 seconds (+ 5 / 2 full Ranks).
Blinded characters have their Strike Chance reduced by 50, and they move at half their normal
TMR. The Adept is normally not affected by this
spell.

Enhanced Vision (T-2)
Experience Multiple: 225
Effects: The Adept may see through any obscurement such as smoke, dust, water, fog, cloud or rain
(though not darkness) to a distance of 10 feet (+ 10
/ Rank) as if unobscured. Normal penalties then
accrue from this range. The Adept may see objects
or entities that are invisible — they appear to have
a slight blue sheen around them. If the invisibility
effect (excluding Walking Unseen) is of a higher
Rank than the Enhanced Vision, the object or entity may not be clearly identified or directly magically targeted. The Adept has a Base Chance of PC
(+ 5 / Rank) of Detecting Illusions if they take a
Pass Action — each Illusion may only be checked
once. Detecting an Illusion does not nullify its
effects.

Illusion of Food (G-3)
Range: 15 feet + 15 / Rank
Duration: 10 minutes + 10 / Rank
Experience Multiple: 100
Base Chance: 50%
Resist: None
Storage: Investment
Target: Object
Effects: The Adept may change the taste, appearance and smell of up to 1 cubic foot of food or
drink (+ 1 / Rank). At Rank 11, the Adept may
create (Rank - 10) pounds of food or drink. This
magically created food will recover fatigue for the
duration of the spell. The Adept’s Cooking Rank
will effectively be improved by 2, or to a minimum
of Rank / 2. No other properties (e.g. poison,
magic) may be changed or created with this spell.

Traditional Colours
The more flamboyant of the Illusionists will often
wear bright, even garish clothing incorporating as
many colours as possible. These colours will usually not clash, but the Illusionist wearing them will
stand out in any crowd. At the very least, most
Illusionists will wear extremes of colours: fire,
blood, sable, snow, charcoal, emerald, gold, etc.

Project Image (T-3)
Experience Multiple: 200
Effects: This Talent creates a visual Illusion of the
Adept, who becomes Invisible as per (G-4) except
that attacking in melee will always disrupt this
Talent. Initially, the image forms over the Adept.
The image will reflect the current appearance of
the Adept, in any desired stance. The image may
only be changed or moved (at TMR) when the
Adept takes a pass action. The image must remain
in line of sight and within 15 feet (+ 15 / Rank),
and may not pass through solid objects. At Rank 5,
the image may make appropriate environmental
noises, e.g. foot-steps. At Rank 10, the Adept may
move while maintaining the relative distance and
direction to their image. At Rank 15, the Adept
may project their voice through the image. At Rank
20, the Illusionist may also see through the image.
The abilities gained at Ranks 15 and 20 require
pass actions to use.

Invisibility (G-4)
Range: 15 feet + 15 / Rank
Duration: 5 minutes + 5 / Rank
Experience Multiple: 450
Base Chance: 30%
Resist: None
Storage: Investment, Potion, Ward
Target: Entity or Object
Effects: The target becomes invisible. Unless cast
at rank 16 or higher, the spell ceases whenever the
target makes a strike check in melee or close
(whether successful or not). The target may choose
to end the spell at any time.

Traditional Symbols
The College of Illusions is associated with a plethora of symbols adopted from individual performers. Theatrical symbology is common.
On Illusions
Most Magics in this College are Illusions. Illusions
cannot be “disbelieved”; the images are there and
will interact with light and sound in the same manner as the objects they represent. Mirrors and light
sources may not be created. Unless stated, the
Illusions will not be substantial, and any solid
objects or spells will pass through them. Illusions
will interact with each other as if real. Illusion
spells may be dispelled by anyone with the appropriate counterspell, although Illusions on an entity
gains their passive magic resistance.

15.1 Restrictions
Adepts of the College of Illusions may practice
their arts without restrictions.
The MA requirement for becoming an Adept of the
College of Illusions is 13.

15.2 Base Chance Modifiers
The Base Chance of performing any talent, spell or
ritual of the College of Illusions may be modified
by conditions.
As the successful performance of Illusion magic is
as much finesse and presentation as raw power, a
knowledge of the Performing Arts is considered
essential by many Adepts. Furthermore, willing a
change in reality to the level of detail necessary for
deception, requires good concentration and Willpower. The following conditions modify the base
chances of the College of Illusions:
For each point the Adept’s Willpower
varies from 15
For each two full Ranks in Troubadour

+/1
+1

15.3 Talents
Concealed Casting (T-1)
Experience Multiple: 150
Effects: This Talent allows the Adept to cast without any apparent movement or noise. The Adept
may not speak or make meaningful or extravagant
actions. They may not cast without a hand free or if

15.4 General Knowledge Spells
Audile Illusion (G-1)
Range: 15 feet + 15 / Rank
Duration: 10 minutes + 10 / Rank
Experience Multiple: 125
Base Chance: 30%
Resist: None
Storage: Investment, Ward, Magical Trap
Target: Area
Effects: This Illusion allows the Adept to create up
to Rank simple sounds, each coming from a location fixed either absolutely or relative to an object.
Whilst taking a pass action, the Adept may cause
any combination of these sounds to occur. Sounds
and locations are defined as they are first used, not
at time of casting. No language or musical instrument may be imitated. The maximum volume at
Ranks 0–4 is that of a stage whisper (usually
clearly audible at 25’), at Ranks 5–9, normal conversation (50’), at Ranks 10–14, shouting (200’), at
Ranks 15–19, screaming (500’), and at Rank 20,
thunder (1 mile). The range at which the sound will
be clearly heard is affected by the ambient level of
noise.
Flash of Light (G-2)
Range: 15 feet + 15 / Rank
Duration: Immediate
Experience Multiple: 150
Base Chance: 20%
Resist: Passive
Storage: Potion
Target: Self
Effects: This spell creates a blinding flash of light
emanating from the Adept’s body. This light is
magical in nature, and may cut through areas of
47

The effects of this spell are identical to the E & E
College spell of the same name.
Personalised Illusion (G-5)
Range: 30 feet + 30 / Rank
Duration: 1 hour + 1 / Rank
Experience Multiple: 75
Base Chance: 40%
Resist: None
Storage: Investment, Ward, Magical Trap
Target: Area
Effects: This Illusion creates an image of a specific
type peculiar to the Adept. At Rank 0, the image
must be of a rock. At each of Ranks 4, 8, 11, 14,
16, 18, 19 and 20, another image may be chosen.
The Adept may select the colouring, size and conformation of the image at cast time (e.g. a closed
iron chest, or an open wooden casket) up to a
maximum diameter of 5’ and height of 8’. The
Illusion is not affected by physical laws except that
it is opaque to light. The Adept may move or rotate
(though not manipulate) the Illusion by taking a
pass action. The workmanship is as if the Adept
were an Artisan of one-third Rank.
Illusion of Seeming (G-6)
Range: 5 feet + 5 / Rank
Duration: 10 minutes + 10 / Rank
Experience Multiple: 200
Base Chance: 20%
Resist: None
Storage: Investment, Ward
Target: Object or Illusion
Effects: This Illusion creates a visual image of up
to 2 cubic foot (+ 2 / Rank) on an object, which
will move with the object if the object changes
shape or position, and dissipate if the object is
destroyed. The Adept may move or change the
image by taking a pass action to concentrate on it.
The object has not been changed, and still has its
original physical properties. The workmanship is
as if the Adept were an Artisan of one-third Rank.
If the target is an Illusion, the target may be
changed to any other form that it could normally

15 COLLEGE OF ILLUSIONS
take. Up to 1 cubic foot of the target (+ 1 / Rank)
may be changed during each pass action. This
change will revert if the Seeming ceases before the
target Illusion. The Adept may only affect their
own Illusions.
Static Illusion (G-7)
Range: 15 feet + 15 / Rank
Duration: 30 minutes + 30 / Rank
Experience Multiple: 200
Base Chance: 30%
Resist: None
Storage: Investment, Ward, Magical Trap
Target: Area
Effects: This spell creates a visual Illusion within a
volume no larger than (Rank + 3) feet by (Rank / 2
+ 2) feet by (Rank
/ 3 + 1) feet. The Illusion is fixed at casting, and
may not be changed or moved with this spell. The
workmanship is as if the Adept were an Artisan of
one-half Rank.
Illusionary Wall (G-8)
Range: 15 feet + 15 / Rank
Duration: 30 minutes + 30 / Rank
Experience Multiple: 125
Base Chance: 20%
Resist: None
Storage: Investment, Ward, Magical Trap
Target: Area
Effects: This spell creates an Illusion of a nonsolid, linear or smoothly curved wall. It may have
all the characteristics of any other wall spell (except solidity, damage or fear), or mimic an adjacent
surface instead. The wall is 20’ by 10’ by 12", with
each Rank altering its height or width by 2’ or
thickness by 4". The wall must be attached along
the entirety of one of its thinnest edges.

15.5 General Knowledge Rituals
Illusionary Cloaking (Q-1)
Range: 5 feet + 5 / Rank
Duration: 6 hours + 6 / Rank
Experience Multiple: 100
Base Chance: 40% + 4% / Rank
Target: Area
Cast Time: 1 hour
Material: None
Concentration Check: Standard
Effects: This ritual creates a subtle visual Illusion
over all non-living matter within range according
to the Adept’s wishes. Only surface appearance is
changed, and this ritual does not provide invisibility. This means that objects will retain their silhouette, and the Illusion will fill the same volume as
the object. The new appearance will fade over the
last hour of the duration. The workmanship is as if
the Adept were an Artisan of one-half Rank.
Illusory Fog (Q-2)
Range: 30 feet + 30 / Rank
Duration: 1 hour + 1 / Rank
Experience Multiple: 75
Base Chance: 40% + 4% / Rank
Target: Area
Cast Time: 30 minutes
Material: None
Concentration Check: Standard
Effects: This ritual creates Illusory fog to a height
of 10 feet + 2 / Rank within range. The fog will
rise within five minutes of the ritual’s completion,
centred on the spot where the Adept performed the
ritual. The fog has a visibility of 20 hexes (1 /
Rank).

15.6 Special Knowledge Spells
Illusionary Animal (S-1)
Range: 15 feet + 15 / Rank
Duration: 10 minutes + 10 / Rank
Experience Multiple: 150
Base Chance: 35%
Resist: None
Storage: None
Target: Area

Effects: An Illusion of a non-magical, non-sentient
creature of no larger than 10 lb + 10 lb × Rank
squared is created. The Illusion will have the instincts of the creature it is based on, though it will
obey most simple mental instructions from the
Adept. This communication is one-way, and may
only occur while the animal is within range of the
spell. No other communication is possible — it has
no mind. The animal has the minimum PS, MD,
AG, TMR and PC for the selected creature. The
animal is solid, and can bear weight. It cannot
attack effectively, nor cause fear. Only creatures
previously observed by the Adept may be created.
Smell, sound and feel are created as appropriate. If
damaged, the Illusion is dissipated. The Adept may
perceive through the animal’s senses as follows:
Rank 3 taste, Rank 7 smell, Rank 11 touch, Rank
15 hearing, Rank 19 sight.
Illusionary Bolt (S-2)
Range: 15 feet + 5 / Rank
Duration: 5 seconds + 5 / Rank
Experience Multiple: 200
Base Chance: 30%
Resist: Active, Passive
Storage: Investment
Target: Entity, Object or Area
Effects: An Illusion of any dangerous-looking
object of “throwable” size is created in the Adept’s
hand. When thrown, it flies in a straight line, doing
(D-2) + 1 / 2 full Ranks magical damage to the first
Entity or object struck. Illusions take double damage from this bolt.
Illusory Creation (S-3)
Range: Touch
Duration: 10 minutes + 10 / Rank
Experience Multiple: 250
Base Chance: 20%
Resist: None
Storage: Investment
Target: Area
Effects: This spell creates the Illusion of one small
object. The object will have the physical attributes
of one of the following: cloth, leather, wood,
bronze to a maximum of 1 lb (+ 1 / 3 full Ranks).
The object will be of a single form, with no moving or removable pieces, though of any combination of colours. The object will interact normally
with its surrounds. The workmanship is as if the
Adept were an Artisan of one-third Rank.
Illusion of Deep Pockets (S-4)
Range: Self
Duration: 1 hour + 1 / Rank
Experience Multiple: 200
Base Chance: 40%
Resist: None
Storage: Potion
Target: Entity
Effects: This Illusion creates 1 magical pocket (+ 1
/ Rank) about the Adept’s clothing. Each pocket
can hold an object with a largest dimension of no
more than 1 inch (+ 1 / Rank). No entity can be
placed in a pocket. The total weight held may not
exceed 1 lb (+ 1 / Rank), and no individual object
can weigh more than half this amount. The pockets
are not obvious, but thorough investigation may
find them: searchers get a (2 × PC) - Rank check to
notice. Items within the pockets may not be located
by spells of a lesser rank. The pockets may only be
accessed while the Adept is clothed; however, a
new set of clothes will still contain the pockets. At
the end of the spell, the pockets expel their contents.
Disguise Illusion (S-5)
Range: Touch
Duration: 1 hour + 1 / Rank
Experience Multiple: 200
Base Chance: 20%
Resist: Active, Passive
Storage: Investment, Potion, Ward
Target: Entity
Effects: The Adept may change the target’s appearance. The target may not be located by loca48

tion spells of a lesser rank. The target’s height may
vary by 1% / Rank, and their weight by 2% / Rank.
The target’s voice or clothing may be changed at
Rank 5 or both of these at Rank 10; an individual
may be duplicated or gender or race changed at
Rank 15, or two of these at Rank 20. If clothes are
included in the spell, they revert to normal if discarded. The disguised form has the PB of the target
+/2 / 3 full Ranks, within racial limitations.
Illusionary Euphonia (S-6)
Range: 15 feet + 15 / Rank
Duration: 10 minutes + 10 / Rank
Experience Multiple: 150
Base Chance: 25%
Resist: None
Storage: Investment, Ward, Magical Trap
Target: Area
Effects: This Illusion creates the sound of harmonious voices or musical instruments. One instrument (+ 1 / 3 full Ranks) may be played, or one
voice (+ 1 / 6 full Ranks) may speak or sing. Each
may come from a different location fixed either
absolutely or relative to an object. The effective
rank of the performance or conversation is Rank/2,
to a maximum of the Adept’s rank in the instrument or language. Pass actions are required for
musical performance at a skill level above Rank/5,
or conversation. The maximum volume at Ranks 04 is that of a stage whisper (usually clearly audible
at 25’), at Ranks 5-9, normal conversation (50’), at
Ranks 10-14, shouting (200’), at Ranks 15-19,
screaming (500’), and at Rank 20, thunder (1 mile).
The range at which the sound will be clearly heard
is affected by the ambient level of noise.
Hallucination (S-7)
Range: 15 feet + 15 / Rank
Duration: 10 minutes + 10 / Rank
Experience Multiple: 225
Base Chance: 25%
Resist: Active, Passive
Storage: Investment, Ward, Magical Trap
Target: Entity
Effects: This spell enables the Adept to select 1
entity (+ 1 / 3 full Ranks) as an audience for their
Illusions. Whenever the Adept casts an Illusion,
they must decide whether everyone can perceive it,
or just the targets of this spell. This spell does not
circumvent any resistance checks by the targets.
The Illusion must be cast during the duration of
this spell although the effects may last after the
Hallucination duration has expired.
Heroism (S-8)
Range: 15 feet + 15 / Rank
Duration: 10 minutes + 10 / Rank
Experience Multiple: 200
Base Chance: 30%
Resist: None
Storage: Investment, Potion
Target: Entity
Effects: This spell charges the target with great
might and heroism. They seem taller and stronger,
and their actions gain a natural grace and power.
All allies (in line-of-sight) led by the target gain a
bonus die modifier of 1 (+ 1 / Rank) to all fear
checks, while neutral observers gain a bonus of 1
(+ 1 / Rank) to reaction rolls. The target gains a
defence bonus of 1% + 1% / Rank.
Illusion of Innocence (S-9)
Range: 1 foot + 1 / Rank
Duration: 10 minutes + 10 / Rank
Experience Multiple: 250
Base Chance: 30%
Resist: Active
Storage: Investment, Potion
Target: Entity
Effects: This Illusion changes the mien of 1 entity
(+ 1 / 4 full Ranks). This influences the initial
reactions of all entities who perceive the target:
they will react as if the target is an innocent (this
does not necessarily mean a favourable reaction).
The spell does not force people to act stupidly — if
the target does not act appropriately, their opinion

15 COLLEGE OF ILLUSIONS
may be revised rapidly. Extreme behaviour terminates these effects. For example, should a target
prepare a weapon, cast a passively resistible spell,
or attack, the spell will dissipate.
Maze (S-10)
Range: 15 feet + 15 / Rank
Duration: 5 seconds + 5 / Rank
Experience Multiple: 450
Base Chance: 1%
Resist: Active, Passive
Storage: Investment, Ward, Magical Trap
Target: Entity
Effects: The target is transported into a hedge-maze
of perpetual fog, where they may wander as they
will. The hex that they occupied is filled with
opaque mist that disperses in 5 seconds. At the end
of the spell, the target reappears in their original
hex (they are displaced by solids but displace fluids); any items dropped inside the maze reappear
along with the target. Whilst in the maze, they
must make a 2 × WP check each pulse to perform a
non-pass action. Personal magical effects continue,
(e.g. while a Phantasm would follow the target into
the maze, Agony wouldn’t). For all purposes, the
Maze is deemed to be on the same plane that the
target just disappeared from. Each target appears
within their own unique Maze. The Adept may still
see (and thus cast upon) the target, but not physically contact them.
Illusion of Metamorphosis (S-11)
Range: 1 foot + 1 / Rank
Duration: 5 minutes + 5 / Rank
Experience Multiple: 225
Base Chance: 15%
Resist: Active, Passive
Storage: Investment, Potion
Target: Entity
Effects: The target’s appearance alters to that of a
one hex creature of at least (target’s PS + EN)
pounds, over a period of 5 seconds. Their actual
form and characteristics do not change. They do
not gain any special abilities (e.g. poison, magic,
fear, flight) associated with that animal. The Adept
must be familiar with the desired animal. The spell
will cease when the target takes effective Endurance damage. Creatures available by Rank: Rank 0
Small Land Animals; Rank 4 Felines, Apes &
Prehumans; Rank 8 Avians, Fish, Lizards &
Snakes; Rank 12 Horses, Large Land Animals;
Rank 16 Earth Dwellers, Fairy Folk, Humans;
Rank 20 Giants.
Illusion of Mist (S-12)
Range: 20 feet + 20 / Rank
Duration: 10 minutes + 10 / Rank
Experience Multiple: 100
Base Chance: 30%
Resist: None
Storage: Investment, Ward
Target: Area
Effects: The Adept creates an Illusion of 1000
cubic feet of mist (+ 500 / Rank) of any shape
entirely within the Adept’s range. The mist must be

least 1 foot in any dimension and may have detail
no finer than one foot in size. The visibility in this
mist is 20 feet (1 / Rank). The mist is not affected
by wind. By taking a pass action, the Adept may
move the mist up to 1 hex / pulse so long as it
remains within range.
Illusion of Multiple Images (S-13)
Range: 1 foot + 1 / Rank
Duration: 1 minute + 1 / Rank
Experience Multiple: 200
Base Chance: 20%
Resist: None
Storage: Investment, Potion
Target: Entity
Effects: One image of the target (+ 1 / 5 full Ranks)
appears in the target’s hex. Each image disappears
upon receiving damage. All images are identical to
the target, and will imitate their actions faithfully.
The images have the same defence as the target,
but no magic resistance. Any targeted attack has an
equal chance of hitting each image and the target.
More than one image may be struck by a multitarget or area effect attack.
Nightmare Illusion (S-14)
Range: 15 feet + 15 / Rank
Duration: 10 seconds + 10 / Rank
Experience Multiple: 400
Base Chance: 5%
Resist: Passive
Storage: Investment, Ward, Magical Trap
Target: Area
Effects: The Adept creates a terrifying, hostile and
dangerous Illusion. The Nightmare is initially
under the Adept’s control, and will obey their
commands until the Adept’s concentration is broken or the Nightmare is slain or dissipated. The
Nightmare is totally insubstantial and invisible to
all who resist. The Nightmare has a combined EN
and FT of 20 (+ 5 / 2 full Ranks), but no defence or
armour. It can automatically hit up to two different
targets a pulse for [D - 2] (+ 1 / 2 full Ranks)
magical damage. It has a TMR of 10 and an Initiative of 30 (+ 2 / Rank). It may not be stunned. At
Rank 15, every target must make a fear check upon
first being struck. The Nightmare may move in any
direction without restriction, including through air,
walls, water, etc. except through the area of an
Illusion Special counterspell. If concentration is
lost, the Nightmare will attempt to kill as many
people as possible, and cannot be controlled by any
means.

15.7 Special Knowledge Rituals
Illusionary Aura (R-1)
Range: 5 feet
Duration: 1 day + 1 / Rank
Experience Multiple: 200
Base Chance: 30% + 3% per Rank
Target: Entity
Cast Time: 1 hour
Material: None
Concentration Check: Standard

49

Effects: This ritual creates an Illusion to alter the
Aura of the target. The target’s apparent Aura is
altered by fooling the DA, DE or Divination with a
stereotype defined in Rank + 1 words, e.g. Rank 4
— “Master Thatcher and Evil Necromancer”. The
questioner then receives appropriate answers based
on the stereotype; if the question is not covered by
the stereotype, the real answer is returned. The
Illusory Aura will not hide information, merely
alter it. The Ritual of Warding is not affected.
Illusionary Terrain (R-2)
Range: 5 feet + 5 / Rank
Duration: 2 hours + 2 / Rank
Experience Multiple: 150
Base Chance: MA + 3% per Rank
Target: Area
Cast Time: 1 hour
Material: None
Concentration Check: Standard
Effects: This ritual creates an Illusion such that
observers outside the range cannot sense any sign
of entities or their trappings within the area. Temporary campsites will not be seen and permanent
dwellings appear deserted and dilapidated. However, some effects may be seen, e.g. the smoke and
light of a campfire rising above the ritual’s range.
The ritual will not hide entities from each other if
both are outside the effect, even if the area lies
between them. Magic that is not targeted on entities or their possessions may not be hidden through
this ritual. Any Location or Scrying magic lower
than the Rank of Illusionary Terrain may not detect
into the ritual’s area.
Permanency (R-3)
Range: 5 feet
Duration: 1 Day (+1/Rank) or Permanent
Experience Multiple: 550
Base Chance: MA + 4% per Rank
Resist: None
Target: Spell or Ritual
Cast Time: 1 hour
Material: Special
Concentration Check: Standard
Effects: This ritual enhances the duration of one of
the Adept’s Illusion spells or rituals. The magic
must have been cast previously, and have sufficient
duration to last throughout the ritual. This ritual
may not be used with any magics with concentration-based or indefinite, condition based duration.
Once the duration of an Illusion has been enhanced, no changes are possible. The Illusion may
be overridden by any similar spell for the duration,
rather than queuing. This ritual may be performed
without material components to change the duration of the target Spell or Ritual to 1 day (+1 per
Rank). Or the Adept may use the material components to change the duration to permanent. Material
costs for this ritual are 5,000 sp (200 / Rank). A
spell that has been enhanced by this ritual may
only be removed by a Ritual of Dissipation or the
Adept’s own Special Knowledge counterspell.

50

16 COLLEGE OF SORCERIES OF THE MIND

16 The College of Sorceries of the Mind (Ver 1.6)
The College of Sorceries of the Mind deals primarily with controlling or influencing the minds of
others. Adepts of this College are variously known
as Mind Mages, or Sorcerers. It is widely believed
that Sorcerers can peer into the darkest nooks and
crannies of the soul, and are near-universally
feared.
Traditional Colours & Symbols
This college has no traditional colours or symbols
as Sorcerers prefer to blend into society.

16.1 Restrictions
Adepts of the College of Sorceries of the Mind
may practice their arts without restriction.
The Magical Aptitude requirement for becoming a
member of this College is 11.

16.2 Base Chance Modifiers
The Base Chance of performing any talent, spell or
ritual of the College of Sorceries of the Mind
against a single entity target is affected by relative
Willpower:
For each point the Adept’s Willpower is
above the target’s Willpower
For each point the Adept’s Willpower is
below the target’s Willpower

+1
-2

16.3 Talents
Resist Pain (T-1)
Range: Self
Experience Multiple: 300
Storage: Potion
Effects: The Adept is largely immune to pain. They
cannot be tortured or stunned by physical means.
In addition, if there is a chance that their concentration may have been broken, 5 (+ 1 / Rank) is always added to their Concentration Check. No
matter what the circumstances the Adept will always have a chance of maintaining concentration.
Like all magic, this talent is affected by cold iron.
However, for a weapon to affect the talent, the
weapon must be made of cold iron, must have done
endurance damage, and must remain in contact
with the body. If cold iron is used in the torture of
a mind mage, the Adept adds their concentration
bonus to the base chance of resisting the torture
attempt.
Resist Temperature (T-2)
Range: Self
Experience Multiple: 250
Storage: Potion
Effects: The Adept is immune to climatic extremes
of temperature from arctic cold to desert heat
whether generated naturally or by magic. They
therefore do not suffer from exposure, frostbite,
wind-chill, heat stress, hypothermia or hyperthermia, and their fatigue loss is unaffected by temperature. If the Adept is attacked by fire, ice, heat
or cold (whether magical or not) then the damage
points are reduced by 1 + 1 per 5 full ranks.
Sense Danger (T-3)
Base Chance: PC + 3 / Rank
Experience Multiple: 300
Storage: Potion
Effects: The Adept may sense the presence of a
hazard to the Adept’s life or wellbeing. Generally
this may be interpreted as a “bad feeling” about a
person or an object, or a sense of foreboding about
a situation. This talent operates continuously and
unconsciously. The Adept may also add 2 + 2 /
Rank to the chance of detecting an ambush (see
Ranger) and to the chance of detecting a trap (see
Spy/Thief).

16.4 General Knowledge Spells
Control Animal (G-1)
Range: 30 feet + 15 / Rank
Duration: Concentration: no maximum
Experience Multiple: 100
Base Chance: 40%

Resist: Passive
Target: Non-sentient entity
Storage: Investment
Effects: The Adept gains control over one nonsentient entity within range. Control is lost if the
entity leaves the range but recommences if range is
re-entered. The Adept does not receive any sensations from the animal. If the Adept releases the
animal or their concentration is broken the animal
may attack the Adept. The Adept controls the
animal’s mind and therefore need not know how to
make a bird fly: the Adept commands the animal to
do something and the animal knows how to do it.
The animal will comply within the spirit of the
command.
Control Person (G-2)
Range: 30 feet + 15 / Rank
Duration: Concentration: no maximum
Experience Multiple: 650
Base Chance: 30%
Resist: Active, Passive
Storage: Investment
Target: Sentient Entity
Effects: The Adept gains absolute control over
every action of one sentient entity within range.
The Adept does not receive any sensations from
the target. The Adept does not need to know the
target’s language to control them. Control is lost if
the target leaves the range but recommences if
range is reentered. They may control the physical
activities of that target, and use the target’s physical skills where known, but may not cause the
target to cast spells.
The Adept may not use their own skills through the
target. The target always acts as if they were also
affected by a Spell of Slowness (E&E S-11) unless
they choose not to resist the spell, or subsequently
decide to co-operate with the Adept’s commands.
The spell continues in effect until the Adept’s
concentration is broken or the Adept chooses to
release the target. The target is aware of the control, and although the target is unable to do anything about it at the time, they will remember being
controlled. The Adept may force the target to act in
direct opposition to the target’s own wishes. Suicidal instructions however, grant the target another
magic resistance which, if successful, dissipates the
spell.
Empathy (G-3)
Range: Touch until Rank 10
Duration: Immediate
Experience Multiple: 200
Base Chance: 20%
Resist: None
Storage: Investment
Target: Entity
Effects: The spell allows the Adept to feel the
emotions and physical sensations which the target
of the spell is currently experiencing. It also allows
the Adept to absorb wounds from Endurance and
Fatigue at a rate of 2 points cured for every 1
which the Adept agrees to subtract from their own
Fatigue (never Endurance). The additional fatigue
subtracted by the Adept is damage fatigue, not
spell fatigue. This spell may not be used to regenerate spell or tiredness fatigue, but only reduces/transfers damage. At Rank 10 or higher, the
Adept can cast this spell on a target who is within
15 feet (+ 15 / Rank over 10) or less from them.
This spell may be cast on self, but only to cure
endurance damage. On a ‘Double’ effect, 3 Damage Points are cured for each point inflicted on the
Adept; a ‘Triple’ effect allows for 4 points per
point inflicted on the Adept.
ESP (G-4)
Range: 30 feet + 15 / Rank
Duration: 30 seconds + 10 / Rank
Experience Multiple: 100
Base Chance: 40%
Resist: None
51

Storage: Potion
Target: Self
Effects: The Adept can sense the presence and
general mood (but not the exact nature) of all entities within range of the spell. The Adept has a
general idea of where each entity is, say to within
90 or 60 degrees, the accuracy of which may increase with rank. ESP will detect the presence of
an entity even if that entity is mind cloaked. Spells
cannot be targeted through ESP.
Hypnotism (G-5)
Range: 15 feet + 15 / Rank
Duration: Concentration: no maximum
Experience Multiple: 200
Base Chance: 40%
Resist: Active, Passive
Storage: Investment
Target: Entity
Effects: The spell causes an entity to accept suggestions from the Adept. The spell may only be
cast over a target with whom the caster is normally
able to communicate verbally. It can never be cast
over a totally hostile target. The target may be
enabled to remember otherwise forgotten details
through appropriate questioning.
Once the target has been hypnotised, the Adept can
make suggestions which the target will readily
accept unless they conflict directly with their best
interests. The target will never remember where
these suggestions came from.
The Adept may choose up to 1 (+ 1 / 5 Ranks) of
the above suggestions to be active in a posthypnotic manner, which the target will continue to
implement for 3 (+ 3 / Rank) hours after the spell
ceases. The target will stop following all other
suggestions once the spell ceases.
Limited Precognition (G-6)
Range: Special
Duration: Immediate
Experience Multiple: 150
Base Chance: 20%
Resist: May not be resisted
Storage: Potion
Target: Self
Effects: The Adept may see (unclearly) up to 1 (+ 1
/ Rank) hours into their own future and may foresee possible events. The Adept’s player should
describe a plan of action, and the GM will describe
a vision based on the consequences of those actions, which can be as literal or metaphorical as the
GM wishes. The clarity and detail of the vision will
be dependent on the Adept’s rank. Should the
Adept foresee their own death, a fright check may
be appropriate. This spell cannot be recast immediately with a different plan; the Adept must wait
until they have taken some action to change their
situation. This spell works at any range.
Mind Shield (G-7)
Range: Self
Duration: 1 hour + 2 / Rank
Experience Multiple: 250
Base Chance: 30%
Resist: May not be resisted
Storage: Potion
Target: Self
Effects: The Adept cloaks their own mind so that
their thoughts cannot be detected or “read”, e.g. by
Telepathy or other mind reading talents, spells or
rituals. The Adept’s Resistance versus Mental
Attack is increased by 10 (+ 2 / Rank) while the
spell is in effect. Mind Cloak does not block ESP
or Empathy. The Adept can cast Mind Speech as
normal, and may choose to decline Mind Speech
from an external source.

16.5 General Knowledge Rituals
Binding Will (Q-1)
Range: 10 feet
Duration: 1 day + 1 / Rank

16 COLLEGE OF SORCERIES OF THE MIND
Experience Multiple: 500
Base Chance: 10% + 5% / Rank
Resist: Passive
Cast Time: 1 hour
Effects: The Adept may bind the will of an entity
so that they become the loyal retainer of the Adept
and serve them in all things. The target must be in
range during the ritual but may be unconscious or
otherwise restrained. The target gets an additional
resistance check at the beginning of each following
day and if required to do anything suicidal. If they
successfully resist, the binding is broken and they
are free. The target is then aware that they have
been under an external influence. The Adept may
never release the binding voluntarily, and until the
duration expires the binding continues, even if the
Adept dies (the target will do everything possible
to get the Adept resurrected). If the ritual backfires,
the Adept loses D10 from their Willpower in addition to any other backfire effect; Hypnotism (G-5)
cures this loss.

16.6 Special Knowledge Spells
Disruption (S-1)
Range: 15 feet + 15 / Rank
Duration: Immediate
Experience Multiple: 400
Base Chance: 25%
Resist: Passive
Storage: Investment, Ward, Magical Trap
Target: Object or Entity
Effects: The Adept causes an object or corporeal
entity (substantial rather than insubstantial) to
pulsate, doing [D + 1] (+ 1 / Rank) points of damage if the target fails to resist, and half damage to
any target that successfully resists. At Rank 10 or
above, the target’s internal structure and surface
will be partially disrupted, requiring the arts of a
healer or artisan of rank equal to (spell Rank / 3 ,
or Rank / 6 if the target resisted the spell) to prevent or remove the scarring (damage can be cured
normally).
Force Shield (S-2)
Range: Self
Duration: 10 minutes + 10 / Rank
Experience Multiple: 250
Base Chance: 40%
Resist: None
Storage: Potion
Target: Self
Effects: This spell creates an invisible protection
around the entire body which increases the Defence of the Adept by 5 (+ 1 / Rank) against all
attacks at any range.
Healing (S-3)
Range: Touch
Duration: Immediate
Experience Multiple: 375
Base Chance: 40%
Resist: None
Storage: Investment, Potion
Target: Entity
Effects: The Adept may heal the target of 2 (+ 1 /
Rank) Damage Points that have been removed
from the target’s Fatigue or Endurance. Endurance
damage is healed first, wrapping to Fatigue damage. On a multiple effect, the Damage Points restored may be multiplied.
Mental Attack (S-4)
Range: 15 feet + 15 / Rank
Duration: 10 seconds + 10 / Rank
Experience Multiple: 350
Base Chance: 25%
Resist: Active, Passive
Storage: Investment, Ward, Magical Trap
Target: Entity
Effects: The Adept may cast this spell at any entity
within range who can be seen or whose position
has been pinpointed via Telepathy. If the target
fails to resist, they fall unconscious and at Rank 15
and above the Target loses [D - 5] from Willpower.
The Willpower loss is recoverable via Hypnotism,
Remove Minor Curse or naturally at one point

every 3 days. Repeated casts may further reduce
the targets Willpower to a minimum of 1.
Mind Speech (S-5)
Range: 30 feet + 30 / Rank
Duration: 10 minutes + 10 / Rank
Experience Multiple: 400
Base Chance: 20%
Resist: May not be resisted
Storage: Investment
Target: Entities
Effects: Allows the Adept to choose a Principal
Target, plus up to 1 + 1 / Rank other targets, who
may then communicate mentally to each other as if
talking aloud in a group. The entities may be targeted via the Spell of Telepathy (at the time of
casting). If an entity subsequently goes further than
30 feet + 30 / Rank from the Principal Target, their
participation ceases until they re-enter that range.
A language in common is required for normal
speech. At Rank 10 and above visual images may
be transmitted. All communication requires the
same level of concentration as for normal speech.
The spell dissipates if the Principal Target is killed.
Multiple casts are a special case. If one or more
entities are part of two or more spells, and are
within range of all of their Principal Targets, then
every target of those spells may speak to each other
as if in one group. This can effectively increase the
working range of the effect.
Phantasm (S-6)
Range: Special
Duration: Concentration: no maximum
Experience Multiple: 400
Base Chance: 1%
Resist: May not be resisted
Storage: Investment
Target: Entity
Effects: This spell creates an insubstantial, magical
beast that must be targeted at a particular victim
which is either visible to the Adept or located
through the spell of Telepathy. The target may
interpret the beast as being a phantasm from their
own worst nightmare. Lack of imagination, however, does not offer any protection or immunity.
The Adept has no control over the shape of the
creature.
The phantasm appears beside the Adept, and then
travels straight towards the target at a TMR equal
to 1 + 1 / Rank, passing through walls, air etc.
When in melee range of the target the Phantasm
always does [D - 4] (+ 1 / Rank) damage each
pulse, including the pulse when it arrives at the
target.
The phantasm is only substantial to, and visible to,
the target and thus can be directly affected by the
target (using spells or magical weapons). Anyone
may cast a Mind College Special Counterspell over
the area currently occupied by the phantasm to
dissipate it. The phantasm is immune to all other
spells and weapons.
The Phantasm’s magic resistance, defence and
initiative are all zero and it always acts last in the
pulse. The phantasm has a single combined Endurance and Fatigue of 20 (+ 5 / 3 Ranks).
The phantasm dissipates when: its endurance drops
to zero, it is dispelled by counterspell, its target
dies or leaves the plane or the Adept drops or loses
concentration. The Adept has some awareness of
the phantasm’s location, and is aware of its dissipation.
Telekinesis (S-7)
Range: 15 feet + 15 / Rank
Duration: Concentration: maximum 10 seconds +
10 / Rank
Experience Multiple: 250
Base Chance: 25%
Resist: None
Storage: Investment
Target: Entity or object
Effects: Allows the Adept to lift and manoeuvre a
target weighting up to 2 pounds and move it at the
52

rate of 2 TMR. The Adept may move an additional
5 pounds or move the chosen target at an additional
3 TMR per Rank. The Adept may increase both
mass and speed by applying separate ranks to each.
Gross movement is automatic, although fine
movements require an MD check and should attract negative modifiers for range.
Telekinetic Rage (S-8)
Range: 30 feet + 5 / Rank
Duration: Instantaneous
Experience Multiple: 750
Base Chance: 1%
Resist: Passive
Storage: Investment, Ward, Magical Trap
Target: Area
Effects: The Adept creates an instantaneous, magical storm that fills the area within range with roiling, body-wrenching forces. All objects and entities that fail to resist, are hurled away from the
Adept until they are out of the area of effect, or
they strike a solid barrier.
The storm inflicts [D - 5] (+ 1 / foot travelled) of
magical damage and is inflicted once only, at the
time of casting. The damage is not contingent on
striking an object. Additional damage may be done
on hitting a wall, entity or other substantial object.
Telepathy (S-9)
Range: 30 feet + 30 / Rank
Duration: 10 minutes + 10 / Rank
Experience Multiple: 550
Base Chance: 10%
Resist: Passive
Storage: Potion
Target: Self
Effects: The spell gives the Adept the ability to
read the surface thoughts of any entity within range
who fails to resist. Each target may only resist each
Telepathy once, regardless of whether they move
in or out of the range of the spell.
Each pulse, the Adept may use their telepathy to
either locate the minds that are within range, or to
read the thoughts of a particular mind. The Adept
need not know the language of the entity whose
thoughts are being read, but will have difficulty
comprehending the thoughts of alien minds. Animal thoughts are usually very primitive and can
summed up in a single word, e.g. “food”, “danger”,
“sex.”
It is not possible to utilise a target’s senses and the
Adept may only gain an impression of physical
sensations if the entity is concentrating on them,
e.g. savouring a meal.
Telepathy may be used to target the Mind Speech,
Mental Attack and Phantasm spells.
Targets are not aware that their thoughts are being
read. Concentration checks should be required of
targets who try to control their thoughts by reciting
poetry etc.
Transmutation (S-10)
Range: 15 feet + 15 / Rank
Duration: 10 minutes + 10 / Rank
Experience Multiple: 1000
Base Chance: 1%
Resist: Active, Passive
Storage: Investment, Ward, Magical Trap
Target: Entity or Object
Effects: The Adept may transmute and rearrange
the elemental components of any target that fails to
resist, transforming it into any form of the same
mass that the Adept desires. A living entity can
only be rearranged into another living form, and
objects into objects. This spell cannot confer skills
or magical abilities, but the target will acquire the
non-magical features of the new form such as
wings, gills etc, and may use them. If the new form
is maintained without disruption for the duration of
the spell, the target will revert to its original shape.

17 COLLEGE OF NAMING INCANTATIONS

17 The College of Naming Incantations (Ver 2.0)
The College of Naming Incantations is concerned
with the essential truths and underlying realities
that make up the world and with the knowledge of
auras and true names. Naming Incantations is one
of the two oldest Colleges of Magic and just as the
Entities branch grew out of runic magic so too do
the Thaumaturgical Colleges have their roots in
Naming. Adepts of this College are commonly
called Namers.

Ranking Names
• Learning a Generic True Name takes a day, while
learning an Individual True Name takes one week
(see §3.9)

rank all Counterspells as General Knowledge spells
of the Namer College. Once a Namer begins to
rank a Counterspell it will count towards the Namers MA limit for spells and rituals as normal.

• Further ranking of both Generic and Individual
True Names takes 1 week × Rank to be achieved.

For full details on the use of Counterspells see
§10.2.

• Ranking Names costs no EP.

17.6 General Knowledge Rituals

• The maximum Rank for True Names is 20.

The Naming College holds that magic is a form of
deception, a manipulation of reality, whereby
Mages use mana to impose their will on the world.
The College’s abilities include divining the true
nature of things and enforcing those truths by
protecting against and preventing magic. Living
beings express their true nature and intrinsic essence in their auras and names, and Namers study
these in order to understand, protect, restore, and
gain control over them. It is said that in the ancient
days Namers were capable of commanding the sea,
the wind, and the rocks by their names — but if
this is true then that knowledge has long been lost.
Namers still learn the names of the plants that grow
in the earth, but they have little influence over
them.

• Ranking of Names may be done in combination
with ranking either a magical or non-magical ability.

Dissipation (Q-1)
Target: Entity, Object, Area, Volume
Base Chance: As per Counterspell + Ritual preparation
Cast Time: 1+ hours, maximum 10
Actions: Concentration
Concentration Check: Standard
Effects: By engaging in Ritual Spell Preparation a
Namer may use a Counterspell to dissipate the
effects of a spell. The Namer must perform at least
one hour of Ritual Spell Preparation at the end of
which they must cast the appropriate Counterspell,
and specify the name of the spell to be dissipated.
Only spells (not rituals) may be dissipated using
this technique. It is not possible to achieve Rank
with this ability since it is not an independent ritual, but rather a specialized use of Ritual Spell
Preparation.

Many Namers also learn the healing arts, and perhaps this is linked to a desire to restore beings to
their true state.
Given its abilities in neutralising magic, and the
low Magical Aptitude requirement, the College
attracts considerable interest from individuals
engaged in the arts of war. Many Adepts use it as a
means to protect themselves against hostile magics,
while they operate in a more physical manner.
Traditional Colours
No particular colour has a strong association with
the College, as astrologically magic is of all colours, and of none.
Traditional Symbols
Members of this College sometimes wear small
symbols made of iron, (insufficient to cause them
any inconvenience), symbolising their ability to
neutralise magic. Circles or spheres are very common, harkening perhaps to circles of protection.

17.1 Restrictions
Adepts of the College of Naming Incantations may
practise their arts without restriction. Some abilities may require that the Namer know a particular
Generic or Individual True Name, or have learned
a particular Counterspell.
The MA requirement for this College is 1.

17.4 Talents
Detect Aura (T-1)
Experience Multiple: 75
Base Chance: 2 × PC ( + 5 / Rank)
Resist: May only be actively resisted
Target: Entity, Object, Area, Volume
Effects: The Base Chance is reduced by 1% for
every foot after the first five from the Adept. See
the Detect Aura Talent (§9.1) for the results of this
talent. In addition to other information gained, the
Namer also receives the target’s Generic True
Name, if any.
Expel Magic (T-2)
Experience Multiple: 75
Resist: May only be passively resisted
Target: Object, Area, Volume
Effects: This talent allows the Namer to dissipate a
magical spell stored in a Ward, Magical Trap,
Potion, or Invested Item. In order to use this talent,
the Namer must specify the name of the spell to be
affected and cast the appropriate Counterspell on
the target with the specific intent of dissipating the
stored magic. The chance of the stored magic resisting destruction is 50% [( + 3 / Rank of the
stored magic) (3 / Rank of this Talent)]. If successful all of the magic of the same type stored within
the target is destroyed. The appropriate Counterspell is the one that affects the magic stored not the
storing magic. For example, a Ward of Enchanted
Sleep would require the use of an E & E General
Knowledge Counterspell. Possessions gain a single
Resistance Check but use the better of the chance
above, or their wielder’s MR.
Quick Cast (T-3)
Effects: Namers may cast any Counterspell that
they know without preparing it first.

17.5 General Knowledge spells

17.2 Base Chance Modifiers
The following numbers are added to the Base
Chance of performing any talent, spell or ritual of
the College of Naming Incantations:
Never before encountered target’s generic
type
Has not learned target’s Generic True
Name
Each Rank achieved with target’s Generic
True Name
Each Rank achieved with target’s Individual True Name
All modifiers are cumulative.

• Namers may Rank one Name in addition to other
forms of ranking.

15%
10%
+1%
+2%

In addition, each Rank achieved with the target’s
Individual True Name reduces the target’s Magic
Resistance by 1%.

17.3 Benefits
Language
Due to their knowledge of True Names, Namers
may Rank any Language in the Protonic Language
Group as if they already know another language in
that group at Rank 5. (See Languages §39.7).

The entire general spell knowledge of the Namer
college consists of the ability to cast Counterspells.
A Namer may cast a Rank 0 Counterspell against
any College of magic with which they are familiar.
Counterspells at Rank 0 do not count towards the
Namers MA limit for spells and rituals.
Familiarity with all of the commonly encountered
Colleges will be taught to a Namer during their
apprenticeship, and beginning Namers will have
the ability to cast all of the Counterspells of the
standard Colleges at Rank 0. If a Namer encounters
Colleged magic of a form with which they are not
familiar they may familiarize themselves with the
College by one of the following methods:
• By using the Ritual of Divination on an Adept of
that College.
• By Divinating a magical effect produced by that
College, provided that it is still in effect.
• By spending a day’s study with a Namer who is
already familiar with the College.
Once they have done this they will be able to cast
Rank 0 Counterspells against that college.
Unlike other Adepts, Namers may gain Rank with
Counterspells that are not of their College. Namers
53

17.7 Special Knowledge Spells
Bane (S-1)
Range: 10 feet + 10 / Rank
Duration: 30 seconds + 5 / Rank
Experience Multiple: 300
Base Chance: 20%
Resist: Passive
Storage: Investment, Ward, Magical Trap
Target: Area
Effects: This spell strengthens reality and stabilizes
the mana in an area 15 feet in diameter (+ 10 / 5
Ranks) such that all magical Cast Chances are
reduced within the area by 5% (+ 3 / Rank). This
will affect spells and rituals, and talents with base
chances. The spell has no affect on stored magics
(such as invested items), shaped items, or magic
without base chances.
Banishment (S-2)
Range: Self
Duration: 10 seconds + 10 / Rank
Experience Multiple: 200
Base Chance: 20%
Resist: No
Storage: Potion
Target: Self
Effects: Through use of this spell, a Namer may
banish a summoned being back to its own plane.
While the banishment spell is in effect, the appropriate Counterspell cast by the Namer at a summoned entity will cause the entity to return to its
own plane, unless it resists. The Counterspell must
match the type of spell or ritual used in the summoning of the creature. In general the spells/rituals
affected are the elemental summonings (Summon
Fire Elemental, etc.), all Greater Summonings,
Dark/Light Sphere conjuration and Fire college
Efreeti and Salamander summoning. The Call
Patron ability of Agents is not classified as a summoning spell and is not affected. The being to be
banished may actively and passively resist the
Counterspell.
Compel Obedience (S-3)
Range: 15 feet + 5 / Rank
Duration: Concentration: max. 10 minutes + 10 /
Rank
Experience Multiple: 400
Base Chance: 20%
Resist: Active, Passive
Target: Entity
Storage: None
Effects: The Adept may cast this spell over 1 (+ 1 /
4 Ranks) targets whose Generic True names are
known to them. Those targets who fail to resist

17 COLLEGE OF NAMING INCANTATIONS
may be commanded by the Adept to perform actions that are both within their physical capabilities
and in their true natures. Commands are given in
the Namer tongue and will be understood by all
entities. Commands must be short and simple, such
as: “Stop!”, “Wait here”, “Follow me”, “Hide
under the table”. Entities can only be compelled to
perform actions that they might perform naturally.
For example, brigands who were involved in a
combat might be compelled to “Flee!”, but if those
same brigands felt they were winning the fight,
they would heed no such compulsion but could
perhaps be directed to a different target. If the spell
is cast at targets with different GTNs the Namer
must use the lowest applicable base chance modifier.
If the Adept chooses to pronounce a target’s Individual True Name as part of the spell then only one
entity may be affected but the Namer is vested with
much greater control over that entity, even against
its nature. It is possible for the target to defy the
Namer, but there are serious consequences for
disobedience. Should the target decline to obey any
command of the caster that is not obviously suicidal, they must make a Willpower check of [1 × WP
- (1% per Rank that the Namer has with the target’s
ITN)]. This check does not break the spell. Should
the target fail their check, they will feel great pain
and immediately take damage equal to half of the
Namer’s Rank with this spell. This damage cannot
be resisted.
Disjunction (S-4)
Range: 10 feet + 10 / Rank
Duration: 1 minute + 1 / Rank
Experience Multiple: 300
Base Chance: 30%
Resist: Passive
Storage: Investment, Ward, Magical Trap
Target: Object, Area
Effects: This spell prevents stored magics within an
object or area from coming into effect. Magics that
are affected by this spell include Wards, Invested
items, Potions, Magical Traps, and permanent
magics that need to be triggered. If a potion under
the effects of a Disjunction is consumed, the potion
will take effect after the spell effect ceases, provided it is still inside an entity. Other items will
simply be unable to be triggered, and no charges
will be lost.
Dispel Magic (S-5)
Range: Self
Duration: 5 seconds + 5 / 4 Ranks
Experience Multiple: 400
Base Chance: 5%
Resist: No
Storage: Potion
Target: Self
Effects: While the Dispel Magic is in effect, the
appropriate Counterspell cast at a target may dissipate magic. The Counterspell must match the type
of spell to be dissipated, and the Adept must specify the name of the spell that they wish to remove.
If the Counterspell is successfully cast, the chance
of the magic being dispelled is 50% [(+ 3 / Rank
with Dispel Magic) (3 /Rank of the target magic)].
This spell cannot remove the effects of rituals, or
remove curses.
Forbidding (S-6)
Range: 10 feet + 10 / Rank
Duration: 10 minutes + 10 / Rank
Experience Multiple: 250
Base Chance: 30%
Resist: Passive
Storage: Investment, Ward, Magical Trap
Target: Area
Effects: This spell creates a thin, invisible wall, 10
feet high and 20 feet long. The Adept may increase
either height or length by 1 foot per Rank. This
barrier obeys all of the usual rules for insubstantial
walls. A single Generic or Individual true name is
crafted into the forbidding. To those entities whose
names are contained therein, or if seen by means of

Witchsight or similar, the wall appears bluish and
crackling with magical energy. If a Generic True
Name is in the forbidding, then to those named
who fail to resist upon initial contact the forbidding
is completely solid to them and they are unable to
pass through it. If they resist, the barrier is insubstantial, as it is those who are not named by it. If an
Individual True name is placed in the forbidding
then in addition to the Generic effects, the entity
must resist each contact with the barrier or suffer
[D - 4] + 1 / Rank damage, even if they are able to
pass through the wall because they initially resisted.
Mana Sense (S-7)
Range: Self
Duration: 5 minutes + 5 / Rank
Experience Multiple: 200
Base Chance: 20%
Resist: No
Storage: Potion
Target: Self
Effects: This spell allows the adept to “sense” the
mana flows within 10 feet (+ 10 / Rank). If a spell
is cast or magic triggered within range, the Adept
will see it flying off towards its target. Similarly, if
the target of any spell is within range, the Adept
will see the magic impact. If the Adept chooses
magical Pass actions of Concentration with this
spell in effect, they will be able to see Adepts
drawing mana, and be able to see if a target resists
a spell or not. While concentrating the Adept will
have a (2 × PC) chance of being able to distinguish
the College of the magic they can see, the name of
the spell, and whether the spell is low, medium,
high or very high in rank.
Scry Shield (S-8)
Range: 10 feet + 5 / Rank
Duration: 10 minutes + 10 / Rank or Special
Experience Multiple: 300
Base Chance: 20%
Resist: No
Storage: Investment, Ward, Magical Trap
Target: Volume
Effects: This spell protects an area from scrying by
Wizard Eyes, Crystals and Waters of Vision,
Bard’s Ear and similar divinatory magics of a Rank
equal to or less than the rank of the Scry Shield. It
does not prevent normal vision, infravision, Witchsight and similar spells. A Scry Shield is a shell
over the protected volume, so once the area is
penetrated by any means, e.g. on foot or by flying,
spells cast inside the protected volume work normally. At Rank 20 this spell alarms the Adept that
an attempt to divine into the volume by magical
means has taken place, provided that the Adept is
within the volume at the time. This spell may be
cast as a ritual if the Adept so chooses. In this form
casting takes 10 hours and the duration is increased
to 4 weeks (+ 1 / Rank).
Spell Barrier (S-9)
Range: 10 feet + 5 / Rank
Duration: 1 minute + 1 / Rank
Experience Multiple: 300
Base Chance: 30%
Resist: No
Storage: Investment, Ward, Magical Trap
Target: Volume
Effects: The Adept creates a thin, glowing, translucent wall which blocks the passage of magic. The
barrier is either 10 feet high and 20 feet long as a
wall, or 10 feet high and 5 feet in radius as a ring.
The Adept may increase any dimension — other
than thickness — by 1 foot per Rank. This barrier
obeys all of the usual rules for insubstantial walls.
Any magic cast in such a way that a direct line
drawn from the caster to their target passes through
the wall (from either side) has a 40% [(+ 3 / Rank
with this spell) (3 / Rank of the target magic)]
chance of having its energies dissipated. If a spell
passes through more than one Spell Barrier only a
single roll for dissipation should be made, with the
highest dissipation chance being used.

54

True Seeing (S-10)
Range: 10 feet + 5 / 2 Ranks
Duration: 30 seconds + 10 / Rank
Experience Multiple: 300
Base Chance: 25%
Resist: No
Storage: Investment, Ward, Magical Trap
Target: Area
Effects: All entities within the area of effect have a
chance of detecting things in the area that have
been magically altered to appear other than they
truly are. True Seeing may reveal the visual component of illusions, and entities or objects that are
invisible, insubstantial, or have been magically
rearranged or transformed (such as by curses). The
chance of an observer detecting such alterations or
concealments will depend upon the rank of this
spell, the rank of the concealing or transforming
magic, and the perception of the observer. For odd
magical abilities without a Rank the GM should
substitute some appropriate alternative (such as
[MA - 10]).
True Seeing is of lesser or equal Rank: Slight
imperfections may be revealed, (e.g. invisible
figures shimmer a little, the colour of an illusion
may appear a bit off, etc.), and there is a (1 × PC)
chance of an observer scrutinizing the area noting
this.
True Seeing is of higher Rank: More major imperfections may be noticed. (e.g. invisible figures
have a slight will-o’-the-wisp glow, toads that are
really Princes may have tiny gold crowns, ). The
detection chance rises to (2 × PC).
True Seeing is 10+ Ranks higher: The imperfections in concealing and transforming magics become quite obvious, (e.g. invisible figures appear
ghostly, illusions may appear painted or translucent, etc.). Detection is automatic.

17.8 Special Knowledge Rituals
Divination (R-1)
Range: 5 feet + 1 / Rank
Duration: Immediate
Experience Multiple: 250
Base Chance: 40% + 10 / Rank
Cast Time: 1 hour or 3 hours
Resist: No
Target: Entity, Object, Area
Material: None
Actions: Concentration
Concentration Check: Standard
Effects: There is no possibility of backfire from
this ritual. By use of this ritual a Namer may determine if an individual, object, or area is currently,
or has been recently, under the effects of magic a
spell by employing the Ritual of Magic Divination.
If the ritual is successful, the nature of all magic in
effect (exact names and Colleges) is revealed to the
Namer. If the magic is of non-college origin general effects are revealed. In the case of magic that
is no longer in effect, for each 5% under the Cast
Chance that the Namer rolled, magic that expired
an extra week ago is revealed. For example if a
Namer rolled 12% under their Cast Chance magic
that expired up to two weeks ago would be revealed — in addition to all magic currently in
effect.
If the Namer wishes they may perform an Ancient
Divination. The Base Chance of the ritual is reduced to 40% (+ 2 / Rank), and the Cast Time
increased to 3 hours. If successful the Namer will
learn the exact nature of all enchantments, magical
mechanisms, triggering conditions, curses, sideeffects, etc., placed upon an entity or object even if
they are of non-college origin. If an object has an
Individual True Name the Ancient Divination will
reveal its existence, though not the actual name.
Expulsion (R-2)
Range: 5 feet + 1 / Rank
Duration: Immediate
Experience Multiple: 300
Base Chance: 10% + 5 / Rank

17 COLLEGE OF NAMING INCANTATIONS
Cast Time: 1 hour
Resist: Active, Passive
Target: Entity
Material: None
Actions: Concentration
Concentration Check: Standard
Effects: This Ritual will return one entity to its
plane of origin, regardless of how it got to the
current plane. Upon completion of the ritual, if the
target fails to resist, they will be immediately returned to the point from which they left their plane
of origin.
Interregnum (R-3)
Range: 10 feet
Duration: Special
Experience Multiple: 250
Base Chance: MA + 4% / Rank
Cast Time: 2 hours
Resist: Active, Passive
Target: Entity or Object
Material: None
Actions: Concentration
Concentration Check: Standard
Effects: The targeted entity or object has all magical effects currently upon them, that are of lesser or
equal rank to the Interregnum, suspended. Whilst
suspended their durations will not reduce, but the
magic will have no effect. The duration of the
Interregnum may be chosen by the Adept at the
time of casting, from a minimum of 1 day to a
maximum of:
Rank

Duration (maximum)

0–10
11–15
16–19
20

1 day (+ 1 / Rank)
1 month
3 months
1 year

Remove Curse (R-4)
Effects: Namers have a greater ability to remove
curses than do the Adepts of other Colleges. Nam-

ers learn the normal Remove Curse ritual (see
§11.3) as R4 of this College, but gain a bonus to
Base Chance of + 2 / Rank for Minor curses and +
1 / Rank for Major curses (including Death
Curses).
Sealing (R-5)
Range: 20 feet +20 / Rank
Duration: 1 day + 1 / Rank
Experience Multiple: 300
Base Chance: 20% + 4 / Rank
Cast Time: 1 hour
Resist: None
Target: Area
Material: Chalk, paint, blood, cornmeal or other
symbol making materials
Actions: Chanting and inscribing symbols
Concentration Check: Standard
Effects: This Ritual seals an area against entities
from a single, specific, named plane. The name of
the plane must be known to the Adept, and the
name of the plane that the Adept is currently occupying cannot be used. No entity whose plane of
origin has been sealed against can voluntarily enter
the sealed area. They will stop at the boundary and
refuse to go any further. Any entity taken into the
area against their will (or without their knowledge,
e.g. unconscious) will attempt to leave the area as
quickly as possible. If an attempt is made to summon an entity from the named plane into the area
the summoning will fail. Any entities from the
named plane who are inside the area when the
sealing is created are unaffected, but should they
leave the area they will be unable to re-enter it.
True Form (R-6)
Range: 5 feet
Duration: Immediate
Experience Multiple: 300
Base Chance: 20% + 3 / Rank
Cast Time: 3 hours
Resist: Active

55

Target: Entity, Object
Material: None
Actions: Concentration
Concentration Check: Standard
Effects: By means of this ritual the Adept may
force a target that has been magically altered,
cursed, or rearranged into a form other than their
natural one to assume their true form and nature. It
will not remove effects that could occur naturally.
For example, the ritual would restore the form of a
human that had been cursed into the shape of a
toad, and would return to flesh a human turned to
stone but would do nothing to remove a curse of
weeping sores or restore a lost limb.
True Speaking (R-7)
Range: 10 feet
Duration: 30 minutes
Experience Multiple: 300
Base Chance: 40% + 3 / Rank
Cast Time: Special / 1 hour
Resist: Active, Passive
Target: Entity
Material: None
Actions: Asking questions
Concentration Check: Standard
Effects: By means of this ritual the Adept may
attempt to force an entity who is present to speak
the truth. The Adept must prepare for 30 minutes
after which they may question the entity for the
remaining 30 minutes of the 1 hour ritual. The
effects of the ritual do not last beyond the hour.
The target need not answer or speak at all, but if
they fail to resist and they choose to answer the
Adept’s questions, they must, to the best of their
knowledge, speak no falsehoods. They need not
volunteer information. The GM rolls for the success of this ritual and need not inform the Adept’s
player of the result.

56

18 COLLEGE OF AIR MAGICS

18 The College of Air Magics (Ver 2.1)
The College of Air Magics is concerned with the
environment of air and the weather. It should be
noted that there used to be a third aspect to this
college, but the Ice Mages’ increasing interest in
ice and cold (rather than air) was considered totally
unacceptable and they were expelled (to form the
independent College of Ice Magics).
When trying to ascertain the effect of spells which
change the weather, especially wind, please read
the notes accompanying the Weather Table. An
area may be affected by more than one Air College
spell or ritual at the same time providing that each
change affects a different aspect of the weather.
Traditional Colours
Air Mages traditionally dress in blues and greys.
Due to their habit of being outdoors in all weathers,
Adepts of this college tend to wear practical clothing and avoid long trailing cloaks or skirts.
Traditional Symbols
Given the ephemeral nature of their element, Air
Mages have little use for symbols, usually allowing
their traditional colours to advertise their profession.

18.1 Glossary
Totally Enclosed is defined as being surrounded by
walls or earth in a windowless area of less than 100
feet in any dimension where there is no direct and
immediate communication with the air outside.
Partially Enclosed is defined as being in a cave or
building or similar walled or earth enclosed area of
greater than 100 feet in its smallest dimension, or
in a smaller area, but having means of direct contact with the air outside, such as through an open
window or portal.
On top of a Mountain is defined as being within ten
minutes walking distance of the peak. Mountains
have an enormous impact on the weather patterns.

18.2 Restrictions
Adepts of the College of Air Magics may only
practise their arts if they are in contact with the air.
They may never practise Air magics while underwater or in a vacuum. They may never summon
avians into an environment where they could not
survive.

Example
Hypothermia (at high altitudes), flying
sickness, altitude sickness.

3. General knowledge relating to being airborne.
Example
hazards.

Safe velocities, altitude, down drafts, flight

4. Due to their close association with the environment, the Adept can modify the Force downwards
and Gauge upwards on the Weather Scale Table by
1 per 4 ranks, for themselves.
Detect Fumes (T-2)
Range: 10 feet + 10 / Rank
Experience Multiple: 100
Effects: The Adept may detect the presence of
fumes and vapours and has a chance equal to the
Adept’s modified perception (+ 3 / Rank) of accurately identifying each smell present. The Adept
must spend a pulse sniffing the air while applying
this talent. If the Adept’s sense of smell is not
operating, for whatever reason, then this talent may
not be used. This talent may only be attempted
once per set of smells. When new smells are introduced, the Adept may be able to identify these new
smells by attempting this talent again, but any
previously unidentified smells will remain so.
Predict Weather (T-3)
Experience Multiple: 50
Base Chance: 30% + 4% / Rank
Effects: The Adept may predict the local weather
over the next day (+ 1 / 3 Ranks). The Adept must
be outside or able to see the sky to use this ability.
If the Adept has resided in an area for a length of
time they will be more familiar with the local
weather. If they have lived in the area for over one
month they will receive a bonus for predicting
weather in the season they are familiar with. If they
have lived in the area for over one year they will
receive a bonus in any season.
Time

Familiarity

Bonus

1 month Current season + 5% †
1 year
Complete
+10% †
†only one modifier applies.

18.5 General Knowledge Spells

The MA requirement for this college is 13.

18.3 Base Chance Modifiers
The following modifiers are added to the Base
Chance of performing any talent, spell or ritual of
the College of Air Magics. Only one modifier may
be applied:
Caster is underground or totally enclosed
Caster is only partially enclosed
Caster is flying, or otherwise not in contact
with any liquid or solid surface
Caster is standing in natural winds of 40 mph
or greater
Caster is more than 100’ above ground (flying or falling etc.)
Caster is above all landmarks within 10 miles
and at least 500’ above the ground (flying or
falling etc.)
Caster is on top of a mountain

2. They will not suffer from the effects of high
altitude. This talent only works up to a height of
2,000 feet per rank.

-15
-5
+5
+10
+10
+20

+20

18.4 Talents
Aerial Affinity (T-1)
Experience Multiple: 75
Effects: Due to their close association with the
environment of air, Air Mages have the following
abilities:
1. The Adept can modify any flying or landing
rolls by 1 step better per 5 full ranks.
Example
The Air Mage comes in for a landing and
the GM tells the player that they need to make a 3 × AG
landing roll, but because the Air Mage has this talent at
rank 12 they need only make a 5 × AG landing roll.

Calm (G-1)
Range: 50 feet + 50 / Rank
Duration: 15 minutes + 15 / Rank
Experience Multiple: 100
Base Chance: 50%
Resist: None
Storage: Investment, Ward, Magical Trap
Target: Volume
Effects: The Adept halts all gross or extreme nonmagical air movement within a volume of 10 foot
(+ 10 / Rank) cubed. The entire volume must be
within the Adept’s range. The air within the volume will stay fresh and will slowly intermingle
with air outside the volume. Once cast the volume
may not be moved.
Feather Falling (G-2)
Range: 10 feet + 10 / Rank
Duration: 30 minutes + 30 / Rank
Experience Multiple: 200
Base Chance: 40%
Resist: None
Storage: Investment, Ward, Magical Trap
Target: Entity
Effects: This spell causes the target to drift gently
downwards if they fall more than 5 feet (should the
target be falling faster than 5 feet per pulse they
will undergo a magical deceleration over one
pulse). The target will be subject to wind effects
while drifting downwards.
Mage Wind (G-3)
Range: 10 feet + 10 / Rank
57

Duration: 1 hour + 1 / Rank
Experience Multiple: 125
Base Chance: 30%
Resist: None
Storage: Investment, Ward, Magical Trap
Target: Object
Effects: The Adept causes 1 wind-driven object to
be affected by a magical wind with an effective
speed of up to 5 miles per hour (+ 1 / Rank). Only
the object targeted is affected by the magical wind.
By concentrating the Adept may alter the direction
and strength of the Mage wind.
Mist (G-4)
Range: 20 feet + 20 / Rank
Duration: 10 minutes + 10 / Rank
Experience Multiple: 100
Base Chance: 20%
Resist: None
Storage: Investment, Ward, Magical Trap
Target: Volume
Effects: The Adept conjures 1000 cubic feet of mist
(+ 500 / additional Rank) of any shape entirely
within the Adept’s range. The mist must be at least
1 foot in any dimension and may have detail no
finer than 1 foot in size. The visibility in this mist
is normally 30 feet (6 hexes). If the wind is
stronger than a light breeze, the Adept must actively concentrate to maintain the mist. Otherwise,
while the Adept actively concentrates, the mist
may be changed in one of the following ways:
1. The Adept may move the mist up to 1 hex/pulse
so long as it remains within range.
2. The Adept may change the visibility within the
mist down to a minimum visibility of 6 hexes (-1 /
4 full Ranks).
Speak to Avians (G-5)
Range: 10 feet + 10 / Rank
Duration: 1 hour + 1 / Rank
Experience Multiple: 75
Base Chance: 50%
Resist: None
Storage: Investment, Magical Trap, Potion
Target: Self
Effects: The spell allows the target to communicate
with any one type of aerial life within range of the
spell. This communication usually consists of
sound and gesture. If the Adept wishes to converse
with several different types of aerial life concurrently then they may cast this spell multiple times.
Storm Calling (G-6)
Range: Special
Duration: 60 minutes + 30 / Rank
Experience Multiple: 200
Base Chance: 40%
Resist: None
Storage: Investment, Magical Trap
Target: Special
Effects: The Adept may summon any storm front
which may exist anywhere in sight. If no front can
be seen, the spell can still be cast but the Base
Chance is modified by -20. Generally, a storm
front can be seen for 20 to 30 miles. Upon reaching
the spot occupied by the Adept at the time of casting, the storm front will slow and finally cease
moving and begin to downpour (snow, rain, hail,
sleet or whatever else the GM feels the clouds may
contain). The storm front will take D10 × 3 minutes (1 / Rank) to arrive, minimum 1. Once the
duration has lapsed the weather will gradually
return to normal over a similar amount of time.
Summon Avians (G-7)
Range: 5 miles
Duration: Immediate
Experience Multiple: 125
Base Chance: 30%
Resist: Active, Passive
Storage: Investment, Magical Trap
Target: Avians

18 COLLEGE OF AIR MAGICS
Effects: The Adept may summon one avian per
Rank (minimum 1) to their current location. The
avians must be native to or present in the area and
the Adept must specify the species (one per spell)
that is being summoned. The Adept may summon a
specific avian providing it is within line of sight
(the species need not been known in this case). The
avian will arrive by the shortest flight path and
their initial reaction will be wary. The avians must
have a clear flight path to the Adept and will spend
up to 5 minutes attempting to reach the Adept.
Note that not all avians will have a movement rate
which will allow them to reach the Adept’s location within this time. This spell may not be cast if
the Adept is totally enclosed. If the avian being
summoned is fantastical it gains a +20% to its base
chance to resist this spell.
Vapour Breathing (G-8)
Range: 10 feet + 10 / Rank
Duration: 30 minutes + 30 / Rank
Experience Multiple: 150
Base Chance: 35%
Resist: Active, Passive
Storage: Investment, Ward, Magical Trap
Target: Entity
Effects: The target can breathe any gas including
otherwise poisonous vapours (e.g. Knockout Gas,
Noxious Vapours etc). This spell will not enable
the target to breathe in a vacuum or underwater.
Also, this spell will not protect the target from any
non-gaseous contents of the atmosphere, e.g. Sleep
Dust.
Wind Whistle (G-9)
Range: Special
Duration: D10 hours + 1 / Rank
Experience Multiple: 125
Base Chance: 40%
Resist: None
Storage: Investment, Potion
Target: Special
Effects: The Adept is able to create a wind over an
open space of up to 100 feet (+ 100 / Rank) diameter centred on the Adept. Outside of this area, the
wind will fade back to the prevailing wind over
half again the distance. The wind will build up over
D + 5 minutes (1 / Rank, minimum 1) and the
Mage must choose at the time of casting which
direction the wind will blow. The speed of the
wind is determined by rolling a D100 roll as follows:
Dice

Force

Speed (mph)

01–05
5
19–24
06–15
4
13–18
16–30
3
8–12
31–50
2
4–7
51–70
3
8–12
71–85
4
13–18
86–95
5
19–24
96–100 6
25–31
The Adept can choose to modify the dice roll up or
down by up to 2 × Rank points to determine the
actual wind speed once the dice have been rolled.
The effects of this spell override any existing wind
effects including already existing Wind Whistle
spells.
If the resulting wind is Force 5 or over, missile fire
will be affected, reducing base chances by (wind
speed/2)%.

18.6 General Knowledge Rituals
Windspeak (Q-1)
Range: 400 feet + 400 / Rank
Duration: 1 hour + 1 / Rank
Experience Multiple: 150
Base Chance: 2 × MA + 3% / Rank
Resist: None
Storage: None
Target: Area
Cast Time: 1 hour
Material: None
Actions: None
Concentration Check: Standard

Effects: The Adept can speak with the whispering
spirits of the wind, learning what they have seen or
heard and even soliciting their aid. All winds
within a range of 300 feet (+ 300 per rank) can be
communed with in this manner. There is no backfire.

18.7 Special Knowledge Spells
Air Blast (S-1)
Range: 10 feet + 10 / Rank
Duration: Immediate
Experience Multiple: 200
Base Chance: 30%
Resist: Active, Passive
Storage: Investment, Ward, Magical Trap
Target: Entity, Object
Effects: This spell projects a narrow and extremely
strong blast of air at the target. The blast will impact either on the target or on the first obstruction
blocking the path from the Adept to the target. If
the target weighs less than 100 pounds (+ 20 /
Rank) they will be thrown back 10 feet (+5 per 3
full rank). On landing the target suffers damage of
[D - 5] + 1 per 3 full ranks and must make a 1
times PS + AG check to remain standing. The
target is thrown in an arc away from the Adept and
the highest point of the arc equals (Rank + 1) feet.
Arrow Flight (S-2)
Range: 5 feet
Duration: 20 minutes + 20 / Rank
Experience Multiple: 200
Base Chance: 35%
Resist: None
Storage: Investment
Target: Arrows or Quarrels
Effects: The Adept can temporarily improve the
flying quality of 2 (+ 1 / 2 Ranks) arrows or quarrels. Anyone firing arrows or quarrels affected by
this spell has their Base Chance modified by +2%
(+ 1 / Rank). At rank 16+ the arrows are counted as
magical for determining which entities may be
harmed.
Avian Control (S-3)
Range: 100 feet + 100 / Rank
Duration: Concentration: maximum 10 minutes +
10 / Rank
Experience Multiple: 200
Base Chance: 20%
Resist: Active, Passive
Storage: Investment, Potion
Target: Avian
Effects: The Adept may control 1 (+ 1 / Rank)
avian within range unless the avian successfully
resists. If the avian is fantastical, it gains +20% on
its base chance to resist this spell. If the Adept fails
to gain control or loses control, the avian will
immediately attack the Adept. An avian which is
still controlled when the spell duration expires, will
flee the Adept’s presence. The Adept may recast
this spell before its duration has expired without
breaking concentration. When the duration of the
first casting expires the target gets another resistance check. The Adept may also choose to release
an avian from their control before the spell’s duration is up, whereupon it will flee as above. If the
avian is sentient and is made to take an action that
would endanger itself, another resistance check is
made.
Barrier of Wind (S-4)
Range: 5 feet + 1 / Rank
Duration: 30 minutes + 30 / Rank
Experience Multiple: 150
Base Chance: 30%
Resist: None
Storage: Investment, Ward, Magical Trap
Target: Entity
Effects: This spell forms a swirling pattern of wind
around the entity in all directions. Thrown and
missile weapons passing through the barrier have a
chance of being deflected from their course, to
impact harmlessly elsewhere. The deflection adds
5 (+ 2 / Rank) to defence against missiles. This

58

spell provides a bonus of 5 (+ 1 / Rank) in melee or
close combat.
Conjuring Air (S-5)
Range: 10 feet + 10 / Rank
Duration: 5 minutes + 5 / Rank
Experience Multiple: 150
Base Chance: 25%
Resist: None
Storage: Investment, Ward, Magical Trap
Target: Volume
Effects: The Adept may conjure anywhere within
their range 5 cubic feet (+ 5 / Rank) of sweet
smelling breathable air in any shape or shapes of
their choice. No dimension of the shape may be
less than 1 foot. The volume of air will hold its
shape versus any non solid intrusions for the full
duration. At the end of the duration the volume will
rapidly disperse into the surrounding area, mixing
with whatever atmosphere was already there.
Flying (S-6)
Range: 5 feet + 5 / Rank
Duration: 30 minutes + 30 / Rank
Experience Multiple: 250
Base Chance: 30%
Resist: Active, Passive
Storage: Investment, Potion
Target: Entity
Effects: This spell enables the target to fly at a
speed of 30 mph (+ 1 / Rank) by walking winds
and air currents. Initially it will take [D + 10 Rank] minutes for the wind currents to arrive.
After the wind currents have arrived the target can
take off and land as many times as they desire, but
to recall the winds to take off after landing will
take [D + 2] pulses. When taking off, it takes one
pulse to accelerate to full speed. If the target tries
to land, it will take one pulse to slow to a standstill.
The target can only travel into places where air
currents or winds are possible. This will not normally occur inside buildings or tunnels.
Gaseous Form (S-7)
Range: Self, touch at Rank 11+
Duration: 5 seconds + 5 / 2 Ranks
Experience Multiple: 300
Base Chance: 20%
Resist: Active, Passive
Storage: Potion, Invested, Magical Trap
Target: Entity
Effects: When this spell is cast the target (and
possessions) turn into a cloud of vapour, that appears as a heavy mist. The target may be affected
by magic spells. Normal or silvered weapons do
not harm the target, but magical weapons may. The
target may not use any possessions such as weapons, nor may the target cast or trigger magic.
Magical or Racial Talents may be used however.
The target may move up to TMR 2 in any direction
and pass through any opening that is not airtight.
When the duration of the spell expires, the target
will reform in the nearest hollow space large
enough to accommodate the target’s body.
Gliding (S-8)
Range: 5 feet + 5 / Rank
Duration: 5 minutes + 5 / Rank
Experience Multiple: 200
Base Chance: 30%
Resist: None
Storage: Investment, Potion
Target: Sentient Entity
Effects: This spell enables the target to glide at an
angle of approximately 1 + Rank feet horizontally
for every foot of descent. The target’s maximum
forward speed will be approximately 80 feet / pulse
(+ 1 / Rank) and their minimum vertical speed
(relative to the air around them) will be approximately 40 feet / pulse (-2 / Rank, minimum 1)
downwards. If the target is gliding in thermals or
down-drafts, their actual rate of descent (or ascent)
may vary. The actual rate of descent is up to the
GM. The target must keep their arms (or equivalent) outstretched until landing or they will fall
downwards, although the spell will resume if the

18 COLLEGE OF AIR MAGICS
target returns their arms to an outstretched position.
The target flies with their body horizontal and may
turn at a rate of 30 degrees / pulse (+ 5 / Rank)
while gliding. The spell will cause the target to
automatically become vertical when within 5 feet
of a horizontal surface. The target must then make
a flying roll when landing.
Knockout Gas (S-9)
Range: 30 feet + 10 / Rank
Duration: 5 minutes + 5 / Rank
Experience Multiple: 450
Base Chance: 15%
Resist: Active, Passive
Storage: Investment, Ward, Magical Trap
Target: Area
Effects: This spell creates a heavy cloud of knockout gas that rises 10 (+ 1 / Rank) feet above the
ground. The area has a diameter of 15 feet (+ 5 /
Rank). At ranks: 0 to 9 the cloud appears as a
translucent mist (i.e. does not inhibit vision) and is
pungent smelling, 10 to 15 the gas is invisible and
pungent smelling, 16 and above the cloud becomes
invisible and odourless. Any entity (including the
Adept) within the gas must resist. If they fail to
resist, they must roll under 2 × EN each pulse or
fall unconscious (not asleep) while they remain
within the gas. Any wind over 15 mph will disperse the gas in D-Force pulses. Once the gas is
dispersed, the duration expires, or a victim is removed from the area, any unconscious entities will
recover in 1 pulse (+ 1 / 5 full Ranks).
Lightning Bolt (S-10)
Range: 60 feet
Duration: Immediate
Experience Multiple: 225
Base Chance: 30%
Resist: Active, Passive
Storage: Investment, Ward, Magical Trap
Target: Entity, Object
Effects: The Adept may throw a single bolt of
lightning 60 feet long from their fingertips. The
bolt must extend the entire 60 feet and will deflect
off stone until it reaches its full extent. All targets
that are in the path of the bolt must resist or suffer
[D + 5] (+ 1 / 3 Ranks) damage (save for half damage). In addition, any target who fails to resist is
automatically stunned.
Lightning Strike (S-11)
Range: 15 feet + 5 / Rank
Duration: 30 minutes + 30 / Rank
Experience Multiple: 250
Base Chance: 35%
Resist: Active, Passive
Storage: Potion
Target: Entity
Effects: This spell creates sheets of lightning that
slowly arc around the entity’s body doing no damage. This is extremely obvious. The first time the
entity takes damage from a blow in combat and the
attacker is within range, a bolt will strike out at the
entity’s attacker doing [D - 4] (+ 1 / 2 Ranks)
damage (save for half damage). The spell then
dissipates. The Range is 5 feet (+ 5 / Rank). At
rank 10 or greater, any attacker who fails to resist
and is capable of being stunned is automatically
stunned. This spell cannot be cast on a target if
they are already under the effect of a Lightning
Strike spell.
Resist Cold (S-12)
Range: Touch
Duration: 1 hour + 1 / Rank
Experience Multiple: 100
Base Chance: 40%
Resist: None
Storage: Investment, Ward, Potion
Target: Entity
Effects: This spell protects the target from the
effects of cold temperature by increasing the
Gauge by 1 (+ 1 / 4 full ranks) up to a maximum of
Gauge 7 (Comfortable). It will totally protect the
target from the effects of Hypothermia at rank 11+.
In addition, the target suffers 1 (+ 1 / 4 full ranks)

less damage due to magical or non-magical cold
based attacks.
Shaping Cloud (S-13)
Range: 5 miles + 1 / Rank
Duration: Concentration: maximum 1 minute + 1 /
Rank
Experience Multiple: 150
Base Chance: 40%
Resist: None
Storage: Investment, Potion, Ward
Target: Cloud band
Effects: The Adept can cause some of a cloud bank
(cloud degree 5+) within range and line of sight to
slowly shape a different image in it each minute.
Each new shape will be gradually formed from the
old over the course of the minute. The shape in the
clouds is recognisable up to a range of 5 miles (+ 1
/ Rank). Measurements for this spell should be
taken from the ground (i.e. disregarding the height
of the cloud bank).
Whirlwind Vortex (S-14)
Range: 15 feet + 15 / Rank
Duration: Immediate
Experience Multiple: 650
Base Chance: 1%
Resist: Active, Passive
Storage: Investment, Ward, Magical Trap
Target: Entity
Effects: This spell creates a tornado around one
human sized target for every 3 or fraction ranks
(minimum 1). If the target fails to resist they are so
tossed and torn by the winds that muscles and
ligaments tear, joints dislocate, bones break, organs
rupture, and they perish. If the target successfully
resists they suffer [D - 4] (+ 1 / Rank) damage due
to excessive forces (instead of perishing). Targets
that fail to resist may be resurrected.
Whispering Wind (S-15)
Range: 100 miles + 100 / Rank
Duration: Special
Experience Multiple: 150
Base Chance: 40%
Resist: None
Storage: Potion
Target: Self
Effects: This spell sends a message to an entity
using the winds. The Adept must know the name of
the entity and be able to pick them out from a
crowd. The maximum length of the message is 5
words per rank. The messages will be in the voice
of the Adept at the same volume as when spoken,
and can only be heard by the target. There is a
noticeable but minor effect of swirling air when the
message is sent and received. The time taken for
the message to reach the entity (the duration) is 1
hour (+ 2 minutes per mile) - rank hours (minimum
of 1 hour).
Windstorm (S-16)
Range: 30 feet + 30 / Rank
Duration: 10 seconds + 10 / Rank
Experience Multiple: 200
Base Chance: 40%
Resist: None
Storage: Investment, Ward, Magical Trap
Target: Area
Effects: The Adept creates a windstorm of Force 9
centred on the Adept and extending out to the full
range of the spell. Once cast the Windstorm will
not move. The winds in the area of a windstorm are
random and violent, they do not prevail in a particular direction. All entities within this area except
the Adept and those in the same hex must check
against 2 × (Physical Strength + Agility) - 2 ×
Rank, every pulse, to regain their feet and/or remain upright. Every time an entity within the area
falls prone, they take [D - 2] damage. This damage
is physical. For the duration of this spell they have
their TMR halved. All entities attempting to use
missile or thrown weapons through or inside the
area of effect have their Base Chance reduced by 5
per every 2 hexes of Windstorm travelled through.

59

Wind Walking (S-17)
Range: 10 feet + 10 / Rank
Duration: 30 seconds + 30 / Rank
Experience Multiple: 350
Base Chance: 25%
Resist: Active, Passive
Storage: Investment, Ward, Magical Trap
Target: Entity
Effects: The target of this spell (and possessions)
turns into wind, keeping their approximate size and
appearance. In the first pulse, the spell’s effect is to
accelerate the target to full speed and in the last
pulse, the target will gradually slow. The distance
travelled over the first and last pulse is half that of
normal. The target moves through the air at 50 mph
(+ 5 / Rank). The target can only pass through an
area that they could pass through normally. The
target chooses the direction of the flight but the
speed may not be altered. The target will not collide with stationary objects during the flight, but if
rushing directly towards an object or wall, will
brush along or past it as a wind would. If blown
down a corridor to a dead end or similar situation,
the target will not stop but double back in a tight
loop. Anything dropped by the target will go in a
random direction.
Ball of Lightning (S-18)
Range: 35 feet + 10 / Rank
Duration: Immediate
Experience Multiple: 350
Base Chance: 30%
Resist: Passive
Storage: Investment, Ward, Magical Trap
Target: Entity or Object
Effects: This spell creates a ball of lightning which
shoots from the caster to the target. The ball is
utterly silent and moves in a straight line. Anything
standing between the caster and the target will be
struck instead. Upon striking anything the ball
explodes, in a bright flash, causing [D - 1] (+ 1 per
Rank) electrical damage. If the target successfully
resists the damage is halved (round up), otherwise
they are also blinded for Rank / 4 pulses (round
down).
Thunderclap (S-19)
Range: 20 feet + 20 / Rank
Duration: Immediate
Experience Multiple: 325
Base Chance: 30%
Resist: Passive
Storage: Investment, Ward, Magical Trap
Target: Area
Effects: The Adept causes the air in the targeted
area to violently compress with a loud crash. The
target area has a diameter of 5’ at Ranks 0–5, 15’ at
Ranks 6–12, 25’ at Ranks 13–19, and 35’ at Rank
20. The entirety of the affected area must be within
the caster’s spell range for the spell to be effective.
All those within the area suffer [D + 1] (+ 1 per 2
full Ranks) concussive damage (resist for half —
round up). Those failing to resist can hear nothing
except a loud ringing for Rank pulses. On a Double
or Triple effect any entities which fail to resist are
also stunned (normal stun recovery applies). Note
that this spell can be heard from a distance as per
normal thunder.

18.8 Special Knowledge Rituals
Air Spring (R-1)
Duration: Rank × Rank hours (minimum 1)
Experience Multiple: 150
Base Chance: MA + 5% / Rank
Resist: None
Storage: None
Target: Area
Cast Time: 1 hour
Material: None
Actions: None
Concentration Check: Standard
Effects: The Adept can open a one way portal 2
feet in diameter from the elemental plane of Air
from which fresh clean air flows. The rate of flow
is 5 cubic feet (+ 5 / Rank) per second. Addition-

18 COLLEGE OF AIR MAGICS
ally any modifier for being enclosed is reduced by
5.
Conjuring Air Elemental (R-2)
Range: 20 feet
Duration: Concentration: No maximum
Experience Multiple: 200
Base Chance: MA + 5% / Rank
Resist: None
Storage: None
Target: Air Elemental
Cast Time: 1 hour
Material: None
Actions: None
Concentration Check: Standard
Effects: The Adept may summon an Air Elemental
and bind it to temporary service by performing this
ritual. At the end of the ritual a Cast Check is performed. If the ritual is successful the Elemental is
summoned and controlled. If the ritual backfires
then the Elemental is summoned but no controlled
and will attack the summoner and companions.

Actions: Dance (2 FT when unencumbered)
Concentration Check: None
Effects: The Adept may change one or more of the
three components which make up the weather by
performing a ritual dance. The three components of
weather are:
• Precipitation, Degree
• Temperature, Gauge
• Wind, Force
The GM should consult the weather table and
advise the Adept of the current level of each of
these three components before they start dancing.
The Adept may change the current components by
a total of 1 +1 per 2 full ranks.
Example
At rank 12, the Adept can change a Force 3
wind into a Force 10 wind or Force 3 into Force 7 and
Degree 5 into Degree 2 cloud cover.

All the changes may be in any direction on the
table.

The Air elemental will always appear within 20
feet of the summoner. If has a combined Endurance and Fatigue equal to 15 (+ 15 / Rank). The
Elemental will remain until it is sent back to its
home plane by the Adept (with Special Knowledge
Counterspell of the college of Air Magics) or banished. If it is controlled by the summoner it will
remain controlled until the summoner’s concentration is broken.

This ritual will not cause weather effects outside
the normal climatic range of the area (as determined by the GM). The weather will change
gradually over (30 - 1 / Rank) minutes per level
shifted on the table and the three components will
change simultaneously. The area of effect is circular with a radius of 1 mile / Rank (minimum 1).
Casting this ritual counts as strenuous activity and
the Adept will lose fatigue. This ritual cannot backfire.

Control Weather (R-3)
Duration: 8 hours + 8 hours / Rank
Experience Multiple: 300
Base Chance: 2 × MA + 3% / Rank
Resist: None
Storage: None
Target: Area
Cast Time: 1 hour

Summon and Bind Cloud (R-4)
Range: 5 miles + 5 / Rank
Duration: 5 hours + 5 / Rank
Experience Multiple: 300
Base Chance: 2 × MA + 3% / Rank
Resist: None
Storage: None

60

Target: Clouds
Cast Time: 1 hour
Material: None
Actions: None
Concentration Check: Standard
Effects: This ritual will summon a quantity of
cirro-cumulus cloud and change its consistency to
that of packed cotton wool to provide a method of
transport. Items will pass through the cloud after
30 seconds unless supported by an Entity. The
cloud arrives during the last half hour of casting
and spends the first and last half hour of travel
rising to and descending from its natural travelling
height. The natural height of cirro-cumulus clouds
is between 2 and 4 miles above sea level (10,000 to
20,000 feet). The cloud may support 1 entity (+ 1 /
Rank) and transports them in a comfortable and
oxygenated environment, although it may be
slightly cold if improperly clothed. Flying entities
may take off or land on the cloud as long as the
above limit is maintained. If more entities land on a
cloud than it can carry, it will immediately start to
descend taking half an hour and dissipate upon
reaching the ground. The cloud appears to be a
normal cloud but is sufficiently soft to prevent
injury to anything that impacts it (it’s also nonflammable). The clouds will move with the prevailing winds or can be moved with magical winds
such as Mage Wind. By actively concentrating the
Adept can cause the clouds to move at a different
speed or in a different direction to the prevailing
wind. The Adept can alter the movement of the
clouds by 2 miles per hour (+ 2 / Rank) in any one
direction. The altitude of the clouds may not be
controlled. While in contact with the ground the
clouds will not move.

18 COLLEGE OF AIR MAGICS

18.9 Weather Scale Table
Force

Wind

Specification

Speed (mph)

0

Calm

Smoke rises vertically

<1

1

Light air

Direction indicated by smoke only

1–3

2

Light breeze

Wind felt on face, leaves rustle

4–7

3

Gentle breeze

Leaves and twigs in constant motion, wind extends light flag

8–12

4

Moderate breeze

Wind raises dust and loose paper, small branches move

13–18

5

Fresh breeze

Small trees in leaf start to sway, crested wavelets on inland waters

19–24

6

Strong breeze

Large branches in motion, whistling through trees

25–31

7

Near gale

Whole trees in motion, inconvenient to walk against wind

32–38

8

Gale

Twigs break from trees, difficult to walk

39–46

9

Strong gale

Slight structural damage occurs

47–54

10

Storm

Trees uprooted, considerable structural damage occur

55–63

11

Violent storm

Widespread damage

64–73

12+
Gauge

Hurricane
Temperature

Widespread damage
Specification

74 and above
Degrees Celsius

0

Arctic

Dangerous

-20 and below

1

Arctic

Salt water freezes

-15

2

Arctic

Branches become brittle

-10

3

Arctic-like

Dangerously cold if not dressed in winter clothing

-5

4

Sub-arctic

Water freezes, numbness, precipitation becomes hail, snow, sleet

0

5

Sub-arctic

Cold, uncomfortable if poorly clothed

5

6

Sub-tropical

Cool

10

7

Sub-tropical

Comfortable

15

8

Sub-tropical

Comfortable, warm

20

9

Tropical

Hot, uncomfortable if poorly clothed

25

10

Tropical

Uncomfortably hot, sub-tropical plants wither

30

11

Desert-like

Extreme heat, sub-tropical plants die, tropical plants wither

35

12+
Degree

Desert
Cloud

Debilitating, tropical plants die
Precipitation

40 and above
mm / Hour

0
1

Clear
Clear

Dry, high fire danger
Dry, fires easy to start

0
0

2

Clear

Comfortable

0

3

Sparse

Humid, uncomfortable in high temperature

1

4

Light

Damp

2

5

Cloudy

Drizzle, fog in cold conditions

3

6

Overcast

Showers

4

7

Heavy Cloud

Light rain, leaves move, fires difficult to start

5

8

Dark Cloud

Average rain

10

9

Low black cloud

Heavy rain, small branches move, small fires doused

15

10

Oppressive

Torrential rain, river rise, large fires doused

20

11

Oppressive

Flood warning, rivers burst their banks

25

12+
Oppressive
Flash floods
35
T-1 Aerial Affinity (Air Mage Talent) Due to their close association with the environment the Adept can modify the Force downwards, and Gauge upwards on
the Weather Scale Table by 1 per 4 rank, for themselves.
S-12 Resist Cold (Air Mage Spell) This spell protects the target from the effects of cold temperature by increasing the Gauge by 1 (+ 1 / 4 full ranks) up to a
maximum of Gauge 7.
R-3 Control Weather (Air Mage Ritual) The Adept may change the current components by 1 + 1 per 2 full ranks. The resulting weather effects cannot be outside
the normal climatic range of the area.
Example
At rank 12, the Adept can change a Force 3 wind into a Force 10 wind or Force 3 into Force 7 and Degree 5 into Degree 2 cloud cover. Each of the changes may be in any
direction on the table.

61

62

19 COLLEGE OF CELESTIAL MAGICS

19 The College of Celestial Magics (Ver 1.3)
The College of Celestial Magics is concerned with
the practice of those magic arts having to do with
the elements of light and dark and their contrasts.
There are four distinct divisions of the College of
Celestial Magics, each of which is concerned with
a different combination of light and dark
Solar Mages
Light
Star Mages
Light within Dark
Shadow Weaving Dark within Light
Dark Mages
Dark
Solar & Star Mages use the element of Light;
Shadow Weavers and Dark Mages use the element
of Darkness.
All members of the college must be associated with
only one division of this College, and may only
change divisions by forsaking their present division, and relearning the new division as if it were a
different college.
Most entities are aligned with either Light or Dark,
and Celestial Magics will often only affect entities
of the opposed aspect. Entities’ fear of their opposing element gives this College special power.
Whether an entity is Light aspected, Dark aspected,
or has neither aspect, is determined per the rules
given in Light and Dark Aspect (§19.8).

19.1 Restrictions
Adepts of the College of Celestial Magics may not
practise their arts in an area where their element is
not present.
A Magical Aptitude of 14 is required to learn this
College. Note also that certain spells of this college
may only be learnt by specific divisions (as listed
after the Spell Number).

19.2 Base Chance Modifiers
The Base Chance of performing any talent, spell,
or ritual of the College of Celestial Magics is
modified by the following numbers, based on the
division to which the adept belongs.
Aspect Modifiers
Solar Mage with a Solar Aspect
Solar Mage with a Lunar Aspect
Dark Mage with a Solar Aspect
Dark Mage with a Lunar Aspect

+1%
-1%
-1%
+1%

Lighting Condition Modifiers
An Adept of Celestial Magics is affected greatly by
the lighting conditions in their vicinity. The bonuses and penalties gained in this section apply
only to non-magical forms of Light and Dark.
Magical forms of the elements may at best neutralise any penalties suffered due to the natural elements. For the purposes of these modifiers, the
vicinity is deemed to be any bounded area around
the Adept (such as a room) or, if the Adept is in the
open, the area within 30’ of the Adept.
Shadow Weavers must be within a shadow that has
a defined edge within the vicinity (the shadow
must be large enough to contain the Adept, and
cannot be generated from the Adept’s possessions),
and Star Mages must be in direct light from point
sources (eg. casting a shadow), otherwise the lighting condition modifier is -25%.
The lighting modifiers are in the Celestial Lighting
Modifiers table (§19.9).

19.3 Talents
Speak to Creatures of Light/Darkness (T-1)
Range: 10 feet + 10 / Rank
Duration: Immediate
Experience Multiple: 50
Resist: None
Effects: This talent allows the adept to communicate in a limited fashion with non-sentient creatures. A Solar Mage or Star mage may communicate with those creatures who are light aspected,
whereas a Shadow Weaver or Dark Mage may
communicate with those creatures who are dark
aspected. The talent is limited to a range of 10’ (+

10’ / Rank), and the communication is equivalent
to a language skill of 1 (+ 1 for every 5 full Ranks).
The communication is a combination of spoken
and sign language. If either vision or sound is not
possible then the talent operates at half its actual
Rank (round down). Moreover, if neither of these
senses are available then the talent cannot function
at all.
Night Vision (T-2)
Range: 50 feet + 10 / Rank
Duration: Always active
Experience Multiple: 100
Effects: This talent allows the adept to see in the
dark with vision similar to that of a cat. Everything
will appear monochromatic (i.e. shades of grey)
and it is difficult to accurately estimate distance.
The higher the Rank, the less of a problem this will
be. Because the vision is monochromatic it cannot
be used to do a Detect Aura. Note that some
amount of light must be present before any sort of
vision is possible.
Detect Aura (T-3)
Range: Special
Duration: Immediate
Experience Multiple: 75
Base Chance: Perception + 5% / Rank
Resist: Active
Effects: The effects of this talent are described in
§9.1.

19.4 General Knowledge Spells
Blending (G-1)
Range: 15 feet + 1 / Rank
Duration: 1 hour + 1 / Rank
Experience Multiple: 50
Base Chance: 60%
Resist: None
Storage: Investment, Potion, Ward
Target: Entity
Effects: Once this spell is cast, the target must
remain still in order for it to have effect. While
remaining still the target is not able to be seen by
non-magical means (i.e. as for invisibility). If the
target moves, the spell ceases to work. However, if
the target becomes still again during the duration of
the spell, it will resume its effect. The duration of
the spell refers to the time since casting, not the
time that the spell is actually in effect (i.e. while
the target is still).
Keeping still will require (as a minimum) a 4 × WP
check every hour. The target may be required to
make additional willpower checks at the GM’s
discretion.
The spell only has effect while the target is “still”.
This means that the target is unable to move any of
its external surfaces, with limited exceptions. Moving an external surface encompasses such actions
as moving a limb, or opening and closing the
mouth. Blinking and normal bodily movement
caused by normal breathing will not constitute
moving for the purposes of this spell. The following actions will automatically cause the spell to
cease working: talking, spell casting, triggering
(subject to any revision of the Investment ritual)
readying a weapon, altering facing in a hex, moving from the hex, using a silent language, or indeed
any Action other than a Pass action (and Pass actions being restricted as outlined). Note that it is
not relevant if an observer would see the movement for the spell to cease working (e.g. moving
hands behind back, or talking behind hand will
both cause the spell to cease having an effect).
Light (G-2)
Range: 15 feet + 15 / Rank
Duration: 15 minutes × [D - 5] × [Rank]
Experience Multiple: 75
Base Chance: 50%
Resist: None
Storage: Investment, Ward
63

Target: Area
Effects: The Adept creates a volume in which nonmagical darkness is partially suppressed. The volume will be 1000 (+ 500 / Rank) cubic feet, and
may be in any one contiguous area the Adept desires, provided that no dimension is smaller than
one foot. The entire volume must be visible and
within range at time of casting, and may not be
moved. For visibility purposes, the Spell will increase Lighting levels within the volume to 60% +
2% / Rank. Rank 20 Light may not be seen
through. It will not aid in providing bonuses for
casting purposes, though it will neutralise penalties
due to natural darkness, to a maximum of 5% + 1%
/ Rank. The volume counts as direct light for Star
& Shadow Mages. If the lighting conditions are
higher than that provided by the spell, no effect
will be apparent. Note that because darkness is
being suppressed, no light is generated, so any area
outside the volume will not be lit. This spell can
engender silhouettes, though not create shadows. If
it is not possible to see into a lit volume, then objects within the volume are not visible. Any of this
volume may be overridden by a higher ranked
Spell of Darkness, or neutralised (back to original
conditions) by an equal rank.
Darkness (G-3)
Range: 15 feet + 15 / Rank
Duration: 15 minutes × [D - 5] × [Rank]
Experience Multiple: 75
Base Chance: 50%
Resist: None
Storage: Investment, Ward
Target: Area
Effects: The Adept creates a volume in which nonmagical light is partially suppressed. The volume
will be 1000 (+ 500 / Rank) cubic feet, and may be
in any one contiguous area the Adept desires, provided that no dimension is smaller than one foot.
The entire volume must be visible and within range
at time of casting, and may not be moved. For
visibility purposes, the Spell will increase Darkness levels within the volume to 60% + 2% / Rank.
Rank 20 Darkness may not be seen through. It will
not aid in providing bonuses for casting purposes,
though it will neutralise penalties due to natural
light, to a maximum of 5% + 1% / Rank. The volume counts as direct shadow for Star & Shadow
Mages. If the lighting conditions are lower than
that provided by the spell, no effect will be apparent. Note that because light is only being suppressed, it may still pass through, and no shadows
are generated outside the volume. If it is possible to
see through a Darkness, everything beyond it is
normally visible. This spell can engender silhouettes of lit objects against the darkness, though not
create shadows. Any of this volume may be overridden by a higher ranked Spell of Light, or neutralised (back to original conditions) by an equal
rank.
Shadow Form / Coruscade (G-4)
Range: 15 feet + 1 / Rank
Duration: 30 minutes + 30 / Rank
Experience Multiple: 150
Base Chance: 10%
Resist: None
Storage: Investment, Ward, Potion
Target: Entity
Effects: The target of this spell is enveloped in a
confusing pattern of either shadows (for Dark and
Shadow) or coruscating light (for Solar and Star),
which increases their defence versus physical
Melee or Ranged attacks by 2 (+ 2 / Rank). In
Close combat only 1 (+ 1 / Rank) is gained. No
form of magical vision will aid in avoiding the
defence bonus produced as a result of this spell, but
any attack made without using the sense of sight
(e.g. by a blind entity, a trample attack) will not be
affected. It is usually quite apparent when an entity
is under the effect of this spell.

19 COLLEGE OF CELESTIAL MAGICS
Wall of Starlight (G-5)
Range: 15 feet + 15 / Rank
Duration: 10 minutes + 10 / Rank
Experience Multiple: 150
Base Chance: 15%
Resist: Passive
Storage: Investment, Ward
Target: Area
Effects: Creates a 10’ high, 1’ thick, 20’ long wall
of light, or a 10’ high, l’ thick, 5’ internal radius
ring of light, or a 15’ high, 5’ diameter pillar of
light.
The adept may increase any dimension by 1’ /
Rank. The wall cannot be cast so as to include any
entity within it, other than the Adept. Any entity
who is dark aspected must passively resist or suffer
[D - 5] ( + 1 / Rank) damage each time they come
within contact with the wall (not per pulse). Any
entity damaged by the wall must roll on the fright
table. The wall created banishes darkness from
within its bounds in the same manner as a Spell of
Light of the same Rank. The entirety of one edge
must be affixed to a surface. This means that a wall
can be created with a smaller dimension than
would otherwise be possible. For example, casting
a wall of light on a stepping stone that is 3’ square
will result in a wall which is only 3’ long. Any
edge may be affixed but, for the purposes of this
spell, this does not include either face. For example, a wall could not be placed flat on a large open
surface. The surface or surfaces that the Wall of
Light is affixed to do not need to be flat, but the
length of the wall is measured from the deepest
depression on the surface that the wall fills. For
example, a circularly concave wall of 5’ radius,
with a rank 0 wall affixed to it will end in a flit
edge 15’ beyond the end of the curvature. The
entire anchoring edge must be visible to the Adept.
The wall itself cannot be moved. Should an entire
cross-section of the last remaining anchoring edge
be removed then the wall will immediately dissipate.
Example
An Adept casts a wall 1’ off the ground,
attached to a door. As soon as the door is opened the wall
is dissipated. If, however, the adept had cast it so that it
overlapped the door frame, and it projected slightly above
the top of the door, then it would not have been dissipated
because no entire cross-section of the anchoring edge has
been removed. In this case somebody with a sharp implement (and quite a bit of patience) could scratch away at
the stone wall until they had created a groove through the
entire cross-section of the Wall of Starlight in order to
make it dissipate.

Note that this spell will not be affected by a Spell
of Darkness except to reduce its lighting effect.
Solar and Star mages get a reduction to the Experience Multiple of 50 (to 100) and +5% to base
chance.
Wall of Darkness (G-6)
Range: 15 feet + 15 / Rank
Duration: 10 minutes + 10 / Rank
Experience Multiple: 150
Base Chance: 15%
Resist: Passive
Storage: Investment, Ward
Target: Area
Effects: This spell works similarly to the Wall of
Starlight Spell, except that light aspected creatures
are affected by it, and it banishes light in the same
manner as a Spell of Darkness of the same Rank.
Shadow and Dark mages get a reduction to the
Experience Multiple of 50 (to 100) and +5% to
base chance.
Witchsight (G-7)
Range: 15 feet + 15 / Rank
Duration: 30 minutes + 30 / Rank
Experience Multiple: 150
Base Chance: 15%
Resist: None
Storage: Investment, Ward, Potion
Target: Entity
Effects: The Adept may see objects or entities
which are invisible and they appear to have a slight

blue sheen around them. If the invisibility effect
(excluding Walking Unseen) is of a higher Rank
than the Witchsight, the object or entity may not be
clearly identified or directly magically targeted.
The Adept may also see in the dark as a Human
does on a cloudy day, with an effective range of
vision of 150 feet under the open sky, and 75 feet
elsewhere.
Walking Unseen (G-8)
Range: 1 foot + 1 / Rank
Duration: 1 hour + 1 / Rank
Experience Multiple: 100
Base Chance: 50%
Resist: None
Storage: Investment, Potion, Ward
Target: Entity
Effects: The target of this spell may move unnoticed, not invisible. This means that it will not
transmit light. As a consequence the target will cast
a shadow (which may or may not be noticed depending on the lighting etc — even if noticed may
not be connected to the target) and have a reflection in a mirror (or any reflective surface). However the target may not be noticed even if another
entity is looking directly at him/her. It should be
noted that a crystal of vision or similar would
count as looking directly at the target, not as a
reflection. An entity will get a perception check if
the target becomes invasive on that entity’s senses
(e.g. standing in a frontal adjacent hex, or standing
behind the entity with the target’s hands over
his/her eyes). Although the target is not invisible, it
may be detected using any magical means for
detecting invisible entities (e.g. witchsight).
If the target of the spell, or the target’s possessions,
are touched by another entity, or that entity’s possessions, then the spell is broken. The target of the
spell may not break it voluntarily (other than by,
for example, touching another entity). Once broken
the spell must be recast.
Resistance to Light (G-9 Solar)
Range: Self
Duration: 10 minutes + 10 / Rank
Experience Multiple: 200
Base Chance: 15%
Resist: None
Storage: Potion
Target: Self
Effects: While under the effects of this spell, an
adept gains 2% (+ 2 / Rank) to the chance of resisting magical, light-based attacks. This includes
Flash of Light, Wall of Starlight, Bolt of Starfire,
Web of Light, Solar Flare and Whitefire. The target
will also become fully protected from damage
caused by non-magical light (e.g. sunburn, snowblindness), with the exception that it will not protect Greater Undead from sunlight. In addition, it
allows vision in a Rank 20 Light Spell. Only Solar
Mages may learn this spell.
Illumination (G-9 Star)
Range: 15 feet
Duration: 10 minutes + 10 / Rank
Experience Multiple: 200
Base Chance: 15%
Resist: None
Storage: Investment, Ward
Target: Object or area
Effects: This spell causes a 1 inch circle on any
non-living surface to radiate light. The intensity of
light is determined by Rank: at Ranks 0–5 it is
merely a glow; at Ranks 6–10 it is equivalent to the
light of a candle; at Ranks 11–15 it is equivalent to
the light of a torch; and at 16–20 it is equivalent to
that of a lantern. Only Star Mages may learn this
spell. It will not aid in providing bonuses for casting purposes.
Charismatic Aura (G-9 Shadow)
Range: Self
Duration: 10 minutes + 10 / Rank
Experience Multiple: 200
Base Chance: 15%
Resist: None
64

Storage: Potion
Target: Self
Effects: This spell allows the adept to use shadows
advantageously to influence reaction rolls. The
adept can choose one of three different effects at
the time of casting. These are: to appear imposing
or threatening, to appear alluring or seductive, or to
appear helpless and in need of protection. When
used in appropriate circumstances these effects
modify reaction rolls by 5% (+ 1 / Rank). For
example, when dealing with an Orc Chief the first
of the effects would probably be most beneficial. It
is very difficult to perceive that the spell is in effect. Only Shadow Weavers may learn this spell.
Strength of Darkness (G-9 Dark)
Range: 15 feet
Duration: 10 minutes + 10 / Rank
Experience Multiple: 200
Base Chance: 15%
Resist: None
Storage: Investment, Ward, Potion
Target: Entity
Effects: The target’s Physical Strength is increased
by 1 (+ 1 for every 2 (or fraction) Ranks) for the
duration of the spell. This spell may only be cast
by Dark Mages when they are in an area of at least
60% Darkness.

19.5 General Knowledge Rituals
Reading the Night Sky (Q-1)
Experience Multiple: 150
Base Chance: MA + 4% / Rank
Cast Time: 1 hour
Effects: The Adept may read something of the
future by performing this ritual. The ritual may
only be performed from a vantage point with a
clear view of the sky (not indoors or in a hollow),
and it must be a clear night. The GM rolls for
success or failure. The GM provides the answers
writ in the stars in the form of generalised statements. If a successful roll occurs the statements
should be generally accurate. If a failure occurs
then nothing is read. If a backfire occurs then the
statements should be misleading.
Summoning and Binding Creatures of
Light / Darkness (Q-2)
Experience Multiple: 200
Base Chance: 20% + 4% / Rank
Cast Time: 1 hour
Effects: The adept may summon and bind 1 (+1 for
every 5 or fraction Ranks) non-sentient creature
whose aspect is the same as the aspect of the division of the college to which the Adept belongs.
Any creature summoned must be native to the area.
If the ritual succeeds, the creature will arrive bound
to the Adept. In this state the creature will try to
protect and aid the adept to the utmost of its ability
(but it does not automatically know what the adept
wishes it to do). If the ritual backfires then the
creature will arrive and immediately attack the
adept. The creature will arrive after (20 - D10 Rank) minutes (minimum of 0). Bound creatures
will continue to serve the Adept as long as passive
concentration is maintained (the Adept stays conscious and does not attempt any other spell that
requires concentration). If the Adept is stunned, a 3
× Willpower ( + 2 / Rank) Willpower check is
required to maintain concentration. If the concentration is broken the creature will immediately
attack the Adept. The Adept may at any time release any of the creatures, in which case the creature concerned will immediately flee from the
Adept’s presence.
Creatures that may be summoned using this ritual
are those appropriately aspected, non-sentient
beings from the following categories: 66.2 Felines,
66.4 Small Land Mammals, 67.1 Common Avians,
69.1 Lizards and Kindred (except Hydras), 69.2
Snakes, 69.3 Insects and Spiders, and 72 Creatures
of Night and Shadow. Note: Weres can only be
affected by this ritual while they are in their beast
form.

19 COLLEGE OF CELESTIAL MAGICS

19.6 Special Knowledge Spells
Healing (S-1)
Range: Touch
Duration: Immediate
Experience Multiple: 200
Base Chance: 40%
Resist: None
Storage: Investment, Potion
Target: Entity
Effects: Cures 1 point of Endurance or Fatigue ( +
1 / every 2 or fraction Ranks). This spell will not
cure specific Grievous Injuries and the extra endurance points associated with any Grievous Injury, nor will it “cure” tiredness fatigue (including
that lost due to spell casting). However, this spell
can cure as if the curing was attempted by a healer
of Rank equivalent to the Rank of the spell / 5
(round down).
Creating Light/Dark Sword (S-2)
Range: 15 feet + 5 / Rank
Duration: 5 minutes + 1 / Rank
Experience Multiple: 250
Base Chance: 30%
Resist: None
Storage: Investment
Target: Object
Effects: The Adept may cause any sword (as listed
on the weapons chart) within range to become a
sword of the Adept’s element (i.e. light for Solar or
Star Mages, and Dark for Shadow Weavers or
Dark Mages). The sword will then have its Strike
Chance increased by 1% (+ 1 / Rank) and its Damage increased by 1 (+ 1 / every 3 or fraction Ranks)
whenever it is used against a creature of the opposite element (i.e. dark aspected for a Light Sword
and light aspected for a Dark Sword). Light
Swords sparkle with small white sparks, and dark
swords appear blacker than black.
Bolt of Starfire (S-3)
Range: 40 feet + 15 / Rank
Duration: Immediate
Experience Multiple: 200
Base Chance: 35%
Resist: Active, Passive
Storage: Investment, Ward, Magical Trap
Target: Entity, object or volume
Effects: The Adept casts a bolt of starfire towards
any target within range. The target may be a volume of air. The first entity or object that the bolt
hits along its flight path suffers [D - 4] (+ 1 / Rank)
damage unless the target successfully resists. If
fired at figures in Close Combat the bolt will hit a
random target (based on their relative sizes).
Meteorite Shower (S-4)
Range: 60 feet + 15 / Rank
Duration: Delayed effect
Experience Multiple: 200
Base Chance: 10%
Resist: Passive
Storage: Investment, Ward, Magical Trap
Target: Area
Effects: The adept calls down from the heavens a
meteorite shower which peppers a given area that
was entirely within the Adept’s range at the time of
casting. The meteorite shower is targeted to hit a
specific hex and takes 2 minutes (-10 seconds /
Rank) to arrive (minimum of the end of the following pulse). Any entities within a vertical column
that is 25 feet in diameter (centred on the target
hex), with a height equal to the spell’s range) must
resist or suffer [D - 4] (+1 / Rank) damage. The
Adept may counterspell this spell at any time prior
to the meteorite shower arriving by casting the
appropriate counterspell at the targeted hex. The
targeted hex will have a magical aura until the
meteorite shower arrives (or is counterspelled by
the Adept). This spell will have no effect if it is
targeted on a hex which is already a target of this
spell. Note that a solid surface (such as 10’ of
earth) will prevent the meteorite shower from
reaching its target hex.

Star / Shadow Wings (S-5)
Range: 10 feet + 10 / Rank
Duration: 30 minutes + 30 / Rank
Experience Multiple: 250
Base Chance: 25%
Resist: None
Storage: Investment, Ward
Target: Sentient Entity
Effects: The target of this spell receives wings
comprising of the element of the Adept (i.e. Light
if a Solar or Star Mage, and Dark if a Shadow
Weaver or Dark Mage). These wings will carry the
target, and anything that the target can carry, at a
speed of 30 (+1 / Rank) miles per hour. (NB : 1
mph approximately 1.5’ / sec = 1.5 hexes / pulse.
During a takeoff or landing half that distance will
be travelled). The wings have a wingspan of 30’
and are insubstantial. If the wings come in contact
with an object they will cease to work until they
can once more spread unhindered (e.g. 30’ of open
ground is usually necessary in order to start using
them). Note that normal precipitation (i.e. rain,
mist, snow and hail) will not cause the wings to
cease functioning. The Wings will not become
invisible or unseen if the wearer does. Star Wings
are clearly visible at night and barely visible during
the day, and Shadow Wings are clearly visible
during the day and barely visible at night. During
the last 5 seconds of the duration of the wings they
will automatically try to land. There is no earlier
warning of the end of duration of the wings. Only
sentient creatures can control the wings.
Since shadow wings are made of shadow they are
clearly insubstantial and hence can be worn in
confined spaces. However, for them to be able to
be used, they require to be properly and fully extended, that is shaped, and will fly at full speed or
not at all. It requires 1 pulse to start and 1 pulse to
stop. They will only fly a humanoid character and
characters of human size. That is taken to mean
characters of 3 hexes or less. They will carry a
character plus the character’s normal encumbrance.
Web of Light / Darkness (S-6)
Range: 30 feet + 15 / Rank
Duration: Concentration (Maximum of 15 minute
+ 15 / Rank)
Experience Multiple: 250
Base Chance: 25%
Resist: Passive
Storage: Investment, Ward, Magical Trap
Target: Area
Effects: A five foot wide web of the element of the
Adept is projected from the finger tips to a target
hex, object or entity. Any entities not aspected to
the element of the web, and all objects, are ensnared by the web. The web may only ensnare a
number of entities equal to the Adepts rank, so it
will stop at the hex at which this limit is reached
(or at maximum range). Entities ensnared in the
web suffer [D - 2] (+1 / Rank) damage (halved if
they successfully resist) each pulse that they remain in the web after the first. The damage is done
at the end of each pulse. Any ensnared entity must
roll 1 × Physical Strength (2 × if they successfully
resisted) in order to move themselves to an adjacent hex (which may be free of the web), or to
perform an action within the web. A similar check
is required for any entity (regardless of aspect)
attempting to remove an object from the web. If an
entity receives aid in removing themselves from
the web, the PS of the aiding character may be
combined with their own. Five or more points
damage from a single blow from a B-class weapon
will destroy the entire web. Treat the web’s defence as being equal to its Rank. Entities of the
same aspect as the element of the web may ignore
its effects, but consequently may not affect the
web. They may aid other character in getting free
of the web. Any dropped object will become ensnared by the web, as will any entity not aspected
to the element of the web who comes into contact
with it (up to the limit of the web).

65

Fear (S-7)
Range: 15 feet + 15 / Rank
Duration: 15 seconds + 15 / Rank
Experience Multiple: 350
Base Chance: 20%
Resist: Active, Passive
Storage: Investment, Ward, Magical Trap
Target: Entity
Effects: The target of this spell is seized by uncontrollable fear and must roll on the Fright table. At
the time of casting, the Adept may choose to modify the Fright Table roll up or down by an amount
up to the rank of the spell. On a double or triple
effect this modification may be doubled or tripled
respectively.
Increased Gravity (S-8)
Range: 60 feet + 15 / Rank
Duration: Concentration: maximum 1 minute + 1 /
Rank
Experience Multiple: 450
Base Chance: 2%
Resist: Active, Passive
Storage: Investment
Target: Entity
Effects: The spell causes a target of the Adept’s
choice which is within range to suffer the effects of
an increase in gravity unless they successfully
resist. This increase in gravity subtracts 2 (+2 /
Rank) from the target’s strength and 1 for every 2
Ranks (or fraction) from the target’s TMR. The
target must roll under 3 × strength each pulse or
become immediately prone. Once prone, a roll of 1
× modified strength is required in order to stand
up. If the target’s strength is reduced to less than
zero, the target suffers the negative amount as
damage each pulse and must roll under Willpower
+ current Endurance in order to remain conscious.
If the target and Adept become separated by a
distance greater than the range of the spell then the
spell immediately ceases to work. If the target is
under (or comes under) the effect of a flying spell,
the following applies:
• If the rank of the flying spell is greater than the
rank of the Increased Gravity then the target may
be able to fly. However, twice the rank of the Increased Gravity is subtracted from the rank of the
flying spell for purposes of determining speed and
lift of the flying spell. This may make it a negative
modifier which may reduce the speed to 0 or less,
in which case the target may not fly but may stand.
• If the rank of the flying spell is less than or equal
to the rank of the Increasing Gravity, then the
target may not fly. However, half the rank of the
flying spell is subtracted from the rank of the Increased Gravity for purposes of determining the
strength and TMR reductions.
Whitefire (S-9)
Range: 30 feet + 15 / Rank
Duration: Immediate
Experience Multiple: 500
Base Chance: 1%
Resist: Active, Passive
Storage: Investment, Ward, Magical Trap
Target: Entity
Effects: The target of this spell must resist or be
instantaneously subjected to the heat of the interior
of a star, causing death. The target’s body is a
blackened husk, their skin needs to be regenerated
and their chance of resurrection is reduced by 2% /
Rank. If the target’s Willpower is greater than or
equal to the cast chance then the target will not be
affected. Protection from magical fire will not help
against this spell.
Solar Flare (S-10 Solar)
Range: 75 feet + 15 / Rank
Duration: Delayed effect
Experience Multiple: 500
Base Chance: 5%
Resist: Passive
Storage: Investment, Ward, Magical Trap
Target: Area

19 COLLEGE OF CELESTIAL MAGICS
Effects: The Adept calls down an incandescent
lance of sunlight which blasts an area of 1 hex (+ 2
hexes / 4 full ranks) in diameter. All non-sentient
flora is immediately withered and charred. Any
entities within the area must resist or suffer [D +
10] damage (+ 1 / 2 or fraction ranks). The flare
takes 60 seconds (-5 / Rank) to arrive (minimum of
5 seconds). The flare will always arrive at the end
of a pulse and during that pulse the area will be
brightly illuminated (that is 99% Light). This spell
may only be cast when the sun is in the sky. Only
Solar Mages may learn this spell.

the spell’s range, and is 5 feet wide. The Adept
may increase the width by one foot per rank. All
entities occupying hexes through which the fire
passes must resist or suffer [D - 2] ( + 1 per Rank)
damage. Entities who are damaged by this spell
have their base chance of infection increased by 20
(+ 2 / Rank). Any entities wholly within the area of
the fire must also roll under 1 × Willpower (2 ×
Willpower if they successfully resisted) or suffer
the effects of a roll on the fright table. Protection
from magical fires will not help against this spell.
Only Dark Mages may learn this spell.

Falling Star (S-10 Star)
Range: 75 feet + 15 / Rank
Duration: Delayed effect
Experience Multiple: 500
Base Chance: 1%
Resist: Passive
Storage: Investment, Ward, Magical Trap
Target: Area
Effects: The Adept calls from the sky a meteor
which crashes into a given area that was within the
Adept’s range at the time of casting. The meteor is
targeted to hit a specific hex and takes 5 minutes
(20 seconds / Rank) to arrive (minimum of 5 seconds). The meteor will always arrive at the end of
the pulse and will be preceded during that pulse by
a high-pitched whistle in the general area. Any
entities within the target hex suffer [D + 12] (+ 4 /
Rank) damage. Entities within adjacent hexes
suffer [D + 2] (+ 1 / Rank) damage. If an entity
successfully resists it suffers only half damage
(round up). The Adept may counterspell this spell
at any time prior to the meteor arriving by casting
the appropriate counterspell at the targeted hex.
The targeted hex will have a magical aura until the
meteor arrives (or is counterspelled by the Adept).
This spell will have no effect if it is targeted on a
hex which is already a target of this spell. Only
Star Mages may cast this spell.

Shadow Walking (S-10 Shadow)
Range: Self
Duration: Immediate
Experience Multiple: 550
Base Chance: 1%
Resist: None
Storage: Ward, Potion
Target: Self
Effects: The Adept may instantly teleport from
within one shadow to another shadow. The destination must be within sight or must have been carefully memorised beforehand. The destination may
be up to 5 miles (+ 1 / Rank) distant. Only Shadow
Weavers may cast this spell.

19.7 Special Knowledge Rituals
Conjuring and Controlling Light / Dark
Sphere (R-1)
Duration: Concentration: Maximum 5 minutes + 5
/ Rank
Experience Multiple: 450
Base Chance: 1% + 3% / Rank
Cast Time: 1 hour
Effects: The Adept may summon a 12” (+ 1” /
Rank) diameter sphere comprised of the element of
their division of the college. Solar and Star Mages
summon a Light Sphere which is as bright as the
sun and coruscates with sparks of light. Dark
Mages and Shadow Weavers create a Dark Sphere
which is inky black and seems to suck light into it.
If the ritual is successful the sphere will appear
within 15’ of the Adept under the Adept’s control.
Active concentration is required to move the
sphere, which may move at up to 6 (+ 1 / 4 full
Ranks) TMR. Once the Adept stops concentrating,
or the duration of 5 minutes (+ 5 / Rank) is
reached, the sphere will return to its own dimension. If the Adept fails to summon the sphere,
nothing happens. If a backfire results (the Cast

Blackfire (S-10 Dark)
Range: 30 feet + 5 / Rank
Duration: Immediate
Experience Multiple: 350
Base Chance: 5%
Resist: Passive
Storage: Investment, Ward, Magical Trap
Target: Area
Effects: From the Adept’s fingertips erupts a column of black flames which travel to the extent of

Check is more than 30 above the Cast Chance), the
sphere appears, but is not under control so will
move randomly about at maximum TMR. Anything that comes into contact with a Dark Sphere
must resist or be immediately sucked into oblivion.
Anything that comes into contact with a Light
Sphere must resist or be immediately reduced to a
pile of ashes. An entity that resists simply suffers
D10 damage and is thrown to the ground by the
shock of contact. It is believed that if a Light
Sphere comes in contact with a Dark Sphere a
cataclysmic explosion results (however there are
no known witnesses to such an event).

19.8 Light and Dark Aspect
Most creatures are either Light or Dark aspected,
depending on whether they are nocturnal or diurnal
(active during the day). There is no direct connection between the possession of a Light or Dark
aspect and the self-styled “Powers of Light and
Darkness”. Entities of Light aspect are not necessarily “good” nor entities of Dark aspect “evil”.
The aspect refers only to the Entity’s position with
regard to the Elements of Light and Dark. It is also
differs from an entity’s astrological Aspect (see
§1.4), but may be influenced by it.
To determine if an Entity is aspected with either of
the elements of Light or Dark, follow these rules:
• If the entity is a Celestial Mage then this determines their aspect: Solar and Star Mages are Light
aspected; Shadow and Dark Mages are Dark aspected.
• If the entity is not a Celestial Mage, and is Lunar
aspected, then they are Dark aspected. Note that
Shapechangers are Lunar aspected.
• If the entity is not a Celestial Mage, and is Solar
aspected, they are Light aspected.
• If none of the above applies, then an entity’s race
or type may mean that they are aspected with either
Light or Dark. Races or creatures that are nocturnal, crepuscular (active at twilight) or live predominantly underground are Dark aspected (Alusian examples: Dwarves, Orcs, most cats, bats,
wolves). Races and creatures that are diurnal, and
who do not live underground are Light aspected.
Golems, Elementals, and Beings of Mana or Spirit
(or their manifestations) are neither Light nor Dark
aspected.

19.9 Celestial Lighting Modifier Table
Light

Darkness Solar

Dark

Shadow

Star

Natural Lighting

Artificial Lighting

0%

100%

-

+25

-

-

Pitch Blackness

Magical Effect – no vision works

1%

99%

-25

+25

-10

+5

Midnight in a storm

Underground, no lights

5%

95%

-20

+20

0

+15

Overcast night

Single Candle Underground

10%

90%

-20

+15

+10

+25

New Moon, Moonless night

1 Torch Underground, Window less room in day

20%

80%

-15

+10

+20

+20

Night with crescent Moon + stars

1 Lantern Underground

30%

70%

-10

+5

+25

+15

Night in a town

Campfire at night, Shuttered room in day

40%

60%

-5

+5

+15

+10

Night of Full Moon

Torch-lit Underground

50%

50%

0

0

+10

+5

Twilight, Major Storm

Inside on overcast day

60%

40%

+5

-5

+5

0

Bright day in a forest, Solid rain

Lamp-lit Interior

70%

30%

+5

-10

0

-5

Overcast, Mist, Light rain

Brightly lit Underground

80%

20%

+10

-15

-5

-10

Autumn Morning, Light cloud

Brightly lit Interior

90%

10%

+15

-20

-10

-15

Bright sunny afternoon

–

95%

5%

+20

-20

-15

-20

Noon

–

99%

1%

+25

-25

-20

-25

Noon in a desert

–

100%

0%

+25

-

-

-

–

Magical Effect – no vision works

- Adept cannot cast under these conditions.

66

20 COLLEGE OF EARTH MAGICS

20 The College of Earth Magics (Ver 1.2)
The College of Earth Magics is concerned with the
shaping of the powers of the earth itself and of
those entities and things that are rooted in the earth
or in contact with it.
There are two distinct branches of this College, and
a practitioner of Earth Magic college can be a
member of only one branch at a time. They may
alter their affiliation, but are treated as if they had
quit their original branch and lost all knowledge of
Earth Magic until such time as they has spent six
months in study and meditation to relearn the General Knowledge of the College in a new form. The
two divisions of this College are:
Pacifistic Earth Magic Usually practiced by wild
people who live in isolation in the wilderness, this
type of Earth Magic is very powerful, but entirely
defensive in nature. An adherent of this division of
Earth Magic will never attack without provocation
(i.e. unless attacked themselves or unless they see
animals or plants attacked). They are unaffected by
counterspells of their College cast over an area
they which occupy while attempting to work the
magic of their College. In addition, the counterspells of their College give only half the normal
benefit to characters attempting to resist their
magic. However, they may never participate in
rituals of this College which involve human sacrifice.
Druidic Earth Magic This form of Earth Magic is
practiced by strong-willed individuals who have no
objection to the taking of human life to further
their magic. It is often practiced communally since
it is in part ritual magic. Many of the rituals involve courting the darker sides of earth by providing blood to quench its thirst.

20.1 Restrictions
Practitioners of the College of Earth Magics must
always be in contact with the earth to perform
magic of this College.
A practitioner of this College is considered in
contact with the earth if they are in contact with an
item which is, itself, rooted in the earth (tree, plant,
foundation of a house, etc.). They would not be in
contact with earth if they were in the air or in water
where they could not touch bottom, or if they were
atop a piece of furniture or an animal as these are
not rooted in the ground. An Adept in a person’s
arms is not in contact with the earth.
The MA requirement of this College is 15.

20.2 Base Chance Modifiers
The following numbers are added to the Base
Chance of performing any talent, spell or ritual of
the College of Earth Magics.
Adept is wearing a sprig of fresh mistletoe
+5
Adept is in contact with earth, but beneath
-20
water†
Adept is in a manmade dwelling without an
-5
earthen floor
Adept occupies a place of power‡
+20
†For example, beneath the ocean, if the Adept can
find a way of breathing in such a situation, the
Adept’s BC would always be reduced by 20.
‡These can be any places frequented by worshippers of earth. Examples from mythology and literature might include: Stonehenge, Finn McCool’s
Seat, The Hill of Tara, etc. If the Place of Power is
used for ritual sacrifice, the practitioners of Pacifistic Earth Magic would receive no benefit.
All modifiers are cumulative. They are in addition
to the modifiers listed in §7.4.

20.3 Talents
Detect Aura (T-1)
Range: Special
Experience Multiple: 75
Base Chance: PC + 5% / Rank
Resist: Active

Target: Entity, Object, Area, Volume
Effects: The Base Chance is reduced by 1% for
every foot after the first five feet the target is from
the Adept. The results of this talent are the same as
given in §9.1.

20.4 General Knowledge Spells
Converse with Animals (G-1)
Range: 10 feet + 10 / Rank
Duration: 3 hours + 3 hours / Rank
Experience Multiple: 50
Base Chance: 45%
Resist: None
Storage: Potion
Target: Self
Effects: The Adept may communicate with fauna
(whether verbally or symbolically, and to what
extent, are left up to the GM’s discretion). Physical
contact between the animal and the Adept increases the Base Chance of successfully casting
this spell by 5. This spell does not include avians
or fish and has to be recast for each type of animal,
for example talking to wolves does not allow the
Adept to talk to tigers.
Converse with Plants (G-2)
Range: 10 feet + 10 / Rank
Duration: 3 hours + 3 / Rank
Experience Multiple: 50
Base Chance: 25%
Resist: None
Storage: Potion
Target: Self
Effects: The Adept can communicate with flora
with which they are familiar. The mode and extent
of communication is up to the GM’s discretion.
Controlling Animals (G-3)
Range: 10 feet + 10 / Rank
Duration: Concentration: no maximum
Experience Multiple: 100
Base Chance: 20%
Resist: Passive
Storage: Investment
Target: Animal
Effects: The Adept controls the actions of any
animal that does not successfully resist. It will
serve them so long as they continue to concentrate.
If they release the animal or their concentration is
broken, it may attack them or flee. The chance to
cast this spell is reduced by 5% if the Adept cannot
speak to the animal. If the Adept cannot make eye
contact, the Base Chance is reduced another 5%.
Blending (G-4)
Range: 15 feet + 1 / Rank
Duration: 1 hour + 1 / Rank
Experience Multiple: 50
Base Chance: 60%
Resist: None
Storage: Investment, Potion, Ward
Target: Entity
Effects: Once this spell is cast, the target must
remain still in order for it to have effect. While
remaining still, the target is not able to be seen by
non-magical means (i.e. as for invisibility). If the
target moves, the spell ceases to work. However, if
the target becomes still again during the duration of
the spell, it will resume its effect. The duration of
the spell refers to the time since casting, not the
time that the spell is actually in effect (i.e. while
the target is still).
Keeping still will require (as a minimum) a 4 × WP
check every hour. The target may be required to
make additional willpower checks at the GM’s
discretion.
The spell only has effect while the target is “still”.
This means that the target is unable to move any of
its external surfaces, with limited exceptions. Moving an external surface encompasses such actions
as moving a limb, or opening and closing the
mouth. Blinking and normal bodily movement
67

caused by normal breathing will not constitute
moving for the purposes of this spell. The following actions will automatically cause the spell to
cease working: talking, spellcasting, triggering
(subject to any revision of the Investment ritual),
readying a weapon, altering facing in a hex, moving from the hex, using a silent language, or indeed
any Action other than a Pass action (and Pass actions being restricted as outlined). Note that it is
not relevant if an observer would see the movement for the spell to cease working (e.g. moving
hands behind back, or talking behind hand will
both cause the spell to cease having an effect).
Walking Unseen (G-5)
Range: 1 foot + 1 / Rank
Duration: 1 hour + 1 / Rank
Experience Multiple: 100
Base Chance: 50%
Resist: None
Storage: Investment, Potion, Ward
Target: Entity
Effects: The target of this spell may move unnoticed, not invisible. This means that it will not
transmit light. As a consequence the target will cast
a shadow (which may or may not be noticed depending on the lighting etc — even if noticed may
not be connected to the target) and have a reflection in a mirror (or any reflective surface). However the target may not be noticed even if another
entity is looking directly at him/her. It should be
noted that a crystal of vision or similar would
count as looking directly at the target, not as a
reflection. An entity will get a perception check if
the target becomes invasive on that entity’s senses
(e.g. standing in a frontal adjacent hex, or standing
behind the entity with the target’s hands over
his/her eyes). Although the target is not invisible, it
may be detected using any magical means for
detecting invisible entities (e.g. witchsight).
If the target of the spell be touched by another
entity, or that entity’s possessions, then the spell is
broken. The target of the spell may not break it
voluntarily (other than by, for example, touching
another entity). Once broken the spell must be
recast.
Healing (G-6)
Range: Touch
Duration: Immediate
Experience Multiple: 100
Base Chance: 40%
Resist: None
Storage: None
Target: Entity
Effects: Through a combination of magic and the
application of healing herbs and salves, the Adept
can cure 3 ( + 1 / Rank) Damage Points suffered
due to disease or injury. The spell takes about 5
minutes to cast.
Detecting Traps and Snares (G-7)
Range: 20 feet + 5 / Rank
Duration: 1 hour + 1 / Rank
Experience Multiple: 75
Base Chance: 60%
Resist: None
Storage: Potion
Target: Self
Effects: This spell increases the Base Chance of
detecting traps or ambushes while outdoors by 10
(+ 1 / Rank).
Detecting Poisons (G-8)
Range: Touch
Duration: 5 minutes + 5 / Rank
Experience Multiple: 75
Base Chance: 50%
Resist: None
Storage: Investment (in valid target only)
Target: Object
Effects: The spell imbues a piece (usually a wand)
of ash wood, ivory, or unicorn horn with the ability

20 COLLEGE OF EARTH MAGICS
to detect poison. When the wand is touched to an
object or substance for 5 seconds, the wand will
momentarily turn black if poison is present. The
wand may be used to check up to 1 (+ 1 / Rank)
objects or substances. By the shading and patterns
of the wand, further information may be determined about the poison based on rank:
Rank 8+
Rank
10+
Rank
16+
Rank
20+

Type (Alchemical or Natural)
Origin (Mineral, Plant, Animal, Magic)
Effects (Damage, Sleep, Unconsciousness, etc.)
Potency (Effective rank or damage of
poison)

Lesser Enchantment (G-9)
Range: 10 feet + 10 / Rank
Duration: Special
Experience Multiple: 125
Base Chance: 20%
Resist: Active, Passive
Storage: Investment, Ward, Magical Trap
Target: Sentient Entity
Effects: The target of this spell is either blessed or
cursed (Adept’s choice). The spell increases either
the character’s luck or misfortune (depending on
whether it operates as a blessing or as a curse) by 1
on every percentile dice roll in which the character
is directly involved. This spell cannot be cast over
oneself. The duration of the enchantment is as
follows: Ranks 1–10 = a fortnight, Ranks 11–19 =
3 months, Rank 20 permanent until dispelled.
Herbal Lore (G-10)
Range: Self
Duration: 1 hour + 1 / Rank
Experience Multiple: 75
Base Chance: 25%
Resist: None
Storage: Potion
Target: Self
Effects: This gives the Adept Rank 0 Herbalist for
the duration of the spell. It also increases their
chance of finding herbs (as per the Herbalist or
Ranger skills) by 2% (+ 2 / Rank). If the Adept is
already ranked in Herbalist or Ranger, then they
gain an additional +10%.
Tracking (G-11)
Range: Self
Duration: 1 day + 1 / Rank
Experience Multiple: 100
Base Chance: 15%
Resist: None
Storage: Potion
Target: Self
Effects: The Adept adds 10% (+ 2 / Rank) to their
chance of Tracking while outdoors.

20.5 General Knowledge Rituals
Summoning Animals (Q-1)
Duration: Immediate
Experience Multiple: 150
Base Chance: MA + 3% / Rank
Resist: None
Target: Animals
Cast Time: 1 hour
Material: None
Actions: Concentration
Concentration Check: Standard
Effects: If the Ritual Check is successful then a
number of small animals equal to the Adept’s Rank
(minimum 1) are summoned. The animal the Adept
attempts to summon must be a native of the area. If
the Adept chooses to vocalise their summons in the
form of a loud shout or call the Base Chance is
increased by 25%.

20.6 Special Knowledge Spells
Earth Hammer (S-1)
Range: 25 feet + 10 / Rank
Duration: Immediate
Experience Multiple: 200
Base Chance: 40%
Resist: Passive

Storage: Investment, Ward, Magical Trap
Target: Entity
Effects: The Adept conjures a giant maul of stone
which hurls itself at a target of the Adept’s choosing. If the target fails to resist the spell, it takes [D 5] (+ 1 / Rank) damage.
Hands of Earth (S-2)
Range: 20 feet + 5 / Rank
Duration: 1 hour + 1 / Rank
Experience Multiple: 150
Base Chance: 25%
Resist: Passive
Storage: Investment, Ward, Magical Trap
Target: Entity, Area
Effects: The spell causes 1 + (Rank / 2) hands of
stone to materialise out of the ground within range.
Each hand is 7 feet tall and all must be contiguous
to each other. They may clutch anything that enters
the hex they occupy and will continue to do so
until they have caught something. Any entity
standing on a spot where a hand materialises will
be snatched up if they fail to resist, otherwise the
hand appears next to them within the hex. Any
entity caught will still be able to wield a weapon or
cast if they roll under 2 × MD. Entities (or combination of entities) with a combined PS + target’s
AG greater than 40 (+ 2 / Rank) may attempt a 1 ×
PS to escape from the Hands, and may do so every
pulse after the first. Any hand which has been
broken out of will either dissipate or become hard
cover (GM discretion).
Strength of Stone (S-3)
Range: 10 feet
Duration: 1 hour + 1 / Rank
Experience Multiple: 200
Base Chance: 20%
Resist: Passive
Storage: Investment, Potion, Magical Trap
Target: Entity
Effects: The target of this spell has their Physical
Strength or Endurance (Adept’s choice) increased
by Rank points (minimum 1).
Armour of Earth (S-4)
Range: 10 feet
Duration: 30 minutes + 30 / Rank
Experience Multiple: 200
Base Chance: 20%
Resist: None
Storage: Investment, Potion
Target: Entity
Effects: This spell increases the defence from
physical weapons of the target by 2 + 2 / Rank. At
Rank 11 and above, it also absorbs 1 Damage Point
per strike.
Diamond Weapon (S-5)
Range: 5 feet + 5 / Rank
Duration: 5 minutes + 1 / Rank
Experience Multiple: 250
Base Chance: 30%
Resist: None
Storage: Investment
Target: Object
Effects: This spell is cast over a weapon of the
Adept’s choice, increasing its strike chance by
Rank% (minimum of 1). The chance of an attacker’s weapon breaking increases by 5% when
striking a weapon under the effect of this spell (i.e.
break chance becomes 94–99, instead of 99).
Gem Creation (S-6)
Range: 10 feet
Duration: 1 day + 1 / Rank
Experience Multiple: 250
Base Chance: 10%
Resist: None
Storage: Investment
Target: Area
Effects: This spell creates one gemstone of random
value for each 5 (or fraction) Ranks. The gems
dissipate at the end of the spell.
Animal Growth (S-7)
Range: 10 feet + 10 / Rank
68

Duration: 1 day + 1 / Rank
Experience Multiple: 225
Base Chance: 15%
Resist: Passive
Storage: Investment, Ward
Target: Non-sentient mammal
Effects: One mammal of the Adept’s choice is
doubled in size. The effects of this radical change
are determined by the GM.
Enchanting Plants (S-8)
Range: 10 feet + 10 / Rank
Duration: 1 day + 1 / Rank
Experience Multiple: 225
Base Chance: 15%
Resist: None
Storage: Investment, Ward
Target: Plant
Effects: This spell may impart partial mobility to a
number of plants (including trees) equal to the
Adept’s Rank. The plants may not uproot themselves, but may move their branches and leaves
while remaining in the same spot. The plant’s
actions are always under the control of the Adept
so long as they maintain their concentration. If
their concentration is broken, voluntarily or otherwise, the plants will be controlled by the GM until
the Adept reestablishes control and could, conceivably attack the Adept.
Binding Animals (S-9)
Range: 10 feet + 10 / Rank
Duration: Until dispelled
Experience Multiple: 250
Base Chance: 10%
Resist: Passive
Storage: Investment, Ward
Target: Animal
Effects: This spell is similar to the Spell of Controlling Animals except that the Adept does not
have to concentrate on it to maintain it. The spell
will last until broken by the animal or dispelled by
magic. Any animal subject to this spell makes one
check per week against its Willpower to determine
if the spell is broken. This only works on land
animals, it will not work on avians or aquatics.
Conjuring Earth Elemental (S-10)
Range: 20 feet
Duration: Immediate
Experience Multiple: 225
Base Chance: 25%
Resist: None
Storage: Investment, Ward
Target: Earth Elemental
Effects: If the cast is successful, an Earth Elemental with a combined Fatigue and Endurance of 15
(+ 5 / Rank) appears within 20 feet of the Adept.
The Adept automatically casts a Spell of Controlling Earth Elemental to see if they control the
Elemental. The Control spell is a separate spell and
requires additional fatigue to cast, but does not
require any preparation and is cast in conjunction
with this spell.
Controlling Earth Elemental (S-11)
Range: 20 feet
Duration: Concentration: no maximum
Experience Multiple: 225
Base Chance: 20%
Resist: None
Storage: Investment
Target: Earth Elemental
Effects: The Adept may attempt to control an Elemental they have just summoned. The Elemental
does not get to resist. If successful in establishing
control over the Elemental, the Adept maintains
control until their concentration is broken or they
banish the Elemental with a counterspell. If they
fail to gain control of the Elemental or gain control,
but have their concentration broken, the Elemental
will immediately attack them. The Adept cannot
banish an Elemental they do not control.
Sinking Doom (S-12)
Range: 30 feet + 10 / Rank
Duration: Immediate

20 COLLEGE OF EARTH MAGICS
Experience Multiple: 650
Base Chance: 2%
Resist: Active, Passive
Storage: Investment, Ward, Magical Trap
Target: Entity
Effects: This spell opens a circular pit under a
single human sized target per 5 (or fraction) Ranks
and sucks the unfortunate standing over it down to
be encased in rock 5 (+ 5 / Rank) feet underground.
This spell may be used to affect multi-hex targets,
each hex of the target counts as one human-sized
target. If the spell does not affect every hex which
the target occupies then the target will not be affected.
Wall of Stone (S-13)
Range: 20 feet + 10 / Rank
Duration: 10 minutes + 10 / Rank
Experience Multiple: 150
Base Chance: 10%
Resist: None
Storage: Investment, Ward
Target: Area
Effects: Creates a 10 foot high × 20 foot long wall
of granite or a 10 foot high ring of stone with a 20
foot radius or a pillar of stone 15 feet high and with
a 2 foot radius. The Adept increases any dimension
by 1 foot per Rank. They may not attempt to create
a wall on top of an entity.
Wall of Iron (S-14)
Range: 20 feet + 10 / Rank
Duration: 10 minutes + 10 / Rank
Experience Multiple: 250
Base Chance: 5%
Resist: None
Storage: Investment, Ward
Target: Area
Effects: Same as for S-13 (Wall of Stone Spell)
except that the Adept creates a wall of cold iron.
Tunnelling (S-15)
Range: 5 feet + 1 / Rank
Duration: 30 seconds + 5 / Rank
Experience Multiple: 200
Base Chance: 10%
Resist: None
Storage: Investment, Ward
Target: Area
Effects: The Adept creates a circular opening or
tunnel 10 feet in diameter and 20 feet deep in a
wall, ceiling, floor, ground surface, cliff face, etc.

The Adept may add 1 foot to either depth or radius
per Rank.

3 / Rank. Each javelin does [D - 4] (+ 1 / 2 (or
fraction) Ranks) damage.

Trollskin (S-16)
Range: 10 feet
Duration: 30 seconds + 5 / Rank
Experience Multiple: 250
Base Chance: 20%
Resist: None
Storage: Investment, Potion, Ward
Target: Entity
Effects: The spell allows the subject to regenerate
Endurance Points. The target begins to regenerate
the pulse after the spell is cast and continues to
regenerate at the rate of 1 Endurance Point per
pulse for the duration of the spell. The spell will
not help regenerate wounds inflicted by acid or
fire. An entity will not die from damage while
under the effects of a trollskin, even if they are
below negative half Endurance.

Earth Transformation (S-19)
Range: 10 feet + 10 / Rank
Duration: 3 hours + 1 / Rank
Experience Multiple: 400
Base Chance: 10%
Resist: Passive
Storage: Investment, Ward
Target: Volume
Effects: This spell turns 3 (+ 1 / Rank) cubic feet of
stone into mud or vice versa. An entity standing on
a spot containing a mud puddle that is turned to
stone has an opportunity to passively resist the
effects of the spell. If they fail to resist, they will
become trapped in the stone that has taken the
place of the mud they were standing in. If they
resist, the mud turns to stone, but they are not
trapped.

Smoking Magma (S-17)
Range: 25 feet + 5 / Rank
Duration: 10 seconds + 10 / Rank
Experience Multiple: 300
Base Chance: 7%
Resist: Passive
Storage: Investment, Ward
Target: Area
Effects: The Adept creates a pool of molten rock
which wells up from underground. The pool has a
radius of 5 feet (+1 / Rank). Any entity within the
area covered by the pool suffers damage of [D - 5]
(+ 1 / Rank). If the target successfully resists, this
damage is halved. Note that while the entity is
within the area of effect the damage is applied
every pulse.

20.7 Special Knowledge Rituals

Diamond Javelins (S-18)
Range: 30 feet + 10 / Rank
Duration: Immediate
Experience Multiple: 300
Base Chance: 20%
Resist: None
Storage: Investment, Ward
Target: Entity
Effects: The spell causes diamond-tipped javelins
to fly from the earth at the Adept’s feet and travel
toward a target(s) of the Adept’s choice. The number of javelins which appear is 1 (+ 1 / 2 (or fraction) Ranks). The javelins have a Base Chance to
hit equal to the Base Chance of ordinary javelins +

69

Binding Earth (R-1)
Range: 10 feet + 10 / Rank
Duration: Concentration: maximum 1 hour + 1 /
Rank
Experience Multiple: 500
Base Chance: 10%
Resist: None
Target: Earth
Cast Time: 1 hour
Material: A human (or humanoid)
Actions: Sacrifice human
Concentration Check: Standard
Effects: If the is ritual successful, the Adept gains
complete control over a 500 pound weight of earth
and stone (plus an additional 500 pound weight per
Rank). They can shape or move the earth, change
its consistency or instill intelligence in it as they
choose. The Base Chance to successfully employ
this ritual is 10%. It can affect any earth or stone
within 10 feet (+ 10 / Rank) of the Adept. If a ritual
sacrifice of a human (or humanoid) being is performed at the end of the duration, the affected earth
becomes permanently bound (that is, it contains no
life and blocks all earth shaping effects, for example, Hands of Earth, Earth Elementals, Tunnelling).

70

21 COLLEGE OF FIRE MAGICS

21 The College of Fire Magics (Ver 2.0)
The College of Fire Magics is concerned exclusively with manipulating the element of fire. Adepts of this College are referred to as Fire Mages or
Pyromancers.

Within range low level heat sources (such as living
bodies) can be seen. At double range medium level
sources (such as camp fires) are visible and high
level heat sources may be visible at any range.

The affinity fire mages have for their element has
often led to them being considered to be pyromaniacs. Most fire mages, however, have a healthy
respect for fire, and rarely will they wantonly set
people or places alight. They do enjoy being in or
near fire and are generally the first to suggest a
nice camp fire when adventuring. Fire mages are
also thought to be hot-tempered, to enjoy hot and
spicy food, and to dislike getting wet. They frequently have red hair. Generally fire mages are
flamboyant characters, particularly in their spellcasting. The College of Fire Magics is anything but
subtle, and its spells involve more than the usual
amounts of shouting and arm waving.

This talent can sometimes penetrate where normal
vision cannot. At half normal range the Adept can
see heat sources which are warmer than the obscurement, eg. living entities through mist or light
bushes. It may also possible to detect the residual
heat of a source which has recently been moved or
extinguished.

The element of fire is unquestionably destructive,
and as a consequence the majority of the spells of
this college cause considerable amount of damage,
if only as a side-effect. The Fire College is not
noted for being harmless, and so fire mages are
even less popular than most mages among the
common folk. When it is revealed that a person is
an Adept of the Fire College the usual reaction will
generally be one of fear and trepidation. Fire
mages are quite amicable towards the Adepts of
other colleges with the notable exception of those
of the College of Water Magics.

2. Identify the general type of heat source. Only
those heat sources that the Adept has previously
encountered may be identified with this talent. At
higher ranks familiar individuals may be identified.

Unless otherwise specified, Magical Fire will
automatically ignite flammables.

Pyrogenesis (T-2)
Range: Sight
Duration: Immediate
Experience Multiple: 75
Base Chance: MA + 5 / Rank - 1 / 5 feet separating
the Adept from the target (after the first 5 feet)
Resist: Passive
Effects: The Adept may cause to burst into flame a
single mass of dry, flammable material weighing
up to 1oz (+ 1/

Traditional colours
Fire mages generally prefer to wear red, the colour
of their element, often trimmed with black, yellow
or orange. For jewellery the fire mage most frequently chooses garnets or rubies, set in gold.
Traditional symbols
The symbol of the Fire College is the flame, usually depicted in red or gold.

21.1 Restrictions
Adepts of the College of Fire Magics may only
practice their arts in an area where it is possible for
fire to exist. They may not practice fire magic
underwater or in a vacuum, for example.
The MA requirement for this college is 12.

21.2 Base Chance Modifiers
The Base Chance of performing any talent, spell or
ritual of the College of Fire Magics is modified by
addition of the following numbers:
Adept is in light mist, fog, or rain, or is
-5
soaking wet
Adept is in heavy fog, or rain, or is par-10
tially immersed in water
Adept is almost totally immersed in water
-15
Adept is within 30 feet of large campfire
+5
Adept is within 30 feet of bonfire
+10
Adept is in contact with small campfire
+5
Adept is in contact with large campfire
+10
Adept is in contact with bonfire
+15
Adept is in a hot, dry region (e.g. desert)
+5
Only one modifier from each group may be applied
together. These bonuses only apply to normal fire.
Magical fire may only be used to reduce negative
modifiers to zero.

21.3 Talents
Infravision (T-1)
Range: 50 feet + 5 / Rank
Experience Multiple: 75
Resist: None
Effects: The Adept is able to see heat sources as if
they emit normal light and target them (e.g. with
weapons or spells) in the absence of visible light. It
works best in relative darkness, since it is easily
over-powered by visible light.

The Adept has a base chance of PC (+ 5 / Rank) - 1
/ 10 feet between the Adept and the heat source
(after the first 10) of gaining additional information
about a particular heat source. They may:
1. Determine the relative temperature of the heat
source.

3. Determine if the heat radiated by source is being
generated by magic, e.g. the Adept is able to distinguish between normal and magical fire. (Note
only Fire College infravision has this ability).
The Adept may re-attempt any of these abilities
after a period of 41 pulses (2 / Rank) has elapsed or
if the Adept is 20 feet closer to the target.

5 ranks). Once alight it will burn normally and may
be extinguished by either mundane or magical
means. Flammable materials are defined as wood,
paper, cloth etc, but not flesh, except that Pyrogenesis may be used to cremate insects and small
creatures within the maximum weight restrictions.
This talent cannot be used on possessions. Utilising
this talent requires a pass action, and is obvious to
observers.

21.4 General Knowledge Spells
Bolt of Fire (G-1)
Range: 25 feet + 25 / Rank
Duration: Immediate
Experience Multiple: 200
Base Chance: 40%
Resist: Passive
Storage: Ward, Investment, Magical Trap
Target: Entity, Object, Area
Effects: The Adept causes a bolt of Fire to streak
from their hand towards anywhere in range. The
first entity or object the bolt hits in its path must
resist or suffer D (+ 1 / Rank) damage. If the bolt
does not hit anything it will dissipate at the end of
its range. At Rank 20 the Adept may delay releasing the bolt for a pulse if they should choose.
Extinguish Fires (G-2)
Range: 15 feet + 15 / Rank
Duration: Immediate
Experience Multiple: 100
Base Chance: 50%
Resist: None
Storage: Investment, Ward, Magical Trap
Target: Volume
Effects: When successfully cast, this spell will
extinguish all fire in a 10 foot (+ 10 / Rank) radius
sphere. All the volume affected must be within
range of the spell. If the range is doubled or tripled
the volume may likewise be increased.
71

Fire Armour (G-3)
Range: Touch
Duration: 1 hour + 1 / Rank
Experience Multiple: 200
Base Chance: 25%
Resist: None
Storage: Potion
Target: Object, Entity
Effects: This spell protects the target against damage by fire for 4 (+ 4 / Rank) points of protection.
Protection is ablative and when the damage the
spell may absorb is exceeded, the spell is dissipated with any excess damage applied to the target.
Double and triple effects may apply to duration or
degree of protection. At Rank 20 the spell confers
100 points of protection.
Firelight (G-4)
Range: Touch
Duration: 30 minutes + 30 / Rank
Experience Multiple: 75
Base Chance: 50%
Resist: Passive
Storage: Investment
Target: Object, Point
Effects: The Adept creates a source of light emanating in all directions from an object or point
touched by them. At ranks 0 to 5 the light emitted
is equivalent to that of a torch, at ranks 6 to 10 that
of a small campfire, at ranks 11 to 15 a large campfire, and at ranks 16 to 20 a bonfire. The light
emitted will have the appearance of firelight of the
appropriate strength. It is magical light, and will
cast poor contrast shadows.
Fireproofing (G-5)
Range: Touch
Duration: 1 hour + 1 / Rank
Experience Multiple: 150
Base Chance: 30%
Resist: None
Storage: Potion, Investment
Target: Entity, Object
Effects: The spell protects the target from all nonmagical fire and heat effects up to the heat equivalent of a bonfire. An entity or object is also protected against smoke effects (including smoke
inhalation), heatstroke and sunstroke.
Increase Temperature (G-6)
Range: Touch
Duration: Special
Experience Multiple: 100
Base Chance: 35%
Resist: Passive
Storage: Investment
Target: Volume, Object
Effects: The Adept must remain in contact with the
target for heating to occur. When the spell ends the
target will cool normally.
Gases
Duration: 1 hour + 1 / Rank
The Adept may cause the temperature of a closed
contiguous volume of gas (such as air) to increase.
The volume affected is equal to 125 (+ 125 / Rank)
cubic feet. The temperature can be increased by up
to 2(C per Rank. At Rank 20 the volume that can
be affected by Increase Temperature is doubled
Solids
Duration: 5 minutes + 5 / Rank
The Adept may heat 1 lb (+ 1 / Rank) of solid
material up to 50 (+ 50 / Rank) degrees C. The
Adept may choose any combination of temperature
increase increment and mass increment to a total of
the Adept’s rank. The temperature increase takes
10 seconds / 100 degrees / pound mass. At Rank 20
any metal item within the weight limit of the Adept
can be reduced to molten slag in as little as 10
seconds (resistance roll is applicable).
Liquids
Duration: 5 minutes + 5 / Rank

21 COLLEGE OF FIRE MAGICS
The Adept may heat 1 (+ 1 / Rank) pints of liquid
up to 20 (+ 20 / Rank) degrees C. They may
choose any combination of temperature increase
increment and mass increment to a total of the
Adept’s rank. The temperature increase takes 10
seconds / 10 degrees / pint. The volume of material
that may be affected is increased by 6 times at
Rank 20.
Slow Fire (G-7)
Range: 5 feet + 5 / Rank
Duration: 1 hour + 1 / Rank
Experience Multiple: 100
Base Chance: 20%
Resist: None
Storage: Investment
Target: Object
Effects: This spell increases the duration that a
wood, lamp or oil fire will burn for by 1 (+ 1 /
Rank) hours. Light remains constant while the spell
is in effect but heat from the fire source is halved.
This spell may be used to slow the effects of small
fires or to ensure a fire lasts for a considerable
period of time.
Smoke Creation (G-8)
Range: 10 feet + 10 / Rank
Duration: 30 minutes + 30 / Rank
Experience Multiple: 75
Base Chance: 25%
Resist: None
Storage: Investment, Ward
Target: Volume
Effects: The Adept may create a volume of smoke
equal to 1000 cubic feet (+ 500 / Rank). This volume must be closed and contiguous, and has a
minimum thickness of 5 feet. The density of the
smoke may vary from light vapours at Rank 0 to
thick, roiling smoke at Rank 20. The Adept may
choose the density of the smoke when casting. The
smoke produced reduces visual perception by 1
multiplier per 5 full ranks (but may not reduce it to
less than once times perception). The effective
Rank of the spell will be decreased by 1 Rank per
10 miles/hour of wind (minimum Rank 0) in the
target volume. Below rank 10 the smoke created by
this spell is completely transparent to infravision.

21.5 General Knowledge Rituals
Binding Fire (Q-1)
Range: 10 feet + 10 / Rank
Duration: 1 hour + 1 / Rank
Experience Multiple: 500
Base Chance: 10%
Resist: None
Target: Fire
Material: Fire source + Endurance (optional)
Actions: Chanting and dancing
Concentration Check: Standard
Effects: The Adept gains complete control over a
fire source as large as 10 (+ 10 / Rank) cubic feet.
This fire may then be moulded and shaped as desired but with a minimum of one foot thickness in
any dimension. Regardless of the shape created,
the fire retains its normal heat and damaging properties. Damage is as per a large bonfire or [D - 3] if
the entity is within one hex. Bound fire is unaffected by normal rains and winds short of hurricane
conditions, but may be extinguished by a dousing
of water of similar or greater volume than the fire.

nent Bound Fire the object to which it is bound
must be broken, destroyed or seriously defaced.
The Adept may instil basic intelligence in a Bound
Fire if desired and give the entity thus created
simple commands.

21.6 Special Knowledge Spells
Cleansing Flame (S-1)
Range: 10 feet + 10 / Rank above 10
Duration: 125 seconds - 5 / Rank
Experience Multiple: 300
Base Chance: 20%
Resist: Special
Storage: Potion, Ward, Investment
Target: Entity
Effects: The target is wreathed in yellowy green
flames causing hideous suffering, yet curing the
target of the effects of natural poisons (acts as
Rank / 2 vs. synthetic poison), venoms, fevers and
diseases. As the foulness is burnt away, the flames
change colour until they become silvery white.
While cleansing takes place any ongoing harm is
halted. At the end of the duration, the target is
cured and their possessions are cleaned to a high
polish. During the cleansing all the target’s Strike
Chances, Defence and Cast Chances are reduced
by 25 (-1 / Rank). This spell only affects willing
targets. At Rank 20 this spell also cures the effects
of Malignant Flames. This spell can also be used to
cure burns, including third degree burns but scarring from burns are not affected.
Dragon Flames (S-2)
Range: Special
Duration: Immediate
Experience Multiple: 500
Base Chance: 10%
Resist: Passive
Storage: Investment
Target: Area
Effects: The Adept may breathe magical fire like a
Dragon causing D10 (+ 3 / Rank) damage to entities within the area of effect. The area of effect is a
cone issuing from the Adepts mouth and is 20 feet
(+ 5 / Rank) long, and 5 (+ 5 / 3 Ranks) wide at
end farthest from the Adept. If an entity resists the
damage is halved, unless they are more than half
immersed in water (eg. swimming) where they take
no damage. Dragon Flames are analogous to an
instantaneous flash of heat, but there are no overpressure or explosive effects.

If the Adept sacrifices 2 points of Endurance from
a sentient as part of the ritual, the duration of the
Ritual of Binding Fire is enhanced to 1 year (+ 1 /
Rank).

Fire Arc (S-3)
Range: Touch
Duration: Immediate
Experience Multiple: 250
Base Chance: 15%
Resist: Passive
Storage: Potion, Investment
Target: Entity
Effects: The Adept may transform 1 entity (+ 1 / 3
or fraction Ranks) into a bolt of fire that flashes to
any unobstructed point in line of sight up to 25 (+
25 / Rank) feet away. The bolt must land within 5
feet of a potentially flammable substance, an entity
or a solid surface. When the bolt arrives, it bursts
into flame causing D10 damage to all entities
within 1 hex. At Rank 20 the target may instantaneously return to the point of origin of the spell at
the end of the next pulse should they so desire.
Double and triple effects apply to range only. This
spell affects only willing targets and will not work
underwater. Any barriers, wards etc that are passed
over affect the targets as if crossed normally.

If the Adept sacrifices 4 points of Endurance from
a sentient creature the duration of the Ritual of
Binding Fire is permanent. This Fire is bound to a
specific non-movable item from which the fire
appears to issue (e.g. a ruby set in a wall, a line of
runes on a floor etc). Permanent Bound Fire can be
suppressed for [21 - Rank] minutes by a dousing of
water of similar or greater volume than the fire,
attack with magical cold for more damage than the
damage rating of the fire etc. To dissipate a perma-

Fireball (S-4)
Range: 60 feet + 10 / Rank
Duration: Immediate
Experience Multiple: 550
Base Chance: 25%
Resist: Active, Passive
Storage: Investment, Ward
Target: Area
Effects: The Adept conjures a ball of fire 1 foot in
diameter which rushes to a point in line of sight of
72

the Adept, and explodes in a radius of 5 foot (+ 5 /
5 or fraction Ranks). Everything in this radius must
resist or suffer D10 (+ 1 / Rank) damage. If an
entity resists the damage is halved, unless they are
more than half immersed in water (e.g. swimming)
where they take no damage. The spell may set
flammable items afire when it bursts. Hard cover
(e.g. walls, parapets, but not shields) reduces damage to half before resistance. In some instances,
e.g. a character wading through water, resistance
may negate all damage. The fireball is analogous to
an instantaneous flash of heat and there are no
over-pressure or explosive effects. A fireball may
be detonated prematurely by the imposition of
barriers in its line of flight.
At Rank 20 this spell may be cast with a detonation
delay of up to 10 pulses. This extends the casting
time 4 pulses. A delayed blast fireball manifests
itself as a 1 foot wide floating sphere of fire until
detonation occurs. If a second fireball is cast within
the proposed volume of effect of the first, both are
detonated.
Heat Shield (S-5)
Range: Touch
Duration: 1 hour + 1 / Rank
Experience Multiple: 200
Base Chance: 25%
Resist: None
Storage: Potion
Target: Object, Entity
Effects: This spell protects the target against damage by cold or ice for 4 (+ 4 / Rank) points of
protection. Protection is ablative and when the
damage the spell may absorb is exceeded, the spell
is dissipated with any excess damage applied to the
target. Double and triple effects may apply to duration or degree of protection.
Hellfire (S-6)
Range: 10 feet + 5 / Rank
Duration: Immediate
Experience Multiple: 650
Base Chance: 5%
Resist: Active, Passive
Storage: Investment, Ward, Magical Trap
Target: Entity
Effects: This sulphurous fire attacks 1 target for
every 3 (or fraction) Ranks. The target’s Magical
Resistance is reduced by 5 (+ 1 / Rank). The spell
does D10 (+ 2 / Rank) damage to each target. If a
target successfully resists, they suffer only half
damage (round up). Double damage adds an additional 1 / Rank damage and triple damage adds an
additional 2 / Rank damage.
Immolation (S-7)
Range: Self
Duration: 30 minutes + 30 / Rank
Experience Multiple: 200
Base Chance: 15%
Resist: None
Storage: Potion
Target: Self
Effects: The Adept gains the grace and form of a
dancing flame, without in anyway altering their
physical nature, but increasing their defence by 2
(+ 3 / 2 Ranks). Any entity that is within 5 feet
must resist or take 1 / 2 (or fraction) Ranks Fire
damage per pulse. The target will glow with the
same degree of brightness as a Firelight spell 5
Ranks lower, (minimum 1). The target appears to
be a humanoid shape composed of flame. If the
target hides in a fire, the target may only be detected by witchsight or by infravision..
Malignant Flames (S-8)
Range: 10 feet + 10 / Rank
Duration: Immediate
Experience Multiple: 550
Base Chance: 5%
Resist: Active, Passive
Storage: Investment, Ward, Magical Trap
Target: Entity
Effects: The target is swathed in flames that are
only visible to the victim causing D10 (+ 2 / Rank)

21 COLLEGE OF FIRE MAGICS
damage. If the spell is at rank 5 or higher the target
may be cursed with the loss of [D - 4] (+1 / 5
Ranks) points from one characteristic or statistic,
as chosen by the Adept. A characteristic may not
be reduced below 1 as a result of this spell. This
minor curse will last 4 (+ 1 / Rank) days.
Pyrotechnics (S-9)
Range: 60 feet + 10 / Rank
Duration: Immediate
Experience Multiple: 200
Base Chance: 20%
Resist: Passive
Storage: Investment, Magical Trap
Target: Volume
Effects: This spell may be mistaken for a Fireball.
The Adept conjures a ball of fire 1 foot in diameter
which rushes to a point in line of sight of the
Adept, and bursts in a radius of 5 foot (+ 5 / 5 or
fraction Ranks) with an enormous flash of light and
fireworks. Any entity in this radius who can see
must resist or they are dazzled by the flash for [D 5] (+ 1 / 3 or fraction Ranks) pulses. Dazzled entities have their strike chances, cast chances and
perception checks reduced by 1 (+ 2 / Rank). Entities without normal vision (e.g. Undead) are unaffected by Pyrotechnics.
Speak to Fire Creatures (S-10)
Range: 15 feet + 15 / Rank
Duration: 20 minutes + 10 / Rank
Experience Multiple: 75
Base Chance: 40%
Resist: None
Storage: Potion
Target: Self
Effects: This spell allows the Adept to communicate with all fire creatures within range. This
communication is at an effective language rank of
1 (+ 1 / 2 Ranks). For the purposes of this spell,
fire creatures are: Elementals, Salamanders,
Efreets, desert creatures and creatures created
using a Binding Fire Ritual.
Summoning Salamander (S-11)
Range: Unlimited
Duration: Until dispelled
Experience Multiple: 200
Base Chance: 15%
Resist: None
Storage: Investment, Magical Trap
Target: Salamander
Effects: The Adept may summon a salamander
which will then attempt to set afire anything flammable that it can reach. A Salamander may only be
summoned to an environment it can survive in (e.g.
a large fire source, volcano, in a desert). The Adept
has no control over the salamander. Dispelling the
spell returns the salamander to its origin.
Weapon of Flames (S-12)
Range: 5 feet + 1 / Rank
Duration: 5 minutes + 1 / Rank
Experience Multiple: 250
Base Chance: 30%
Resist: None
Storage: Investment
Target: Weapon
Effects: The Adept may cause any weapon to burst
into flame but without causing damage to the
weapon or its wielder. The weapon has its Strike
Chance increased by 1 (+ 1 / Rank), damage increased by + 1 / 2 (or fraction) Ranks. When the
weapon is used against a entity which are creatures
of cold or water or against the Undead, damage
increases to 1 / Rank. A missile may also be the
target (e.g. arrow), but not missile weapons. At
rank 11 any hand-held weapon may be created
from fire. This weapon has normal characteristic
requirements but no weight.
Wall of Fire (S-13)
Range: 10 feet + 10 / Rank
Duration: 10 minutes + 10 / Rank
Experience Multiple: 250
Base Chance: 25%
Resist: Passive

Storage: Investment, Ward, Magical Trap
Target: Area
Effects: The Adept may create a 5 foot high × 20
foot long × 1 foot thick wall of flames, or a 5 foot
high × 1 foot thick circle of flames with a 5 foot
radius, or a pillar of fire 15 feet high with a 2 foot
radius. The Adept can increase any single dimension by 1 foot / Rank. The wall must be anchored
to a surface and cannot easily be seen through. Any
object or entity that passes through the wall must
resist or suffer D10 ( + 1 / Rank) damage. If an
entity resists the damage is halved. The wall will
provide light equivalent to a large campfire. At
Rank 20 the Adept may double the damage of a
Wall of Fire by halving its duration.
Wildfires (S-14)
Range: Touch
Duration: 20 minutes + 20 / Rank
Experience Multiple: 250
Base Chance: 25%
Resist: Active, Passive
Storage: Potion, Investment
Target: Entity
Effects: This spell imbues an entity with the speed
and essence of a forest fire. The target may run
without tiring at a speed of 20 (+ 2 / Rank) miles
per hour over any solid surface. Additionally they
may “run” across any substance that would ordinarily support fire (e.g. across tree tops, up a
wooden wall, or across oil). Like a fire, momentum
effects do not apply to the target: they may take
corners at impossible angles, climb a flammable
surface impossibly quickly, or stop instantly in
place. The target may leap a gap (as a fire leaps a
firebreak) a horizontal distance of PS (+ 1 / Rank)
feet or a vertical distance of 1/2 PS (+ 1 / Rank)
feet. The target will leave charred footprints behind
them but will not normally ignite surfaces they
cross.
An entity under Wildfires is moving too quickly to
undertake normal tactical combat or to otherwise
interact with other entities. If they wish to engage
in melee they must dissipate the spell.
The target must keep moving at least two hexes in
any direction per pulse. If they stop the spell is
dissipated. When the spell ends the target stops
instantly in place.
At Rank 20 the target may extend the duration of
the spell by suffering 1 EN damage / 10 additional
minutes.

21.7 Special Knowledge Rituals
Create Drought (R-1)
Range: 2 miles + 2 / Rank
Duration: 8 hours + 8 / Rank
Experience Multiple: 200
Base Chance: 30% + 3% / Rank
Resist: None
Target: Area
Material: Fire source
Actions: Chanting and dancing
Concentration Check: Standard
Effects: This ritual increases temperature in the
target area by 2C per Rank, slowly building up at 2
degrees C per hour. This may impede rainfall, kills
plants, dry up wells, ponds, etc and ruin crops if
repeated regularly. In cold or icy areas climes it
may make an area tolerable for life for a time or
result in standing fogs if the difference between
climes is too extreme.
Flame Sight (R-2)
Range: 5 feet
Duration: 1 minute + 1 / Rank
Experience Multiple: 250
Base Chance: 40% + 5% / Rank
Resist: None
Material: Small fire
Actions: Staring into flames
Concentration Check: Standard
Effects: The Adept must sit next to a fire and stare
into the flames for half an hour. They may then
attempt one of three visions:
73

• A view as if the Adept was looking out of a fire
within 1/2 a mile (+ 1/4 mile Rank). The Adept
may choose to look out of a specific fire, otherwise
the fire is randomly determined from all those in
range. If no fires are within range the Adept will
see nothing. They may move their viewpoint
within the fire so as to see in any direction. There
is no way to perceive that a fire is looked through
by this ritual.
• A precognitive vision which is controlled by the
GM. The detail of this vision and the amount of
information obtained increases with rank.
• If the Adept chooses to sacrifice an object (which
is destroyed) in the fire, they gain a vision relevant
to that object’s past. This version of the ritual can
only be performed on flammable objects.
Ritual of Summoning a Lesser Efreeti (R-3)
Range: 5 feet
Duration: 1 hour + 1 / Rank
Experience Multiple: 200
Base Chance: 30% + 3% / Rank
Resist: None
Target: Entity
Cast Time: 3 Hours - 10 minutes / Rank (minimum
10 minutes)
Material: none
Concentration Check: Standard
Effects: The Adept may summon a Lesser Efreeti
to act as a steed for themselves. The Lesser Efreeti
will obey all its rider’s commands while it is
mounted. If the rider is not mounted or is out of its
presence then it will only follow simple passive
commands (e.g. wait here, go there). At Rank 14 or
higher it will obey commands which move it away
from the rider (e.g. go attack people behind the hill,
go and pick up Bob).
Its characteristics and statistics are based on either
the Adept or their rank in the ritual.
PS = 30 (+ 2 / Rank) MD = Adept
EN = 30 (+ 1 / Rank) AG = Adept
FT = 30 (+ 2 / Rank) WP = Adept
PC = Adept
MA = Adept
Natural Armour WP / 4 (+ 1 / 3 or fraction
Ranks)
DEF WP + Ritual BC / 2
TMR Walking = 5 (+ Rank / 4); Flying 8 (+ Rank /
3)
Movement Rate = 20 (+ 2 / Rank) miles per hour
Size 1 (+ 1 / 10 Ranks) hexes
Weapons The Lesser Efreeti has three physical
attacks, two claws and a bite. These have a base
chance of 40% and do D + 4 (+ 1 / 2 Ranks) damage. It also has a 65% chance of igniting in battle
with flames extending in a 5 foot radius. Entities
within the radius, except for the rider, must resist
or suffer D (+ 1 / 2 Ranks) damage. Entities that
resist suffer half damage.
Abilities The Lesser Efreeti is immune to damage
from fire but takes double damage from ice or
water and magic resistance to these attacks is reduced by 10%. It cannot be banished while the
rider is mounted.
At Rank 20 the ritual has extra benefits:
• the adept may nominate another entity as the
rider.
• the Lesser Efreeti has Strength and Fatigue of 80
and Endurance of 60.
• NA is 8 (+ WP / 4).
• damage from claws and bite is [D + 16].
• the ritual’s duration is 24 hours.
This ritual may double or triple duration and may
backfire. It may be invested with a material cost of
50,000 silver pieces per charge.
Summon Fire Elemental (R-4)
Range: 20 feet
Duration: Concentration, no maximum
Experience Multiple: 450
Base Chance: MA + 4% / Rank

21 COLLEGE OF FIRE MAGICS
Resist: None
Target: Fire Elemental
Material: Large fire
Actions: Concentration
Concentration Check: Standard
Effects: The Adept may summon a Fire Elemental
with a combined endurance and fatigue of 15 (+ 5 /
Rank) which appears within the fire. The Elemental is under the Adept’s control but strongly resents
being summoned.

21.8 Damage by Burning

If the ritual backfires the elemental arrives uncontrolled and will immediately attempt to kill the
summoner

If the pulse after ignition is spent putting flammables out, no further damage occurs. In general
exposure to a fire source indicates exposure for a
full pulse, therefore standing in a bonfire might
hurt a character but simply jumping through one
might not. This table should be used as a guide to
how much damage exposure to a particular fire
source might cause.

The Elemental is returned to its own dimension if
the summoner’s concentration is broken, it is banished, or the Adept casts a Fire Special Counterspell.

Occasionally a character will suffer damage from
falling in to, or being exposed to, fire. The following table indicates the pulse by pulse damage taken
by an entity being in contact with fire. While ignition of flammables on a person will not always
occur, if an entity is immersed in fire (e.g. passage
through a Wall of Fire) flammable possessions will
catch alight. Unless otherwise specified, Magical
Fire will automatically ignite flammables.

Clothing alight after passage through
a fire
Standing in campfire
Standing in a one Hex bonfire
Very large bonfire, house fire
Raging forest fire
Immersion in boiling water
Immersion in boiling oil
Immersion in molten metal or lava
Sauron’s forge

74

D+3
D+5
D + 10
D + 15
D + 25
D + 35
D + 50
D + 100
D + 150

21.9 Some Useful Temperatures
100°C
113°C
200°C
232°C
300°C
327°C
419°C
800°C
1000°C
1053°C
1083°C

Boiling Water
Molten Sulphur
Burning paper and meths
Tin melts
Boiling oil
Lead melts
Molten zinc
Molten salt
Silver melts
Gold melts
Bronze and copper armour and weapons
melt
1600°C Molten iron
1800°C Sand melts
4000°C Molten graphite
>4000°C Sauron’s jewellery

22 COLLEGE OF ICE MAGICS

22 The College of Ice Magics (Ver 1.5)
The College of Ice Magics is concerned with the
shaping of ice and snow, freezing and the manipulation of cold.
Ice Mages have been most valuable to those fragile
communities living in cold, arid and often dangerous places, where food is hard to grow and
neighbours can be separated by days of travel.
Many of the Ice Mages’ Spells and Rituals are
designed either to enhance living in these inhospitable areas or to aid hunting and defending themselves from arctic predators. Ice Mages are seldom
seen in warm climates, and have been unpopular
with Philosophers of magic who cannot agree
whether to place Cold as an element in its own
right. The College of Ice Magics has been likened
to the Colleges of Water, due to its connections
with Ice and Snow, Fire, as an antithesis, Air, due
to its weather-like effects and Mind, due to the
peculiar mental fortitude its Adepts have demonstrated while living in cold climes.
Philosophy
Ice Mages tend to be of a solitary and quiet nature,
often having lived their lives isolated in areas of
sparse population density. Many are hunters or
only part-time Mages. They generally have fair
relations with most Colleges, but relations with the
Fire and Water Colleges are distinctly icy.
Traditional Colours
Ice Mages are not known to dress flamboyantly.
Whites and Greys are popular colours when dressing in warmer climes, perhaps with a hint of pale
blue. However in their own element they usually
wear undyed leathers and furs.
Notes
All the Ice and Snow generated by the spells of this
college is real unless otherwise specified. It is in no
way magical, and does not have a duration. External effects (e.g. heat) will begin to affect it immediately upon its generation.
Ice Magics is considered “opposite” to Fire
Magics. Many Ice spells and effects are uniquely
vulnerable to the spells of the Fire College, and to
Fire. Additionally Ice Magics have some spells and
effects that are especially effective versus Fire and
Fire creatures. This in no way affects the “opposite” relationship that the Colleges of Water and
Fire Magics have to each other. Water Magics and
Ice Magics are not considered related, rather they
are generally regarded as antagonistic rivals.

22.1 Restrictions
Adepts of the College of Ice Magics may practise
their arts without restriction.
The MA requirement for this College is 13.

22.2 Base Chance Modifiers
All base chances of Ice Mages are affected by
temperature. Only one of the temperature modifiers
may be applied at one time. Consult the Weather
Scale Table for Weather Gauge details.
Temp

Gauge

Temp °C

Mod.

Very Cold 0-3
<= - 5
+ 10
Cold
4-5
<= 5
+5
Average
6-8
6-24
+0
Hot
9-10
>= 25
-5
Very Hot
11 +
>= 35
- 10
If the Adept is standing on Ice or Snow in an area
where it is in abundance they gain an additional
+5%.

22.3 Talents
Cold Affinity (T-1)
Experience Multiple: 100
Effects: This talent allows the Adept to ignore the
deleterious effects of low body temperature. The
adept is treated as if having a Resist Cold spell of
equal Rank to the Rank in this talent in effect at all
times. Should the Adept also be under the effect of

a Resist Cold spell, the higher of the two ranks is in
effect.
Endure Hardship (T-2)
Experience Multiple: 150
Effects: This talent allows the Adept to function
capably even in harsh and forbidding environments. The Adept may go without food (but not
water!) for 1 (+1 / Rank) days every three months
with no ill effects. These days may be taken singly
or consecutively, and the Adept need not consume
extra food later to make up for this time spent
fasting. The Adept may additionally increase the
Base Chance of any concentration checks made in
hostile environments, when the concentration
checks are reduced below 4 × WP due to environmental effects, by 1 × WP / 5 full Ranks the Adept
has in this talent, up to a maximum of 4 × WP.
Finally, the Adept may reduce high fatigue rates
due to environmental and weather effects, other
than those relating to heat or fire, by a 1 row shift
per 10 full ranks the Adept has in this talent on the
Rate of Exercise Chart (see §58.1, under the Fatigue and Encumbrance Chart), down to a minimum rate of medium fatigue, or light fatigue at
rank 20. For example, at Rank 10 a Strenuous
climb up a mountain may be treated as if it is only
a Hard climb.

22.4 General Knowledge Spells
Extinguish Fires (G-1)
Range: 15 feet + 15 / Rank
Duration: Immediate
Experience Multiple: 100
Base Chance: 50%
Resist: None
Storage: Investment, Ward, Magical Trap
Target: Volume
Effects: When successfully cast, this spell will
extinguish all fire in a 10 foot (+ 10 / Rank) radius
sphere, by smothering them with ice crystals. All
the volume affected must be within range of the
spell. If the range is doubled or tripled the volume
may likewise be increased. Magical fires are not
affected. This spell is identical to the Fire College
“Extinguish Fires”, except that the effect is
achieved by physical means (this is merely a cosmetic difference).
Freeze (G-2)
Range: 5 feet + 5 / Rank
Duration: 1 day + 1 / Rank
Experience Multiple: 50
Base Chance: 40%
Resist: None
Storage: Investment
Target: Object
Effects: The Adept may freeze one object of up to
5 lb. (+ 5 / Rank). This freezing will protect the
object from decay while the duration lasts. While
frozen the object will be as cold as ice to the touch
and will drip slightly from condensation. When the
duration has expired the object will defrost at the
rate of 1 minute per pound of weight.
Ice Creation (G-3)
Range: 15 feet + 10 / Rank
Duration: Immediate
Experience Multiple: 100
Base Chance: 25%
Resist: None
Storage: Investment, Ward, Magical Trap
Target: Area
Effects: This spell creates a film of ice 1 inch thick
in a single square of dimensions 1 (+ 1 / Rank) foot
each side or a single cube of ice of dimensions 6 (+
6 / Rank) inches cubed. The ice must be created on
the ground and not on top of an entity. It is nonmagical and will persist until melted, etc.
Ice Traversal (G-4)
Range: 10 feet + 10 / Rank
Duration: 20 minutes + 20 / Rank
75

Experience Multiple: 125
Base Chance: 30%
Resist: None
Storage: Investment
Target: Entity
Effects: This spell enables 1 target (+ 1 / 4 full
Ranks) to travel over ice and/or snow without
slipping or sinking in, as if it were normal earth
and/or rock. For example, this would enable climbers to climb icy slopes. Quadrupeds are treated as
two targets for the purposes of this spell. In addition, if the terrain travelled on is flat ice, each
target’s TMR is increased by 1 (+ 1 / 3 full Ranks)
while on the ice. See Travel on Ice (22.8), for
additional detail.
Refrigeration (G-5)
Range: 25 feet + 5 / Rank
Duration: 1 hour + 1 / Rank
Experience Multiple: 50
Base Chance: 35%
Resist: None
Storage: Investment
Target: Volume
Effects: The caster may cause the ambient temperature of one 15 × 15 × 15 foot cube to lower by 2°C
/ Rank.
Resist Cold (G-6)
Range: Touch
Duration: 1 hour + 1 / Rank
Experience Multiple: 100
Base Chance: 40%
Resist: None
Storage: Investment, Ward, Potion
Target: Entity
Effects: This spell protects the target from the
effects of cold temperature by increasing the
Gauge by 1 (+ 1 / 4 full Ranks) up to a maximum
of Gauge 7 (Comfortable). It will totally protect the
target from the effects of Hypothermia at Rank
11+. In addition, the target suffers 1 (+ 1 / 4 or
fraction Ranks) less damage due to magical or nonmagical cold based attacks. This spell is identical
to the special knowledge Air college spell of the
same name.
Snow Shovel (G-7)
Range: Self
Duration: Concentration: maximum of 15 minutes
+ 15 / Rank
Experience Multiple: 125
Base Chance: 20%
Resist: None
Storage: Potion
Target: Self
Effects: This spell enables the Adept to clear a path
along snow or ice obstructed ground and/or to
tunnel through snow and ice. Any snow or ice up
to 2 feet in front of the Adept undergoes a change
in density to dry snow, and is pushed to either side,
leaving a gap 2 feet wider, and higher (if applicable), than the size of the Adept. No more than 1
hex (+ 1 / 2 full Ranks), may be cleared per pulse
in this manner. This effect moves with the Adept.
The Adept may lean in order to direct the path up
or down. Note that the walls and roof of a tunnel
through snow or ice are merely packed snow and
do not confer any particular structural support or
stability. See Travel on or through Snow (22.8) for
additional detail.
Water to Ice (G-8)
Range: 10 feet + 10 / Rank
Duration: Immediate
Experience Multiple: 100
Base Chance: 15%
Resist: None
Storage: Investment, Ward, Magical Trap
Target: Volume of Water
Effects: The Adept can freeze up to 10 (+ 10 /
Rank) cubic feet of existing water based liquids
into solid ice, or into snow of a density chosen by
the Adept. All the water to be transformed must be

22 COLLEGE OF ICE MAGICS
within the Adept’s range at the time of casting.
This spell may not be cast on or near entities or
their possessions.

22.5 General Knowledge Rituals
Create Igloo (Q-1)
Duration: 1 hour + 1 / Rank
Experience Multiple: 100
Base Chance: MA + 4% / Rank
Effects: The Adept must spend one hour in ritual
construction of a miniature dome made out of snow
or ice cubes. At the end of this time the Adept must
make a successful cast check. If successful, the
dome swells in size to become an igloo of internal
size 5 (+ 1 / Rank) feet radius and 2 feet thick. This
ritual cannot backfire. The Igloo has a single entrance which is chosen by the Adept to be up to
half its internal height in both height and length.
The inside temperature of the igloo always counts
as very cold (-10 degrees) and the following enchantments apply to objects or entities while they
remain inside the igloo:
• All entities and creatures are treated as having a
resist cold spell upon them of equal rank to that of
the Adepts Rank in this ritual
• All organic objects are preserved from decay
In addition the igloo counts as bound snow while
the duration is in effect and will not melt or break
due to non-magical forces (although magical attacks affect it as normal). Once the duration runs
out the igloo reverts to a normal (non-magical)
igloo and will thereafter melt, collapse etc. as
normal due to external conditions.
Bind Ice and Snow (Q-2)
Duration: Concentration: Maximum 1 hour + 1 /
Rank
Experience Multiple: 750
Base Chance: MA + 4% / Rank
Effects: The Adept may bind all of the ice and
snow within a 5 (+ 5 / Rank) feet radius circle of
the Adept. The results of this ritual are similar to
those for the binding of other elements. The Adept
gains control of all of the facets of the element.
The Adept may move or shape the ice and snow,
change its consistency and instil intelligence in it
as desired. Finally, the Adept may sacrifice a point
of MA (this may be bought back with EP) in order
to make a part or all of the bound ice and snow
permanently bound. In this instance, the bound ice
and snow is non-intelligent but magical, and is
enduring. Almost no magical or physical force will
affect it (e.g. it resists Wizard’s Eye and Telepathy), with the exception of magical heat and fire,
against which it has 100 times the resistance of
ordinary ice and snow, and if any part of it remains
it will (slowly) regenerate.

22.6 Special Knowledge Spells
Armour of Ice (S-1)
Range: Touch
Duration: 30 minutes + 30 / Rank
Experience Multiple: 250
Base Chance: 20%
Resist: Active, Passive
Storage: Investment, Ward, Magical Trap
Target: Entity
Effects: The target of this spell is covered by a
magical Armour of Ice, which provides 5 points of
Armour Protection (+ 1 / 4 full Ranks) which fluctuates according to the current temperature: the
Armour gains +2 points of Armour Protection
when the temperature is very cold, +1 when it’s
cold, -1 when it’s hot, and -2 when it’s very hot
(see 22.2). The armour has a weight rating of 5 (see
56.3) and subtracts 2 from AG and 20 from stealth.
Ice Armour may not be cast on entities wearing
armour. This spell will stack with other defensive
spells. The Ice Armour is vulnerable to Fire, and
has an ablative effect; it will absorb up to half of
any fire damage taken, but for every 10 points of
damage taken (before halving) one point of protection is removed from the armour, and if the protec-

tion is reduced to zero the spell is immediately
dissipated.
Icy Transformation (S-2)
Range: 10 feet + 10 / Rank
Duration: 10 minutes + 10 / Rank
Experience Multiple: 300
Base Chance: 25%
Resist: Special
Storage: Investment, Ward, Magical Trap
Target: One (metal or mineral) Object
Effects: This spell turns one metal or mineral object of up to 5 lbs (+ 5 / Rank) entirely into ice (a
wall of iron would have to be entirely turned to ice
but a single brick in a wall could be transformed).
The object is then transparent and vulnerable to
damage, heat etc. At the end of the duration, the
object will revert back, but any damage will not be
repaired.
Freezing Wind (S-3)
Range: 5 feet + 5 / Rank
Duration: 5 seconds + 5 / Rank
Experience Multiple: 225
Base Chance: 30%
Resist: Passive
Storage: Investment, Ward, Magical Trap
Target: Volume
Effects: This spell causes Arctic conditions to
prevail in a 10 (+ 1 / Rank) feet cube. Any entity
within this cube which fails to resist will suffer [D
- 4] (+ 1 / 2 full Ranks) points of magical cold
damage per pulse. Creatures of fire (efreet, salamander or elemental) take half damage even if they
resist.
Frostbite (S-4)
Range: 50 feet + 25 / Rank
Duration: Immediate
Experience Multiple: 200
Base Chance: 20%
Resist: Special
Storage: Investment
Target: Special
Effects: This spell may either be used to kill (5 ×
[Rank])% of all non-sentient plants within the
Adepts range, or the Adept may target and kill 1
individual plant (+ 1 / Rank). There is no visible
effect, but death due to frostbite occurs at the time
of casting and no amount of non-magical effort
will revive plants affected by this spell. Sentient
plants take 1 (+ 1 / Rank) points of damage (instead of dying), and may passively resist for no
damage. Other plants that are magical or especially
resistant to cold may also be entitled to a passive
resistance roll versus the spell’s effects.
Frozen Doom (S-5)
Range: 10 feet + 10 / Rank
Duration: Immediate
Experience Multiple: 500
Base Chance: 5%
Resist: Active, Passive
Storage: Investment, Ward, Magical Trap
Target: Entity
Effects: This spell uses magical cold to freeze solid
the blood of one target entity, which may be up to
1 hex (+ 1 / 2 full Ranks) in size, killing it instantly
if they fail to resist.
Hibernation (S-6)
Range: 5 feet
Duration: Special
Experience Multiple: 250
Base Chance: 20%
Resist: Active, Passive
Storage: Investment, Ward, Magical Trap
Target: Entity
Effects: The target of this spell is placed in suspended animation for up to ([Rank × Rank]) days
as specified by the Adept. At Rank 20 there is no
maximum and the Adept may choose any duration.
All bodily functions including ageing, are suspended for the duration of the spell and the target
feels cool to the touch. The target is immune to
cold and suffocation, and takes no more damage
from existing injuries while this spell is in effect.
76

Additional injuries will still affect the target, but
any damage that would occur due to bleeding,
poison etc. is ignored. When the spell duration runs
out, or the spell is dispelled, the target awakens
with physical strength reduced by 1 / full week
hibernated and immediately begins to suffer from
any existing injuries and conditions (poison, disease, shock, bleeding and suchlike). Physical
strength may be regained at a rate of 1 point per
day, and is not reduced below 1. Entities that naturally hibernate suffer -20 to their Magic Resistance
vs. this spell.
Ice Bolt (S-7)
Range: 20 feet + 10 / Rank
Duration: Immediate
Experience Multiple: 300
Base Chance: 35%
Resist: None
Storage: Investment, Ward, Magical Trap
Target: Entity, Object
Effects: The Adept creates a 2 foot long, 2 inch
diameter non-magical bolt of ice which is projected
at a target with a Strike chance of 30% (+ 3 / Rank)
+ MD, modified by range as if it were a heavy
crossbow. The ice bolt strikes as an A class
weapon doing [D + 4] (+ 1 / Rank) damage which
can stun and inflict specific grievous injuries. Once
the spell is cast the target gets no magic resistance,
but the target’s defence is subtracted from the
chance to hit, since the effect of the spell is to
create a physical bolt. A double or triple effect
cannot affect damage but may add +10% and
+20% to the Strike Chance respectively.
Ice Construction (S-8)
Range: 15 feet + 5 / Rank
Duration: 10 minutes + 10 / Rank
Experience Multiple: 225
Base Chance: 15%
Resist: None
Storage: Investment, Ward, Magical Trap
Target: Volume
Effects: The Adept may conjure 30 (+ 30 / Rank)
cubic feet ice in up to 1 + Rank shapes of the
Adept’s choice. The shapes always appear entirely
within the range of the Adept and may not appear
above or inside (partially or wholly) any entity.
Each shape must appear on the ground in a stable
fashion (not about to topple over) and must have a
minimum thickness of 6 inches in any part. When
the spell duration expires, the ice disappears (returns to whence it came).
Iceberg (S-9)
Range: 10 feet + 10 / Rank
Duration: Concentration: Maximum 30 minutes +
30 / Rank
Experience Multiple: 150
Base Chance: 30%
Resist: None
Storage: Investment, Ward, Magical Trap
Target: Area of Water
Effects: The Adept creates a polyhedral iceberg of
dimensions 10 (+ 2 / Rank) feet cubed. It may only
be successfully created in an existing volume of
liquid sufficient to hold it without it touching the
bottom. 8/9 of the Iceberg will be submerged. The
Iceberg will be flat topped and may stably support
up to 1 hex of entities (+ 1 / Rank) (Additional
entities will cause it to roll in the water). While the
Adept is in contact with the Iceberg the Adept may
move the Iceberg at a speed of 5 (+ 1 / Rank )
miles per hour. The Iceberg spell also has a calming effect on the water around it, reducing the size
of all waves up to Rank feet away by Rank feet.
Ice Pack (S-10)
Range: 10 feet +5 / Rank
Duration: Immediate
Experience Multiple: 150
Base Chance: 15%
Resist: None
Storage: Investment, Potion
Target: Entity

22 COLLEGE OF ICE MAGICS
Effects: This spell immediately halts any damage
being taken due to shock or blood loss and the
target gains 1 / Rank to their chance of recovering
from stun or unconsciousness (if applicable). It will
restore a (live) character on negative endurance to
zero, although any further damage will start the
process of shock and bleeding as usual. In addition,
if the target is suffering from the adverse effects of
a fright roll or similar emotional effect, they get a
second roll (in some cases a second resistance roll)
to recover from (“or snap out of”) it with +1% /
Rank added to their dice roll. The cause of the
fright or shock may have been magical or otherwise. This spell may be recast on each target as
often as desired. This spell will not work on regenerating entities (including those under the effects of
a trollskin spell).
Ice Projectiles (S-11)
Range: 20 feet + 5 / Rank
Duration: Immediate
Experience Multiple: 300
Base Chance: 30%
Resist: Passive
Storage: Investment, Ward, Magical Trap
Target: 1 Entity + 1 / Rank
Effects: Each entity targeted by this spell must
resist or suffer [D - 4] (+ 1 / Rank) points of magical damage due to being pierced by flying A class
shards of ice (armour does absorb damage but there
is no AG loss). If this spell is doubled the adept
may not double damage but may choose to have
the ice projectiles ignore armour instead. If this
spell is tripled, the adept may roll for a possible A
class specific grievous injury for each target that
failed to resist, in addition to ignoring armour as
above (and still doing fatigue damage except as
part of the grievous result), or reduce the targets’
resistances as usual.
Ray of Cold (S-12)
Range: 30 feet + 15 / Rank
Duration: Immediate
Experience Multiple: 300
Base Chance: 30%
Resist: Passive
Storage: Investment, Ward, Magical Trap
Target: Entity, Object
Effects: This spell projects a blast of intense magical cold at the target. The ray of cold will impact
either on the target or on the first obstruction
blocking the path from the Adept to the target.
Anything struck by the ray must either resist or
suffer [D + 1] (+ 1 / Rank) damage (resist for half
damage).
Snowball (S-13)
Range: 10 feet +10 / Rank
Duration: Concentration: Maximum of 10 minutes
+ 10 / Rank
Experience Multiple: 200
Base Chance: 20%
Resist: None
Storage: Investment, Ward, Magical Trap
Target: Adept, Special (chosen each pulse)
Effects: This spell causes the ground within 5’ of
the Adept to be instantly covered with 5 inches of
conjured snow (returns to whence it came when the
spell duration expires). Snowballs then form themselves out of the snow and launch themselves at a
target or targets of the Adepts choice. Up to 1
target (+ 1 / 4 full Ranks) can be pelted with a
flurry of snowballs each pulse. A different set of
targets may be chosen by the Adept each pulse.
Each target must resist or suffer a reduction of -1%
(1 / 2 full Ranks) to their Base Chance of doing
anything while being snowballed and in addition
must make a 4 × WP roll (3 × WP if Adept is
above Rank 10 in this spell) to perform any action
that involves concentrating (e.g. casting a spell).
Each target need only resist this spell once. If the
Adept leaves the 5’ diameter circle of snow while
this spell is in effect, the spell is automatically
dissipated.

Snowfall (S-14)
Range: 10 feet + 10 / Rank
Duration: 10 minutes + 5 / Rank
Experience Multiple: 200
Base Chance: 40%
Resist: None
Storage: Investment, Ward
Target: Area
Effects: This spell causes snow to begin gently
falling in an area of between 5 feet minimum and
40 feet (+ 5 / Rank) maximum diameter (chosen by
the adept). The snow will form in the air at between 10 and 40 feet above the ground, and will
gently float to the ground. The area within this
spell, if above zero degrees in temperature, will be
magically cooled to zero degrees for the spell
duration (nb. This will not count as a positive
modifier to Ice mage base chances as it is magical,
although it may reduce negative modifiers to zero).
For each 5 minutes that this spell is in effect, one
inch of powder will cover the ground. At the end of
the spell the snow will remain, but the temperature
will be restored to normal (and the snow may begin
melting). Note that if this spell is, for example, cast
inside a house with a 20 foot ceiling, as the snow is
formed in the air between 10 and 40 feet up, only a
third of the snow will fall in the room. The other
two thirds will appear above the 20 foot line and
will fall on the roof, with the exception of a small
amount that appears in the attic.
Wall of Ice (S-15)
Range: 20 feet + 10 / Rank
Duration: 10 minutes + 10 / Rank
Experience Multiple: 200
Base Chance: 30%
Resist: None
Storage: Investment, Ward
Target: Area
Effects: This spell conjures a wall of ice that is 10’
high, 20’ long and 2’ thick, or a pillar 15’ high
with a 3’ diameter. The adept may choose to increase the height of the wall by 1’, the width by 2’
or the thickness by 6", or the height or diameter of
the pillar by 1’, for each Rank the Adept possesses
in the spell. The wall may be uniformly curved up
to a half circle. The wall may not be created on top
of an entity, and is subject to the usual restrictions
on physical walls. The wall is translucent but not
transparent. When the spell duration expires the ice
returns to whence it came.
Weapon of Cold (S-16)
Range: 10 feet + 5 / Rank
Duration: 5 minutes + 1 / Rank
Experience Multiple: 250
Base Chance: 30%
Resist: None
Storage: Investment
Target: Weapon
Effects: The weapon over which the spell is cast
becomes infernally cold without harm to either the
weapon or the user of it. The base chance of hitting
with the weapon is increased by 1 (+ 1 / Rank) and
the damage done by the weapon is increased by 1
(+ 1 / 3 full Ranks). This amount of damage is
tripled if the object of the attack is a creature of
fire.
Winter Garden (S-17)
Range: 20 feet + 10 / Rank
Duration: 2 weeks + 2 / Rank
Experience Multiple: 100
Base Chance: 35%
Resist: None
Storage: Investment, Ward
Target: Plant
Effects: This spell bestows resistance to cold on
one plant (+ 1 / Rank) or 25 square feet (+ 25 /
Rank) in one patch of the same species of plant of
the Adepts choice. Plants with this resistance are
immune to the Frostbite spell and will thrive even
in permafrost and arctic temperatures.

22.7 Special Knowledge Rituals
Snow Simulacrum (R-1)
Duration: Concentration: Maximum 1 hour + 1 /
Rank
Experience Multiple: 300
Base Chance: MA + 3% / Rank
Resist: Special
Effects: The Adept spends an hour forming a human or animal figure, no larger than 1 hex (+ 1 / 3
full Ranks) in size, out of snow, which must be
already present.
The animated sculpture will have the same characteristics as the sculptured entity, except that all
characteristics are reduced by 25%. This will generally cause abilities to be lessened by a like
amount (e.g. Flying speed is reduced by 25% (if
the entity can fly) due to Agility being reduced,
Damage from attacks is reduced by 25% due to
Strength reduction, Base Chance to hit is reduced
because of MD reduction, etc.).
Magical abilities are not copied in the copied form,
although quasi-magical abilities may be. The simulacrum will not bear a close resemblance to any
other figure unless the figure chosen to be duplicated is present during the ritual. In the latter instance the figure being duplicated may choose to
actively resist the ritual; otherwise the ritual is nonresistible. The simulacrum has only animal intelligence, and the Adept may give it simple instructions or actively concentrate to control its movements (requires the Adept to perform pass actions).
The simulacrum will have vague inclinations relating to its original (borrowed) form, and although
no longer comprised of snow (it takes on a fleshy
appearance, or whatever is appropriate for the
original entity) will have an adverse reaction to the
presence of heat and flame and will take an additional 2 points from heat and flame attacks. The
simulacrum normally resembles a generic example
of its borrowed form except that it has a snowy
sheen to it and is cool to the touch. Clothes, possessions and suchlike are not duplicated by this
ritual.
When the Adepts concentration ceases, the simulacrum will collapse into a pile of ordinary nonmagical snow.
Summoning and Controlling Ice Elemental
(R-2)
Duration: Concentration: Maximum 1 hour + 1 /
Rank
Experience Multiple: 250
Base Chance: MA + 5% / Rank
Effects: The Adept summons an Ice Elemental
from its home plane (conjectured to be the elemental plane of Cold), which will appear within 20
feet. This ritual takes 2 hours, and may only be
performed in temperatures of 0 degrees or less. If
the ritual backfires the Elemental arrives uncontrolled and will attack the Adept (and others
nearby); however, if the ritual is successful the
Elemental is controlled and must obey the Adept’s
every whim. The Elemental remains until dispelled, which the Adept may do by successfully
casting a Special Knowledge counterspell at it. 1%
is added to the base chance of success on this ritual
for each point of Willpower the Adept has over 15.
Ice Elementals are similar to other Elementals in
that they do not normally exist on this plane, but
are summoned by Ice magics. They will always be
hostile to their summoner and will attack if uncontrolled. Ice Elementals are impervious to attacks
with non-magical weapons. Magic does affect
them. Ice Elementals are vulnerable to fire and can
be damaged by attacks involving this “opposite
element”. Ice Elementals have a combined endurance and fatigue of 15 (+ 6 / Rank), which must be
divided in such a way that they fall into the ranges
indicated below. Any reference to Rank below
refers to the Adept’s Rank in this ritual.
Ice Elementals have the following characteristics:
Habitat Other Planes

77

22 COLLEGE OF ICE MAGICS
Frequency Very Rare
Number 1
Description Ice Elementals appear as lean crystalline humanoids with frosty hair and silvery blue
eyes. They are half as tall, in feet, as their Endurance.
Talents, Skills and Magic Ice Elementals can
disappear into Ice with only a 10% chance of being
detected. They can freeze water within line of sight
at a rate of one cubic foot per pulse for every point
of Physical strength available to the Elemental
(entities within the area get a 3 × AG check to
avoid being caught and trapped while Ice forms
around them), and they can cast a Ice construction
and Wall of Ice at a Rank equal to their summoner’s Rank + 4 in this ritual. These are talents
and cost no fatigue. They may expend 2 fatigue to
fire an Ice bolt (as per the Spell) striking as an A
class weapon with a base chance equal to the Elemental’s combined maximum Endurance and Fatigue, and doing [D + 4] (+ 1 / Rank) points of
damage. Since this is a physical attack formed
from the Ice Elemental, the Ice bolt remains after
firing.
Movement Rates Running 200 + 10 × Summoner’s Rank
PS: 20 + 5 / Rank WP: 14 - 18
MD: 20 - 25
EN: 5 - 50
AG: 15 - 20
FT: 10 - 85
MA: None
PC: 15 - 20
PB: 8 - 10
TMR: 4 + 1 / 5 Ranks
NA: Skin absorbs 5 DP
Weapons Ice Elementals strike their opponents
with open hands and pierce them with their long
icy fingers. They can attack twice in the same pulse
without penalty doing [D + 3] (+ 1 / Rank) damage
per strike.
Ritual of Winter (R-3)
Duration: Concentration: Maximum 1 hour + 1 /
Rank

Experience Multiple: 350
Base Chance: MA + 3% / Rank
Effects: The Adept may change one or more of the
three components which make up the weather by
performing a ritual dance. The three components of
weather are:
1. Precipitation, Degree
2. Temperature, Gauge
3. Wind, Force
The GM should consult the weather table and
advise the player of the current level of these three
components before the Adept starts performing the
ritual. The Adept may change the current components by 1 (+ 1 / 2 full Ranks). All the changes
may be in any direction on the table with the proviso that the Adept may never raise the temperature, and must lower it by at least one degree. All
weather within 2 miles (+ 2 / Rank) of the Adept is
affected by the ritual. This ritual cannot backfire.

22.8 The Element of Ice
Ice
Ice weighs 47.2 lb. per cubic foot and is translucent
(but not transparent).
Breaking through Ice
This applies particularly to the spells Ice Creation,
Ice Construction and Wall of Ice, but may also be
used in natural settings (for example a single inch
of ice may cover a lake).
Ice is deemed to have 10 points of fatigue per inch,
120 points per foot. It takes only half damage from
being hit by B class weapons, full damage from
axes, fire and water based attacks, and double
damage from picks (note that for fire based attacks
bonuses against cold creatures apply, for those
spells that have them). Other weapons only do a
single point. Endurance blows do double damage.
The exception is when a blow by any weapon
exceeds the remaining fatigue of the ice; in this
case all damage is applied (the ice breaks). Successful elimination of the ice’s fatigue makes a

78

human sized hole in the ice, big enough for a one
hex entity to go through. Multiple entities may
attempt to break through the same area of ice,
within reason. Smaller holes may be made but for
the purposes of simplicity are no quicker to make
(when digging far into the ice at least a human
sized hole must be made in order to keep working
anyway).
Travel on Ice
Bipeds must travel at half TMR (round up) while
on ice or make a 2 × AG roll each pulse or go
prone. Quadrupeds may make a 4 × AG roll to
travel at full speed, or go prone. Subtract 1 × AG if
the ice is wet. Entities going prone may also slide
up to half their TMR along the ice in the pulse in
which they fell, depending on how much of their
movement was in one direction (GM’s discretion).
Travel on or through Snow
Travel through snow usually causes entities to lose
1/4 of their TMR per foot of powder, down to a
minimum TMR of 1 (unless the powder is higher
than the entity is tall). In addition there is an encumbrance shift of one column on the encumbrance table for each 1/4 TMR slowed. Note the
reference to powder — if the snow is denser than
freshly fallen powder the effects may be less.
These figures are based on a human sized entity; a
giant (for example) would only suffer the effects of
4 foot of powder (TMR reduced to 1) when up to
its chest; that is, in about 13 feet for a Stone Giant
(normally 20 feet tall). Quadrupeds tend to manage
snowy conditions well, and should be treated as if
they were standing erect for the purposes of height
calculation. Hobbits, and other creatures with large
feet, may have these penalties halved at the GM’s
discretion. GM’s may also allow items such as
snow shoes; a suggestion is a straight halving of
TMR but the entity does not sink into the snow.
Note that such items are not known or common in
warmer climes!

23 COLLEGE OF WATER MAGICS

23 The College of Water Magics (Ver 1.3)
The College of Water Magics is concerned with the
shaping of the element of water. Water Mages, as
Adepts of this college are known, have difficulty in
practising their magics on land without an abundant source of their raw material and thus the college has fallen somewhat into disuse in many locales. The typical vision people have of Water
Mages is of an Adept standing on the sea or guiding a ship around reefs to a safe harbour. While it
is true that some Water Mages do perform such
duties on ships, the majority either live beside large
bodies of water or in more recent times have taken
to living entirely underwater, thus spending all of
their time in a different world from that experienced by normal people. In general, Water Mages
enjoy good relations with other colleges with the
notable exceptions of the elemental College of
Fire, due to its opposing elemental nature.
Traditional Colours
The traditional colours worn by Water Mages have
tended towards the greens and blues that symbolise
the sea. Aquamarine and sea blue are particularly
popular amongst those who wish to advertise their
college. The hermitic type of Water Mage who
spends the majority of their time underwater usually wear very little, a small bathing suit at the
most, as most denizens of the deep are not concerned about nudity.
Traditional Symbols
Water Mage adornments are usually unique items
found only upon the sea floor such as rare shells or
pieces of coral, also wave, fish and whale motifs
are popular or any animal or occurrence specific to
the sea they live in.

23.1 Restrictions
Adepts of the College of Water Magics may only
practice their arts if they are in contact with, or
near, water. With the exception of Dowsing, they
may never practice their arts in a vacuum or in a
totally arid place. They may not summon waterdwelling creatures into an area that does not contain a body of water large enough for the waterdwelling creature to immerse itself entirely. They
may use their magic on land (in a non-arid area)
but suffer some diminution in their abilities. A
character must have 18 MA to become a water
mage.

23.2 Base Chance Modifiers
The Base Chance of performing any Talent, Spell
or Ritual of this College is modified by the addition of the following numbers. Apply one of the
following:
Adept is completely immersed in water
+20
Adept is in physical contact with water
+15
Adept is within 25 feet of water
+10
Adept is within 500 feet of water
+5
Adept is surrounded by mist or fog, or is
+5
standing in the rain
Adept is on land and over 1 mile from
-5
water
Adept is on land and over 10 miles from
-15
water
This modifier may be applied in conjunction with
one of the above:
All water in the vicinity of the Adept is
frozen

20

Notes
1. For the purpose of these modifiers “water” refers
to a body of water such as a sea, lake, ocean, pond,
river, stream, spring or other feature containing
large amounts of water (1000 gallons or more) or
existing as part of a larger system or network of
waterways. A barrel, bucket, or tun of water does
not qualify as a body of water.
2. Apply the most advantageous modifier from the
first group in conjunction with the frozen water
modifier.

23.3 Talents
Predict Weather (T-1)
Range: 10 miles (+10 / Rank)
Experience Multiple: 75
Base Chance: MA + 5% / Rank
Effects: The Adept predicts with some accuracy
what the weather will be like over the ensuing
three days in the area within the range of the talent.
The Adept must be at sea (on an ocean, sea or large
lake) or within 15 miles of an ocean, sea or large
lake. If the Adept makes a successful Predict
Weather check, the GM rolls D100 and checks the
accuracy of the prediction. The following results
may occur:
Dice

Accuracy

Totally wrong (opposite of the prediction
occurs)
Generally incorrect (fairly wide diver06–10
gence)
Generally correct (close but not totally
11–85
accurate)
Almost totally accurate (say within 1
86–
degree)
100
Following the dice roll the GM delivers the prediction as though it were generally correct.
01–05

Dowsing (T-2)
Experience Multiple: 75
Base Chance: 2 × MA + 4% / Rank
Effects: The Adept is able to sense the presence of
water above ground, and within 100 feet (+ 100 /
Rank). The Adept can determine the direction and
distance to the nearest source of water. Detection
of underground water has a range of 1 feet (+ 2 /
Rank). Assessment of freshness increases with
increased Rank, e.g. Rank 3 can tell salt from
fresh, Rank 6 can tell alkali from clear, etc. This
talent may be exercised when the Adept is in a
vacuum or a totally arid place. In general the GM
will roll to determine the success of this talent.
Aquatic Affinity (T-3)
Experience Multiple: 100
Effects: Adepts of the College of Water Magic
have a kinship with their element which gives them
the following abilities:
1. The Adept may modify their swimming rolls by
4% for every Rank they have achieved with this
talent.
2. Above Rank 10, the Adept is considered to be an
Aquatic entity for the purposes of calculating underwater defence.
3. Movement in water is considered to be one row
less difficult on the Encumbrance table. If the
Adept has more than Rank 10 in this talent then it
is two levels less difficult.
4. General knowledge of the Aquatic environment.

23.4 General Knowledge Spells
Buoyancy (G-1)
Range: Touch
Duration: 30 minutes + 30 / Rank
Experience Multiple: 100
Base Chance: 45%
Resist: None
Storage: Investment, Potion
Target: Entity
Effects: This spell allows the target to float at
whatever depth they choose. If they want to float
upon the surface this spell enables them to do so
without using any float actions. The rate of ascent
or descent is 5 feet (+2 additional / Rank) and this
spell also protects the target from any pressure
effects related to being at extreme depths.
Cold Resistance (G-2)
Range: Touch
Duration: 1 hour + 1 / Rank
Experience Multiple: 150
Base Chance: 40%
79

Resist: None
Storage: Investment, Potion
Target: Entity
Effects: This spell protects the target from the
effects of temperature down to 0°C - 2 / Rank. It
will totally protect the target from the effects of
Hypothermia. In addition, the target takes 1 (+1/5
Ranks (round up)) less damage from magical or
non-magical cold-based attacks.
Create Fog (G-3)
Range: 20 feet + 20 / Rank
Duration: 6 minutes + 6 / Rank
Experience Multiple: 100
Base Chance: 20%
Resist: None
Storage: Investment, Ward, Magical Trap
Target: Volume of Air
Effects: This spell allows the Adept to create 1000
cubic feet (+500 / Rank) of fog entirely within
range. The fog must be continuous and connected
to the surface above which it is conjured. In addition the conjured mist has the effect that all entities
within the fog have their Strike Chances reduced
by 5% (+1 / Rank) due to limited visibility. Visibility is reduced to 1 hex. The spell of Water Breathing completely negates the effect of this spell.
Mage Current (G-4)
Range: 10 feet + 10 / Rank
Duration: Concentration: maximum 30 minutes +
30 / Rank
Experience Multiple: 125
Base Chance: 30%
Resist: None
Storage: Investment
Target: Water
Effects: The Adept creates a current in the water
within the range (a volume that moves with the
Adept) of 5 miles per hour (+1 / Rank) in one
direction. All objects/entities, etc. within the range
that are submerged in the water or floating on the
water are carried at the speed of the current. The
Adept may freely alter the direction of the mage
current during the spell (requires active concentration). Passive concentration is required to maintain
this spell.
Navigation (G-5)
Range: Self
Duration: 1 hour + 1 / Rank
Experience Multiple: 125
Base Chance: 15%
Resist: None
Storage: Potion
Target: Self
Effects: The spell attunes the Adept’s mind to the
sea and winds, allowing them to sail with less
chance of mishap. This decreases the chance of
veering off course, running aground, etc., by 5% (+
1 / Rank). In addition the Adept may add 1 (+ 1 / 4
full Ranks), to their effective Rank in the Navigator Skill.
Rehydration (G-6)
Range: Touch
Duration: Immediate
Experience Multiple: 350
Base Chance: 30%
Resist: None
Storage: Investment, Magical Trap
Target: Object
Effects: This spell rehydrates (restores dried substances to their normal water content) including
dehydrated entities.
Saturated Earth (G-7)
Range: 100 feet + 50 / Rank
Duration: 1 hour + 1 / Rank
Experience Multiple: 250
Base Chance: 25%
Resist: None
Storage: Investment, Ward
Target: Area

23 COLLEGE OF WATER MAGICS
Effects: The Adept summons water from deep
within the earth, rising to the surface in 100 seconds (-5 / Rank), causing the ground in an area of
100 feet (+10 / Rank) radius to become totally
sodden. Earth will turn to mud. Sand has a 30%
chance of turning into quicksand (swallowing any
trapped individuals in 20 seconds) and a 70%
chance of turning into firm hard sand facilitating
easy movement. The Adept may alter these
chances by 1% per Rank. Wetlands will be unaffected by this spell, as will solid rocky terrain.
With this spell, all wells in an area can be caused to
fill and all crops will be well watered.
Ship Strength (G-8)
Range: 90 feet + 15 / Rank
Duration: 5 days + 1 / Rank
Experience Multiple: 200
Base Chance: 20%
Resist: None
Storage: Investment
Target: Ship, boat, wood
Effects: This spell may be used to strengthen the
structure of any ship or boat, repair leaks or holes,
restep masts, etc. At Rank 10 and above, a small
sail craft can be constructed out of available wood.
At Rank 15 or above, a large craft (40 feet or more
in length) may be constructed. At the end of the
spell’s duration, its effects are undone; a constructed craft will fall apart (even in mid-ocean). In
addition, any craft which is affected by this spell
has its chance of encountering sea monsters, pirates
or typhoons, decreased by 5% (+1 / Rank).
Speak to Aquatic Life (G-9)
Range: 15 feet + 15 / Rank
Duration: 1 hour + 1 / Rank
Experience Multiple: 150
Base Chance: 40%
Resist: None
Storage: Potion
Target: Self
Effects: This spell allows the Adept to communicate with any one form of aquatic life within range.
This communication usually consists of both
speech and gesture. All members of the selected
generic type of aquatic that are within speaking
range can be understood and can understand the
Adept. The Adept may have more than one spell on
at once, as long as they are for different generic
types.
Summon Aquatic Life (G-10)
Range: Unlimited
Duration: Immediate
Experience Multiple: 100
Base Chance: 20%
Resist: None
Storage: Investment, Potion
Target: Aquatic Species
Effects: The Adept may summon 1 aquatic creature
(+ 1 / Rank). It may take up to 30 minutes (30
seconds / Rank) for the aquatic life forms to arrive.
The species must be native to the area to be summoned and the Adept must be within 100 feet of a
body of water.
Water Breathing (G-11)
Range: Touch
Duration: 1 hour + 1 / Rank
Experience Multiple: 100
Base Chance: 25%
Resist: None
Storage: Investment, Potion
Target: Entity
Effects: This spell forms a set of gills in the subject’s neck and covers their eyes with a transparent
film. This allows the target to breathe and see
equally well under water as on land. The target
may cast spells subject to the restrictions of their
College. The spell does not affect the target’s ability to operate on the surface. This spell allows
vocal communication to a range of the target’s
perception in hexes.
Water Creation (G-12)
Range: Touch

Duration: Immediate
Experience Multiple: 150
Base Chance: 35%
Resist: None
Storage: Investment
Target: Water
Effects: The Adept must touch the substance from
which water is to be drawn. The spell allows the
Adept to extract moisture from the air, or from
plants (providing there is moisture available to be
extracted) to the amount of 1 pint (+ 1 / Rank).
Waterproofing (G-13)
Range: Touch
Duration: Special
Experience Multiple: 150
Base Chance: 30%
Resist: None
Storage: Investment
Target: Entity or Object
Effects: This spell protects any affected entity or
object completely from all forms of non-magical
water damage. It works on any item less than or
equal to 5 pounds (+ 10 / Rank) in weight. Thus,
written scrolls would not run, steel would not rust,
potions would not dilute, clothing would not get
wet etc. The duration is 5 days (+ 5 / Rank) for
objects, and 3 hours (+ 3 / Rank) for entities.
Wave Control (G-14)
Range: 90 feet + 90 / Rank
Duration: 15 minutes + 15 / Rank
Experience Multiple: 125
Base Chance: 25%
Resist: None
Storage: Investment, Ward, Magical Trap
Target: Volume of Water
Effects: The Adept can increase or decrease the
size of all waves within range by 5 feet (+ 1 additional foot / Rank).

23.5 General Knowledge Rituals
Binding Water (Q-1)
Range: 10 feet + 15 / Rank
Duration: 1 hour + 1 / Rank
Experience Multiple: 750
Base Chance: MA + 4% / Rank
Resist: Special
Effects: The Adept can bind the element of water
while maintaining their concentration. It takes an
hour to perform this ritual. The results are similar
to the binding of all other elements. The Adept
gains control of all facets of the element. They can,
for example, create an intelligent water sprite
(which will always have characteristics several
points lower than the Adept), however, its every
action would have to be directed by the Adept. At
Rank 10 or higher, the Adept can create a freewilled water sprite that will be loyal to them and
not require its every move be directed. Such entities will never leave the water. The will not be
resisted unless the area contains a Water Elemental, which may passively and actively resist.

23.6 Special Knowledge Spells
Control Aquatic Life (S-1)
Range: 10 feet + 10 / Rank
Duration: Concentration: no maximum
Experience Multiple: 100
Base Chance: 20%
Resist: Passive
Storage: Investment
Target: Aquatic Entity
Effects: This spell allows the caster to control 1 (+
1 / 2 Ranks) aquatic creatures (which must be of
the same generic type). These creatures will serve
the caster so long as the caster concentrates or until
they are told to go away (move out of range of the
spell). A creature that is no longer controlled but
still in the vicinity may attack its former master. If
the target is of a particularly small, schooling species, the Adept may be able to control the whole or
part of the school. This spell cannot be used to
control sentient creatures. Aquatic Mammals are
usually non-sentient.
80

Dehydration (S-2)
Range: 20 feet + 20 / Rank
Duration: Immediate
Experience Multiple: 500
Base Chance: 1%
Resist: Active, Passive
Storage: Investment, Ward, Magical Trap
Target: Object, Entity
Effects: The target of this spell must successfully
resist or have all of the water removed from their
body, resulting in instant mummification. The
target’s corpse will remain preserved indefinitely
(if kept dry, since little or no tissue damage occurs). After rehydration the body will be able to be
resurrected as if it had just died. If the target is
formerly or never living, the Adept receives +40%
on the Base Chance. When used in this manner the
spell can be used to dehydrate such things as food
for preservation, such things lasting indefinitely if
kept dry.
Flash Flood (S-3)
Range: 600 feet + 600 / Rank
Duration: 30 seconds + 30 / Rank
Experience Multiple: 500
Base Chance: 2%
Resist: None
Storage: Investment, Ward, Magical Trap
Target: Watercourse
Effects: The Adept causes a particular watercourse
within range to swell and burst its banks. The
watercourse can be a stream, dry river bed, small
or large river (i.e. anywhere that might be subject
to such an occurrence naturally, including drains
and sewers). The flood will occur with very little
warning. After 30 seconds of low rumblings, the
water level will suddenly rise, sweeping all before
it. The flood will wipe out any small bridges and
dams within range, wash people away, unhorse
riders, wash wagons and carts away (chance of
destruction dependent upon construction). The
effects are most noticeable on small rivers or dry
river beds. On a large river, the flood might appear
as a large wave which would look rather innocuous
at a distance but which would wreak just as much
havoc. After the spell duration expires, the water
level will drop just as quickly as it rose. All those
caught in the flow must make a successful swimming roll to avoid drowning. If a person is unhorsed they must make a horsemanship roll to stay
with their horse, in which case they may use the
horse’s Rank in swimming (generally 8) to determine drowning.
Geyser (S-4)
Range: 20 feet + 20 / Rank
Duration: 30 seconds + 30 / Rank
Experience Multiple: 350
Base Chance: 15%
Resist: Active, Passive
Storage: Investment, Ward, Magical Trap
Target: Ground
Effects: The Adept calls forth from the ground 1 (+
1 / Rank) jets of hot steam and mud. Each jet must
appear in an adjacent hex to at least one other jet.
Any Entity occupying an affected hex must resist
or suffer [D - 4] (+ 1 / 2 or fraction Ranks), half if
resisted (round down), damage, per pulse. If a
character resists they are allowed an automatic hex
of movement. It is necessary to resist for each
separate hex passed through, however only 1 resistance check is necessary for an individual hex
regardless of the duration it is occupied for. Magical Waterproofing or Protection from Normal Fire
will shield a character completely from any damage.
Liquid Purification (S-5)
Range: 15 feet
Duration: Immediate
Experience Multiple: 350
Base Chance: 30%
Resist: None
Storage: Investment
Target: Aqueous Liquid

23 COLLEGE OF WATER MAGICS
Effects: The Adept may turn 1 quart (+ 1 / Rank) of
any aqueous liquid into drinkable water.

storm front will take D10 × 3 (1 / Rank) minutes to
arrive.

Liquid Transmutation (S-6)
Range: Touch
Duration: Immediate
Experience Multiple: 350
Base Chance: 30%
Resist: None
Storage: Investment
Target: Potable water
Effects: The Adept may turn 1 pint of drinkable
water into any other common liquid of their choice.
The quality of the transformed liquid is dependent
upon Rank (Rank 0, a wine might be made that
was just drinkable, Rank 10 a Silver or Gold Medal
young wine, Rank 20, an unsurpassed wine of its
type).

Walk on Water (S-10)
Range: 10 feet + 10 / Rank
Duration: 5 minutes + 5 / Rank
Experience Multiple: 200
Base Chance: 30%
Resist: None
Storage: Investment, Potion
Target: Entity
Effects: The target is able to walk on the surface of
water as if it were solid and non-slip. They may
travel at their normal TMR. Reduce TMR by 1 for
every foot above 3 in wave height. In rough waters,
agility rolls may be deemed necessary (unless
crawling).

Maelstrom (S-7)
Range: 30 feet + 30 / Rank
Duration: 10 seconds + 10 / Rank
Experience Multiple: 500
Base Chance: 10%
Resist: Passive
Storage: Investment, Ward, Magical Trap
Target: Volume of Water
Effects: The Adept creates a horrifying watery
vortex with a diameter of 10 feet (+ 10 / Rank)
which exists entirely within the spell’s range. All
objects and entities within 20 feet of the vortex
must successfully resist or they are sucked into the
eye of the vortex and down to the sea bottom. This
spell will only be effective if cast over a large body
of water (sea, ocean or lake).
Rainstorm (S-8)
Range: 20 feet + 20 / Rank
Duration: 30 minutes + 30 / Rank
Experience Multiple: 300
Base Chance: 20%
Resist: Active, Passive
Storage: Investment, Ward, Magical Trap
Target: Entity
Effects: The caster summons a miniature rainstorm,
radius 5 feet (+ 1 / Rank), that appears over the
target’s head (little dark rain cloud, tiny lightning
bolts and all). If the target fails to resist, the storm
will follow them for the duration (even inside
buildings), completely soaking everything not
waterproof in 60 seconds (5 / Rank). The distraction caused by the storm will add 2% per Rank to
any rolls the target makes that are of a mental
nature (i.e. require thought or concentration) or 1%
per Rank to other rolls. All characters in the area of
effect will suffer these effects although the effects
are halved if they are not the actual target of the
spell. If the target is waterproofed the effects will
be minimal and if they are a Water Mage they will
suffer no ill effect, but will have their Base
Chances increased by 5%. If the Adept chooses to
cast this spell on himself the duration is increased
to 24 hours.
Storm Calling (S-9)
Range: Works at any Range
Duration: Special
Experience Multiple: 200
Base Chance: 40%
Resist: None
Storage: Investment
Target: Storm Front
Effects: The Adept may summon any storm front
which may exist anywhere in sight. Upon reaching
the spot occupied by the Adept at the time of casting, the storm front will slow and finally cease
moving and begin to downpour (snow, rain, sleet,
or whatever else the GM feels clouds may contain).
Generally a storm front can be seen for 20 to 30
miles. If no front can be seen, the spell may still be
cast but the Base Chance is reduced by 20. The

Waters of Healing (S-11)
Range: Touch
Duration: Special
Experience Multiple: 400
Base Chance: 30%
Resist: None
Storage: Special
Target: Pint of water
Effects: The Adept can turn 1 pint of water into a
half pint healing potion. This potion will immediately neutralise the effects of any venom, plus it
will cure 1 point of Endurance loss due to the
venom. Or, if there are no venoms present, the
Potion will cure [D - 5] (+ 1 / Rank) damage (Endurance then Fatigue). The Potion will last 2 minutes (+ 2 / Rank). Or the Adept may spend an hour
and utilise materials costing 200sp to make a Potion with the same effects that will last indefinitely.
The ingredients are used up regardless of success
or failure.
Waters of Strength (S-12)
Range: Touch
Duration: Special
Experience Multiple: 350
Base Chance: 30%
Resist: None
Storage: Special
Target: Pint of water
Effects: The Adept can turn 1 pint of water into a
half pint strength potion. This potion adds [D - 2]
(+ 1 / Rank) to Physical Strength for 5 minutes (+ 5
/ Rank). Potions are not cumulative. The Potion
will last 2 minutes (+ 2 / Rank). Or the Adept may
spend an hour and utilise materials costing 500sp
to make a Potion with the same effects that will last
indefinitely. The ingredients are used regardless of
success or failure.
Waters of Vision (S-13)
Range: Touch
Duration: 10 seconds
Experience Multiple: 250
Base Chance: 25%
Resist: None
Storage: Investment
Target: Pool of water
Effects: The Adept must touch a pool of water with
their hand. They may then view visions (usually
precognitive in nature) concocted by the GM. At
Rank 5 and above, they may use this technique to
spy into an area to see what is going on there. The
maximum distance from the character to the area
being spied into is 5 miles (+ 15 / Rank). The point
of view of the Waters cannot be changed. If the
Adept is not able to form a clear mental image of
the location to be spied upon or command the
waters to focus on an unambiguously defined point
in space, then it is the GM’s discretion as to what
will be seen
Waterspout (S-14)
Range: 60 feet + 60 / Rank

81

Duration: 30 seconds + 30 / Rank
Experience Multiple: 750
Base Chance: 5%
Resist: Passive
Storage: Investment, Ward, Magical Trap
Target: Volume of water
Effects: The Adept creates a tornado-like formation
with a radius of 5 feet (+ 5 / Rank), over a body of
water. The waterspout kills anyone and destroys
anything which occupies the same space with it
unless they successfully resist. The waterspout
does [D - 1] (+ 1 / Rank) damage to a character
who resists successfully instead of inflicting
enough damage to kill them. A character need only
resist the waterspout once. Boats and ships will be
affected at the GM’s discretion. Small boats may
be broken into match sticks and will almost certainly capsize, large ships may suffer less damage,
but lose rigging, masts etc. If the Adept concentrates they may move the waterspout in any direction they desire at a rate of 10 mph (+ 5 / Rank). It
may never be moved onto dry land.
Wave Riding (S-15)
Range: 10 feet + 10 / Rank
Duration: Concentration: maximum 30 minutes +
30 / Rank
Experience Multiple: 300
Base Chance: 20%
Resist: None
Storage: Investment
Target: Entity
Effects: This spell forms a wave of water under the
target controlled by the Adept, allowing travel up
to 10 mph (+ 2 / Rank). This spell will only form a
wave on a suitably sized body of water. Speed of
current will add/subtract to speed, wind will not.
The wave requires passive concentration to maintain and to keep moving in the required direction.
The wave dissipates if the Adept loses concentration. Multiple casts of this spell may be maintained
by the same passive concentration action, thus
several targets may ride the same wave (not separate ones).

23.7 Special Knowledge Rituals
Summoning and Binding Water Elemental
(R-1)
Range: 20 feet
Duration: passive concentration
Experience Multiple: 250
Base Chance: MA +5% / Rank +1% WP above 15
Resist: None
Effects: The Adept may summon a Water Elemental and bind it to temporary service by performing
this ritual. The ritual takes two hours. It may only
be performed if the summoner is in contact with, or
within 20 feet of, a large body of water (ocean, sea
or lake) and the summoner remains stationary and
takes no other action during the time the ritual is
being performed. At the end of the two hour ritual,
the player makes a Cast Check. If the ritual is
successful the Elemental is summoned and controlled. If the ritual backfires then the Elemental is
summoned but not controlled and will attack the
summoner and their friends.
A Water Elemental always appears within 20 feet
of the summoner. It has a combined Endurance and
Fatigue equal to 15 (+15 / Rank). The Elemental
will remain until it is sent back to its own dimension by the Adept (with a Special Knowledge
Counterspell of the College of Water Magics) or
banished. If it is controlled by the summoner it will
remain controlled until the summoner’s concentration is broken.

82

24 COLLEGE OF GREATER SUMMONINGS

24 The College of Greater Summonings (Ver 1.0)
This is a non-player college only. Player Characters may not learn it, transfer to it, or learn
any of the spells or rituals with the exception of
the counterspells.
The College of Greater Summonings is concerned
exclusively with the summoning and controlling of
entities from other dimensions. All such summonings and associated magical procedures are Ritual
Magic. Members of the College of Greater Summonings possess no Talent or Spell Magic as a
result of their association with the College. Their
power lies exclusively in their ability to summon
and control beings via the performance of special
rituals.
All summonings of this College, regardless of the
type of being they are designed to summon, are
performed in the same manner. First the summoner
must perform a Ritual Cleansing of their body
(requiring from 1 to 10 hours). Then they must
prepare and implement the proper Ritual of Summoning. The Ritual of Summoning itself requires
only one hour to execute. Once the Adept has
performed the Ritual of Summoning, they may
wish to control the entity they have summoned by
implementing either a Ritual of Binding or a Ritual
of True Speaking. Binding and True Speaking
Rituals each require one hour to prepare and implement. If the summoner does not implement
these rituals immediately upon successfully summoning the entity who is the target, they will have
to engage in a new Ritual of Cleansing before
attempting to enact another ritual.
When performing the summoning rituals of this
College, the summoner occupies a Circle of Protection which also contains within it a Pentacle of
Power. Unless the summoner and companions
occupy this protective area, there is a chance that
whatever they summon will be able to turn on them
and destroy them. Only the summoner occupies the
Pentacle. Their companions occupy the Circle, but
are outside the Pentacle.
The summoned entity appears in the vicinity of the
Circle of Protection, but outside it (unless the ritual
backfires and the Circle is broken). In order to
perform (and as part of the performance of) the
Rituals of True Speaking and Binding, the summoned entity is forced into the Triangle (outside
the Circle of Protection).
The Pentacle, Circle of Protection, and Triangle
must all be drawn on the ground by the summoner
before the rituals are performed.

24.1 Restrictions
Members of the College of Greater Summonings
must meet requirements of time, place, equipment, knowledge, and circumstance in order to
perform their College’s magic.
The following list of requirements must be met in
all or most situations wherein the arts of the College of Greater Summonings are practised:
1. The Adept must know the spell or ritual being
employed.
2. If a summoning of an entity other than an Incubus or Succubus is being performed, the Adept
must know the name of the entity being summoned
and speak it during the ritual.
3. Throughout all preparations for the ritual and
throughout the ritual itself, the Adept must remain
stationary. They may engage in no other activity.
4. The space used for the performance of the ritual
must be large enough to accommodate the ritual
symbols and the entity being summoned (usually,
an area 30’ × 30’ and 12’ high is adequate) and
may not have been affected by an area counterspell
in the last 12 hours.
5. If the magic is a Special Knowledge Ritual, it
may only be performed on certain days of the

month and at certain hours of the day (listed in
§24.7).

roll is made for each companion if more than one is
granted.

6. The Adept must possess the necessary tools and
equipment to perform any ritual and must have
access to such substances as clear water for the
cleansing which precedes each ritual.

24.4 General Knowledge Spells

If the Adept fails to meet any of these requirements, they may not attempt to perform any ritual
of the College of Greater Summoning. They may
still exercise any special Talent Magic they may
possess and may cast counterspells.
The MA requirement for this College is 9.

24.2 Base Chance Modifiers
The following numbers are added to the Base
Chance of successfully performing a counterspell
or ritual of the College of Greater Summoning:
Each Rank the Adept has achieved with the
+3
spell or ritual
Adept occupies a Mana-rich place
+15
The following numbers are added to the Base
Chance of successfully performing a ritual of the
College of Greater Summoning:
Each hour (maximum of 10) Adept spends in +3
Ritual Cleansing prior to performing the
ritual
Ritual is part of the College’s body of Gen+20
eral Knowledge and is begun at midnight
Ritual is part of the College’s body of Gen+20
eral Knowledge and is performed while the
moon Luna is 2, 4, 6, 8, 10, 12 or 14 days
into its cycle
Special Knowledge Rituals are affected differently
from General Knowledge Rituals so far as the hour
of the day or day of the moon in which they are
performed is concerned. The effects of the hour
and day on these rituals is discussed in §24.7.

24.3 Talents
There are no Talent Magics granted an Adept
simply because they are a member of this College.
However, there is a possibility that the Adept may
be assigned a “companion” by a demon they have
summoned. Once an Adept has been assigned a
companion, they may automatically call forth (and
later dismiss) that companion. Once called forth,
the companion instantly appears to do the bidding
of the Adept. In this sense, there is some Talent
Magic available to members of this College.
Upon dismissing a demon they have summoned,
the summoner may request that the demon assign
them a companion from among its legions of lesser
spirits. The Base Chance that a demon will grant
this request is listed under the “Lesser Spirits”
heading for each demon. The summoner rolls
D100, and if the resulting number is less than or
equal to the Base Chance, the demon grants a
companion. If the number is half or less than the
Base Chance, the demon grants two companions.
A companion will be either a lesser Devil or an
imp who is constantly “on call” to the summoner.
The companion is available until one of the following events occurs:
1. The summoner requests a companion from any
other Demon except the Demon who granted the
summoner’s current companion (regardless of
whether or not the request is granted).
2. The companion is dispelled by having a General
Knowledge Counterspell of the College of Greater
Summonings cast over it by the summoner.
3. The companion is forced into its own dimension
by a reduction in its Endurance sufficient to cause
unconsciousness.
4. The summoner voluntarily attacks the companion or quits the College of Greater Summonings.
When a companion is granted, the summoner rolls
D10. If the result is 1–4, their companion will be
an imp. If it is 5–10, it will be a devil. A separate
83

The College of Greater Summonings General
Knowledge and Special Knowledge Counterspells
are the only spells usable by members of this College. See §10.2 for descriptions of the nature and
working of counterspells.

24.5 General Knowledge Rituals
There are six rituals that an Adept learns upon
becoming a member of this College. The first is the
Ritual of Cleansing in which the Adept bathes their
body and meditates as a prelude to attempting any
other ritual. It is not possible to attempt any other
ritual of this College unless one has first engaged
in the Ritual of Cleansing. There are three summoning rituals that are part of the body of General
Knowledge of this College. They are the Ritual of
Summoning Succubi, the Ritual of Summoning
Incubi, and the Ritual of Summoning Heroes.
There are two non-summoning rituals associated
with the General Knowledge of this College besides the Ritual of Cleansing. They are the Ritual
of Binding and the Ritual of True Speaking. These
two rituals may only be employed after a summoning ritual has been successfully completed. They
are sometimes necessary to gain the required services of beings from other dimensions.
Also as part of the study of the General Knowledge
of this College, the Adept receives a parcel containing the tools necessary to their magic. The
parcel consists of the following:
Item

Weight

Value

1 sceptre of dogwood
1 lb
250sp
1 broadsword of silvered
6 lb
150sp
steel
1 mitre
1 lb
50sp
1 cap
5 lb
10sp
1 robe of virgin linen
3 lb
20sp
1 girdle of lion’s skin im3 lb
100sp
printed with symbols
1 censor of silver and gems
2 lb
3000sp
In addition to these tools, the Adept is provided
with a supply of materials which together weigh 5
pounds and which must be replenished on the
average of every three months at a cost of 1000
Silver Pennies. These items include: A packet of
charcoal, a packet of powdered agrimony (for
making tea used in the Ritual of Cleansing), a 3
ounce tin of myrrh, a vial of white frankincense
dissolved in white wine, a 6 ounce pot of sandalwood and powdered antimony, a 2 ounce tin of
ambergris salve, an 8 ounce box of multi-coloured
chalks (for drawing the Triangle, Circle, Hexagram
and Pentagram necessary for the various Rituals of
Summoning).
The Adept is also taught the Generic True Names
of all things that occupy dimensions other than
their own and may be summoned to this dimension. They also learn the Individual True Names of
all heroes of other dimensions known to the College and of the 72 great Demons of the Seventh
Plane. The Adept does not learn the rituals necessary to summon Demons from the Seventh Plane
— only the names of the Demons and how to identify them. A member of this College may learn
additional Individual True Names of entities from
other planes by first acquiring them from a Demon
and then studying the True Names in the same
manner as a member of the College of Naming
Incantations.
The following Rituals constitute the major part of
the knowledge the Adept acquires in their general
training.
Cleansing (Q-1)
Effects: The Ritual of Cleansing must be performed prior to any other rituals of this College.
The Adept cleans their body and purifies their

24 COLLEGE OF GREATER SUMMONINGS
mind for from 1 to 10 hours. The Base Chance of
any succeeding Ritual being successful is increased
by 3 for each hour spent in Ritual Cleansing. There
is no Base Chance for this ritual being successful.
The Adept states the number of hours they will
expend on the ritual and at the end of that period of
time, the Adept is cleansed. They may perform no
other activity while engaged in this ritual. If their
concentration is broken, they must restart the
cleansing from the beginning or abandon the effort.
Any rituals the Adept desires to perform after the
cleansing must be performed immediately. They
may perform any number of rituals within three
hours of the cleansing, but these rituals must immediately follow each other. Any time spent in any
other activity destroys the effects of the cleansing
and a new cleansing must be accomplished before
another ritual can be enacted.
Summoning Succubi (Q-2)
Experience Multiple: 300
Base Chance: 5% + 3 / Rank
Effects: This ritual may be used to summon one or
more Succubi, who will arrive on this plane favourably disposed towards the summoner for having summoned them, and thus will not immediately
need to be bound. Succubi arrive on this plane with
only one goal: the enticement and mating with
humans. Any deed the summoner desires of them
which will further their goal will be approved and
encouraged by the Succubi. If the Succubi are
summoned to perform a task not immediately
related to their goal, they will have to be Bound
and forced to do the task. The Base Chance is 5%
(+ 3 / Rank), and it is reduced by 10 for each Succubi above one being summoned at once. If the
ritual backfires, the Succubi will appear and attempt to molest and then devour the summoner and
his or her companions.
Succubi remain on this plane and serve for a number of days equal to [D + 4]. The die is rolled individually for each Succubus successfully summoned. The Succubi may also be forcibly returned
to their own plane whenever their Endurance is
reduced to the point that they are reduced to unconsciousness or a counterspell is cast over them
by their summoner.
Once returned to their own plane, they may not
return to the aid of the summoner unless the summoner performs another Ritual of Summoning
Succubi.
Summoning Incubi (Q-3)
Experience Multiple: 300
Base Chance: 5% + 3 / Rank
Effects: This ritual operates in the same manner of
Q-2 (the Ritual of Summoning Succubi). Incubi are
exactly like Succubi, except that they appear only
in the male form, rather than as females. They will
tend to be 2–3 points stronger and will have 1–2
points less Endurance than Succubi, but will otherwise be the same.
Summoning Heroes (Q-4)
Experience Multiple: 500
Effects: The Adept may summon a great hero from
another dimension to assist them. This hero may be
any character from the body of fantasy literature
known to both the Adept and the GM. The GM
always sets the characteristics of the hero, their
weaponry and armour, the number and type of
companions (if any) and the length of time and
terms under which they will remain in the Adept’s
dimension and assist them. The GM may limit the
use of this ritual to periods when various celestial
bodies are in conjunction (once a game year or so).
The GM need not inform the Adept of any details
concerning the results of the summoning until it
has been performed.
The Base Chance that this ritual will succeed will
vary according to the hero the Adept is attempting
to summon, but should usually be set at less than
20%.

True Speaking (Q-5)
Experience Multiple: 250
Base Chance: 50%
Effects: Whenever any entity except a hero is
summoned, the summoner may wish to ask them
questions. In all cases, the truthfulness of the entity
summoned (Incubus, Succubus, Demon) will be
less than 100%. Sometimes it will be very near
100%, but in the case of a Demon who is particularly resentful that they have been summoned (or
one that is an habitual liar) the chance of the entity
telling the truth may be as low as 5%. The only
way to be sure that the entity will tell the truth is to
perform a Ritual of True Speaking and then ask the
entity for answers to the desired questions during
the ritual. The entity is required to stand within a
triangle and give answers to the summoner’s questions during the ritual. The ritual lasts for one full
hour, and the summoner may do nothing else during the time the ritual is in progress except ask
questions. The effects of the ritual do not extend
beyond the end of the ritual. The Base Chance of
forcing the entity into the triangle and successfully
performing the ritual is 50%. The GM rolls D100
to determine the success of the ritual. They need
not inform players of the result of the dice roll. The
entity may passively resist the effects of the ritual.
Binding (Q-6)
Experience Multiple: 400
Effects: An Adept may perform a Ritual of Binding
in order to bind an entity (other than a hero) to this
plane. They cannot normally control a Demon that
is not bound, except while inside the Pentacle they
occupied when they summoned the demon. Before
they can leave the Pentacle or require any service
of the demon except for conversation and teaching
of skills, the summoner must in most cases perform
a Ritual of Binding. The Base Chance of successfully binding a particular demon is equal to half
(rounded down) the unmodified Base Chance to
summon that demon initially. The Demon can
actively resist the workings of a Ritual of Binding.
Once bound, a demon will remain on this plane for
a number of days equal to [D - 3] and will serve the
summoner more or less willingly during that time.
In some cases, special conditions must be met to
bind the demon. Usually, this means the giving of
presents, most often human life. If the demon
resists such an offering, it will have also broken the
summoning and may turn on the summoner despite
the fact that the summoner is protected by the
Pentacle. When this happens, the summoner rolls
D100. If the result is less than or equal to the summoner’s Magic Resistance, the demon is banished
to its own plane. If the result is greater than the
summoner’s Magical Resistance, the summoner is
overcome by the will of the demon and voluntarily
breaks the circle of protection allowing the demon
to enter the pentacle and attack them.
Some demons will serve more willingly and faithfully than others. The details of which demons
serve willingly and which resist service even when
bound are discussed under the heading dealing
with each particular demon. A summoner may
control only one bound demon at a time, though
they may summon other demons (or non-demonic
entities) for purposes of conversation. They may
voluntarily dismiss a demon at any time prior to
the end of its service and the demon is immediately
returned to its own dimension unless it has already
broken the binding and turned on the summoner.

24.6 Special Knowledge Spells
There are no Special Knowledge Spells for this
College.

24.7 Special Knowledge Rituals
There are six separate Special Knowledge Rituals.
Each is designed to summon a particular Rank of
demon. There are six Ranks of demons: Duke,
Prince, President, Earl, Marquis, and King. Each
Rank is subject to certain limitations as to where
and when it can be summoned.
84

All Ranks of demons are summoned in the same
manner. The summoner announces the demon they
are summoning and that demon’s Rank. They then
perform the appropriate Ritual of Summoning. At
the end of the ritual (that is after one hour), a check
is made to see if the ritual has been effective. The
summoner rolls D100. If the result is equal to or
less than the Base Chance of summoning the particular demon that is the object of the ritual, the
demon is summoned and appears before the summoner. Otherwise, the demon does not appear and
the summoner may not make a further attempt to
summon that demon that day. They may attempt to
summon another demon instead but must first
repeat the Ritual of Cleansing. The description of
each demon lists the Base Chance to summon that
demon.
Special Knowledge Rituals may only be performed
on days 2, 4, 6, 8, 10, 12 and 14 of the cycle of the
moon, Luna. They may be attempted on other days,
but can never have any effect. There is a possibility
that an Adept could lose track of time and attempt
to employ a Special Knowledge Ritual on a day
when it will not work. In such cases, the GM may
choose not to inform the individual that the ritual
can have no effect and may allow them to perform
it anyway, only telling the player why they have
been unable to summon a demon at the end of the
ritual.
An Adept who knows a particular Ritual of Summoning may summon any of the demons of that
Rank Each description includes: the demon’s
name; the Base Chance of summoning (and binding) it; the percentage chance that the demon will
agree to grant the summoner a companion from
among its legions of lesser spirits; the special talents, skills and magical abilities of the demon; a
quantification (given as a span of possible numbers) of the demon’s characteristics; the demon’s
natural armour (given under the heading NA as the
number of hits absorbed for each Strike); the natural weapons of the demon (and any other weapons
habitually carried), and any special comments on
the demon’s nature or abilities. Also included is a
short physical description of the demon.
Once a demon is dismissed (returned to its own
dimension) it cannot return to this plane in less
than a day. A demon who has been dispelled by a
counterspell or rendered unconscious may return to
this plane (by being re-summoned in a new ritual)
only after one full month in its own plane (spent
reforming the scattered energy pulses that make up
its being).
Demons may be controlled while the summoner
stands within the Pentacle which they must draw to
perform the summoning ritual. They will speak to
the summoner under this circumstance and will
sometimes tell the truth. To ensure absolute honesty, a Ritual of True Speaking must be performed.
To ensure that the demon will not destroy the
summoner once the ritual is over, a Ritual of Binding must be performed. A demon who is not bound
must be dismissed at the end of the summoning
(once the summoner has finished speaking to the
demon) and will then usually return to its own
plane. However, a demon who is particularly savage or who has been offered a gift which it rejects
will attempt to devour the summoner before departing. The summoner must then make a check
against the Willpower to determine if the summoner breaks the magical circle protecting them
and fights the demon or remains safely within the
Pentacle (in which case, the frustrated demon
departs).
Adepts are not provided with a Shield when they
are admitted to membership in the College of
Greater Summoning, as a Shield is not necessary to
perform the rituals of the College. However, they
may wish to make or have made a special Shield
(actually a disc of metal engraved with symbols of
occult power) to protect them during the summoning of demons. This shield is made by Adepts of
the College of Shaping Magics — see Arcane

24 COLLEGE OF GREATER SUMMONINGS
Wisdom. A summoner who does not have a Shield
can suffer a backfire. A summoner who is using the
proper Shield cannot suffer a backfire. A backfire
from a Special Knowledge Ritual consists of the
appearance of the demon being summoned, but
inside the circle of protection (Pentacle) so that the
demon is free to attempt to destroy the summoner.
In such cases, the demon may be returned to its
own dimension by a counterspell or by being rendered unconscious, but will otherwise remain on
this plane and freely roam about attacking and
destroying until somehow banished. It will not,
however begin roaming the earth until it has destroyed its summoner. A backfire occurs whenever
an Adept rolls a number which is 30 or more than
the modified Base Chance of summoning a particular demon while making a check to see if that
demon is summoned. A backfire is treated as “no
effect” if the summoner has the proper Shield in
their hand during the ritual.
The individual Shields that will protect a summoner from backfire are discussed under each
individual Ritual of Summoning. Shields only
affect the summoning of demons. There is no backfire due to an ineffective attempt at employing any
other Ritual of Summoning. Rituals of True Speaking and Binding may backfire (regardless of
whether or not a Shield is employed) and result in
the summoner being affected by their own ritual
and forced either to answer all questions of the
demon as truthfully as possible or to serve the
demon so long as the demon remains on this plane.
This backfire result may be passively resisted.
Note: The interaction between a player and the
“demon characters” they summon is the most important aspect of the workings of this College, and
the GM should strive to keep players on their toes
by developing the demonic character as fully as
possible (making it cooperative about some things
and uncooperative about others, for example).
Some demons are savage in the extreme and will
always be out to do what damage they can short of
murdering the summoner (and sometimes that is
not excluded). Others are milder and will pass up a
golden opportunity to devour their summoner. The
notes on individual demons are meant to serve as a
guide to their characters as well as their abilities.
Their descriptions also give the forms in which
they may appear. Often, these forms will be insubstantial and the demon will have no power (nor
will anyone have power over it) while it is in those
forms. However, the information is included as
clues for the GM in structuring the demonic character.
Summoning Demonic Dukes (R-1)
Experience Multiple: 300
Effects: This ritual is used to summon the following demons from the seventh plane: Agares, Aim,
Alloces, Amdusias, Astaroth, Barbatos, Bathin,
Berith, Bune, Crocell, Dantalion, Eligos, Furcalor,
Furcas, Gremory, Gusion, Havres, Murmur, Sallos,
Uvall, Valefor, Vapula, Vephar and Zepar. The
demons of this Rank may only be summoned between sunrise and noon on days when the weather
is clear and the sun can be seen. Any attempt to

perform this ritual at any other time will be totally
ineffective.
The only Shield that will protect against the possibility of backfire while summoning Dukes is a disk
of purest copper
12 inches across, inscribed with the names and
signs of all the Dukes of the seventh plane. This
shield may only be manufactured by a Shaping
mage. It weighs 2 pounds and the average cost of
manufacture will be 3000+ Silver Pennies. It takes
about three months to manufacture.
Summoning Demonic Princes (R-2)
Experience Multiple: 350
Effects: This ritual is used to summon the following demons from the seventh plane: Gasp, Ipos,
Orobas, Seir, Sitri, Siolas, and Vassago. These
demons may be summoned at any time of the day.
The only Shield that will protect against the possibility of backfire while summoning Princes is a
disk of hammered tin inscribed with the names and
symbols of the Princes of the seventh plane. This
Shield may only be manufactured by a Shaping
mage. It weighs 3 pounds and the average cost of
manufacture will be 3000+ Silver Pennies. It takes
about three months to manufacture.
Summoning Demonic Presidents (R-3)
Experience Multiple: 350
Effects: This ritual is used to summon the following demons from the seventh plane: Avnas, Buer,
Camio, Foras, Haagenti, Labolas, Malphas, Marbas, Volac and Voso. These demons may only be
summoned during daylight.
The only shield that will protect against the possibility of backfire while summoning Presidents is a
disk of base metal (other than cold iron) coated
with quicksilver. This shield may only be manufactured by a Shaping mage. It weighs 3 pounds and
the average cost to manufacture will be 4000+
Silver Pennies. It takes about four months to manufacture.
Summoning Demonic Earls (R-4)
Experience Multiple: 500
Effects: This ritual is used to summon the following demons from the seventh plane: Andromalius,
Bifrons, Botis, Furfur, Nlaithus, Marax, Raum and
Renove. These demons may be summoned only in
woods and hills and only in places that are quiet.
They may be summoned at any time of the day or
night.
The only shield that will protect against the possibility of backfire while summoning Earls is a disk
of hammered bronze inscribed with the names of
the Earls of the seventh plane. This shield may be
manufactured by Shaping magicians. It weighs
three pounds and the average cost to manufacture
will be 5000+ Silver Pennies. It will take about
four months to manufacture.
Summoning Demonic Marquis (R-5)
Experience Multiple: 550
Effects: This ritual is used to summon the following demons from the seventh plane: Amon, Andras, Andrealphus, Cimejus, Decarabia, Forneus,

85

Leraje, Marchosias, Naberius, Orias, Phenex,
Samieina, Savnok and Shaz. These demons may
only be summoned between 3 in the afternoon and
sunrise.
The only Shield that will protect against the possibility of backfire while summoning Marquis is a
disk of fine silver inscribed with the names of the
Marquis of the seventh plane. This shield may only
be manufactured by a Shaping mage. It weighs 3
pounds and the average cost of manufacture will be
7000 Silver Pennies. It will take about five months
to manufacture.
Summoning Demonic Kings (R-6)
Experience Multiple: 600
Effects: This ritual is used for summoning the
following demons from the seventh plane: Asmoday, Bael, Balam, Beleth, Belial, Palmon, Purson,
Vine and Zagan. These demons may only be summoned between 9 in the morning and noon and
between 3 in the afternoon and sunset.
The only shield that will protect against the possibility of backfire while summoning Kings is a disk
of hammered gold inscribed with the names of the
Kings of the seventh plane. This Shield may only
be manufactured by Shaping magicians. It weighs
three pounds and the average cost of manufacture
will be 15,000 Silver Pennies. It will take about six
months to manufacture.

24.8 Additional Notes
Hexagram
A Hexagram is a symbol drawn on a parchment of
calf’s skin, covered with a cloth of fine white linen,
and draped from the girdle of lion’s skin outside
the white robe worn by a Greater Summoner. It
helps cause the demon to take physical form and
compels them to be obedient.
Counterspells
Counterspells, as they relate to the College of
Greater Summoning, are used as a form of banishment. When a General Knowledge Counterspell is
cast over an Imp, Devil, Succubi, Incubi, or Hero
by the Adept who either was granted the companion or summoned the entity, that entity or companion is banished back to the dimension from whence
it came. A counterspell cast by anyone else will
have no effect whatsoever. If a Special Knowledge
Counterspell is cast over a Summoned Demon by
the Adept that Demon is banished back to the plane
from whence it came. It should be emphasised here
that counterspells can only be passively resisted,
and it is up to the GM to decide which Demons,
once summoned to this plane, will resist being sent
back.
Knowing the Past, Present or Future
In reference to Imps and Devils, the 7% Base
Chance of knowing past, present and future events
represents the chance they will know the correct
answer to a specific question posed to them. It is
up to the GM to perform the roll and decide what
their answer (if any) will be if the result is above
the Base Chance.

86

25 COLLEGE OF NECROMANTIC CONJURATIONS

25 The College of Necromantic Conjurations (Ver 1.1)
The College of Necromantic Conjurations is concerned with death, decay and especially with the
Undead. Practitioners of the College of Necromantic Conjurations are commonly known as Necromancers. The classic picture of a Necromancer is
one of a black-robed figure, gaunt and pale, leaning
on a gnarled wooden or bone staff, and surrounded
by their slavering Undead minions. While it is true
to say that many Necromancers will tend towards
this archetype, normally any peasant upon seeing
this apparition will arouse the local militia to lynch
said Necromancer as their very names are often
uttered in the same sentence as “Black Mage” and
“Demon” with a like response being shown to all.
As a result Necromancers will tend to neutrality in
their appearance and / or secrecy in their work,
modified by their arrogance and / or sense of style.
Others try to maintain a level of cleanliness and
hygiene not normally associated with a college
whose close ties with decay and corruption can
make things very grubby and the stench often
involved is very hard to erase. Whilst it is true that
Necromancers have no direct affiliation to the
Powers of Light or Darkness, the consecration of a
burial place severely limits the ability of the Necromancer to tamper with the deceased in any way.
Traditional Colours
Necromancers usually tend towards one of the noncolours, black for those that associate themselves
more with their often night-time activities or white
for those who wish to maintain an aloofness from
their own profession which, by its very nature is
dark and dirty.
Traditional Symbols
The College’s traditional symbol is the Grinning
Skull. Other common symbols include any easily
recognisable bone shape. Because of the sometime
secretive nature of Necromancers, given the fear
and loathing commonly shown them by the general
populace, they will tend not to openly tout their
symbols and will quite possibly use symbols which
do not directly link them to their profession.

25.1 Restrictions
Adepts of the College of Necromantic Conjurations
may practice their arts without restriction.
The MA requirement for this College is 16.
Control Limitations
An Adept is limited in the amount of Undead they
can exert control over at any one time. They have a
maximum number of bound Lesser Undead equal
to (Willpower + Rank with the spell of Binding of
Undead) + Military Scientist Rank. Bound Lesser
Undead include those made by the spells of Animation of the Dead, and Animating Bodily Parts as
well as those bound by using the spell of Binding
Lesser Undead. Non human-sized Undead will
count as more than one Undead, or a fraction of
one Undead, for the purposes of this limitation, as
per the spell of Animation of the Dead. An Adept
may also never have more than one bound Greater
Undead at any one time.

25.2 Base Chance Modifiers
The Base Chance of performing any talent, spell or
ritual of this College is modified by the addition of
the following numbers:
The Adept occupies an unconsecrated
+5%
burial place (e.g. pagan graveyard, barrow)
The Adept is standing on, or attempting
-30%
to affect, consecrated ground
During a Death Festival
+15%
During a Life Festival
-20%
All modifiers are cumulative. They are in addition
to the modifiers listed in §7.4.

25.3 Talents
Ask the Dead (T-1)
Range: 10 feet

Experience Multiple: 150
Base Chance: 20% + 4% / Rank
Effects: The Adept may, whenever they occupy the
place in which an entity has died or has been buried, attempt to communicate with the spirit of the
entity. Such communication is only possible if the
Adept could have communicated with the entity
when it was alive (knew its language, etc.). The
Adept must also be aware that the place is the site
of an entity’s death or burial. Questions put to the
entity may only be answered with a simple yes or
no, and the dead may only provide knowledge of
events which transpired while they were alive.
Once the dead have responded initially, they will
continue to answer all questions until dismissed, or
until the Adept has asked 20 ( + 5 / Rank) questions. Whether or not the Adept is successful, they
may not attempt to use this Talent again, on the
same dead entity, until 24 hours have passed.
Death Sense (T-2)
Range: 1. 50 feet + 10 / Rank; 2. 10 feet
Experience Multiple: 75
Base Chance: 1. Automatic; 2. 40% + 3% / Rank
Effects: This talent has two distinct uses:
1. The Adept will always be aware of the death of a
Sentient Entity within range, provided they are
awake and not engaged in an activity requiring
active concentration. The Adept will be aware of
the approximate distance (within 20 feet) and approximate direction (within 45 degrees) of the
demise.
2. The Adept may attempt to determine whether
the location they occupy is the site of a death or
burial. If the Adept is unsuccessful they may not
attempt to use this Talent on the same location
until 24 hours have passed.
Necrogeny (T-3)
Range: Self
Experience Multiple: 150
Base Chance: Automatic
Effects: Due to their close association with death,
decay and the Undead, Necromancers become
somewhat resistant to:
Fear Effects The Adept gains a 5% (+3 / Rank)
bonus to any Willpower check to resist fear effects,
caused by lesser Undead, hideous sights, etc. This
Talent does not aid in resisting magical fear (e.g.
Spell of Fear, Mass Fear Spell).
Infection The Adept reduces their chance of becoming infected by 5% (+ 1 / Rank).
Undead Draining Damage done to the Adept due
to the touch of a Greater Undead is reduced by 1 (+
1 per 4 or fraction Ranks).

25.4 General Knowledge Spells
Animating Bodily Parts (G-1)
Range: Touch
Duration: 60 minutes + 10 / Rank
Experience Multiple: 175
Base Chance: 30%
Resist: Passive
Storage: Investment
Target: Parts of Corpses
Effects: The Adept may instill 1 humanoid bodily
part (+ 1 / Rank) with a gruesome semblance of
life. A single corpse consists of 6 bodily parts: 2
legs, 2 arms, 1 head, 1 torso. The Adept may join
together previously unconnected parts before animation. The animated bodily parts will be able to
follow simple commands given by the Adept. The
exact mode of locomotion of the parts, and their
speed and other abilities is left to the discretion of
the GM. They will be slower and weaker than
normal Skeletons or Zombies. The exact results of
this spell are hard to predict, and even two identically constructed “animates” may not perform in
precisely the same manner.

87

Binding Lesser Undead (G-2)
Range: 15 feet + 15 / Rank
Duration: 1 hour + 2 / Rank
Experience Multiple: 200
Base Chance: 35%
Resist: Passive
Storage: Investment
Target: Lesser Undead Entity
Effects: The Adept may gain control of 1 (+ 1 per 2
or fraction Ranks) Lesser Undead, that fail to resist. If the Undead to be affected are already controlled or bound, the rank of the Adept with this
spell is compared to the rank of the original control
or binding. If the original rank is equal or higher,
the Undead are unaffected by this spell, otherwise
the Undead must resist normally. The Undead will
serve the Adept in all ways, obeying simple commands communicated verbally by the Adept. No
target may resist an order once it has failed to
initially resist the spell.
Conjuring Darkness (G-3)
Range: 15 feet + 15 / Rank
Duration: 15 minutes × [D - 5] × Rank ( × 1 if
unranked)
Experience Multiple: 100
Base Chance: 50%
Resist: None
Storage: Investment, Ward
Target: Area
Effects: The Adept creates a volume in which nonmagical light is partially suppressed. The volume
will be 1000 (+ 500 / Rank) cubic feet, and may be
in any one contiguous area the Adept desires, provided that no dimension is smaller than one foot.
The entire volume must be visible and within range
at time of casting, and may not be moved. For
visibility purposes, the Spell will increase Darkness levels within the volume to 60% + 2% / Rank.
Rank 20 Darkness may not be seen through. It will
not aid in providing bonuses for casting purposes,
though it will neutralise penalties due to natural
light, to a maximum of 5% + 1% / Rank. The volume counts as direct shadow for Star & Shadow
Mages. If the lighting conditions are lower than
that provided by the spell, no effect will be apparent. Note that because light is only being suppressed, it may still pass through, and no shadows
are generated outside the volume. If it is possible to
see through a Darkness, everything beyond it is
normally visible. This spell can engender silhouettes of lit objects against the darkness, though not
create shadows. Any of this volume may be overridden by a higher ranked Spell of Light, or neutralised (back to original conditions) by an equal
rank.
Detecting Undead (G-4)
Range: 50 feet + 50 / Rank
Duration: Immediate
Experience Multiple: 150
Base Chance: 30%
Resist: None
Storage: Potion
Target: Self
Effects: The Adept becomes immediately aware of
the class (Lesser or Greater), approximate number
(to the nearest 5), and general location (to the
nearest 20 feet) of all Undead within range. If the
Undead are within 10 feet (+ 10 / Rank) the Adept
is aware of their exact types, numbers, and locations.
Fear (G-5)
Range: 15 feet + 15 / Rank
Duration: Immediate
Experience Multiple: 250
Base Chance: 25%
Resist: Active, Passive
Storage: Investment, Ward, Magical Trap
Target: Entity
Effects: The Adept instills in the target an uncontrollable fear. Unless the target successfully resists

25 COLLEGE OF NECROMANTIC CONJURATIONS
they must roll on the Fright Table. If a double
effect is achieved, the Adept may modify the
Fright Table roll up or down by an amount equal to
the Rank of the spell. If a triple effect is achieved,
the Adept may modify the Fright Table roll up or
down by twice the Rank of the spell. See the Fright
Table for the exact results of the fear.
Noxious Vapours (G-6)
Range: 15 feet + 15 / Rank
Duration: 10 minutes + 10 / Rank
Experience Multiple: 200
Base Chance: 25%
Resist: None
Storage: Investment, Ward, Magical Trap
Target: Area
Effects: The entire area affected by the spell exudes
a charnel stench, and all entities within it, except
the Adept, must make a Will Power check or become nauseous. The Difficulty Multiplier for the
Check is dependent on the Rank of the spell:
Rank

Multiplier

0–7
4.0
8–13
3.0
14–18 2.0
19–20 1.0
Those entities who become nauseous have their
Strike Chances reduced by 10 (+ 1 / Rank), and
must make a Willpower concentration check to
utilise Spell magic. The multiplier for this check is
the same as that for resisting the nausea. In addition, this spell causes a thick, roiling white mist to
rise from the ground. The mist is 6 inches high (+ 6
inches / Rank), and reduces the range of all forms
of vision, within the mist, to 20 feet (1 foot /
Rank).
Putrescence (G-7)
Range: 15 feet + 15 / Rank
Duration: Immediate
Experience Multiple: 100
Base Chance: 50%
Resist: None
Storage: Investment, Ward, Magical Trap
Target: Volume of Food and / or Drink
Effects: This spell causes up to 1 cubic foot (+ 1 /
Rank) of food and/or drink to putrefy, spoil, decay
and rot. The foodstuffs will thereafter be unfit for
consumption. If a Double or Triple effect is
achieved, the amount of food and / or drink to be
affected may be doubled or tripled.
Putrid Wound (G-8)
Range: 15 feet + 15 / Rank
Duration: Immediate
Experience Multiple: 250
Base Chance: 20%
Resist: Active, Passive
Storage: Investment, Ward, Magical Trap
Target: Living Entity
Effects: The Adept may cause [D - 4] (+ 1 / Rank)
damage in the form of a putrid wound, to one entity within range, unless the target successfully
resists. The wound is automatically infected. This
spell will only affect living entities (i.e. it will not
affect Undead, Animates, Demons, etc.)
Rigor Mortis (G-9)
Range: 15 feet + 15 / Rank
Duration: 10 seconds + 10 / Rank
Experience Multiple: 200
Base Chance: 20%
Resist: Active, Passive
Storage: Investment, Ward, Magical Trap
Target: Entity
Effects: If the target entity fails to resist they are
affected with a corpse-like stiffening. Any attempt
to move will result in pain and accompanying
cracking noises. The victim suffers a penalty to
their Cast Chances and Strike Chances equal to 5
(+ 1 / Rank). In addition, their Agility is reduced
by 1 (+ 1 per 3 or fraction Ranks). This Agility
reduction will affect initiative and TMR.
Spectral Hand (G-10)
Range: 15 feet + 5 / Rank

Duration: 30 seconds + 10 / Rank
Experience Multiple: 200
Base Chance: 25%
Resist: None
Storage: Investment
Target: Special
Effects: The Adept conjures a giant, invisible,
spectral hand, which executes a two word command e.g. “Smash that, Lift that, etc.” The hand
may move at TMR 4, and may not leave the
Adept’s range. The hand may exert force in one
direction equal to a Physical Strength of 15 (+ 2 /
Rank). Entities or Objects caught between the hand
and an immovable object suffer 1 (+ 1 per two full
Ranks) physical damage per Pulse (armour may
reduce this). Entities may break away from the
hand by executing a Withdraw action. If seen by
Witchsight, the hand appears coldly blue and skeletal and is roughly 3 feet in diameter.
Warping (G-11)
Range: Touch
Duration: Immediate
Experience Multiple: 175
Base Chance: 30%
Resist: Passive
Storage: Investment
Target: Volume of Object
Effects: The Adept may twist and warp up to 1
cubic foot (+ 1 / Rank) of any formerly living
matter, into the shape of their choosing. The Volume may contain varied items.
Example
An Adept could warp a collection of flowers, bones, and sticks together so as to form a funeral
wreath.

The spell confers no special artisan abilities on the
Adept. Once warped the object(s) will remain in
their new shape unless remoulded.

25.5 General Knowledge Rituals
Converse with the Dead (Q-1)
Duration: Special
Experience Multiple: 200
Base Chance: MA + 4% / Rank
Resist: None
Target: Spirit
Cast Time: 1 hour
Material: Drawn Pentacle
Actions: Concentration
Concentration Check: Standard
Effects: This Ritual summons a friendly Spirit, who
will answer 3 questions (+ 1 per two full Ranks)
for the Adept. The Adept must first draw a Pentacle, and remain within it throughout the Ritual.
Each question will be answered with a riddle or
puzzle. The GM creates all riddles and puzzles in
answer to the Adept’s questions. As the Adept’s
Rank with this ritual increases, the replies will
become less cryptic and confusing. Undead may
not be summoned or conversed with using this
ritual.
Summoning Lesser Undead (Q-2)
Duration: Immediate
Experience Multiple: 250
Base Chance: MA + 3% / Rank
Resist: None
Target: Lesser Undead
Cast Time: 2 hours
Material: Drawn Pentacle
Actions: Concentration
Concentration Check: Standard
Effects: To use this Ritual the Adept must first
draw a Pentacle in or near an area where Lesser
Undead may be found (near tombs, graveyards,
barrows, etc.). The Adept must remain within the
Pentacle during the entire course of the ritual. The
ritual summons 1 (+ 1 / Rank) Lesser Undead.
Undead summoned successfully (i.e. not as a result
of a backfire) will appear within 20 feet of the
Adept’s Pentacle, and will be unable to cross into
the pentacle. The Undead are not bound or controlled in any way. If the ritual backfires the Lesser
Undead will appear inside the pentacle and will
attack the Adept.
88

25.6 Special Knowledge Spells
Agony (S-1)
Range: 30 feet + 15 / Rank
Duration: 10 seconds + 10 / Rank
Experience Multiple: 350
Base Chance: 10%
Resist: Active, Passive
Storage: Investment, Ward, Magical Trap
Target: Area
Effects: This spell causes all entities in the affected
area, except the Adept, to suffer extreme agony.
Entities who fail to resist may only take Pass actions for the duration of the spell, or until such time
as they leave the area of effect. Entities who successfully resist reduce all Strike Chances by 30,
and take twice as long to perform any action. Note
that Mind Mages gain a bonus to resist this spell
equal to 2 × Rank with their Talent of Resisting
Pain.
Animation of the Dead (S-2)
Range: 30 feet + 15 / Rank
Duration: 1 hour + 1 / Rank
Experience Multiple: 250
Base Chance: 20%
Resist: None
Storage: Investment, Ward, Magical Trap
Target: Corpses
Effects: The Adept may fill 3 (+ 1 per Rank) human-sized corpses within range, with the power of
undeath, giving them a gruesome semblance of
life. The Undead so created will serve the Adept in
all ways. The Undead will obey simple commands
communicated verbally by the Adept. Corpses that
possess most of their flesh will become Zombies,
those that are mostly devoid of flesh will become
Skeletons. If this spell is used on corpses of other
than human size, the following strictures apply:
Larger than human sized corpses count as 1 corpse
per hex. Smaller than human sized corpses count as
fractions of a corpse: Dog, 0.5, Cat, 0.2, Rat, 0.1.
No more than 10 small corpses, even if smaller
than rat sized, may be animated in the place of 1
human sized corpse. All corpses animated by a
single casting of this spell must be of the same
type. Note that if this spell is cast as the result of a
Ward or Magical Trap the reanimated dead will be
uncontrolled and will attack any living beings they
can reach.
Binding Greater Undead (S-3)
Range: 15 feet + 15 / Rank
Duration: 30 minutes + 30 / Rank
Experience Multiple: 300
Base Chance: 20%
Resist: Active, Passive
Storage: None
Target: Greater Undead Entity
Effects: The Adept may order one Greater Undead,
that fails to resist, to do anything that is within its
physical capabilities, and is not obviously suicidal.
No target may resist a valid order once it has failed
to initially resist the spell. If the Undead to be
affected is already controlled or bound, the rank of
the Adept with this spell is compared to the rank of
the original control or binding. If the original rank
is equal or higher, the Undead is unaffected by this
spell, otherwise the Undead must resist normally.
This spell does not grant the Adept any particular
ability to communicate with the target.
Bone Construction (S-4)
Range: 5 feet + 5 / Rank
Duration: 15 minutes + 15 / Rank
Experience Multiple: 250
Base Chance: 15%
Resist: None
Storage: Investment
Target: Area
Effects: The Adept may create 25 cubic feet of
interlocking bones (+ 25 / Rank) in any shape or
shapes of the Adept’s choosing. Any dimension
that is less than 1 foot is considered to be 1 foot for
the purposes of computing volume. The bones
always appear entirely within range of the Adept

25 COLLEGE OF NECROMANTIC CONJURATIONS
and may not appear on top of, or inside (partially
or wholly), any entity. The bones become increasingly strong with higher Rank:
Rank

Strength

0–5
6–10
11–15
16–20

Bone
Wood
Bronze
Iron

Dark Vision (S-5)
Range: 15 feet + 15 / Rank
Duration: 1 hour + 1 / Rank
Experience Multiple: 100
Base Chance: 60%
Resist: None
Storage: Investment, Potion
Target: Entity
Effects: The Adept causes the target to develop
excellent vision in the dark. Everything will appear
monochromatic (ie shades of grey) and it is difficult to accurately estimate distance. The higher the
Rank, the less of a problem this will be. Some
amount of light must be present for this vision to
operate. The range of the vision is 50 feet (+ 10 /
Rank).
Hand of Death (S-6)
Range: 15 feet + 15 / Rank
Duration: 5 seconds + 5 / Rank
Experience Multiple: 250
Base Chance: 20%
Resist: Active, Passive
Storage: Investment
Target: Entity
Effects: The target suffers [D + 1] damage each
pulse that the Adept takes a Pass action and makes
visible squeezing motions with their hand to simulate the squeezing of the victim’s heart. Only at the
time of casting, and on subsequent pulses when the
Adept takes a pass action, must the target be in the
Adept’s line of sight. On Pulses that their heart is
squeezed the target suffers extreme pain, and may
only perform Pass actions.
Life Draining (S-7)
Range: Touch
Duration: Special
Experience Multiple: 300
Base Chance: 15%
Resist: Special
Storage: Potion
Target: Self
Effects: The Adept’s hand becomes charged for 5
seconds (+ 5 / 4 or fraction Ranks) and will drain 1
(+ 1 / Rank) Fatigue from the next entity to be
touched, if the target fails to resist. This thereby
discharges the spell. The Fatigue may be used to
repair the Adept’s own Fatigue and/or Endurance.
If the target has no Fatigue remaining the Adept
may drain from Endurance instead. The Adept may
not drain from both Fatigue and Endurance with
one cast of the spell. Note that only living entities
will be affected by this spell and the Adept may not
drain themselves.
Mass Fear (S-8)
Range: 30 feet + 15 / Rank
Duration: 30 seconds + 10 / Rank
Experience Multiple: 400
Base Chance: 10%
Resist: Passive
Storage: Investment, Ward, Magical Trap
Target: Area
Effects: The spell instills in all entities within
range, other than the Adept and those who successfully resist, an unreasoning and uncontrollable fear.
All entities that fail to resist must roll on the Fright
Table (see 54.1).
Necrosis (S-9)
Range: 15 feet + 15 / Rank
Duration: Immediate
Experience Multiple: 450
Base Chance: 5%
Resist: Active, Passive
Storage: Investment, Ward, Magical Trap

Target: Entity
Effects: This spell causes 1 target for every 3 or
fraction Ranks to suffer [D + 1] (+ 2 / Rank) damage in the form of internal haemorrhaging and
rotting. If a target resists, they suffer only half
damage (round up). Wounds inflicted by this spell
will automatically be infected.
Note that only living entities will be affected by
this spell.
Petit Mort (S-10)
Range: Touch
Duration: Variable
Experience Multiple: 250
Base Chance: 30%
Resist: Active, Passive
Storage: Investment, Potion
Target: Entity
Effects: By means of this spell, the Adept suspends
all of the target’s bodily functions, causing them to
take on the semblance of death. The target will
have no discernible signs of life, even so far as to
appear dead to a Healer using empathy. The target’s body will slowly cool to room temperature.
At Rank 10 or below, the target’s aura will still
show them to be living, but at Rank 11 or greater,
they will detect as “Formerly Living”. A Healer
will only be able to detect that the target is not
truly dead if they attempt to either Preserve Dead
or Resurrect. The target’s body will require no
sustenance of any sort, nor will it decay or age. The
duration of the spell must be decided by the Adept
at the time of casting, up to a maximum of:
Rank

Duration

0–3
1 day
4–8
1 week
9–11
1 month
12–16 6 months
17–19 1 year
20
Any duration
Note that if the spell is made into a Potion, the
target of the spell is the imbiber, and they may only
passively resist the spell’s workings.
Scarring Terrain (S-11)
Range: 50 feet + 25 / Rank
Duration: 1 month + 1 / Rank
Experience Multiple: 225
Base Chance: 20%
Resist: None
Storage: Investment, Ward
Target: Area
Effects: The Adept causes terrible ruin to all
ground within range. The ground will be so damaged that it will be unable to support any flora for
the duration of the spell (grass will turn to dust,
small shrubs will shrivel, trees will lose all foliage
and slowly die). Though the spell will wither any
flora in the area, fauna will be unaffected.
Spectral Warrior (S-12)
Range: Sight
Duration: Concentration: No maximum
Experience Multiple: 400
Base Chance: 5%
Resist: None
Storage: Investment
Target: Entity
Effects: The Adept conjures to this plane a spectral
warrior, and directs the warrior to hunt down and
slay one target, who must be within sight when the
spell is cast. The warrior is completely insubstantial and invisible except to its intended victim. It
may be seen by others with Witchsight. It appears
as a glowing spectre in baroque armour, with piercing red eyes. The warrior will continue with its
mission until the Adept’s concentration is broken,
its intended victim dies, or it is dissipated or slain.
The warrior can unerringly locate its intended
victim and will always move towards them at its
full movement rate, and engage them in melee
combat. The warrior has a single, combined, Endurance and Fatigue Characteristic with a value of
20 (+ 5 per 3 full Ranks). The spectral warrior
automatically hits every Pulse for [D - 4] (+ 1 /
89

Rank) damage. Its Initiative is 30 (+ 2 / Rank). The
warrior has no defence or armour value. It may not
be Stunned. The warrior’s movement rate is 650
yards per minute, and its TMR is 13. The warrior
may move in any direction without restriction,
including through the air, walls, water, etc., except
through the area of a Necromantic Special Counterspell. The warrior may be dissipated by having a
Necromantic Special Counterspell cast on the area
it occupies.
Spectral Weapon (S-13)
Range: 5 feet + 5 / Rank
Duration: 5 minutes + 1 / Rank
Experience Multiple: 250
Base Chance: 15%
Resist: None
Storage: Investment
Target: Object
Effects: The Adept may increase the usefulness
of any weapon within range. The weapon begins to
faintly glow with a cold, blue light. The weapon
has its Base Chance increased by 1 ( + 1 / Rank),
and the damage increased by 1 per 3 or fraction
Ranks. In addition, the weapon may affect targets
that are insubstantial, such as Spectres, Spectral
Warriors, etc.
Stream of Corruption (S-14)
Range: 30 feet + 5 / Rank
Duration: Immediate
Experience Multiple: 350
Base Chance: 5%
Resist: Passive
Storage: Investment, Ward, Magical Trap
Target: Area
Effects: From the Adept’s fingertips erupts a column of rotting blood, lacerating bone shards, maggots, and corrosive pus, which travels to the extent
of the spell’s range, and is 5 feet wide. The Adept
may increase the width by 1 foot per Rank. All
entities occupying hexes through which the stream
passes must resist or suffer [D - 2] (+ 1 / Rank)
damage. Living entities who are damaged by this
spell have their Base Chance of Infection increased
by 20 (+ 4 / Rank). The putrid matter remains for
about a minute after the spell is cast.
Wall of Bones (S-15)
Range: 15 feet + 15 / Rank
Duration: 10 minutes + 10 / Rank
Experience Multiple: 150
Base Chance: 30%
Resist: Passive
Storage: Investment, Ward, Magical Trap
Target: Area
Effects: The Adept conjures a wall of interlaced
bones, 1 foot thick, 10 feet high, and either 20 feet
long, or in a circle with a diameter of 10 feet. The
Adept may increase height, length, or diameter, by
1 foot per Rank. Every time an entity comes in
contact with the wall, they must passively resist or
suffer a roll on the Fright Table. The wall can
withstand 100 points of damage before crumbling
to dust. “A” Class weapons cannot damage the
wall.
Wraithcloak (S-16)
Range: Self
Duration: 30 minutes + 30 / Rank
Experience Multiple: 200
Base Chance: 15%
Resist: None
Storage: Potion
Target: Self
Effects: The Adept becomes shadowy and less
corporeal, becoming increasingly ethereal with
Rank. 2% (+ 3 per two full Ranks) is added to their
defence, and if they are struck by weapon that is
not magical or silvered, 1 point per three full Ranks
is subtracted from the damage. In addition, the
Adept gains a 1% (+ 1 / Rank) bonus to their
Stealth.

25 COLLEGE OF NECROMANTIC CONJURATIONS

25.7 Special Knowledge Rituals
Becoming Undead (R-1)
Duration: Permanent
Experience Multiple: 250
Base Chance: 10% + 3% / Rank
Resist: Active, Passive
Target: Entity
Cast Time: 2 hours
Material: Drawn Pentacle, Sacrifice
Actions: Ritual Sacrifice
Concentration Check: None
Effects: The Adept may transform a sentient entity
into an Undead by employing this ritual. Once the
entity joins the ranks of the Undead the effects may
not be reversed. The Adept must draw a Pentacle
and remain within it throughout the ritual, along
with the entity to be transformed (who may be the
Adept themselves) and an entity to be sacrificed.
During the course of the ritual, the Adept must
sacrifice an entity of the same race as the entity to
be transformed. The sacrifice may be bound, but
must be conscious during the ritual. Upon the
performance of sacrifice, a Ghost appears over the
corpse. This Ghost becomes chained to the place in
which the ritual was performed until freed by the
death of the Adept. Once the sacrifice is performed, the Adept’s Player checks to determine if
the ritual was a success. If it is a success, the entity
to be transformed joins the ranks of the Undead as
the Greater Undead type of the Adept’s choice. If
the ritual backfires, the Ghost is transformed into a
Wraith, who may immediately attack the Adept.
Greater Undead created by this ritual retain all of
their prior knowledge and magical abilities. They
will also retain those physical abilities as may be
used by their new form.
Life Prolonging (R-2)
Duration: Special
Experience Multiple: 350
Base Chance: 5% + 5% / Rank
Resist: None
Target: Entity
Cast Time: 8 hours
Material: None
Actions: Concentration
Concentration Check: Standard
Effects: The Adept may prolong an entity’s life,
including their own, causing them to remain unchanged and unaging for the duration of this ritual.
Once the effects of the ritual wear off, the target
will begin to age at the rate of 1 year per minute
until they reach the proper biological age to match
their chronological age. Normally, the ritual’s
effects last for 2 years (+ 2 / Rank), but at Rank 10
and above, the Adept can immediately upon the
ritual being completed reduce the target’s biological age by up to 1 year (+1 / Rank) in addition to
halting aging. Any backfire causes accelerated
aging (this effect may be passively resisted).
Permanency (R-3)
Duration: Permanent
Experience Multiple: 350
Base Chance: MA + 4% / Rank
Resist: None
Target: Undead, Bones
Cast Time: 1 hour
Material: Precious Ingredients

Actions: Anointing Undead or Bones
Concentration Check: None
Effects: The Adept may use this ritual to increase
the duration of the animation of any Lesser Undead, or any Bone Construction, created by them to
“Permanent”. This can be applied to the spells of
Animating Bodily Parts, Animate Dead, and Bone
Construction. The ritual may only be performed by
the same Adept who made the animate or bones to
be affected. The Adept must first cast the spell that
they wish to make permanent and then immediately begin this ritual. A spell that has been enhanced by the use of this ritual may not be Counterspelled, but may still be Dissipated. The Adept
must expend precious ingredients to perform this
ritual. The ingredients will cost 300sp (5 / Rank),
per cubic foot of bones or Undead to be affected.
Any dimension that is less than 1 foot is considered
to be 1 foot for the purposes of computing volume.
A standard human-sized Undead will be about 12
cubic feet.
Shaping Flesh Golems (R-4)
Duration: Permanent
Experience Multiple: 500
Base Chance: Special
Resist: None
Target: Parts of bodies
Cast Time: 24 weeks
Material: Parts of corpses & (20,000 - 900 / Rank)
Silver Pennies
Actions: Constructing Golem
Concentration Check: None
Effects: This ritual is used to fashion a Golem from
parts of different corpses. All pieces that are to go
into the Golem must be found before the ritual can
begin. In addition to the pieces of corpses and
precious ingredients, the services of a Taxidermist
or Undertaker of at least Rank 6 are needed for this
ritual. Once the ritual is complete the Flesh Golem
will become animate and will remain active until it
is killed. A Flesh Golem will always obey its creator and is barely sentient, but the GM must determine the complexity of commands it is capable of
executing based on the intelligence of the brain
used to fashion the Golem. This ritual is automatic,
but there is a 1% chance of an accident each week
during a Flesh Golem’s creation. It must be noted
that the creation of a Flesh Golem is an inexact
science, and no two Flesh Golems seem to turn out
the same. Even building a second Golem using
identical materials rarely results in the same final
creature. To perform this ritual the Adept will
require access to an Alchemist’s laboratory or a
Binder’s workshop. Flesh Golems have the following characteristics:
Description A Flesh Golem can be put together
from different creatures within a species, or from
any number of different species. For example, a
troll could be fashioned with the brains of a human,
and the wings of a gargoyle, or an ogre could be
made with the strength of one ogre (presumably a
powerful one), the endurance of another hardy one,
and the brain of a bright ogre. Seams will show
wherever parts are sewn together, and at low Ranks
the Golem will exude a horrible, rotting odour. By
Rank 6, the odour will only be noticeable in very
close Proximity to the Golem, and above Rank 10

90

the odour will only be noticed by canines and other
creatures with an exceptional sense of smell.
Talents, Skills and Magic Flesh Golems never
retain any of the skills or spell casting abilities
(although they do retain some weapon skills) of
any of the creatures from which they are made, but
do retain any talents appropriate to the parts of a
creature used. Thus, a Golem made with the eyes
of a basilisk would be able to turn creatures to
stone in the same manner as a basilisk.
Movement Rate (yards per minute): Variable,
according to the Golem’s size and method(s) of
locomotion.
PS Average the Physical Strength of the creatures
from which the Golem’s muscle tissues were taken.
MD Average the Manual Dexterities of the creatures from which the Golem’s muscles and brain
were taken.
AG Average the Agilities of the creatures from
which the Golem’s muscles and brain were taken.
MA 0
EN Average the Endurances of the creatures from
which the Golem’s muscles and internal organs
were taken.
FT As for Endurance, but the creature from which
the Golem’s lungs were taken counts twice within
the average.
WP Add 5 to the Willpower of the creature from
which the Golem’s brain was taken. Maximum 30.
PC Variable, depending on the quality of the eyes,
ears, and mental faculties of the creatures that were
used, but never more than 10 (+ Rank / 2, rounded
up).
PB Variable, depending on the creatures used, but
never more than 6 (+ Rank).
TMR Variable depending on Agility and GM’s
discretion.
NA Variable, depending on the creature from
which the skin was taken.
Weapons A Flesh Golem can use any natural
weapons (such as claws and fangs) built into it, as
well as any weapons it can carry. It will have the
same Ranks in weapons as the creature who contributed its brain divided by 2 and rounded down.
This is knowledge only, and there is no guarantee
that the Golem will have the requisite PS or MD,
or will even be able to manipulate weapons.
Comments The GM will have to use discretion in
allowing combinations of creatures. It would be
absurd to allow a Golem to be built with a dragon’s
head on a human body – due to discrepancies in
size, for instance. Flesh Golems can be harmed by
weapons or magic as per normal.
Summoning Greater Undead (R-5)
Duration: Immediate
Experience Multiple: 350
Base Chance: MA / 2 + 3% / Rank
Resist: Passive
Target: Greater Undead
Cast Time: 2 hours
Material: Drawn Pentacle
Actions: Concentration
Concentration Check: Standard
Effects: This ritual works in the same manner and
under the same conditions as the Ritual of Summoning Lesser Undead (Q-1) except that the Ritual
summons only one Greater Undead.

26 COLLEGE OF RUNE MAGICS

26 The College of Rune Magics (Ver 2.2)
College in Playtest
The Rune College is currently in play test, and
this is the test version in use at time of publication. Significant changes from this version are
expected to happen at irregular intervals. Check
http://www.dq-nz.org/dqwiki/index.php?title=Rune
for the latest version.
All characters that join this College do so under
the understanding that it may be withdrawn or
radically changed. Contact a member of the
Character tribunal for advice before taking this
College.
The College of Rune Magics is concerned with the
use of special symbols of power to shape mana into
desired forms. A Rune is a graphic symbol representing some actual, elemental, or mystical force.
In rare cases, additional Runes may be developed
or discovered which employ parts of existing
Runes. However, much of the power of the Runes
derives from their constant usage over many centuries, and most useful Runes will be known to all
Adepts of this College (or at least be readily available to them with very little research). It is believed
that the origins of Runes come from the original
written script of the dragons. As the dragons investigated the world they attempted to codify this
knowledge as written symbols. Ages later early
mortals discovered fragments of these writing.
From these discoveries, prehistoric shamans developed primitive magic to give them simple power
over the world around them.

26.1 Totem Animals
As shamans, primitive Rune mages often chose
totem animals to aid and guide them. This binding
of their spirit to that of their totem enable them to
sense when their totem animal is in the vicinity and
their totem will never make an unprovoked attack
on the Adept. Using the Spell of Summon Totem
Helper the Adept may gain assistance from the
Totem spirits.

26.2 Rune Wands and Staves
A wand is defined as a length of wood or bone one
foot long. It cannot be used in combat. It has negligible weight. A staff is defined as a quarterstaff in
terms of weight, length and damage.
Materials
Material Area
Willow
Poplar
Bone/Ivory
Pine
Elm
Beech

Bonus (+5% to)

Healing
Divination
Control
Creation
Warning
Spirit

healing spells
runes of sight
control spells
rune wall, weapon
purification, warding
spirit spells (-5% to
others)
Oak
Strength
stores extra Ft
Ash
Destruction elemental
Blackthorn Curse
curses
Redwood Travelling sending, visitation etc.

26.3 Restrictions
Adepts of the College of Rune Magics may use
their talent magic without restriction. Many spells
require inscribing the appropriate Rune on a surface or item to be enchanted. This location is indicated (as ‘Rune:’) in the spell’s description, and
full details are given under the spell’s ‘Effects:’.
In order to write the Rune, the Adept may use any
substance that will mark the surface of the object to
be enchanted. Any tool may be used to carve a
Rune into a substance, so long as the tool is hard
enough to do the job and it is not composed of
Cold Iron.
The MA requirement for this college is 14.

26.4 Ritual Casting
Some spells may be ritually cast. The spell is cast
as a ritual, taking at least one hour. The adept

spends the same amount of fatigue as they would if
the spell was cast normally.

1 hour (+ 1 / Rank), unless it is ritually cast, when
it lasts for 1 day (+ 1 / Rank).

26.5 Base Chance Modifiers

Darkness Rune (G-2)
Range: 5 feet + 1 / Rank
Duration: 15 minutes + 15 / Rank
Experience Multiple: 100
Base Chance: 40%
Resist: None
Target: Point
Rune: Object/Runestaff
Effects: The Adept creates a volume which elemental darkness fills like fog. The volume is a
sphere with a radius equal to the spells range,
centred on the Rune drawn by the caster and may
not be moved unless the Darkness Rune is inscribed on the Adept’s Runestaff.

The Base Chance of performing a talent, spell, or
ritual of the College of Rune Magics is modified
by the addition of the following numbers:
Adept takes a minute to inscribe a
Rune on a surface
Adept uses their own blood to inscribe
a Rune on a surface (1 pt tiredness FT,
minimum 1 minute)
Adept employs Ritual Spell Preparation or Casting (maximum 10 hours)
Adept uses fresh Dragon’s blood to
inscribe a Rune
Adept uses a wand or staff

+5
+5

+5 + (5 /
hr)
+50
as material table

All modifiers are cumulative.

26.6 Talents
Interpret Runes and Symbols (T-1)
Range: 5 foot + 1 / Rank
Experience Multiple: 150
Base Chance: MA + PC + 3% / Rank
Effects: This talent allows the adept to divine the
meaning of any symbols, maps or writings etc
which are in range and can be clearly seen. This
will supply vague definitions about the piece of
information. It may be only attempted once per
piece of information (GMs discretion). If a double
effect is rolled, the adept may ask 1 question about
the information. If a triple effect is rolled, the adept
may ask 2 questions.
If the symbol is magical then Adept will discern its
general effect. If a double effect is rolled, the adept
can ask for 1 of the attributes of the spell (e.g.
Rank, specific name, etc.). If a triple effect is
rolled, 2 attributes may be discovered.
Spirit Vision (T-2)
Range: 50 feet + 10 / Rank
Experience Multiple: 200
Effects: The Adept may attempt to see into the
spirit world. They can see spirits, such as the souls
of the dead (which normally remain close to their
bodies for 3 days before travelling to the lands of
the dead), those travelling outside their bodies (e.g.
via the Spell of Visitation or the Herbalist Potion),
incorporeal or insubstantial undead etc. (e.g. a
vampire in the form of a cloud of mist as an undead spirit), insubstantial Fae (e.g. dryads, sylphs),
summoned spirits (e.g. whispering wind, speak
with dead), as though they were normally visible.
Although the Adept cannot normally see the spirit
of a living being (inside their body) they may, at
the GM’s discretion, gain some inkling into a characters soul should they have attracted any spirit
followers.

26.7 General Knowledge Spells
Control Entity (G-1)
Range: Touch
Duration: Special
Experience Multiple: 500
Base Chance: 10%
Resist: Passive
Target: Entity
Rune: Entity
Effects: This spell requires the blood of either the
target or the Adept to be used to paint a Rune of
Compulsion onto the forehead of the target. If
target fails to resist, then they are compelled by the
Adept.
The compulsion does not in any way affect the
mindset or opinion of the target, but they are forced
to obey an direct command given to them. Should
the target be opposed to the Adept, then they will
interpret any command in the narrowest and least
useful manner possible. This spell has a duration of

91

The darkness created will be 60% + 2 / Rank). At
Ranks 0-4 the darkness is like evening twilight, at
Ranks 5-9 it is like moonlit night, at Ranks 10-14 it
is a starlit night, at Ranks 15-19 like pitch dark
room and at Rank 20 (100% dark) no vision is
possible. Although infravision works off heat and
elvish and dwarvish visions work in total darkness,
it is still not possible to see at all at rank 20. This is
elemental darkness and will cast shadows. However it does not give Celestials bonuses (but may
give penalties).
Lesser Healing Rune (G-3)
Range: Touch
Duration: Immediate
Experience Multiple: 200
Base Chance: 35%
Resist: None
Target: Living entity
Rune: Living Entity
Effects: The Adept paints Runes of Healing over
the body of the target. The spell takes at least a
minute to cast and heals 1 + (Rank / 2) points of
damage.
Light Rune (G-4)
Range: 5 feet + 1 / Rank
Duration: 15 minutes + 15 / Rank
Experience Multiple: 100
Base Chance: 40%
Resist: None
Target: Point
Rune: Object/RuneStaff
Effects: The Adept draws the Light Rune (or has it
inscribed on their staff). The Rune will emit light
as a point source and cannot be moved unless
inscribed on a Runestaff. The light within the
specified range will be 60% + 2 / Rank. At Ranks
0-4 this light is equivalent to a small lamp and will
clearly illuminate the immediate hex, Ranks 5-9
the light is like that of a camp fire and will clearly
illuminate the surrounding megahex, Ranks 10-14
the light is like a large bonfire and will brightly
illuminate a radius of 15 feet, Ranks 15-19 the light
like a searing forge and brightly illuminates a
radius of 20 feet and Rank 20 the light as if the sun
on a bright day and will be blinding within a 25
feet radius.
This light is elemental light and is a point source so
extends beyond the specified range (at naturally
reducing levels). This will create shadows but does
not give Celestials bonuses (but may give penalties). At Rank 10 and beyond the actual point
source is over a foot in diameter and creates shadows without a defined edge.
Liquid Purification (G-5)
Range: Touch
Duration: Immediate
Experience Multiple: 100
Base Chance: 30%
Resist: May not be resisted
Target: Liquid
Rune: Runestaff/Container
Effects: The Adept may turn any aqueous substance into potable water by touching the substance

26 COLLEGE OF RUNE MAGICS
with their Runestaff which has the Purification
Rune incised into it. The Adept may purify 1 (+ 1 /
Rank) gallon by volume with this spell. This spell
may be used to neutralise poison in solution. Note:
This spell is not intended for use in combat and
will not work on anything with magic resistance. If
the Rune is drawn on a vessel of maximum capacity 1 (+ 1 / Rank) quart then any liquid within the
vessel is purified. At Rank 11 or above, the Rune
may cause the vessel to shatter if it contains poison.
This spell may be cast reversed to pollute a liquid.
Pyrogenesis (G-6)
Range: 5 feet + 1 / Rank
Duration: Immediate
Experience Multiple: 75
Base Chance: 40%
Resist: None
Target: Object or area
Rune: Point/Object/Runestaff
Effects: A Fire Rune is drawn and all eligible
things (small flammable objects, or entities no
larger than a mouse) within range burst into flame.
The flames are fuelled by the object or entity, and
may be extinguished normally.
If the Rune is on an object, then only the object
will ignite. If the Adept has the Rune on their staff,
then they may target a hex up to 5 feet (+ 5 / Rank)
away.
Smite (G-7)
Range: Touch
Duration: 1 hour + 1 / Rank
Experience Multiple: 200
Base Chance: 35%
Resist: Passive
Target: Entity
Rune: Entity
Effects: The Adept must paint a Smite Rune on the
target. This Rune is then activated by the Smite
Spell. Should the target make a successful strike,
then the opponent must make a magic resistance
against the Smite Spell. If the opponent fails to
resist, they suffers [D + 1] (+ 1 / Rank) damage. If
the spell is Rank 10 or above, should the opponent
fail to resist then they are thrown prone. At Rank
20 should the opponent fail to resist, they are also
stunned.
Once a successful strike has been made the spell
ceases to be in effect. The strike should be considered as being performed with a magical weapon.
If the spell is ritually cast then target may make 1
(+ 1 / 6 Ranks) successful strikes before the spell
ceases to be in effect.

26.8 General Knowledge Rituals
Fashioning Runestaff (Q-1)
Experience Multiple: 300
Base Chance: 30% + 3% / Rank
Cast Time: 1 week
Rune: Staff or Wand
Effects: The Adept may employ this ritual to create
a Runestaff or Runewand out of any of the materials listed for Rune Wands (§26.2). The implement
is fashioned by inscribing Runes into the material’s
surface, which describe its use, name, and history.
Once it has been fashioned and consecrated in this
ritual, it remains fully effective unless and until it
is broken or otherwise destroyed. A rune mage
may only have one rune staff or Runewand at any
time.
All materials used in an unsuccessful ritual (or a
ritual that backfires) are destroyed or ruined. If the
ritual is successful, the Adept may use the
Runestaff or Runewand thereafter to cast spells and
perform rituals that require the use of a Runestaff.
The adept may also inscribe runes upon the
Runestaff or Runewand that aid in the casting of
certain spells. The adept may inscribe 1 rune into a
wand or up to 1 + (Rank / 3) runes into a staff.

In addition, the Adept may store a maximum of 2
Fatigue Points in the Runestaff at Rank 0, and an
additional 1 Fatigue Point for every Rank they
have with the ritual of Fashioning Runestaff at the
time the Runestaff is fashioned. (This amount is
doubled if the staff is made of Oak, or halved if the
item is a wand. An oak wand holds the standard
amount.) Fatigue is stored may be used by the
Adept to cast spells at any time that they are holding the Rune staff while making a Cast Check. The
Staff will be restored to full fatigue at midnight on
the night of a full moon.
Runes of Sight (Q-2)
Experience Multiple: 300
Base Chance: 40% + 3% / Rank
Cast Time: 1 hour
Rune: Self/Area, Object, Entity
Effects: The Adept may gain insight into the future
by drawing the Runes of Sight (Runes which represent the cosmic balance). There is no possibility of
backfire from this ritual. The performance of this
ritual allows the Adept to exercise one of the following functions during its course:
Limited Precognition The Adept draws a Rune of
Sight on themselves. This ritual produces the same
results as for the Spell of Limited Precognition of
the Mind College.
Divining Enchantment The Adept draws Runes of
Sight around the target to attempt to determine if
an entity or object is currently, or had been recently, under the effects of a spell. The object or
entity must be present for the entire duration of the
ritual, and be within 5 feet (+ 1 / Rank). The ritual
may not be resisted. The Base Chance is reduced
by 5 for every week or part thereof since the spell
that is being divined was cast. Permanent magic
(e.g. invested items still with charges) or spells
currently in effect carry no modifier. The Adept
gains knowledge of those spells that fall within
their cast chance.
If the Adept can divine the spell, its exact name
and college are revealed. If the spell is noncolleged in origin, its general effects are revealed.
Only one of these two options may be performed at
each casting of the Ritual.
Sending (Q-3)
Range: 10 miles + 5 / Rank
Experience Multiple: 250
Base Chance: 30% + 5% / Rank
Cast Time: 5 hours
Resist: The ritual can be only passively resisted
Rune: Self
Effects: The Adept must paint their forehead with a
Sending Rune before retiring to sleep. They will
then require a five hour period of sleep with no
disturbances sufficient to wake them or the ritual
will fail. The target of the spell is likewise required
to be asleep for five undisturbed hours or the ritual
will not work. The time asleep counts as resting for
Fatigue recovery purposes. During the time asleep,
the Adept will be in communication with one entity
of their choice that they have seen and studied
sufficiently (as per College of Ensorcelments and
Enchantments Spell of Location for “seen and
studied”). Alternatively, the Adept may employ the
target’s Individual True Name if it is known.
If the Cast Check is successful and the target fails
to resist then it will answer all questions asked of it
in a yes / no fashion. This ritual does not allow
communication with entities on other planes of
existence. Upon completion of the ritual the Adept
may receive the answers to Rank questions.
Warding with Runes (Q-4)
Range: 70 feet
Duration: 1 week + 1 / Rank
Experience Multiple: 200
Base Chance: 30% + 5% / Rank
Cast Time: 2 hours (10 minutes / rank) minimum
10 minutes
Resist: None
Target: Area
92

Rune: Around area
Effects: The Adept must draw Rank Rune of Warding symbols in a roughly circular configuration
around the area to be warded (the Adept must
remain inside the area while the ritual is being
prepared). At the end of the ritual, if it is successful, a Rune Ward exists that will help to protect
those inside it from magic.
No magical item (amulet, weapon, etc.) can enter
the warded area unless it is a possession, though
items already inside the warded area can be taken
out.
Any magical creature, spirit or Adept attempting to
enter the warded area must make a Passive Resistance (-2 / Rank of ritual) check, or it will be unable to enter the area. In addition, an entity which
is wholly or partially of another plane (such as
demons, devils, imps, hellhounds) decreases its
Magic Resistance by 3 / Rank when it attempts to
enter the warded area.
If the ward is breached then one of the Runes supporting the ward momentarily glows and then
disappears. When the last Rune disappears then the
ward dissipates.
In addition, so long as it is in effect, all targeted
spells cast into (not out of) the warded area have a
30% + 2 / Rank of ward chance of being dissipated
harmlessly when striking the warded area.
Backfire from this ritual results in D10 damage to
the Adept’s Endurance.
All entities which were in area of the ward for the
duration of its casting of the ward are not subject to
it.

26.9 Special Knowledge Spells
Banishment (S-1)
Range: Touch with Runestaff
Duration: Immediate
Experience Multiple: 250
Base Chance: 30%
Resist: Passive
Target: Entity
Rune: none
Effects: The Adept may banish any one entity back
to its own plane of origin. In order to do so the
Adept must touch the target entity with their
Runestaff at the moment the spell is completed. If
successful, the spell results in the entity immediately returning to its own plane unless the entity
successfully resists. The touch is automatic unless
the target is actively avoiding being touched, in
which case the target must make a successful strike
at the moment of casting. The spell must be prepared normally. The target returns to a random
spot, in an appropriate medium, on its own plane.
The exact whereabouts is GM’s discretion, however, entities banished at approximately the same
time will appear in approximately the same area.
Control Corpse (S-2)
Range: Touch
Duration: 1 hour + 1 / Rank
Experience Multiple: 300
Base Chance: 15%
Resist: None
Target: Corpse
Rune: Corpse
Effects: The Adept inscribes the Animate Rune on
the target corpse (can be either sentient or nonsentient but must be formally living). With a successful cast check the Adept will animate the
corpse into a zombie under their control. The zombie will work at (4 / Rank)% of their living physical ability. The zombie is completely mindless and
requires at least passive concentration for the
Adept to function. The maximum size of the entity
is 1 hex + (1 / 5 ranks)
Converse with Spirits (S-3)
Range: 10 feet + 5 / Rank
Duration: 5 minutes + 5 / Rank
Experience Multiple: 200
Base Chance: 30%

26 COLLEGE OF RUNE MAGICS
Resist: None
Target: Self
Rune: Self
Effects: The Adept inscribes the Converse Rune
and a Rune representing the target spirit on their
face. Should the Adept successfully cast this spell
they will be able to “converse” with a single spirit
which is within range and falls within the Rune of
representation. For example, the Adept could use a
rune representing lesser undead, and then any ghost
or other lesser undead could answer, or use a number of runes to represent say Girden Bloodaxe, a
fallen dwarven warrior. Then, if the spirit of Girden is within range then only he would answer.
This spell does not compel any spirit to answer any
question and if they do answer then it does not
compel them to speak the truth.
Creating Rune Weapon (S-4)
Range: Touch
Duration: 5 minutes + 1 / Rank
Experience Multiple: 200
Base Chance: 20%
Resist: None
Target: Weapon
Rune: Entity
Effects: The Adept may create a magically poisoned weapon by inscribing a Rune of Acid on a
weapon and activating it with the Rune Weapon
spell. If at least one point of effective damage is
inflicted on a target, they will take [D - 5] (+ 1 / 3
or fraction ranks) damage per pulse for D10 pulses.
The target can only have such affect in effect at
any one time, i.e. acid from different strikes is not
cumulative. The acid is considered magical in
origin and will affect creatures not normally affected by such things. The normal rules for using
poisoned weapons apply but the Adept is immune
to their own weapon spell.
The Adept may choose instead to draw a Weapon
rune in the air and create a magical weapon of their
choosing. The weapon will be insubstantial and
magical in natural and will hit everything, including creatures of a spiritual or spectral nature, but
otherwise will be completely normal.
Greater Heart Rune (S-5)
Range: Touch
Duration: 1 day + 1 / Rank
Experience Multiple: 300
Base Chance: 25%
Resist: Active, Passive
Target: Living Entity
Rune: Circle
Effects: The Adept expends 5 Ft and takes 10 minutes to paint a Rune of Healing on the skin over the
heart, of at least 4 inches in diameter. The rune will
heal the target 3 (+ 1 / 2 Ranks) Endurance, immediately, or when the target next takes endurance
damage. The rune can be washed of easily with
water.
The Adept can make the rune semi-permanent by
tattooing the Rune onto the target. This reduces the
target’s Endurance by 3 points until the rune is
used or the spell ends.
Rune Curse (S-6)
Range: 5 feet + 5 / Rank
Duration: Special
Experience Multiple: 200
Base Chance: 15%
Resist: Active, Passive
Target: Entity or Object
Rune: Target/Runestaff
Effects: The Adept must first have the Curse Rune
inscribed on their Runestaff for this spell to work
at range, otherwise they can carve the Curse Rune
into the victim (taking a minute). The duration of
the curse is based on the cast time.
Cast time

Duration

Pulse
Minute
Hour
Day

Rank minutes
Rank hours
Rank days
Permanent

The Adept curses any one target with a particular
unpleasantness as listed below. If the effects of the
curse are doubled or tripled, the Adept may inflict
2 or 3 different results. If a Ritual of Remove
Curse is employed, the Rune Curse is considered a
Minor Curse. Ritual of Remove Curse must be
used on each separate curse. Identical Rune Curse
effects are not cumulative. The Adept may always
choose to inflict a curse of lesser Rank than their
actual Rank. The Curses that the Adept may inflict
are dependent on the Rank of the spell:
Rank Curse
0–4 The victim will suffer hallucinations that will
reduce their Perception by 5 in addition to any
specific effects. The GM and the Adept must work
out the exact nature of the hallucination at the time
that the curse is made. Hallucinations should, however, be of a minor, generalised nature, seeing
coloured lights in the distance, hearing sounds like
the clanking of weaponry, smelling meat cooking
from time to time, and so forth.

Rune of Truth (S-8)
Range: Touch
Duration: 10 minutes + 10 / Rank
Experience Multiple: 300
Base Chance: 30%
Resist: None
Target: Entity
Rune: Entity
Effects: Prior to casting this spell the Adept must
draw a Truth Rune on the forehead (or equivalent)
of the target. The Rune of Truth causes the target
to be unable to speak a falsehood for the duration
of the spell. The target must not knowingly say
anything false, but may refuse to answer a question
put to them.
In addition, the bearer of the Truth Rune may
attempt to see the true nature of all things with a
(PC + 2 × Rank) chance of noticing deceptions,
such as illusions, invisibility, shape or skin changing, traps, and any other deception the GM sees fit.
Only one attempt may be made per object.

5–9 The victim will suffer from terrible migraines
and must make a concentration check for every
complex action (such as casting or using a skill, but
not standard combat).

The Truth Rune does not necessarily help the target
see through the deception, for example, an Illusory
Wall will still be opaque, but the target will know
it is an illusion.

10–13 The victim will suffer from limited Amnesia. Any complex activity (using a weapon, casting
a spell etc) will require a Magic resistance check.
Should the victim fail they will be unable to remember how to perform that action will not be able
to remember it again for a period of (Rank × Cast
time). The victim has not forgotten anything but
simply temporarily can’t remember how to do
something.

Rune of Willow Healing (S-9)
Range: 15 feet + 15 / Rank
Duration: Special
Experience Multiple: 450
Base Chance: 35%
Resist: Passive
Target: Any living creature
Rune: Over heart of target
Effects: The Adept first paints The Rune of Healing over the heart of the target. At any time within
twice rank hours the adept activates the Rune
within range. The rune will heal the target 3 endurance damage per pulse for (Rank + 2) pulses then
fades.

14–16 The victim is afflicted with Creeping Senility and will lose 1+ (Rank / 5] points of MA immediately and a similar amount every day afterward.
17–19 The Adept may afflict the target with extreme paranoia and nightmares. The target will
recover only one fatigue point per hour from taking
a nap, and only 2 per hour from sleeping. In addition, the target will feel hag-ridden and imagine
themselves pursued by phantasms. They will, until
the curse is removed, become more and more estranged from reality, distrustful of friends and
companions, and obsessed with the idea of destroying their enemies (who they think are “all
around”). If the curse is not removed within D10 ×
[target’s Willpower 2 × Rank] days, the target will
completely lose touch with reality. They will then
plot to destroy their friends in the belief that they
are “out to get them” and will exhibit other bizarre
behaviour. They will be cured of the advanced
stage of this affliction only by having the curse
removed and then spending a number of days equal
to the Adept’s Rank × D10 in rest and recuperation.
20 Total Amnesia
Rune Lock (S-7)
Range: Touch
Duration: 1 hour + 1 / Rank
Experience Multiple: 200
Base Chance: 30%
Resist: None
Target: Portal
Rune: Portal
Effects: This spell may be cast over any portal
(door or window) inscribed with the Lock Rune
that can normally be opened or closed and is in
sight. It effectively locks the portal with an unpickable lock. The spell can be dispelled by anyone
casting the Rune College Special Counterspell or
Spell of Opening of at least equal Rank.
If the portal is destroyed by brute force (or by
magical means) then the spell will dissipate. It will
take rank × rank points of damage to destroy the
Rune locked portal.
The Adept may open any portal they have locked
without dissipating the lock.
93

Rune Shield (S-10)
Range: Touch
Duration: 1 hour + 1 / Rank
Experience Multiple: 250
Base Chance: 40%
Resist: None
Target: Entity
Rune: Entity
Effects: The Adept must inscribe a Rune of Protection onto the target. The magic will create a shield
of protection around the target, giving 5% + Rank
to defence and absorbing Rank / 4 points of physical damage. Any Grievous Blow to the target will
disrupt the shield but, in that case, the specific
grievous injury will not be applied to the target.
Rune Wall (S-11)
Range: None
Duration: 30 minutes + 30 / Rank
Experience Multiple: 250
Base Chance: 20%
Resist: Passive
Target: Area
Rune: Point
Effects: The Adept may, by drawing a Rune of
Protection, not necessarily on an object, create a 20
(+ 2 / Rank) feet radius, transparent, shimmering
wall of force 1 inch thick, centred on the Rune.
The wall can be of any orientation and need not be
anchored. It will expand around solid objects but
will not pass through them and will not form
touching an entity (the spell will fail immediately if
it comes in contact with an entity while forming).
Any entity who comes into contact with the wall
must resist or be thrown back prone and will suffer
[D - 2] (+ 1 / Rank) damage.
Sacrifice (S-12)
Range: Touch
Duration: 10 seconds + 10 / Rank
Experience Multiple: 650
Base Chance: 5%
Resist: Passive

26 COLLEGE OF RUNE MAGICS
Target: Any living creature
Rune: Self
Effects: The Adept first draws the Death Rune
across their forehead and then activates it. They
must then touch their victim (successful unarmed
strike) and release the spell. If the victim fails to
resist the Adept immediately gains all current
Fatigue and Endurance from the victim. Up to
Rank points each of this can be used to heal damage and restore fatigue respectively. If the victim
has zero or less current fatigue and Endurance
nothing is gained.
In addition, if the Adept then spends a hour making
a meal of their victim they can temporarily increase
their following characteristics to up to (5 / Rank) %
of that the victim.
Organ / Body part

Stat

Brain
WP
Heart
EN
Arms
PS
Legs
AG
Hands
MD
Eating the genitals will give a Rank% increase to
the Adept’s virility. See Conception (§4.8) for
conception chances.
If the victim is skinned then the Adept may “wear”
skinned. This will cause superficial physical
changes such as snout nose, hoofed feet, clawed
hands, hairy skin.
All transformations last 10 (+ 10 / Rank) minutes.
This spell does not work on plants.
Summon Totem Helper (S-13)
Range: Unlimited
Duration: 10 minutes + 10 / Rank
Experience Multiple: 200
Base Chance: 30%
Resist: None
Target: Spirit
Rune: Self
Effects: If successful a Totem spirit will arrive in
D10 pulses to aid the Adept. This aid may include
asking the spirit to summon a totem animal to the
Adept, give basic geographical knowledge and do
simple scouting tasks. The spirit will automatically
warn the Adept of any immediate danger to the
Adept that they see (with a base chance of Adept’s
PC + 2 / Rank). A Totem spirit cannot be summoned again until (24 - Rank) hours have passed.
Torment (S-14)
Range: 15 feet + 15 / Rank
Duration: Immediate
Experience Multiple: 250
Base Chance: 15%
Resist: Active, Passive
Target: Entity
Rune: Entity
Effects: The Adept can, by pointing their
Runestaff, inscribed with the Pain Rune, at one
entity, cause that entity extreme pain. Entities who
fail to resist may only take a Pass action every
second pulse until they recover. Entities who successfully resist reduce all Strike Chances by 30,
and take twice as long to perform any action until
they recover. Note that Mind Mages gain a bonus
to resist this spell equal to 2 × Rank with their
Talent of Resisting Pain.
Each pulse that the Adept continues to point the
Runestaff at the entity (requiring a pass action) it
suffers [Rank / 4] point of damage and may not
attempt to recover from the spell.
Rank

Difficulty

0–4
5–9
10–14
15–
20

4 × WP
3 × WP
2 × WP
1 × WP
0.5 × WP

Trapping Spirit (S-15)
Range: 10 feet + 5 / Rank
Duration: 1 minute + 1 / Rank

Experience Multiple: 250
Base Chance: 5%
Resist: Active, Passive
Target: Spirit
Rune: Circle
Effects: The Adept must draw a circle or at least
one foot radius with the Runes of Protection
around its circumference and then draw a pentacle
using fresh blood and inscribe the Runes of Binding and Representation (see Converse with Spirits
above on specifying the target spirit) within. If the
named spirit is within range, they will be drawn
into the pentacle where they are trapped for the
duration of the spell.
If the spell is ritually cast then the duration increases to 10 (+ 10 / Rank) minutes.
Visitation (S-16)
Range: 1 mile + 1 / Rank
Duration: Concentration: maximum 1 hour + 1 /
Rank
Experience Multiple: 300
Base Chance: 15%
Resist: None
Target: Entity
Rune: Self
Effects: The Adept must draw the Runes of Farseeing on themselves while performing the spell. If
successful, the Adept is able to send a ghost-like
image of themselves instantly to a previously
drawn Rune of Location, within range. They are
present in that location in all ways except bodily
(i.e. the Adept may communicate and use all their
senses while the image is there, but may not be
harmed by any attack). The image may move no
more than 10 feet (+ 10 / rank) from the specific
Rune of Location, and may materialise anywhere
within that area. The Adept may not cast any spells
or rituals. When the visitation time has expired (or
anytime prior that the Adept wishes), the image
quickly fades and travels back to the Adept.
Warning Stones (S-17)
Range: Touch
Duration: 1 hour + 1 / Rank
Experience Multiple: 100
Base Chance: 20%
Resist: None
Target: Self
Rune: Stone
Effects: The Adept draws the Rune of Warning and
at least one of the Runes of Body and/or Mind on a
stone (which must weigh at least 1/2 lb. per Rune).
The Adept may then leave the stone somewhere
and will instantly know if an entity comes within 5
feet ( + 1 / Rank) of the stone. The stone will detect
living and/or sentient entities depending on which
Runes it is inscribed with. The Adept may use as
many warning stones as they wish, but will be
unable to tell which of their stones has detected an
entity.

26.10 Special Knowledge Rituals
Binding Elements (R-1)
Duration: 2 hours + 2 / Rank
Experience Multiple: 500
Base Chance: MA + 3 / Rank
Cast Time: 30 minutes
Rune: Runestaff
Effects: The Adept may gain control of any element by using this ritual. They must have had the
Binding Rune and the Rune representing the element to be bound inscribed on their Runestaff and
they must touch the element with their Runestaff at
the conclusion of the ritual. The Adept may bind
500 pounds of earth (+ 500 / Rank), 500 gallons of
water (+ 500 / Rank), 1000 cubic feet of air (+ 500
/ Rank), or all fire within a 10 foot radius (+ 15 feet
/ Rank). They may do anything with the element
except form an elemental. This ritual may not be
used over an area occupied by an elemental and
cannot be used in any way to control an elemental.
Binding Spirits (R-2)
Duration: Permanent
94

Base Chance: MA + 3 / Rank
Cast Time: 4 hours
Resist: May not be resisted
Rune: Skull
Effects: To perform this ritual the Adept must
possess the skull of the spirit they wish to bind, and
the spirit must be present (e.g. within 100 hours of
death, the body has been preserved by a Healer, the
spirit is held in a spirit trap, the spirit has remained
on plane as a greater undead or ghost). The Adept
must spend four hours cleaning the skull (boiling
off any remaining flesh etc.) and etching it with
Runes to bind the spirit. Note that although they
may not resist this ritual the spirit may, should they
be able, attempt to disrupt the ritual or slay the
Adept. If the Adept has the victim’s heart, they
may burn this during the ritual to gain an extra +20
on Base Chance.
Upon successful completion of this ritual the spirit
is bound to the skull and may not leave unless and
until the skull is destroyed. Although a bound
greater undead would be able to drain anyone who
touched the skull, in general the bound spirit will
be unable to affect the material world. The Adept
can use the Spell of Converse with Spirits to question the spirit, and may gain useful answers / advice should the spirit have any expertise in the
area. Although the spirit cannot lie it may refuse to
answer and can mislead by omission or neglecting
to correct false assumptions and the like. A backfire result destroys the skull and the Adept’s Endurance value is reduced by (D-5) (minimum 1)
points which may only be recovered by the expenditure of Experience Points. The Adept will be
unable to attempt to bind that spirit again. NOTE:
Unless the spirit has some reason to wish to remain
as an adviser it is unlikely to be happy about being
kept trapped on this plane.
Casting the Runes (R-3)
Experience Multiple: 500
Base Chance: 5% + 5% / Rank
Cast Time: 1 hour
Rune: Paper
Effects: The Adept must prepare a piece of paper
or vellum on which are written the Runes of Doom.
At the end of the ritual, the Adept chooses a creature (see below) from the Seventh Plane to be the
executor of the doom and also writes this name on
the paper. The name must be capable of being read.
The Adept’s player must actually write this information down, since it will only come into play in
the future. Once the ritual is prepared, the Adept
then passes the sheet of paper on to the victim
whose name is written on the paper. The victim
must voluntarily accept the paper (though they
need not know what is on it). Once they accept it, a
creature named on the paper will turn up in [20 +
D10 - Rank] days and hunt them down and kill
them. Even if the creature is destroyed another will
return within a similar time.
Only by passing the paper on to another entity who
voluntarily accepts it can the doom be transferred.
If the paper is destroyed, the doom can never be
transferred. The Doom may be lifted by the Adept
by ritually casting a Rune Special Counterspell on
the target or a remove curse may be performed.
The curse is considered a Major Curse with an MA
of (MA of Adept + 2 × Rank of ritual) and can be
removed by Ritual of Remove Curse.
This ritual requires the expenditure of one point of
Endurance (permanently) regardless of success. If
the ritual backfires, the named creature will immediately turn up and attempt to kill the Adept, but
will not return once destroyed.
Rank

Creature

0–4
5–9
10–14
15–19
20

Imp
Half devil
Devil
Succubus or Incubus
Named Demon

26 COLLEGE OF RUNE MAGICS
Creeping Doom (R-4)
Experience Multiple: 450
Base Chance: 20% + 4 / Rank
Cast Time: 1 hour
Resist: Special
Rune: Bones
Effects: The Adept creates 13 Runes of Destruction
by carving the appropriate maledictions into human
bones. They then perform a ritual over them and
bury the sticks beneath the dwelling of someone
they wish to curse. It is best if the victim’s name is
carved in the bones as well, otherwise others in the
house may become ill instead. For each month that
the bones remain in or under the victim’s dwelling,
they must make a Resistance Check, the Base
Chance for which is composed of the victim’s
Endurance multiplied by the Difficulty Rating of
the resistance.
Rank

Difficulty

0–5
4 × EN
6–10
3 × EN
11–15 2.5 × EN
16–18 2 × EN
19–20 1.5 × EN
If the victim fails to resist, they suffer a wasting
disease and lose [D – 3] Endurance points for the
purposes of future resistance (only). If they fail to
resist for three straight months, they die.
Generally, the victim of these maledictions does
not know exactly what is wrong with them. Should
they discover the bones, they may remove the
curse by removing the bones from the house. Other
means of ending a curse do not normally suffice,
although the sufferer would show immediate improvement upon leaving the house and sleeping
elsewhere for a few weeks. There is no chance of
this ritual backfiring.
Rune Healing (R-5)
Range: Touch
Duration: Immediate
Experience Multiple: 300
Base Chance: 2 × MA + 3 / Rank
Resist: None
Cast Time: 30 minutes
Target: Living entity
Rune: Living Entity
Effects: The Adept paints Runes of Healing over
the body of the target. The rune will heal the target
3 + 3 / rank. In addition, all afflictions which can
be cured by a (Rank / 3) healer will be fixed, with

exception of preserve dead, which cannot be done
as the target is not living.

All entities passing through a Rune portal lose all
their Fatigue, in the form of tiredness fatigue.

Rune Portal (R-6)
Range: Special
Duration: Special
Experience Multiple: 400
Base Chance: MA + 5 / Rank
Cast Time: 30 minutes
Rune: Stone circle/circle
Effects: Rune portals allow a Rune Mage to transport themselves and Rank other entities with them
to any other portal which the Rune Mage has visited, has sufficient knowledge (uniquely distinguishable) about, or the source portal’s “linked
destination”. The Adept can create two types of
Rune portals.

Transformation (R-7)
Duration: Special
Experience Multiple: 500
Base Chance: MA + 3% / Rank
Resist: Passive only
Cast Time: 1 hour
Rune: Circle
Effects: By the performance of this ritual the Adept
merges a living sentient entity with a non-sentient
animal. Both entities must be living (though they
need not be conscious), and must remain within a
circle of runes for the entirety of the performance
of the ritual. Upon completion of the ritual the
animal will shrivel and wither away to dust, while
the sentient entity’s body will writhe and transform
into that of the animal. Both entities may choose to
resist and should either be successful the ritual will
fail. The sentient will remain trapped in the form of
the animal (having the animals physical characteristics but retaining their own Magical Aptitude,
Willpower, and Perception) until subject to a Ritual
of Remove Curse (this ritual counts as a Major
Curse).

Permanent A permanent portal is constructed by
the Adept inscribing Runes of Translocation on
large stones, placing them on a flat surface to create a circle of a size, in hexes, at least equal to that
of the number of entities the portal can transport.
Permanent portals take a day per hex to construct
and cannot be destroyed unless all the stones forming the portal are smashed. At the time the portal is
created, a “link” destination can be imbued in it.
The destination must be a permanent portal the
Adept has previously visited.
Temporary The portal is constructed by the Adept
painting Runes of Translocation onto a surface is a
circular fashion (taking half an hour). The circle
must be of a size, in hexes, at least equal to that of
the number of entities the portal can transport.
Temporary portals last 1 week (+ 1 / Rank), unless
a Rune Special Counterspell is cast into the area, in
which case, it will immediately dissipate.
The ritual to create the portal takes half an hour to
perform and has a Base Chance of MA + 5 / Rank.
If the ritual fails then nothing happens and ritual
can be performed again without additional work
but a backfire will ruin the entire ritual and a new
portal will have to be constructed.
To utilise a Rune portal, the Adept performs a half
hour ritual. The Adept may transport a maximum
of themselves and Rank others (multihex creatures
count as size, in hexes, entities). Base Chance to
transport is MA + (5 / Rank) - (1 / 5 miles). Of the
ritual is successful, the Adept must spend 1 FT per
entity transported. A backfire will only result the
expenditure of the Fatigue. Transportation is instantaneous.

95

They will only be able to perform those skills and
abilities which the GM deems feasible in animal
form, and will be unable to perform Spells or Rituals.
At Rank 10, should the Adept have a sample of
blood, hair, or nail clippings etc. from an entity
they wish to target as prey, they may include this in
the Ritual. In this case, instead of the above, the
animal and sentient’s bodies will writhe together
and merge to form an anthropomorphised version
of the animal with each characteristic being the
higher of the two. The hybrid is imbued with an
irresistible hunger for the targeted entity. They will
always know in which direction their target is, and
will be unable to perform any action except hunting for and eating their prey. They will not sleep or
eat (except for their target) and will die in 8 hours
(+8 / Rank). The sentient will be aware of their
actions, but will have no control. At Rank 20 the
Adept may use this ritual to make a chimera from
any two non-fantastical living creatures. The exact
effects are up to the GM. Some obvious examples
are a minotaur from a human and a bull, a gryphon
from a eagle and a horse, etc. Should both creatures be sentient it will retain both heads.

96

27 COLLEGE OF WITCHCRAFT

27 The College of Witchcraft (Ver 1.1)
The College of Witchcraft is concerned with natural magics, the rhythms of the world, and especially with blessings and curses. Practitioners of
the College of Witchcraft are commonly known as
Witches if female, Warlocks if male, or collectively as Wicca. This College is without doubt the
most primitive and least formal in its approach to
magic. The Wicca generally feel themselves to be
more in tune with the world than the various Elemental Colleges, who dedicate themselves to only
part of the whole, and certainly more than the
Thaumaturgists who practise a sterile and scientific
magic. The widespread use of Witchcraft predates
the present renaissance of magic on Alusia and
Wicca are much more often found practising their
trade in small towns and villages than in cities.
Novice Wicca will usually have been apprenticed
to a local Witch or Warlock rather than having
attended any form of Magical Academy.
The College’s magic touches on alchemy, herbalism and astrology, and many Wicca possess some
of these skills. The other Colleges often treat
Wicca with a degree of contempt as they view this
dabbling with “quasi-magic” to be less than
worthwhile. This is not to say that the Wicca are
without power, as experienced Adepts of Witchcraft have available to them powerful magics, fully
the equal of any other College. In these destructive
and powerful magics there lies danger however,
and some Wicca become so obsessed with the
“darker” side of natural magic that they begin to
follow the Dark Path of magic and make agreements with the Powers of Darkness so as to further
their material goals. These Adepts are often known
as “Black Mages” and are greatly feared. Many
“Black Mages” view the other members of the
College as weak or foolish for failing to exploit the
powers they possess. At the other extreme, some
Wicca follow a path of pacifism, and eschew the
curses and destructive side of Witchcraft. These
Adepts are sometimes referred to as “White
Mages” or “White Witches”, in contrast to their
darker brethren. Most Wicca view both of these
extremes as unfortunate deviances from the College’s holistic path. Many of the Agents of the
Powers are Wicca as the College’s general and
undivided outlook does not conflict often with a
Power’s interests.
Being highly in tune with magic as a whole, Wicca
are more sensitive to changes in the “mana flow”
than other Colleges. The Wicca’s magic is somewhat affected by the presence of large amounts of
spirit such as from proximity to many entities, or
from a lessening of the “mana flow”, such as on
the certain “mana poor” days of the year, which
have universally become known as “High Holidays” of the Elohim, the spirit Powers. Conversely,
a Wicca’s powers increase when away from
sources of spirit and on certain “magic rich” days,
sometimes referred to as “Faerie days”. Finally,
due to the ancient nature of the College and the
equally ancient association between the elements
of light and spirit, a Wicca’s powers are slightly
decreased during the hours of daylight, whether the
Wicca is in light or not, and are slightly enhanced
during the hours of darkness.
Traditional Colours
Wicca usually wear clothing in the colours of
nature itself, in much the same way as the Elemental mages, but often in combination that the Elementals do not use. Blues and greens reminiscent
of sky and sea are worn with the light browns of
the earth, and with the warm golds and oranges of
autumn or fire. “Black Mages” traditionally cloak
themselves in midnight robes sometimes embroidered with pentagrams and other Daemon associated symbols, whilst “White Mages” often wear
robes of bleached or unbleached wool or linen.

Traditional Symbols
Animals are most commonly associated with the
Wicca as many Adepts of this College keep animal
familiars with cats, ravens, toads and owls being
the most usual. There is only one symbol that is
often connected with the Wicca, the Great Wheel
of Being, representing light and darkness, earth and
air, water and fire, spirit and mana. This symbol
sometimes appears as an eight pointed star or eight
arrows radiating from a central point, and at other
times as two circles passing through each other at
right angles, or even simply, as two intersecting
circles. Black Mages often use and wear Daemonic
symbols, but very seldom use the Great Wheel.
White Wicca seldom use any symbols, but when
they do, the Great Wheel is the most common.

27.1 Restrictions

100 and its cost is 50 Silver Pennies. The Base
Chance of effectively preparing it is 60% (+ 2 /
Rank).
Making Amulets The ability to make the following amulets:
Amethyst Wards bad dreams and assists the wearer
in achieving a restful sleep. Increases the wearer’s
Fatigue recovery during sleep periods by 10%
(round down). Cost: 3000 SP.
Aquilegius The wearer subtracts 10 from all rolls
on the Fear Table. Cost: 2400 SP.
Beryl Increases the wearer’s ability to detect traps
and ambushes by 5. Cost: 4000 SP.
Betony Decreases the Base Chance of infection by
15. Cost: 2200 SP.

Adepts of the College of Witchcraft may practice
their arts without restriction.

Bloodstone Prevents miscarriage and decreases
Base Chance of infection by 20. Cost: 3000 SP.

The MA requirement for this College is 18.

Carbuncle Decreases damage done by poison by 2
points of damage per pulse or day. Cost: 9600 SP.

27.2 Base Chance Modifiers
The Base Chance of performing any talent, spell or
ritual of this College is modified by the addition of
the following numbers:
From sunrise to sunset
-5
From sunset to sunrise
+5
In large town or city (pop. greater than
-5
2000)
In small town or village (pop. 500 to 2000)
+0
In hamlet or rural (pop. less than 500)
+5
Mana poor day
-5
Mana rich (Faerie) day
+5
In a high mana area
+5
In a low mana area
-5
All modifiers are cumulative. Those modifiers
pertaining to sunrise and sunset are applied depending on the hour of day, and not on whether the
Wicca is standing in light or darkness. A Wiccan
underground with no light during the day still
receives the negative modifier. Modifiers pertaining to population, refer to the population of sentient beings, within 1 mile of the Wicca.

27.3 Talents
Farsensing (T-1)
Range: 15 feet + 15 / Rank
Duration: Active concentration
Experience Multiple: 150
Target: Familiar
Effects: The Adept can, by remaining stationary
and actively concentrating for the duration of the
talent’s workings, see, hear, taste, smell and feel
the same things as their familiar, provided that
their familiar is within 15 feet (+ 15 feet / Rank) of
their position. This talent allows no special communication with the familiar, merely the ability to
utilise their senses. The Adept must have already
acquired a familiar through the use of Finding
Familiar Ritual (Q-1) for this talent to be effective.
It takes about 10 seconds (- 1 / Rank) for the Adept
to tune in to the familiar’s senses. If the familiar is
killed while the Adept is using this talent the magical backlash is harsher, due to the tighter link, and
the amount of magical damage incurred is increased by 5 points, see Q-1.
Special Alchemy (T-2)
Effects: The Adept gains certain knowledge of
Alchemy. The specific benefits accruing to the
Adept are:
Distilling Venoms The ability to distill venoms
from such plants as belladonna. The Adept functions as a Rank 1 Alchemist for this purpose. See
the Alchemist Skill.
Distilling Toad’s Sweat The ability to distill a
dose of a potion of Toad Sweat that will remove
blemishes, warts, corns, pimples, etc., at the rate of
1 disfigurement (wart, corn, etc.) (+ 1 / Rank) per
dose. The Experience Multiple for this potion is
97

Chalcedony No undead will willingly approach
closer than 10 feet to the wearer in most cases.
Cost: 4800 SP.
Diamonds Increases all of the wearer’s Strike
Chances by 2. Cost: 8000 SP.
Elder Flowers Makes the wearer proof against the
Evil Eye. Cost: 400 SP.
Hypericum Increases the wearer’s Magical Resistance by 10 to any magical act performed by a
Demon or Daemonic being. Cost: 800 SP.
Iron No Demon or Daemonic being will willingly
approach closer than 10 feet to the wearer in most
cases. Cost: 4000 SP.
Jade No undead will willingly approach closer
than 30 feet to the wearer in most cases. Cost: 4000
SP.
Jet No Demon or Daemonic being will willingly
approach closer than 50 feet to the wearer in most
cases. Cost: 4800 SP.
Luck Made from tiger’s or alligator’s teeth. It
increases the wearer’s Magical Resistance by 3 and
adds 2 to the wearer’s defence. Cost: 2400 SP.
Note that the “cost” is the cost of material necessary to manufacture the amulet. Each amulet requires 3 days to manufacture once the necessary
materials have been gathered or purchased. Amulets are usually sold at (cost + 25%). The time
taken to prepare an Amulet is full-time work, and
no training may be undertaken at the same time.
Those Amulets that prevent the “willing approach”
of certain creatures create a “circle of protection”
around the wearer. The creatures protected against
will not willingly cross the circle’s boundary, but if
forced across it, for instance by the approach of the
wearer, are no longer inconvenienced by the protection.
Love Philtre The ability to distill from a variety of
substances a Love Philtre which will cause the
imbiber to fall in love with the first entity upon
whom he or she sets eyes after drinking it (regardless of species or sex). The Base Chance to prepare
the Philtre is 30% (+ 3 / Rank) and the Experience
Multiple is 200. The cost of the materials will
average 600 silver pieces. The effects of the substance will last for 1 week (+ 1 week / Rank),
unless dispelled by the casting of the General
Knowledge Counterspell of the College of Witchcraft by the creator of the Love Philtre, or by the
successful use of the Curse Removal Ritual. In the
latter case, the curse is treated as Minor.
Fertility / Infertility Potion The ability to distill
from a variety of substances a Potion of Fertility or
Infertility that increases or decreases the chances of
conception by 5% (+ 5 / Rank). It has a 30% (+ 3 /
Rank) chance of working and may be passively

27 COLLEGE OF WITCHCRAFT
resisted by the imbiber. The effects of the Potion of
Fertility last 1 day (+1 per 3 or fraction Ranks)
whilst that of the Potion of Infertility last 1 week
(+1 per 3 or fraction Ranks), unless dispelled by
the casting of the General Knowledge Counterspell
of the College of Witchcraft by the creator of the
Potion or a Ritual of Remove Curse is employed. If
the latter option is taken, the curse is considered a
Minor Curse. The Experience Multiple for this
potion is 200 and its cost is 100 Silver Pennies. See
Conception (§4.8) for conception chances.
Witchsight (T-3)
Experience Multiple: 200
Effects: The Adept may see objects or entities
which are invisible and they appear to have a slight
blue sheen around them. If the invisibility effect
(excluding Walking Unseen) is of a higher Rank
than the Witchsight, the object or entity may not be
clearly identified or directly magically targeted.
The Adept may also see in the dark as a Human
does on a cloudy day, with an effective range of
vision of 150 feet under the open sky, and 75 feet
elsewhere.

27.4 General Knowledge Spells
Damnum Minatum (G-1)
Range: 15 feet + 15 / Rank
Duration: Special
Experience Multiple: 200
Base Chance: 40%
Resist: Active, Passive
Storage: Investment, Ward, Magical Trap
Target: Entity
Effects: The Adept curses any one target within
range with a particular unpleasantness as listed
below. Some of the effects are identical to backfire
results; such effects are cross referenced to the
backfire table (§53). If the effects of the curse are
doubled or tripled, the Adept may inflict 2 or 3
different results. The curse is permanent until a
General Knowledge Counterspell of the College of
Witchcraft is cast over the afflicted entity, a Ritual
of Remove Curse is employed, the duration expires, or the effect is cured by a healer of the appropriate Rank. Curses that have a duration, or may
be cured by a Healer are indicated in their descriptions. If a Ritual of Remove Curse is employed, the
Damnum Minatum is considered a Minor Curse. A
separate Counterspell or Ritual of Remove Curse
must be used on each separate curse. Identical
Damnum Minatum effects are not cumulative.
Note that the Adept may always choose to inflict a
curse of lesser Rank than their actual Rank. The
Curses that the Adept may inflict are dependent on
the Rank of the spell:
Rank Curse
0–3 Boils 1 (+ l / Rank); Warts 1 (+1 / Rank).
4–6 Clumsiness (-l AG); Maladroitness (-l MD).
7–9 Weakness (-2 PS); Poor health (-3 EN).
10–11 Cowardice (-3 WP & +5 Fright/Awe rolls);
Lose Smell & Taste (B:73).
12–13 Deafness (B:67); Lose Tactile Sense (B:75);
Virulent Skin Disease (B:79-80).
14–15 Insomnia (B:77); Wasting Disease (B:81);
Periodic Hallucinations (B:88).
16–17 Periodic Muscle Spasms (B:82-83); Asthma
(B:93); Migraines (B:86-87).
18–19 Creeping Senility (B:94-95); Struck Mute
(B:71); Arthritis (B:89-90); Enfeeblement (B:9192).
20 Blindness (B:63);
Amnesia [Roll D10]:
1–2 Partial — Magic (B:96)
3–5 Partial — Skills (B:97)
6–7 Partial — Recent (B:98)
8–10 Total (B:99)
Darkness (G-2)
Range: 15 feet + 15 / Rank
Duration: 15 minutes + 15 / Rank
Experience Multiple: 100
Base Chance: 60%
Resist: None
Storage: Investment, Ward

Target: Volume
Effects: The Adept creates a volume in which nonmagical light is partially suppressed. The volume
will be 1000 (+ 500 / Rank) cubic feet, and may be
in any one contiguous area the Adept desires, provided that no dimension is smaller than one foot.
The entire volume must be visible and within range
at time of casting, and may not be moved. For
visibility purposes, the Spell will increase Darkness levels within the volume to 60% + 2% / Rank.
Rank 20 Darkness may not be seen through. It will
not aid in providing bonuses for casting purposes,
though it will neutralise penalties due to natural
light, to a maximum of 5% + 1% / Rank. The volume counts as direct shadow for Star & Shadow
Mages. If the lighting conditions are lower than
that provided by the spell, no effect will be apparent. Note that because light is only being suppressed, it may still pass through, and no shadows
are generated outside the volume. If it is possible to
see through a Darkness, everything beyond it is
normally visible. This spell can engender silhouettes of lit objects against the darkness, though not
create shadows. Any of this volume may be overridden by a higher ranked Spell of Light, or neutralised (back to original conditions) by an equal
rank.
Decay (G-3)
Range: 15 feet + 15 / Rank
Duration: Immediate
Experience Multiple: 100
Base Chance: 50%
Resist: None
Storage: Investment, Ward, Magical Trap
Target: Object
Effects: The Adept may cause an amount of food,
produce or beverage to quickly age, moulder, spoil
and rot. Upon casting the spell the targeted matter
will decay, causing parasitic fungi to spring forth,
and an odour of decay to prevail. The spell may
ruin up to 1 cubic foot of food and drink (+ 1 cubic
foot / Rank). If a double or triple effect is achieved
the amount of food that is spoiled may be doubled
or trebled. Once affected by the spell the food and
drink will thereafter be inedible.
Fear (G-4)
Range: 15 feet + 15 / Rank
Duration: Immediate
Experience Multiple: 350
Base Chance: 20%
Resist: Active, Passive
Storage: Investment, Ward, Magical Trap
Target: Entity
Effects: The Adept instills in the target an uncontrollable fear. Unless the target successfully resists
they must roll on the Fright Table (§54.1). If a
double effect is achieved, the Adept may choose to
modify the Fright Table roll up or down by an
amount equal to the rank of the spell. If a triple
effect is achieved the Adept may modify the Fright
Table roll by twice the rank of the spell. See the
Fright Table for the exact results of the Fear.
Harming Entity (G-5)
Range: 15 feet + 15 / Rank
Duration: 10 second + 10 / Rank
Experience Multiple: 200
Base Chance: 20%
Resist: Active, Passive
Storage: Investment, Ward, Magical Trap
Target: Entity
Effects: Unless successfully resisted, the Adept
causes the target intense pain for the duration of
the spell. The target must check to see if their
concentration is broken and must subtract 10 (+ 3 /
Rank) from their Strike Chances whilst suffering
the pain. The difficulty multiplier for the Concentration Check is dependent on the Rank of the
spell:
Rank

Multiplier

0–5
6–10
11–15

3.0
2.5
2.0

16–19 1.0
20
0.5
No actual damage is inflicted as a result of this
spell. Note that Mind Mages will be somewhat
unaffected by this spell, and may halve the reduction to their Strike Chances.
Hypnotism (G-6)
Range: 15 feet
Duration: Concentration: maximum 5 minutes + 5
/ Rank
Experience Multiple: 200
Base Chance: 40%
Resist: Active, Passive
Storage: Investment
Target: Entity
Effects: The Adept may lull an entity that is within
range into a trance-like state in which they will be
subject to suggestion. The spell may not be cast
over a totally hostile entity. Once the subject has
been hypnotised, the Adept may make suggestions
(provided that they can communicate verbally with
the subject) that will be readily accepted unless
they directly conflict with the subject’s best interests. The subject will remain suggestible so long as
the Adept maintains concentration and the subject
remains in range. The subject will continue to
implement implanted suggestions for 3 hours (+ 3 /
Rank) after the suggestions have been made, even
when no longer hypnotised. The subject will never
have any idea where the suggestion that it is implementing came from.
Igniting Flammables (G-7)
Range: 15 feet + 15 / Rank
Duration: Immediate
Experience Multiple: 150
Base Chance: 20%
Resist: Passive
Storage: Investment, Ward, Magical Trap
Target: Object
Effects: The Adept may call forth fire and cause
flammable material (cloth, paper wood, and similar
items, but not flesh) to burst into flames. The material will thereafter burn normally and the flames
may be extinguished by normal means.
Mind Cloak (G-8)
Range: Self
Duration: 1 hour + 2 / Rank
Experience Multiple: 250
Base Chance: 30%
Resist: None
Storage: Potion
Target: Self
Effects: The Adept creates a cloak around their
own mind so that their thoughts cannot be detected
or read. This spell does not prevent the Adept’s
presence or emotions from being detected, but their
mind will simply not appear to be there when an
attempt is made to “read” it.
Protection Against Were-Creatures (G-9)
Range: 15 feet
Duration: 30 minutes + 30 / Rank
Experience Multiple: 200
Base Chance: 20%
Resist: None
Storage: Investment, Ward, Magical Trap
Target: Area
Effects: The spell creates an invisible Circle of
Protection with a radius of 15 feet (+ 1 / Rank)
which will not willingly be crossed by any Werecreature or Shapechanger in beast form unless they
successfully resist the circle’s effects upon first
encountering it. Even if the Circle’s effects are
resisted, the Were-creature or Shapechanger will
be discomforted while within the Circle, and will
have their Strike Chances reduced by 10. If the
circle is seen with the use of Witchsight, it will
appear as a glowing red circle, similar to a ring of
fire.
Storm Calling (G-10)
Range: Works at any range
Duration: 60 minutes + 30 / Rank
Experience Multiple: 200

98

27 COLLEGE OF WITCHCRAFT
Base Chance: 40%
Resist: None
Storage: Investment, Magical Trap
Target: Special
Effects: The Adept may summon any storm front
which may exist anywhere in sight. Upon reaching
the spot occupied by the Adept at the time of casting, the storm front will slow and finally cease
moving and begin a downpour (snow, rain, hail,
sleet, or whatever else that the GM feels the clouds
may contain). Generally a storm front can be seen
for 20 to 30 miles. If no front can be seen the spell
may still be cast, but the Base Chance is reduced
by 20. The storm front will take [D × 3 - 1 / Rank]
minutes to arrive. Once the duration has expired,
the weather will gradually return to normal over a
similar amount of time.
Summoning Enchanted Creatures (G-11)
Range: 5 miles + 1 / Rank
Duration: Immediate
Experience Multiple: 200
Base Chance: 20%
Resist: None
Storage: Investment, Magical Trap
Target: Entity
Effects: The Adept may summon 1 enchanted
fantastical creature (+ 1 per 5 or fraction Ranks).
Only creatures that are native to the area may be
summoned. It will take them D10 minutes (15
seconds / Rank), minimum 1 minute, to arrive and
they will be uncontrolled when they do arrive. If
more than 1 creature is summoned, all must be of
the same type.
Walking Unseen (G-12)
Range: 1 foot + 1 / Rank
Duration: 1 hour + 1 / Rank
Experience Multiple: 100
Base Chance: 60%
Resist: None
Storage: Potion, Investment, Ward, Magical Trap
Target: Entity
Effects: The target of this spell may move unnoticed, not invisible. This means that it will not
transmit light. As a consequence the target will cast
a shadow, which may or may not be noticed, depending on the lighting conditions, etc, and will
have a reflection in a mirror or other reflective
surface. However, the target may not be noticed
even if another entity is looking directly at them.
An entity will get a Perception check to notice the
target if the target becomes invasive on the entity’s
senses (e.g. standing next to the entity and putting
their hands over the entity’s eyes). Note that a
Crystal of Vision or similar means of viewing is
considered direct viewing and is affected by this
spell. If the target, or the target’s possessions, are
touched by another entity, or an entity’s possessions, then the spell is broken. Although not truly
invisible, the target may be detected by using
magical means to detect invisible entities (e.g.
Witchsight).
Wind Whistle (G-13)
Range: Self
Duration: D10 hours
Experience Multiple: 100
Base Chance: 40%
Resist: None
Storage: Investment, Potion
Target: Self
Effects: The Adept is able to create a wind over an
open space of up to 100 feet (+ 100 / Rank) in
diameter, centred upon themselves. Outside of this
area the wind will fade back to the prevailing wind
over half again the distance. The wind will build up
over [D - 2] minutes and the Adept must choose
before that time which direction the wind will
blow. The speed of the wind is determined by a
D100 roll as follows:
Dice

Velocity

01–10
11–25
26–50

35 mph
15 mph
10 mph

25 mph
51–75
35 mph
76–90
91–100 45 mph
The Adept may add or subtract a number equal to
the Rank of the spell from the dice roll used to
determine velocity. This need not be done until
after the dice have been rolled and the result ascertained. If a double or triple effect is achieved the
Adept may add or subtract double or treble the
Rank of the spell. If the resulting wind is over 30
mph missile fire will be affected, reducing Base
Chances by the (wind speed / 2) but extending
ranges by a similar number of hexes if firing with
the wind, or reducing them respectively if firing
into the wind.

27.5 General Knowledge Rituals
Finding Familiar (Q-1)
Duration: Special
Experience Multiple: 250
Base Chance: 40% + 4% / Rank
Resist: None
Target: Animal
Cast Time: 1 hour
Material: A piece of food acceptable to the type of
animal being summoned
Actions: Concentration
Concentration Check: Standard
Effects: The Adept may attempt to summon a small
animal that will serve them as a familiar. The type
of animal is chosen by the Adept and may be any
natural, unenchanted, small animal such as a cat,
dog, bat, rat, toad, weasel, falcon, owl, goat, monkey, trout, etc, and must be native to the area in
which the summoning is performed. If the summoning is successful, an animal of the chosen sort
will arrive at the Adept’s location in (25 - Rank)
minutes. The ritual allows the Adept to communicate with the animal when it first arrives. The
Adept must promise to feed and protect the animal.
The GM should roll a reaction check for the animal. If the result is Enraged, the animal attacks, if
Belligerent it leaves immediately. If neither of
these results are achieved, the animal agrees to
serve the Adept as a familiar. Regardless of the
result, the Ritual confers no further ability to communicate with the animal. If the Adept fails to feed
the familiar on a regular basis, or mistreats it in any
way, the familiar may run away, and a new familiar must be found. The familiar will serve the
Adept to the best of its ability, warning them of
danger, and so forth. If the Adept is unable to
communicate with the familiar magically it will
attempt to warn them by tugging at their cloak,
whimpering, or whatever, as appropriate. If the
familiar is killed, the Adept suffers [D + 5] points
of damage in the form of a magical backlash. This
damage may not be resisted. An Adept may only
have one familiar at any one time. A familiar is not
an enchanted creature. The range of the summoning caused by this ritual is 1 mile (+1 / Rank).
Tarot Reading (Q-2)
Duration: Immediate
Experience Multiple: 500
Base Chance: Special + 3% / Rank
Resist: Special
Target: Special
Cast Time: 30 minutes
Material: 78 card Tarot deck
Actions: Laying out & reading Tarot cards
Concentration Check: None
Effects: The Adept may read the tarot to gain insight and information. The Tarot may be used in
one of four ways, and only one of these four options may be chosen per reading. Once one of these
options has been successfully implemented, a new
reading must be begun in order to implement another. There is no Backfire except as specifically
noted. The four options are:
Divining Aspects The Adept may use the Tarot to
attempt to divine the Aspect or Aspects of an entity
that is present for the entire ritual and within 5 feet
(+1 / Rank). The entity may actively but not passively resist the reading. The Base Chance of the
99

reading is 40% and if successful, the Tarot will tell
the Adept the entity’s basic Aspect (autumn air,
lunar, death, etc), and whether the entity is light or
dark aspected. Failure will result in no sensible
answer and Backfire in an incorrect reading.
Divining Enchantment The Adept may use the
Tarot to attempt to determine if an entity or object
is currently, or had been recently, under the effects
of a spell. The object or entity must be present for
the entire duration of the ritual, and be within 5
feet (+ 1 / Rank). The ritual may not be resisted.
The Base Chance of the ritual being successful is
45%. The Base Chance is reduced by 5 for every
week or part thereof since the spell that is being
divined was cast. Permanent magic (e.g. invested
items still with charges) or spells currently in effect
carry no modifier. The Adept gains knowledge of
those spells that fall within their cast chance.
If the Adept can divine the spell, its exact name
and college are revealed. If the spell is noncolleged in origin, its general effects are revealed.
Divining the Future The Adept may use the Tarot
to attempt to learn something about future events.
The Adept must decide on a question to be posed
or a general course of action being considered
before attempting this divination. The GM may
make the reading as simple or as complex as they
desire, but in all cases the information gained
should be vague.
The Base Chance of successfully Divining the
Future is 20%. If the Adept fails this option, the
reading will be gibberish and obviously a failure,
but if a Backfire occurs, a sensible but otherwise
false, reading will be gained.
Questioning the Dead The Adept may attempt to
communicate with the spirit of a deceased entity
provided that they occupy the place that the entity
died or was buried. The Adept may only attempt
this if they are aware that the place they occupy
was the site of the entity’s death or burial. The
Base Chance of the spirit responding to the Adept’s
questioning is 10%. If the spirit responds the Adept
may ask it questions and interpret its answers by
turning over cards. Only questions that can be
answered with yes or no should be asked, and the
spirit’s answer is indicated by the orientation of the
card turned. The dead can only provide knowledge
of events that transpired while they were alive.
Once the dead initially respond they will continue
to answer all questions until dismissed, or the
entire deck has been used.

27.6 Special Knowledge Spells
Blessing Crops (S-1)
Range: Sight
Duration: 1 year + 1 / Rank
Experience Multiple: 125
Base Chance: 40%
Resist: None
Storage: Investment
Target: Area
Effects: The spell increases the richness of the soil
of 1 acre (+ 1 acre / Rank). For the duration of the
spell everything grown in that soil will be proof
against locusts, droughts, flooding, frost, and other
natural disasters. This spell will also dissipate the
effects of a Spell of Blighting Crops which has
previously been cast on the target area of this spell.
Blessing/Curse on Unborn Child (S-2)
Range: Sight
Duration: Until birth of target’s child
Experience Multiple: 200
Base Chance: 20%
Resist: Active, Passive
Storage: Investment, Magical Trap, Potion
Target: Entity
Effects: The Adept may mar or bless any unborn
child whose mother is in sight while she is pregnant. The Adept may increase or decrease any one
characteristic of the child by 1 (+ 1 per 3 or fraction Ranks). This spell may only be cast on the
same unborn child more than once if it is cast by

27 COLLEGE OF WITCHCRAFT
different Adepts, and is used on different characteristics. The spell may raise characteristics above
normal racial maximums. If cast so as to curse, it is
a Major Curse and may only be removed before the
child is born. Note that if this spell is made into a
potion, the target of the spell is the imbiber. The
imbiber may only passively resist the effects of the
potion’s magic.
Blessing Livestock (S-3)
Range: Sight
Duration: 1 month + 1 / Rank
Experience Multiple: 150
Base Chance: 45%
Resist: None
Storage: Investment
Target: Livestock
Effects: The spell may be cast on up to 5 (+ 1 /
Rank) livestock that are within sight. These animals will then be resistant to natural disorders,
such as rabies, dysentery, worms, and hoof and
mouth disease for the duration of the spell. This
spell will also dissipate the effects of a Spell of
Pestilence which has previously been cast on the
targets of the spell.
Blighting Crops (S-4)
Range: Sight
Duration: 1 year + 1 / Rank
Experience Multiple: 125
Base Chance: 45%
Resist: None
Storage: Investment
Target: Area
Effects: The spell causes 1 acre + 1 / Rank of land
within sight to become sour and lose fertility.
There is a 20% (+ 1 / Rank ) chance of future crops
failing while this spell is in effect. Those years that
the crops do not fail, they will be stunted and approximately half a normal yield will be obtained.
This spell is a minor curse. This spell will also
dissipate the effects of a Spell of Blessing Crops
which has previously been cast on the target area
of this spell.
Cat Vision (S-5)
Range: 15 feet + 15 / Rank
Duration: 1 hour + 1 / Rank
Experience Multiple: 100
Base Chance: 60%
Resist: None
Storage: Investment, Ward, Potion
Target: Entity
Effects: The Adept causes the target to develop
vision similar to that of a cat. Everything will
appear monochromatic (i.e. shades of grey) and it
is difficult to accurately estimate distance. The
higher the Rank, the less of a problem this will be.
Some amount of light must be present for this
vision to operate. The range of the vision is 50 feet
(+ 10 / Rank).
Controlling Animals (S-6)
Range: 15 feet + 15 / Rank
Duration: Concentration: maximum 1 hour + 1 /
Rank
Experience Multiple: 100
Base Chance: 20%
Resist: Passive
Storage: Investment
Target: Animal
Effects: The Adept controls the actions of one
normal and unenchanted animal, bird or aquatic,
that does not successfully resist. The creature will
serve the Adept as long as they maintain their
concentration. If the Adept chooses to release the
animal or has their concentration broken, the creature may attack them or flee. The chance to cast
this spell is reduced by 5 if the Adept cannot
Communicate with the creature. If the Adept cannot make eye contact, the Base Chance is also
reduced by 5.
Converse With Animals (S-7)
Range: Self
Duration: 1 hour + 3 / Rank
Experience Multiple: 50

Base Chance: 60%
Resist: None
Storage: Investment, Potion
Target: Self
Effects: The Adept may communicate with any
natural and unenchanted, animal, bird, or aquatic.
Whether this communication is verbal or symbolic,
and to what extent the communication may be
carried is left up to the GM’s discretion. The Adept
must specify at the time of casting what particular
type of animal, bird or aquatic is to be conversed
with. The spell must be re-cast to speak to another
type of animal, bird, or aquatic.
Creating Plague (S-8)
Range: 15 feet
Duration: 1 day + 1 / Rank
Experience Multiple: 200
Base Chance: 20%
Resist: Active, Passive
Storage: Investment, Ward, Magical Trap, Potion
Target: Entity
Effects: The spell infects any one target with any of
the following diseases:
Rank

Disease

0–5
Measles
6–10
Consumption
11–15 Typhoid
16–18 Bubonic Plague
19–20 Pneumonic Plague
The target will not die of the disease, but will become habitually ill and all who come in contact
with them (except the Adept who cast the spell)
may contract a potentially fatal dose of the disease.
In effect, the target becomes a carrier. This spell is
a major curse. Note that if this spell is made into a
potion, the target of the spell is the imbiber. The
imbiber may only passively resist the effects of the
potion’s magic.
Creating Restorative (S-9)
Range: Touch
Duration: Immediate
Experience Multiple: 200
Base Chance: 30%
Resist: None
Storage: Potion
Target: Water
Effects: The spell creates out of drinkable water a
potion which, when imbibed, subtracts 2 from
Endurance and repairs
4 lost Fatigue. The amount subtracted from Endurance is increased by 1 and the amount of Fatigue
repaired is increased by 2 per Rank. The fatigue so
restored may have been lost through damage or
tiredness, including spell casting. The potion will
only restore lost Fatigue. This spell can be prepared in two ways:
1. The Adept can turn water into a restorative
potion that will last 2 minutes (+2 / rank).
2. The Adept may spend an hour and burn oils
costing 500sp to make a potion with the same
effects that will last indefinitely.
The effects of drinking the potion may be resisted.
The Endurance damage caused by this potion may
be healed by normal means.
Damnum Magnatum (S-10)
Range: 20 feet + 15 / Rank
Duration: Special
Experience Multiple: 600
Base Chance: 5%
Resist: Active, Passive
Storage: Investment, Ward, Magical Trap
Target: Entity
Effects: The Damnum Magnatum is a Major Curse
and may take one of three forms, as chosen by the
Adept. The Damnum Magnatum may normally
only be removed by the use of a Remove Curse
Ritual, by a Counterspell cast by the Adept that
laid the curse, or by the death of the target. This
spell may not be dissipated.

100

Affliction The Adept may choose to torment or kill
their target. If the effects of the Affliction are intended to be deadly, the target may not die as a
result of the curse before (24 - Rank) hours have
passed. The Adept’s player states what the Affliction is to do, and then the exact effects and results
must be decided by the GM. In addition to the
normal ways of lifting a curse, afflictions may have
durations or conditions worded into them, in which
case the curse is lifted when the duration expires or
the condition is met. Players should note that Afflictions are particularly capricious, and can never
be relied upon to operate in precisely the same
manner twice. Some sample Afflictions are:
1. Target begins to age at 10 years per day. Target
may die of old age. Once the curse is lifted the
target will age backwards to their correct age, at
the same rate.
2. Target contracts a deadly disease (including
open running sores) that may not be cured by the
arts of a Healer.
3. Target is transformed into a frog or other small
creature (but retain their own mind). Condition: the
Curse may be lifted by the kiss of a member of
royalty of the opposite gender.
4. Target is cursed with Lycanthropy (random
species).
5. Target will fall into a century long sleep (see
Hibernation, College of Ice Magics S-6).
Ill Luck Add two times the Rank of the Adept with
this spell to any percentile dice roll involving the
target’s use of their abilities. This may never be
applied favourably. Note that this is an addition to
the dice roll, not a subtraction from Base Chances.
Doom A Doom is a pronouncement, by the Adept,
upon an event that will occur in the target’s future,
such as: “You will die by the hand of a loved one.”
The statement, which must be indefinite, will come
true in not less than (24 - Rank) weeks. The Doom
remains until it is fulfilled, and may not be removed by a Remove Curse Ritual, or even by the
death of the target, unless that death fulfils the
Doom. The target is immediately aware of the
nature of the Doom, and its wording. A Doom may
be modified, so as to decrease the severity, make
the time factor longer, etc., by the casting of a
modified Doom on the same target, by an Adept
with Rank in this spell at least equal to the Rank at
which the original Doom was cast. The exact effects of the Doom must be decided by the GM, and
players should note that two Dooms, even if
worded the same, need not have precisely the same
effects.
Earth Tremor (S-11)
Range: 15 feet + 15 / Rank
Duration: 5 seconds + 5 / Rank
Experience Multiple: 350
Base Chance: 20%
Resist: None
Storage: Investment, Ward, Magical Trap
Target: Area
Effects: By the use of this spell the Adept causes
the very earth to pitch and roll uncontrollably as
though in a tremendous earthquake. The area that
may be affected is a one hex area of ground (+ 1 /
Rank). Any Entities within the Area must roll less
than or equal to 1 × AG to retain their footing.
Those who fail to remain standing fall prone immediately and may not rise for the duration of the
tremor. Objects within the Area will tend to topple
and roll around. If the spell is cast under part of, or
all of, a building, wall, or other such construction,
significant structural damage will occur, probably
causing partial or total collapse.
Hex (S-12)
Range: 15 feet + 15 / Rank
Duration: 1 day + 1 / Rank
Experience Multiple: 300
Base Chance: 20%
Resist: Passive

27 COLLEGE OF WITCHCRAFT
Storage: Investment, Ward, Magical Trap
Target: Entity
Effects: By use of this spell, the Adept curses the
target with ill-fortune. Unless the target resists, all
their Base Chances, Strike Chances, and their
Magic Resistance are reduced by the Rank of the
spell (1 if unranked). This spell is a minor curse.
Hellfire (S-13)
Range: 10 feet + 5 / Rank
Duration: Immediate
Experience Multiple: 650
Base Chance: 5%
Resist: Active, Passive
Storage: Investment, Ward, Magical Trap
Target: Entity
Effects: This sulphurous fire attacks 1 human-sized
target for every 3 (or fraction) Ranks. The target’s
Magical Resistance is reduced by 5 (+ 1 / Rank).
The spell does D10 ( + 2 / Rank) damage to each
target. If a target successfully resists, they suffer
only half damage (round up). Double damage add
an additional 1 / Rank damage and triple damage
adds an additional 2 / Rank damage.
Instilling Flight (S-14)
Range: Touch
Duration: Concentration: maximum 30 minutes +
30 / Rank
Experience Multiple: 350
Base Chance: 20%
Resist: None
Storage: None
Target: Object
Effects: This spell enables the Adept to instil a
possession of up to 5 lbs (+5 / Rank) with the
power of flight. The spell will dissipate if the object stops being a possession of the Adept, the
Adept loses concentration, or if the object is broken. The Adept may cause the object to fly at 20
miles per hour (+ 2 / Rank). It will take off and
accelerate up to full speed, or halt and land, in a
single pulse. The object may support 150 lbs (+ 50
/ Rank) in addition to its own weight. Naturally
flexible or fragile items gain sufficient strength and
rigidity to support the load. Any object or entity
that falls from the flying object will move off in a
random direction. If the object is about to crash
into a surface, it will attempt to land, although
some surfaces may be inappropriate for this (lava,
sheer walls, etc.).
Mass Fear (S-15)
Range: 10 feet + 15 / Rank
Duration: 30 seconds + 10 / Rank
Experience Multiple: 400
Base Chance: 10%
Resist: Passive
Storage: Investment, Ward, Magical Trap
Target: Area
Effects: The spell instills in all entities within
range, other than the Adept, and those who successfully resist, an unreasoning and uncontrollable
fear. All entities that fail to resist must roll on the
Fright Table (see §54.1).
Pestilence (S-16)
Range: Sight
Duration: 1 month + 1 / Rank
Experience Multiple: 150
Base Chance: 45%
Resist: Special
Storage: Investment
Target: Livestock
Effects: The spell may be cast on up to 5 (+ 1 /
Rank) livestock that are within sight. All livestock
so cursed that do not resist (individually) are infected (see §4.7). Any new stock which comes into
contact with the infected stock while the curse is in
effect must also resist (individually) or become
infected. This spell is a minor curse on each individual. This spell will also dissipate the effects of a
Spell of Blessing Livestock which has previously
been cast on the targets of this spell.
Skin Change (S-17)
Range: Touch

Duration: Until dispelled by the appropriate counterspell
Experience Multiple: 350
Base Chance: 30%
Resist: Passive
Storage: Special
Target: Entity
Effects: The Adept may enchant an animal pelt or
skin so that anyone who touches, or is touched by
the “inside” will turn into the type of animal to
whom the pelt originally belonged, but will retain
their own mind and memories. The spell is, in
effect, stored in the pelt or skin, and may be retained unused for an length of time dependent upon
the Rank of the spell:
Rank

Duration

0–6
1 week (+ 1 / Rank)
7–12
2 weeks (+ 2 / Rank)
13–19 1 month (+ 1 / Rank)
20
permanent until used
The wearer of the pelt may only resume their own
form by having the Special Knowledge Counterspell of the College of Witchcraft cast over them.
The pelt is destroyed by the process of returning
the wearer to their original form. This spell may be
used to enchant the pelts or skins of sentient entities, but if used for that purpose its Base Chance is
reduced by 20%. Note that if a backfire result is
achieved when casting this spell, the pelt or skin is
destroyed in addition to any other backfire effects.
The Skin Change spell itself may be stored by
Investment (so that it may later be triggered on a
pelt) but not by other means. The “inside” of a pelt
is the side that was closest to the animal’s body.
The skin must be applied to the target for at least
30 seconds before they will change, where upon
the target gets to resist its effects.
Virility (S-18)
Range: 5 feet + 5 / Rank
Duration: 2 hours + 1 / Rank
Experience Multiple: 100
Base Chance: 30%
Resist: None
Storage: Investment, Ward, Magical Trap, Potion
Target: Male Entity
Effects: The spell is cast over a male entity and
greatly increases the target’s virility. In addition
the chance of the target’s female partner conceiving is increased by 5 (+ 5 / Rank). Note that if this
spell is made into a potion, the target of the spell is
the imbiber.
Wall of Thorns (S-19)
Range: 15 feet + 15 / Rank
Duration: 30 minutes + 30 / Rank
Experience Multiple: 150
Base Chance: 20%
Resist: None
Storage: Investment, Ward, Magical Trap
Target: Area
Effects: The Adept causes a wall of tangled briars
and thorns to spring forth from the ground. The
wall is 10 feet high × 20 feet long × 2 feet thick.
The Adept may increase the length or height by 1
foot per Rank. The wall may not be cast on top of
an entity. A human-sized hole may be made in the
wall by causing 20 points of damage in one area.
This damage need not come from a single attack.
All “A” class weapons and most “C” class will
have little effect on the tough springy vines. An
entity forcibly pressed against the wall, or attempting to force their way through it will suffer [D + 2]
damage for every Pulse they are so engaged. This
damage is entirely physical, and armour may protect against it. An entity can normally force its way
through the wall in 6 Pulses. The wall is very
dense, and may not be seen through.

27.7 Special Knowledge Rituals
Controlling Weather (R-1)
Duration: 8 hours × Rank (minimum 1)
Experience Multiple: 300
Base Chance: 30% + 3% / Rank
101

Resist: None
Target: Area
Cast Time: 1 hour
Material: None
Actions: Dance
Concentration Check: None
Effects: The Adept may change one or more of the
three components that make up the weather by
performing a ritual dance. The three components of
weather are: Precipitation / Cloud Cover, Temperature, and Wind. The GM should consult the
Weather Table and advise the player of the current
level of each of the three components before they
start dancing. The Adept may change one component by Rank / 2 (round down), or two components
by Rank / 3 (round down) levels each, or all three
components by Rank / 4 (round down) levels each.
The changes are independent and may be in any
direction. The weather will change gradually over
30 minutes (l / Rank) per level shifted, and all three
components will change simultaneously. The area
of the effect is circular and the diameter is 2 miles /
Rank (minimum 2). This ritual counts as Strenuous
activity and the Adept will lose Fatigue. This ritual
may not Backfire.
Creeping Doom (R-2)
Duration: Special
Experience Multiple: 450
Base Chance: 20% + 4% / Rank
Resist: Special
Target: Entity
Cast Time: 2 hour
Material: 13 bones
Actions: Carving bones
Concentration Check: None
Effects: The Adept creates 13 Rune-bones by carving the appropriate maledictions into bones from
an entity of the same race as the target. The Adept
then buries the bones beneath the dwelling of the
entity that they wish to curse. It is best if the victim’s name is carved in the bones as well. If the
intended victim’s name is not carved on the bones,
and there are 1 or more other entities inhabiting the
dwelling, there is a 20% (- 1 per Rank) chance that
the curse will settle on someone other than the
intended victim. For each month that the bones
remain in or under the victim’s dwelling, they must
make a Resistance Check, the Base Chance for
which is composed of the victim’s Endurance
multiplied by the Difficulty Multiplier of the resistance. The Difficulty Ratings are:
Rank

Multiplier

0–5
4.0
6–10
3.0
11–15 2.5
16–18 2.0
19–20 1.5
If the victim fails to resist, they suffer a wasting
disease and lose [D - 4] Endurance for purposes of
future resistance (only). If they fail to resist for
three straight months they die.
Dead Man’s Candle (R-3)
Duration: Special
Experience Multiple: None Base Chance: Automatic
Resist: None
Target: Materials
Cast Time: Variable
Material: As detailed
Actions: As detailed
Concentration Check: None
Effects: By means of this ritual the Adept creates
an horrific and evil candle. Only the darkest of the
Wicca would ever perform this Ritual. The Adept
makes a Dead Man’s candle by severing the left
hand of a convicted murderer who has been hung.
The hand must be severed during a full moon and
wrapped in a burial shroud. It must then be dried in
the sun until desiccated. The Adept must render the
fats and oils from the body of a stillborn baby, so
that the hand can be coated with them and a candle
made. The wick of the candle is then made from
the hair of the same murderer. The Adept says

27 COLLEGE OF WITCHCRAFT
words of darkest power over the candle. Thereafter, it may be lit as part of any spell or ritual of this
College and will increase the chance that the spell
or ritual is successful by 20, provided that the ritual
is being performed with malign intent. This Ritual
may not be Ranked, and it always works, if it is
correctly performed. A Dead Man’s Candle will
burn for about 10 hours before it is no longer usable, and may be extinguished and relit an indefinite number of times.
Hand of Glory (R-4)
Duration: Permanent
Experience Multiple: None Base Chance: Automatic
Resist: None
Target: Severed hand
Cast Time: Variable

Material: Murderer’s hand
Actions: As detailed
Concentration Check: None
Effects: This gruesome ritual creates an amulet of
great and malign potency. Many Wicca consider
this ritual to be evil and it would certainly never be
studied or performed by a “White Mage”. To successfully perform the ritual, the Adept must sever
the right hand of a convicted murderer who has
been hung. The hand may only be severed during
the new moon and must be wrapped in a winding
sheet. It must then be dried in the sun and the blood
entirely removed. When the desiccated hand is
worn as an amulet, thereafter, it will subtract 10
from the Cast Check of creating any Plague,
Blight, or Curse. This Ritual may not be Ranked,
and it always works if it is correctly performed.

102

Summoning Animals (R-5)
Duration: Immediate
Experience Multiple: 150
Base Chance: MA + 5% / Rank
Resist: None
Target: Animals
Cast Time: 1 hour
Material: None
Actions: Concentration
Concentration Check: Standard
Effects: The Adept may summon a number of
small animals equal to the Rank of the ritual (1 if
unranked). The animals that the Adept attempts to
summon must be native to the area. The animals
are not controlled in any way when they arrive.

28 SKILLS

28 Skills
A character may acquire and refine skills during a
campaign. They can hone their talents in a series of
interrelated non-magical and quasi-magical abilities, which combine to form a single skill. A character’s degree of talent is measured by their Rank
in a skill. They begin with the simplest abilities at
the lowest Ranks, and gain the more difficult ones
as they progress through the Ranks. Their percentage chance of successfully performing tasks associated with a skill will increase as their Rank becomes higher.
The possession of a skill does not necessary imply
any character traits associated with that skill.

28.1 Acquiring and Using Skills
The rudiments of a skill are learned by dint of hard
practice and diligent study. A character must spend
a good deal of time and effort before they can use a
skill at novice level (Rank 0). The character’s
ability with a skill can improve only if they continue to work with it during and between adventures.
Any skill may be acquired at Rank 0 at a variable cost of Experience Points and 8 weeks of
game time.
All eight weeks must fall within a period of six
game months. Time spent on adventure may not
count toward the necessary eight weeks.
The method by which a character learns a skill
affects the Experience Point cost to acquire that
skill or to increase the character’s Rank.
If the character is taught by someone of greater
Rank in the skill, decrease any Experience Point
cost by 10%. If the character learns from a book,
verbal descriptions, or practises with someone of
equal or lesser Rank in the skill, any Experience
Point cost is unmodified. If the character practices
with no useful outside assistance, any Experience
Point cost is increased by 25%. The availability of
qualified teachers, and the fees they charge the
character for their services, are left to the discretion
of the GM. Some skills have additional requirements (e.g. literacy) before learning some ranks.
Check each skill for details.
A character may attempt to employ a skill any
number of times during a day.

The use of a skill does not, in and of itself, prevent
a character from using the same or any other skill
immediately afterwards. However, a character
might suffer adverse effects (for example, lose
Fatigue Points) while executing a skill, which
would inhibit their ability to act.
The use of a skill is rarely automatic.
A character usually has a chance of failure when
using a non-magical skill. Unless the ability is
described as an exception to this rule, the maximum chance to succeed with it is never greater
than 90 (+ Rank)%. A character always fails to use
an ability if the roll is greater than the modified
chance or 100 (regardless of Rank).
Some of the abilities associated with the various
skills are quasi-magical.
The following are the only quasi-magical abilities
to be found in the skills section: Alchemist, Astrologer, Healer, Herbalist, Ranger Bump of North.
Supervision of subordinates
The possessor of a Skill, other than an Adventuring
skill, is able to supervise the work of subordinates
in that Skill. The supervisor may instruct and supervise a number of subordinates equal to their
Rank. Subordinates must be practising the same
Skill as their supervisor and may themselves be
supervising underlings, thus creating a “chain of
command”. A subordinate may be replaced by a
work-gang. A work-gang is a group of up to ten
labourers working as a team. Labourers may not
supervise others. A character need not supervise
their maximum number of subordinates or labourers, and may work in proportion to their unused
supervision capacity.
Example
A character with Rank 6 in Artisan (Carpenter), may instruct up to 6 other Carpenters or 6 workgangs (up to 60 labourers), or some combination thereof.
If they were supervising 2 Carpenters and 1 work-gang,
they would only be using half their supervision capacity,
and could themselves work about half of the time.

Expert Knowledge
The possessor of a skill, other than an Adventuring
skill, also gains an in-depth knowledge of the field
associated with their skill. This is equivalent to
having Knowledge in that skill

103

28.2 Knowledge (area)
This is a skill that can be taken many times — once
for each area of knowledge. A character with this
skill knows most of the common lore and traditions
concerning their chosen area. An area may include:
a particular city or territory, a culture, an historical
period, or a race, or species. In addition, an area of
knowledge may be taken from the Philosopher
skill. If this is done, the area is equivalent in size to
a Sub-field, and any Subfields except Advanced,
Experimental or Ancient are available as areas of
knowledge.
A character is limited to the knowledge available to
their culture. The knowledge held by the character
may not be entirely factual, and may contain certain popular misconceptions or superstitions. This
skill mostly gives the character a much wider general knowledge about their area, some history of it,
and perhaps some biographical knowledge of famous figures associated with it, both historical and
contemporary. This skill is entirely one of knowledge, and confers no special ability to perform a
craft or trade.
Generally there is no success percentage; the GM
simply gives far more information regarding a
certain topic to a character who has knowledge of
that area. If there is doubt as to whether or not a
character should know something from their specific area, the Base Chances are:
Rarity of Information

Base Chance

Common
WP + 70%
Uncommon
WP + 40%
Rare or Obscure
WP + 10%
These chances may be further modified by the GM
to reflect the individual rarity of the knowledge. A
character will not know the theories behind the
lore.
If a character learns an area of Knowledge that is
also a Philosopher Sub-field, and that character is,
or becomes, a Philosopher, the area of Knowledge
may be used as the appropriate Sub-field. See the
EP cost table note A (§55.2 ) for details on Ranking.

29 ADVENTURING SKILLS

29 Adventuring Skills (Ver 1.2)
These skills may be ranked as with any other skill.
The only differences are that all characters start
with swimming, climbing, stealth and horsemanship at Rank 0, and if the skill is used conspicuously during an adventure it can be ranked once
without the need for training time, but there must
be a tutor with a similar skill who is present to
advise the character on the technique they should
employ.

29.1 Climbing
This skill allows a character to climb anything
from walls to mountains without the aid of specialised equipment, if this is at all possible. The Base
Chance to use this skill is (4 × MD + 8 × Rank [structure height in feet / 10])%. A character using
this skill should make a roll at approximately 20’
intervals, but if the climb is especially difficult,
every 10’. Note that the GM may modify the formula in certain instances.
A climber suffers ([Height of fall (in feet) / 10]
Squared) Endurance Points when they fall.
Various items of equipment may be used to improve a character’s chance of climbing as follows:
1. Climbing Claws add 15% to BC but have no use
for rock climbing where hands are more use.
2. Rope allows the user to climb the structure making only one roll but are only useful where ropes
may be practically used.

29.2 Horsemanship

The actual occurrence must be decided by the GM
and should become worse the farther the roll is
above the modified percentage.
If the GM judges the rider has totally lost control
of their mount, the rider may take no other action
until they have regained control (presuming they
manage to stay mounted).
Using horsemanship while in combat may be done
in combination with any other Action. A trained
rider receives certain abilities as they rise in Rank:
Rank
3

May use two-handed weapons

Rank
5

May fire a missile weapon or cast a spell
while moving

Rank
7

May use two one-handed weapons at
once

29.3 Flying
Flying is the skill of performing aerial manoeuvres
using magical flying. As a rule aerial combat is
difficult. Flying is an adventuring skill.
A character may always take off, fly, or land in an
appropriate manner and reasonable conditions, and
under such circumstances no roll is necessary. Note
that landing appropriately is not precise. The success chance to perform a complex aerial manoeuvre with precision is (3 × AG + 10 × Rank). This
base chance may be modified by the following:

A character will use horsemanship to direct animals which they ride. A character may use their
horsemanship with any animal or monster which
they would ordinarily ride (such as horses, donkeys, camels, elephants, etc.). Enchanted or Fantastical monsters do not necessarily fall into this category, and the GM must make rulings governing
these situations.

Environmental conditions. 0 to -50
Type of flight used.
+10 to -50
Speed.
0 to -m/hr
Flying into an obstacle causes up to [D + (relative
speed in miles per hour / 10) squared] endurance
damage. The nature of the obstacle may reduce the
damage. Specific Grievous injuries (normally C
class) may also be incurred. See Climbing (§29.1)
for falling (as opposed to flying) damage.

The character’s player will roll percentile dice
whenever their horsemanship is called into play. A
character’s horsemanship is equal to [(modified
AG + WP) / 2 + Rank × 8], round down.

As a rule of thumb, an airborne clothed humanoid
who falls through the air drops 350ft in the first
pulse, 650ft in the second, and 1000ft in each subsequent pulse.

The type of mount a character is riding will modify
their horsemanship as follows:

Note that a speed of one mile per hour is equal to
30 yards per minute in the chase sequence and 1.5
hexes per pulse in combat.

Donkey
-10 Palfrey
+15
Mustang †
-12 Warhorse † -5
Quarterhorse -10 Camel
-15
Dire Wolf
-10 Mule
-8
Draft Horse
-5
Pony
+10
Elephant
-10
†Rating unless trained by rider; in that case, 0.
The GM should also take into account the familiarity the character has with the individual animal
type and apply modifiers thereby (i.e. the first time
a character finds themselves atop a camel should
be worth at least an additional - 15).
A character’s horsemanship is called into play
whenever they wish their mount to perform an
unusual or difficult action. Any mount can be
directed into moving at a walking pace or even a
brisk trot; an unusual or difficult action would be
to break into a gallop or charge, jump an obstacle,
etc. During combat, horsemanship is called into
play during every Pulse to a) keep the mount controlled, b) regain control if it is lost, and c) direct
the mount to take any specific Action. Remember
only a Warhorse can be directed to enter into Close
Combat by its rider, and all other mounts will only
attack if directly assaulted.
A successful roll will result in the mount obeying
the directions of the rider. A roll above the modified percentage but less than the modified percentage plus the rider’s WP indicates the mount either
does nothing or continues to do whatever it was
doing. A roll above both of these indicates the
mount will either disobey the rider, buck, attempt
to throw the rider, or some other unpleasant result.

A trained magical flier receives certain combat
abilities as they rise in rank.
Rank
3
Rank
5
Rank
7

May use two-handed weapons
May fire a missile weapon or cast a spell
while moving
May use two one-handed weapons at
once

29.4 Stealth
A character can use stealth to move as soundlessly
and unobtrusively as possible.
A character may use their stealth ability only if
they have adequate cover (i.e. space in which to
conceal or obscure themselves) in the area they
wish to traverse, they are appropriately clad (e.g.
not in plate armour or luminescent clothing), and
they are not currently under observation by the
entities from whom they are attempting to conceal
their presence.
The GM will roll percentile dice to determine if a
character is able to use their stealth ability successfully. The GM only makes such a check if there is
a reasonable possibility that the character could be
detected. The GM makes one check each time the
character attempts one continuous action, or each
time an unexpected change of condition has a
significant effect upon the character’s chance of
remaining hidden (e.g. one of the entities under
surveillance heads for a room which happens to be
through the doorway in which the character is
hidden). The GM may modify the success percentage.
104

A character’s base chance of using their stealth
ability is (3 × Agility + 5 × Rank + Thief Rank + 2
× Spy Rank + 2 × Assassin Rank)%. The greatest
Perception value of the entities who may be able to
discover the character using the stealth ability is
subtracted if those entities are unaware of the character’s presence, or three times that Perception
value if they are.

29.5 Swimming
This skill is required in order to perform any actions in the water. All player characters start off
with Rank 0. This, under good conditions, will
allow the character to tread water in order to stay
afloat. The higher the rank, the more the character
will be able to do until they are at the stage where
they can swim like a fish and survive even in adverse conditions.
Base Chance
The base chance for swimming is PS + AG + EN +
8 × Rank and is modified by the following (all
adjustments cumulative):
Wearing no or little clothing +10
Encumbered (per pound)
-1
Water Temperature
+5 to -25
Water Conditions
+10 to -25
May not swim freely
-10 to -50
Other modifiers may be applied by GM as appropriate. An unsuccessful skill roll does not imply
drowning (yet) but the character could be in serious
trouble. If they are trying to float and the roll is
failed then they need to make another successful
skill roll in order to stay afloat. Two failed skill
rolls implies they are underwater, holding their
breath, without preparation.
If an Adept is attempting to cast then they can do
so, within the restrictions of their College, if
breathing water or if they make a successful skill
roll. A concentration check (3 × WP) may also be
required in adverse conditions.
Breath Holding
The base time a character can hold their breath is
(current EN / 3 + swimming Rank / 2) pulses
rounded up. This time is doubled if a Pass Action
is used in the previous pulse to prepare.
Drowning
Once that time is expired then the character must
make a 5 × WP check in order to continue holding
their breath. At the end of subsequent pulses, the
WP factor is reduced by 1 until the roll fails.
At that point the character starts drowning, taking
physical damage at a rate of D10 EN per pulse
until death or rescue. A drowning character needs
to make a 2 × (WP + swimming rank) check before
being able to perform useful activity as above.
Sight and Communication
The character can see PC hexes in clear water. This
is halved in lakes and rivers because of algae and
silt.
Communication is by sign language, or a range of
one hex if speaking.
Movement Rates
Swimming TMR = (Land TMR + Rank) / 3. Walking on the bottom (if weighted) = Land TMR / 3.
Swimming is generally a hard or strenuous activity
unless the entity concerned is an aquatic.
Characters that are encumbered by non-buoyant
materials descend at the following rates:
Unencumbered to 5 lbs 0 ft per pulse
5–10 lbs encumbrance
1 ft per pulse
10–15 lbs
2 ft per pulse
15–20 lbs
3 ft per pulse
20–25 lbs
4 ft per pulse
25+
5 ft per pulse
Unencumbered characters floating to the surface
(e.g. if unconscious) do so at 1 ft per pulse.

30 ALCHEMIST

30 Alchemist (Ver 1.1)
Almost all natural chemicals can be combined into
a variety of useful mixtures by expert hands. The
potions which will be in most demand by characters will be those that affect the bodily functions of
humanoids. The effects of these potions range from
stimulation and depression of emotions to deadly
poisons. In a sense, alchemy is a “poor man’s
magic”; it is more cost-efficient in affecting the
actions of beings than the use of mana, albeit not as
easily applied to the victim.
There are five main areas of study within alchemy.
The first is that of chemical analysis, the ability to
determine the effects of ingestion or application of
a given liquid substance. The others are: standard
chemicals, medicines and antidotes, poisons (including venoms) and potions. The creation of a
potion requires the aid of an Adept or a Healer.
As a character gains experience in the field of
alchemy, they will increase the efficacy of the
mixtures they produce. The character will also
decrease the cost of goods (to manufacture).

30.1 Restrictions
An alchemist must know how to read and write in
one language if they wish to advance beyond Rank
0.

30.2 Benefits
An alchemist gains the ability to analyse chemicals at Rank 0.
An alchemist may identify a liquid by its type (e.g.
medicine, poison). If the liquid is not a common
one, the alchemist must spend (110 - 10 × Rank)
minutes using the proper equipment to analyse the
liquid’s type.
If a liquid to be analysed is particularly wellknown to the alchemist (e.g. water or wine), they
will recognise it almost immediately. If an alchemist wishes to determine the exact nature of a not
readily identifiable substance, the GM rolls D100.
If the roll is equal to or less than (Perception + 8 ×
Rank), the alchemist is told the common name of
the substance in question (e.g. hemlock, quicksilver). If the roll is greater than the success percentage, the GM either informs the alchemist that they
are not sure or gives an incorrect answer. The
greater the roll, the more likely the GM is to give
false information.
An alchemist can injure themselves while working with dangerous chemicals.
Whenever an alchemist uses or analyses a potentially dangerous liquid, there is a chance that they
will come in contact with some of the substance.
The GM incorporates the accident chance into any
other alchemy-related percentile roll; should there
not be one, they roll D100. The chance of no accident is (70 + 2 × Rank + Manual Dexterity)%. If
the roll is within the span of numbers for accident,
the alchemist suffers from the chemical. A roll of
100 always causes an accident.
Example
An alchemist character with a Manual
Dexterity of 17 and of Rank 3 would have a 7% chance of
failure. Any roll from 94 to 100 will cause the alchemist to
have an accident.

The GM will determine the exact effects upon the
unfortunate character. The minimum damage will
be from formaldehyde type chemicals, which will
cause about 1 Damage Point and causes blisters.
The maximum damage from a non-magical liquid
will be from something on the order of non-dilute
hydrochloric acid, which will cause about 12 Damage Points per
pulse, and possibly permanent bone and tissue
damage. The effects of certain chemicals are described in the following sections. Unless either the
GM or the player have a fair knowledge of chemistry, the alchemist should restrict themselves to
common liquids.

If the alchemist is dabbling with dangerous chemicals without using the proper equipment (see
§30.3), double the chance of accident. If an alchemist is working in their lab they may prevent damage due to chemicals after the first pulse (unless
they are incapacitated during the first pulse) by
pouring the appropriate counter-agent upon the
affected area.
If a combination of chemicals forms a gas or a
solid, the character’s Agility value is substituted
for their Manual Dexterity when rolling for accident.
An alchemist can mix standard chemicals at
Rank 3, and may add one additional ability to
their repertoire at Ranks 5, 7 and 9.
An alchemist chooses their additional ability from
the following: medicines and antidotes, poisons
(including venoms) and potions.
The ability to mix standard chemicals allows the
alchemist to produce mixtures which can prove
useful on expeditions.
An alchemist may produce well-known chemical
combinations (e.g. oil and vinegar, water and anything) at any Rank. The standard chemicals ability
allows the alchemist to perform most distillations
and extractions, and mix the simplest of compounds.
For example, an alchemist can produce Greek Fire
and methane with the standard chemicals ability.
The components for 12 ounces of Greek Fire
(enough to fill a grenado) cost 600 Silver Pennies.
Enough methane to fill a grenado can be manufactured at a cost of 300 Silver Pennies. If a creature is
directly hit by a grenado filled with Greek Fire,
that creature will suffer [D + 7] Damage Points per
Pulse until the flames are extinguished (the virtue
of Greek Fire as a weapon is that it sticks to the
target). A partial hit will cause [D - 3] Damage
Points per Pulse; if a shield is interposed between
target and grenado, the shield catches fire, though
the intended target suffers no more than 2 Damage
Points. A methane grenado creates a ball of fire in
the hex in which it detonates and the adjacent six
hexes. Any creature in one of those hexes will
suffer [D - 3] Damage Points, but will be able to
avoid further damage by exiting the fire hexes
(methane is not a persistent inflammable).
Whenever an alchemist wishes to manufacture
standard chemicals, they must spend [D + 7] hours
in a laboratory and pay for the components. The
quantity mixed does not affect the time required,
but an alchemist is limited to the manufacture of
one end product during a given laboratory session.
An alchemist can produce standard chemicals for
the use of local businessmen (e.g. embalming fluid
for the undertaker), and earn between 50 and 75
Silver Pennies per full week of labour. Alternately,
they may produce chemicals which are likely to be
put to illegal uses (e.g. a corrosive for iron) or
manufacture addictives (e.g. cocaine, heroin). The
alchemist must discover an outlet to sell such
chemicals, and the return on the goods is up to the
GM’s discretion.
The cost for a standard chemical will range from 1
Silver Penny for a quart of flammable oil to 2000
Silver Pennies for a fluid ounce of non-dilute hydrochloric acid. The GM should scale the costs of
other chemicals appropriately.
Medicines and antidotes are used to cure a being suffering from either disease, fever or poison.
An alchemist may manufacture three types of
medicine:
• bactericide (remedy for disease)
• antipyretic (remedy for fever)
• salve (remedy for skin inflammation)

105

A bactericide or antipyretic must be ingested, while
one dose of salve can cover up to two square feet
of skin. Salves will cure minor skin inflammations,
irritations (eg sunburn, rashes & insect bites) and
may cure severe burns.
Whenever a being uses a medicine to counteract an
affliction from which they are suffering, the GM
rolls percentile dice. If the roll is equal to or less
than (8 × Alchemist’s Rank + User’s Endurance),
the user is completely cured. If the roll is above the
success percentage, the user subtracts 10 from their
next dice-roll to see if they naturally recover from
their infection (see §4.7). The failure of one medicine to work has no effect upon any subsequent
medicines used by a being.
A medicine costs (150 - 10 × Rank) Silver Pennies.
An alchemist can produce up to three doses per
day.
When an alchemist manufactures an antidote, they
must specify the type of poison they are negating.
Natural poisons are classified by their source.
Thus, a snake antidote will cure all poison from
snakes, and so on. Synthetic poisons (those manufactured by alchemists) are cured by an antidote
from an alchemist of equal or higher Rank than the
alchemist who created the poison. When a being
ingests the proper antidote, the poison in their
system will no longer affect them.
An antidote costs (250 - 15 × Rank) Silver Pennies.
An alchemist can produce up to three doses per
day.
Poisons cause damage when introduced into the
blood stream of a being.
Poisons come from two sources: those which occur
in nature (venoms from animals and plants) and
those which are created in a laboratory (synthetic
poisons). An alchemist may distill venoms and
synthesise poisons. A venom is distilled from
either the poison sacs of a poisonous animal (the
most common being a snake), or from certain
plants. An alchemist may distill [D - 1] doses of
poison from poison sacs. The amount they may
distill from plants depends on the type of plant
(GM’s discretion). An alchemist requires (11 Rank) hours to distill one dose of venom from
either source. The cost of a poison plant or sac is
(750 + 150 × average damage per Pulse) Silver
Pennies, and there is no cost for the distillation
process.
Venoms come in two forms: nerve agents and
blood agents. Nerve agents work quickly (doing
damage every Pulse) while blood agents (such as
arsenic) work over a long period of time. The
effects of slow acting (blood agent) poisons function in the same manner as infections except there
is no roll for cure. The damage a being will suffer
from a dose of nerve agent venom is equal to the
damage it would suffer from the venom of the
source animal or plant.
An alchemist may also manufacture synthetic
poisons (both venoms and paralysants) in their
laboratory. A synthetic venom will do [D + Rank 5] damage points per Pulse and costs (1000 - 75 ×
Rank) Silver Pennies to manufacture. If a synthetic
paralysant is used to affect a being, the Willpower
Check of the victim is (4 × Willpower + 20 - 5 ×
Rank). A synthetic paralysant costs 1750 - (60 ×
Rank) Silver Pennies to manufacture. An alchemist
can produce up to three doses of synthetic poison
per day.
Potions are created by an alchemist with the aid
of either an Adept or a Healer.
Potions are designed to create a specific effect
when imbibed by a being. They are manufactured
in one-use doses and the entire dose must be swallowed for the effect.
Magical potions are created by the concerted efforts of an Adept and the alchemist (who may be

30 ALCHEMIST
the same person). Any spell or talent which the
Adept knows and which is designed to affect only
the Adept or some facet of their own person may
be imbued into a potion. It takes two whole days of
continuous combined effort to create the potion. It
is successfully created if at the end of the time the
player rolls less than (10 × Alchemist’s Rank) +
Adept’s Rank with the spell or talent). A roll above
this indicates the potion is useless and the process
must be repeated with new ingredients. The effect
of a successful potion for the imbiber is as if the
Adept had already made a successful Cast Check
and the spell had taken effect. The workings of
magical potions are immediate. The cost to manufacture a magical potion is equal to [(Experience
Multiple of spell or talent × 20) - (Alchemist’s
Rank × 10)].
An alchemist and a healer working together may
create a healing potion (again, they may be the
same person). The potions possible and their Base
Value are:
Base Healer Ability

Value

Cure Disease

600

Cure Fever
600
(Graft) Skin Salve
650
Neutralise Poison (specify type) 700
Cure Endurance Points
1500
Prolong Life
2500
The time required to produce the potion is the same
as a magical one, and the equation to see if the
process was successful is (10 × Alchemist’s Rank
+ 3 × Healer’s Rank). If successfully created, the
potion will act on the imbiber as if a healer of the
healer’s Rank was attempting to heal them (any
success rolls must still be attempted). The cost to
manufacture a healing potion is (Base Value - 50 ×
Alchemist’s Rank) Silver Pennies.
The duration of a potioned talent, once imbibed, is
1 hour × Rank of Talent (minimum 1).

30.3 Costs
An alchemist will be able to better perform their
skill when using the proper equipment or when
working in a laboratory.
It costs 2500 Silver Pennies to construct a lab, and
1000 Silver Pennies per year to maintain it. An

106

alchemist can only manufacture medicines, antidotes, poisons, or potions or distill venoms in a lab.
A laboratory may be rented at a cost of 15 Silver
Pennies per day.
The chance of an alchemist correctly analysing a
chemical (see §30.2) is increased by 10 when they
perform the analysis in a laboratory.
The GM and an alchemist player should scale costs
and effects of improved alchemical support material to the above rules.
An alchemist must purchase the components
necessary to manufacture each product.
The costs for poisons and potions are given with
their rules. All costs given are for one creation
attempt; if that attempt fails, new ingredients must
be purchased.

31 ARMOURER

31 Armourer (Ver 1.3)
31.1 Restrictions
The skill is related to that of weaponsmith, and an
armourer who is a more skilled weaponsmith expends only three-quarters of the necessary Experience Points to acquire or improve this skill. The
reverse is also true.

harder to affect, and this is reflected in the number
of ranks an armourer must have to do so. Also,
some of the attributes have maximums (e.g. the
Agility Modifier may not be decreased beyond 0).
The ranks required and the attribute maximums
are:

An armourer’s progress in their skill is inhibited by
a low Manual Dexterity, and aided by a high Manual Dexterity. An armourer has an increased Experience Point cost of 5% for each point of Manual
Dexterity less than 16. An armourer decreases their
Experience Point cost by 5% for each point of
Manual Dexterity greater than 20.

Weight 1/2 a factor per 3 full ranks. Never lighter
than WT 1. This attribute may not be affected for
the cloth, leather or mithril categories.

31.2 Benefits

Agility Modifier 1 per 6 full ranks. Never better
than 0.

An armourer acquires the ability to make one
category of armour every two ranks.
Some categories require other categories as prerequisites and cannot be learned before their prerequisites. All armourers begin with the cloth category
at rank 0.
Categories

Prerequisites

Cloth
Leather (leather, soft leather and
furs)
Scale (scale and full scale)
Chain mail
Partial plate
Plate I (full plate and heavy plate)
Plate II (improved plate, jousting
armour)
Dragon skin

None
Cloth
Cloth
Cloth
Chain
Chain
Plate I

Scale,
Leather
Mithril
Chain
Additional categories may be gained without increasing in rank by spending 5,000 Experience
Points and 4 weeks training time per category.
These costs are discounted by 25% if the armourer
has reached rank 8, or by 50% if they have reached
rank 10.
An armourer can make increasingly effective
armour as their rank increases.
An armourer may positively affect any of the 4
attributes of armour (Weight factor, Protection,
Agility Modifier and Stealth Modifier) or any
combination thereof. Some of the attributes are

Protection +1 per 4 full ranks. This attribute may
not be affected for cloth, furs or soft leather, and no
more than 1 additional point of protection may be
added to hard leather.

Stealth Modifier +1% per rank. Never better than
+5%.
Note: These effects are not cumulative. For example a rank 7 armourer could make a suit of armour
with 1 less weight factor and 1% better stealth, or
1/2 a weight factor less and 1 point more protection, or any of the other non-cumulative combinations. An armourer may always make a suit of
armour at a lower effective rank than their true
rank.
Armour statistics shown on the Alusian Armour
Chart are for armours manufactured with an effective rank of 0, i.e. of the mass-produced, off the
peg variety. The armourer who made them may
have been of greater rank but the level of skill used
was elementary.
The time and cost required for an armourer to
construct a suit of armour is dependent on the
effective Rank used and the category of armour.
Time The number of days required to construct a
suit of armour in a properly equipped and staffed
workshop is effective rank plus the base number
for the armour listed below:
Categories

Time

Cloth or leather
Scale
Mail
Partial plate
Plate I

5 days
10 days
20 days
25 days
30 days

107

Plate II
35 days
Dragon skin
20 days
Mithril
30 days
An Armourer with greater rank than the effective
rank being applied may reduce the construction
time by (Rank - Effective Rank) days (minimum 1
day).
The fitting time for the armour (the time spent with
the armourer by the wearer-to-be) is a number of
hours equal to the base number of days (e.g. 20
hours to fit a suit of mail). The hours are not consecutive and may be reduced by (Rank - Effective
Rank) hours (min. 1).
Cost 80% of the Base Cost as shown on the Armour Chart x (Effective Rank + 1) silver pennies.
Note that this is the cost to the armourer, not the
sale price.
Fixing and Modifying Armour
The time taken to repair a suit of armour damaged
by a Grievous blow, or to modify a suit to fit a new
(but appropriately sized) wearer, is usually the
same as the original fitting time.
An armourer is treated as a merchant of their
armourer rank when attempting to buy or value
armour from categories with which they are
familiar.
If the armourer is not familiar with an armour
category they act as a merchant of half their rank
(rounded down).

31.3 Costs
An armourer can only perform their skill in a
properly maintained workshop.
It costs 2000 silver pennies to construct a workshop and 500 silver pennies per year to maintain it
with tools and materials. A basic tool kit will cost
(100 + 100 × Rank) silver pennies. It costs only
20% of the above amount to add to a weaponsmith’s workshop so as to make it usable by an
armourer as well. The reverse is also true. A workshop may be rented at the rate of 10 silver pennies
a day.

32 ARTISAN

32 Artisan (Ver 2.0)
Artisan is not a skill in itself. It is a heading under
which many craft, trade and service skills may be
grouped, as they all function in a similar manner
within the rules. Any, or indeed all, of the skills
listed below may be learned, but each is learned
and ranked as a separate skill. Knowledge of any of
the artisan skills confers no benefit with regard to
learning or ranking another.

Hatter / milliner design and construction of men’s
and women’s hats.

32.1 Artisan Skills

Leatherworker making of leather into garments
and articles such as saddles.

The most common skills under the heading of
artisan are:
Apiarist bee breeder, keeper, honey collector.
Artist, Painter formal, perspective painting.
Artist, Sculptor sculpture design and construction.

Hunter / trapper use of gin, or other animal traps,
skinning, animal collection.
Husbander breeding, raising, tending of animals.
Lapidary gem and semi-precious stone cutting,
polishing, finishing.

Locksmith design and construction of simple
locks.
Lumberjack tree felling, hewing, sawing for
planks, replanting.

Barber / coiffeur simple hairstyling through to
massive structures of hair, wire and glue, made to
resemble ships in full sail, castles, etc.

Mason stone quarrying, cutting, finishing and
fitting.

Basketmaker basket weaving, making wicker.
Blacksmith iron smelting and fashioning, simple
founding.

Miner quarrying, prospecting, tunnelling, not mine
design. Musical instrument maker design and construction.

Brewer brewing beer, ale, stout, mead, creating
new recipes.

Papermaker plant collection, pulping, screening,
drying, grading, creating new recipes.

Brickmaker/bricklayer mixing the ingredients for
bricks, using brick moulds, checking integrity,
making brick constructions, designing paving
stones.

Perfumer extraction of scents, perfume recipe
creation, perfume mixing.

Butcher killing, butchering and preparation of
animals.
Carpenter / cabinetmaker joints and wood-joining,
woodworking, making basic wooden constructions,
wooden furniture.
Calligrapher / illuminator fancy script, book
illustration, drafting official documents.
Cartographer / chart maker map and sea chart
making and copying.
Cartwright / wheelwright basic wagon and cart
design, wagon, cart and wheel construction.
Carver / bone / etching / wood carving, engraving, woodcuts, plates for printing.
Caster / pewterer / tinsmith complex and intricate casting by sand, mould or "lost wax" methods.
Making and casting pewter and smelting and fashioning of soft base metals.
Chandler / lampmaker design and construction of
lamps and lanterns.
Charcoaler making charcoal from partially burnt
wood and bones.
Cheesemaker turning milk into curds and whey,
pressing curd to form cheese, creating new recipes.
Clothmaker / fuller / weaver material collection,
cleaning, spinning, weaving, hammering in dirt to
soften, cloth design and creation.
Cobbler / cordwainer shoe and boot design and
construction.
Cook / baker food preparation and cooking, baking, pastry making, creating new recipes.
Cooper design and construction of barrels and
casks.
Dyer / inkmaker extraction of natural dyes, mixing
of mineral compounds to make inks and dyes.
Farmer/gardener ploughing, planting, tending,
harvesting, food crops or ornamental plants.
Fisher fishing, basic net repair, fish identification.
Glass-blower glass mixing, blowing, window
construction, staining.
Gold / silversmith smelting and fashioning of
gold, silver, platinum and other precious metals.

Miller milling grains into flour, millwheel use.

Plasterer mixing and application of plaster.
Potter clay collection and mixing, pottery design
and construction, firing, glazing.
Printer / bookbinder setting printing type and
plates, press operation, binding books.
Roofer / thatcher material collection, bundling,
binding, attaching roofs of thatch, sod, or tile.
Rope / netmaker plant collection, unravelling,
winding and braiding, net design and construction.
Rug / carpetmaker pattern design, material collection, weaving, and finishing.
Sail/tentmaker sail construction, sewing tents,
tarring and waterproofing.
Sailor operating small boats and crewing ships.
Basic sail repair and knots.
Salter salt collection from inland sources or by
evaporating seawater.
Shipwright boat and ship construction from standard designs, not creating new designs.
Tailor/seamstress cutting, fitting, designing and
constructing, men’s or women’s clothing.
Tanner/hideworker/furrier cleaning, scraping,
preserving, grading, leather or pelts, making of
preserved pelts into garments, or garment trims..
Tattooist tattoo design, pigment mixing and implementation.
Taxidermist pithing, preserving, stuffing and
mounting of animals and trophies.
Tinker basic metal implement repair, knife grinding.
Toymaker design and construction of puppets,
dolls and simple mechanical toys.
Undertaker / embalmer funeral preparation and
celebration, body preservation and reconstruction,
mixing embalming fluids.
Vintner manufacture of wine, sherry, port, brandy
and other fermented or distilled beverages, creating
new recipes.

32.2 Benefits
An artisan becomes increasingly more effective at
their skill as their rank increases. An artisan may
always work at an effective Rank lower than their
true rank. Standard items, as shown on the DQ
Equipment List, are manufactured with an effective
Rank of 0. The artisan creating the item may have

108

had a higher rank than this, but the skill used was
elementary. Generally, work produced at a higher
effective Rank will appear better, be more aesthetically pleasing, be more durable, taste better, or
result in a higher yield, as appropriate.
Applicable base
chance
Applicable characteristic
Difficulty modifier

+1% / Rank applied.

+ 1 / 5 full Ranks
applied.
-0.5 / 5 full Ranks
applied.
+5% / Rank applied.
Yield
These effects are not cumulative, but the effective
Rank used may be spread between these aspects.
The benefits of the improved quality will only
accrue if the skill (or item created) is used correctly
and in appropriate circumstances.
Example
A seamstress with Rank 8 in her skill
creates a ball gown at an effective Rank of 8 for a courtesan. She must make the gown out of very high quality
cloth (of an appropriate type) and can either create a gown
which confers +8% on reaction rolls or one that gives +1
PB and +3% on reaction rolls, provided that the courtesan
wears it both correctly and in a situation for which it was
designed.
Example
A carpenter with Rank 7 may build a door
(to resist the gentle ministrations of adventurers) that
either reduces their chance of kicking it down by 7% or is
half a difficulty factor harder to kick in and further reduces
their chances by 2%.
Example
A fisher with Rank 10 may catch 150%
(100 + (Rank 10 × 5%)) of the normal amount of fish, in a
day’s fishing.

No more than one artisan bonus may applied to a
specific Base Chance or Difficulty Modifier, be
gained to any one characteristic, or be added to a
Yield. If there is a conflict the better of the bonuses
may be employed.
Example
If the courtesan in the example above were
to wear her Rank 8 (+1 PB, +3%) ballgown in conjunction
with a tiara she had made at Rank 5 that also grants +1 PB,
she would still only gain +1 to PB.

If the skill (or item created) is used incorrectly or
in inappropriate circumstances then no bonus will
be gained and negative modifiers may apply.

32.3 Time & Cost
The time and cost for an artisan to perform their
skill is dependent on the effective Rank used and
the Base Time required for that skill.
The time required is: (Base Time × ((effective
Rank / 2) + 1)).
Example
If the base time to make a Rank 0 ball
gown is 1 week, then an Rank 7 one will take 1 week × ((7
/ 2) + 1) = 4.5 weeks. The Cost is (80% of Base Cost ×
(effective Rank + 1)) silver pennies. Note: This is the cost
to the artisan, not the sale price.

Exceptions
Those Ranks used to gain extra yield do not count
in the time calculation. Also, half of any Ranks
possessed by the artisan above the rank being used
may be subtracted from the effective Rank in the
time calculation, to a minimum of the base time.

32.4 Artisan as Merchant
An artisan is treated as a merchant of half their
Rank (rounded down) when attempting to buy or
value equipment or materials with which someone
with their skill would be familiar.

32.5 Requirements
An artisan will usually require a workshop, or at
least a toolkit to perform their skill properly. The
cost of tools and basic materials will vary, but will
usually be (100 + (50 × Rank)) silver pennies. An
artisan may not perform their skill at a higher rank
than that of their workshop or tool kit.

33 ASSASSIN

33 Assassin (Ver 1.0)
Assassin is not a skill which should be carelessly
chosen, as the skill is not looked upon with great
favour by members of society (at least until they
need one). Assassins will vary in philosophy and
methods; they may be cold-hearted but not necessarily evil. The GM must allow assassins to practice their art as they wish, and this may result in
solo adventures.

33.1 Restrictions
An assassin must be able to use the sap or garotte at a minimum of Rank 1 before advancing
past Rank 2.

33.2 Benefits
An assassin is trained in using envenomed
weapons.
An Assassin may envenom weapons and use them
in combat without making the 3 × MD check required by others (§6.13 Envenomed Weapons).
An assassin increases their chance of causing a
Grievous Injury as their Rank increases.
If the assassin is attacking in a surprise situation,
their chance of causing a Grievous Injury is increased by 2% per Rank. If an assassin attacks a
victim through a rear hexside during combat, their
chance of causing a Grievous Injury is increased by
1% per Rank. If an assassin attacks a victim
through a front hexside during combat, their
chance of causing a Grievous Injury is increased by
1% per three full Ranks. An assassin does not gain

the above bonuses when engaging in Ranged
Combat.

tion into their description as the roll approaches
100.

An assassin may gain information from a victim
through torture.
The assassin must torture their victim for a period
equal to (Victim’s Willpower / Assassin’s Rank)
hours to try to gain an important secret. The assassin’s chance of forcing the victim to reveal the
secret is ( 10 × Assassin’s Rank - 4 × Victim’s
Willpower)%. If the GM’s roll on percentile dice is
equal to or less than the success percentage, the
assassin gains the exact information they require. If
the assassin fails to gain the desired information,
they may try another torture attempt. A victim can
withstand a number of torture attempts equal to
one-fifth their Endurance (round down) before they
die.

The GM may decrease the success percentage for
difficult feats of memory.

If the information an assassin seeks is not of great
importance to the victim, the GM should decrease
the time required to gain it and increase the assassin’s success chance appropriately.
An assassin is trained to improve their memory.
Whenever an assassin wishes to recall the details of
a place or routine they have surveyed, the GM rolls
percentile dice. If the roll is equal to or less than (5
× Perception + 2 × Rank), the assassin has a perfect
memory of the place or routine. If the roll is greater
than the assassin’s success percentage, the GM
should include more and more erroneous informa-

109

An assassin is able to buy poisons, distilled venoms and acids at cost (i.e. no mark-up) from an
alchemist.
An assassin inflicts increased damage on their
target when attacking through a rear hexside in
Melee Combat.
Increase the damage caused by a blow from an
assassin by 1 per Rank when they strike their target
through a rear hex in Melee Combat.
An assassin increases their chance of knocking
out a target.
An assassin may attempt to automatically knockout
a target when using a sap. They must make a successful strike while attacking from behind or surprise against an unhelmeted opponent.
In addition their chance of knocking out a target
(see §6.9) with a sap is increased by 2% for each
Rank they have achieved in assassin.
An assassin increases their chance of performing any action involving stealth.
An assassin increases stealth (see §29.4) by 2% per
Rank.

34 ASTROLOGER

34 Astrologer (Ver 1.0)
The celestial bodies have a definite, if not entirely
understood, effect upon the lives of the inhabitants
of a DragonQuest world. These Great Powers seem
to impose predestination upon all but the strongwilled, and determine the aspect of each being. The
Sun, the Moon, and the Planets regularly cause
perturbations in the flow of mana; the mighty Stars
affect a world across the vast reaches of space by
their positions relative to it. The study of the purpose and method of the Powers is the science of
astrology.
An astrologer’s main talent is a limited ability to
predict and shape the future. An astrologer will be
able to make clear, general assertions, but will only
be able to give obscure clues when asked for specific details.

34.1 Restrictions
An astrologer must be able to read and write in one
language at Rank 8 if they wish to advance beyond
Rank 0.
An astrologer may not make a general prediction or ask a specific question concerning only
themselves.
One must consult another astrologer in these
weighty matters.
An astrologer may only try once to answer a
particular question or to forecast the outcome of
an event.
Once an astrologer has made a reading (i.e. a determination about the future), they may not seek to
change or influence the reading through their art.
Other astrologers who attempt to read the same
future will receive the same information that the
first astrologer did. A second astrologer may, however, receive some clarification about the first’s
reading.

The results of a reading will affect the pertinent
course of events.
The GM is expected to modify the outcome of an
adventure or happening in their world to conform
with a determination made by an astrologer or by
an astrologer at the behest of a character. The
determination does not preclude the characters’
actions from affecting the outcome of the adventure or event: to the contrary, the GM must interpret the reading as they see fit, and alter a few of
the random dice-rolls engendered by the characters’ actions accordingly.
A prophecy cannot be avoided by the affected
character(s) changing their plans. The doom
(which may be good) will follow them to the undertaking they substitute for that which was predicted. However, if a character asks a specific
question predicated upon a given action, the prophecy need not come to pass unless and until that
action is taken.

34.2 Benefits
An astrologer’s Rank determines how many
beings they can directly affect with a single
prediction.
A being is directly affected by an astrologer’s art
when the GM modifies the result of an action taken
by the being due to a prophecy.
An astrologer can directly affect up to (5 + 10 ×
Rank) beings with a single prophecy. If an astrologer attempts a prediction which would directly
affect more beings than their Rank allows, they
receive no answer.
An astrologer may make (and possibly modify)
a general prediction during a reading.
When an astrologer wishes to make a general prediction about a particular venture or being, the
player (or the GM) actually uses a divinatory technique at their disposal. Such a technique could be
reading the tarot, casting the I Ching, or any mutually agreed upon method.

110

The result of the divination becomes the astrologer’s prediction. If the astrologer does not wish to
make the prediction, they may immediately attempt
to change it. The GM rolls percentile dice, and if
the roll is less than or equal to (5 × Willpower + 4
× Rank - 30), the astrologer makes a second divination (which may not be changed). If the roll is
greater than the success percentage, the astrologer
is stuck with their first prediction.
An astrologer may seek to answer up to Rank
specific questions per month.
When a being poses a specific question to an astrologer willing to attempt an answer, the GM rolls
percentile dice. If the roll is equal to or less than (6
× Rank + 4 × Perception), the astrologer is able to
give a correct answer. If the roll is greater than the
success percentage, they mutter meaningless gibberish.
All answers given to specific questions must be, at
the very least, obscure. The GM may respond with
cryptic poetry, much like the Oracle at Delphi, or
may choose to have the astrologer supply a riddle
(though the Player of the astrologer does not know
the answer themselves).
An astrologer can determine the aspect of a
being after observing them.
After an astrologer has spent (60 - Rank) consecutive minutes observing a being, the GM informs
the astrologer of the being’s aspect.
An astrologer expends Fatigue points when
practicing their art.
Action
Fatigue
Make general prediction
Try to change general prediction
Try to answer specific question
Determine being’s aspect

10
10
17
5

35 BEAST MASTER

35 Beast Master (Ver 1.1)
A loyal animal or monster is likely to serve its
master far better than adventuring comrades ever
will. A beast master is one who trains these creatures to obedience. They take a wild animal and,
from an adversarial relationship, develop a rapport
with it. They train young animals from birth until
they heed their every command. A beast master
will, in almost all cases, become very fond of animals. They will defend them against wanton cruelty and slaughter, and will treat their personal
charges as family.
A beast master will encounter three kinds of animals: the easily domesticated (such as the horse),
the naturally wild (such as the pegasus) and the
intelligent or rebellious creature (such as the unicorn). The latter can never be steadfastly loyal to
the beast master; such creatures always have at
least a subconscious desire to escape. A beast
master can be a slaver if they specialise in training
humanoids.

35.1 Restrictions
A beast master must have at least 15 Willpower.
A beast master will normally use their skill to train
or domesticate animals for their own use. Animals
which spend their lives with a beast master and are
trained by them will be loyal to their master and
serve and protect them as much as possible. If
necessary, an animal can be trained to temporarily
serve another master (if one week of mutual training is undergone), but the animal will always obey
the original master before any new one.
If a beast master’s Rank is 5 or greater, they may
train animals for other people. The being who is
acquiring the trained creature must spend (12 Rank) weeks before it will accept them as its new
master, during which the beast master must be
present at least one day per week. The creature will
heed the beast master’s commands before those of
its new owner for as many years as the beast master’s Rank at the time the creature’s ownership is
transferred.
A beast master of any Rank may domesticate,
rather than train, animals. Such animals can be
commanded by any other person, but will tend to
wander off or revert to their wild state if not super-

vised, tied up, or stabled. Note that horses and
dogs, the most common domestic animals, are
governed by this rule.

35.2 Benefits
A beast master acquires the ability to train one
type of animal or monster at Ranks 0, 5 and 10.
A beast master may acquire the ability to train
additional types of creatures without increasing in
rank by spending 5,000 Experience Points and 4
weeks training per type. These costs are discounted
by 25% if the beast master has reached rank 8, or
by 50% if they have reached rank 10.
A type consists of all creatures listed within one
subsection of the beastiary (e.g. avians). A beast
master may choose, instead, all creatures subsumed
under a single animal family (e.g. canines).
A beast master must spend (12 - Rank) months to
train an animal or monster, or a like number of
weeks to domesticate one.
Creature to be trained is

Time

Easily domesticated
× 0.5
Naturally wild
× 1.0
Intelligent or rebellious
× 3.0
Raised by beast master from adolescence × 0.5
Domesticated by another beast master
× 1.0
Caught in wilderness
× 1.5
The unmodified number of months required is
multiplied by all applicable modifiers. The time to
train a monster or animal is always dependent on
the beast master’s Rank when they begin the process. Any increases in Rank during the training or
domestication period have no effect on the time
required.
Loyalty Checks
A trained animal or monster must make a loyalty
check whenever it recognises that its master is
endangering it, or whenever its master commands
an action that runs counter to its instincts. Whenever a loyalty check is required, the GM rolls percentile dice. The base chance is 2 × beast master’s
WP + (4 × Rank if the creature is intelligent or
rebellious, 6 × Rank if the creature is naturally
wild, and 8 × Rank if the creature is easily domesticated). If the owner is not a beast master, use their

111

WP and the Rank of the beast master when they
trained the creature. If the roll is less than or equal
to this success percentage, the trained creature will
do as its master commands. If the roll is greater
than the success percentage, the creature’s reactions will range from balking to fleeing to turning
on its master, as the roll increases (GM’s discretion).
A domesticated creature must make a loyalty check
if the circumstances described above arise. The
GM rolls D100. If the resulting number is less than
or equal to current master’s WP + beast master’s
Rank, the domesticated creature will perform the
action. If the roll is greater than the success percentage, but less than or equal to twice that percentage, the creature will balk. If the roll is greater
than two times the success percentage, but less
than three times that percentage, the creature will
take flight. If the roll is greater than three times the
success percentage, the creature will turn on its
master. A roll of 100 always indicates that a domesticated creature turns on its master. A roll of 96
through 99 indicates that the creature takes flight if
the success percentage is 47 or greater.
A beast master who intimidates their animals adds
one to their Rank when calculating training or
domestication time, but the GM adds 10 to any
loyalty check dice-roll for one of their animals.
A beast master may train or domesticate as
many creatures as their Rank at one time.
All creatures being trained or domesticated concurrently must be of the same type.

35.3 Cost
A beast master must pay 100 Silver Pennies per
creature trained and 25 Silver Pennies per creature domesticated.
They may halve the cost for upkeep of creatures if
they build a stable. A horse-sized stable costs (500
+ 150 × Stalls) Silver Pennies to construct, and
costs (Stalls) Silver Pennies for repairs after the
first year.

36 COURTIER

36 Courtier (Ver 2.1)
A courtier may be an attendant at and frequenter of
courts and palaces, or merely a most civilised
student of polished and refined manners. Courtiers
learn to survive in the Machiavellian political
situation prevalent in most courts and places of
high estate, and to be obsequious and intimidating
by turns. They may learn musical or creative skills
to enhance their status, and may indulge in manipulation and seduction.

36.1 Restrictions
A courtier pays 10% more EP to increase their rank
if:
• their AG is less than 12,
• their PB is less than 15.
A courtier pays 10% less EP to increase their rank
if:
• their AG is more than 22,
• their PB is more than 20. All modifiers are cumulative.

36.2 Benefits
A courtier gains 2 abilities at Rank 0, and 1 further
ability per Rank. All abilities are usually performed
at the overall Rank of the courtier. However, a
courtier may choose to specialise. If, upon gaining
a new Rank (or an additional ability without increasing in rank), the courtier wishes to forego
gaining a new ability, they may specialise in one of
the abilities that they already possess. That ability
then operates at (courtier’s Rank + 1), maximum
10. A courtier may specialise more than once with
the same ability, gaining Rank + 2, Rank + 3, etc.
Additional abilities may be gained without increasing in rank by the expenditure of 1,000 Experience
Points and 4 weeks of training per ability. These
costs are discounted by 25% if the courtier has
reached rank 8, or by 50% if they have reached
rank 10.
Individual Base Chances are not provided for the
various courtier skills; rather, there is a generic
Base Chance of 3 × appropriate characteristic (+ 5 /
Rank), modified by the GM to reflect the difficulty
of the feat being attempted.
The abilities available to a courtier are:

Bureaucracy an understanding of organisations
and hierarchies, how to get information, which
wheels to oil, and which palms to grease.
Carousing the ability to socialise informally with
persons of all social classes, without being seen as
an outsider. Also includes the ability to drink considerably less than most observers would think.
Compose Music the ability to create musical
works, using instruments that the composer is
familiar with.
Dress sense the knowledge of what to wear, how
to wear it, and when. This skill includes dressing
formally, seductively, or ridiculously, as the occasion and culture require. Also includes what cosmetics and scents to wear, what accessories, and
even when to not wear things.
Entertaining the ability to organise events, ranging from intimate parties, to state dinners, grand
fetes, and balls. The courtier may supervise caterers, and menials, arrange the entertainments, and
will know whom not to seat next to the Duke.
Etiquette the knowledge of what to do, how to do
it, and when. This skill includes courtly graces,
correct forms of address, and which fork to use for
the fish. Etiquette must be learnt separately for
different cultures.
Formal dance a good grounding in formal courtly
dances, particularly suitable for fetes and balls.
Gaming an understanding of the rules of such
recreational pursuits as backgammon, chess, go,
fox-and-geese, nine-mens-morris, and tafl, as well
as various card and dice games.
Hunting & Hawking a familiarity with the practice and styles of falconry, riding to hounds, and
similar courtly sports.
Intimidation the ability to rule subordinates
through terror, and knowing character flaws and
weaknesses. Also includes a good general grounding in methods of personal manipulation, such as
blackmail.
Oratory presenting a point of view or a set of
information in a formal and forceful manner, to an
audience. Includes rhetoric and declamation, and

112

also the ability to handle interjection and questioning.
Play an instrument the ability to play one musical
instrument; the music taught will tend to be mostly
formal and structured. This ability may be taken
several times with different instruments. A courtier
can usually play similar instruments to the ones
they have chosen at (Rank / 2). A Singer is one
who selects Voice as their instrument.
Poetry creating poetry, often of formal and highly
complex structure.
Seduction see below.
Simulate emotions the ability to keep careful
check on the emotions being displayed, so as to
deny observers information (such as when playing
poker), or to give false information (feigned surprise, apparent pleasure).

36.3 Seduction
Whilst seduction may be used to entice an entity
who is compatible with the seducer into a sexual
relationship, it may also be used to create a sense
of friendship and trust, even with a being not sexually compatible with the seducer. The skill mostly
consists of flattery and gentle coaxing, and a seducer will greatly benefit from being skilled at
etiquette, dress sense, dance, playing music, or
whatever is appropriate to the type of seduction
undertaken. Seduction is not a rapid skill, requiring
hours or even days to achieve the desired result.
Often there is no skill check made since the GM
will decide the results of the seduction based on the
character’s Rank and the way the Player describes
the attempted seduction. If a Base Chance is used,
it is seducer’s PB (+ 10 / Rank), modified by the
GM to reflect the difficulty of the seduction. If the
attempt succeeds the seduction is generally successful. If the attempt fails but is close to the Base
Chance the seduction may be attempted again, at a
later time. A particularly high roll indicates that the
target is unimpressed or repulsed by the seducer.
Player Characters are not bound by the result of
seduction attempted on them, but the GM should
give them strong hints as to how their character
feels about the seducer.

37 HEALER

37 Healer (Ver 1.4)
The healer skill is pseudo-magical and healers are
able to cure all physical ills and perform miracles.
It is a highly skilled profession and healers are not
common. However their existence means the
health and life span of people in the DQ world are
considerably better than their medieval counterparts.
A healer’s empathy often gives them a distaste for
causing pain to others.
A healer will charge whatever their client can
afford for their lower Ranked abilities. The charge
for a miracle (the performance of an ability Rank 8
or greater) will normally exceed 2000 Silver Pennies.
For the purposes of the Healer skill, an entity’s
body parts are the head, torso, and each limb. Vital
Organs are the heart, stomach, viscera (liver, small
and large intestines and kidneys), genitalia, brain,
and eyes.
A healer may also use their abilities upon animals
which they have Beastmaster (§35) familiarity.
However non-sentients cannot be resurrected.
The section §4 Health and Fitness details the effects of injury and illness on patients.
Field Operations
Once a healer begins work on curing a patient the
condition of the patient is “stabilised” while that
curing is continuing. This means that no Endurance
or Fatigue loss will occur for the condition that is
being cured. Other afflictions will be unaffected.
After each healing attempt a pulse effectively
passes prior to any other attempt beginning.
Potions & Unconscious Patients
An entity cannot drink a healing potion when they
are unconscious or below zero endurance but one
can be massaged down their throat. The chance of
doing this is equal to the Manual Dexterity + Perception of the person administering the potion, or if
a healer, 90 + Healer Rank. If successful then D10
per 10 points of the healing potion’s curing (round
down) will be received. If the person fails the roll,
the potion is wasted, but no harmful effects occur
to the patient.

37.1 Benefits
A healer gains specific abilities at each Rank as per
the following table:
0.
1.
2.
3.
4.

Empathy, Ranged Empathy (optional)
Cure Infection, Disease, Headaches, Fever
Soothe Pain, Prolong Life
Heal Endurance, Transfer Fatigue
Neutralise Poison, Cure Burns and Repair
Scars
5.
Repair Muscle, Preserve Dead
6.
Repair Bones
7.
Repair Tissues and Organs
8.
Resurrect the Dead
9.
Regenerate Limbs and Joints
10. Regenerate Trunk, Head and Vital Organs
NB. A healer must choose at Rank 0 whether or
not to learn Ranged Empathy.

37.2 Restrictions
A healer must expend as many Fatigue Points as
the Rank at which they acquired the ability they are
using (except Empathy, Ranged Empathy).
A healer may use any of their abilities (with the
exception of resurrection) upon themselves.
A healer must “lay hands” (place their hands) on
an entity on whom they are to use any of their
abilities (except Ranged Empathy).
A healer has the following modifications to their
combat strike chances:
Tactile Empathy (only):
-1 / 2 Ranks
-1 / 5 Ranks

Close Combat strike chance
Melee Combat strike chance

Ranged Empathy:
-1 / Rank
-1 / 2 Ranks

Close Combat strike chance
Melee Combat strike chance

37.3 Ability Descriptions
Empathy
Base Chance: automatic
Time: 5 seconds
When a healer lays on hands they immediately
invoke empathy.
A healer uses empathy to identify which of the
healing abilities is required to heal the patient.
The healer automatically detects the surface emotions of the entity being healed. An entity’s surface
emotions are those which currently occupy their
conscious mind. The GM informs the healer of the
general feelings of the being with which they have
empathy.
Ranged Empathy
Base Chance: Perception + 10 / Rank
Time: 5 seconds
If the healer has learnt Ranged empathy they may
attempt to detect the surface emotions of an entity
no more than (2 × Rank) feet away from them at a
cost of 1 Fatigue Point. If the entity actively resists
then subtract twice the target’s Willpower from
this success chance.
Cure Infection, Disease, Headaches and
Fever
Base Chance: 15 × Rank + Patient’s Endurance
Time: 30 minutes - 2 / Rank
A healer neutralises poisons and cures fevers and
diseases in much the same manner that medicines
and antidotes (see §30.2) do.
If the healing attempt is unsuccessful the patient
subtracts 10 from their next die roll to see if they
naturally recover from their affliction.
Soothe Pain
Base Chance: 90 + Rank
Time: 60 seconds - 5 / Rank
Duration: Rank squared hours
When a healer uses their soothe pain ability, they
numb their patient’s nervous system so that it will
not transmit pain sensations to their brain. The
ability also has a soporific effect upon the patient,
so that they will not inadvertently injure themselves while unable to distinguish hurtful actions.
The GM may, at their discretion, permit the healer
to use this ability as if they had fed or injected their
patient with a local or general anaesthetic, tranquilliser, etc.
Prolong Life
Base Chance: 90 + Rank
Time: 60 seconds - 5 / Rank
When a healer uses the prolong life ability add D10
× (Healer’s Rank + Patient’s Endurance) days to
the life of their patient. A patient’s life may not be
prolonged to over three times their natural life. An
entity with a prolonged life has a reduced chance
of resurrection.
Heal Endurance and Transfer Fatigue
Base Chance: 90 + Rank
Time: 11 minutes - 1 / Rank
Heal Endurance will cure the patient of [D + Rank
- 5] Endurance points. It will not heal damage
associated with a specific grievous injury.
When a healer transfers fatigue the patient gains
one Fatigue point for each point the healer expends
(above the fatigue cost to use the ability).
An entity may never have more Fatigue or Endurance Points than the value of the relevant characteristic and excess points cured have no effect upon
the patient.
Neutralise Poison
Base Chance: 90 + Rank (natural) or 50 - 5 ×
difference in Rank (synthetic)
113

Time: 5 seconds
A healer may neutralise the effects of a natural
venom or the effects of a synthetic poison created
by an alchemist of equal or lesser Rank. If a synthetic poison is produced by an alchemist of greater
Rank they must roll the second Base Chance
above.
Cure Burns and Repair Scars
Base Chance: 90 + Rank
Time: 30 minutes - 2 / Rank
A healer may immediately halt ongoing burning
damage, and prevent scarring from a fresh burn.
They may also reduce the effects of old scars;
minor scars can be completely removed, but major
scars would take (2 × PB Modifier) successful
cures to be removed.
Repair Damage
Base Chance: 90 + Rank
Time: 50 hours - 3 / Rank
A healer may repair torn, damaged, or broken
muscles, bone, tissues and organs. Generally these
abilities will be used to repair the effects of Grievous Injuries.
A Healer may repair all injuries to one body part at
the same time. To use Repair, a body part must be
mostly intact and no more than half damaged.
Regeneration is required for more significant healing.
A healer can act as a cosmetic surgeon. First, they
sedate their patient with the soothe pain ability.
They then slice and reshape the skin, muscles, and
bones which are deemed unsightly, and make them
whole with the appropriate repair ability. Unless
the healer has also learnt regeneration, it is best
that they work with a partner.
Preserve Dead
Base Chance: 90 + Rank
Time: 60 minutes - 5 / Rank
A healer can suspend the time limit on resurrection
by preserving the dead body of a being. Each time
a healer uses the preserve dead ability, the body
will not “age” for a number of days equal to the
healer’s Rank. This ability may be repeated by the
same healer on the same body.
Reattach Body Parts
Base Chance: 90 + Rank
Time: 30 minutes - 2 / Rank
A Healer able to Repair Muscle may reattach a
severed body part. The reattached body part may
be Repaired to full functionality provided the
wound meets the requirements of repair and it was
severed for no more than Rank minutes. Or it may
be Regenerated to full functionality provided it was
severed for no more than Rank hours.
Resurrection
Time: 60 minutes - 5 / Rank
Base Chance: Patient’s Endurance + 8 / Rank
minimum = Rank
maximum = 90 + Rank
regardless of the total modifiers.
Base Chance Modifiers:
+5
+5
-5
-5
-1

healer is life aspected
patient is life aspected
healer is death aspected
patient is death aspected
per year (or fraction) the patient’s life has
been prolonged
-1
per day of regeneration it would normally
require to make the body whole
-10 body is whole but has suffered Damage
Points equal to or greater than twice its
Endurance (including after death damage)
-10 per unsuccessful resurrection attempt since
patient died
Effects: A resurrection will cure the body of all ills
and damage done to it provided that Rank 8 healing or below would have been sufficient had the

37 HEALER
patient been alive. For example, poison and nonspecific wounds will be cured automatically.
If the resurrection is successful, the patient is resurrected with their body whole. Their Endurance
characteristic is decreased by one, although all of
their other characteristics remain as before they
died.
After a resurrection the patient will have 1 Endurance point and 0 Fatigue. The Endurance is considered to be grievous damage and the Fatigue loss is
deemed to be due to tiredness. This means that the
Fatigue loss may only be recovered by sleep, rest,
hot meals or some form of fatigue transfer and the
Endurance loss may be cured by a healer, magic, or
by letting the body heal itself naturally.
If the resurrection is unsuccessful the patient is not
resurrected and their Endurance characteristic is
decreased by one. The body is preserved for one
full day after the attempt. When an entity’s Endurance is reduced to zero or less, that entity may no
longer be resurrected.
If the roll for resurrection is equal to or greater than
(90 + Rank), the healer has summoned a malignant
spirit, rather than the patient’s life-force. The spirit
will drain the healer’s Endurance characteristic by
[D - 5]. The spirit will then return to the netherworld.
Restrictions:
1. A healer may attempt the resurrection of an
entity who is less than 10 × Rank hours dead.
2. A healer must have a body part at least the size
of a torso to attempt the resurrection of an entity. A

healer will not succeed if they attempt the resurrection of a living being from a severed body part
(there is only one life force). If a body is completely destroyed (perhaps burned), which prevents
the resurrection of the entity, the entity may become a revenant.
3. If the patient has wounds that require regeneration (Rank 9 or 10) healing, these need to be healed
separately.
4. Most vital organs will need to be healed prior to
the resurrection otherwise the body will die again
immediately (the notable exception being the
eyes).
5. The healer need not know what the patient
looked like since the healing of the body is governed by its own characteristics. Hence any
changes that had been made to the body (for example facial changes or embedding of items) will be
gone after the resurrection.
6. A player may take no action with their character’s dead body.
7. In rare instances a healer may be able to resurrect a life force into a different body. The resurrected entity has the physical characteristics of the
new body and the mental characteristics from the
life force. It will take some months for the entity to
get used to the new body and this will affect base
chances of physical abilities. The GM will advise
the specifics.
8. A body that has been animated (e.g. Zombie)
may still be resurrected provided it is no longer
animated and all the other conditions have been

met (for example length of death, condition, etc.).
Note that the Healer Preserve Dead will not affect
Zombies.
Regeneration
Base Chance: 90 + Rank
Time: 50 hours - 3 / Rank (refer below)
A healer can regenerate every portion of an entity’s
body including vital organs and severely damaged
or severed body parts. A healer must regenerate
each vital organ or body part separately.
Regeneration will regenerate [Healer Rank] percent of a missing body part or vital organ per day
costing the target [Healer Rank] FT per day. A
Healer must initiate the Regeneration of a body
part or vital organ, the regeneration will then continue unattended until the regrowth is complete or
the healing area sustains further significant damage.
A regenerated vital organ will immediately begin
to function if enough of the rest of the entity’s
body is in working order. Otherwise, the vital
organ will be dormant until the healer can repair or
regenerate the necessary body parts. The regeneration time does not need to be consecutive, but the
damaged part will not function nor continue growing until the regeneration time is complete.

37.4 Potions
A healer can manufacture certain potions in conjunction with an alchemist. See Alchemist (§30.2)
for more details.

37.5 Healer Summary Chart
Ability

Base Chance

Time

Empathy
Ranged Empathy
Cure Infection, Disease, Headaches & Fever
Soothe Pain
Prolong Life
Heal Endurance
Transfer Fatigue
Neutralise Poison
Cure Burns and Repair Scars
Repair Muscle, Bones, Tissues and Organs
Preserve Dead
Reattach Body Parts
Resurrection
Regeneration

Automatic
PC + 10 / Rank
(15 × Rank) + Patient’s EN
90 + Rank
90 + Rank
90 + Rank
90 + Rank
90 + Rank or 50 - 5 × difference in Rank
90 + Rank
90 + Rank
90 + Rank
90 + Rank
special
90 + Rank

5 seconds
5 seconds
30 minutes - 2 / Rank
60 seconds - 5 / Rank
60 seconds - 5 / Rank
11 minutes - 1 / Rank
11 minutes - 1 / Rank
5 seconds
30 minutes - 2 / Rank
50 hours - 3 / Rank
60 minutes - 5 / Rank
30 minutes - 2 / Rank
60 minutes - 5 / Rank
1 week

114

38 HERBALIST

38 Herbalist (Ver 1.1)
Herbalism is an old profession, and herbalists of
varying quality can be found throughout the land.
Whilst most of the magical power of this skill is
derived from the herbs themselves, herbalists will
have quasi-magical abilities to derive the greatest
benefit from the herb. To some extent the ranger’s
healing herbs are a simple aspect of the herbalist’s
skill. Through the use of special herbal mixtures
the herbalist can mimic some of the abilities of
other skills, in particular those of the healer and of
the alchemist. In general a herbalist cannot perform
their skills at short notice since the number of
herbs and preparations required to meet all needs
would be too large to carry, and many preparations
have a limited life. But given enough time a competent herbalist can usually find and prepare a
herbal mixture to deal with most problems.
A herbalist may make some potions in advance but
these potions will only last for a limited time.
Potions are always roughly a quarter of a pint per
dose and weigh, with their container, ½ pound.
Salves with their container weigh ¼ pound. A
herbalist may make up to Rank / 3 (round up)
batches (a single batch may produce multiple doses
if enough herbs are used) of potion or salve simultaneously. Should they attempt to make more potions at once then there is a 50% chance for each
extra batch that all potions being made will fail.
Should the batches being made fail there is a further 30% chance that some form of accident will
occur (GM’s discretion). While the herbalist is
producing potions to their full capacity no other
actions are permitted, but if the herbalist is working at less than full capacity then they can interrupt
their work for training, meeting visitors etc., providing they tend the potions at least every hour.

38.1 Herbs
Locate and Identify Herbs
A herbalist will learn how to locate and identify
herbs and spices growing in any terrain. Their
chance of finding fresh herbs suitable for performing a given skill is (Perception × 2 + Rank × 15 10 × the minimal rank at which they would be able
to perform that skill) + the following modifiers
Rarity
common
uncommon
rare
very rare
Season

-0
-20
-70
-100

spring
summer
autumn
winter
Search Area

-20
-5
+0
-25

The herbalist must state that they are looking for
herbs required for a particular ability before setting
out to search. If the herbalist then rolls over their
modified chance, they will find no useful herbs. A
herbalist may search the same area a number of
times, providing that they search for herbs required
for a different ability each time.
Common and uncommon herbs may be found in
quantities larger than a single dose depending on
conditions, but it would be rare to find sufficient
for more than three or four doses at once.
In general, rare and very rare herbs are found in
small quantities sufficient for a single application,
and these magical herbs are more frequently found
associated with places of mana.
After harvesting herbs there, the herbalist cannot
return at a later date to harvest more, since the
original harvest will almost certainly kill the plant.
If the herbalist does not harvest the herbs and returns at a later time to harvest the herbs, there is a
5% cumulative chance per days delay that the herb
will have died or been consumed.
A herbalist may alway successfully identify herbs
(in fresh or prepared form) used in abilities which
the herbalist can perform. If the herbalist does not
have the ability for which the herbs are used they
still have a Perception + 5% / Rank chance of
knowing what the herb is used for providing that
the herb is the usual one used for that ability, and
not one of the rarer substitutes.
If the herbalist is so far from the environment in
which they learned their skill (for example off
plane) that the plants and herbs are unknown, then
they will not know what abilities the unknown
herbs could be used for.
Herbal Garden
A herbalist may establish a garden where common
and uncommon herbs and spices are grown. Such a
garden will require tending by a herbalist for 1
week every month. If left untended for 2 months
uncommon plants will die or be overgrown and if
left untended for 6 or more months such a garden
will be only as fruitful as wilderness (see above)
for finding herbs.
Preserving Herbs
A herbalist will learn how to preserve herbs and
spices and how to prepare them for maximum
efficacy. Such preparation is simple and will only
take a few minutes. Only common herbs will last
for any length of time. For common herbs preservation is Rank / 2 months, for uncommon herbs
Rank /

Less than ¼ of a square mile (roughly 1
hour of searching per ¼ of a square mile)
Between ¼ and ½ square miles
Between ½ and 1 square mile
Between 1 and 1½ square miles
Over 1½ square miles
each subsequent hour or ¼ square mile
Other

-30

For rare and very rare herbs, area is high
mana
Area is wilderness
Area is settled
Area is civilised
For each rank of ranger skill
or for each rank of ranger skill if the environment is the ranger’s speciality as per
rule §45.2

+25

-20
-10
+0
+10
+5%

+10,
-15
-30
+1%
+2%

In some environments (for example off plane or
tropical jungle) the GM may choose to add extra
modifiers. Also some environments are simply
unsuited to plant growth and a herbalist will not be
able to find herbs.

3 weeks, for rare herbs Rank / 2 days and for very
rare herbs Rank × 2 hours. The herbs will last at
full efficacy for the time indicated but after that
time will decay at a rate proportional to the time of
preservation.

38.2 Benefits
A herbalist gains specific abilities at each Rank as
per the following table:
0.
1.
2.
3.
4.
5.
6.

7.
8.
9.

Herbs, Cooking
Restorative Meal; Cure Infections, Headaches
and Fevers; Relaxation Tea
Cure warts, boils and pimples; Healing Sleep;
Sleep Draught; Animal Perfume; Mild Poison
Enhance Endurance Healing; Fertility Brew
Cure Disease; Sentient Perfume; Strong Poison
Enhance Organ Healing
Enhance Bone Healing; Hallucination Libation; Fantastical Perfume; Multi-part Deadly
Poison
Enhance Characteristic Potion; Spirit Travel
–
Restore Life

115

38.3 Cooking
A herbalist, should they choose to take the artisan
cooking skill, gains a reduction of 25% to their
experience for ranking that artisan skill. In addition
they will be able to use their knowledge of herbs to
enhance their cooking skill and prepare an invigorating broth. As their rank increases a herbalist
who is also an artisan Cook, or acting in concert
with an artisan Cook, will be able to prepare superb
food, providing there are sufficient ingredients,
common herbs and spices available. Even with
limited ingredients a herbalist will enhance the art
of preparing enjoyable food.
Restorative Meal or Broth
At Rank 1 and above the herbalist may include
common herbs that will turn a normal hot meal into
an invigorating restorative meal (capable of restoring fatigue loss due to damage or spell use) that
will add 2 + (Rank / 4, round up) lost fatigue
points, or the herbalist may make a broth or tea that
will restore 1 + (Rank / 2, round up) fatigue points.
The restorative meal or the broth/tea replace a
normal hot meal and if taken in addition to a hot
meal no extra fatigue will be restored. Like a hot
meal, the restorative meal or the broth / tea can be
taken once every 4 hours.

38.4 Healing
A herbalist will be able to prepare herbal mixtures,
and salves that will heal, or enhance the normal
healing processes.
Cure Infections, Headaches and Fevers
At Rank 1 a herbalist may prepare, using common
herbs, antiseptic salves that will disinfect wounds
and cure infections, and potions that will cure
headaches and fevers. These salves and potions
will last Rank months, and it will take 24 hours to
prepare Rank / 2 (round up) doses.
Cure Warts, Boils and Pimples
At Rank 2 a herbalist may prepare, using common
herbs, salves that will cure warts, boils and pimples. These salves will last Rank months, and it
will take 24 hours to prepare Rank / 2 (round up)
doses.
Healing Sleep
At Rank 2 a herbalist may prepare, using common
herbs, a potion that will cause the drinker to fall
into a healing sleep that will heal fatigue or endurance damage at the rate of 4 fatigue per hour or 1
endurance per hour if the sleeper has lost endurance. Note that if the sleeper is recovering endurance they will recover no fatigue. The sleeper can
be woken at any time but must sleep at least 4
hours to receive any benefit at all. These potions
will last Rank months, and it will take 24 hours to
prepare Rank / 3 (round up) doses.
Enhance Endurance Healing
At Rank 3 a herbalist may prepare, using uncommon herbs, healing salves that will enhance the
healing time of wounds. Divide the normal rate by
their rank. For example a Rank 10 herbalist treating a deep sword cut that would normally take
20 days to heal would cause it to heal in 2 days.
This healing will include repair of muscle, ligament, tendon and skin, all without scarring provided the wound is open and the salve can be applied directly to the wound. Note that one preparation of salve will heal one serious wound or a
number of minor wounds. If the herbalist uses
uncommon herbs these salves will last for Rank
days, and it will take 6 hours to prepare Rank / 3
(round up) doses. If the herbalist uses rare herbs
these salves will last for Rank weeks, and it will
takes 12 hours to prepare Rank / 4 (round up)
doses.

38 HERBALIST
Cure Disease
At Rank 4 a herbalist may prepare, using uncommon herbs, potions that will cure diseases (note a
specific disease will require a specific potion).
These potions will last for Rank weeks, and it will
take 12 hours to prepare Rank / 3 (round up) doses.
Enhance Organ Healing
At Rank 5 a herbalist may prepare, using uncommon herbs, potions that will enhance the healing of
bruising and internal injuries to Rank × the normal
rate. If the herbalist uses uncommon herbs these
potions will last for Rank days, and it will take 30
minutes to prepare one dose. If the herbalist uses
rare herbs then these potions will last for Rank
weeks, and it will take 12 hours to prepare Rank / 4
(round up) doses.
Enhance Bone Healing
At Rank 6 a herbalist may prepare, using rare
herbs, potions that will enhance the healing of
broken bones to Rank × the normal rate. These
potions will last for Rank weeks, and it will take 12
hours to prepare Rank / 5 (round up) doses.
Restore Life Elixir
At Rank 9 a herbalist may prepare, using the very
rarest of herbs, a potion that will restore life to a
sentient being providing that the being died not
more than 12 hours previously or if the body was
preserved as per the healer skill within 12 hours of
death, and the body is capable of supporting life.
Thus the body must be substantially whole (ie no
more than an arm or a leg lost) and any fatal
wounds (for example sword thrust through the
heart) must have the appropriate healing salves or
potions applied immediately prior to the restoration
of life. After life has been restored, the being will
be too weak to move for a number of days equal to
the number of hours which they were dead after
which they will regain one point of endurance as
per the healer skill. This potion will last Rank × 3
hours and it will take 30 minutes to prepare a single drop, which is then placed on the tongue of the
patient to restore life.

38.5 Alter Natural Rhythms
A herbalist may prepare herbal mixtures that alter
the normal rhythms of a sentient being’s body.
Relaxation Tea
At Rank 1 a herbalist may prepare, using common
herbs, a potion that will reduce tension and promote relaxation. This effect will last for Rank × 4
hours. This potion will last for Rank weeks, and it
will take 24 hours to prepare Rank / 2 (round up)
doses.
Sleep Draught
At Rank 2 a herbalist may prepare, using uncommon herbs, a potion that will induce sleep for up to
Rank × 2 hours unless the drinker strongly resists
(rolls under 4 × willpower every
10 minutes). This potion will last for Rank weeks,
and it will take 24 hours to prepare Rank / 3 (round
up) doses.
Fertility Brew
At Rank 3 a herbalist may prepare, using common
herbs, a potion that will increase the likelihood of
fertilisation by 1 + 1% per Rank for a period of
Rank days, or reduce the likelihood of becoming
pregnant by 2 + 2% per Rank, for a period of Rank
weeks. These potions will last for Rank months,
and it will take 24 hours to prepare Rank / 3 (round
up) doses. Note these potions will be gender and
race specific.

Hallucination Libation
At Rank 6 a herbalist may prepare, using uncommon herbs, a potion that will cause the drinker to
hallucinate, unless they roll less than (2 × Willpower - 3 × herbalist’s Rank), for up to Rank
hours. The herbalist can usually (75% chance)
control the extent (i.e. subtle, with only a slight
dreamlike character to reality, mild, with reality
moderately distorted, or total, with no limits at all)
and the general nature (i.e. pleasant or unpleasant)
of the hallucination. If the herbalist uses uncommon herbs these potions will last for Rank days,
and it will take 6 hours to prepare Rank / 3 (round
up) doses. If the herbalist uses rare herbs these
potions will last for Rank weeks, and it will take 12
hours to prepare Rank / 4 (round up) doses.
Enhance Characteristic Potion
At Rank 7 a herbalist may prepare, using very rare
herbs, a potion that will allow the drinker to enhance any one characteristic by 2 × (Rank - 6)
points. This has a duration of (Rank 6) hours. Only
one Herbalist characteristic increasing potion may
be in effect on an entity at any one time. There is
no reduction of characteristics or sleep period
required after the duration of the positive effects —
i.e. no down-side. These potions will last for Rank
× 2 weeks, and it will take 48 hours to prepare
Rank / 4 (round up) doses.

38.6 Perfumes and Odours
A herbalist may prepare perfumes that will attract
or repel animals or sentients. Such perfumes will
only mildly modify (plus or minus Rank × 4% to
reaction roll) the reactions of the creature being
affected (except that in this case a low roll indicates the degree of repulsion rather than hostility).
These perfumes are subtle and only if the being
makes the appropriate willpower roll will they
consciously notice the perfume.
The perfumes will not change the basic nature of
the creature (i.e. no matter how nice you smell a
dragon will still kill and eat you, although she may
recommend the herbalist to other dragons as a
supplier of fine condiments). All perfumes will last
(in the bottle) for Rank weeks. Thorough washing
will reduce the duration of a perfume by 10 fold. A
clean cantrip will remove the perfume.
Animal Perfume
At Rank 2 a herbalist may prepare, using uncommon herbs, a perfume that will attract or repel a
given animal, the animal must roll under (WP herbalist’s Rank) to resist the effects of the perfume. The perfume will continue to act for Rank
days once applied. It will take 12 hours to prepare
Rank / 3 (round up) doses.
Sentient Perfume
At Rank 4 a herbalist may prepare, using rare
herbs, a perfume that will attract or repel a given
sentient creature (including character races), the
creature must roll under (2 × WP - herbalist’s
Rank) to resist the effects of the perfume. The
perfume will continue to act for Rank × 3 hours
once applied. It will take 24 hours to prepare Rank
/ 4 (round up) doses.
Fantastical Perfume
At Rank 6 a herbalist may prepare, using very rare
herbs, a perfume that will attract or repel fantastical or magical creatures, the creature must roll
under (3 × WP - herbalist’s Rank) to resist the
effects of the perfume. The perfume will continue
to act for Rank hours once applied. it will take 48
hours to prepare Rank / 3 (round up) doses.

116

38.7 Poisons
A herbalist may prepare toxic mixtures of herbs
that may harm and ultimately kill animals and
sentients that ingest them. All poisons will last (in
the bottle) for Rank weeks.
Mild Poison
At Rank 2 a herbalist may prepare, using uncommon herbs, a poison that will cause [D - 6] + (Rank
/ 4, round up) damage per minute for Rank × 2
minutes (damage is applied at the end of each
minute). It will take 24 hours to prepare Rank / 3
(round up) doses.
Strong Poison
At Rank 4 a herbalist may prepare, using uncommon herbs, a poison that will cause [D - 6] + (Rank
/ 3, round up) damage per pulse, for Rank × 2
pulses. Or using rare herbs a poison that will cause
[D - 3] + (Rank / 2, round up) damage per pulse,
for Rank × 2 pulses. If the herbalist uses uncommon herbs it will take 24 hours to prepare Rank / 3
(round up) doses. If the herbalist uses rare herbs it
will take 36 hours to prepare Rank / 4 (round up)
doses.
Multi-part Deadly Poison
At Rank 6 a herbalist may prepare, using rare
herbs, a poison in multiple parts (2 or more) that
will cause [D - 4] + Rank damage per pulse, for
Rank pulses. The time between application of first
and last part must be no more than 24 hours. Using
very rare herbs, the herbalist may prepare a poison
that will kill a human sized creature in 3 pulses.
Larger creatures have a percentage chance of dying
in proportion to their body size e.g. a creature four
times the size of a human has a 25% chance of
dying. If the creature does not die then it takes [D 4] + Rank damage per pulse, for Rank pulses. It
will take 72 hours to prepare Rank / 4 (round up)
doses of multi-part poison. It will take 48 hours to
prepare Rank / 4 (round up) doses of instant kill
poison.

38.8 Spirit Travel
A herbalist can prepare herbal mixtures that can
free the mind from the body.
At Rank 7 a herbalist may prepare, using very rare
herbs, a potion that will allow them to separate
their spirit from their body and travel in spirit
anywhere within 10 miles of the body for up to
Rank hours at a cost of 3 endurance points per
hour. After such spirit travel the herbalist may
regain lost endurance by healing or normal recovery except for 1 point per hour travelled which can
only be regained by expenditure of experience
(2500 EP per point lost). While in spirit form the
herbalist may, move at normal walking or running
speed, pass through any solid objects, see as they
would normally see (i.e. racial talents apply), hear
as they would normally hear, use any talent (including magical), cast any spell that affects only
the Adept, and fly at normal walking speed. The
spirit form cannot touch nor move any object by
non-magical means, nor can it speak. The spirit
form is insubstantial and cannot be touched or
harmed in any way (magically or physically). The
spirit form is unseen but may be detected by
Witchsight as per the spell of Walking Unseen. An
entity that loses their last point of endurance as a
result of this potion will become a spectre. This
potion will last for Rank × 2 days, and it will take
48 hours to prepare Rank / 4 (round up) doses.

39 LANGUAGES

39 Languages (Ver 2.1)
The campaign has many languages. Each sentient
race usually has one language intrinsic to itself, or
more if that race is split into various populations.
There is no universal language, but Common is the
first language of several nations.

39.1 Restrictions
A language may not be known above its maximum
rank. Characters may not speak a tongue for which
they do not have the vocal apparatus. Characters
may not learn a language without instruction from
a source of at least the same rank as that being
learnt.

39.2 Structure
Family Each language belongs to one particular
Family of intrinsically related tongues (see §39.6).
Group History, geography, and custom all transform languages. Languages with a common history
or interaction share the same language Group (see
§39.7). A language may belong to several Groups,
and a Group may link languages from different
Families.
Learning a language is easier if one already knows
a related or similar language at a higher rank. The
EP discount is:

Not all languages have a written form. It is not
possible to attain literacy in a language that does
not have an established written form. One may
attempt to transcribe that language, adapting a
known script, but the “writing” produced is ineffectual for communicating with others.
Phonetic Reading & Writing
Most Alusian languages are written using a phonetic alphabet — a set of signs representing, oneto-one, all the sounds of that language. Historically, a recently literate language usually re-uses an
established alphabet with minor variations. Therefore there are many languages, but few alphabets.
For each alphabet, the cost is 1000 EP and 4 weeks
the first time you learn literacy using it; literacy in
a subsequent language, using the same alphabet, is
only 500 EP and 2 weeks. Sometimes, for different
cultures, one language is written in different phonetic alphabets. If so, you must pay the time and
EP for each one you learn.
Table of known Alusian alphabets
Bedouin script (human, flowing, cursive).
b
d

Drakonic.

e

Elvish script.

• 20% if in the same Family or Group,

i

Island (used near the land-locked ocean).

• 30% if in both the same Group and the same
Family.

k

Kingdom (used near the Azurian Empire).

m

Mer (suited to underwater use)

n

Nagan (elaborate, but versatile).

o

Ogham (human, rune-like).

r

Dwarvish runes.

w

Westron (usual Western human alphabet, also
adopted by many newly literate societies).

39.3 Benefits
Languages vary in their complexity; a low maximum rank may indicates less versatility, vocabulary, or foreignisms.
At Rank 0 in a language, you cannot speak it, but
can usually sense the general mood of plain statements: a threat, a greeting, etc. Thereafter, with
increasing rank, your competency and vocabulary
progressively increase, as compared to humans
using a typical human language to talk about everyday things in their village.
Rank

Effect (& approximate Vocabulary).

1

Some of the simple, common words (2%).

2

A few simple statements (5%).

3

Common phrases, including basic directions; several tenses; effectively rank 0 in
all other languages of that Group (20%).

4

Common idioms; more tenses; can give
passable descriptions of events or people;
effectively rank 0 in all other languages of
the same Family (70%).

5

Rarer idioms; most tenses; sufficient to
use most professional skills (90%).

6

Normal, every-day fluency & usage; can
give clear & accurate descriptions of
events or people; effectively rank 1 in all
other languages of that Group (100%).

7

Courtly or professional speaker (120%+).

8

Can express any conceivable thought;
may cast college magic; effectively rank 1
in all others of the same Family (200%+).

9

Effectively rank 2 in all other languages
of that Group (400%+).

10

Maximum mastery of the language
(500%+).

Note that some languages are very limited. For
example, many concepts or emotions cannot be
articulated in Troll.

39.4 Literacy
Literacy in a language is distinct from the skill of
speaking. It is easily learnt if the written form is
alphabetic. Most cultures have a large proportion
of the population that is illiterate.

Orthographic languages
A literate language not using a truly phonetic alphabet is orthographic (e.g. it uses pictograms, or
an elaborate spelling structure). The written form is
so complex that it must be learned as if it were, in
effect, another language of the same language
family (e.g. written and spoken Erehleine are
treated as two separate members of the Eldar Family). Hence one often speaks and writes an orthographic language at different ranks. Orthographic
languages are indicated in §39.6 by an asterisk (*).
Each orthographic system is functionally unique to
its particular language.

39.5 Special Rules
Common It is easy to learn Common. Knowledge
of any other language at a higher rank gives a 50%
EP discount.
Accent Every speaker has an accent which reflects
a mixture of their native language and the tutors
from whom they learnt the language. At Rank 6 or
higher, any speaker may gain a particular accent by
spending 500 EP and 1–3 weeks studying or being
tutored (the GM decrees how much time is necessary).
Unpronounceable Tongues All languages of the
Dragon Family (except Saurime) require unusual
vocal apparatus. No humanoid race may normally
speak these tongues. However, you may rank the
language at twice normal cost, to gain comprehension. Alphabetic literacy in an unpronounceable
language costs 2000 EP and 8 weeks. If you do
have the physiological or magical ability to speak
such languages, you may rank them without penalty.
Immersion If character spends a number of weeks
listening to a particular language being spoken
daily and frequently by speakers who use it at a
rank higher than the character knows it, the GM
may allow that character to use those weeks as
ranking time for that language in addition to any
other activity undertaken (e.g. going on adventure,
117

other training, etc). The EP must still be paid. A
character may only rank one language by immersion at any one time.
New Languages When a new language is introduced into the campaign, the GM concerned must
determine the following:
1. Family and any language Groups.
2. Written forms, if any. Are they phonetic or
orthographic? If alphabetic, what alphabet is used?
3. Its maximum rank.

39 LANGUAGES

39.6 Language Families

39.7 Language Groups

The figure within [ ] represents the maximum rank
that can be achieved with the language; the letter(s)
represent the phonetic alphabet(s) used, and *
identifies orthographic languages. If no letter or
asterisk is given, the language does not have an
established written form.

Archaic Eldaran, Quenchan, Tenochan.

Common Common [9i,k,w].
Western-Human Alman[9o,w], Brett[9o,e], Destinian[8w], Ebolan[9w], Folksprach[9w], Lalange[10w], Raniterran[9e], Reichspiel [9w],
Saxony[9w].
Central-Human
Arabiq[9b],
Domani[9w],
Draknbrger[9w], Ellenic[10i], Kipchak[8], Kravonian[9*],
Panjari[9*],
Pasifikan[8],
Sanddweller[9e], Sea-of-Grass[9], Themiskryan
[9i,*].
Eastern-Human Five-Sisters [10*], Hindian [9b],
Lunar Empire[9*], Ruskan [9k].

Austronesian Jhavanese, Madyrese, Mylae.
Draconic Culhuan,
Draconic, Wyvern.

Draconic,

Dravidic
Drow,
Sanddweller.

Five-Sisters,

Nagan,

Old-

Raniterran,

Dwarvic Dwarvish, Gnomish, Halfling.
Dwarvidic Alman, Brett, Ebolan, Reichspiel,
Folksprach, Ruskan, Saxony.
Ellenic Centaur, Ellenic.
Elvic Drow, Eldaran, Elvish, Erehleine, Terranovan-Drow.
Elvidic Elvish, Lalange, Eloran.
Gnomic Fossegrim, Gnomish.
Herpetic Culhuan, Saurime.

Bestial Dawon [7], Dimasa [10b,n], Gnoll [7],
Karbi [9] Rabari [8b], Sasquatch[3], Sora [6],
Vanaran [9b].

Hiin Dawon, Dimasa, Doppleganger, Gnoll, Hindian, Karbi, Rabari, Sora, Vanaran.

Bhasa Mylae[10i], Jhavanese[9i], Madyrese[8i].

Low Gigantic Hill-Giant, Ogre, Stone-Giant.

Dragon Old-High-Draconic[10d], Culhuan[10*],
Draconic[10d], Nagan[10n], Saurime[7d], Wyvern[4].

Nomadic Domani, Draknbrger, Kipchak, Kravonian, Sea-of-Grass.

Eldar Drow[9e], Eldaran[10d], Eloran[9e,w],
Elvish[10e], Erehleine[10*], Quenchan[10*], Terranovan-Drow[9*], Tenochan[8*].
Faerie Brownie[7], Centaur[9i], Dryad[6],
Fossegrim[6], Leprechaun[6], Nixie[6], Nymph[7],
Pixie[7], Satyr[7], Sylphine[6].
False-Fey
Doppelganger[8],
Harpy[7], Medusa[6].

Gargoyle[6],

Littoral Destinian, Ebolan.

Orcal Goblin, Hobgoblin, Kobold, Ogre, Orcish.
Panic Centaur, Dryad, Nymph, Satyr, Sylphine.
Perfidic Fossegrim, Merfolk, Nixie, Pixie.
Protonic Eldaran, Old-Draconic, Draconic.
Rustic Brownie, Leprechaun.
Titanic Cloud-Giant, Lunar-Empire, Storm-Giant,
Titan.

Earth-Dweller
Gnomish[9r],
Goblin[8w],
Halfling[9r], Hobgoblin[8w], Kobold[8], Dwarvish[9r], Ogre[6w], Orcish[9w], Troll[4].
Giant Cloud[9w], Fire[9w], Frost[9w], Hill[8w],
Stone[8w], Storm[9w], Titan[10i].
Merfolk [8m].
Signing Silent-Tongue[6], Bandito [5].

118

40 MECHANICIAN

40 Mechanician (Ver 2.2)
Mechanicians are a blend of engineer and builder
who possess both design knowledge and crafting
ability so that they may plan and personally manufacture devices. Even without modern power
sources and techniques, mechanicians can still
build quite sophisticated devices using systems
such as springs, hydraulics and wind-based motor
systems to drive well greased moving parts.
Mechanicians may also be called on to devise locks
and traps to foil the efforts of thieves. They often
practice a particular trade and are called locksmiths, shipwrights, architects, etc. A skilled
mechanician may master several such professions.
Mechanicians often build overly large and complex
devices that are frequently non-functional and
occasionally dangerous. Their profession is usually
considered more of an art than a science.

40.1 Restrictions
A character must be literate in at least one language at Rank 6 or above to acquire the mechanician skill.
MD affects a mechanician’s Experience costs. A
mechanician pays 10% extra EP if their MD is less
than 15 and pays 10% less if their MD is more than
22.
A mechanician must pay money for the upkeep of
a studio or workshop, tools, work-in-progress, and
possibly guild fees.
The more complex, dangerous and experimental a
mechanician’s project is, the more likely that it is
to be temperamental, expensive to upkeep and
prone to breakdowns.

40.2 Benefits
Drafting A mechanician may draft and use plans
accurately. A mechanician may draw freehand
sketches and may draft, read and use plans and
diagrams, provided that they relate to an ability
with which the mechanician is familiar and that the
mechanician is literate in the language used.
Supervision Many projects will require the assistance of artisans and labourers, as well as other
mechanicians. Mechanicians gain the ability to
supervise subordinates who are practising either
the mechanician skill or an artisan skill necessary
to the mechanician’s project.
Artisan discount Many of the mechanician abilities give the character a grounding in an artisan
skill. A character may rank artisan skills that are
listed under abilities they have learned, at half of
the normal experience cost and time (round up), up
to the same Rank as mechanician. Artisan skills are
shown in the mechanician ability listings as [craft].
Combinations A mechanician may combine
known abilities. A mechanician may combine any
or all of their areas of expertise in the design and
execution of a project. A mechanician may also
combine their skills with other crafters to produce
items. The GM must decide, based on the abilities
possessed by the mechanician and other assistants,
whether or not they may design and build a certain
project.
Example
A mechanician who knew bridge building,
stoneworking, earthworks and hydro engineering, could
design and build an aqueduct that spanned a gorge. A
mechanician who knew chronometrical engineering, fine
materials and spell containment, might design and build a
“magical trap” that has a time delay in the trigger mechanism. A mechanician, an armourer and a weaponsmith
could combine abilities to build a suit of plate armour with
retractable blades at various locations. A mechanician
wishing to build a waterwheel-powered mill would need
architecture, complex mechanics, stoneworking, and
woodworking. A mechanician who knew boat building
and animal and textile products could design and build a
sail-powered coracle made of leather, but not a wooden
dinghy.

40.3 Abilities
All mechanicians have certain rudimentary abilities. At Rank 0 a mechanician gains an in-depth

knowledge of basic mechanics (including levers,
wedges, simple gears and pulleys, balances and use
of ropes) and basic foundations (including simple
earthworks, digging and shoring pits, piled stone
walls and brick making and laying). [Brick maker
/layer].

Ships new designs for boats and ships. [Shipwright].

After Rank 0 a mechanician acquires one new
ability per Rank. Additional abilities may be
gained without increasing in rank by the expenditure of 2,500 Experience Points and 4 weeks of
training per ability. These costs are discounted by
25% if the mechanician has reached rank 8, or by
50% if they have reached rank 10.

Stoneworking quarrying, cutting, finishing and
fitting. [Mason].

Some mechanician abilities give abstract comprehension of the theory, design and construction
techniques involved in crafting different projects.
Others offer an understanding of materials along
with a basic practical knowledge and ability in
crafting those substances. Each ability lists the
particular crafts or substances with which knowledge is gained. Special abilities are fully explained
in later sections of this skill.
Mechanician knowledge is of a more practical and
less esoteric nature than that gained through
equivalent philosopher fields and may be complemented by the acquisition of philosophic knowledge.
The abilities available are:
Animal and textile products includes material
such as horn, furs and leather, natural fibres and
other non-wooden plant products, heavy cloth and
ropes. Does not include venoms and alchemical
extracts. [Rope / Netmaker], [Sail / Tentmaker],
[Leatherworker], [Tanner / Hideworker / Furrier].
Architecture unfortified buildings of any size.
Bridges includes suspension, span, swing, humpback and floating bridges.
Carriages wagons, carriages and coaches. [Cartwright / Wheelwright].
Chronometers clocks, time-pieces and other timing devices. Complex locks special ability. (See
below). [Locksmith]. Complex mechanics includes
stresses, valves, pumps, power transmission (complex gears, compound pulleys, pistons, hydraulics,
etc.) and power generation (springs, wind, water,
etc.).
Earthworks complex earthworks, foundations and
landscaping. Civil engineering (including road,
ramp and town square building, as well as town
planning). Earthworks will be required to build
most large structures. [Lumberjack].
Fine materials fine and delicate materials, wirepulling and small component manufacture.
[Gold/Silversmith].
Fortifications defensive military works. Includes a
basic knowledge of siege warfare.
Glassworking glass mixing, blowing, window
construction and staining. [Glass-blower].
Hydro-mechanics devices (pumps, pistons,
valves, waterscrews, etc.), canals, sealocks, drainage, irrigation, sewage systems and plumbing.
Metalworking the forging and casting of base
metals. [Blacksmith], [Caster / Pewterer / Tinsmith].
Mines mine design & construction, pneumatic
devices (air pumps, fans, ventilators, etc.), knowledge of air shafts, ventilation and basic geology.
[Miner].
Optics optical devices (telescopes, magnifying
glasses, spectacles, mirrors, etc.), knowledge of
light, optics, and lens making, grinding and finishing.
Prosthetics articulated artificial limbs.
Traps special ability. (See §40.5).
119

Siege engines offensive military machines. Includes a basic knowledge of siege warfare.
Spell containment special ability. (See §40.6).

Woodworking carpentry, joints and wood-joining.
Also making basic wooden constructions. [Carpenter / Cabinetmaker].
Experimental engineering this ability may be
learnt any number of times with different experimental areas. It may first be learnt when acquiring
Rank 8. Experimental engineering areas may include: aeronautics, steam, geo-thermal, gases,
explosions, perpetual motion, vacuum, sub-marine
and advanced versions of any other mechanician
ability they already possess.

40.4 Complex Locks
Rank A Complex Lock is considered to have a
Rank, which is the Effective Rank that the
mechanician used in the construction of the lock.
The Rank of a lock may be less than or equal to the
Rank of the mechanician constructing it.
Time & cost The time to construct a Complex
Lock is (11 + Lock Rank - mechanician Rank)
hours.
The cost is (25 × Lock Rank [minimum 10]) sp.
A mechanician may always open one of their own
Locks in (12 - mechanician Rank) minutes.

40.5 Trap Construction
Rank A trap is considered to have a Rank, which
is the Effective Rank that the mechanician used in
the construction of the trap. The Rank of a trap
may be less than or equal to the Rank of the
mechanician constructing it.
Time & cost The time and cost to create a trap will
vary greatly, depending on the complexity, size
and nature of the trap.
The most commonly encountered type of mechanician trap is the precision trap. This is the type of
small needle or blade trap that may be set into or
adjacent to locks or other precision devices.
A lock or similar device may have up to Rank / 3
(round up) traps on or adjacent to it.
The time to build each trap is (11 + Trap Rank mechanician Rank) hours.
The cost is (125 × Trap Rank) sp, minimum of 50,
plus the cost of poisons, alchemical materials.
A triggered trap may be reset by any Mechanician
whose Rank is at least half that of the trap. This
will take (11 - mechanician Rank) hours. A trap
may need refuelling.
A mechanician may disable or re-enable one of
their own traps in (12 - mechanician Rank) minutes.
Triggering The precise actions that will trigger a
trap must be specified at the time that the trap is
constructed. Traps on a lock or other precision
device are automatically triggered if the device is
operated in the pre-specified manner and the traps
have not been removed or disabled.
Damage A trap may be built that causes physical
damage or explosively discharges its contents in a
cone up to (Trap Rank + 1) feet wide and (10 +
Trap Rank) long, or activates the mechanical trigger of a “spell container”.
A precision trap that causes physical damage may
cause up to [D10 + Trap Rank] Damage Points. It
may also be poisoned, coated with acid, etc., so as
to cause additional damage.

40 MECHANICIAN

40.6 Spell Containment
A mechanician may use this ability to create a spell
container, or magical trap, with a mechanical trigger.
Construction A mechanician with the “spell containment” and “fine materials” abilities can build a
mechanical device into which a spell can be stored.
The device is usually referred to as a magical trap,
or a spell container. The device is made out of
silver, truesilver or starsilver. Often the device is
built inside or incorporating other materials with
which the mechanician is familiar.
Spell Storage A single charge of a suitable spell
may be stored in the trap or container by an Adept
successfully casting a spell into the device after

performing Ritual Spell Preparation. A double or
triple effect stores an enhanced spell as specified
by the Adept. A failure has no effect. A backfire
affects the Adept as normal, and also results in the
device being damaged so that 20% more time and
materials are required before another try at storing
a spell may be attempted. The spell to be stored
must include a Storage type of “Magical Trap”.
Triggering The precise actions that will trigger the
device must be specified at the time that the device
is constructed. When these actions are performed
the spell is released. The spell stored must either
affect only the entity or object that triggered the
release of the spell, or affect an area in relation to
the device. All variable spell effects, such as direction and volume affected, must be defined at the

120

time of storage. Once the spell has been triggered
the device is useless, although metal equal to 10%
of the cost may be recovered. If the spell is dissipated, then 20% of the cost may be recovered.
Time to construct 25 - (2 × mechanician Rank)
hours.
Time to store spell (Spell Rank - mechanician
Rank) hours (minimum 1, maximum 10).
Cost (Spell Rank (minimum 1) × Spell EM) + 100
sp.
Minimum Weight (16 - mechanician Rank)
ounces.

41MERCHANT

41 Merchant (Ver 1.1)
Since adventurers are highly talented individuals
who often risk their lives, and a person is usually
compensated for the value of the work they do, the
player characters will fare better than most economically. A merchant character, blessed with the
ability to earn even more Silver Pennies, has the
best of all worlds. Their business acumen enables
them to command a stiff price for those goods they
vend, and to acquire that which they covet at bargain rates. The merchant is not often fooled in
monetary matters, for they can be an expert in
evaluating the worth of rare and costly goods.
The economies of most DragonQuest worlds do
not promote the growth of capitalism. Basically,
the nobility has a vested interest in all rural lands,
which comprise the vast majority of human-settled
areas. An ambitious, dynamic merchant could
perhaps own the entirety of a large town, but it is
quite likely that a jealous duke or prince would
twist justice to break the merchant’s power. Therefore, it behooves a merchant to cultivate powerful
allies when their holdings burgeon.

41.1 Restrictions
A merchant must be able to read and write in at
least three languages at Rank 6 in order to use their
assaying ability.

41.2 Benefits
The merchant’s ability to buy and sell a particular item is dependent upon its type.
Any item will be classified as one of three types:
common, uncommon, and rare or costly. Items
listed in the Players’ Handbook are of the common
type. Jewellery set with semiprecious stones, spices
from another continent, and fine paintings are
examples of the uncommon type. Rare and costly
items include magic-invested objects, diamonds,
roc’s eggs, giant slaves, etc. The GM must classify
each item with which a merchant wishes to deal.

A merchant can purchase items at a cost
cheaper than the asking price.
Item Type
Discount to Merchant
Common
[5 × Rank] %
Uncommon
[2 × Rank] %
Costly or Rare [1 × Rank] %
If the GM is actively playing the role of the seller,
or another player is the seller, the merchant must
do their own haggling. There will also be those
items which the vendor cannot afford to sell at the
usual discount to the merchant. The GM should use
their discretion here.
A merchant may mark up the price of an uncommon or rare item.
A merchant can gain (1.5 × Rank)% above the
value of an uncommon item they are selling. They
can gain (0.5 × Rank)% above the value of a costly
or rare item they are selling.
A merchant can assay an item to determine its
exact worth.
The player characters will generally receive a fair
quote on the price of basic goods, but must accept
the word of the being with whom they are dealing
when conducting a transaction involving uncommon, rare or costly items. The odds of the player
characters being bilked increase as they venture
forth from their native land(s). However, if a merchant is amongst them, they can assay the value of
any item after (11 - Rank) minutes.
The success percentage for assaying a common
item is equal to the merchant’s (Perception + 12 ×
Rank)%, to assay an uncommon item equal to
(Perception + 9 × Rank)%, and to assay a rare or
costly item equal to (Perception + 6 × Rank)%. If
the GM’s roll is equal to or less than the success
percentage the merchant is told the exact value of
the item in question. If the roll is greater than the
success percentage, the GM’s quote increasingly
diverges from reality as the result approaches 100.

121

If the result is odd, the quote is below the actual
asking price; if even, it is above.
A merchant may use their skill to affect transactions involving up to (250 + 50 × Rank Squared)
Silver Pennies per month, or a single transaction of
any amount.
The merchant must buy and sell at the asking price
for any transactions over their monthly limit.
A merchant can specialise in a specific category
of item assaying for every three full ranks.
The merchant chooses their speciality from the
following list (and any the GM should add):
1.
2.
3.
4.
5.
6.
7.
8.
9.
10.

Ancient Writings
Antiques
Archaeological Finds
Art
Books
Gems
Jewellery
Land
Magic Items
Monster and Animal Products (e.g. furs,
eggs)
11. Precious Metals
12. Slaves
When a merchant assays an item of a category in
which they specialise, they add (2 × Rank)% to
their success percentages. It is possible for a merchant to attain a 100% chance of accurately pricing
a speciality item (exception to 90% + Rank limit).
If a merchant wishes to add additional specialities
without increasing in rank, they must expend 4,000
Experience Points and 4 weeks of training per
speciality. These costs are discounted by 25% if
the merchant has reached rank 8, or by 50% if they
have reached rank 10.

42 MILITARY SCIENTIST

42 Military Scientist (Ver 2.1)
A military scientist can capably lead an increasing
number of troops as they improve their skill. They
can prevent their troops from fleeing after they
have gained their confidence. The main ability of a
military scientist is to anticipate and react to enemy
manoeuvres quickly because of their knowledge of
tactics.

42.1 Restrictions
A military scientist must be able to read and write
in at least one language at Rank 6 or above if they
wish to advance beyond Rank 2.

42.2 Specialised Fields
This skill has a number of specialised fields. One is
gained at each of Rank 0, 3, 6, 8, and 10.
Additional fields may be learnt without increasing
in rank by an expenditure of 3,000 EP and 4 weeks
of training per field. These costs are discounted by
25% if the military scientist has reached rank 8, or
by 50% if they have reached rank 10.
The fields are:
Aerial planning for or against magically or naturally flying troops.
Battlefield formulating and implementing battlefield level tactics, involving from hundreds to tens
of thousands of troops.
Logistics the ability to organise and control a
military organisation.
Naval tactics involving from one ship up to fleet
actions. Siege conducting or defending against
siege actions.
Skirmish tactics involving from one to fifty troops,
includes guerrilla and resistance tactics, and operating behind enemy lines.
Strategy overall campaign level command of a
military force.

42.3 Benefits
Command
A military scientist may control a much larger
number of subordinates than is possible with most
skills. Also, a military scientist’s subordinates need
not be practising this skill, nor need all be using the
same skill. A military scientist could thus command a mechanician, who was in turn in charge of
building siege engines, and a healer who was supervising other healers and teams of stretcherbearers. Any subordinate may be replaced by a unit
of up to 10 labourers or soldiers. A military scien-

tist may have up to (WP / 2 [+ 1 / Rank]) subordinates. A military scientist with the Battlefield
specialisation may have up to (WP + 2 / Rank)
subordinates.
Personal guard
After drilling for (12 - Rank) months, or being in
combat situations for a like number of weeks, a
military scientist may form a personal guard of
(WP + 5 + [2 × Rank]) troops. These troops will be
steadfastly loyal to the military scientist. The military scientist gains a (2 × Rank)% bonus when
attempting to command, rally, etc. their personal
guard. A personal guard will automatically follow
all rational commands from the military scientist in
all but the most stressful situations. In addition, a
personal guard may be commanded as a single
unit, replacing only one subordinate, even if there
are more than 10 individuals in the guard.
Rally troops
A Military Scientist may attempt to rally fleeing
troops that have been fleeing for less than 30 (+ 5 /
Rank) seconds. The military scientist must declare
how many troops are being rallied during one
pulse. If the military scientist is on the Tactical
Display, a rally attempt requires a Pass Action. The
Base Chance of rallying is (2 × WP) + (10 / Rank)
- number of troops to be rallied. If the roll is within
the Base Chance the troops rally, and will begin to
follow orders again; if the roll is greater than the
Base Chance, the troops continue to flee. The
chance of a being rallying is decreased by 25% for
each time after the first that it has broken during
the battle. Because of this it is possible for a successful rally attempt to affect only some of the
fleeing troops.
Raise morale
A military scientist may temporarily increase each
of their direct subordinates’ WP values by (Rank /
2) round down, provided that the military scientist
takes a Pass action every second pulse. To use this
ability, the military scientist may not be engaged,
stunned, or otherwise incapacitated.
Perceive tactics
A military scientist may be able to perceive the
tactics being employed by the enemy as they are
put into use, but before they come to fruition. To
use this ability, the military scientist must be unengaged, in a position to see the majority of the combat, and the combat must be of a type with which
they are specialized. In addition, if the Combat is

122

on the Tactical Display, the military scientist must
take a Pass action to implement this ability. The
Base Chance of Perceiving Tactics is PC (+ 7 /
Rank). The GM rolls D100; if the roll is within the
Base Chance, the GM informs the player of the
enemy’s plan in general terms. If the roll is greater
than the Base Chance but less than twice the Base
Chance, the military scientist is unsure of the enemy plan. If the roll is greater than twice the Base
Chance, the GM should mislead the player as to the
enemy’s plan, with the information becoming
completely false as the roll approaches 100.
Initiative
If a group involved in combat on the Tactical Display are led in combat by a military scientist with
the Skirmish field, the Military Scientist may add
(2 × Rank), minimum 1, to the group’s initiative
die roll, provided that they are not stunned or otherwise incapacitated, or engaged in melee or close
combat.
Time out
If a group involved in combat on the Tactical Display have a military scientist with the Skirmish
field leading them, they may have more time to
plan their actions between rounds of combat. The
Military Scientist may request a break period of up
to 20 seconds (+ 10 / Rank) between each and
every pulse, in which to plan their actions and
those of their companions. This time simulates the
orders and pre-arranged battle plans of the military
scientist. The players may speak with the military
scientist, and with each other, but should limit their
conversation to the matters at hand. Only the military scientist leading the group in combat may use
this ability.
Logistics
If the military scientist learns the Logistics field,
they gain knowledge of logistics management,
billeting and supplying troops, organising foraging
parties, posting watches, running patrols, and the
general day-to-day smooth running of a complex
organisation. The number of people that may be
effectively controlled by one organiser is 100 ×
([WP / 2] + Rank). This need not be an army, but
could also be an exploratory expedition, merchant
caravan, etc. If the military scientist has the Naval
field they may also control the logistics for (Rank
+ 1) ships.

43 NAVIGATOR

43 Navigator (Ver 1.1)
The art of piloting a sea-going vessel and that of
ascertaining one’s location are inextricably linked.
Humanoids must venture across the waters in
awkward ships, and are unable to survive immersion in the sea except for relatively short periods of
time. Yet there are many beings who dwell beneath
the surface of the ocean, and it is profitable for
land-bound peoples to engage in commerce with
them. Adventurers, with the assistance of an Adept,
will probably choose to try to despoil some of the
treasures of the deep.
A navigator can manage ships of increasing size as
they become more experienced. There is a limit to
the size of ships constructed, because of their relative fragility (sea creatures are wont to destroy
those vessels they consider overly large). The
navigator’s other chief ability allows them to locate
directions with instruments and read maps.

43.1 Benefits
A navigator can determine all compass directions if they can view the stars.
If the night is cloudy, or during the day, the navigator’s chance of correctly locating the compass
direction is equal to (25 + 7 × Rank)%. If the roll is
less than or equal to the success percentage, the
navigator has an exact reading on the compass
directions. If the roll is greater than the success
percentage, the reading is off by one degree for
each percentage point by which exceeds the success percentage (the GM must decide in which
direction the error is made).
A navigator may always determine the compass
direction of a landmark relative to their position.
A landmark is defined as any object which can be
seen or to which a being can precisely point. A
navigator may also judge the distance between
their position and a visible landmark. Their chance
to precisely gauge the distance is equal to (PC + 10
× Rank)%. When the roll exceeds the success

chance, the estimate is off by the percentage difference between the roll and the chance to accurately
judge, randomly long or short.
A navigator can read a map if they can relate
their physical surroundings to the symbols on
that map.
This skill allows a navigator to read a map, chart or
rutter if they can relate their physical surroundings
to the symbols on that document. Even the best
quality maps are not particularly accurate or standardised. Interpreting each new map is a challenge
of the navigator’s wits and experience. If a character does not have a map-reading skill, they may not
read maps.
If a navigator tries to read a map which is of the
area in which they are presently located or is of an
area with which they are quite familiar, they
clearly understand at least (2 × PC + 8 × Rank)%
of the map. Further, they are baffled by up to (2 ×
PC + 2 × Rank)% of the map. They may misinterpret the remainder of the map. If a navigator tries
to read a map of an area with which they are not
familiar, they clearly understand only (PC + 4 ×
Rank)% of what they would have had they known
the area. If the map is inaccurate, it is unlikely that
the character will detect the flaw unless it was
relatively major.
The navigator may place themselves on a map if
they can determine the direction of two marked
landmarks.
Map Creating
The navigator may draw a map or chart or which
shows the major landmarks and features of the area
in which they are presently located or of an area
with which they are quite familiar, or write a rutter
describing a route that they are travelling or are
familiar with. At least (2 × PC + 8 × Rank)% of the
map will be accurate, a further (2 × PC + 2 ×
Rank)% will be confusing and unclear, and the rest
will be inaccurate and misleading.

123

A navigator can competently pilot a ship of up
to (25 + 25 × Rank) feet in length.
A competent pilot of a ship has a negligible chance
of damaging or sinking a ship when faced with
normal weather and sea conditions. When a ship is
not steered by a competent pilot, it is in very real
danger of experiencing an accident in choppy seas
or during a storm.
A navigator can consistently maintain a ship’s
speed at (50 + 5 × Rank)% of its optimum
speed.
If the ship is undercrewed, the optimum speed is
calculated for the ship with its current crew complement.
A navigator can predict weather at sea with (PC
+ 5 × Rank)% chance of accuracy.
The GM rolls percentile dice; if the roll is equal to
or less than the success percentage, a navigator can
correctly predict the weather for the following (4 +
2 × Rank) hours. If the roll is greater than the success percentage, the navigator’s version of the
upcoming weather becomes more and more inaccurate as the roll approaches 100.
A navigator can sometimes recognise nonmagical danger at sea before subjecting the ship
to it.
A navigator’s success percentage to use their perceive danger ability is (3 × Perception + 7 ×
Rank)%. If the GM’s roll is equal to or less than
half the success percentage (rounded down), the
GM informs the navigator character of the precise
danger the ship is facing. If the roll is between onehalf and the full success percentage, the navigator
intuitively senses the direction and distance of the
danger. If the roll is greater than the success percentage, the navigator is unaware of impending
doom.

44 PHILOSOPHER

44 Philosopher (Ver 2.0)
Philosophers become familiar with the general
characteristics of their world, within the limits of
the knowledge available to their culture, discarding
many popular misconceptions. They acquire extensive knowledge on a wide range of subjects, and
are, in many ways, the encyclopaedias and expert
opinions of the medieval world. Philosophers are
also well versed in using the unusual and obscure
indexing methods employed in medieval libraries,
and so may research and answer enquiries that they
do not immediately know the answers to.

44.1 Library
Any place with 50 or more books may be considered a library, for the purposes of study. Libraries
are rated for the number of days that a philosopher
may study in them to answer any particular question. This rating is usually equal to Books divided
by 50. Once a philosopher has exhausted the possibilities of a library they must either find another
and continue their study or attempt to answer the
question anyway. A day of study is 10 hours, and is
full-time work. Some libraries with specific collections may be rated higher for some Realms than
others. GMs should bear in mind that the books in
some libraries will be predominately in particular
languages, and that if the philosopher is not literate
in those languages, the library may be of reduced
usefulness.

44.2 Requirements
Language restriction
A character may not become a philosopher unless
they possess at least one language at Rank 8, and
are literate in that language.
Books
A philosopher must possess (or have frequent
access to) at least Rank times 10 books, written in
languages that they are literate in.

44.3 Structure
The philosopher skill is designed as a tree-like
structure, with several separate Realms of knowledge, each of which has its own Fields, which in
turn, have Sub-fields.
Realms
These are the largest and least detailed divisions of
knowledge. There are 5 Realms of knowledge: the
Social World, the Material World, the Magical
World, the Animal World, and the Plant World.
Fields
Realms are divided into large blocks of knowledge,
called Fields. GMs should not need to add new
Fields to the Realms, but may do so if they wish.
Sub-fields
Small, and quite specific divisions of a Field, these
are not limited to only those suggested below. A
philosopher may learn almost any sub-division of a
Field as a Sub-field, with the GM as the final arbiter. The most common Sub-fields concern a particular race or area within a Field.

44.4 Language Benefits
Philosophers gain a reduction in the EP costs to
learn languages, in addition to any other reductions
available to the character.

(see the individual Realms for more details). If a
philosopher wishes to forego learning a Realm,
they receive an extra 8 Sub-fields (which may be
traded for Fields as below).
Fields & Sub-fields
At each Rank above 0, the philosopher receives a
number of Sub-fields. They receive: at Ranks 1 to
4, 3 Sub-fields; at Ranks 5 to 7, 5 Sub-fields; and
at Ranks 8 to 10, 7 Sub-fields.
3 Sub-fields may be traded for 1 Field. Any part of
the Subfield allotment may be retained and used in
conjunction with the allotment received for further
Ranks. Once a philosopher has achieved Rank 10
they may not Rank their Skill further, but may
acquire new areas of knowledge. A new Realm
costs 8 weeks and 4000ep, a new Field 3 weeks
and 1500ep, and a new Sub-field 1 week and
500ep.
Field Restrictions
A philosopher may not learn a Sub-field if they
have not already learned the Field that it is part of.
They may not learn a Field if they have not already
learned the Realm that it is part of.

Final Result
If the question is of a yes/no nature, the Accuracy
is the Base Chance that the philosopher will arrive
at the correct answer. If the question is more open,
the Accuracy is the amount of relevant information
that the philosopher will come up with. It is also
possible that some questions (as determined by the
GM) are simply unanswerable. If this is the case,
the Accuracy becomes the Base Chance that the
philosopher will become aware of this fact.

44.7 Realms & Fields
Each of the five Realms is listed below, along with
its associated Fields. Some Fields are followed by
a list of suggested Sub-fields.
The Social World
Standard Sub-fields include: Area, Race, History.
The Fields of this Realm are:
• Art & Music — Style
• Ethnology

Overlaps & Connections
In some cases it is possible to reach the same Subfields by different routes. These duplicated Fields
may be treated as identical and no benefit accrues
from having the same Subfield more than once.

• Heraldry & Genealogy — Tinctures, Furs

44.6 Research Benefits

• Philosophy & Ethics

Philosophers may attempt to answer questions put
to them. These questions may be posed by themselves, or by other characters. If the philosopher
does not already know the answer, their chance of
success depends on the difficulty of the question
and the relevant Realms and Fields of the philosopher.

• Politics & Customs

Difficulty
Questions that may be answered by a philosopher
fall into one of seven categories: Automatic, Very
Easy, Easy, Standard, Hard, Very Hard, and Impossible. The first step in determining the difficulty
of answering the question is for the GM to determine which Realm(s) the question pertains to, and
the level of difficulty of the question.
A Standard question is one of average difficulty,
relative to a given Realm, as determined by the
GM. They usually deal with a reasonably large
sub-set of the knowledge of the Realm. If the philosopher possesses the Realm to which the question pertains, but has no more in-depth knowledge
applicable to the question, the difficulty is as set by
the GM. If the philosopher has a Field within that
Realm that the GM determines is relevant to the
question, the difficulty decreases by one step. If the
philosopher has a Sub-field within that Field, and
the GM determines that it is relevant to the question, the difficulty decreases by another step. If a
philosopher does not even possess the Realm of the
question, it becomes two steps harder. A philosopher will immediately know the answer to an
Automatic question. A philosopher may not answer
an Impossible question.

44.5 Knowledge Benefits

Answers
The accuracy of the answer that a philosopher can
offer is dependent on Rank and the difficulty of the
question. To increase their accuracy, a philosopher
may also undertake a course of study. For each
study period (the length of which is determined by
difficulty), +1% is added to the philosopher’s Base
Chance. A philosopher may, at any time, attempt to
answer the question. The base Accuracy, Rank
bonus, and length of study period is shown on the
Answer Table (§44.8).

Realms
At Ranks 0, 4, 7, and 10, the philosopher may learn
a Realm of knowledge. Each Realm provides a
thorough grounding in the basics associated with it

Even though philosophers keep notes during their
course of study, an extended interruption may
prove a setback. If a philosopher ceases a course of
study but resumes it within Rank weeks there are

Provided that the Rank of language being learnt is
not greater than their Rank of philosopher, philosophers may learn to speak the language or to
read an orthographic language at a 10% discount.
If the philosopher has chosen the Field of Linguistics, the discount is 20% instead, increasing to 30%
if the philosopher has chosen the appropriate Language Group subfield (see §44.3).

no adverse effects. If the interruption is longer than
this, then half of the percentage amount that they
had achieved from study is lost.

124

• History — Ancient
• Legends & Folklore
• Linguistics — Language Group

• Theology & Mythology
The Material World
Standard Sub-fields include: Area, Race, History,
Advanced. The Fields of this Realm are:
• Alchemy — Experimental
• Architecture — Experimental, Ancient
• Astronomy
• Cartography
• Engineering — Experimental
• Geography
• Geology & Mineralogy — Group of Minerals
• Mathematics
• Metallurgy — Experimental
• Oceanography
The Magical World
Standard Sub-fields include: Area, History. The
Fields of this Realm are:
• Artefacts & Magical Items — Shaper, Legends
• (Any College) — Politics, Famous People
• Demi-Powers — Groups, Races
• Deities — Pantheon, Religion
• Dragons — Type, Genealogy, Behaviour
• Elements — Any element or amalgam
• Fantastical Beings — Any group
• History & Theory — College Divisions, Backfires
• Naming — Structure
• Magical Animals — Type
• Magical Plants — Type
• Mana Zones — Places of Power
• Other Planes — Plane
• The Powers — Pacts, Invocations, Agency, Factions
• Undead — Lesser, Greater

44 PHILOSOPHER
The Animal World
Standard Sub-fields include: Area, Type. The
Fields of this Realm are:
• Amphibians
• Aquatics
• Avians

44.8 Answer Table
Difficulty

Accuracy

Per rank

Period

Very Easy
Easy
Standard
Hard
Very Hard

90%
70%
40%
20%
0%

+1%
+2%
+3%
+3%
+3%

1 minute
5 minutes
15 minutes
30 minutes
1 hour

• Insects & Spiders
• Land Animals
• Magical Animals
The Plant World
Standard Sub-fields include: Area, Type. The
Fields of this Realm are:
• Aquatic Plants
• Flowers
• Grasses & Cereals
• Herbs
• Magical Plants
• Root Plants
• Shrubs & Bushes
• Trees

125

45 RANGER

45 Ranger (Ver 2.1)
Rangers are trained to survive, and perhaps thrive,
in wilderness. They can feed themselves, shelter
from the elements, choose the best way to travel
and identify natural dangers. Rangers’ general
training is useful in any outdoors environment but
they benefit further from learning the specifics of
particular environments.

45.1 Benefits
Primary Environment
A ranger knows far more about the environment
with which they are most familiar. While in this
primary environment the ranger’s base chances and
formulas should be calculated as if they were 2
Ranks higher. A ranger’s initial primary environment is that in which they learnt the skill. A ranger
may later choose to change their primary environment during ranking.
The ranger must train in the new environment, the
ranking time is increased by 1 week and the EP
cost for this rank is increased by 50% (to maximum of +3000 EP). To return to a previously
learnt primary environment the ranger must spend
500 EP and 2 weeks in the environment. After rank
10 a ranger may learn a new environment by training for 4 weeks in the new environment and spending 3000 EP.
Stealth Bonus
While using any of the abilities in this skill a
ranger gains a bonus to stealth of + 3 / Rank. No
other skill bonuses to stealth may be applied at the
same time as this bonus.
Finding Food
Foraging A ranger knows how to find water, edible plants, and animals suitable for the pot. Foraging includes finding plants, setting snares, hunting
small animals, fishing etc. Snares should be left
overnight (or even days) to be successful. A ranger
does not need to make an attack roll but may automatically kill small animals that were caught during foraging.
In an average area in one hour a ranger can find
enough food to feed one person for a day (+ 30
minutes per extra person). The volume of food
available is dependent on fertility and season so the
GM should adjust the time to suit the environment.
If a ranger wants to hunt larger animals they should
use the Tracking ability to locate game and then
use an appropriate hunting weapon to kill it (i.e.
ranged weapons, spear). If they make a successful
attack then their quarry is immediately killed. If
they miss the animal will flee. If a ranger is hunting predators, extremely large animals or sentients
they must use the combat rules to kill their quarry.
A ranger and mechanician may combine their
abilities to build and conceal large traps, pit falls,
etc at the GM’s discretion.
Identify and Find Plants and Animals A ranger
can recognise common plants and animals. They
have a (Perception + 10 / Rank)% chance of resolving whether a strange plant or animal is suitable for food. If they roll 10% or less than their
success chance they may also notice other properties of the item (e.g. poisonous, valuable etc).
A ranger can identify the types of entities living in
an area from the traces they leave behind (tracks,
game paths, grazing signs, prey remains etc). This
takes about 15 minutes and gives them an idea of
the variety of animals in the area (e.g. the primary
carnivore is a wolf pack; there is a large herd of red
deer and a flock of pigeons).
A ranger may search for a specific plant or animal
(including herbs required in the First Aid ability),
provided it is native to the region. The base chance
is 2 × PC + 5 / Rank (- 0 if common, - 25 if uncommon and - 75 if rare). This roll should be made
once per hour of searching.

Tracking A ranger can follow the tracks left by
entities moving on the ground. In calm weather,
tracks normally last around 10 days but the clarity
and duration of tracks will be enhanced by the
number of entities, or soft ground, and reduced by
hard ground, rough weather, or if the entity is
trying to hide their tracks. The base chance of
following tracks is Perception (+ 5 / Rank) (+2 /
entity in group) (4 / Rank of quarry’s ability to hide
tracks). If a ranger is following a fresh track they
will be aware when they are close enough to be
detected. They may then use stealth to sneak up on
their quarry and they will be able to get 25% closer
than a non-ranger before there is a possibility of
being detected.
Camping
Preparing Food A ranger knows how to get a fire
going, gut and skin animals, and cook simple meals
over an open fire.
Campsites A ranger knows where to set up camp
so that they are sheltered from the elements, close
to water, or other by criteria they may choose (e.g.
hidden or defensible).
A ranger can easily erect tents, they can add extra
comfort to a campsite by setting up tarps to protect
from wind or water, and they can take advantage of
nearby resources to build a crude shelter.
Travelling
Orientation A ranger has a sensitivity towards
north. They are able to pinpoint true north to within
(10 - Rank) degrees and from this they can work
out the other compass directions.
Map Reading A ranger can read a simple map if
they can relate their physical surroundings to the
symbols on that map. There are no standard symbols or keys so interpreting a new map is a challenge of the Ranger’s wits and experience. A
ranger may place themselves on a map if they can
determine the direction of two marked landmarks.
Route Finding A ranger is rarely lost and can
normally back track to a known point. They learn
to recognise landmarks from unfamiliar directions
and estimate the time and effort required to travel
through various terrain. A ranger can pick a route
through unknown terrain based on ease of travel,
speed, stealth, or safety etc. The base chance of the
ranger picking the best route for their purpose is 2
× Perception (+5 / rank)%. The roll should be
made by the GM and if the ranger fails then the
route travelled should be hard or longer or dangerous as appropriate. This roll should only be made
once per day.
After a ranger has travelled through an area several
times they do not need to use known routes but can
freely take shortcuts or choose better routes.
Distance Estimates A ranger can estimate distance
travelled overland to within (90 + Rank)% accuracy.
Safety
Detect Hidden In a natural setting a ranger may
notice hidden entities, or recognise an ambush or
trap before they walk into it. The base chance is 3
× Perception (+ 5 / Rank) (-5 / Rank of person who
did the hiding or set the ambush or trap).
Hide Tracks A ranger can obscure the tracks of 1
(+ 1 / Rank) entities moving in the same direction.
It takes 30 (1 / Rank) minutes to obscure 100 yards
of track. This time may be reduced if the ground is
rocky or naturally hard.
Hide Entities A ranger can attempt to hide 1 (+ 1 /
Rank) entities in natural cover. The ease of hiding
someone is dependent on the available terrain. The
GM should advise a modifier based on the terrain
of 1 (e.g. flat open ground) to 10 (e.g. thick bushes
or jungle). The base chance of hiding is (modifier ×
Rank) - 5. (NB this ability does not imply that the
ranger can set up ambushes).
126

First Aid A ranger knows simple first aid to prevent minor accidents in the wilderness becoming
severe. They know how to:
• Stop external bleeding
• Splint broken bones
• Treat minor burns
• Recognise the effects of common natural poisons
They also know how to brew tisanes (herbal teas)
which help reduce the effects of headaches, nausea,
fevers and food poisoning. To make tisanes the
ranger requires fresh common herbs (which have
been picked within 24 hours of use).
The First Aid abilities cannot be used in combat.

45.2 Environments
The environments a ranger may choose as their
primary environment are dominated by similarities
of climate, terrain and fertility. These environments
cover lightly populated areas eg. open farmland,
moors, but do not include towns, cities, etc. Some
environments overlap.
Arctic Includes tundra, steppes, permafrost and ice
caps and other infertile lowlands in cold climates.
Fertility: Infertile, Seasons: standard, note that
winter has no daylight & summer has no night
time.
Caverns Includes all caves, tunnels, natural caverns, and other substantial underground areas.
Fertility: Infertile, Seasons: always low season.
Coastal Includes land adjacent to saltwater, estuaries, coastal marshes etc. Fertility: Average or poor,
Seasons: standard.
Highlands Includes hills and mountains, moors,
high plateaus. Also includes evergreen forests on
steep ground. These areas are fertile in summer but
snow or ice covered and hostile in winter. Fertility:
Poor, Seasons: standard.
Jungle Includes hot climate forests of any sort.
They are particularly characterised by heavy undergrowth and high rainfall. Fertility: Rich, Seasons: wet/dry.
Plains Includes grasslands, plains, pampas, savannah, prairie, veldt, and other more or less open and
flat or rolling terrain. May include low hills where
the land is open and not wooded. Fertility: Poor,
Seasons: standard.
Rural Generally mild climate cultivated terrain,
lightly inhabited. Includes cultivated fields, grazing
lands, vineyards, heaths, etc. Fertility: Average,
Seasons: standard.
Waste Includes all deserts, wastelands, salt flats,
and other infertile lowlands in mild to hot climates.
Fertility: Infertile, Seasons: Reversed in hot regions as the most fertile period is autumn and the
least fertile summer.
Wetlands (freshwater) Includes marshes &
swamps, and land adjacent to freshwater rivers,
lakes & ponds, etc. Fertility: Rich, Seasons: standard.
Woods Includes mild climate deciduous and evergreen forests or large wooded areas with few sentient inhabitants, in mild to cold climates. Fertility:
Average, Seasons: standard.

45 RANGER

45.3 Ranger Summary Chart
Ability

Base Chance

Notes

Brew Tisanes

90 + 1 / Rank

Reduces effects of headaches,
nausea, fevers and stomach upsets

Choosing campsites

90 + 1 / Rank

Detect hidden & traps

3 × PC + 5 / Rank (-5 / Rank of opposing ability)

Distance Estimates

90 + 1 / Rank

Find specific
plant/animal
First Aid

2 × PC + 5 / Rank (- 25 if uncommon / -75 if
rare)
90 + 1 / Rank

Foraging

90 + 1 / Rank

Hide Entities

(Modifier × rank) - 5

Hide Tracks

90 + 1 / Rank

Identify Local
Inhabitants
Map Reading

90 + 1 / Rank

Orientation

90 + 1 / Rank

Preparing Food

90 + 1 / Rank

Recognise
plants/animals
Route Finding

PC + 10 / Rank
2 × PC + 5 / Rank

Rolled by GM

Tracking

PC + 5 / Rank (+ 2 / entity) (- 4 / Rank opposing
ranger)

Tracks last 7 - 10 days

Time

incl. hidden entities, ambushes
Roll per hour
Stop external bleeding, treat
burns, splint bones, recognise poison
Modified by season and fertility

60 mins + 30 per extra
person

Hides 1 + 1 / Rank entities. Modifier based on available
cover
Obscures tracks of 1 + 1 / Rank
30 - 1 / Rank minutes
entities
15 minutes

90 + 1 / Rank

127

Roll per day

46 SPY

46 Spy (Ver 2.0)
Amongst the many professionals encountered in
everyday life, there will be a scattering of those
with the covert skill of Spy. This is a profession
dealing with obtaining and distributing information. There are basic information gathering and
remembering abilities in common with all spies,
but the methods of operation, and spheres of influence vary from the court ambassador to the military scout to the pub minstrel. A spy may specialise
in a particular field of operation and if they have
the accompanying skills appropriate for their
cover, will perform better, with less chance of their
actions being discovered and their cover blown.

46.1 Restrictions
There are no restrictions on learning the spy ability, or to using it, except perhaps the fear of being
discovered. Note that having learnt the Spy skill is
not the same as actually spying.
Discovery
If the spy catastrophically fails in their skill by
rolling greater than 30% over the base chance for
the ability, or 90% + Rank (whichever is the lesser
BC), then they may have been discovered. The
repercussions of this discovery is based on their
situation: what exactly they were attempting, who
the discoverer is, the difficulty of the task compared to their rank in the skill, whether the spy has
other skills to back up their cover etc. The repercussions may range from a slap in the face, being
given disinformation, expulsion from an inn/town,
a beating in the back alley, through to the traditional punishment for an exposed traitor: to be
drawn and quartered, (although nobles are sometimes beheaded). Thus, a good spy is the one least
likely to be discovered.

46.2 Benefits
All spies gain grounding in basic spy craft, including memory enhancement, moving quietly, observing closely and communicating with their peers.
Enhanced Memory
• A spy may memorise and recall visual details,
such as those of a room, a person, or a piece of
parchment, etc. Memorisation requires (120 - 10 ×
Rank) seconds of undisturbed concentration, studying the object.
• A spy’s chance to recall a memorised image
accurately is (2 × Perception + 12 × Rank)% rolled
by the GM. If the failed recall attempt occurs
within 1 + (1 × Rank) days of the memorisation,
the spy merely cannot remember. After this period,
their effective rank for recall reduces by one per
subsequent day, and with a failed roll, erroneous
information may be remembered instead. If a spy
fails to recall an object or place, they may not
attempt to recall it again until they study it again.
• A spy may also use this ability to recall spoken
phrases and combination of sounds. Even if a spy
does not know the language used, they can reproduce the phrases phonetically.
• A spy’s enhanced memory gives them an effective understanding of any spoken language that
they have at Ranks 0-3 as if one rank higher.
Stealth
A spy increases their chance of acting stealthily by
2% per Rank.
Fieldcraft
A spy is trained to notice, recognise, and appropriately interact with other spies. This includes handing off messages discretely, the use of dead-drops,
safe-houses and identification phrases, as well as
foiling the less subtle attempts to interfere with

such exchanges. Also, they may identify the weak
willed, manipulable, morally suspect or gullible
individuals that a spy prefers to associate with, and
has a greater chance of enticing them to a course of
action by coaxing, flattery, wheedling, blackmail
or other enticement.
Optional Abilities
In addition, a spy gains an ability chosen from the
list below with each Rank (including Rank 0).
Additional abilities may be gained without increasing in rank by the expenditure of 2,500 Experience
Points and 4 weeks of training. These costs are
discounted by 25% if the Spy has reached rank 8,
or by 50% if they have reached rank 10. Individual
Base Chances are provided for some of the various
spy abilities; for the other skills there is a generic
Base Chance of 3 × appropriate characteristic (+ 5 /
Rank), modified by difficulty.
Assess: a spy can infer some information from
observing the grouping and activities of people.
The level of information gained is logistical, rather
than the in depth knowledge that knowing the
associated skill would give. Example uses include:
• Estimate Entourage ascertain the likely social
ranking of the target based on the size, quality,
snootiness etc of accompanying servants. Evaluation of who they are or the implications of the
household makeup would require Courtier.
• Estimate Goods estimate the number of boxes,
and the people, wagons, time and other logistics
required to move them. Evaluation of quality or
value would require Merchant.
• Troop Estimation estimate size, equipment and
quality of a military force or navy. Inferring the
tactics this represents would require Military Scientist.
Befriend: over a period of days or weeks, a spy
may target individuals to engender trust in themselves, and possibly distrust in others. The “friend”
begins to willingly and unknowingly reveal information to the spy.
Bribery: a spy may recognise appropriate people
and which “gifts” they prefer, to gain information
or access. They may also recognise situations when
bribing will not work, before making the attempt.
Codes: a spy can recognise and use simple codes.
They may attempt to break others’ codes and ciphers. The length of study required to do this depends on the familiarity of the code’s style and the
difficulty set by the code maker.
Counterspy: a spy may perceive other current
spying activity and recognise the other spies. Also,
a spy may create convincing artifices that fit with
available information, to spread disinformation,
misunderstanding or confusion.
Disguise: with the use of physical props and resources, the spy can apply makeup, dyes, false hair
etc to convincingly alter the appearance of people.
The success of disguising race and gender depends
on the physical similarity of the person, and how
closely the disguised entity is inspected. A spy may
only attempt to imitate a specific person’s appearance after prolonged study of the target. Animals
the spy is familiar with may also be disguised.
Forgery: With the right materials to hand, a spy
can create convincing replicas of personal letters,
official documents and the like. This includes the
ability to open and re-seal letters, produce false
seals, or move real ones to a forgery. The chance
for this is MD + PC + (4 × Rank). If a spy is literate with the language of the document, then the

128

language rank can be added to the base chance for
forgery.
Hiding: a spy can find unlikely hiding places and
conceal themselves for long periods of time, keeping still and quiet. The spy must roll WP + EN + (5
× Rank) to maintain this for extended periods of
time.
Imitation: the spy can study and copy behaviour,
mannerisms, and accents. This will allow the spy to
maintain the roles of ordinary people and not stand
out as a foreigner. As a guide, this takes 6 (Rank/2) hours of exposure to the society, with
modifiers based on the apparent familiarity or
strangeness of the society.
Information: by using other skills and knowledge
a spy has an increased chance of getting the most
relevant information they are seeking. They have
an increased chance of noticing disinformation
about their area of knowledge, and distinguishing
fabrication and pretence from fact and reality. The
spy may recognise information as having worth to
other individuals or spies in their network, even
though it is not useful to them.
Lip Reading: a spy may understand spoken conversations outside of their hearing range. A clear
line of sight to the targets’ face and knowledge of
the language being spoken (min rank 6) is required.
The spy may garner fragments but not all of the
conversation if they are not fluent with the language, can only see one of the participants, or an
unlikely topic is being discussed.
Pick Locks: while using appropriate tools, a spy
can spend (240 - (20 × Rank) ) seconds to attempt
to pick a lock to either unlock or lock it. The base
chance of the attempt is (MD + (4 × Rank) ) - (6 ×
Lock Rank). If the attempt fails, the lock resists the
attempt. A catastrophic failure may damage the
lock.
Resist Torture: being familiar with extraction
techniques, a spy may add their Rank to their effective WP for resisting torture attempts. They may
also choose to apparently lower their WP by up to
Rank to release false information.
Shadowing: a spy learns the skills of following
individuals at a distance without being observed. In
addition they have an increased chance of noticing
when they themselves are being followed, and may
attempt to lose their followers if the appropriate
terrain is available. To spot a tail the Spy must roll
under (2 × PC) + (5 × Rank) - 5 times opposing
Spy’s rank. To successfully tail another spy requires a roll of PC + AG + (5 × Rank) - 5 times
opposing Spy’s rank.
Sleight of Hand: a spy can palm, swap or place
small objects, without attracting notice. Removing
objects from people requires the Pick Pockets skill
from Thief.
A GM may give the following abilities to a Spy
over the course of play, or they may be requested
by a player as part of their character knowledge
and backgrounds.
Network (area): the spy has joined a spy ring in a
specific area, and knows the specific field craft
routines used by that ring.
Spy Master (area): the spy has set up a spy ring in
a specific area, and knows where to place spies,
how to store information for later reference, and
how to manage other spies. This ability requires
Rank 8 plus.

47 THIEF

47 Thief (Ver 1.2)
Thieves practice their trades covertly, in order to
avail themselves of the well-guarded wealth of the
powerful. The thief has a task to accomplish: the
(hopefully) undisturbed removal of property from a
supposedly secure place of storage. A thief usually
seeks monetary rewards for their efforts, and a
thief cultivates contacts in the underworld of their
area of operations. These contacts will enable them
to discover where the choicest items are stored, and
aid them in disposing of their ill-gotten gains.
If a thief character wishes to use their skill while
not accompanied by the rest of the party, the GM
should run a solo adventure (unless the task the
thief sets themselves is very easy). A thief who is
caught in the act of burglary is liable to the stiff
penalties of medieval times: a hand is removed for
the first (known) offence, a second time merits the
removal of the other hand or the eye opposite the
missing hand, with a greater degree of dismemberment for each succeeding offence.

47.1 Restrictions
A thief must be able to read and write in one language at Rank 3 if they want to advance beyond
Rank 3.
When a character is both a spy and a thief, the
player may use the better of the two percentages to
perform a given ability.

47.2 Benefits
A thief can pick locks or open safes with the aid
of tools.
The time a thief must spend to implement the pick
lock ability is (120 - 10 × Rank) seconds, and (15 Rank) minutes to use the open safe ability.
If the GM’s roll on percentile dice is equal to or
less than the success percentage the thief has
opened the safe or picked the lock. If the roll is
greater than the success percentage, the safe or
lock resists the thief’s best efforts. If any trap remains in place when a thief attempts to open a safe
or pick a lock, it is triggered by that action.
For Thief to Pick
Lock

(2 × MD + 6 × Rank) - (6 ×
Lock Rank)

For Thief to
Open Safe

(2 × MD + 5 × Rank) - (7 ×
Safe Rank)

A thief may attempt to detect traps and should
the thief succeed, may try to remove them.
A thief may make one attempt to detect traps
(which requires 10 seconds) in a particular location
per day. A thief must spend (12 - Rank) minutes to
use their remove trap ability.
The GM must make one percentile roll for each
trap to see if the thief detects it. If the roll is less
than or equal to the success percentage, the thief
notices the location of the trap. If the roll is above
the success percentage, they remain blissfully
unaware of the trap’s presence.

For Thief to
Detect Trap

(Perception + 11 × Rank)

For Thief to
Remove Trap

(2 × MD + 11 × Rank) - (5
× Trap Rank)

When a thief attempts to remove a trap, the GM
rolls percentile dice. If the roll is less than or equal
to the success percentage the thief has removed the
trap without triggering it. If the thief has a trap
container, they may store the removed trap. If the
GM’s roll is greater than the success percentage,
the trap is triggered (see §40.5).
A thief can sometimes detect a secret or hidden
aperture.
Any thief can try to find a secret or hidden aperture
if they spend time sounding and searching the
appropriate wall, floor, or ceiling. A thief has a (2
× Perception + 5 × Rank)% chance of noticing that
a secret or hidden aperture is within (5 + Rank) feet
of them.
If the GM’s roll on percentile dice is equal to or
less than the success percentage, the thief senses
that at least one hidden or secret door is in their
detection area (but is not told how many). If the
roll is greater than the success percentage, the thief
does not notice the aperture(s).
A thief can attempt to pick the pocket of another being without being detected.
A thief has a base success percentage equal to (3 ×
Manual Dexterity + 6 × Rank)% to pickpocket a
being. The following modifiers are applied to the
success percentage:
The victim is unconscious
The victim is sleeping or stunned
The victim cannot see well in current circumstances (e.g. human at
night)
The victim is inebriated
The pickpocket attempt is made in
an uncrowned area and the victim
has at least a slight suspicion of the
thief’s intentions
The object to be pickpocketed is in a
sealed pocket, pouch or compartment
The object to be pickpocketed is
affixed to the victim’s person or is
something used constantly during
the day by the victim
The object to be pickpocketed
makes noise when removed
The victim wears metal armour or
garments
The victim is an assassin, thief or
spy

+50%
+25%
+10%

+5%
-15%

-20%

-30%

-25%
-5%

-5 × Victim’s
Rank%
It is assumed that the thief attempting to pickpocket is not handicapped by their physical condition; if they are, the GM should modify the success
percentage accordingly.

129

If the GM’s roll of percentile dice is equal to or
less than the success percentage, the thief filches
the object they desire without their victim noticing.
If the roll is between one and two the success percentage, the thief is detected by the victim just after
the object has been removed from its storage place.
If the roll is equal to or greater than twice the success percentage, the thief is caught with their hand
in the victim’s pocket.
A thief will develop a photographic memory as
they gain experience.
A thief’s success percentage is (Perception + 10 ×
Rank)%. A thief may use the ability without error
for up to (1 + Rank) days. When a thief uses the
ability after the error-free time limit is expired,
reduce the Rank for success percentage calculation
(only) by one for each day over that time limit.
If the GM’s roll on percentile dice is equal to or
less than the success percentage, the thief can
recall visual details, such as those of a room or a
piece of parchment, etc., if they observed it for the
requisite length of time. A thief must have observed the object in question for (240 - 20 × Rank)
seconds to use the ability. If the roll is greater than
the success percentage, the thief’s memory has
more and more gaps in it as the roll approaches
100. If the thief is attempting to recall past their
error-free time limit, the GM introduces erroneous
information into the memory gaps as the roll approaches 100.
A thief tests their photographic memory ability
whenever they try to verbally describe an object or
place, whenever they call on their memory to gain
a mental image of the object or place, or whenever
they record it in writing. If a thief fails to recall an
object or place once, they may not use the ability
again to try to recall the image of that object or
place.
A thief increases their chance to perform an
action involving stealth by 1% per Rank.
A thief can, as long as they may find a purchase
sufficient to bear their weight, climb any structure.
The success chance when climbing on a structure
not made for that purpose is (4 × MD + 10 × Rank)
- (Structure Height in Feet / 10)%. Round the structure height down. If the GM’s roll is greater than
the success percentage, the thief has fallen in
climbing the structure. To determine the height at
which the thief falls, roll D100. Round the number
off to the nearest
10% (a roll of 5 is rounded down), and multiply the
height the thief sought to attain by that percentage.
See §29.1 for falling damage.

48 TROUBADOUR

48 Troubadour (Ver 2.1)
A troubadour is a multi-talented entertainer and
performer, and a well skilled troubadour may be an
actor, poet, mimic and musician. The most powerful ability that a troubadour can gain is the Bardic
Voice, which enables them to influence all but the
deaf.

48.1 Benefits
All troubadours gain a grounding in stagecraft.
They become able to size up an audience and determine what form of entertainment will be the best
to perform, and how to handle interjection and
ridicule.
In addition, a troubadour gains 3 abilities at Rank
0, and one further ability per Rank. All abilities are
usually performed at the overall Rank of the troubadour. However, a troubadour may choose to
specialise. If, upon gaining a new Rank (or an
additional ability without increasing in rank), the
troubadour wishes to forego gaining a new ability,
they may specialise in one of the abilities that they
already possess. That ability then operates at (troubadour’s Rank + 1), to a maximum of (Rank +
specialisation) of 10. A troubadour may specialise
more than once with the same ability, gaining Rank
+ 2, Rank + 3, etc. Additional abilities may be
gained without increasing in rank by the expenditure of 1,000 Experience Points and 4 weeks of
training per ability. These costs are discounted by
25% if the troubadour has reached rank 8, or by
50% if they have reached rank 10.
Individual Base Chances are not provided for the
various troubadour skills; rather, there is a generic
Base Chance of 3 × appropriate characteristic (+ 5 /
Rank), modified by the GM to reflect the difficulty
of the feat being attempted.
The abilities available to a troubadour are:
Acrobatics mostly involves tumbling across the
ground, but also performing manoeuvres after
swinging from a trapeze, rope, or bar; jumping
from a springboard, or high ledge.
Acting Portraying fictitious personalities and devising rationales for assumed identities. Usually
involves accentuated and exaggerated actions and
emotions.
Bardic Voice (see below). Note: A troubadour
may not specialise in Bardic Voice.
Comedy The use of timing, inflection and language to cause merriment or laughter. Also writing

both jokes and skits. Comedy may also be combined or included in many other art forms.
Dance Mostly traditional, often rural dances, performed for an audience; also includes creating new
dances.
Fire Eating Appearing to swallow and/or produce
flame, usually, from the mouth. To do this, a fire
eater requires a special liquid, which may be purchased from an Alchemist for a modest fee.
Juggling Throwing and catching objects. A juggler
is able to keep up to 1 (+ 1 / Rank) items, of equal
weight and size, in the air at the same time. If the
items juggled are of a different size and/or weight,
each difference counts as another item juggled.
Make-up using props, stage makeup, and items
such as wigs, fake beards, and wax noses, a troubadour can portray a character of a different age,
race, sex, or profession to their own.
Mime Using only the performer’s body, and its
movements, to convey an idea, describe a scene,
tell a tale, or entertain.
Mimicry Imitating sounds and voices accurately
and believably.
Patter Talking interestingly, seemingly non-stop,
either as advertising for a show or as a misdirecting
part of a performance.
Play an Instrument This ability may be taken
several times with different instruments. A singer
is one who has play instrument (voice). A troubadour can usually play similar instruments to the
ones they have chosen at (Rank / 2).
Poetry Creating and reciting poetry, including
lengthy epics running to hundreds of lines.
Prestidigitation Manipulation of small articles
such as coins, eggs, or pebbles to make them move,
disappear and reappear in unusual and entertaining
ways. This ability also gives a bonus to the casting
of all Cantrips of 2% (+ 2 / Rank).
Production Play writing and turning a play into a
successful production. Includes set design and
sound effects. The higher the Rank, the less likely
it is that a major catastrophe will befall the production through something having been forgotten or
overlooked.
Puppetry Writing a story to be performed by
puppeteers, and performing a story or play with
puppets.

130

Stilt Walking Balancing and walking on stilts of
up to 50% (+ 20% / Rank) of the troubadour’s
height.
Storytelling Creating and reciting stories for an
audience.
Sword Swallowing Controlling the mouth, tongue
and throat such as to be able to allow long, rigid
props to pass into the throat.
Tightrope Walking Walking, balancing, and
turning on a taut raised rope, or narrow beam.
Ventriloquism The ability to speak without moving the lips and make the voice seem to come from
any location up to (Rank / 2) feet away.

48.2 Bardic Voice
A troubadour may use their Bardic Voice in an
attempt to influence an audience. Beings who are
affected will see the troubadour as their friend, and
the troubadour’s words as wise and well meant.
Bardic Voice may be used, for example, to calm a
lynch mob, or to begin a riot against a cruel tyrant.
The troubadour begins speaking to key elements in
the crowd, stirring their emotions and playing upon
their beliefs and feelings. All beings to be affected
must be within earshot, and capable of understanding the language used by the troubadour. When the
troubadour begins to use this ability they may
enthrall up to (4 + 6 / Rank) beings, with (15 Rank) minutes being required to work their skill.
Once they have spent the required time, the troubadour makes a Check to see if they are having the
desired effect. If successful, the troubadour may
elect to use their voice again on the same crowd.
By doing this they may double the number of beings whose attention they have captured.
Using Bardic Voice is tiring, and a troubadour
must expend 4 FT each time that they use this
ability. A troubadour may use their voice continuously upon a crowd until they exhaust their FT,
they reach the limit of the size of audience, they
fail a Bardic Voice roll or they have doubled (Rank
/ 2, round down) times. The Base Chance is 50%
(+ 5 / Rank), modified by the GM for the reasonableness of the troubadour’s suggestions and the
audience’s predisposition to certain actions.

49 WARRIOR

49 Warrior (Ver 3.0)
Warriors train in advanced combat and weapon
techniques, and practice the art of melee combat
until it becomes a series of finely honed instincts.
They practise on all terrains with all types of
weapons against different configurations and tactics, so that they may always respond optimally
without the need for thought.
Warriors usually join and train within one Warrior
Guild (or school), most guilds are exclusive and
will not train those of other guilds, many have
strong ethical codes such as Chivalry and training
is often only available after indoctrination in the
code.

weapons, with which they have ranks. If they
have a passing familiarity with the weapon
(i.e. has Ranks in the same category but not
the actual weapon) they will act as a merchant of half their warrior Rank.

Overstrike (Weapon) - may apply damage bonuses for both skill and strength as a Special Attack.

A Rank 5 Warrior may start to acquire Warrior
Special Abilities.

Advanced Disarm (Weapon) - advanced training
in disarming opponents. Reduces Disarm SC%
penalty by 2% per rank. In addition if the Disarm
attempt is an EN hit then the chance of retaining
the weapon is halved, on a SpecGrev the target
may not attempt to retain the weapon.

While some abilities are general combat abilities
and only need to be learned once, most are weapon
specific and must be learned once per weapon. A
warrior gains one special ability at ranks 5, 6, & 7,
two at ranks 8, 9, & 10.
Additional special abilities may be gained without
increasing in rank by the expenditure of 10,000
Experience Points and 8 weeks of training per
ability. These costs are discounted by 25% if the
warrior has reached rank 8, or by 50% if they have
reached rank 10.

49.1 Restrictions
A Warrior may never train in the Warrior skill
without a training partner of at least equal Warrior
rank.
Assassin teaches a method of fighting that does not
stack with Warrior. The benefits of Warrior OR
Assassin (not both) may be applied to any specific
attack or defence.

49.3 Warrior Special Abilities
Close-Evasion (General) - may Evade while in
close combat. Standard defence bonuses for evasion and potential Parry.

A Warrior must have at least 3 weapons at rank 4
or higher including 1 melee rated weapon and 1
close rated weapon.

Close-Withdraw (General) - may add their Warrior Rank to their Strength for the purposes of
Withdraw from Close.

A Warrior requires various minimum weapons
skills per Rank. Before being able to achieve any
new Rank in the warrior skill, a Warrior must
achieve the Ranks summarised in table 49.5 below.

Pre-Engage (General) - may choose to act on their
Engaged IV provided they are within half Warrior
rank hexes of an engagement and they are joining
the engagement as their action.

49.2 Benefits

Two Step (Weapon) - may move an additional 1
TMR before executing a Melee Attack, Special
Attack, or Charge.

A Warrior gains a bonus to defence in melee
and close combat of 2% (+2% per rank).
When using ranked melee or close rated weapons, or weapons from a category where they
have a higher ranked weapon, a Warrior:

gains +1 Engaged IV per rank (no benefit at
rank 0)

gains +2% (+2% per rank) to Strike Chance

gains +1 damage at ranks 5 & 10

gains +1 effective weapon rank for the Parry
Calculation at ranks 4 & 8.

Advanced Multi-Hex (Weapon) - advanced training in striking multiple opponents. Reduces MultiHex Strike SC% penalty by 2% per rank.
Double-Hex Strike (Weapon) - May make a Special Attack (as per Multi-Hex Strike) against 2
adjacent opponents with a 1-handed B or C Class
weapon. May be combined with Advanced MultiHex to reduce the SC penalty.
Multi-Hex Bash (Weapon) - May make a Special
Attack (as per Multi-Hex Strike) with a 2-handed C
Class weapon. May be combined with Advanced
Multi-Hex to reduce the SC penalty.

For categories where the Warrior has a ranked
weapon, they:

may use any weapon from the category at
half the Rank of their maximum Ranked
weapon in that category or at half their warrior Rank, whichever is lower.

may Rank weapons in the category without a
tutor for an additional 10% EP cost.

is treated as a merchant of their warrior Rank
when attempting to buy or value non-magical

Draw and Strike (Weapon) - may combine a
Prepare Weapon Pass Action and Melee Attack as
a Special Attack.

Repulse (Weapon) - add 2 to effective weapon
rank when attempting to repulse a closing figure.

Strong Guard (Weapon) - add 2 to effective
weapon rank when defending against a
Parry/Riposte. And may add 2% per Warrior Rank
to chance of resisting a Disarm.
True Riposte (Weapon) - may add 1 per 2 full
ranks in Warrior to the Parry/Riposte Calculation
when evading.
Shield Block Shield an adjacent figure as a free
action. Loose all defence from their shield and
contribute half of it to an adjacent target for the
remainder of the pulse.
Quasi-Magical Warrior Abilities
These abilities cost Fatigue to use (exertion/tiredness FT, not damage). They may only be
performed if the FT cost can be paid.
Fortitude Costs 2 FT, must be declared at the start
of the pulse. Adds Warrior Rank to the amount of
damage required to be Stunned and 2% per rank
bonus to fear checks for the remainder of the pulse.
Unstoppable Costs 2 FT. May re-roll a failed
Consciousness or Stun Recovery check. Or, may
make a 1xWP check each pulse to stay conscious
when between 0 and -rank Endurance.
Quick Stand Costs 2 FT. May leap to their feet
from a Prone position as a free action.
Solid Strike (Weapon) Costs 1 FT per damage
D10 rolled, must be declared before making the
attack roll. Any roll of less than half Warrior rank
is increased to half Warrior rank. I.e. A damage
dice roll of 2 by a rank 7 Warrior becomes 4.
Lightning Strike (Weapon) Costs 1 FT, must be
declared at the start of the pulse. May add twice
Warrior Rank to engaged initiative for a pulse.
Full Defence (Weapon) Costs 1 FT, must be declared at the start of the pulse. Add Warrior rank to
defence, or twice Warrior rank when evading.

Off-hand (Weapon) - special training in using the
weapon in either hand, treated as ambidextrous
when using this weapon.

49.4 Weapon Categories
Note: These definitions only apply to the warrior skill, not combat in general.
1. Shortswords

4. A Class Swords

7. Blunt

10. B Class Pole




















Dagger
Main Gauche
Shortsword
Sai

Rapier
Estoc

War Club
Mace
War Hammer
War Pick
Mattock

Halberd
Poleaxe
Glaive
Quarterstaff

2. Single Edged

5. Oriental Swords

8. Entangling / Chain

11. Unarmed




















Tulwar
Falchion
Scimitar
Sabre

Katana/Wakizashi
O-Dachi
Ninjato

Flail
Morningstar
Bola
Net
Manriki Kusari
Nunchaku

Unarmed
Cestus

3. Double Edged

6. Axes

9. A Class Pole

12. Shield






















Hand & a half
Two-Handed Sword
Claymore
Broadsword

Hand Axe
Battle Axe
Great Axe

131

Javelin
Spear
Pike
Lance
Trident

Tower
Kite
Large Round
Small Round
Buckler

49 WARRIOR

49.5 Ranking Requirements
Warrior Rank

No. of Categories

Minimum Rank

No. Weapons ≥
Rank 4

0
1
2
3
4
5
6
7
8
9
10

4
4
7
4
7
4
7
4
7
4
7

0
1
1
2
2
3
3
4
4
5
5

1

No. Weapons ≥
Rank 5

No. Weapons ≥
Rank 6

No. Weapons ≥
Rank 7

1
1
2
2
1
2
1
2
3
4

132

50 WEAPONSMITH

50 Weaponsmith (Ver 1.1)
50.1 Restrictions
The skill is related to that of armourer, and a
weaponsmith who is a more skilled armourer expends only three-quarters of the necessary Experience Points to acquire or improve this skill. The
reverse is also true.
A weaponsmith’s progress in their skill is inhibited
by a low Manual Dexterity, and aided by a high
Manual Dexterity. A weaponsmith has an increased Experience Point cost of 5% for each point
of Manual Dexterity less than 16. A weaponsmith
decreases their Experience point cost by 5% for
each point of Manual Dexterity greater than 20. A
weaponsmith will have great difficulty passing
their apprenticeship if their Manual Dexterity is
less than 10.

50.2 Benefits
A weaponsmith acquires one ability every two
Ranks. The character begins with one of the
following abilities at Rank
0. All abilities can be performed skillfully.
1. Make and maintain swords including daggers.
2. Make and maintain hafted weapons.
3. Make and maintain thrown weapons.
4. Make and maintain pole weapons.
5. Make and maintain missile weapons.
6. Make and maintain entangling weapons.
7. Make and maintain experimental weapons.
8. Make and maintain siege engines.
9. Make and maintain shields.
Additional abilities may be gained without increasing in rank by the expenditure of 5,000 Experience
Points and 4 weeks of training per ability. These
costs are discounted by 25% if the weaponsmith
has reached rank 8, or by 50% if they have reached
rank 10.

A weaponsmith can build increasingly more
effective weapons as their Rank increases.
For every Rank that a weaponsmith achieves, they
may create weapons that have an increased base
chance of 1%. For every Rank divisible by five that
a weaponsmith achieves, they may create weapons
that cause an extra point of damage. These two
effects are not cumulative.
For example, a Rank 8 weaponsmith may construct
a weapon with a Base Chance increased by 3% and
a Damage Modifier increased by 1, or a weapon
with a BC increased by 8% and no increase in DM.
Note: The weapons statistics as shown in the
weapons chart are manufactured at an effective
Rank of 0 i.e. they are the mass-produced variety.
They may have been manufactured by a weaponsmith of greater Rank than this, but the skill level
used was elementary.
The time and cost required for a weaponsmith
to construct a weapon is dependent on the Rank
that is used, and the type of weapon.
A Weapon may be manufactured at any Rank up to
the weaponsmith’s Rank.
1. The time required is (10 × (Effective Rank +
DM)) hours, with a minimum of 10 hours.
2. The cost is 80% of the Base Cost as shown in
the weapons table × (1 + effective Rank + DM
increase) silver pennies.
3. For every rank that a Weaponsmith has beyond
the effective rank of the of the weapon, they reduce
the time required, as given above, by 5%. For
example, a Rank 8 weaponsmith churns out a Rank
0 weapon in only 6 hours, or produces a +1 Damage sword, no BC modifier, in 51 hours rather than
60.

133

A weaponsmith is treated as a merchant of their
weaponsmith Rank when attempting to buy or
value weapons which are part of their abilities.
If the equipment concerned is unfamiliar, then they
operate as a merchant of half their Rank (rounded
down).

50.3 Costs
A weaponsmith, with the exception of some
missile weapons, can only perform their skill in
a properly maintained workshop.
It costs 2000 silver pennies to construct a workshop and 500 silver pennies per year to maintain it
with tools and materials. A basic tool kit will cost
100 + (100 × Rank) silver pennies. A workshop
may be rented at a cost of 10 silver pennies per
day.

50.4 Silvering Weapons
A weaponsmith may incorporate an additional
metal during the manufacture of a weapon. This
improves the appearance of the weapon, and may
provide other benefits as given elsewhere in the
rules.
Cost is 80% of Base Cost as shown in the weapons
table × (metal factor + effective rank + DM increase). The metal factor is 1 for cold iron weapons, 10 for silvered, 120 for gilded, and 180 for
truesilvered.
For example: a truesilvered rank 10 Hand & a half
with +2 DM costs (80% × 85) × (180 + 10 + 2) =
13,056sp or a gilded Rank 6 Great Axe with +1
DM and +1% SC costs (80% × 30) × (120 + 6 + 1)
= 3,048sp.

134

51 GRIEVOUS INJURY TABLE

51 Grievous Injury Table
A class weapons do grievous injuries on rolls of 01
through 20.
B class weapons do grievous injuries on rolls of 21
through 80.
C class weapons do grievous injuries on rolls of 70
through 00.
01–05 Congratulations! It’s a bleeder in your
primary arm! Take 1 Damage Point from
Endurance immediately and 1 per pulse
thereafter until the flow is staunched by a
Healer of Rank 0 or above or you die.
06–07 Oh no! Your opponent’s weapon has
entered your secondary arm’s elbow joint
and the tip has broken off. Take 2 Damage Points immediately from Endurance
and that arm is useless until the sliver has
been removed by a Healer of Rank 3 or
above. Also, increase the chance of infection by 30.
A vicious puncture wound in your groin!
08
Take 3 Damage Points immediately from
Endurance and reduce your TMR by 2
until fully recovered, which will take two
months. In addition, add 30 to your
chance of being infected (assuming you
live long enough for such things to matter).
09–10 You have been stabbed in your secondary
arm. Drop whatever you were holding in
it and take 2 Damage Points immediately
from Endurance. It will take a full week
for the arm to be of any use to you whatsoever.
Your aorta is severed and you are quite
11
dead. Rest assured your companions will
do their best to console your widow(er).
A stomach puncture. Nasty. You suffer 3
12
Damage Points immediately from Endurance and lose 2 from your TMR until
fully recovered, which will take two
months. Also, you are automatically
stunned for the next pulse (if you aren’t
already), after which you may recover.
Add 20 to the chance to be infected.
Your opponent’s weapon has entered your
13
eye. Roll D10. On a roll of 1, the weapon
has entered your brain and you are dead.
On a roll of 2–5, your left eye is blinded.
On a roll of 6–10, your right eye is
blinded. If you are lucky enough to be
blinded instead of killed, you have suffered 2 Damage Points to Endurance. In
addition, a figure who is blind in one eye
suffers the following subtractions: -1 from
MD, -2 from PB, -4 from Perception. A
figure blinded in one eye reduces their
base chance with any missile or thrown
weapon by 30.
14–18 Tsk, tsk. A wound of the solid viscera.
Usually fatal. Take 3 Damage Points to
Endurance immediately and 1 per pulse
thereafter until the bleeding is stopped by
a Healer of Rank 2 or above or you die.
Add 30 to the chance of infection.
19–20 Take a stab in the leg (your choice as to
which one) resulting in a deep puncture of
the thigh muscle. Suffer 1 Damage Point
to Endurance immediately and reduce
your TMR by 1 until you heal, which will
take 4 weeks.
21–25 A chest wound. Take 2 Damage Points to
Endurance immediately and reduce your
TMR by 1 until recovered (about 2
months). Look on the bright side, though.
Your attacker’s weapon is caught in your
rib cage and has been wrenched from
their grasp.

26–27

28–30
31–34

35

36
37–40

41–42

43

44–50

51–52

53–60

61–67
68–69

70–74

Bad luck! Your secondary hand has been
severed at the wrist. Take 2 Damage
Points to Endurance immediately and 1
point per pulse from Fatigue thereafter
(Endurance when Fatigue is exhausted)
until you are dead or the bleeding is
staunched by a Healer of Rank 0 or
above. If you live, reduce your MD by 2.
Worst luck! Your primary hand has been
severed. See result 26–27 for effects.
A minor wound. Your face is slashed
open, ruining your boyish good looks and
causing blood to spurt into your eyes.
Reduce your PB by 4 permanently.
Your secondary arm is sliced off at the
shoulder. Take 5 Damage Points immediately from Endurance and 1 per pulse
thereafter from Fatigue (Endurance when
Fatigue is exhausted) until you are dead
or the bleeding is staunched by a Healer
of Rank 1 or above. Reduce your MD by
2 and your AG by 1.
The same as 35, except it’s your good
primary arm that has been lopped off.
You have been eviscerated! Take 4 Damage Points immediately from Endurance
and 1 point per pulse from Fatigue thereafter (Endurance when Fatigue is exhausted) until you are unconscious. Increase your chance of infection by 40.
A glancing blow lays open your scalp and
severs one ear (your choice as to which
one). Take 2 Damage Points immediately
from Endurance. Reduce your Perception
by 2.
A savage slash rips open your cheek and
jaw. Take an automatic pass action next
pulse due to the shock of the blow. Your
PB is increased by 1, since your disfigurement will bring out the maternal/paternal instincts in the opposite gender.
A slash along one arm, and it’s a bleeder!
Take 2 Damage Points immediately from
Endurance and lose 1 point from Fatigue
(Endurance when Fatigue is exhausted)
each pulse until the bleeding is stopped by
a Healer of Rank 1 or above or you die.
Hamstrung! Roll D10. On a roll of 1–4, it
is your left leg. On a roll of 5–10 it is your
right. Take 4 Damage Points immediately
from Endurance and fall prone. You may
not stand unassisted until the wound is
healed (which should take three months).
Reduce your AG by 3 permanently.
Your primary arm is crippled by a wicked
slash! Take 2 damage Points immediately
to Endurance and drop anything you have
in your primary hand. The arm is unusable until healed, which should take 2
months.
Your secondary arm is crippled; see 53–
60 for details.
A nasty slash in the region of the shoulder
and neck. Roll D10. On a roll of 1–3, your
head is severed and your corpse tumbles
to the ground. On a roll of 4–6, your
secondary collar bone is crushed; on a roll
of 7–10 your primary collar bone is
crushed. If your collar bone is crushed,
the results are identical to 53–60, except
you suffer 4 Damage Points to Endurance.
A crushing blow smashes your helmet and
causes a concussion. Take 3 Damage
Points from Endurance and suffer a reduction of 4 in both MD and AG lasting for 3
days.

135

75–80

81–84

85–87

88–89
90–92

93–94
95–97

98–
100

A massive chest wound accompanied by
broken ribs and crushed tissues. Very
ugly, this. Take 5 Damage Points immediately from Endurance. Reduce your MD
and AG by 3 each until this wound heals
(which should take about 4 months).
Increase your chance of infection by 10.
A crushing blow smashes tissue and produces internal injuries. You suffer 2
Damage Points immediately to Endurance
and 1 per pulse thereafter to Fatigue (Endurance when Fatigue is exhausted) until
unconscious or you receive the attention
of a Healer of Rank 2 or above.
A jarring blow to your primary shoulder
inflicts 2 Damage Points immediately to
Endurance. Roll D10; the result is the
number of pulses the arm is useless. You
immediately drop anything held in that
hand.
Similar to 85–87 except it is your secondary shoulder.
Your right hip is smashed horribly. Take
5 Damage Points immediately to Endurance and fall prone. You will be unable to
walk until the damage has healed (which
should take about 6 months). Good fun.
When healed, you will still have a limp
which will reduce your TMR by 1 and
your AG by 2.
The same as 90–92 except it is your left
hip that is smashed.
Your opponent’s weapon has come crashing down on your head and fractured your
skull. You fall prone and are unconscious,
and take 8 Damage Points to Endurance.
If you survive, you lose 2 from AG, 2
from MD and 2 from Perception. It will
take a year in bed to recover.
Crushing blow to your pelvis breaks bone
and tears tissue. Take 7 Damage Points
immediately to Endurance and fall prone.
Make a WP check to avoid falling unconscious. If you survive, you will be unable
to move for D10 months.

Notes
The suggested recovery times are a guideline for
the GM to use in determining how long characters
should be kept out of action. The actions of a competent Healer may alter these times in some instances.
These Grievous Injuries are designed for combat
between human-sized opponents; any injuries
sustained involving larger monsters should be
applied judiciously by the GM, taking into account
size and mass differences, etc. In some situations,
the GM may have to disallow the Grievous Injury
or change its effects.

52 FUMBLE TABLES

52 Fumble Tables
When an attacker fumbles, they lose 10 from their
Initiative Value until the end of the next pulse.
Then they make a totally unmodified D100 roll. If
that roll is under their current Initiative Value, they
suffer no further penalty for their slight fumble; if
it is not under their current Initiative Value, apply
the corresponding result from the appropriate table
below. See (§6.10).

52.1 Consequences
A broken weapon is useless until repaired; a shattered weapon is useless until reforged. Any combat
spell on a broken or shattered weapon is dissipated.
A damaged weapon is bent, dented, nicked, or
similarly flawed. You may still use the damaged
weapon but it does 1–2 points less damage and has
0–20 penalty to its strike chance (GM decides these
figures), until repaired.
A damaged magical weapon must be repaired by a
Weaponsmith of at least Rank 6. A damaged
weaponsmithed weapon loses all its weaponsmith
bonuses to Strike chance and/or Damage, until
repaired.
Any self-inflicted damage ignores your armour
(including magic); it usually represents bruising,
minor strains, etc. Naturally, take EN damage if
you have no FT available. If a specific injury is
stated (e.g. pulled groin muscle, or broken bones),
then healing requires a lot of time or the appropriate minimum rank of Healer.

52.2 Special cases
An innately magical weapon
• ignores asterisked results (*).
• does not break.
• may shatter or be damaged, but less often.
• does not include any non-magical weapon merely
under the effects of magic.
Unarmed Combat
• “Shattered” = broken bone(s); lose 2 EN; useless
in combat until healed.
• “Broken” = Seriously bruised; lose 2 FT; may be
used in combat, but (until healed) each successful
blow does 2 damage points less to opponent and 2
FT to yourself.
• “Damaged” = Ouch! Lose 2 FT; no further effect.
Strike Chance over 100
If the fumble indicates a broken or damaged
weapon, but your modified Strike Chance was over
100, you have also struck your opponent; roll [Fatigue] damage as normal.

52.3 Bows and Crossbows only
01–12 Bowstring snaps and lashes you; lose 2 EN.
Bowstring snaps and lashes you in the eye;
lose 2 EN; you are blinded in one eye for 3
weeks or until cured by a Rank 7 Healer. A
figure who is blind in one eye suffers the
following subtractions: 1 from MD, 2 from
PB, 4 from Perception and reduces their
base chance with any missile or thrown
weapon by 30.
14–29 Bowstring snaps; no further penalty.
13

Traditional Hunting accident. Clumsy release causes arrow / quarrel to fly towards a
random “friendly” back in approximately
the same direction as you were aiming:
Strike Chance = Weapon BC + weapon
bonuses + 30 - target’s defence.
31–33 Clumsy release; bolt/arrow flies wide missing friends and foes.
34–36 Brief twinge of pain in your arm or back;
Lose 1 EN.
30

37–39 as per 34-36, but lose 2 EN.
40–59 Dropped bolt or quarrel.
60–99 Bowstring snaps; no further damage.
00

Bowstring snaps and lashes you; lose 2 EN.

51–52 Your vigorous swing causes a slight twinge.
Make 3 × EN or lose 2 FT.
Make 3 × EN or pull a groin muscle, lose 2
53
FT and have half Base TMR until healed.
54–55 as per 51–52, but make 2 × EN.
as per 53, but make 2 × EN.

52.4 All Other Weapons

56

The following results are generalised. Therefore
the GM is free to ignore or downgrade any result
which is inapplicable to a specific case. Some
outcomes are avoidable through a successful characteristic check.

57–58 as per 51-52, but make 1 × EN.

01–09 Shattered weapon.
Shattered weapon; some slivers fly at you,
potentially causing you a grievous injury —
roll on the Grievous Injury Table (§51), but
ignore any result over 13.
11–12 Shattered weapon, flying splinters; you and
your opponent(s) lose 1 EN each.
Playing the Roman fool? You just did your13
self an Endurance blow; fortunately you
rolled minimum damage (but don’t forget
the extra damage from poison, magic, etc).
14–16 Your wild swing possibly connects with
someone other than your intended target or
yourself – immediately make a strike check
at your new victim, the nearest being in
range other than you or your intended target.
Hope you weren’t mounted.
17–18 Lose 1 EN. Feels as if you pulled something.
Lose 2 EN. You really pulled something.
19
10

Oops! You’ve flung your weapon in a high
parabolic arc. Normally a flung or dropped
weapon falls without hurting anyone —
however, in this case, it falls on a random
target, possibly even you, and maybe hurts
them: Strike Chance = [Weapon’s BC] +
[magical / weaponsmith bonuses] - [random
target’s defence].
21–26 Butterfingers! Make 3 × MD to avoid your
weapon flying 2–3 hexes in a random direction.
27–28 Klutz! Make 3 × MD to avoid dropping your
weapon in your hex.
Whoops! You’ve caught your weapon in
29
your own armour or gear. You may choose
to automatically free it in the next pulse, in
which case you may not attack or cast magic
until after the end of the next pulse. Or else
you may wish to prepare another weapon in
your next action.
Overly enthusiastic lunge. GM moves you to
30
an unoccupied forward hex (make 3 × AG to
choose your new facing) — but if no empty
hex is available, you just tried to close on an
opponent, who gets a free chance to keep
you out of close. If you did close, you don’t
have to drop any non-close weapon, but it
may not be used to attack effectively.
31–32 Poor balance; make 3 × AG or no offensive
action until after the end of the next pulse.
Stumble; make 3 × AG or fall prone.
33
20

34–35 as per 31–32, but make 2 × AG.
36

as per 33, but make 2 × AG.

37–38 as per 31–32, but make 1 × AG.
39

as per 33, but make 1 × AG.

40

Broken weapon.

41–49 Damaged weapon.
50

Momentary dizziness; make 3 × EN or you
may not attack or cast magic until after the
end of next pulse.

136

59

as per 53, but make 1 × EN.

No effect unless you used a A or B-class
melee weapon against an opponent with a
non-magical shield. In which case, you have
spectacularly wedged you weapon into their
shield. Make 1 × PS to immediately wrench
your weapon out, or it will be torn from
your grasp in the fracas. Don’t worry if you
fail — perhaps their shield is now useless?
61–62 Your melee weapon is stuck, caught, or
entangled in your opponent’s armour or gear
(and you didn’t even hurt them). Make 3 ×
PS to immediately disengage your weapon,
or it will be yanked from your grasp in the
fracas.
You palpably hit a tree, rock, wall-hanging,
63
furniture, or some other adjacent “scenery”.
Make 3 × PS to immediately disengage/extract your weapon. You may try
again, as a future action; but perhaps you
should prepare a new weapon instead.
64–65 as per 61-62, but make 2 × PS.
60

66

as per 63, but make 2 × PS.

67–68 as per 61-62, but make 1 × PS.
69

as per 63, but make 1 × PS.

Shattered weapon, if it is not at least Rank 1
weaponsmithed.
71– Your weapon breaks unless you roll under
73* its weaponsmith rank on D10. Indeed it
shattered if you failed the roll by 5 or more.
74–77 Twinge of pain. Take (D10 - rank in
weapon) FT damage.
78–79 as per 74–77, but also you may not attack or
cast magic for remainder of the pulse.
Your weapon flies from your grasp. You
80
may choose to drop whatever is in your
other hand; in which case, make 3 × MD to
catch the weapon in that other hand.
81–82 Butterfingers! Make 3 × MD to avoid your
weapon flying 2–3 hexes in a random direction.
Klutz! Make 3 × MD to avoid dropping your
83
weapon in your hex.
84–85 as per 81-82, but make 2 × MD.
70*

86

as per 83, but make 2 × MD.

87–88 as per 81–82, but make 1 × MD.
89

as per 83, but make 1 × MD.

90

Broken weapon.

91–99 Damaged weapon.
00

Your bizarre but highly spectacular fumble
is mistaken for an obscure martial technique.
All engaged melee opponents hastily elect to
neither attack or cast magic as their next
action. If you have another action before
they actually perform their next action, you
may choose to run away (retreat up to full
TMR) as your action without the need for a
Withdrawal manoeuvre — you are no longer
engaged with those particular opponents.

53 BACKFIRE TABLE

53 Backfire Table
Extra Fatigue The following effects result in the
Adept losing extra Fatigue in order to contain the
Backfire effect. If no Fatigue was expended during
the cast then there will be no apparent effect. If the
Adept has insufficient Fatigue, then they lose all
remaining Fatigue and re-roll on this table for each
Fatigue point they were unable to expend, with any
further rolls of Fatigue loss taken as damage due to
internal injuries.
01–09 Fatigue loss equal to that already expended.
10–16 Fatigue loss equal to twice that already
expended.
17–21 Fatigue loss equal to 3 times that already
expended.
22–24 Fatigue loss equal to 4 times that already
expended.
Fatigue loss equal to 5 times that already
25
expended.
Spell Awry The following effects result in the
spell working in a manner that was not intended by
the Adept. The spell may still be resisted by the
eventual target, if the spell is resistible. Note that
the Adept will not necessarily be aware of the
outcome, since in many cases there will be no
apparent effect.
26–30 The Adept becomes the target of the spell.
31–32
33
34–35
36
37–39
40–41
42
43–45

46–48

49
50
51

52–53

The Adept becomes the target of the spell
with some or all effects doubled.
The Adept becomes the target of the spell
with some or all effects tripled.
The Adept becomes the target of the spell
and the spell is delayed by D10 pulses.
The Adept becomes the target of the spell
and the spell is delayed by D100 pulses.
The spell has opposite or different effect
to that which it was designed.
The spell’s effects are delayed by D10
pulses.
The spell’s effects are delayed by D100
pulses.
The spell’s effects are intermittent with
D10 pulses or minutes on, followed by
D10 pulses or minutes off.
The spell affects a random target or area
within range, or goes in a random direction.
The spell affects a random target or area
within twice range.
The spell affects a random target or area
within three times range.
The spell is cast at random as though the
caster is a random entity within D10
hexes.
The spell affects a random target or area
within range with some or all effects
doubled.

The spell’s effects are delayed by D10
pulses and affects a random target or area.
The spell’s effects are delayed by D100
pulses and affects a random target or area.
The spell works with some or all effects
halved.
The spell works as normal.

82–83

The spell works with some or all effects
doubled.
The spell works with some or all effects
60
tripled.
Minor Curses The following effects result in the
Adept being afflicted by a minor curse. The Adept
may resist the curse by using their Magical Resistance against non–College Magic. Some curses can
be cured by healing skills, and all may be dispelled
by a curse removal. Any result specifying a gradual
loss of statistics will also cause an immediate loss.
Blind for D10 pulses.
61

54
55
56–57
58
59

84

Periodic muscle spasms lasting D10
pulses cause a loss of 1 Fatigue each
pulse. There is D10 × D10 minutes between spasms. This can be cured by Repair Muscles.
A deep sleep for D10 pulses.

85

A deep sleep for D10 × D10 minutes.

86–87

Recurring migraines cause a loss of 2
Magical Aptitude and 2 Willpower. Each
minute of concentration requires a 4 ×
Willpower concentration check. The effects can be treated by Soothe Pain and
cured by Repair Vital Organs.
Periodic hallucinations for D10 hours.
Each hallucination lasts D10 pulses and
there is D10 × D10 minutes between
them. Can be cured by Repair Vital Organs.
Arthritis causes -4 Dexterity, -4 Agility
and increases by 1 per hour the Fatigue
loss due to exercise, until treated by Repair Tissues.
Enfeeblement causes -4 Strength, -4 Endurance and doubles the Fatigue loss due
to exercise, until treated by Repair Muscles.
Asthma causes TMR to be halved, doubles the Fatigue loss due to exercise, and
the Adept cannot perform strenuous exercise until treated by Repair Vital Organs.
Creeping senility will cause a loss of 1
Magical Aptitude every two days until
treated by Regenerate Vital Organs.
Partial Amnesia causes the loss of all
Magical abilities for D10 days.
Partial Amnesia causes the loss of all
Skills (excluding Magic and Weapons) for
D10 days.
Partial Amnesia causes the loss of all
memories from the past 2D10 months.
The Adept will operate at lower ranks in
the abilities that have been ranked during
this period. The memories will return at a
rate of 1 month each day.
Total Amnesia causes the loss of all
memories for D5 × D5 days. All magic
and skills other than the primary language
will be lost, and all weapon ranks will be
halved (round down) or lost if Rank 0.
The Adept’s original personality will
come to the fore and they may need to
make a reaction roll to determine their
initial feelings towards each person.
Roll two more times and apply both effects.

88

89–90

62

Blind for D10 × D10 minutes.

63

Blind for D10 × D10 hours.

64

Blind for D10 days.

65

Deaf for D10 pulses.

66

Deaf for D10 × D10 minutes.

67

Deaf for D10 × D10 hours.

68

Deaf for D10 days.

69

Mute for D10 pulses.

70

Mute for D10 × D10 minutes.

71

Mute for D10 × D10 hours.

72

Mute for D10 days.

96

73

Lose smell and taste for D10 days.

97

74

Lose smell and taste for D10 × D10 days.

75

Lose tactile sense for D10 days.

76

Lose tactile sense for D10 × D10 days.

77

Insomnia such that only 1 Fatigue is recovered for each hour of sleep for D10
days.
Insomnia such that only 1 Fatigue is recovered for each hour of sleep for D10 ×
D10 days.
A virulent skin disease halves Physical
Beauty and causes intense itching which
will increase the difficulty of concentration checks by 1, until stopped by Cure
Disease.
Wasting disease causes -1 Strength and -1
Endurance per day until stopped by Cure
Disease. The Strength and Endurance lost
will be recovered at 1 point per day, or by
being treated by Repair Muscles.

78

79–80

81

137

91–92

93

94–95

98

99

00

54 FRIGHT & AWE TABLES

54 Fright & Awe Tables
Rolls against these tables can made as the result of
magic, when meeting entities who are extremely
ugly or beautiful, and in other surprising situations.
Modifiers
All modifiers are cumulative lasting 24 hours or
until 8 hours sleep, but each division only counts
once. For example, a target gets feared 3 times.
The first is a hysterical which adds 10%; second is
catatonic which adds 10%. Total so far is 20%. The
third, however, is another hysterical but as a hysterical component is already included in the 20%,
the total remains at 20%. A minor heart attack
within 24 hours would push the total to 30%.
Familiarity
In both the cases of extremely low and high beauty,
the effects are reduced with familiarity with the
creature causing the fear or awe. Once a character
has successfully made their Willpower check, or
has recovered from a failed check, they need not
check again (for an effect from the same creature)
for the remainder of the encounter. If the same
creature is encountered again another Willpower
check must be made, but the GM may add 1 or
more to the difficulty factor, thus making it easier
to succeed. This bonus should not be applied to
encounters with other similar creatures, only to the

same ones (it is not true to say that when you have
seen one Troll, you have seen them all).
PB Fright Checks
Whenever characters encounter a creature whose
Physical Beauty is less than 5, and whose description states a Fear causing ability, they must make a
Willpower check to determine if they are frightened. The difficulty factor for this test is equal to
the creature’s PB (use a factor of 1 if PB is 0). If
this test is failed, the character must then roll on
the Fright Table (see §54.1) and apply any results
before they take another action. They may attempt
to recover every pulse, by trying to succeed at the
same Willpower check. Until that time they will
act as the Fright Table indicates.
PB Awe Checks
Whenever characters encounter a creature whose
Physical Beauty is above 26, and whose description states an Awe causing ability, they must make
a Willpower check to determine if they are awed.
If this Willpower check is failed the character must
then roll on the Awe Table (see §54.2) and apply
any results before they take another action. Recovery from awe effects is as indicated on the Awe
Table. Awe can be caused by a creature with a
Physical Beauty greater than 26. The range of the

54.1 Fright Table (Ver 1.1)

effect is line of sight to the creature. The character
will only be affected if the creature is within their
line of sight, and they are facing towards it. Once
seen, however, facing becomes irrelevant, i.e. if the
Target turns away, so as to run, they do not lose the
Awe effect simply because they are no longer
facing the source. The effect will remain until the
Target is out of line of sight of the source. Should
the character return into line of sight of the source,
the same result as before will be applied, unless
their Willpower has increased, in which case, they
receive another Willpower check to resist the effect
of the Awe.
The difficulty factor of the Willpower check is
dependent on the creature’s PB:
PB

Difficulty PB Difficulty
Factor
Factor

27–28
29
30
31

4.0
3.5
3.0
2.5

32
33
34
35+

2.0
1.5
1.0
0.5

54.2 Awe Table

Wary The target will not voluntarily approach the source of their
fear. If they are not aware of the source they will be very cautious
and seek to optimise safety.
Berserk They immediately charge to attack the object of their
21–25
rage. If the source is not apparent they will charge about noisily
looking for it. Add +10 to Strike Chance and -10 to Defence.
Panic They will attempt to maximise their safety in relation to the
26–75
source of their fear. This usually involves fleeing as rapidly as
possible, but could also include cowering in the centre of the
party, curling up in a small ball, hiding under a bed, etc. While a
state of panic prevails, some sanity is present and the target would
not normally do anything suicidal (e.g. running over the edge of a
cliff) but they might use abilities to increase their safety (e.g.
flying away). If the target wishes to use an ability, (e.g. casting a
spell) the GM should give a suitable negative modifier to their
base chance (e.g. -20).
Frozen They may take no action until snapped out of it (e.g.
76–90
slapped on the face, attacked, etc). The target can attempt to break
out of it themselves by making a 1 × WP check per pulse. On
recovery, the target rolls again at -30 (with no other modifiers) to
determine their next action. Add +10 to subsequent rolls on the
fright table.
Hysterical They stand and scream and may take no other action
91–95
until snapped out of it (as for 76–90). On recovery, roll again at 20 (with no other modifiers). Add +10 to subsequent rolls on the
fright table.
96-100 Catatonic Target becomes catatonic. Their hair turns white and
they may take no other action until snapped out of it (as for 76–
90). On recovery, roll again at -20 (with no other modifiers). Add
+10 to subsequent rolls on the fright table.
101–110 Faints The target faints into unconsciousness and loses 5 Fatigue.
At the end of each minute they roll 1 × WP in order to regain
consciousness. Add +10 to subsequent rolls on the fright table.
111-115 Collapses The target collapses into unconsciousness and loses all
of their Fatigue. After (30 - Endurance) minutes, or being tended
by a Healer, they will regain consciousness. All their Characteristics and Ranks will be reduced by 2, and they will not be able to
recover Fatigue, until they have had comfortable bed rest for (40 Endurance - tending Healer Rank) hours. Add +10 to subsequent
rolls on the fright table.
Heart Attack The target suffers a heart attack and must receive
116+
the attention of a Healer of at least Rank 3 within Endurance
pulses or they are dead. If they survive they will be on 0 Endurance and 0 Fatigue, and will be unconsciousness for (30 - Endurance) minutes. All their Characteristics and Ranks will be reduced
by 5, and they will not be able to recover Fatigue or more than
half their Endurance, until they have had comfortable bed rest for
(60 - Endurance - tending Healer Rank) hours. Add +10 to subsequent rolls on the fright table.
< 20

Awe Target is slightly awed. Will not voluntarily approach the
source of the Awe, unless requested to do so by the source. If not
aware of the source, will be slightly cautious.
Enamoured Target is completely enamoured of the source. Will
21–25
do anything that the source requests, to the extent of attacking
comrades and friends, but not to the extent of killing themselves if
so requested. If the source is not apparent they will rush around
noisily looking for it.
Panic Target is panicked by the Awe. They will attempt to maxi26–76
mise their safety, as they perceive it. This may involve fleeing,
cowering, hiding, pleading and whimpering. They will not usually
further endanger themselves in their attempt to escape, by running
off a cliff for example. Add +5 to subsequent Awe Table rolls.
Humble Target is completely humbled and prostrates themselves
77–90
before the source of the Awe. They may take no other actions
until they are snapped out of their grovelling by an outside
agency, or by rolling less than or equal to their Willpower on
D100. They may attempt this roll every second Pulse after being
affected. Add +10 to subsequent Awe Table rolls.
Hysterical Target becomes hysterical and falls to the ground,
91–95
weeping, laughing, singing and/or praying as appropriate until
snapped out of it by an outside agency. They may attempt to
break out of it by themselves, by rolling less than or equal to their
Willpower on D100, at the end of each minute following the
affect. On recovery, roll D100: 1–50 slightly awed (as for 1–20,
this table), 51–55 enamoured (as for 21–25), 56–100 panic (as for
26–76). Add +10 to subsequent Awe Table rolls.
96–100 Catatonic Target collapses, becomes catatonic and may take no
further action until snapped out of it by an outside agency. Upon
recovery, roll D100: 1–26 slightly awed, 27–31 enamoured, 32–
95 panic, 96–100 hysterical. Add +15 to subsequent Awe Table
rolls.
101–106 Faints Target faints dead away and will remain unconscious for
[D + 6] minutes. Add +15 to all subsequent Awe Table rolls.
107–110 Mild Heart Attack Target suffer a mild heart seizure. The result
is the same as for 101–106 except that the Target may not move
about under their own power for the remainder of the day and
suffers a decrease of 2 to PS, MD, AG, EN, and FT, until either
they spend one month resting in bed, or their heart is repaired
using the Healer ability “Repair Tissues and Organs”. Add +15 to
subsequent Awe Table tolls.
Severe Heart Attack Target suffers a severe heart attack and
111+
must have the attention of a Healer of at least Rank 2 within one
minute (12 pulses) or they will die. Otherwise this result is the
same as for 107–110, except that +20 is added to subsequent Awe
Table rolls.
01–20

138

54 FRIGHT & AWE TABLES
Spell Effects
1. Wall of Bones (Necromancy): A target is not
affected by the wall until they touch it. If they fail
to resist they are affected any time they are in line
of sight of the wall but facing is not important (if a
target turns to flee the wall then they do not lose
the fear because they are no longer facing it). If the
target touches the same wall again they make another resistance check. Duration is that of the spell;
once the wall is gone, so is the fear.
2. Fear (Necromancy, Wicca, Celestial): single
target spell.
If duration is immediate then after (Rank of spell /
5) pulses, rounded up, the target may make a 3 ×

WP check each pulse to recover from fear, otherwise the fear lasts the duration of the spell. A further check may be required to recover from the
secondary effects of 76–90, 91–95 and 96–100.
Once recovered from, the spell has no further effect. The range of the spell only determines the
possible targets, so the target being more than this
distance from the Adept will not affect the duration.

Example
A group of adventurers are in the middle of
a forest. Rolf the Harrier gets struck by a Fear spell from
Nasty the Necro, and fails to resist. A roll of 45 indicates
panic. Rolf’s player thinks Rolf has 3 options:

3. Mass Fear (Necromancy, Wicca): has set range
and duration. Range is a sphere around the Adept.
Targets going out of range of the spell are no
longer affected. Each time they re-enter the area of
effect of the spell they make another magic resistance (and roll on the Fright Table if they fail).

Rolf’s player and the GM decide on percentages for the
options and one is determined. In this case Rolf flees. The
spell is Rank 7 so Rolf gets to attempt to snap out of the
fear each pulse after the first two. Six pulses after the spell
is cast, on his third try, Rolf succeeds his 1 × WP check
and may turn around and attack Nasty the Necro.

139

1. Run back to a clearing just left by the party.
2. Feeling no direction to be safe, Rolf clings to Hu’ug the
giant’s leg and trembles.
3. Being a “If I can’t see them they can’t see me” believer,
Rolf puts a sack over his head.

55 EXPERIENCE POINT COSTS –WEAPONS

55 Experience Point Costs
55.1 Weapons
Sword
Dagger
Main gauche
Short sword
Falchion
Scimitar
Tulwar
Rapier
Sabre
Broadsword
Estoc
Hand & a half
Claymore
Two-handed sword
Hafted weapons
Hand axe
Battle axe
Giant axe
Great axe
Crude club
War club
Giant club
Mace
Giant mace
War hammer
War pick
Flail
Morningstar
Mattock
Quarterstaff
Pole arms
Javelin
Spear
Giant spear
Pike
Lance
Halberd
Poleaxe
Trident
Glaive
Giant glaive
Missile weapons
Sling
Self bow
Short bow
Long bow
Giant bow
Composite bow
Crossbow
Heavy crossbow
Spear thrower
Blowgun
Thrown weapons
Throwing dart
Boomerang
Grenado
Entangling weapons
Net
Bola
Whip
Lasso
Special weapons
Rock
Cestus
Garotte
Sap
Shield
Unarmed
Unarmed

0
25
50
100
25
100
100
200
150
50
75
100
50
50
0
100
75
75
150
25
25
25
50
50
50
75
25
100
50
75
0
50
100
100
200
250
100
100
200
50
50
0
200
100
100
300
300
200
100
100
25
25
0
200
100
25
0
150
200
150
150
0
25
30
100
25
25
0
150

1
25
50
100
25
100
100
200
150
50
75
100
50
50
1
100
75
75
50
25
75
75
50
50
50
75
25
100
50
75
1
50
100
100
200
400
100
100
200
50
50
1
200
100
100
200
200
200
100
100
25
25
1
100
300
50
1
150
200
150
150
1
25
40
200
75
25
1
300

2
50
100
200
50
200
200
200
200
100
150
200
100
100
2
200
150
150
100
50
150
150
100
100
100
150
50
200
100
150
2
100
200
200
400
700
200
200
400
100
100
2
400
200
200
500
500
400
200
200
50
50
2
200
500
75
2
300
400
500
500
2
100
50
300
150
50
2
450

3
100
200
400
100
400
400
200
500
200
200
400
200
200
3
500
200
200
200

4
200
400
700
200
700
700
200
1000
400
500
500
400
400
4
1500
500
500
500

5
400
1100
1500
400
1500
1500
500
2000
700
1000
900
700
700
5

6
700
1500
3000
700
3000
3000
500
2000
1500
2000
1700
1500

7
1500
3000

8
3000
3000

9
4000
3000

10

1500
3000
3000
2000
2000

3000
3000
3000
4000

4000

3000

4000
2000
1800

4000

3000

6

7

8

9

10

1000
1000
1000

2000
2000
3000

4000
4000
5000

300
300
200
200
200
200
100
400
200
200
3
200
400
400
700
1000
400
400
800
200
200
3
700
400
400
1000
1000
700
400
400
100
100
3
500
1000
100
3
600
700
900
900
3
150
100
600
250
100
3
600

500
500
400
400
400
500
200
700
400
500
4
400
700
700
1500
1700
700
700
1400
200
200
4
1500
700
700
2000
2000
1500
800
800
200
200
4
1000
1200
150
4
1300
1500
1400
1400
4
200
200

700
700
700
700
700
1000
400
1800
700
1000
5
800
1800
1800
3000
3500
1500
1500
3000
200
200
5
3000
1500
1500
2000
2000
3000
1000
1000
400
400
5
2000
1500

2000
6
1400

4000
7
2000

4000
8
2000

3000
9
2000

10
3000

500
500
6
3000
3000
3000
2000
2000
3000

800
800
7
3000
3000
3000
2000
2000
3000

1500
1500
8
3000
3000
3000
3000
3000
3000

3000
3000
9

10

700
700
6
2000
1500

1500
1500
7
2000
1500

3000
3000
8
2000

4000
4000
9
2000

5000
5000
10
2000

5

6

7

8

9

10

2000
2000
2000
5
300
400

5000
3500
3500
6
500
700

4000

5000

5000

6000

7

8

9

10

1500

3000

6000

5
900

6
1500

7
3000

8
4000

9
5000

200
4
800

140

4000

10
4000

55 EXPERIENCE POINT COSTS – SKILLS & CHARACTERISTICS

55.2 Skills
0

1

2

3

4

5

6

7

8

9

10

extraB

§29

AdventuringA

0

125

250

375

500

625

750

875

1000

1125

1250

NA

§30

Alchemist

800

350

1200

2650

4350

6500

8650

11100

12750

14500

17000

NA

C

§31

Armourer

600

300

800

1600

3000

5500

6200

7300

8800

10800

14000

5000

§32

Artisan

250

100

150

350

700

950

1500

1850

2500

3200

4000

NA

§33

Assassin

600

250

750

1700

2900

4200

5750

7550

9500

11700

14100

NA

§34

Astrologer

400

150

500

1150

2050

3100

4400

5900

7500

9400

11500

NA

§35

Beast Master

600

250

750

1650

2800

4300

5600

7350

9300

11400

13750

5000

§36

CourtierC

250

100

200

500

950

1450

2050

2800

3600

6300

8000

1000

§37

Healer

1000

400

1600

3500

5800

8400

11400

14700

18500

22500

26750

NA

§38

Herbalist

800

350

1200

2650

4350

6500

8650

11100

12750

14500

17000

NA

§39

Languages

200

75

125

300

550

850

1350

1700

2250

2900

3500

NA

C

§40

Mechanician

600

250

650

1500

2600

3900

5300

7000

8850

10900

13000

2500

§41

Merchant

300

125

300

850

1400

2200

3400

4200

5300

6800

9500

4000

§42

Military Scientist

300

125

350

950

1500

2350

3100

4150

5400

6750

10000

3000

§43

Navigator

400

150

400

900

1550

2400

3350

4450

5750

7100

10500

NA

§44

Philosopher

1400

700

1400

2100

2800

3500

4200

4900

5600

6300

7000

NA

§45

Ranger

600

250

800

1650

2750

4100

5650

7350

9300

11400

13250

NA

C

§46

Spy

500

200

600

1400

2400

3600

5000

6600

8400

10400

12600

2500

§29.4

Stealth

0

500

1000

1500

2000

2500

3000

3500

4000

4500

5000

NA

§47

Thief

750

300

1050

2350

4000

5750

7900

10250

12900

14850

16000

NA

§48

Troubadour

250

100

200

500

1050

1450

2100

2800

3900

4600

7000

1000

§49

Warrior

600

250

750

1700

2900

4200

5750

7550

9500

11700

14100

NA

§50

WeaponsmithC

600

300

800

1600

3000

5500

6200

7300

8800

10800

14000

5000

Notes
A Adventuring skills includes Horsemanship, Climbing, Flying, and Swimming. See the above table for Stealth.
B Some skills have subskills which may be gained at any rank for the EP cost
given and 4 weeks. These costs are discounted by 25% if the character has
reached rank 8 in the given skill, or by 50% if they have reached rank 10. Subskills gained in this way are in addition to any gained by an increase in the rank
of the skill. See each skill for details.
C Depending on the character’s personal characteristics, they may pay more or
less EP to rise in Rank. See each skill for details.
D Knowledge (§28.2) is a skill which takes 4 weeks and 500ep to learn Rank 0.
It cannot be Ranked beyond Rank 0.

55.3 Characteristics
Stat

First Point

Extra Points

Fatigue

2500

2500

Endurance

5000

2500

Perception

1000

750

All others

5000

5000

141

56 WEAPON, SHIELDS & ARMOUR CHARTS

56 Weapons, Shields and Armour Charts
56.1 Weapons
Special Notes for Weapons
Weapons are normally wielded one-handed, and
the exceptions are noted with a (2) after the name
of the weapon. Some may be used either one or
two-handed, and these are noted with a (1–2).
When weapons of this type are wielded two
handed, increase their Damage Modifier by 1.
- indicates that a weapon has no Class for purposes
of Grievous Injuries; when a possible Grievous
Injury is rolled, only damage affecting Endurance
results.
V indicates that the characteristic is variable.
A When attacking a foe whose modified AG is
between 12 and 9 (inclusive) the weapon may be
used to attack twice in one pulse without penalty; if
the modified AG is 8 or less, the weapon may
attack three times in a pulse.
B Only giants can use giant weapons.
C A torch is not actually a weapon, but may be
used as such in emergencies. Also, brandishing a
burning torch in the face of an animal may cause it
to flee. Any animal whose WP is 10 or less may be
scared off if it fails a roll of 4 × WP. A successful
roll indicates the animal is not impressed. No Rank
may ever be achieved with a torch.
D The sap may only be used to knock out targets
wearing only leather, cloth or no armour. Used by
an assassin, any hit knocks out the target; for anyone else, any hit stuns and 4 or more points of
effective damage knocks out the target. This will
not work on targets larger than human size.
E The following weapons also function as thrown
weapons: dagger, hand axe, battle axe, giant axe,
crude club, war club, giant club, mace, war hammer, javelin, spear, giant spear, net, bola and rock.
F Up to three throwing darts may be thrown at one,
two, or three targets in one action with no penalty.
G A boomerang returns to the thrower if it does not
hit anything during its flight.
H A grenado is filled with any substance (manufactured by an alchemist) designed to burst into
flames on impact. These substances include greek
fire, methane, and anything else the GM will allow.
It bursts on landing (it need not be thrown at a
particular figure), and its effects are determined by
the substance contained within. If a “miss” is rolled
for the strike check, the GM should randomly
determine whether the grenado landed short, long,
left, or right of the target (or any combination
thereof).

Swords
DaggerA
Main gauche
Shortsword
Falchion
Scimitar
Tulwar
Rapier
Sabre
Broadsword
Estoc
Hand & a half (1-2)
Claymore (1-2)
Two-handed sword (2)
Hafted weapons
Hand axe
Battle axe (1-2)
Great axe (2)
Giant axeB
Crude club
War club
Giant club
TorchC
Mace
Giant maceB
War hammer
War pick (1-2)
Flail
Morningstar (1-2)
Mattock (2)
Quarterstaff (2)
Thrown weaponsE
Throwing dartF
BoomerangG
GrenadoH
Pole weapons
JavelinI
Spear (1-2)
Giant spear(1-2)B
Pike (2)J
LanceK
Halberd (2)
Poleaxe (2)
Trident (1-2)
Glaive (2)
Giant glaive (2)B
Missile weaponsL
Sling (2)
Self bow (2)
Short bow (2)
Long bow (2)M
Composite bow (2)
Giant bow (2)B
Crossbow (2)
Heavy crossbow (2)
Spear thrower (2)
Blowgun (2) U
Entangling weapons
NetN
BolaN
WhipN O
Lasso (2)
Special weapons
Rock
CestusP
Garotte (2)Q
SapD
Shield
Unarmed
UnarmedR

Wt
10oz
1
2
4
4
4
2
3
3
2
6
5
9
Wt
2
5
6
25
4
3
10
3
5
25
4
5
4
5
6
3
Wt
3oz
1
2
Wt
3
5
15
8
7
6
6
5
7
14
Wt
1
2
4
6
8
14
7
10
4
1
Wt
2
2
3
4
Wt
V
3
1
1
V
Wt
0

142

PS
7
8
10
12
11
13
11
14
15
15
17
16
22
PS
8
14
19
29
16
14
25
8
16
27
15
17
14
18
19
12
PS
9
11
9
PS
12
15
22
18
16
16
18
14
16
22
PS
7
10
14
16
17
25
14
20
11
7
PS
11
11
10
12
PS
5
12
12
9
10
PS

MD
10
15
12
11
15
15
18
15
15
17
16
13
14
MD
11
14
17
12
10
10
9
12
9
10
13
13
15
15
14
16
MD
15
15
15
MD
15
14
16
16
18
16
15
16
18
18
MD
15
15
15
15
15
17
14
14
14
16
MD
16
15
16
18
MD
10
14
15
11
12
MD

BC
40
45
45
50
50
50
45
60
55
45
60
50
55
BC
40
60
65
65
45
50
50
40
50
50
45
45
50
60
55
55
BC
40
40
40
BC
45
50
55
45
45
55
55
45
55
65
BC
40
45
45
55
55
55
60
60
50
30
BC
30
35
40
30
BC
30
35
30
40
40
BC
V

DM
+0
+1
+3
+2
+3
+4
+3
+3
+4
+5
+5
+4
+7
DM
+1
+4
+6
+10
+2
+2
+8
+1
+4
+7
+3
+4
+2
+4
+6
+2
DM
+0
+0
V
DM
+2
+3
+7
+5
+6
+3
+4
+2
+5
+9
DM
+1
+0
+2
+4
+4
+7
+3
+5
+2
-3
DM
-5
-3
-3
-4
DM
-1
-1
+3
+1
-2
DM
-4

Range
8
P
P
P
P
P
P
P
P
P
P
P
P
Range
8
6
P
6
6
7
9
P
5
8
6
P
P
P
P
P
Range
12
20
15
Range
12
6
12
P
P
P
P
5
P
P
Range
60
30
60
180
225
45
80
90
15
7
Range
5
10
P
6
Range
8
P
P
P
P
Range
P

Class
A
A
A
B
B
B
A
B
B
A
B
B
B
Class
B
B
B
B
C
C
C
C
C
C
C
C
C
C
C
C
Class
A
C
Class
A
A
A
A
A
B
B
A
B
B
Class
C
A
A
A
A
A
A
A
A
Class
Class
C
C
C
C
Class
C

Use
RMC
MC
M
M
M
M
M
M
M
M
M
M
M
Use
RMC
RM
M
RM
RM
RM
RM
M
RM
RM
RM
M
M
M
M
M
Use
R
R
R
Use
RM
RM
RM
M
M
M
M
RM
M
M
Use
R
R
R
R
R
R
R
R
R
R
Use
RM
R
M
RM
Use
RMC
MC
C
MC
M
Use
MC

Cost
10
20
40
35
60
65
35
40
50
65
85
80
100
Cost
15
20
30
50
3
5
10
1
15
40
15
20
15
20
18
3
Cost
1
2
V
Cost
4
10
20
15
4
15
20
8
15
30
Cost
1
20
20
25
80
80
15
20
5
3
Cost
4
5
6
5
Cost
15
3
2
V
Cost

Rk
9
10
6
8
8
8
10
7
6
9
7
7
5
Rk
4
7
7
7
2
5
5
5
5
5
5
5
5
5
9
Rk
10
7
4
Rk
10
5
5
5
5
5
5
5
9
9
Rk
8
8
8
8
8
8
5
5
10
10
Rk
4
6
10
6
Rk
6
9
3
3
4
Rk
10

56 WEAPON, SHIELDS & ARMOUR CHARTS
I A javelin functions as a thrown weapon unless it
is launched by a spear thrower, in which case the
spear thrower’s characteristics are used and it
functions as a missile weapon.

AccessoriesS

Qty

Wt

Cost

Shot

20

4

1

Use in sling

Dart

20

2

5

Use in blowgun

Arrows

20

2

10

Use in draw bows

Quarrels

20

7

15

Use in crossbows

Cranequin

1

3

10

Use with crossbowsT

Notes

J A pike may be used to melee attack any figure
within two hexes; its melee zone extends into what
would normally be the first hexes of the figure’s
ranged zone.
K A lance may be used only by a mounted figure.

56.2 Shields
Shield type

Wt

Def / Rank

MD loss

Cost

Buckler

3

Small round

5

2%

-

5

3%

-2

Large round

8

10

4%

-3

10

Kite

15

5%

-4

15

Tower

25

6%

-6

20

Main gauche†

1

2%

-

20

Weight The weight of the shield in pounds.
Defence / Rank The percentage by which the
figure’s defence is increased per Rank while the
shield is prepared (Rank 0 is counted as a
Rank†).
Manual Dexterity Loss The number of points
the figure’s MD is reduced by for all purposes
while that shield is prepared.
Cost The cost in Silver Pennies for a shield of
average workmanship.

When a shield is not prepared, it is considered
slung on the back of the figure carrying it. All
shields except the tower shield and main gauche
are constructed of wood and hides and do not
affect the flow of mana in regard to Adepts.
†The main gauche does not subtract its defence
from any fire attack, and cannot make a shield
rush attack. The main gauche functions both as a
weapon and a shield, and only one EP expenditure is used to rise in Rank in both. No defence
bonus is gained at Rank 0.

56.3 Armour Chart with Guild Prices
Armour Type

Wt

Prot

AG mod

Cost

Stealth

Cloth

1

1

0

30

+5

Heavy Furs

2

2

0

40

+5

Soft Leather

3

3

0

50

0

Leather

3

4

-1

50

0

Scale†

4

5

-3

600

-5

Full Scale†

6

5

-2

750

-10

Chainmail†

7

6

-2

1,200

-10

Partial Plate†

6

6

-2

1,500

-15

Full Plate†

8

7

-3

2,000

-20

Improved Plate†

7

8

-3

2,850

-20

Heavy Plate†

8

9

-3

3,500

-25

Jousting Armour†

9

10

-4

5,000

-30

Heavy Jousting Armour†

15

15

-8

12,000

-50

Mithril

2

10

-2

Quest

-10

Dragon Skin

4

‡

-1

Quest

0

†Cold Iron

‡As per Dragon - 3

Weight The number by which a figure’s size is
multiplied to find the weight of the armour in
pounds. Size number for the character races are:
Halfling (3), Dwarf (4), Elf (5), Orc (6), Human
(6), Hill Giant (9). For all others, the multiple is
their height in feet (round up). Females should
subtract 0.5 from the multiples.
Protection The number of Damage Points the
armour absorbs.
Agility Loss The number of points the figure’s
AG is reduced for all purposes when the armour
is worn. Does not include possible additional AG
loss for the weight of the armour.
Cost The cost in Silver Pennies for the armour.
Costs assumes average workmanship and mansized armour; larger or smaller armour should
cost proportionally more or less.

Stealth Adjustment The amount by which a
figure wearing that type of armour has their
stealth percentage adjusted.
Note Cloth armour is worn underneath all other
armours and its protection and weight are factored into those armours.
1. Silvered Armour has the same protection –
cost is + 30,000 sp. Permits magic at -10% to
Base Chance.
2. Truesilvered Armour has the same protection
– cost is + 180,000 sp. Permits magic at no modification to Base Chance.
3. Bronze Armour has 2 points less protection –
cost is the same. Permits magic at no modification to Base Chance.

143

L All missile weapons must be loaded before firing; this action is in addition to preparing the
weapon itself. A pass action must be taken in order
to load the sling, any draw bow, the spear thrower,
and the blowgun. Two consecutive pass actions
must be taken to load a crossbow (three if using a
cranequin).
M Longbows may not be used while mounted, nor
by small figures, including Dwarves & Halflings.
N An entangling weapon may be used as a garotte
in Close Combat.
O The whip may be used to entangle and do damage in the same pulse to the same target in melee
combat. In close combat, it functions as a garotte.
Once the target is entangled, the attacker may
choose to leave them entangled (thus letting go of
the whip), or disentangle the target themselves, and
retain possession of the weapon.
P Cesti are worn on the hands and need not be
prepared in order to be used.
Q The garotte is used to strangle the target and
may only be used against human-sized or smaller
victims. When used by a trained assassin, once a
successful hit has been scored, it will continue to
do damage every pulse from then on until the
victim is dead or the assassin has taken effective
damage from either the victim or an outside source.
If the victim’s PS is greater than the assassin’s the
GM may permit them to attempt to break the hold,
similar to the attempt to restrain. If the attempt is
successful, the hold is broken and the assassin will
have to make another successful strike check to
continue the strangulation. Some types of plate
armour may, at the GM’s discretion, prevent the
successful use of this weapon due to protection
around the neck area. A non-assassin has to roll a
strike check every pulse to see if any damage can
be done.
R Unarmed has a base chance of 2 × AG + PS over
15. The damage modifier is -4 (+ 1 for every 3 full
points of PS over 15).
S All shot, darts, arrows, and quarrels come in
appropriate pouches or quivers of 20; the weight
and cost of the pouch or quiver is not included in
the information given for the accessory.
T A cranequin is used to cock crossbows; it requires a PS of 11 and two free hands.
U If a blowgun dart does any effective damage, the
damage is not scored against their victim but rather
they suffer the effects of the substance which coats
the tip (poison, for instance; see Alchemist §30).

57 COMBAT TABLES

57 Combat Tables
57.1 Strike Chance Modifiers Summary

57.2 Combat Equation Summary

Close Combat Modifiers

Engaged
Initiative

PC + modified AG + weapon
Rank + warrior Rank.

Unengaged
Initiative

D10 + (PC + 2 × Military Scientist).

Strike Chance
with ranked
weapon

Weapon Base Chance + attacker’s modified Manual Dexterity + (4 × Rank with weapon)
+ Magic.

each point attacker’s PS is greater than target’s PS
+20 target has no Fatigue
+1

+20 target is stunned
each point target’s PS is greater than attacker’s PS
-20 attacker has no Fatigue
-1

Melee Combat Modifiers
+10 target has no Fatigue

Modified Strike Strike Chance + Modifiers
(§57.1) opponent’s Defence.
Chance
Defence

Modified AG + Shield defence
+ Modifiers (§57.1) + Magic.

Repulse a
charge attack

D10 ≥ prepared weapon Rank.

+20 target is kneeling or prone
+20 attacker is charging with pole weapon or
shield
+30 target is being attacked through rear hex

Withdraw from (D10 + total friendly Physical
Strength - total hostile Physical
close combat
Strength) ≥ 10.

-4

Strike chance to 40% + attacker’s modified
Manual Dexterity + (4 × Rank
trip
with weapon) - opponent’s
Defence (Damage: D10).

+10 target is being attacked though a flank hex
+15 target is stunned

each Rank the target has with prepared
weapon if evading
-10 target is evading
-10 attacker has no Fatigue
-15 attacker is charging with non-pole weapon
-20 attack is Melee attacking while withdrawing

40% + attacker’s modified MD
+ (4 × Rank with shield) - opponent’s Defence (Damage: [D
- 2]).

Disarm

-20 to strike chance.

57.4 Special Damage Chart

Entangle

Same as normal strike chance
with weapon (Damage: [D - 4]).

Success

Triple

Double

Knockout

Must roll under (15% × Modified Strike Chance).

01 - 09

-

01

10 - 16

01

01 - 02

Fumble

Reduce Initiative by 10. Roll
under Initiative on D100 or roll
on the Fumble Table (§52.4).

17 - 23

01

01 - 03

24 - 29

01

01 - 04

30 - 36

01 - 02

01 - 05

MD + Rank on D100. Chance
doubled if two handed.

37 - 43

01 - 02

01 - 06

44 - 49

01 - 02

01 - 07

50 - 56

01 - 03

01 - 08

57 - 63

01 - 03

01 - 09

3 or less: Successful parry;
evader must pass next action.

64 - 69

01 - 03

01 - 10

70 - 76

01 - 04

01 - 11

4–7: Disarm, 1 EN damage.

77 - 83

01 - 04

01 - 12

8 or greater: Disarm plus a
riposte; evader may melee
attack, 1 EN damage.

84 - 89

01 - 04

01 - 13

90 - 96

01 - 05

01 - 14

97 - 103

01 - 05

01 - 15

104 - 109

01 - 05

01 - 16

110 - 116

01 - 06

01 - 17

117 - 123

01 - 06

01 - 18

124 - 129

01 - 06

01 - 19

130+

01 - 07

01 - 20

Ranged Combat Modifiers
+10 target being attacked through a flank hex
+10 target is stunned
+10 attacker is kneeling

Avoid Disarm

+20 target is being attacked through rear hex
+20 aim (also chances of Endurance or Specific
Grievous damage are increased to 10% and
20%, see §6.9)
-3 every hex through which a thrown weapon
travels
-3 each 5 hexes (or fraction) after the first 5
through which a missile travels
-15 snapshoot
-5

target is currently moving

-10 target is kneeling or prone

Unengaged figures
• Move up to full TMR
• Step and Melee Attack
• Charge
• Charge with Polearm or Shield
• Charge and Close
• Evade
• Retreat
• Pass
• Cast
• Throw
• Fire
• Recover from Stun

Shield Rush

-40 pitch blackness
-50 target is invisible or undetectable

Close figures
• Grapple
• Withdraw
• Pass
• Recover from Stun

3 × ((PS+AG of attacker) (PS+AG of defender)).

-10 starry night or shadowy interior
-30 cave or unlit interior

Engaged figures
• Melee Attack
• Close and Grapple
• Evade
• Offensive Withdraw
• Defensive Withdraw
• Flee
• Pass
• Cast
• Throw
• Recover from Stun

Restrain

Visibility Condition Modifiers

-20 cloudy night

57.3 Action Summary

Stun recovery

(2 × WP) + current Fatigue.

Parry result

D10 + evader’s Rank - attacker’s Rank.

-20 target is evading
-20 target occupies a sheltered hex
Miscellaneous Modifiers
-20 striking weapon held in attacker’s secondary
hand
-10 multiple strike; attack with weapon in primary hand
-30 multiple strike; attack with weapon in secondary hand
-10 multiple strike; attack with each weapon if
ambidextrous
-20 multiple strike with B class two-handed
weapon
-20 attacking an airborne figure
-15 airborne figure attacking ground figure
-10 airborne figure attacking another avian
Each modifier is added to the Strike Chance of the
attacker in each instance where it applies; all modifications are cumulative.

144

58 MISCELLANEOUS TABLES

58 Miscellaneous Tables
58.1 Fatigue, Encumbrance and Movement Charts
PS

Weight of Load (lbs)

3-5

Max

0

0

5

14

21

30

37

45

50

6-8

0

5

12

17

25

40

55

67

75

9-12

5

12

17

25

40

60

75

90

100

13-17

12

17

25

40

60

80

95

112

125

18-20

17

25

35

50

75

105

125

140

150

21-23

25

40

55

70

100

140

165

185

200

24-27

35

50

65

85

120

160

185

202

225

28-32

45

65

85

105

140

180

205

230

250

33-36

55

80

110

140

180

220

245

262

275

37-40

65

85

135

170

207

247

280

307

325

Light

0

0

0

1/2

1/2

1

2

3

5

Medium

0

0

1/2

1/2

1

1

3

4

6

Hard

1/2

1/2

1

1

2

3

5

6

8

Strenuous

2

2

3

3

4

5

6

7

9

0

1

2

3

5

7

9

10

12

Fatigue loss from Exercise

Agility Loss in Combat
Loss

Weight of Load (lbs) The mximum weight, in pounds, that a character can
carry (excluding clothing worn), to fall into that category. Note: A mount can
carry weight for a character while they are riding.

Fatigue loss from Exercise Tiredness Fatigue loss per hour of encumbered
exercise, see §4.4.
Agility Points Lost The temporary Agility Point loss suffered by a character
toting the given weight in combat. Use the procedure in rule §4.4 to use this
chart.

Max The maximum load, in pounds, that a character can carry for a sustained
period of time.

58.3 Overland Movement Rate

58.2 Tactical Movement Rate

Rate of Exercise

Modified Agility

TMR

<1

0

Terrain

Light Medium Heavy Strenuous

1–2

1

Cavern

5/-

3–4

2

Field

15/15 25/25

30/40* 35/50*

5–8

3

Marsh

-/-

10/10* 15/15*

9 – 12

4

Plain

15/15 25/25

30/40* 40/50*

13 – 17

5

Rough

10/5

15/10

20/15* 25/-

18 – 21

6

Waste

10/5

15/10

20/10* -/-

22 – 25

7

26 – 27

8

> 27

†

Woods
10/5 15/10
20/15* 25/The number before the slash indicates movement in miles per day on foot; the number following the slash
indicates mounted movement (assuming horses). Rates for other animal types must be adjusted by the GM.
The day assumes a total of 8 hours marching. Effects of adverse weather must be adjudicated by the GM.
Any paths or roads negative the effect of other terrain, and the Plain movement rates are used.

† TMR = 9 + 1 for every two points of AG over
28, e.g. AG 32 gives 11 TMR

10/5/5

15/-

20/-

(-): Movement type impossible at this exercise rate.
* In these exercise rate categories, horses’ maximum rates will deteriorate 33% per day. They can travel at
these rates for approximately 4 consecutive days and then they will die.

145

VERSION HISTORY

Version History
History of Edition 2.0
September 2014
Blowgun damage added and note changed to similar to venomous snakes
Counterspell of other college clarified as being
special knowledge
Spy stealth bonus changed back to 2%/rank to
match stealth skill
May 2014
Jono's version of Evil Eye (E&E G-9).
Rename Wicca Evil Eye to Hex.
Remove requirement to cast 5 times per when
ranking.
Remove the 100 charge rule from investment.
Warrior 3.0 basic and special abilities.
Rewording of Spirit Vision (Rune T-2) re visibility
restriction.
Change the Background Experience in Character
Generation from 250 experience to 2,500.
Add decreasing Racial Modifiers to Character Gen:
Race.
Interrupted Ranking: replace completed with resumed.
Reworded E&E and Illusion invisibility.
Moved damage by burning to end of fire college.
Add adventuring skill combo summary.
In Heritage change Baronies to Western Kingdom
and Cazarla.
Change to Armourer to V1.3.
Removed base chance from wizardsight.
History of Edition 1.8
January 2013, EPub Edition. Spelling and grammar
edits. College intro hierarchy rearrangement in
Mind, Namer, Fire, Necro, Celestial. Removed first
level heading as inconsistent and had orphans .
Removed cytogenesis. Changed non-tactile empathy to ranged empathy. Changed collegiate to
college. Checked semi-colons.
Moved sap from hafted weapons to special weapons in 56.1.
Changed wording of spirit vision removing sentence "This vision is blocked by material objects
(even if invisible) and magical darkness of Rank
20.", adding "as though they were normally visible.
Fixed Detect Enchantment (T-2) binding college
"Range: 1. 30 feet (+ 5 / Rank) 2. Touch "Base
Chance: 1. PC + 3% / Rank 2. PC + 5% / Rank" "If
the Adept is in contact with the target then the base
chance of this talent is higher."
The Ice elemental has AG 15-2. Assumed that it
should be 15-20.
Removed non-magical from these two lines as
otherwise it implies that the it doesn't apply to the
quasi-magical abilities mentioned just below them.
“A character may attempt to employ a *nonmagical* skill any number of times during a day.”
“The use of a *non-magical* skill is rarely automatic.”
6.10 Fumbles If the weapon is not Magical, or
made of cold iron, or a Bow, or a Crossbow, increase this chance of fumbling by: any silver or
truesilver alloy of iron 1% etc changed to: This
chance of fumbling is increased if the weapon is
made of a material other than cold iron, as listed
below, unless it is magical, or a Bow or Crossbow.
Removed reference in Ice College spell Refrigeration (G-5) to fire college temperature alteration
spell.
Corrected Wiccan Damnum Minatum reference to
Ice College from Air College.
History of Edition 1.7
July 11, 2010 Large Print Edition.
June 10, 2010 Enhance Indexes. Move Combat
section to before Magic. Reorder character-related
sections. Move change log to end of Rulebook.
Add ToC to Combat, Magic.
June 9, 2010 Miscellaneous edits for greater clarity
and readability. Allow Courtier and Troubadour
specialisation when taking subskill. Adjust bonus
for rank in Stone Golems. Re-order Binder Golem

info. Add back clothing exemption to Encumbrance.
June 8, 2010 Changes to Healer Regeneration,
Graft Skin to Cure Burns, adjust and correct summary table. Update Languages. Waters of Healing
neutralise venom if present. More info gained from
Detect Poisons. Standardise to ‘Shapechanger’.
Masterwork exceptions for Adventuring and Language Skills. Move Knowledge Skill from Adventuring Skills to Skill Intro. Add Salve details to
Alchemist, and time to make Medicines/Antidotes.
Clarify Combat Equation Summary. Add Compose
Music to Courtier.
June 7, 2010 Repeat lost June 1-7, 2006 miscellaneous edits.
History of Edition 1.6
June 7, 2006 Miscellaneous edits for clarity and
readability.
May 25, 2006 Move Conception from Character
Generation to Health and Fitness. Added 6 month
limit and Subskills to Ranking. Permanency
changes to Binder and Illusion Colleges. Weather
table rationalised. Rune v2.2 in playtest and available for new characters. Modified Sub-skill rules
for all applicable skills. Remove Ropes cost and
Orienteering from Adventuring skills. Move Healer
potion costs to Alchemist. Cost formula clarifications in Armourer. Added new Alphabet and Languages to Languages. Removed Spy discount from
Thief. Cost formula changes in Weaponsmith.
Added EP costs for Lasso and removed Katana.
Rationalised Skill notes.
January 11, 2006 Add Spy 2.0
December 6, 2005 Include Subskill costs in the EP
table.
History of Edition 1.5
June 3, 2004 Miscellaneous edits for greater clarity
and readability. Acknowledge contributors for the
August 13,
2001 Mind College revision. Remove partial contributors list from the cover.
May 26, 2004 College Magic: The Investment
Ritual (Ver 1.2). New Version of the E&E Ritual
of Greater Enchantment. Geas changed to max
rank 20 and Full Geas at rank 15. Mind College
spell of Undetectability removed. Necromancy
spell of Necrosis: limitation of only affecting living
restored. Remove Probation tags from the colleges
of Binder, Namer, Mind, and Fire. Add Shields to
Weaponsmith.
History of Edition 1.4
December 19, 2001 All files revised to version
5.10 in RCS.
September 28, 2001 Ranger version 2.1 added.
This is a complete replacement for the current
ranger. Thanks to Rosemary Mansfield for this.
September 17, 2001 More changes to stun in combat to fix editor’s misunderstanding about the free
recovery at the end the pulse. Fix typo in Namer
probation paragraph and copy it to the Mind College.
September 3, 2001 Change wording in restrictions
on magic to enforce the limitation of cold iron on
casting. Fix wording in Unarmed combat for kick
so that the wording is now Rank 3 or above. Fix
reference to DA in Namer. Fix wording under
Encumbrance in Adventure. Try to fix Recover
from Stun so that figures get a Recover from Stun
action at the end of the pulse they were stunned in.
Text of Mind Speech tightened up. The Ranking of
Names modified to reflect new Namer. The section
on Namer Ranking has been modified accordingly.
The cost of a trainer for Ranking a weapon to Rank
0 has been included (10sp).

146

The sections Auras and Names merged into a new
section “Auras & Names”. Names wasn’t really
significant to have its own section and adding it
threw out all the section numbering. Text of Scry
Shield tightened up thanks to Terry Spencer. Minor
corrections of Mind by Ian Wood.
August 30, 2001 Minor corrections to combat
suggested by Errol Cavit. Order credits by alphabetical order and add Errol Cavit to it. Celestial
Light and Dark aspect rewritten by Errol Cavit.
This had previously been munged by Ross Alexander from the GM’s Guide. Fix logic error in Name
ranking in Namer spotted by Terry Spencer and
fixed by Martin Dickson. Resistance on Wall of
Darkness corrected from None to Passive.
August 13, 2001 The Mind College revision 1.6
done by Struan Judd, Jacqui Smith and Ian Wood.
Detect Aura rewritten and a new section on True
Names added. The Counterspell subsection of
general magic rewritten with respect to the changes
to Namer. The College of Naming Incantations
completely rewritten by Martin Dickson and is
version 2.0.
July 11, 2001 Change EM of Fireball from 500 to
550. Check duration of Weapon of Flames is 5
minutes + 1 / Rank. Add duration to Speak to Fire
Creatures of 20 minutes + 10 / Rank. Duration of
Alchemist’s potion of Talents is now based on
Rank of Talent, not Rank of Alchemist. Change
text of reference of Health and Fitness in Healer.
Significant changes to Recovery from Stun introduced into Combat. Add Recover from Stun to
combat actions table.
June 13, 2001 Fix typo in history. Fix problem
with section headings in character generation and
conception. Add note to Aspect about Celestial
college affecting light/dark aspect.
History of Edition 1.3
August 31, 2000 Looking at Fire, the EM of Fireball is 500, which is what was agreed by Paul
Schmidt just be the final release. Duration of
Weapon of Flames corrected to 5 minutes + 1 /
Rank. Duration of potioned talents changed from
Rank of alchemist to Rank of talent.
June 8, 2000 Final checkin for version 1.3. Last
minute change to EM of Fireball from 350 to 550
as a preventive measure. Requested by Jim Arona
and put in by the editor on the principle that it is
easier to refund overspend EP then to increase the
EM of a spell and try to sort out the deficit.
June 6, 2000 Change warrior initiative to +1/Rank
(except Rank 0). Healer and adventure retypeset.
Minor magic replaced with Cantrips. All weapon
spells (except Runeweapon) to have duration 5
minutes + 1 / Rank. Weapons of Flames toned
down.
May 23, 2000 Lots of minor corrections. Conception added to race descriptions.
May 19, 2000 Notes on grievous injury moved into
adventure under the new subsection of Health and
Fitness. The addendum of Healer has been merged
into the text of healer where possible. The notes
about the Guild healing services have been removed and should be in the Players Guide.
March 29, 2000 Illusion Talent of Enhanced Vision
changed (Andrew Withy). The description for
Wizard Sight changed (Andrew Withy). Range and
EM of Witchcraft Hellfire changed to match new
Fire college.
March 2, 2000 Language bonus for Namers
changed to fit new language document. Celestial
Witchsight description changed. E&E Witchsight
renamed to Wizard Sight. Fire College rewritten
(Paul Schmidt) and updated to version 2.0 and
probationary notice added. Wiccan Hellfire and
Instilling Flight changed to match Fire and Binder
respectively. Expert changed to Knowledge in EP

VERSION HISTORY
table notes. Symbols of power for curse removal
are now portable. Version 2.0 removed from title
of Explanation of Characteristics.
February 29, 2000 Languages skill updated to
version 2.0. Change languages in ranking section
of adventuring. The Talent of Speaking to Creatures of Dark/Light, Bardic College and Philosopher changed to fit in with new Languages document. Healers can now use limited abilities on nonsentient animals. Runeweapons made from yew are
to be treated as natural poisons. Mechanician
changed to version 2.1 and spell containment rewritten (Martin Dickson).
The Bardic spell Great Shout replaced with a new
spell Shout of Thunder. Bardic and Ice now made
full colleges with Binder still probationary.
October 20, 1999 Add new fumble rules to combat
and fumble table.
October 18, 1999 Fix rules where the term “round”
appears and should be pulse. Changes from Kelsie.
October 8, 1999 Delete Spell of Echosense and
cascade reference numbers down.
October 7, 1999 Change to ritual description in
general magic. Rituals now backfire on BC + 30%
and can double and triple effect. The description to
Ritual of Water Elemental (R-1) and Ritual of
Summoning Creatures of Light and Dark (Q-2)
modified. Add note that Purification cannot backfire. Add note of backfire of Investment ritual and
add note that Remove Curse does not backfire
normally. Probation note removed from Bardic and
Ice colleges. Backfire description on Ritual of
Resounding Instrument (R-2) changed. Binding
and Animating Ritual of Divination (R-3) changed
so it no longer backfires. Ritual of Fire Elemental
(R-1) description changed. Description of Ritual of
Illusory Fog (Q-2) changed. Active Talents now
require a pass action rather than one pulse to use.
August 18, 1999 Banishing demons summoned by
a Greater Summoner using Special Counterspell
can now only be done by the Adept. Counterspelling multiple instances of a spell clarified.
August 2, 1999 Conjuring the Controlling Air
Element (R-2) modified.
History of Edition 1.2
June 4, 1998 Various editing changes before final
print.
May 11, 1998 Add aquatic adventuring changes
from Keith Smith. Fix e.g. and i.e. problems.
Change descriptions of Greater Enchantment,
Sleep, Purification, Strength of Stone and Damnum
Minatum. Herbalist stat increase potions changed.
Namer encounter method removed and generic
names now teachable.
May 7, 1998 Alter navigator find landmark back to
DQII original rule (Michael Parkinson). Change
Sinking Doom, Whitefire and Incinerate to no
longer be irresurrectable. (Stephen Martin).
May 6, 1998 New combat section by Andrew
Withy.
May 5, 1998 New character generation from
Rosemary Mansfield. Major typographical changes
in magic.
April 15, 1998 Artisan and Mechanician completely rewritten (Martin Dickson).
March 30, 1998 Expert Knowledge and Supervisor
added to skills (Martin Dickson). Courtier, Philosopher, Military Scientist and Troubadour completely rewritten (Martin Dickson).
September 3, 1997 Add minor typographical
changes to Ice, Water and corrections from Keith
Smith.
June 23, 1997 Typesetting changes to force Binder
and Ice to put in an empty page if end on odd page.
History of Edition 1.1
6 June, 1997 Final print for 1997 rulebook.

5 June, 1997 Witchcraft control weather modified.
Initial stat generation modified so player can
choose 90 points rather than rolling. Bardic 1.1
added (Jacqui Smith and Martin Dickson).
22 May, 1997 Parts of Stephen Martin’s document
on eating, recovery and infection added. Infection
removed from combat. Falling removed from Thief
and put into Adventure. Binder 1.1 from Stephen
Martin added.
15 May, 1997 Aquatic Affinity talent added to
Water (Keith Smith). Paragraph about Bardic and
Ice being probationary until June 1999. Skills
imply knowledge of the subject area added (Martin
Dickson). Skills do not necessarily imply traits
added (Ian Wood).
Notes on blood agents moved from infection in
combat. Infection removed from combat.
7 May, 1997 Walls of Darkness and Walls of Starlight changed so that BC/EM switched for Solar/Star (Andrew Withy). Version changed to 1.3.
Necro and Witchcraft Darkness modified to Celestial Darkness. Ice 1.5 from Carl Reynolds introduced (May Gods meeting).
29 April, 1997 Numerous minor changes. Damage
on heavy crossbows changed from +4 to +5.
22 April, 1997 Weight of invested items changed
to minimum of one ounce (April Gods meeting).
9 April, 1997 Weight table modified so that values
represent maximums rather than median values.
This should make it simpler to use. Mages now
cannot cast while prone. Gilding removed from
restrictions on magic and cold iron.
6 April, 1997 Ice College 1.3 added by Carl Reynolds 12/1/96. Bardic 1.0 by Jacqui Smith added.
April 1, 1997 Typo in Waters of Vision and Crystal
of Vision fixed.
March 20, 1997 Notes on shaped items added to
introduction to magic.
March 19, 1997 Monsters introduction and aspect
moved to GM’s Guide.
March 14, 1997 DQ Swimming Skill Ver 2.3 by
Keith Smith added. Flying added and EP added to
EP table (as 125 / Rank). Notes on Ranking added
to adventuring section.
March 13, 1997 Extended rituals and notes on
possessions moved from Binder to general magic.
Counterspells made wardable and trappable. Storages on Namer spells fixed.
February 28, 1997 Guild banking removed from
adventure section and appending to player’s guide.
February 25, 1997 Spelling mistake in combat
summary fixed. Invisibility added to modifiers (50). Water fixed by Clare West, with new version
(1.3).
February 11, 1997 Giants changed to weight multiple 9 (to be confirmed). Notes of other creatures
added. Minor fixes to Illusion 1.4.
February 7, 1997 Change maximum Rank of Silent
Tongue to six.
January 15, 1997 Spelling mistakes corrected in
Namer Name List, Light and Dark Aspect and
Minor Magics. Additional indices added to Introduction to Magic.
December 17, 1996 New version of the Investment
ritual (v1.0) added. Extra indices added to college
and non college ritual section.
November 5, 1996 Awe table appended to fright
table. Fright from low PB now minimum 1 × WP.
November 1, 1996 Additional indexes added and
notes on orienteering changed in the skills cost
tables. Ordering of tables changed. Daylight time
and holiday times moved to Player’s Guide.
October 8, 1996 Map reading added to ranger. Map
reading and drawing maps changed in navigator.

147

Orienteering removed and artisan cartographer
added instead (Andrew Withy).
October 1, 1996 Tidy up files in general. Targeting
in magic changed so that an Adept can attempt to
cast at a target that is potentially out of range in the
hope of a multiple effect. Notes on magical storage
(not approved at this time) added to magic. Distract
removed from minor magic.
September 26, 1996 Greater Summonings altered.
The descriptions of imps, devils, succubi and incubi removed and put into the monster manual,
while the descriptions of the special rituals (summoning Dukes, Princes, Presidents, Earls, Marquis
and Kings) added. Half devils also added to monster manual, under summonables.
September 16, 1996 Add two new combat spells to
Air (Jon McSpadden), and change version to 2.1.
Fix resistance in Damnum Magnatum. Add notes
on magical storage to introduction to magic. Add
note that learning a college for the first time takes
6500 ep and six months. Dragon Flames and Necrosis now actively resistible.
September 12, 1996 Changes to Celestial Lighting
modifiers and the spells of Light and Darkness
added. Version updated to 1.2.
August 12, 1996 Add new bow rules. Missile
weapons before unarmed combat in combat section
and tables updated.
August 9, 1996 Change the description of orcs and
their racial multiplier to 1.1. Change hill giants size
multiplier to 11 (rather than 15). This has to be
verified. Be able to target outside normal range in
the hope of a double or triple.
July 11, 1996 Add Monsters section introduction
and notes on aspects.
March 27, 1996 Add Name list of Namer. Add
holidays and sunrise / sunset table. Add combat
equation summary to tables.
March 25, 1996 Rework coinage table to make
clearer. Make first reprint.
February 22, 1996 Change weight of coins and
remove personal names from colleges.
February 17, 1996 Parameters tweaked for formatting, and tables moved to back of book, after skills.
January 12, 1996 Initial release for players and
GMs. Release document does not contain combat.
January 5, 1996 E & E and Mind revised by Ross
Alexander and Brent Jackson.
October 14, 1995 Air rewritten by Jono Bean, Carl
Reynolds, Phil Judd and Rosemary Mansfield.
Version 2.0 released.
October 14, 1995 Illusion rewritten by Andrew
Withy to version 1.3.
October 1, 1995 General magic revised by Ross
Alexander, Andrew Withy and Brent Jackson.
June 8, 1995 Water revised by Clare West to version 1.2.
June 6, 1995 First draft is released to GMs for
comments and corrections.
June 4, 1995 Earth revised by Ross Alexander to
version 1.2.
\end{document}
