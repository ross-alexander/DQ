\chapter*{Monsters}

\section{Introduction}

\begin{multicols}{3}

The player characters represent only an infinitesimal fraction of the
inhabitants of the DragonQuest world.  The GM is responsible for
playing the part of those inhabitants that the characters meet during
their adventures. These inhabitants will be of roughly two types:
non-player characters (NPC's) and monsters.

Non-player characters are those inhabitants who are of races or
species from which a player character could also come or which are
closely-related to those races.  Monsters consist of those inhabitants
who come from races or species from which a player character could not
come.  These two classes are further broken down into other categories
depending upon the element in which the Players are most likely to
encounter them and whether or not they are common or fantastical.
Common monsters are those that dwell throughout the DragonQuest world
(as interpreted by the GM) while fantastical monsters are those rare
species which are highly magical and will tend to be found only in
isolated areas (especially areas that are mana-rich).

The GM pre-generates some monsters and NPC's prior to play so that they
may be brought into play whenever the players' characters arrive at
their destination or otherwise stumble upon them.  They need not keep
detailed records on all monsters.  Instead, they may write the most
important information concerning a monster on a 3' \x 5' index card,
with a master list of all monsters kept on a single sheet.  The GM may
wish to save time and effort by using the same numbers for all
NPC's/monsters of the same race or species found in the same place,
possibly varied for one or two characteristics if additional flavour is
desired.  The GM creates these characters and monsters by choosing (or
randomly generating) a number which falls within the parameters given
for each type of character or monster under the monster descriptions
in this Section.

Alternatively, the GM may wish to keep index cards for various
monsters, and pull one at random whenever the characters are due to
encounter an NPC/monster.  The GM may wish to present his own monsters
and NPC's to characters whenever they randomly encounter wandering
monsters or NPC's (those not placed in advance) or he may wish to use
the mechanic provided in this rule section.

\subsection{Encountering Monsters and Non-player Characters}

The frequency with which characters will encounter monsters and NPC's
will be determined by the GM.  There are two types of encounters the
players' characters may have: encounters with prepositioned monsters
and NPC's (usually in their lair or dwelling) and encounters with
wandering monsters and NPC's in a more or less random pattern.  GMs
may choose to use their own system for determining when and how the
latter are encountered or they may choose to use the following system:
The GM chooses the ``Danger Level'' of the area through which the
players are adventuring.  This level determines how frequently the GM
must check to see if the characters encounter a random (not previously
emplaced) monster or NPC, the Base Chance of encountering anything
(dependent in part on the terrain), and the number which is added to
the dice roll to determined the type of encounter the characters have.
The GM may wish to vary slightly the regularity with which they make
Encounter Checks so as to keep the players from anticipating danger
too easily.

\subsection{Reactions to Encounters}

Unless the GM has established a reaction for the NPC/monster the
characters have encountered, they consult the Reaction Table, rolling
D100 and adding to or subtracting from the result whatever number they
believe appropriate to the situation in addition to those modifiers
listed on the Humans column of the Encounter Table, where appropriate.
The GM determines the modifier before rolling the dice.  It should
seldom exceed 30.  The GM may add negative numbers to the dice roll.
The modified dice roll number indicates the reaction of the monster(s)
or NPC(S) to the encounter as follows:

\textbf{Dice}\hspace{5.0em}\textbf{Reaction}
\begin{description}

\item[01--10]
Enraged: Immediately attacks party.

\item[11--20]
Belligerent: Immediately attacks unless somehow mollified.

\item[21--30]
Wary: Inclined to attack, but does not immediately charge.

\item[31--40]
Unfriendly: Willing to communicate on a limited basis, but will not
cooperate and may attack if patience is tried too severely.

\item[41--60] Neutral: Willing to communicate or to allow the party to
pass by without hindrance.  Has no positive or negative feelings
about the party.

\item[61--75] Pleasant: Willing to communicate, including in his
conversation useful hints about the area, but still intent upon his
own business.

\item[76--85] Friendly: Willing to communicate and provide minor
assistance (such as providing temporary lodging).

\item[86--95] Charmed: Willing to assist the party in any way which
does not imperil the NPC/monster's own interests.  He may even be
talked into joining the party temporarily.

\item[96--100] Enraptured: Willing to join the party immediately upon
being asked.  Will totally identify with the party and its
interests even to his own peril.
\end{description}

The nature and degree of any modification will depend upon the race or
species of the monster or NPC encountered, on the manner in which the
characters habitually treat entities they encounter, and on such
unpredictable details as whether the monster currently has its young
in tow and is thus primarily concerned with their welfare. Once the
initial reaction has been determined, the ensuing interaction of the
characters with the monster or NPC will depend upon the actual
interaction of the players and the GM, as modified by their respective
perceptions of the prejudices, perceptions, and characteristics of
their characters.

\subsubsection{The Physical Beauty of a Monster (or lack thereof) may
cause characters to react in unpredictable ways.}

Whenever characters encounter a monster whose Physical Beauty is less
than 6, they must make a Willpower check of (4 \x WP). If they roll
above this result, they must then roll on the Fright Table (see
\S\ref{tables:fright}), and apply any results before they take any
other action.  If affected then, receive another Willpower check every
second Pulse until they recover. Until that time, they will act as the
result on the Fright Table indicates.

\textbf{Note:} The relative Physical Beauty, of monsters will in
part determine character interaction with them and will also determine
in part the interaction of a party of characters accompanied by such a
monster with other random encountered NPC's or monsters.

\subsection{How to read the Monster Descriptions}

The actual Monsters sections list the various types of fauna which
may be encountered in the DragonQuest World.  The sections describe
one type of fauna and provides detailed information on some specific
representative examples of that type.  These sample creatures are
discussed in detail according to the format given below.

\begin{description}
\item[Name] The name of the monster (or NPC type, hereafter called
simply ``monsters'').

\item[Natural Habitat] The environment(s) in which the monster is most
likely to be found, including subclasses of the 10 basic terrain types
discussed in \S\ref{ranger}.

\item[Frequency of Appearance] There are 4 designations given under
this heading, each representing the relative rarity of the monster as
a Guide to the GM in placing them in the adventure. In ascending order
of rarity, they are: Common, Uncommon, Rare, Very Rare.

\item[Number]
The average number of specimens of the monster which will be found
together in one place, usually expressed as a span of numbers.  In
some cases, this span will be followed by a single number which
indicates that this is the number most frequently found together.

\item[Description] A description of the monster as perceived by human
senses.

\item[Talents, Skills, and Magic] Includes a list and description of
all the talents possessed by the monster as well as any skills
mastered and whether the monster possesses any magical talents or is
an Adept of a College of magic.

\item[Movement Rates] A list of the Flying, Swimming, Running,
Climbing, Crawling, and Tunnelling speeds of the monster.  These are
given in yards (usually hundreds) per minute.  These numbers are used
primarily in the Adventure Sequence for purposes of establishing chase
speeds.  The Movement Rate of humanoids is Running: 250.

\item[PS] Physical Strength.
\item[MD] Manual Dexterity.
\item[AG] Agility.
\item[MA] Magical Aptitude.
\item[EN] Endurance.
\item[FT] Fatigue.
\item[WP] Willpower.
\item[PC] Perception.
\item[PB] Physical Beauty.
\item[TMR] Tactical Movement Rate, TMR's are listed in the same order
as t1hey are listed in movement Rates.  A monster's TMR is equal to
its Movement Rate divided by 50.
\end{description}
These characteristics function in the same manner as the
characteristics of player characters except for Physical Beauty, which
measures the relative emotional response (in ascending order of
approval from 1) of player characters to the physical appearance of
the monster (but not NPC).  These characteristics are given as a span
of number in most cases.  The GM may choose to pick a number from the
span or he may randomly generate a modifier to the lowest number in
the span (which serves as a base).

\begin{description}
\item[NA] The monster's Natural Armour, given as the number of Damage
Points (DP's) absorbed by the monster's skin, scales, etc., for each
Strike.

\item[Weapons] The natural weapons of the monster in the forms of
claws, teeth, talons, etc.  The damage done by each natural weapon,
its Base Chance and, in some cases, its possible Rank, are listed
along with each weapon.  Monsters always add their Manual Dexterity to
their Base Chance with any natural weapon whether Ranked or not.  For
purposes of Grievous Injury, all teeth, horns, and tusks do A class
damage.  Talons and claws do B class damage.  Hooves and other
appendages to butt or kick do C class damage.

\item[Comments] Any special characteristics of the monster, including
its preferences in diet, treasure that it may have scavenged, etc.,
are discussed under this heading.

\end{description}
\end{multicols}
