\chapter*{Adventure}

\begin{multicols*}{3}
\index{adventure|(}

\section{Equipment and Monetary Matters}

\subsection{Purchase of goods and items}

The GM will be guided in determining the price (in Silver Pennies) of
the various goods produced by craftsmen by the Price List (see Players
Handbook and Tables \S\ref{tables:weapons}, \S\ref{tables:armour} and
\ref{tables:shields}. The three factors which determine the price of
finished goods are the quality of the material used, the hours spent
in construction, and the estimated Rank of the artisan (if one person
produces the goods) or of the overseer (if the effort is a team
project). However, if a character wishes to purchase a custom-made or
rare item, then they will have to negotiate with the artisan
(represented by the GM), and may defray costs by providing some of the
scarcer components themself. The barter system is acceptable when
dealing in costly or rare items.

The value of a coin is determined by its weight and metal of which it
is made.

\begin{tabularx}{\linewidth}{Xll}
\textbf{Name}		& \textbf{Weight}	&\textbf{Value} \\
Copper farthing (cf)	& 1/5 oz \\
Silver penny (sp)	& 1/20 oz	& 4 cf \\
Gold shilling (gs)	& 1/20 oz	& 12 sp \\
Truesilver guinea (tg)	& 1/10 oz	& 21 gs \\
\end{tabularx}

Other common coins include the halfpenny, three-pence, and
sixpence. The values and weights of these coins correspond to those of
the Silver Penny.

\subsection{Encumbrance Modifies Agility}

The weight borne by a character may temporarily reduce the character's
Agility.

To calculate modified Agility user the Fatigue and Encumbrance Table
(\S\ref{tables:encumbrance}) and:
\begin{Enumerate}
\item
Cross-reference the character's Physical Strength and the weight they
are carrying.

\item
Read down this column until it intersects with the row which reads
``Agility Loss.''

\item
Deduct the resulting number from the character's Agility to give
Modified Ability.

\item
Re-calculate this number if there is a change in the weight they bear.
\end{Enumerate}

The character's Modified Agility is used as a basis for determining
their current TMR. A character is considered to have a minimum Agility
of 1 for all other game functions.

\section{Health and Fitness}
\label{adventure:health}
An character's Fatigue will vary depending upon the amount of food and
rest they get compared to their activities.

A character's Endurance may be temporarily reduced by lack of
sustenance, extreme activities, damage, or illness.

\subsection{Eating and Drinking}
\label{adventure:eating}
The amount of food and water required per day is dependent on many
factors. These include the person (endurance, weight, build, metabolic
rate and race) and the level of activity they are involved in (light,
medium, hard or strenuous).

On average 1 lb of food and 2 pints of water per day is required.

\subsection{Starvation}

Starvation occurs when a character does not have at least 1 nourishing
meal a day.

If a character is starved they will have their Fatigue maximum and
Endurance temporarily reduced by 1 each day. This decrease will last
until the character starts receiving proper nourishment.

A starved character's Fatigue maximum and Endurance will recover by 1
point each day, after the first, that they receive proper nourishment

\subsection{Dehydration}

Dehydration occurs when a character does not have at least 2 pints of
water a day. This will increase in high temperatures, plus 1 pint per
10 degrees above 20.

If a character is dehydrated they will have their Fatigue maximum and
Endurance temporarily reduced by 5 each day. If the character receives
part of their water requirement, the penalty is reduced. For every
20\% (or fraction) less than the daily requirement they lose 1 from FT
max and EN. This decrease will last until the character starts
receiving adequate quantities of water.

A dehydrated character's Fatigue and Endurance maximums will recover
by 5 points each day, after the first, that they receive adequate
quantities of water.

\subsection{Tiredness and Rest}

Characters have a tendency to lose Fatigue points on adventure. A
fatigued character must rest to recover Fatigue points. Sleep, as
might be expected, is the best way to become refreshed, but food and
rest will also help.

The Fatigue point loss and recovery rates given in these rules assume
that the character is in good health and is well fed. If the character
is not in condition, the GM may adjust the effects of activity, the
effects of weight carried and the rate of recovery.

\subsubsection{Fatigue Loss}
\label{adventure:exercise}
A character can lose Fatigue points when they engage in any activity
more stressful than a leisurely walk.

There are four classes of activity which can fatigue a character:
\begin{Enumerate}
\item
Light Exercise includes moderate to brisk walking, riding slowly or at
a moderate pace on a docile mount, etc.

\item
Medium Exercise includes jogging, riding on a cantering mount, light
construction or precision work, etc.

\item
Hard Exercise includes paced running, riding at a gallop, hard manual
labour, etc.

\item
Strenuous Exercise includes constant sprinting, breakneck riding, and
generally those actions with which the character pushes their body to
its practical limits.
\end{Enumerate}

It is possible for a character's actions to be more taxing than
Strenuous Exercise, which requires superhuman exertion. This Fatigue
loss from this activity will be determined by the GM.

A character's degree of exertion is judged each hour. 

The GM should indicate to players the level of exertion of their
activities (averaging where necessary). If the GM gives consistent
guidelines the players will be able to keep an ongoing track of
fatigue loss.

\subsubsection{Encumbrance}

A character is limited in the weight they can bear and if they engage
in exercise, they may become fatigued more quickly because of the
weight they are carrying.

The Fatigue and Encumbrance Table (\S\ref{tables:encumbrance}) lists
the maximum weight a character may carry.

A player must determine the total weight their character is carrying
if the character is to engage in light or more stressful exercise for
a significant length of time during a day.

When an entity has a Physical Strength value greater than 40, the GM
divides that value by 40. Multiply the quotient by the entry for 40,
and add the entry corresponding to the remainder to determine that
entity's capabilities.

\subsubsection{Damage}

A character may lose Fatigue by being damaged.  This may be recovered
naturally or by being healed.

\subsubsection{Spell Casting}

A character may lose Fatigue by using magical abilities.  This may be
recovered naturally but may not be healed.

\subsubsection{Calculating Current Fatigue}
\label{adventure:fatigue}

The Fatigue status of a character only needs to be calculated before
they enter into combat, wish to perform magic or if they perform
fatiguing activities for long periods.  To calculate current Fatigue
use the Fatigue and Encumbrance Table (\S\ref{tables:encumbrance}):
\begin{Enumerate}
\item
Cross-reference the character's Physical Strength and the weight they
are carrying.

\item
Read down this column until it intersects with the row corresponding
to the character's rate of exercise.

\item
Multiply the resulting number (Fatigue points lost per hour) by the
number of hours at this exercise level.

\item
Perform this calculation once for each time one (or more) of the three
factors changes.

\item
Add each separate subtotal to determine the total Fatigue points
expended by the character so far.
\end{Enumerate}

\subsubsection{Exhaustion}

If a character's Fatigue point total is reduced below zero, they are
exhausted. An exhausted character is limited in the activities they
may choose to do and the performance of their abilities is adversely
effected. Their Fatigue is considered zero for the purposes of combat
or magic use.

A character may choose to exert themselves after their Fatigue points
are reduced to zero until they have expended a nominal one-half their
initial Fatigue points (round down). When they reach this limit they
will collapse unless they succeed a 1 \x WP check every (2 \x
Endurance) minutes.

An exhausted character must sleep for as much time as they were
performing any light exercise while exhausted \emph{before} they may
recover \emph{any} Fatigue points.

If an exhausted character wishes to engage in Strenuous Exercise, they
must succeed a separate 1 \x WP check.

\subsubsection{Exhaustion Modifier}

The character must subtract 1 / half hour (or fraction) of exhaustion
to any base chance.

\subsection{Fatigue Recovery}

A character may regain Fatigue points naturally by eating a hot meal
or resting.

A character may never have a Fatigue total greater than their Fatigue
Characteristic.

A character naturally recovers Fatigue points as follows:

\begin{tabular}{ll}
\textbf{Activity} & \textbf{Fatigue points / Hour} \\
Eat Hot Meal	& 2 \\
Relaxation 	& 1 \\
Nap		& 2 \\
Sleep		& 3 \\
\end{tabular}

\begin{Enumerate}
\item
A character may benefit from a hot meal no more than three times
during a 24 hour period, and each meal must be separated by at least 4
hours.

\item
A character that does not get at least 6 hours of rest and/or sleep
per day will have their Fatigue maximum temporarily reduced by 1 FT /
hour (or fraction) of sleep under 6 hours. This may be recovered at
the rate of 1 FT / 4 hours sleep.

\item
If a character's Endurance is less than 10, they recover one-half of a
FT point less per hour or meal, and if their Endurance is less than 5,
they recover one less FT point. However, a character always recovers a
minimum of one-half a FT point when resting.

\item
If a character's Endurance is from 21 to 30, they recover an
additional one-half of a FT point per hour or meal. Each succeeding
ten point Endurance bracket carries an additional one-half FT point
bonus.

\item
Fatigue loss from damage may also be recovered by magical healing (but
not the Healer skill Heal Endurance).
\end{Enumerate}

\subsection{Damage and Illness}

\subsubsection{Effects of Low Endurance}
\begin{Description}
\item[Unconsciousness] When an entity's Endurance reaches 3 or less,
they must make a (current EN) \x WP check or fall unconscious; this
WP check is repeated every minute or if their EN changes.

An entity on 0 Endurance is unconscious, but stable. An entity with a
full Endurance of 5 or less does not make consciousness checks. They
remain conscious until they fall to 0 or less Endurance.

\item[Below Zero Endurance] An entity on negative Endurance will lose
one point of Fatigue (Endurance when no Fatigue remains) until the
bleeding is stanched by a Healer, or until dead. They will continue to
take damage from any further blows, spells, grievous wounds which are
bleeders, burning, etc.

When an entity is below zero Endurance they are on the very brink of
death.  It takes time and skill to tell the difference between this
state and death (\eg empathy, DA). GMs should not let players take
advantage of out of character information when another player's
character is below 0 Endurance.

\item[Death] When an entity's Endurance falls below negative one-half
their full Endurance, they are dead. Once dead, ongoing damage (\eg
poison or bleeding) ceases but further damage may be inflicted on the
body.
\end{Description}

\subsubsection{Endurance recovery}

There are many causes of a character losing En-durance
points. However, once lost there are two primary methods of recovering
them.

\subsubsection{Healers and Magical Healing}

Healers, herbalists, potions, medicines and some magics may aid the
recovery of Endurance. The exact effects can be found under the
appropriate skill or magic.

\subsubsection{Natural Healing}

The rate at which Endurance Points recover naturally primarily depends
on how active the injured being is.

If an entity is resting they regain 1 Endurance point every three full
days.

This rate is reduced to 1 / 4 days if the entity:
\begin{Itemize}
\item takes any further EN damage 
\item uses more than half their FT
\item does not receive adequate nourishment
\end{Itemize}

If an entity is given ministrations from a physicker's kit, their body
requires one less day to regain an Endurance Point.

Injuries which are not quantified as Endurance point losses or
grievous injuries (\eg hamstrung muscles) heal at the same rate as
they do in this world.

These healing rates are based on average Endurance value of 15. The GM
may chose to increase the healing rate if an entity's full Endurance
is very high or decrease it for a low Endurance entity.

\subsubsection{Potions \& Unconscious Patients}

An entity cannot drink a healing potion when they are unconscious or
below zero endurance but one can be massaged down their throat. The
chance of doing this is equal to the Manual Dexterity + Perception of
the person administering the potion, or if a healer, 90 + Healer
Rank. If successful then D10 per 10 points of the healing potion's
curing (round down) will be received. If the person fails the roll,
the potion is wasted, but no harmful effects occur to the patient.

\subsubsection{Grievous Injuries}

Endurance loss resulting from specific grievous injuries may not be
healed separately from the underlying specific injury.  When the
specific injury is fully cured the related endurance is recovered
automatically.

\subsubsection{Natural healing of Specific Injuries}

Major injuries take a long time to heal and some will not heal
naturally but require a healer. Here are guidelines for the healing
requirements of some common major injuries.
\begin{Description}
\item[Broken bones] will knit in 4 weeks for a simple fracture, or up
to 10 weeks for a compound fracture.  A bone must be properly set
before the bone may knit together.

\item[Internal injuries] an entity will usually die from internal
injuries. If the patient is comfortable, unmoving, and kept alive by a
healer or physician, internal injuries will heal 1 Endurance point per
week

\item[Open wounds] will heal at half the normal rate, provided that
they are kept free of infection. Open wounds will leave scars.

\item[Removed body parts] will not regrow naturally. However, the
remaining wound will heal over at quarter the normal rate, provided it
is kept free from infection.

\item[Magical healing of Specific Injuries] Healers and certain magics may
heal specific injuries. The time taken and effects of these magics may
be found under the appropriate skill or magic.
\end{Description}

\subsection{Infection}
\label{adventure:infection}
\index{infection}
If a character is wounded there is the possibility that they have
become infected as a result of their wounds.

An Infection Check must be performed to determine whether they are
infected or not.

\subsubsection{Becoming Infected}

The chance of becoming infected depends on the entity's health, the
type of injury and the environment the entity is in, modifiers are
cumulative but only one from each category:

\begin{tabbing}
\hspace{2em}\= \kill
There is a wound which is $\ldots$ \\
\> Dirty			\` +20\% \\
\> Heavily contaminated		\` +50\% \\
The environment is $\ldots$ \\
\> Dry   			\` -5\% \\
\> Humid			\` +20\% \\
The average temperature is $\ldots$ \\
\> Below 0			\` +20\% \\
\> 1 -- 5			\` +10\% \\
\> 30 -- 40			\` +10\% \\
\> Above 40			\` +20\% \\
\end{tabbing}

Some specific grievous injuries also increase the chance of infection.

\subsubsection{Effects of Infection}

An entity with an infection will be slowly poisoned by the
infection. The damage is [D - 5] Endurance per day, until the infection
is cured. An infected wound will not heal until the infection is
cured.

\subsubsection{Curing Infection}

There are two ways to recover from infection. The first is to tough it
out. The second is to be healed by a healer.

\begin{Description}
\item[Toughing it out] An infected character may make a 1 \x Endurance
check every day to recover naturally.

\item[Healing] An infected character may be cured by the arts of a
Healer or by magic. The rank at which this is possible, and the chance
of success can be found under the appropriate skill or spell.
\end{Description}

\section{Ranking}
\label{adventure:ranking}
\index{ranking|(}

Experience points are required to advance in anything.  Time spent
training is required to increase proficiency in spells, skills and
weapons.  Adventuring time is required to advance in characteristics
and talents.

EP is spent as per below but note the following.
\begin{itemize}

\item
Spells must be cast at least five times before the next rank can be
achieved.  Casting chambers are available in the Guild, and monitored
on request.

\item
Talents may be ranked only once per game adventuring week.

\item
Weapon skills take 1 week of training to reach Rank 0, and 2 weeks \x
(rank to be achieved) to improve.

\item
Skill ranks 8, 9, and 10 must be ratified by a GM and in general
required significant use of the skill.

\item
The character may rank any combination of two things at the same time,
providing the character does not rank magic (\ie spells or rituals)
at the same time as non-magic (\ie weapons or skills).

\item
Namers may rank 1 name in addition to other forms of ranking.  They
may also substitute ranking names for any other form of ranking.
Hence it is possible for a Namer to be ranking up to three names at
any one time.

\end{itemize}

All adventurers can learn any skill or weapon, and any magic within
their college. Learning costs EP, time and money.  The following can
be ranked: Skills, Weapons, Spells, and Languages, Names, Rituals,
Talents, Adventuring Skills and Characteristics.

\subsection{Characteristics}

A Characterisitic may only be raised by five points over its
\emph{starting value} to a maximum of 25 (modified by racial bonuses /
penalties), except Fatigue, and Physical Beauty (which cannot be
raised).  The exception to this is Perception, which can be raised to
racial maximum.

To calculate the maximum for Fatigue, take the racial maximum for
endurance and find the maximum Fatigue from the chart (see character
generation).  For example, dwarves have +2 to EN so their maximum EN
is 27, hence their maximum FT is 24 from the chart.  Then apply any
additional racial modifiers to FT.  For example, orcs gain +2 to FT so
their maximum is $24 + 2 = 26$.

A Characteristic point may only be increased once per adventuring
session (if a session took more than the normal session length, then
this rule should be applied appropriately).  If a character did not
adventure during a session then they cannot raise any characteristic
points, and if they participated on more than one adventure during the
session, then they can still only raise a characteristic point by one.
Any or all of the stat points may be raised simultaneously if
permissable.

If a character has lost characteristic points for any reason, they may
buy back as many points as they wish in addition to any normal
increase.  The cost of buying a characteristic point back is the same
as buying an extra point (see table \S\ref{epcosts:stats}).

\subsection{Talents}

For each week of actual, out in the field, adventuring, you can rank
each of your talents once.  No training time is required to rank
Talents. Like Spells and Rituals, each Talent has an EM. No MA
discounts apply to any Talents.

\subsection{Spells}

If you are an Adept (\ie cast magic), you can rank your spells.
Each spell has an EM, or Experience Multiplier. This is multiplied by
the Rank that you wish to achieve, to give a total EP cost. If the
Adept has MA $>$ 15, (MA - 15) \x 5\% of the EP cost of General Spells
may be discounted. Training time for spells is (Rank to be achieved)
days.  All spells must be cast five times before each new rank is
achieved. As these spells may backfire, these rolls will need to be
made at some stage.  Learning a new spell to Rank 0 takes (EM / 100
rounded up + 1) weeks, but no experience points. See Handbook for
availability of special knowledge spells.  You can not have more
spells and rituals below rank 6 than your MA characteristic.  Rank 20
is the Maximum Rank achievable with any Spell (except Geas).

\subsection{Rituals}

Rituals are learnt and ranked just like spells, except that Ranking
time is (Rank to be achieved) weeks, rather than days.  MA discount
applies to General Knowledge Rituals.

\subsection{Skills}

All skills are assumed to be unranked (\ie unknown) initially. The
first level of competence is Rank 0, and will take eight weeks to
learn. Each subsequent rank will take that number of weeks to reach
(eg; to get to Rank 8 from Rank 7 will take 8 weeks). The EP cost for
ranking Skills is listed in the rulebook.  Some skills require minimum
Characteristic requirements to Rank, or impose EP penalties (or
discounts) for exceptional Characteristics.

If the character is taught by someone of greater Rank in the skill,
decrease any Experience Point cost by 10\%.  If the character learns
from a book (the availability of which is up to the GM), verbal
descriptions or practices with someone of equal or lesser Rank in the
skill, any Experience Point cost is unmodified.  If the character
practices with no useful outside assistance, any Experience Point cost
is increased by 25\%.  If training is done at the Guild, it costs
150sp \x (Rank to be achieved minimum 1). Achieving Ranks 8, 9 and 10
is difficult.  You must find and complete a special task relating to
your skill, with the assistance of a GM, for each of these Ranks. Rank
10 is the maximum achievable Rank in all Skills.

\subsection{Adventuring Skills}

Adventuring Skills are skills used every day by adventurers to
survive, and thus are continually honed. These skills include
Knowledge, Horsemanship, Swimming, Flying, Stealth and Climbing.

If you have extensively used an adventuring skill while on adventure,
you may rank this skill without any time requirements. Otherwise,
Ranking time is as per normal skills.

Adventurers are assumed to start off with Rank 0 or more in all these
skills, unless specifically told otherwise. Experience Point costs are
set out in the rulebook.  The Maximum Rank in all these Skills is 10.

\subsection{Languages}

Languages have the same time requirement as normal skills, except that
the time for Rank 0 in a language is only 1 week.

The undiscounted EP costs are set out in Table \S\ref{epcosts:skills}.
Note that knowing the Philosopher Skill may grant an EP discount. The
maximum total EP discount applicable is 50\%, regardless of how many
individual discounts are available to the character.

\subsection{Weapons}

All weapons are assumed to be unranked initially.  Rank 0 in a weapon
takes 1 week.  All higher ranks take 2 \x Rank weeks. Weapons have
individual maximum Ranks.  EP costs are detailed in \S\ref{tables:ep}.
All Weapons require minimum PS and MD Characteristics. If you do not
fulfil both requirements, you may not rank a weapon.  You may not get
an EP discount for training, but if no trainer is available, you may
not increase in Rank.  The cost of a trainer is 10 \x Rank squared
(minimum 1) silver pennies.

\subsection{Names}

Anyone can learn Names, but only Namers can Rank them beyond Rank 0.
Once acquired, an Individual or Generic Name may be studied and fully
learnt. For Ranking Names beyond see the College of Naming
Incantations (\S\ref{namer:ranking}).

\begin{Itemize}
\item Generic Names take one day of study to be learned (\ie Rank 0).

\item Individual Names take on week of study to be learned.

\item Names can be Ranked alongside any other Ranking.
\end{Itemize}

\index{ranking|)}
\index{adventure|)}
\end{multicols*}
