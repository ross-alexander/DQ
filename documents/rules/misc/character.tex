\chapter*{Character Generation}

\begin{multicols*}{3}

\index{Character Generation|(}
\section{Explanation of Characteristics}

This section is an explanation of a character's characteristics and
how they are used in the game.

All characteristics are calculated when the character is generated but
Adventurers in a world of magic can expect them to change from time to
time.  A ``temporary'' change indicates an increase or decrease of
limited duration to the value of a characteristic; a ``permanent''
change indicates an increase or decrease of indefinite duration to the
value of a characteristic.

The first six characteristics are the \emph{primary} characteristics.
These can be increased temporarily by magic or permanently by training
(expenditure of experience points), and can be decreased temporarily
by magic or injury, or permanently by injury to the character. These
primary characteristics can never be trained more than 5 above their
starting value, and never above racial maximum, except by unusual
magical means.

All other characteristics are \emph{secondary} characteristics. The
manner in which a secondary characteristic can be changed will be
covered in the appropriate explanation.

Generally, a high characteristic value indicates a character's ability
to perform a certain task well, while a low value indicates a relative
lack of such ability. A characteristic's effect is almost always
translated into numerical terms for the purposes of resolving action
during play.  Adventurers generally have higher characteristics than
normal people in the world, that is what makes them heroes after all.

\subsubsection{Effects of Characteristics}

A character develops specific skills during the game, and their
characteristics influence their base chances with these skills.
However there are also many feasible tasks that a character may wish
to perform without having a specific skill to do so.  The GM then uses
the most appropriate characteristic to generate a base chance to
perform that task.

\subsubsection{Difficulty Factors (Characteristic Multipliers)}

When a player declares that their character will attempt a task which
the GM acknowledges as dependent upon a particular characteristic, the
GM assigns the task a difficulty factor. This difficulty factor will
be a number from 1/2 through to 5.

The greater the difficulty factor value, the easier a task will be to
perform.

The player multiplies the difficulty factor by the appropriate
characteristic, arriving at the percentage chance of the character
performing the task.  The maximum base chance is (70 + characteristic
+ difficulty factor)\%.  The player then rolls D100, and if the roll
is less than or equal to the percentage then the character has
successfully performed the task. If the roll is greater than the
percentage, the character has failed. If the roll fails by at least
the value of the characteristic or exceeds the maximum base chance,
the character has failed miserably and may have injured themself. The
GM may wish to determine the extent of the injury by how much the roll
exceeds the percentage plus the characteristic.

\subsection{Strength (PS)}

\textbf{Physical Strength} is a measure of a character's muscle
co-ordination and strength.  The Physical Strength characteristic
represents the brute force a character can exert from the thews of
their arms, the thrusting power of their leg muscles, and their lift
and weight capacity.

\subsubsection{Specific Influences}
\begin{Itemize}
\item Effects of weight carried
\item Minimum PS requirement for weapons
\item Damage
\end{Itemize}
\subsubsection{Generic Uses}
\begin{Itemize}
\item Breaking objects
\item Lifting heavy or awkward objects
\end{Itemize}
\begin{example}
Consider the sturdiness of the object and the implement being used to
break it for the former, and consider the weight and bulk of the
object plus the purchase afforded the character for the latter.
\end{example}

\subsection{Manual Dexterity (MD)}

\textbf{Manual Dexterity} is a measure of a character's control with
their hands.  The Manual Dexterity characteristic represents the
character's hand-to-eye co-ordination, the speed at which the
character can perform a complex task with their hands, and the ability
to manipulate their hands.

\subsubsection{Specific Influences}
\begin{Itemize}
\item Minimum MD requirement for weapons
\item Thievery
\item Strike Chance in Combat
\end{Itemize}
\subsubsection{Generic Uses}
\begin{Itemize}
\item Handling dangerous substances
\item Fine handicrafts \& other delicate tasks.
\end{Itemize}
\begin{example}
Consider the delicacy of the task when a character seeks the careful
manipulation or removal of an object.
\end{example}

\subsection{Agility (AG)}

\textbf{Agility} is a measure of a character's ability to manoeuvre
their whole body and their speed of movement.  The Agility
characteristic represents the character's litheness of body, the speed
at which the character can run, and their ability to dodge with or
contort their body.
\subsubsection{Specific Influences}
\begin{Itemize}
\item Tactical Movement Rate
\item Speed in combat
\item Defence
\item Most physical skills
\end{Itemize}
\subsubsection{Generic Uses}
\begin{Itemize}
\item Manoeuvring
\end{Itemize}
\begin{example}
Consider speed, distance, and complexity of the manoeuvre, as well as
the nature of any obstacles or features they are using.
\end{example}

\subsection{Magical Aptitude (MA)}

\textbf{Magic Aptitude} is a measure of a character's ability to
harness and direct magical energies.  The Magic Aptitude
characteristic represents the character's control over the flow of
mana (the stuff of magic), and their ability to remember spells and
rituals.
\subsubsection{Specific Influences}
\begin{Itemize}
\item Magic Colleges have a minimum MA requirement
\item Cost of training magic
\item Base chances of magical skills
\end{Itemize}
\subsubsection{Generic Uses}
\begin{Itemize}
\item Noticing arcane mana effects
\end{Itemize}

\subsection{Willpower (WP)}

\textbf{Willpower} is a measure of a character's self control of mind
and body, especially in stressful situations.  The Willpower
characteristic represents a character's ability to concentrate, their
ability to resist the imposition of another's will upon their own, and
the degree to which their will can be used to counter their instincts
(when, for instance, the character might be attempting an action which
could be suicidal).

\subsubsection{Specific Influences}
\begin{Itemize}
\item Magic resistance
\item Fear resistance
\item Concentration checks to perform magic
\item Recovering from being stunned
\end{Itemize}
\subsubsection{Generic Uses}
\begin{Itemize}
\item Resisting suffering
\item Persevering with boring or dangerous tasks
\end{Itemize}

\subsection{Endurance (EN)}

\textbf{Endurance} is a measure of the punishment a character's body
can absorb before the character becomes unconscious, sustains mortal
wounds, or dies.  The Endurance characteristic represents the
character's capacity to withstand wounds, their resistance to disease
and infection and their rate of recovery from same, and directly
affects their ability to over exert themselves.
\subsubsection{Specific Influences}
\begin{Itemize}
\item Starting fatigue
\item Damage capacity
\item Stunning from damage
\end{Itemize}
\subsubsection{Generic Uses}
\begin{Itemize}
\item Resisting poison, infection \& disease
\end{Itemize}

\subsection{Fatigue (FT)}

\textbf{Fatigue} is a measure of a character's physical and mental
fitness.  The Fatigue characteristic represents the degree to which
the character can exert themself before becoming exhausted, the number
of minor cuts and bruises they can take before their abilities are
affected, and the mental energy that can be used to cast spells. This
characteristic directly reflects a character's current level of
tiredness as it is reduced temporarily with any strenuous activity and
restored to normal with rest.  Fatigue may be permanently increased by
training up to 5 points or to racial maximum.

\subsubsection{Specific Influences}
\begin{Itemize}
\item Sustained activity
\item Minor damage capacity
\item Spell casting energy
\end{Itemize}
\subsubsection{Generic Uses}
\begin{Itemize}
\item Ignoring cold
\item Coping with missing meals or sleep
\end{Itemize}

\subsection{Physical Beauty (PB)}

\textbf{Physical Beauty} is a measure of a character's exterior
attractiveness (or repulsiveness) as perceived by the humanoid races.
Physical Beauty is a characteristic representing a character's
appearance compared to the aesthetic standards of the main sentient
races.  It is in no way a reflection of a character's personality.
Specific reactions to PB are also influenced by the observer's race
and gender.  The Physical Beauty values for monsters describe how that
monster appears to a character, and not to another monster of the same
race.  Physical Beauty can be increased or decreased temporarily by
magic, and decreased permanently by disfigurement. It cannot be
increased by training.
\subsubsection{Specific Influences}
\begin{Itemize}
\item Reaction rolls
\end{Itemize}
\subsubsection{Generic Uses}
\begin{Itemize}
\item Influencing NPCs
\end{Itemize}

\subsection{Perception (PC)}

\textbf{Perception} is a measure of a character's intuition developed
as a result of their experience.  The Perception characteristic
represents the character's ability to note peculiarities in a given
situation, their ability to deduce a person's habits or customs from
scant information, and their general knowledge of the world.

The Perception value can be increased or decreased temporarily, and
can be increased permanently through training up to racial maximum.
Magic, certain natural or alchemical preparations, and the character's
condition can cause a temporary increase or decrease in the Perception
value.
\subsubsection{Specific Influences}
\begin{Itemize}
\item Detecting ambushes or traps
\item Detecting hidden things
\item Initiative
\end{Itemize}
\subsubsection{Generic Uses}
\begin{Itemize}
\item Picking up information from conversation or observation
\item Peripheral vision
\item Noticing things out of the ordinary
\item Remembering vague information
\item Making connections between new clues and previous knowledge
\end{Itemize}

\subsection{Tactical Movement Rate (TMR)}

The \textbf{Tactical Movement Rate} is the fastest speed a character
can move in combat.  A character's Tactical Movement Rate (TMR)
characteristic is based on their Agility and influenced by any weight
carried or restricting clothing.  It may be temporarily modified by
magic or injury, but cannot be trained.
\subsubsection{Specific Influences}
\begin{Itemize}
\item Distance moved in combat
\end{Itemize}
\subsubsection{Generic Uses}
\begin{Itemize}
\item comparative speeds
\end{Itemize}
\end{multicols*}

\pagebreak

\begin{multicols*}{3}
\section{Character Generation}

There are six sections in Character Generation:
\begin{Description}
\item[\ref{chargen:points}] Characteristic Points
\item[\ref{chargen:race}] Race
\item[\ref{chargen:description}] Description
\item[\ref{chargen:aspect}]Aspect
\item[\ref{chargen:heritage}] Heritage
\item[\ref{chargen:abilities}] Starting Abilities \& Possessions
\end{Description}

Sections \ref{chargen:points} -- \ref{chargen:heritage} may be done
in any order. Each section is designed so that a player may choose
from a range of options or randomly generate their character. Section
\ref{chargen:abilities} should be done last.

\subsection{Characteristic Points}
\label{chargen:points}

A character has 6 primary statistics which are generated by allocating
points from a total, and 4 secondary statistics which are either
derived from the primary statistics or are generated randomly. The
higher the number, the better the characteristic.

\subsubsection{Generating Characteristic Points}

The player may choose to allocate the primary statistics from a total
of 90 points or may roll 2D10 once against the following table. If
they choose to roll the result must stand.

\begin{tabular}{ccl}
\textbf{Die Roll} & \textbf{Points Total} \\
2	& 81 \\
3	& 82 \\
4	& 83 \\
5	& 84 \\
6	& 85 \\
7	& 86 \\
8	& 87 \\
9	& 88 \\
10	& 89 \\
\textbf{11} & \textbf{90} & \textbf{(default choice)} \\
12	& 91 \\
13	& 92 \\
14	& 93 \\
15	& 94 \\
16	& 95 \\
17	& 96 \\
18	& 97 \\
19	& 98 \\
20	& 99 \\
\end{tabular}

\subsubsection{Assigning Characteristic Points}

This total of points need to be spent on the following
characteristics: Physical Strength, Manual Dexterity, Agility, Magical
Aptitude, Willpower \& Endurance. These characteristics may change
during the game, and may be raised up to 5 points through training,
though not past the character's racial maximum.

The human range for each of these characteristics is 5 -- 25; this
range is adjusted for non-humans (see the Characteristic Modifier
tables for the non-human races). These ranges represent the minimum
and maximum capabilities of the races. The player should assign the
points and then make any adjustment for race.

Prior to assigning the characteristic points, the player should give
some thought to what kind of character they wish to have and what
weapons, spells, and/or skills are desired for the newly created
individual. Some weapons require a great deal of Physical Strength or
Manual Dexterity, and the player should be sure to assign enough
points in those areas to use the weapons of their choice. All magical
colleges require a minimum Magic Aptitude to join and the player
should be aware of these restrictions. Most skills do not have any
special requirements, but many give bonuses for exceeding a minimum
value in certain characteristics.

When the player has chosen the values for the character, they must
record them on a Character Sheet. The total value of the six primary
characteristics (before racial modifiers) must equal the amount
received in the Generating Characteristic Points section; thus, a
player cannot ``save'' Characteristic Points and assign them to
characteristics at a later date. The value of each of the six primary
characteristics must be recorded before any secondary characteristics
are generated.

\subsubsection{Generating Secondary Characteristics}

Fatigue, Physical Beauty, Perception and Tactical Movement Rate are
secondary characteristics. They may be modified if the character is
non-human (see the Characteristic Modifier tables for the non-human
races).

\textbf{Fatigue}

The value of a character's Fatigue is a direct function of their
Endurance. The player enters the Fatigue value corresponding to the
character's Endurance value after their Endurance has been modified
for race.


\begin{tabular}{cc}
\textbf{Endurance} & \textbf{Fatigue} \\
\textbf{3 or 4}	& \textbf{16} \\
 5 to 7		& 17 \\
 8 to 10	& 18 \\
11 to 13	& 19 \\ 
14 to 16	& 20 \\
17 to 19	& 21 \\
20 to 22	& 22 \\
23 to 25	& 23 \\
\textbf{26 to 27} & \textbf{24} \\
\end{tabular}

Endurance and Fatigue values in bold type can be achieved only by
members of certain non-human races.

From this point on, a change in a character's Endurance value will not
affect their Fatigue value and vice-versa. Fatigue may be raised by up
to 5 points, though not past the character's racial maximum.

\subsubsection{Physical Beauty}

The value of the Physical Beauty characteristic is generated randomly
by rolling 4D5 + 3. This characteristic can never be increased by
training.

\subsubsection{Perception}

A character's perception value begins at 5. This may be trained up to
racial maximum.

\subsubsection{Tactical Movement Rate}

A character's Tactical Movement Rate (TMR) is a direct function of
their Agility. It is based on the character's Agility value and is
recalculated when Agility is modified by encumbrance and armour
penalties; see the TMR table (\ref{tables:tmr}) for values.

\subsection{Race}
\label{chargen:race}

\textbf{A player must choose the race of their character.}

The majority of people in Alusia are human, but the player may choose
one of the common non-human races: dwarf, elf, halfling, or orc.

If the player wishes their character to be giant or shape changer they
must roll D100.  They may roll once per race and if the roll is lower
than the race chance \% they must take that race.  If they fail then
the character must be of one of the common races.  If the player is
attempting to be a shape changer they must decide what type of shape
changer they want prior to rolling (\ie wolf, tiger, bear or boar).

\begin{tabular}{lc}
\textbf{Race} &	\textbf{Chance (\%)} \\
Hill Giant	& 06 \\
Shape-Changer	& 04 \\
\end{tabular}

A player may wish to play one of the very rare sentient races.  To do
so they must get the agreement of both the generating GM and a member
of the character tribunal.  They will decide which of the common races
has the appropriate racial modifiers.  For example Erelheine
characters are generated using the Elf option.

Humans learn faster than non-humans.  Learning is represented in game
by spending Experience Points (EP). \emph{Divide} any experience
points a character gains by the ``racial modifier'' and then spend the
result normally.

\begin{tabular}{lc}
\textbf{Race}	& \textbf{Modifier} \\
Dwarf		& 1.1 \\
Elf		& 1.2 \\
Halfling	& 1.1 \\
Hill Giant	& 1.5 \\
Human		& 1.0 \\
Orc		& 1.1 \\
Shape-Changer	& 1.4 \\
\end{tabular}

Each race has a description of a stereotypical member of the race and
any special abilities and characteristic modifiers that apply to a
character of that race.

\subsubsection{Dwarf}

\textbf{A dwarf is a short, bearded humanoid, usually taciturn who
frequents mountainous areas.}

\textbf{Description:} Pride and attention to detail are important to
dwarves. They form strong community ties, and are distrustful of
strangers, especially those of other races. Their strongest
antipathies are towards orcs and elves. Although dwarves are greedy by
nature, they are essentially honest and stand by their word.  Dwarves
covet precious stones and metals, and appreciate fine, detailed
workmanship. Dwarven warriors favour the axe as weapon.

\subsubsection{Special Abilities}

\begin{Enumerate}
\item
Dwarves close vision is exceptionally sharp, but many have poor
distance vision. They can see in the dark as a human does at
dusk. Their effective range of vision in the dark is 50 feet under the
open sky, 100 feet inside man-made structures, and 150 feet inside
caves and tunnels.

\item
Dwarves can assess the value of and deal in gems and metals as if they
are a Merchant of Rank 5. If a dwarf character progresses in the
Merchant skill, their ability to assess the value of gems and metals
is five greater then their current Rank, to a maximum of ten.

\item
If a dwarf character is a Ranger specialising in mountains or caverns,
they pay half the EP cost necessary to advance ranks.

\item
A dwarf's capacity for alcohol is twice that of a human's.
\end{Enumerate}

\begin{tabularx}{\linewidth}{Xc}
\textbf{Characteristic} & \textbf{Modifier} \\
Physical Strength	& + 2 \\
Agility			& - 2 \\
Endurance		& + 2 \\
Magical Aptitude	& - 2 \\
Willpower		& + 2 \\
Perception		& + 1 \\
Physical Beauty		& - 2 \\
Tactical Movement Rate	& - 1 \\
Starting Age:		& 20 + \\
Average Life Span:	& 125 -- 150 years \\
Base Conception Rate	& 3\% \\
\end{tabularx}

\subsubsection{Elf}

\textbf{An elf is a slim agile humanoid, who frequents wooded areas.}

\textbf{Description:} Elves are virtually immortal and generally take
the long term view. They are insular, indifferent to others and tend
to be traditional.  Elves are great respecters of nature and
learning. Their Elders are repositories of great wisdom while elvish
youth are enthusiastic merry makers. Elven warriors favour bow weapons
and disdain metal armour. Members of other races generally find elves
attractive.

\subsubsection{Special Abilities}

\begin{Enumerate}
\item
Elves have superior vision especially over long distances or in poor
lighting.  An elf can see in the dark as a human does on a cloudy
day. Their effective range of vision in the dark is 150 feet under the
open sky, and 75 feet elsewhere.

\item
If an elf character is a ranger specialising in woods, they pay
one-half the EP to advance ranks.

\item
An elf receives a racial Talent which functions in all respects as the
Witchcraft Witchsight Talent.

\item
An elf makes little or no noise while walking and adds 10\% to their
chance to perform any activity requiring stealth.

\item
If an elf character takes the healer skill, the elf pays
three-quarters the EP to advance ranks, though they cannot resurrect
the dead.

\item
An elf is impervious to the special abilities of the lesser undead.

\item
If an elf character takes the courtier skill, the elf pays one-half
the EP to advance ranks.
\end{Enumerate}

\begin{tabularx}{\linewidth}{Xc}
\textbf{Characteristic} & \textbf{Modifier} \\
Physical Strength	& - 1 \\
Agility			& + 1 \\
Endurance		& - 1 \\
Magical Aptitude	& + 1 \\
Willpower		& + 1 \\
Perception		& + 1 \\
Physical Beauty		& + 2 \\
Tactical Movement Rate	& + 1 \\
Starting Age		& 30 -- 300 + \\
Average Life Span	& Circa 10,000 years \\
Base Conception Rate	& 1\% \\
\end{tabularx}

\subsubsection{Halfling}

\textbf{A halfling is a short, cheerful humanoid, who will be an
active participant in village life.}

\textbf{Description:} Halflings appreciate the good life more than
most; a successful halfling will arrange a schedule of much sleep,
good food, and relaxed study or conversation. Halflings are shy around
other races, preferring to merge into the background.  Amongst
themselves they are a friendly folk who form into small communities
where everyone knows everyone else's business. While Halflings take
their social responsibilities seriously, they are renowned for their
practical jokes and light fingers. Halflings are noted for their
tough, hairy feet and usually go barefoot. Halflings avoid the rigours
of military life but when forced to defend themselves they favour
small weapons.

\subsubsection{Special Abilities}

\begin{Enumerate}
\item
A halfling has infravision, which allows them to see faint red shapes
where living beings are located in the dark. Their range of vision is
100 feet.

\item
A halfling adds 20\% to their chance to perform any activity requiring
stealth.

\item
If a halfling takes the thief skill, they pay half the EP cost to
advance ranks.

\item
A halfling may drop jewellery down active volcanos without anyone
thinking the worse of them.
\end{Enumerate}

\begin{tabularx}{\linewidth}{Xc}
\textbf{Characteristic} & \textbf{Modifier} \\
Physical Strength	& - 3 \\
Manual Dexterity	& + 3 \\
Agility			& + 1 \\
Endurance		& - 2 \\
Magical Aptitude	& - 1 \\
Willpower		& + 1 \\
Physical Beauty		& - 1 \\
Starting Age		& 21 + \\
Average Life Span 	& 80 -- 90 years \\
Base Conception Rate	& 4\% \\
\end{tabularx}

\subsubsection{Hill Giant}

\textbf{A hill giant is a huge, coarse featured humanoid, who has no
patience for laborious learning.}

\textbf{Description:} Giants are lusty types, preferring nothing
better than to go through life brawling, drinking, and wenching. They
tend to gather together in a clan arrangement, building huge halls (or
steadings) in out-of-the-way locations. They are not overly
intelligent, and resent humans and elves particularly. Giants enjoy
riddling and bartering. Giant warriors favour simple weapons scaled to
their size.

\subsubsection{Special Abilities}

\begin{Enumerate}
\item
A giant has infravision, which allows them to see faint red shapes
where living beings are located in the dark. Their range of vision is
250 feet.

\item
A giant's magic resistance is increased by 10\%.

\item
Whenever a giant attempts minor magic, the GM should increase the
difficulty factor by one, making it easier.

\item
Giants may use the giant weapons listed in the Weapons Table
(\S\ref{tables:weapons}).
\end{Enumerate}

\begin{tabularx}{\linewidth}{Xc}
\textbf{Characteristic} & \textbf{Modifier} \\
Physical Strength	& + 7 \\
Manual Dexterity	& - 1 \\
Agility			& - 2 \\
Endurance		& + 8 \\
Magical Aptitude	& - 1 \\
Willpower		& - 1 \\
Fatigue			& + 1 \\
Physical Beauty		& - 1 \\
Tactical Movement Rate	& + 3 \\
Natural Armour		& + 1 \\
Starting Age		& 26 + \\
Average Life Span	& 500 years \\
Base Conception Rate	& 2\% \\
\end{tabularx}

\subsubsection{Human}

\textbf{Humans are by far the most common race on Alusia, frequenting
most areas and climes.}

\textbf{Description:} Humans have a great diversity of cultures,
languages and sub-racial traits, such as hair and eye colour or skin
tone. Human behaviour is an odd mix.  They can be superstitious and
distrustful of the unknown, but they are also insatiably curious and
look for new knowledge.  Many also seek personal fame and fortune as
most human social structures are less rigid than those of non-humans
and a person's birth need not permanently define their place in
society.  This odd combination of attributes has lead them to become
great explorers and sailors, and they will venture boldly into
unexplored areas in search of knowledge and wealth.  Humans build
great cities and are far more welcoming of other races than most.
Outside of their own culture they are social chameleons; adept at
adapting their behaviour to match local customs.

\subsubsection{Special Abilities}

\begin{Enumerate}
\item
Humans can ingratiate themselves with strangers more readily than
other races.  A human character has +10 to any reaction roll in an
encounter with sentient creatures.
\end{Enumerate}
\begin{tabularx}{\linewidth}{Xc}
\textbf{Characteristic} & \textbf{Modifier} \\
Starting Age			& 16 + \\
Average Life Span (Varies widely with wealth and culture) & 40 -- 90 years \\
Base Conception Rate		& 6\% \\
\end{tabularx}

\subsubsection{Orc}

\textbf{An Orc is a stoop-shouldered, surly humanoid and a pack
member by nature.}

\textbf{Description:} Orcs are a cruel, violent folk, liking nothing
better than to loot and pillage. Individuals test themselves against
their peers, bullying anything weaker but cowering away from anything
stronger. A strong individual will form a pack around them, and the
pack leader's word is law. Orcs enjoy the sensual pleasures of life,
and reduce their already short life spans through hard living. They
have a robust digestion and will eat foods that others turn their nose
up at. Orc warriors favour the great axe and glaive. Orcs are
considered unattractive by other humanoid races.

\subsubsection{Special Abilities}

\begin{Enumerate}
\item
An orc's eyes are highly light-sensitive. They must decrease their
chance of hitting a target with Ranged Combat by 10\% in daylight.

\item
An orc has infravision, which allows them to see faint red shapes
where living beings are located in the dark. Their range of vision is
150 feet.

\item
Orcs are either back-stabbing scum or brutal bully-boys. An orc may
take one of either Assassin Skill or Warrior Skill and pay
three-quarters the EP to advance in Ranks.

\item
An orc's seed is highly fertile. The orc and hybrid orc population
increase mitigates against the high orc fatality rate.
\end{Enumerate}

\begin{tabularx}{\linewidth}{Xc}
\textbf{Characteristic} & \textbf{Modifier} \\
Physical Strength	& + 2 \\
Endurance		& + 1 \\
Magical Aptitude	& - 2 \\
Willpower		& - 2 \\
Fatigue			& + 2 \\
Physical Beauty		& - 4 \\
Natural Armour		& + 1 \\
Starting Age		& 12 + \\
Average Life Span	& 40 -- 45 years \\
Base Conception Rate	& 10\% \\
\end{tabularx}

\subsubsection{Shape Changer}

\textbf{Shape Changers are a hidden race amongst humans, with the
ability to change into the form of a particular animal.}

\textbf{Description:} Shape Changers are identical in appearance to
humans when not in animal form. They are somewhat bestial in nature,
adopting traits one might expect from an anthropomorphised wolf,
tiger, bear or boar. There exists a love/hate relationship between
humans and shape changers: shape changers possess some degree of
animal magnetism, but, if discovered, can expect severe treatment at
the hands of humans. Shape Changers are, on the whole, bitter towards
humans, and are not above using humans to their advantage. There are
very few ways to tell a shape changer from a human (\eg they will be
discomforted by wolfbane) and these vary by shape changer type. Shape
Changers are a ruthless lot.

\subsubsection{Special Abilities}

\begin{Enumerate}
\item
A shape changer can change from human to animal form (or
\emph{vice-versa}) in 10 seconds during daytime and 5 seconds during
the night-time.

\item
A shape changer possesses a dual nature. While in animal form, human
inhibitions will be muted; and while in human form, animal instincts
will be dulled.

\item
A shape changer cannot be harmed while in animal form, unless struck
by a silvered weapon, magic or by a being with a Physical Strength
greater than 25. Five Damage Points are automatically absorbed in the
latter case.

\item
A shape changer will regenerate 1 Endurance Point every 60 seconds
while in animal form.

\item
The player must devise a set of characteristics for their animal
form. Take the difference between the average for each characteristic
in animal and human form, and modify the human characteristics
appropriately.

\item
A shape changer is automatically lunar aspected.

\item
A shape changer can remain in animal form for a quarter of the night
times the quarters of the moon showing (\ie at full moon they may
remain in animal form all night). During the day a shape changer can
remain in animal form for one hour times the quarter of the moon. A
shape changer can make one set of transformations times the quarter of
the of the moon per day (\ie dawn to next dawn).

\item
If a shape changer is in animal form during the day, there is a 1\%
cumulative chance for each 5 minutes they remain in animal form that
they will never be able to change back into human form. Similarly, if
the shape changer exceeds the time limits given above, there is a 1\%
cumulative chance (per 5 minutes) of their not being able to return to
human form.

\item
A shape changer will be inconvenienced by those wards which can be
used against were-creatures.

\item
A shape changer's magic resistance is increased by 5\%.

\item
If a shape changer takes the courtier skill they pay three-quarters
the Experience Points necessary to advance ranks.
\end{Enumerate}

A new set of characteristics must be generated (see Ability 5).

\begin{tabularx}{\linewidth}{Xc}
\textbf{Characteristic} & \textbf{Modifier} \\
Physical Beauty		& + 1 \\
Starting Age		& 16 + \\
Average Life Span	& 55 -- 65 years \\
Base Conception Rate	& 5\% \\
\end{tabularx}

\subsection{Description}

This section covers height, weight, gender, primary hand, and general
description.

\subsubsection{Height and Weight}

A player should choose their character's height and weight.

The character's height and weight should be chosen according to the
player's idea of the character, with due regard to the character's
primary characteristics, race and background.

The following chart give a range of heights and weights within which
90\% of adventurers fall, and the average values within that
range. Please modify your chosen height and weight according to gender
and racial adjustments as below.

\begin{tabular}{ccc}
\multicolumn{3}{c}{\textbf{Normal Base}} \\
\textbf{Height} & \textbf{Weight} & \textbf{Range} \\
5'3"	& 130	& 100--170 \\
5'6"	& 140	& 110--185 \\
5'9"	& 150	& 120--200 \\
6'0"	& 165	& 130--220 \\
6'3"	& 180	& 145--240 \\
\end{tabular}

\begin{tabular}{lcc}
\textbf{Adjustments} & \textbf{Height} &\textbf{Weight} \\
Human Male	& +0"	& 100\% \\
Human Female	& -4"	& 80\% \\
Orc Male	& -4"	& 110\% \\
Orc Female	& -6"	& 100\% \\
Elf Male	& +5"	& 80\% \\
Elf Female	& +2"	& 65\% \\
\end{tabular}

\begin{tabular}{ccc}
\multicolumn{3}{c}{\textbf{Short Folk Base}} \\
\textbf{Height} & \textbf{Weight} & \textbf{Range} \\
3'9"	& 85	& 65-110 \\
4'0"	& 95	& 75-125 \\
4'3"	& 105	& 85-140 \\
4'6"	& 115	& 95-155 \\
4'9"	& 125	& 105-170 \\
\end{tabular}

\begin{tabular}{lcc}
\textbf{Adjustments} & \textbf{Height} &\textbf{Weight} \\
Dwarf Male	& +0"	& 100\% \\
Dwarf Female	& -2"	& 90\% \\
Halfling Male	& -12"	& 65\% \\
Halfling Female	& -13"	& 60\% \\
\end{tabular}

\begin{tabular}{ccc}
\multicolumn{3}{c}{\textbf{Hill Giants Base}} \\
\textbf{Height} & \textbf{Weight} & \textbf{Range} \\
8'4"	& 370	& 295--490 \\
8'8"	& 420	& 335--555 \\
9'0"	& 470	& 375--625 \\
9'4"	& 525	& 420--700 \\
9'8"	& 580	& 465--780 \\
\end{tabular}

\begin{tabular}{lcc}
\textbf{Adjustments} & \textbf{Height} &\textbf{Weight} \\
Giant Male	& +0"	& 100\% \\
Giant Female	&  -4"	&  90\% \\
\end{tabular}

\subsubsection{Gender}

A player may choose whether their character is male or female. It is
recommended the character be the same gender as the player, as playing
the opposite gender convincingly is difficult.

Optionally, some characteristics may be adjusted for a female
character. This would also modify her appropriate racial maximums.

\subsubsection{Female Characteristic Modifier}

\begin{tabular}{lc}
\textbf{Characteristic} & \textbf{Modifier} \\
Physical Strength 	& -2 \\
Manual Dexterity	& +1 \\
Endurance		& +1 \\
\end{tabular}

\subsubsection{Primary Hand}

A player must determine whether their character's Primary Hand is
their right or their left. This determination affects which hand a
weapon is held during combat, and any penalties assigned for attacking
with a weapon in a secondary hand.

They may choose either right or left, or roll randomly. If they choose
to roll, the result must stand.

The player rolls D5 and D10. If the D10 result is greater, the
character's right hand is primary. If the D5 result is higher, their
left hand is primary. If the two results are equal, the character is
ambidextrous.

\subsubsection{Description}
\label{chargen:description}

The player will sometimes need to describe their character and should
therefore think about the character's physical appearance based on the
generated characteristics. They should choose hair, eye and skin
colour (based on race and family background).

\subsubsection{Conception}

Any rolls against Conception Chance should only be made once in any 48
hour period.


\subsection{Aspects}
\label{chargen:aspect}
\index{aspect!astrological}

The timing of a character's birth orients them towards one of several
astrological influences, or aspects. A character will benefit during
the time their aspect is powerful, and will suffer when the opposite
aspect is powerful.

The times of high noon and midnight are extremely important when
applying the effects of aspects. The GM should allow characters to
perform actions at precisely those instants, though the passage of
time must be properly monitored.

\subsubsection{Generating an Aspect}

The player may choose an Aspect as if they had rolled any number up to
80, or roll D100 once against the following table. If they choose to
roll on the table, any roll over 80 may be re-rolled.

If the character is joining one of the elemental colleges the player
may choose any aspect between 1 and 80 that is neutral to their
college, or they may roll.


\begin{tabular}{ll}
\textbf{Die} & \textbf{Aspect} \\
01--05		& Winter Air \\
06--10		& Winter Water \\
11--15		& Winter Earth \\
16--20		& Winter Fire \\
21--25		& Spring Air \\
26--30		& Spring Water \\
31--35		& Spring Earth \\
36--40		& Spring Fire \\
41--45		& Summer Air \\
46--50		& Summer Water \\
51--55		& Summer Earth \\
56--60		& Summer Fire \\
61--65		& Autumn Air \\
66--70		& Autumn Water \\
71--75		& Autumn Earth \\
76--80		& Autumn Fire \\
81--85		& Solar \\
86--90		& Lunar \\
91--95		& Life \\
96-00		& Death \\
\end{tabular}

\subsubsection{Effects of Aspects}

Apart from elemental aspects, all modifiers apply to percentile rolls,
not base chances.

\subsubsection{Elemental Aspects}

Characters gain a bonus of 1\% on the Base Chance of performing any
magic of the same College as their elemental aspect, and a penalty of
-1\% on the opposed College. Air opposes Earth and Fire opposes
Water. Ice magic is not affected.

\subsubsection{Seasonal Aspects}

A character is affected by their seasonal aspect during their aspect's
season and the opposite season. The following table lists the seasonal
aspect effects and when they apply.

\begin{tabularx}{\linewidth}{Xr}
\textbf{Time} & \textbf{Effect} \\
Midnight, Aspect's Season & -10 \\
Midnight, Equinox or Solstice of Aspect's Season & -25 \\
Midnight, Opposite Season & +10 \\
Midnight, Equinox or Solstice of Opposite Season & +25 \\
\end{tabularx}

The effect is applied for 30 seconds before and after midnight.

\subsubsection{Solar and Lunar Aspects}

A character of solar or lunar aspect is affected by their aspect at
high noon and midnight. The following table lists the Solar aspect
effects, and when to apply them.

\begin{tabularx}{\linewidth}{Xr}
\textbf{Time} & \textbf{Effect} \\
Noon				& -5 \\
Midnight			& +5 \\
Noon, Summer Solstice		& -25 \\
Midnight, Winter Solstice	& +25 \\
\end{tabularx}

Lunar aspected characters gain opposite bonuses and penalties for the
same times. The effect is applied for 10 seconds before and after high
noon or midnight. If the sky is cloudy, the effect may be reduced to a
minimum of +/- 1 and 5.

\subsubsection{Life and Death Aspects}

Life and Death aspected people are affected by the creation and
destruction of life force.

The following table lists the Death aspect effects, and when to apply
them.


{\small \begin{tabularx}{\linewidth}{@{\hspace{0em}}X@{\hspace{0.5em}}
c@{\hspace{0.5em}}l@{\hspace{0.5em}}r@{\hspace{0em}}}
 & \textbf{Range} & & \\
 & \textbf{is less} & & \\
\textbf{Event}		& \textbf{than} & \textbf{Aspect} & \textbf{Effect} \\
Birth of mammal		& 100'	& Death	& +5 \\
 & & &  \\
Birth of humanoid & 250' & Death & +10 \\
 & & &  \\
Birth of close relative\dag & 500' & Death & +25 \\
 & & &  \\
Death of mammal	& 50' & Death & -5 \\
 & & &  \\
Death of humanoid & 125' & Death & -10 \\
 & & &  \\
Death of close relative\dag & 250' & Death & -25 \\
\end{tabularx}}

\dag The close relative can be no more distant than second cousin.

Life aspected characters gain opposite bonuses and penalties for the
same times. Deaths are non-cumulative (only one can be in effect at a
given time), though births are cumulative. A stillbirth does not
affect a life or death aspected character. A resurrection is treated
as a birth.

A death event is applied for as many seconds as the effect range in
feet. A life event is applied for 3 times as long.

A female life aspected character will suffer no pain after giving
birth, and will be as healthy and active as she was before she became
pregnant.

\subsubsection{Light and Dark Aspects}

All living creatures have an additional celestial Light or Dark
Aspect.  This is fully explained in an addendum to the College of
Celestial Magics (\S\ref{celestial:aspect}).

\subsection{Heritage}
\label{chargen:heritage}

This section is relevant to humans, primarily from the Baronies, and
should be adapted for other races or regions.

\subsubsection{Social Status}

The ``social status'' is that of the character's \emph{parent(s)},
usually the father.  The table does not represent the population,
merely the proportion of backgrounds from which accepted Adventurer's
Guild applicants originally come. Most social classes are present in a
variety of environments (city; town/village; rural; court;
castle/stronghold; maritime).  A player may choose any social category
in the 01-80 range for the character's background, or roll; however
any such dice-roll must be accepted.  In general, the higher the
number rolled, the higher the social status within each band.  A roll
of 40-90 optionally may, but need not necessarily, indicate a
respectably retired ex-adventurer.


\begin{tabularx}{\linewidth}{lX}
\textbf{Die} & \textbf{Social Status} \\
01--14	& Trash / Criminal \\
15--20	& Bonded \\
21--29	& Skilled retainer\\
30--40	& Goodman \\
45--54	& Master \\
55--70	& Military \\
71--84	& Gentry \\
85--94	& Lesser Nobility \\
95--98	& Merchant-prince \\
99--00	& Greater Nobility \\
\end{tabularx}

\subsubsection{Explanation of Classes}

\begin{Description}

\item[Trash/Criminal] No legitimate employment.

\begin{example}
Thug; body-snatcher; bandit; pirate; beggar.
\end{example}

\item[Bonded] There is no slavery in the Baronies. This is the next
best thing: enforced servitude to one master for a long period [up to
life], through birth, contract, or debt.

\begin{example}
Serf; villain; unskilled or semi-skilled servant; labourer;
indentured apprentice or journeyman in a craft or trade guild;
dependent artisan contracted to a master; lay member of an accepted
religious community; ordinary soldier or sailor.
\end{example}

\item[Skilled Retainer] Voluntarily employed person, physically and
legally capable of seeking a position elsewhere.  Owns the tools of
the trade and has other, limited possessions.  Usually works under the
direction of a goodman or master.  Occasionally an itinerant artisan
of low status.

\begin{example}
Clerk; court musician; religious acolyte; freeborn shepherd or farm
hand; merchant's assistant; family chaplain; tinker; fisher.
\end{example}

\item[Goodman {[}Goodwife, Goody{]}] Head of a household;
more possessions and commitments than a mere retainer; comparatively
independent.  Usually leases or owns a smallholding (if in the
countryside) or a few rooms (if in a town). Much contact with social
peers and superiors. Often employs skilled retainers.  Includes
itinerant professionals and artisans of high status.

\begin{example}
Miller; pilot; established artisan, minor trader, innkeeper;
accredited witch; priest in an accepted temple; shop-owner; poor
freeman-farmer; forester; gamekeeper; itinerant or privately employed
alchemist, healer, magician or blacksmith.
\end{example}

\item[Master: {[}Mistress, Mother{]}] Like a goodman, but with a
larger establishment, more employees, more commitments to subordinates
and equals.  Tied to one place as direct contact with, and obligations
to, social superiors and Guilds may make impolitic any relocation or
other changes in social conduct, despite theoretical liberties and
rights.

\begin{example}
Guildmaster of a smaller craft/trading guild, or councillor in a more
powerful one; wealthy freeman farmer; professional (alchemist, healer
etc) trading publicly, with own shop and apprentices; Alphonse the
famous chef; a Ducal Kapellmeister; high-priest of an accepted temple;
captain-owner of a trade ship; mayor of a medium town.
\end{example}

\item[Military] A socially sanctioned, trained fighter or skilled
ancillary. This includes sergeants and low-born lesser officers
(lieutenants, etc); high-ranking officers are ex officio gentry.

\begin{example}
Town guardsman; skilled scout or military spy; army blacksmith;
(legal) mercenary captain.
\end{example}

\item[Gentry] By birth or service entitled to a coat of arms;
significant social or military duties. There are often many social
gradations of gentry not comprehensible to persons outside that
class. Often possesses an estate or ``independent means'' but is not of
lordly rank; such persons may, technically, be employed (but usually
to a lord, or in service to their country). May have difficulty
ensuring all children have an acceptable start in society (especially
in larger families).

\begin{example}
Knight; country squire; beneficed parson; port-reeve; courtier of
significance; respected \& influential magician; judicial officer of a
town or district; tax farmer; non-noble army or navy officer
(generally Captain \& up); cadet member of a noble family.
\end{example}

\item[Lesser Noble] Of lordly rank. Similar to the gentry, but
definitely a cut above. Normally owes feudal service to, or through, a
greater noble.

\begin{example}
Non-independent Baron; Lord Admiral of a small navy; General; ordinary
Abbot or other Head of an established, accepted, religious house;
former gentry ennobled for extraordinary or personal services to a
great noble or royalty.
\end{example}

\item[Merchant-Prince] Extremely wealthy city-based merchant, head of
an extended trade/family.  Controls a nationally significant
trade-empire and / or monopoly. Has significant power in the local
guilds. Extensive resources (especially in his/her home city), with
contacts and enemies in several countries. Capable of ordering actions
deemed criminal in less influential personages. On a roll of 98, the
family head is the character's parent; on 95--97, the head is a little
more distant (perhaps uncle or cousin).

\begin{example}
Owner of a trade-fleet; trader with a national monopoly on a
commodity (\eg silk, wine), Guildmaster of a powerful guild.
\end{example}

\item[Greater Noble] Ruler of a minor country, or head of one of the
``Great'' families in a larger country.  Will have several estates and
titles. Usually has subordinates of lordly rank. Children may have
courtesy titles.

\begin{example}
An independent Baron; Marshall of a Duke's or independent Count's
armies; Bishop; Abbot of a mother-abbey; Marshall or vicar-general of
a powerful order; Count within a duchy; Lord Admiral of a maritime
nation.
\end{example}

Greater Noble and Merchant-prince families impact seriously on the
campaign; the generating GM may need time to consult with other GMs
before the character's background is finalised.  Characters who wish
to retain an acknowledged, good social-standing may have to devote
time and money to maintain their position by indulging in appropriate
behaviour - noblesse oblige.
\end{Description}

\subsubsection{Birth Order}

Players should now choose their birth order, or roll on the following
table. Note that it is unlikely that an heir will go adventuring (at
least not without active encouragement from the next-in-line).

\begin{tabular}{cl}
\textbf{Die} & \textbf{Birth Order} \\
1	& 1st or 2nd \\
2--3	& 3rd \\
4	& 4th \\
5	& 5th \\
6	& 6th \\
7--8	& 7th \\
9	& 8th or later \\
0	& bastard \\
\end{tabular}

\subsubsection{Disinheritance}

Beginning characters never start with an estate, magic possessions, or
other ``real'' wealth.  For game reasons, characters seldom inherit
while actively adventuring. Most classes will happily pass over an
adventurer in favour of more deserving and capable stay-at-home
siblings.  If the heir or heiress can not be passed over (\eg a noble
estate) and the player does not wish to retire the character, a
trusted kinsman or tenant must be appointed as trustee or warder, to
administer and enjoy the inheritance until it is reclaimed.

A noble or wealthy parent may disown adventuring children --- either
through disfavour, or for mutual protection.  A beginning character
doesn't want to be set upon by family enemies; and no parent wants the
social stigma of refusing to pay a ransom. The guild fully supports
such characters adventuring under an alias, just as it also supports
gifted adventurers who fled legal restraints in order to join the
guild (\eg a runaway serf turned mage).  Both classes do have the
obligation not to expose their fellow adventurers to unnecessary risks
arising from their backgrounds.

In most cases, achievement begets amnesty.  A serf who has spent a
year and a day in a town becomes a freeman; a now wealthy prodigal is
welcomed back into the family fold.

\subsection{Starting Abilities and Possessions}
\label{chargen:abilities}

This section covers abilities and possessions a character has prior to
starting life as an adventurer. None of the experience points awarded
in this section are adjusted by any racial experience modifiers but
the player must use their character's race and heritage as a guideline
to the allocation or choices they make. Except where noted, the normal
acquisition and ranking rules apply to the spending of experience
points. This section must be started after all other sections are
completed, and each sub-section must be completed in order.

\subsubsection{Language Skills}

Every character knows their native language, the Alusian trade
language (Common) and possibly another language. A Guild member will
be literate in at least one language and literacy is required to learn
magic.

The player should get the GM's assistance to determine what their
character's native language is and then choose one of the following
options for their starting language skills:

\begin{Description}
\item[Option A] Rank 8 and literate in either native language or
common; Rank 6 in the other of native language or common; Rank 4 in
any other common language.

\item[Option B] Rank 8 and literate in either native language or
common; Rank 7 and literate in the other of native language or common;
Rank 1 in any other language.

\item[Option C] Rank 9 and literate in either native language or
common; Rank 6 and literate in the other of native language or common.
\end{Description}

\subsubsection{Adventuring Skills}

A character starts with Rank 0 in the Adventuring skills of
Horsemanship, Stealth, Climbing and Swimming. The player now receives
1250 experience points that may be spent on improving these
skills. Any experience points left over are lost. They also gain Rank
0 Flying, but may not raise it at any stage during Character
Generation.

\subsubsection{Mage or Non Mage?}

The player must decide whether the character will be a magic user or
not. (This choice can be made at any time during character
generation).

\subsubsection{Mage}

If the character is to be a magic user then the player must choose a
college of magic for the character to belong to. Remember that there
is a minimum Magical Aptitude requirement for each college.

\begin{tabular}{lc}
\textbf{College}	& \textbf{MA} \\
Naming Incantation	& 1 \\
Mind			& 11 \\
Fire			& 12 \\
Air			& 13 \\
Ice			& 13 \\
Illusion		& 13 \\
Celestial		& 14 \\
Earth			& 15 \\
Bardic			& 16 \\
E \& E			& 16 \\
Necromancy		& 16 \\
Binding \& Animating	& 17 \\
Water			& 18 \\
Witchcraft		& 18 \\
(Rune)			& 18 \\
\end{tabular}

The character now receives all of the general knowledge abilities of
their college including talents, general knowledge spells, general
knowledge rituals, both counterspells, the purification ritual and
ritual spell preparation.

The player should list these on their character sheet.

\subsubsection{Non Mage}

If a player decides that their character will not be a magic user then
they receive 6500 experience points to be spent in the following
order:

\begin{Enumerate}
\item
2500 must be spent on either 1 point of Fatigue or 3 points of
Perception.

\item
The character must acquire one new skill at rank 2, and may acquire a
second new skill at no more than rank 1. The Warrior skill may not be
chosen at this time.

\item
The character must acquire one weapon at rank 2, and may acquire up to
two weapons at no more than rank 1.

\item
The player may save up to 500 points to spend later.

\end{Enumerate}

The player must spend any remaining points on any of:
\begin{Itemize}
\item
1 rank in any known adventuring skills

\item
more ranks in any known languages

\item
more perception.
\end{Itemize}

Any remaining points (other than the permissible 500) are lost.

\subsubsection{Background Experience}

A character now chooses any one Artisan skill at Rank 0. This reflects
knowledge gained through childhood and must be appropriate to their
family background.

They also receive 250 Experience Points which, together with any left
over from the non-mage generation, can be spent freely.

At this time the character may acquire any one new skill at Rank 0 for
the cost of only 100 EP (rather than the usual cost).

If there is any EP remaining it may be saved for spending later in the
game.

\subsubsection{Background Possessions}

The character will have two sets of clothing of a quality appropriate
to their family background. They also have goods up to the value of
500 sp which may be chosen from the Basic Price List in the Players
Guide. Up to 50 sp may be saved as cash.

\subsubsection{Modified Agility and Manual Dexterity}

The player should calculate any agility modifiers from the weight of
their possessions and any armour penalties; see the Encumbrance Table
(\S\ref{tables:encumbrance}) for values. They should then calculate
their modified TMR from this value, see the table (\S\ref{tables:tmr})
for values. If the character uses a shield, they should modify their
Manual Dexterity as well.

\subsubsection{Finishing the Character}

The player must choose a name for their character.

They should enter every piece of relevant data onto their Character
Sheet, and calculate base chances and other variables. The generating
GM will check it, and then sign \& date it as complete.

\index{Character Generation|)}

\end{multicols*}
