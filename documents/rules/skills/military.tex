\begin{skill}{Military Scientist}{2.0}{milsci}

A military scientist can capably lead an increasing number of troops
as they improve their skill. They can prevent their troops from
fleeing after they have gained their confidence. The main ability of a
military scientist is to anticipate and react to enemy manoeuvres
quickly because of their knowledge of tactics.

\subsection{Restrictions}

A military scientist must be able to read and write in at least one
language at Rank 6 or above if they wish to advance beyond Rank 2.

\subsection{Specialised Fields}

This skill has a number of specialised fields. One is gained at each
of Rank 0, 3, 6, 8, and 10. Once a character has achieved Rank 10 they
may learn extra fields at a cost of 1500 EP and 4 weeks of training,
each. The fields are:

\begin{Description}
\item[Aerial]
planning for or against magically or naturally flying troops.

\item[Battlefield]
formulating and implementing battlefield level tactics, involving from
hundreds to tens of thousands of troops.

\item[Logistics]
the ability to organise and control a military organisation.

\item[Naval]
tactics involving from one ship up to fleet actions.

\item[Siege]
conducting or defending against siege actions.

\item[Skirmish]
tactics involving from one to fifty troops, includes guerrilla and
resistance tactics, and operating behind enemy lines.

\item[Strategy]
overall campaign level command of a military force.

\end{Description}

\subsection{Benefits}

\subsubsection{Command}

A military scientist may control a much larger number of subordinates
than is possible with most skills. Also, a military scientist's
subordinates need not be practising this skill, nor need all be using
the same skill. A military scientist could thus command a mechanician,
who was in turn in charge of building siege engines, and a healer who
was supervising other healers and teams of stretcher-bearers.  Any
subordinate may be replaced by a unit of labourers or soldiers. A
military scientist may have up to (WP / 2 [+ 1 / Rank]) subordinates.
A military scientist with the Battlefield specialisation may have up
to (WP + 2 / Rank) subordinates.

\subsubsection{Personal guard}

After drilling for (12 - Rank) months, or being in combat situations
for a like number of weeks, a military scientist may form a personal
guard of (WP + 5 + [2 \x Rank]) troops.  These troops will be
steadfastly loyal to the military scientist.  The military scientist
gains a (2 \x Rank)\% bonus when attempting to command, rally, etc.
their personal guard.  A personal guard will automatically follow all
rational commands from the military scientist in all but the most
stressful situations. In addition, a personal guard may be commanded
as a single unit, replacing only one subordinate, even if there are
more than 10 individuals in the guard.

\subsubsection{Rally troops}

A Military Scientist may attempt to rally fleeing troops that have
been fleeing for less than 30 (+ 5 / Rank) seconds.  The military
scientist must declare how many troops are being rallied during one
pulse.  If the military scientist is on the Tactical Display, a rally
attempt requires a Pass Action.  The Base Chance of rallying is (2 \x
WP) + (10 / Rank) - number of troops to be rallied. If the roll is
within the Base Chance the troops rally, and will begin to follow
orders again; if the roll is greater than the Base Chance the troops
continue to flee.  The chance of a being rallying is decreased by 25\%
for each time after the first that it has broken during the battle.
Because of this it is possible for a successful rally attempt to
affect only some of the fleeing troops.

\subsubsection{Raise morale}

A military scientist may temporarily increase each of their direct
subordinates' WP values by (Rank / 2) round down, provided that the
military scientist takes a Pass action every second pulse.  To use
this ability, the military scientist may not be engaged, stunned, or
otherwise incapacitated.

\subsubsection{Perceive tactics}

A military scientist may be able to perceive the tactics being
employed by the enemy as they are put into use, but before they come
to fruition.  To use this ability, the military scientist must be
unengaged, in a position to see the majority of the combat, and the
combat must be of a type with which they are specialized. In addition,
if the Combat is on the Tactical Display, the military scientist must
take a Pass action to implement this ability.  The Base Chance of
Perceiving Tactics is PC (+ 7 / Rank).  The GM rolls D100; if the roll
is within the Base Chance, the GM informs the player of the enemy's
plan in general terms.  If the roll is greater than the Base Chance
but less than twice the Base Chance, the military scientist is unsure
of the enemy plan.  If the roll is greater than twice the Base Chance,
the GM should mislead the player as to the enemy's plan, with the
information becoming completely false as the roll approaches 100.

\subsubsection{Initiative}
\index{initiative!military scientist}

If a group involved in combat on the Tactical Display are led in
combat by a military scientist with the Skirmish field, the Military
Scientist may add (2 \x Rank), minimum 1, to the group's initiative die
roll, provided that they are not stunned or otherwise incapacitated,
or engaged in melee or close combat.

\subsubsection{Time out}

If a group involved in combat on the Tactical Display have a military
scientist with the Skirmish field leading them, they may have more
time to plan their actions between rounds of combat.  The Military
Scientist may request a break period of up to 20 seconds (+ 10 / Rank)
between each and every pulse, in which to plan their actions and those
of their companions.  This time simulates the orders and pre-arranged
battle plans of the military scientist. The players may speak with the
military scientist, and with each other, but should limit their
conversation to the matters at hand. Only the military scientist
leading the group in combat may use this ability.

\subsubsection{Logistics}

If the military scientist learns the Logistics field, they gain
knowledge of logistics management, billeting and supplying troops,
organising foraging parties, posting watches, running patrols, and the
general day-to-day smooth running of a complex organisation. The
number of people that may be effectively controlled by one organiser
is 100 \x ([WP / 2] + Rank). This need not be an army, but could also
be an exploratory expedition, merchant caravan, etc. If the military
scientist has the Naval field they may also control the logistics for
(Rank + 1) ships.
\end{skill}
