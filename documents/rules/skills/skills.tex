\chapter{Skills}

\begin{multicols*}{3}
% \section{Introduction to Skills}
\label{skills}
\index{skills|(}
\index{skills!introduction}

A character may acquire and refine skills during a campaign.  They can
hone their talents in a series of interrelated non-magical and
quasi-magical abilities, which combine to form a single skill.  A
character's degree of talent is measured by their Rank in a skill.
They begin with the simplest abilities at the lowest Ranks, and gain
the more difficult ones as they progress through the Ranks.  Their
percentage chance of successfully performing tasks associated with a
skill will increase as their Rank becomes higher.

The possession of a skill does not necessary imply any character
traits associated with that skill.

\subsection{Acquiring and Using Skills}
\index{skills!acquiring}
\index{skills!usage}

The rudiments of a skill are learned by dint of hard practice and
diligent study.  A character must spend a good deal of time and effort
before they can use a skill at novice level (Rank 0). The character's
ability with a skill can improve only if they continue to work with it
during and between adventures.

\subsubsection{Any skill may be acquired at Rank 0 at a variable cost
of Experience Points and 8 weeks of game time.}
\index{skills!rank 0}

All eight weeks must fall within a period of six game months.  Time
spent on adventure may not count toward the necessary eight weeks.

\subsubsection{The method by which a character learns a skill affects the
Experience Point cost to acquire that skill or to increase the
character's Rank.}

If the character is taught by someone of greater Rank in the skill,
decrease any Experience Point cost by 10\%.  If the character learns
from a book, verbal descriptions or practices with some of equal or
lesser Rank in the skill, any Experience Point cost is unmodified.  If
the character practices with no useful outside assistance, any
Experience Point cost is increased by 25\%.  The availability of
qualified teachers, and the fees they charge the character for their
services, are left to the discretion of the GM.

\subsubsection{A character may attempt to employ a non-magical skill any
number of times during a day.}

The use of a skill does not, in and of itself, prevent a character
from using the same or any other skill immediately afterwards.
However, a character might suffer adverse fortune (for example, lose
Fatigue Points) while executing a skill, which would inhibit their
ability to act.

\subsubsection{The use of a non-magical skill is rarely automatically
successful.}

A character usually has a chance of failure when using a non-magical
skill.  Unless the ability is described as an exception to this rule,
the maximum chance to succeed with it is never greater than 90 (+
Rank)\% .  A character always fails to use an ability if the roll is
greater than the modified chance or 100 (regardless of Rank).

\subsubsection{Very few of the abilities associated with the various skills
are quasi-magical.}

The following are the only quasi-magical abilities to be found in the
skills section: Alchemist, Astrologer, Healer, Herbalist, Ranger Bump
of North.

\subsubsection{Expert Knowledge}
\index{skills!expert knowledge}

The possessor of a skill, other than an Adventuring skill, also gains
an in-depth knowledge of the field associated with their skill.  This
is equivalent to having Knowledge in that skill (see \S
\ref{skills:knowledge}).

\subsubsection{Supervision of subordinates}
\index{skills!supervision}

The possessor of a Skill, other than an Adventuring skill, is able to
supervise the work of subordinates in that Skill.  The supervisor may
instruct and supervise a number of subordinates equal to Rank.
Subordinates must be practising the same Skill as their supervisor and
may themselves be supervising underlings, thus creating a ``chain of
command''.  A subordinate may be replaced by a work-gang, consisting
of a group of up to 10 labourers, who must be working as a team, and
may not be supervising others. A character need not supervise their
maximum number of subordinates or labourers, and may themselves work,
proportional to their unused supervision capacity.

\begin{example}
A character with Rank 6 in Artisan (Carpenter), may instruct up to 6
other Carpenters or 6 work-gangs (up to 60 labourers), or some
combination thereof.  If they were supervising 2 Carpenters and 1
work-gang, they would only be using half their supervision capacity,
and could themselves work about half of the time.
\end{example}

\index{skills|)}
\end{multicols*}

