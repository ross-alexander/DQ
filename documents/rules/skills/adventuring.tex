\begin{skill}{Adventuring Skills}{1.0}{adventuring}

These skills may be ranked as with any other skill.  The only
differences are that all characters start with swimming, climbing,
stealth and horsemanship at Rank 0, and if the skill is used
conspicuously during an adventure it can be ranked once without the
need for training time, but there must be a tutor who with a similar
skill who is present to advise the character on the technique they
should employ.


\subsection{Climbing}
\label{climbing}
\label{falling}

\index{climbing}
\index{skills!climbing}

This skill allows a character to climb anything from walls to
mountains without the aid of specialised equipment, if this is at all
possible.  The Base Chance to use this skill is (4 \x MD + 8 \x Rank -
[structure height in feet / 10])\%.  A character utilising this skill
should make a roll at approximately 20' intervals, but if the climb is
especially difficult, every 10'. Note that the GM may modify the
formula in certain instances.

Various items of equipment may be used to improve a character's chance
of climbing; however they will only function if a character has
familiarity with them as follows:
\begin{Enumerate}
\item
Climbing Claws add 15\% to BC but have no use for rock climbing where
hands are more use. The additional experience cost to become familiar
with these is 1000.

\item
Rope allows the user to climb the structure making only one roll but
are only useful where ropes may be practically used.  The additional
experience cost to gain familiarity with rope use is 500.

\item
A climber suffers ([Height of fall (in feet) / 10] Squared) Endurance
Points when they fall.
\index{falling}
\end{Enumerate}


\subsection{Horsemanship}
\label{skills:horsemanship}

\index{horsemanship}
\index{skills!horsemanship}

An adventurer will use horsemanship to direct animals which they ride.
An adventurer may use their horsemanship with any animal or monster
which they would ordinarily ride (such as horses, donkeys, camels,
elephants, etc.).  Enchanted or Fantastical monsters do not
necessarily fall into this category, and the GM must make rulings
governing these situations.

The character's player will roll percentile dice whenever their
horsemanship is called into play.  A character's horsemanship is equal
to [(modified AG + WP) / 2 + Rank \x 8], round down.

The type of mount a character is riding will modify their horsemanship
as follows:

\begin{tabularx}{\linewidth}{@{}XrXr@{}}
Donkey        &  -10 & Palfrey   & +15 \\
Mustang       & -12\dag & Warhorse  & -5\dag \\
Quarterhorse  &  -10 & Camel     & -15 \\
Dire Wolf     &  -10 & Mule      &  -8 \\
Draft Horse   &   -5 & Pony      & +10 \\
Elephant  & -10 & & \\
\end{tabularx}

\dag Rating unless trained by rider; in that case, 0.

The GM should also take into account the familiarity the character has
with the individual animal type and apply modifiers thereby (\ie the
first time a character finds themself atop a camel should be worth at
least an additional - 15).

A character's horsemanship is called into play whenever they wish
their mount to perform an unusual or difficult action.  Any mount can
be directed into moving at a walking pace or even a brisk trot; an
unusual or difficult action would be to break into a gallop or charge,
jump an obstacle, etc.  During combat, horsemanship is called into
play during every Pulse to a) keep the mount controlled, b) regain
control if it is lost, and c) direct the mount to take any specific
Action.  Remember only a Warhorse can be directed to enter into Close
Combat by its rider, and all other mounts will only attack if directly
assaulted.

A successful roll will result in the mount obeying the directions of
the rider.  A roll above the modified percentage but less than the
modified percentage plus the rider's WP indicates the mount either
does nothing or continues to do whatever it was doing.  A roll above
both of these indicates the mount will either disobey the rider, buck,
attempt to throw the rider, or some other unpleasant result.  The
actual occurrence must be decided by the GM and should become worse
the farther the roll is above the modified percentage.

If the GM judges the rider has totally lost control of their mount,
the rider may take no other action until they have regained control
(presuming they manage to stay mounted).

Using horsemanship while in combat may be done in combination with any
other Action.  A trained rider receives certain abilities as they rise
in Rank:

\begin{Description}
\item[Rank 3] May use two-handed weapons 
\item[Rank 5] May fire a missle weapon while moving or cast a spell
\item[Rank 7] May use two one-handed weapons at once
\end{Description}


\subsection{Flying}
\index{flying}
\index{skills!flying}

Flying is the skill of performing aerial manoeuvres using magical
flying. As a rule aerial combat is difficult. Flying is an adventuring
skill.

A character may always take off, fly, or land in an appropriate manner
and reasonable conditions, and under such circumstances no roll is
necessary. Note that landing appropriately is not precise. The success
chance to perform a complex aerial manoeuvre with precision is (3 \x
AG + 10 \x Rank). This base chance may be modified by the following:

\begin{tabularx}{\linewidth}{lX}
0 to -50       & Environmental conditions. \\
+10 to -50     & Type of flight used. \\
0 to -m/hr     & Speed. \\
\end{tabularx}

Flying into an obstacle causes up to [D + (relative speed in miles per
hour / 10) squared] endurance damage. The nature of the obstacle may
reduce the damage. Specific grievous injuries may also be incurred
(normally C class).

Note that a speed of one mile per hour is equal to 30 yards per minute
in the chase sequence and 1.5 hexes per pulse in combat.

A trained magical flier receives certain combat abilities as they rise
in rank.

\begin{Description}
\item[Rank 3] May use two-handed weapons 
\item[Rank 5] May fire a missle weapon while moving or cast a spell
\item[Rank 7] May use two one-handed weapons at once
\end{Description}

\subsection{Knowledge (area)}
\label{skills:knowledge}
\index{skills!knowledge}

This is a skill that can be taken many times --- once for each area of
knowledge. A character with this skill knows most of the common lore
and traditions concerning their chosen area. An area may include: a
particular city or territory, a culture, an historical period, or a
race, or species. In addition, an area of knowledge may be taken from
the Philosopher skill. If this is done, the area is equivalent in size
to a Sub-field, and any Sub-fields except Advanced, Experimental or
Ancient are available as areas of knowledge.

A character is limited to the knowledge available to their
culture. The knowledge held by the character may not be entirely
factual, and may contain certain popular misconceptions or
superstitions. This skill mostly gives the character a much wider
general knowledge about their area, some history of it, and perhaps
some biographical knowledge of famous figures associated with it, both
historical and contemporary. This skill is entirely one of knowledge,
and confers no special ability to perform a craft or trade.

Generally there is no success percentage, the GM simply giving far
more information regarding a certain topic to a character who has
knowledge of that area.  If there is doubt as to whether or not a
character should know something from their specific area, the Base
Chances are:

\begin{tabular}{lc}
\textbf{Rarity of Information}	& \textbf{Base Chance} \\
Common			& WP + 70\% \\
Uncommon		& WP + 40\% \\
Rare or Obscure		& WP + 10\% \\
\end{tabular}

These chances may be further modified by the GM to reflect the
individual rarity of the knowledge. A character will not know the
theories behind the lore.

If a character learns an area of Knowledge that is also a Philosopher
Sub-field, and that character is, or becomes a Philosopher, the
Knowledge (area) may be used as the appropriate Sub-field.  See the EP
cost table note A (\S\ref{tables:ep}) for details on Ranking.


\subsection{Orienteering}
\label{skills:orienteering}

\index{orienteering}
\index{skills!orienteering}

The adventuring skill of orienteering no longer exists.  Instead,
artisan Cartographer (\S\ref{artisan}) exists.  A Cartographer may use
Landmarks and Read \& Draw Maps as per a Navigator. They may not use
Charts or Rutters.

\subsubsection{Conversion}

Characters with the current Orienteering skill have spent precisely
the same amount of EP as a Rank 5 Artisan. They will gain Rank 5
Cartography. This normally takes 23 weeks, but they may spend only 10
weeks to convert the skill. There is no requirement as to when they
have to spend the weeks. If they partially complete the ten weeks
training, they may act as an Orienteer equal to half the weeks already
spent training, until the training is complete. A character with
Orienteering and Cartography will have 2,500ep and 13 weeks towards
further ranks in Cartography.


\subsection{Stealth}
\label{stealth}
\index{stealth}

An adventurer can use stealth to move as soundlessly and unobtrusively
as possible.

An adventurer may use their stealth ability only if they have adequate
cover (\ie space in which to conceal or obscure themselves) in the
area they wish to traverse, they are appropriately clad (\eg not in
plate armour or luminescent clothing), and they are not currently
under observation by the being(s) from whom they are attempting to
conceal their presence.

The GM will roll percentile dice to determine if a character is able
to use their stealth ability successfully.  The GM only makes such a
check if there is a reasonable possibility that the character could be
detected.  The GM makes one check each time the character attempts one
continuous action, or each time an unexpected change of condition has
a significant effect upon the character's chance of remaining hidden
(\eg one of the beings under surveillance heads for a room which
happens to be through the doorway in which the character is hidden).
The GM may modify the success percentage.

A character's base chance of using their stealth ability is (3 \x
Agility + 5 \x Rank + Thief Rank + 2 \x Spy Rank + 2 \x Assassin
Rank)\%.  The greatest Perception value of the beings who may be able
to discover the character using the stealth ability is subtracted if
those beings are unaware of the character's presence, or three times
that Perception value if they are.


\subsection{Swimming}
\label{swimming}
\index{swimming}
\index{skills!swimming}

\subsubsection{Introduction}
This skill is required in order to perform any actions in the water.
All player characters start off with Rank 0. This, under good
conditions, will allow the character to tread water in order to stay
afloat. The higher the rank, the more the character will be able to do
until they are at the stage where they can swim like a fish and
survive even in adverse conditions.

\subsubsection{Base chance}

The base chance for swimming is: PS + AG + EN + 8 \x Rank and is
modified by the following (all adjustments cumulative)

\begin{tabularx}{\linewidth}{Xr}
Wearing no or little clothing		& +10 \\
Encumbered (per pound)			& -1 \\
\\
Water Temperature			& +5 to -25 \\
Water Conditions			& +10 to -25 \\
May not swim freely			& -10 to -50 \\
\end{tabularx}

Other modifiers may be applied by GM as appropriate.  An unsuccessful
skill roll does not imply drowning (yet) but the character could be in
serious trouble. However if they are trying to float and the roll is
failed then they need to make another successful skill roll in order
to stay afloat.  Two failed skill rolls does imply they are
underwater, holding their breath, without preparation.

If an Adept is attempting to cast then they can do so, within the
restrictions of their College, if breathing water or if they make a
successful skill roll. A concentration check (3 \x WP) may also be
required in adverse conditions.

\subsubsection{Breath holding}

The base time a character can hold their breath is (current EN / 3 +
swimming Rank / 2) pulses rounded up. The time is doubled if a Pass
Action is used in the previous pulse to prepare.

\subsubsection{Drowning}

Once that time is expired then the character must make a 5 \x WP check
in order to continue holding their breath. At the end of subsequent
pulses, the WP factor is reduced by 1 until the roll fails.

At that point the character starts drowning, taking physical damage at
a rate of D10 EN per pulse until death or rescue. A drowning character
needs to make a 2 \x (WP + swimming rank) check before being able to
perform useful activity as above.

\subsubsection{Sight and Communication}

The character can see PC hexes in clear water. This is halved in lakes
and rivers because of algae and silt.

Communication is by sign language or a range of one hex if speaking.

\subsubsection{Movement rates}

Swimming TMR = (Land TMR + Rank) / 3. Walking on the bottom (if
weighted) = Land TMR / 3.  Swimming is generally a hard or strenuous
activity unless the entity concerned is an aquatic.

Characters that are encumbered by non-buoyant materials descend at the
following rates:
\begin{tabular}{lr}
Unencumbered to 5 lbs	& 0 ft per pulse \\
5--10 lbs encumberance	& 1 ft per pulse \\
10--15 lbs		& 2 ft per pulse \\
15--20 lbs		& 3 ft per pulse \\
20--25 lbs		& 4 ft per pulse \\
25+			& 5 ft per pulse \\
\end{tabular}

Unencumbered characters floating to the surface (\eg if unconscious) do so
at 1 ft per pulse.

\subsubsection{Benefits of Rank}

A trained swimmer receives certain combat abilities as they rise in
rank.

\end{skill}
