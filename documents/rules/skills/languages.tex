\begin{skill}{Languages}{2.0}{languages}

The campaign has many languages. Each sentient race usually has one
intrinsic to itself, or more if that race is split into various
populations. There is no universal language, but Common is the first
language of several nations.

\subsection{Restrictions}

A language may not be known above its maximum rank. Characters may not
speak a tongue for which they do not have the vocal apparatus.
Characters may not learn a language without instruction from a source
of at least the same rank as that being learnt.

\subsection{Structure}

\begin{Description}

\item[Family] Each Language belongs to one particular Family of
intrinsically related tongues (see \S\ref{languages:families}).

\item[Language Group] History, geography, and custom all transform
languages --- therefore languages with a common history or interaction
share the same Language Group (see \S\ref{languages:groups}). A
language may belong to several groups, and a group may link languages
from different Families.

\end{Description}

Learning a language is easier if one already knows a related or similar
language at a higher rank. The EP discount is:
\begin{Itemize}

\item 20\% if in the same Family or Group;

\item 30\% if in both the same Group and the same Family. 
\end{Itemize}

\subsection{Benefits}

Languages vary in their complexity; a low maximum rank may indicates
less versatility, vocabulary, or foreignisms.

At Rank 0 in a language, you cannot speak it, but can probably sense
the general mood of plain statements: a threat, a greeting, etc.
Thereafter, with increasing rank, one's competency and vocabulary
progressively increase, as compared to humans, using a typical human
language, to talk about every-day things in their village.

\begin{Description}

\item[Rank] Effect (\& approximate Vocabulary).

\item[1] Some of the simple, common words (2\%).

\item[2] A few simple statements (5\%).

\item[3] Common phrases, including basic directions; several tenses;
Effectively rank 0 in all other languages of that Group (20\%).

\item[4] Common idioms; more tenses; can give passable descriptions of
events or people; Effectively rank 0 in all other languages of the
same Family (70\%).

\item[5] Rarer idioms; most tenses; sufficient to use most professional skills (90\%).

\item[6] Normal, every-day fluency \& usage; can give clear \&
accurate descriptions of events or people; Effectively rank 1 in all
other languages of that Group (100\%).

\item[7] Courtly or professional speaker (120+\%).

\item[8] Can express any conceivable thought; may cast college magic;
Effectively rank 1 in all others of the same Family (200+\%).

\item[9] Effectively rank 2 in all other languages of that Group (400+\%).

\item[10] Maximum mastery of the language (500+\%).

\end{Description}

Note that some languages are very limited. For example, many concepts
or emotions cannot be articulated in Troll.

\subsection{Literacy}

Literacy in a language is distinct from the skill of speaking. It is
easily learnt if the written form is alphabetic. Most cultures have a
large proportion of the population that is illiterate.

Not all languages have a written form. It is not possible to attain
literacy in a language that does not have an established written
form. One may attempt to transcribe that language, adapting a known
script, but the ``writing'' produced is ineffectual for communicating
with others.

\subsubsection{Phonetic Reading \& Writing}

Most Alusian languages are written using a phonetic alphabet: a set of
signs representing, one-to-one, all the sounds of that language.
Historically, a recently literate language usually re-uses an
established alphabet with minor variations. Therefore there are many
languages, but few alphabets.

For each alphabet, the cost is 1000 EP and 4 weeks the first time you
learn literacy using it; literacy in a subsequent language, using the
same alphabet, is only 500 EP and 2 weeks. Sometimes, for different
cultures, one language is written in different phonetic alphabets. If
so, you must pay the time and EP for each one you learn.

\subsubsection{Table of known Alusian alphabets}
\label{languages:scripts}
\begin{Description}
\item[b] Bedouin script (human, flowing, cursive).

\item[d] Drakonic.

\item[e] Elvish script.

\item[i] Island (used near the land-locked ocean).

\item[k] Kingdom (used near  the Azurian Empire).

\item[n] Nagan (elaborate, but versatile).

\item[o] Ogham (human, rune-like).

\item[r] Dwarvish runes.

\item[w] Westron (usual Western human alphabet; also adopted by many
newly literate societies).


\end{Description}

\subsubsection{Orthographic languages}

A literate language not using a truly phonetic alphabet is
orthographic --- \eg it uses pictograms, or an elaborate spelling
structure. The written form is so complex that it must be learned as
if it were, in effect, another language of the same language family
(\eg written and spoken Erehleine are treated as two separate members
of the Eldar Family). Hence one often speaks and writes an
orthographic language at different ranks. Orthographic languages are
indicated in \S\ref{languages:families} by an asterisk (*). Each
orthographic system is functionally unique to its particular language.

\subsection{Special rules}

\begin{Description}

\item[Common] is easily learnt. Knowledge of any other language at a
higher rank gives a 50\% EP discount.

\item[Accent] Every speaker has an accent which reflects a mixture of
their native language and the tutors from whom one learnt the
language.  At Rank 6 or higher, any speaker may gain a particular
accent by spending 500 EP and 1--3 weeks study or being tutored (the
GM decrees how much time is necessary).

\item[Unpronounceable Tongues] all languages of the Dragon Family
(except Saurime) require unusual vocal apparatus. No humanoid race may
normally speak these tongues. However, you may rank the language at
twice normal cost, to gain comprehension. Alphabetic literacy in an
unpronounceable language costs 2000 EP and 8 weeks. If you do have the
physiological or magical ability to speak such languages, you may rank
them without penalty.

\item[Immersion] If character spends a number of weeks listening to a
particular language being spoken daily and frequently by speakers who
use it at a rank higher than the character knows it, the GM may allow
that character to use those weeks as ranking time for that language in
addition to any other activity undertaken \eg going on adventure,
other training, etc. The EP must still be paid. A character may only
rank one language by immersion at any one time.

\item[New languages] when a new language is introduced into the
campaign, the GM concerned must determine the following:
\begin{Enumerate}
\item Its Family and any language groups. 

\item Whether it has a written form --- and, if so, is it phonetic or
orthographic? If it is alphabetic, what alphabet is used?

\item Its maximum rank. 
\end{Enumerate}
\end{Description}


\subsection{Language Families}
\label{languages:families}


The figure in [ ] represent the maximum rank that can be achieved with
the language; the letter(s) represent the phonetic alphabet(s) used,
and * identifies orthographic languages. If no letter or asterisk is
given, the language does not have an established written form.

\begin{Description}

\item[Common] Common [9i,k,w].

\item[Western-Human] Alman[9o,w], Brett\-[9o,e], Des\-tin\-ian\-[8w],
Ebolan\-[9w], Folk\-sprach\-[9w], Lalange\-[10w], Rani\-terran\-[9e],
Reich\-spiel\-[9w].

\item[Central-Human] Arabiq[9b], Draknbrger[9w], Ellenic[10i],
Kravonian[9*], Panjari[9*], Pasifikan[8], Sanddweller[9e],
Sea-of-Grass[9], Themiskryan [9i,*].

\item[Eastern-Human] Five-Sisters-Courtly[10*],
Five-Sisters-Trader[9*], Lunar-Empire[9*].

\item[Merfolk] [?,?].

\item[Eldar] Drow\-[9e], Eldaran\-[10d], Elv\-ish\-[10e], Ereh\-leine\-[10*],
Frog\-elf\-[8*], Purple-Drow\-[9e], Purple-Old-Drow\-[9e], Quenchan\-[10*],
Terra\-novan-Drow\-[9*].

\item[Faerie] Brownie[7], Centaur[9i], Dryad[6], Fossegrim[6],
Leprechaun[6], Nixie[6], Nymph[7], Pixie[7], Satyr[7], Sylphine[6].

\item[False-Fey] Doppel\-ganger[8], Gargoyle[6], Harpy\-[7], Medusa[6].

\item[Earth-Dweller] Gnomish[9r], Goblin[8w], Half\-ling\-[9r],
Hob\-goblin[8w], Khuzdul[9r], Kobold[8], Dwarv\-ish[9r], Ogre[6w],
Orcish[9w], Sasquatch[3], Troll[4], Yeti[3].

\item[Giant] Cloud[9w], Fire[9w], Frost[9w], Hill[8w], Stone[8w],
Storm[9w], Titan[10i].

\item[Dragon] Culhuan\-[10*], Draconic\-[10d], Nagan[10n],
Old-High-Draconic\-[10d], Saurime\-[7d], Wyvern\-[4].

\item[Signing] Silent-Tongue[6], Bandito [5].

\end{Description}

\subsection{Language groups}
\label{languages:groups}

\begin{Description}

\item[Archaic] Eldaran, Purple-Old-Drow, Quenchan.

\item[Draconic] Draconic, Nagan, Old-Draconic, Wyvern.

\item[Dravidic] Drow, Raniterran, Sanddweller.

\item[Dwarvic] Dwarvish, Gnomish, Halfling, Khuzdul.

\item[Dwarvidic] Alman, Brett, Ebolan, Folksprach, Reichspiel.

\item[Ellenic] Centaur, Ellenic.

\item[Elvic] Drow, Eldaran, Elvish, Erehleine, Purple-Drow, Terranovan-Drow.

\item[Elvidic] Elvish, Lalange.

\item[Gnomic] Fossegrim, Gnomish.

\item[Herpetic] Culhuan, Saurime.

\item[Littoral] Destinian, Ebolan.

\item[Low Gigantic] Hill-Giant, Ogre, Stone-Giant.

\item[Nomadic] Draknbrger, Kravonian, Sea-of-Grass.

\item[Orcal] Goblin, Hobgoblin, Kobold, Ogre, Orcish.

\item[Panic] Centaur, Dryad, Nymph, Satyr, Sylphine.

\item[Perfidic] Fossegrim, Nixie, Pixie.

\item[Protonic] Eldaran, Old-Draconic, Draconic.

\item[Purpuric] Purple-Drow, Purple-Old-Drow.

\item[Rustic] Brownie, Leprechaun.

\item[Titanic] Cloud-Giant, Storm-Giant, Titan.

\end{Description}
\end{skill}
