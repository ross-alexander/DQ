\begin{multicols*}{3}
\section{Names \& Auras}

In the DQ universe entities have an Aura. Those objects that were once
alive retain traces of their living auras.  The strength and
composition of the Aura reflects the amount of life-force and magic
that the entity or object possesses and the other properties that are
intrinsically part of their being.

The base element of any Aura is strength. The strength of an Aura is
always revealed by any magic that detects or reads Auras. The
categories of strength are, from weakest to strongest:
\begin{Enumerate}
\item Magic (magical wall, illusion)
\item Formerly Living Composite (chair, stew)
\item Formerly Living (dead orc, log)
\item Non-Sentient Animates (stone golem, skeleton)
\item Living Plants (rose, oak)
\item Living Animals (dog, cat)
\item Sentient Animates (flesh golem, vampire)
\item Living Sentient (human, sphinx)
\item Long Living Sentient (dragon, titan, elf)
\item Avatar (material form of a Demon, etc.)
\end{Enumerate}
The strongest Aura will be detected. Thus a human covered by an
illusion (without an Aura component) will still be detectable as a
Living Sentient.

The rest of an Aura consists of information intrinsic to the
possessor. This information will vary depending upon the Generic type
of the target, but will either affect the life-force of the target, or
be magical. Only effects that are still current or continuos in nature
will be detected.  Information may be gained in descriptive terms,
values or even proportions as appropriate for the type of information
being read.

Information that may be gained from a living being includes: its
Generic True Name, its plane of origin, approximately how far it is
through its life-span (\eg juvenile, 50\%, about 100 years old), its
general state of health (\eg healthy, diseased, 1/2 Endurance),
aptitude with a magical ability (\eg low overall, Rank of specific
ability) and to which College of magic (if any) it is attuned. These
last two facts are discernible because the skills that they represent
have an affect on the level and type of magic in the entity's
Aura. Relatively little information can be divined regarding the
non-magical learned abilities of an entity. It will be possible to
learn what is the being's most intrinsic skill or ability, but lesser
skills may not be sufficiently intrinsic so as to have made an
impression on the being's aura.

Magical auras will include information such as College, exact name of
a spell or other effect, level of magical ability (low, medium, high,
very high), approximate length of time that the magic has been in
effect (providing it is still present), and approximately how much
duration remains.

Any one part of an object will be representative of the entire object,
for example the Aura of a toe sticking out from underneath a blanket
will reveal the same information as if the entire being were
visible. A detached thumb could reveal some information about its
former owner, up to the time it was detached, providing it is
intrinsic to the thumb, for example Generic True Name, plane of
origin, or age --- when it was removed.  The thumb would not reveal a
magical college or a skill, as these are properties of living beings
only.

\subsection{Detect Aura Talent}
\label{magic:detect-aura}
\index{detect aura}

\begin{talent}{Detect Aura}

\range{Special}
\multiple{75}
\basechance{Perception (\x 2 for Namers) + 5\% / Rank 
- 1\% for every foot after the first five feet the target is 
from Adept}
\resist{Active}
\target{Entity, Object, Area, Volume}
\begin{effects}
If the talent is successful the Adept learns which of the aura
categories they are seeing (with the strongest taking precedence), and
optionally learns the answer to one question of the Adept's choice
about the target. The answer to a DA question will consist of a single
concept or "bit" of information. If the information sought is not
intrinsic to the target the Adept will receive no answer. It is not
possible to determine the Individual True Name of an entity. If the
Adept achieves a double or triple effect, the Adept may ask the GM two
or three questions respectively. The process of reading an aura and
asking a question entails concentration on the part of an Adept and
requires a magical Pass action.  Re-reading a previously seen aura, or
learning the category of the aura without asking a question may be
combined with other actions, as for other talents.

Only one attempt at Detect Aura may be made per object. An individual
object will change over time, however, and a fresh attempt may be made
when the aura has changed sufficiently to class it as a ``new'' object.
If an aura has been successfully read, the same information will be
available without a new Cast Check being made, until such time that
the object changes sufficiently to be considered a ``new'' object.

Aura require direct line of sight to be read.  It is not possible to
use Detect Aura through a mirror, crystal ball, Wizard's Eye, or by
any other indirect means.
\end{effects}
\end{talent}

\subsection{Interpretation and Examples}

\begin{Enumerate}
\item
In general the more specific the question the more 
specific the answer - the exact nature is left to the GM's 
discretion. For example:
\begin{Itemize}
\item ``What was the last type of magic to impact on this person?''
might get ``A spell''.

\item ``What is the nature of the most recent magical affect on 
this perso''?" might get ``fortitude''.

\item ``What was the last spell to impact on this person?'' might 
get ``Strength of Stone''. 
\end{Itemize}

\item
All things change over time, even if outwardly they look the
same. While the times may vary from object to object depending on
circumstances, they tend to follow a seasonal cycle following the
seasons. An object whose aura was read in Spring will have changed
sufficiently by Summer to be able to be DAed again.

\item
The single attempt rule also applies to multi-hex objects. Only one DA
may be performed per item --- one cannot DA a wooden floor three times
just because it covers 3 hexes. Attempts to do so will get the
response (even before rolling the dice) - ``you see an aura, formerly
living, and the answer is oak'', that is the same aura they had
previously read until a change of season. After the change of season a
new attempt to detect the aura would have to be made.

\item
Once an aura has been detected, it is available to the detector for
the looking, as is the answer to the question/s asked, until the aura
changes. For example, having detected the Aura of a Ward, the Adept is
able to re-read the same information freely. If the Ward was then
dissipated, the aura would vanish.

\item
A DQ aura is located very close to the skin surface.  Thus a person in
a full suit of plate armour, with the visor down, and no part of the
body visible whatsoever would not be able to have their aura
read. Cloth will hide an aura, but make up will not.

\item
The size of the object will be determined by its utility. Thus 100' of
corridor could be a single object, while the next 5 feet, because it
has a wooden floor, or is magical say, may be an object. The GM may
deliberately break things up so as not to give away too much
information from the groupings chosen.
\end{Enumerate}

\subsection{True Names}

\label{names}
\index{true names}
\index{true names!generic}

\subsubsection{Generic True Names}

All living things in the DragonQuest world have a Generic True
Name. This name is present in their aura. Formerly living things
retain traces of their aura and the name that they had when alive. All
True Names are in an ancient language, believed by some to be the
language that the gods used when they made the world, and by others to
be the original language of the first mortal race, the Dragons. A
translation of these names into the common tongue yields such terms as
human, elf, tiger, oak, bee, rattlesnake, and rose.

A Generic True name identifies the entity or object as being of a
distinct type. Very similar things with much the same form and
function, will probably have the same Generic name. For example, many
small, harmless, plains-dwelling snakes will have the Generic True
name ``Grass Snake'' even if they look somewhat different. A venomous
variety of similar nature will have a different Generic True name.
Generic True Names may be learnt:
\begin{Itemize}
\item By means of a Detect Aura.

\item From another being who has studied the name (\eg knows the name
at Rank 0 or higher).
\end{Itemize}

\subsubsection{Individual True Names}

\index{true names!individual}

All sentient entities (player character races, dragons, mer-people,
naga, etc.) have an Individual True Name. This becomes known to them
upon reaching maturity. All sane sentient entities will know their own
Individual True Name and no force --- physical or magical --- can coerce
the entity to reveal it. They may choose to reveal it, however.  An
entity will be called by a given (or use) name, often given to then by
parents or peers. The entity will know their Individual True Name in
their native tongue and a Namer would have to spend time translating
the Individual Name into the Namer language before it could be used.
Entities will protect their True Name vigorously as this knowledge can
be used both defensively and offensively.  Indeed, the very mention of
an entity's True Name would strike great fear into their heart.

Player characters and even members of the College of Naming
Incantations will know only their own Individual True Name upon
completing their education. All other Individual True Names must be
acquired and learnt before they can be used.  Four methods exist for
acquiring an entity's True Name:

\begin{Itemize}
\item The entity may choose to reveal it.

\item It may be obtained from a Namer Demon --- when an entity is born,
their Individual True Name becomes known to some of the Naming Demons
(see the demon descriptions in the College of Greater Summoning).

\item It may be obtained from another other being who has studied the
True Name, if they choose to disclose it.

\item It may be found in written form --- Adepts of various Colleges
have been known to record important entity's names in magic tomes.
\end{Itemize}

The aura of the entity contains both the Generic and Individual True
Names, and so the training of the College of Namers concentrates
greatly on the study and interpretation of auras, from all living
beings and formerly-living objects. Whilst other Colleges use
abilities to detect auras, only the Namer is trained to make maximum
use of the information gained from perceiving auras. The Generic Name
is instantly identified when a Namer perceives an aura, although this
Name must still be studied and Ranked to be of use. The information is
coded into the aura in a form that Namers are trained to recognise,
but other Colleges, through use of the same Detect Aura spell /
talent, would need to inquire specifically to receive the same
information.

The Individual True Name is also coded into the aura, but is so
complex and varied that they cannot be deciphered and used by even a
Namer. If the Namer is given the Entity's True Name, then it is
possible for the Namer to identify the auric characteristics that make
up the Individual True Name. The study of these components takes
considerable time due to their complexity.

\end{multicols*}
