\begin{Chapter}{Astrologer (Ver 1.0)}

The celestial bodies have a definite, if not entirely 
understood, effect upon the lives of the inhabitants 
of a DragonQuest world. These Great Powers seem 
to  impose  predestination  upon  all  but  the  strong-
willed, and determine the aspect of each being. The 
Sun,  the  Moon,  and  the  Planets  regularly  cause 
perturbations in the flow of mana; the mighty Stars 
affect  a  world  across  the  vast  reaches  of  space  by 
their  positions  relative  to  it.  The  study  of  the  pur-
pose  and  method  of  the  Powers  is  the  science  of 
astrology. 

An  astrologer’s  main  talent  is  a  limited  ability  to 
predict and shape the future. An astrologer  will be 
able to make clear, general assertions, but will only 
be  able  to  give  obscure  clues  when  asked  for  spe-
cific details. 

34.1 Restrictions 
An astrologer must be able to read and write in one 
language at Rank 8 if they wish to advance beyond 
Rank 0. 

An  astrologer  may  not  make  a  general  predic-
tion  or  ask  a  specific  question  concerning  only 
themselves. 

One  must  consult  another  astrologer  in  these 
weighty matters. 

An  astrologer  may  only  try  once  to  answer  a 
particular question or to forecast the outcome of 
an event. 

Once  an  astrologer  has  made  a  reading  (i.e.  a  de-
termination about the future), they may not seek to 
change  or  influence  the  reading  through  their  art. 
Other  astrologers  who  attempt  to  read  the  same 
future  will  receive  the  same  information  that  the 
first astrologer did. A second astrologer may, how-
ever,  receive  some  clarification  about  the  first’s 
reading. 

The results of a reading will affect the pertinent 
course of events. 

The  GM  is  expected  to  modify  the  outcome  of  an 
adventure  or  happening  in  their  world  to  conform 
with  a  determination  made  by  an  astrologer  or  by 
an  astrologer  at  the  behest  of  a  character.  The 
determination  does  not  preclude  the  characters’ 
actions  from  affecting  the  outcome  of  the  adven-
ture  or  event:  to  the  contrary,  the  GM  must  inter-
pret  the  reading  as  they  see  fit,  and  alter  a  few  of 
the  random  dice-rolls  engendered  by  the  charac-
ters’ actions accordingly. 

A  prophecy  cannot  be  avoided  by  the  affected 
character(s)  changing 
their  plans.  The  doom 
(which  may  be  good)  will  follow  them  to  the  un-
dertaking  they  substitute  for  that  which  was  pre-
dicted.  However,  if  a  character  asks  a  specific 
question predicated upon a given action, the proph-
ecy  need  not  come  to  pass  unless  and  until  that 
action is taken. 

34.2 Benefits 

An  astrologer’s  Rank  determines  how  many 
beings  they  can  directly  affect  with  a  single 
prediction. 

A  being  is  directly  affected  by  an  astrologer’s  art 
when the GM modifies the result of an action taken 
by the being due to a prophecy. 

An  astrologer  can  directly  affect  up  to  (5  +  10  × 
Rank) beings with a single prophecy. If an astrolo-
ger  attempts  a  prediction  which  would  directly 
affect  more  beings  than  their  Rank  allows,  they 
receive no answer. 

An  astrologer  may  make  (and  possibly  modify) 
a general prediction during a reading. 

When an astrologer  wishes to make a general pre-
diction  about  a  particular  venture  or  being,  the 
player (or the GM) actually uses a divinatory tech-
nique  at  their  disposal.  Such  a  technique  could  be 
reading the tarot, casting the I Ching, or any mutu-
ally agreed upon method. 

The  result  of  the  divination  becomes  the  astrolo-
ger’s  prediction.  If  the  astrologer  does  not  wish to 
make the prediction, they may immediately attempt 
to  change  it.  The  GM  rolls  percentile  dice,  and  if 
the roll is less than or equal to (5 × Willpower + 4 
×  Rank  -  30),  the  astrologer  makes  a  second  divi-
nation  (which  may  not  be  changed).  If  the  roll  is 
greater  than  the  success  percentage,  the  astrologer 
is stuck with their first prediction. 

An  astrologer  may  seek  to  answer  up  to  Rank 
specific questions per month. 

When  a  being  poses  a  specific  question  to  an  as-
trologer willing to attempt an answer, the GM rolls 
percentile dice. If the roll is equal to or less than (6 
× Rank + 4 × Perception), the astrologer is able to 
give a correct answer. If the roll is greater than the 
success  percentage,  they  mutter  meaningless  gib-
berish. 

All answers given to specific questions must be, at 
the very least, obscure. The GM may respond with 
cryptic  poetry,  much  like  the  Oracle  at  Delphi,  or 
may  choose  to  have  the  astrologer  supply  a  riddle 
(though the Player of the astrologer does not know 
the answer themselves). 

An  astrologer  can  determine  the  aspect  of  a 
being after observing them. 

After an astrologer has spent (60 - Rank) consecu-
tive  minutes  observing  a  being,  the  GM  informs 
the astrologer of the being’s aspect. 

An  astrologer  expends  Fatigue  points  when 
practicing their art. 

Action 

Fatigue 

Make general prediction  
10 
Try to change general prediction   10 
17 
Try to answer specific question 
Determine being’s aspect  
5 
\end{Chapter}
