\begin{Chapter}{Grievous Injury Table}
\label{table:grievous}
\begin{Enumerate}
\item A class weapons do grievous injuries on rolls of 01 through 20.  
\item B class weapons do grievous injuries on rolls of 21 through 80.  
\item C class weapons do grievous injuries on rolls of 70 through 00.
\end{Enumerate}

\smallskip

\makeatletter
\patchcmd{\@item}
  {\addvspace\itemsep}
  {\par\kern\dimexpr.7\itemsep-.7\parskip-.7\baselineskip\relax%
   \hrulefill%
   \par\kern\dimexpr.3\itemsep-.3\parskip-.3\baselineskip\relax}
  {}{}%
\makeatother
  
\begin{Description}
\item[01–05] Congratulations! It’s a bleeder in your primary arm! Take
  1 Damage Point from Endurance immediately and 1 per pulse thereafter
  until the flow is staunched by a Healer of Rank 0 or above or you
  die.

\item[06–07] Oh no! Your opponent’s weapon has entered your secondary
  arm’s elbow joint and the tip has broken off. Take 2 Damage Points
  immediately from Endurance and that arm is useless until the sliver
  has been removed by a Healer of Rank 3 or above. Also, increase the
  chance of infection by 30.

\item[08] A vicious puncture wound in your groin!  Take 3 Damage
  Points immediately from Endurance and reduce your TMR by 2 until
  fully recovered, which will take two months. In addition, add 30 to
  your chance of being infected (assuming you live long enough for
  such things to matter).

\item[09–10] You have been stabbed in your secondary arm. Drop
  whatever you were holding in it and take 2 Damage Points immediately
  from Endurance. It will take a full week for the arm to be of any
  use to you whatsoever.

\item[11] Your aorta is severed and you are quite dead. Rest assured
  your companions will do their best to console your widow(er).

\item[12] A stomach puncture. Nasty. You suffer 3 Damage Points
  immediately from Endurance and lose 2 from your TMR until fully
  recovered, which will take two months. Also, you are automatically
  stunned for the next pulse (if you aren’t already), after which you
  may recover.  Add 20 to the chance to be infected.

\item[13] Your opponent’s weapon has entered your eye. Roll D10. On a
  roll of 1, the weapon has entered your brain and you are dead.  On a
  roll of 2–5, your left eye is blinded.  On a roll of 6–10, your
  right eye is blinded. If you are lucky enough to be blinded instead
  of killed, you have suffered 2 Damage Points to Endurance. In
  addition, a figure who is blind in one eye suffers the following
  subtractions: -1 from MD, -2 from PB, -4 from Perception. A figure
  blinded in one eye reduces their base chance with any missile or
  thrown weapon by 30.

\item[14–18] Tsk, tsk. A wound of the solid viscera.  Usually
  fatal. Take 3 Damage Points to Endurance immediately and 1 per pulse
  thereafter until the bleeding is stopped by a Healer of Rank 2 or
  above or you die.  Add 30 to the chance of infection.

\item[19–20] Take a stab in the leg (your choice as to which one)
  resulting in a deep puncture of the thigh muscle. Suffer 1 Damage
  Point to Endurance immediately and reduce your TMR by 1 until you
  heal, which will take 4 weeks.

\item[21–25] A chest wound. Take 2 Damage Points to Endurance
  immediately and reduce your TMR by 1 until recovered (about 2
  months). Look on the bright side, though.  Your attacker’s weapon is
  caught in your rib cage and has been wrenched from their grasp.

\item[26–27] Bad luck! Your secondary hand has been severed at the
  wrist. Take 2 Damage Points to Endurance immediately and 1 point per
  pulse from Fatigue thereafter (Endurance when Fatigue is exhausted)
  until you are dead or the bleeding is staunched by a Healer of Rank
  0 or above. If you live, reduce your MD by 2.

\item[28–30] Worst luck!  Your primary hand has been severed. See
  result 26–27 for effects.

\item[31–34] A minor wound. Your face is slashed open, ruining your
  boyish good looks and causing blood to spurt into your eyes.  Reduce
  your PB by 4 permanently.

\item[35] Your secondary arm is sliced off at the shoulder. Take 5
  Damage Points immediately from Endurance and 1 per pulse thereafter
  from Fatigue (Endurance when Fatigue is exhausted) until you are
  dead or the bleeding is staunched by a Healer of Rank 1 or
  above. Reduce your MD by 2 and your AG by 1.

\item[36] The same as 35, except it’s your good primary arm that has
  been lopped off.

\item[37–40] You have been eviscerated! Take 4 Damage Points
  immediately from Endurance and 1 point per pulse from Fatigue
  thereafter (Endurance when Fatigue is exhausted) until you are
  unconscious. Increase your chance of infection by 40.

\item[41–42] A glancing blow lays open your scalp and severs one ear
  (your choice as to which one). Take 2 Damage Points immediately from
  Endurance. Reduce your Perception by 2.

\item[43] A savage slash rips open your cheek and jaw. Take an
  automatic pass action next pulse due to the shock of the blow. Your
  PB is increased by 1, since your disfigurement will bring out the
  maternal/paternal instincts in the opposite gender.

\item[44–50] A slash along one arm, and it’s a bleeder!  Take 2 Damage
  Points immediately from Endurance and lose 1 point from Fatigue
  (Endurance when Fatigue is exhausted) each pulse until the bleeding
  is stopped by a Healer of Rank 1 or above or you die.

\item[51–52] Hamstrung! Roll D10. On a roll of 1–4, it is your left
  leg. On a roll of 5–10 it is your right. Take 4 Damage Points
  immediately from Endurance and fall prone. You may not stand
  unassisted until the wound is healed (which should take three
  months).  Reduce your AG by 3 permanently.

\item[53–60] Your primary arm is crippled by a wicked slash! Take 2
  damage Points immediately to Endurance and drop anything you have in
  your primary hand. The arm is unusable until healed, which should
  take 2 months.

\item[61–67] Your secondary arm is crippled; see 53–60 for details.

\item[68–69] A nasty slash in the region of the shoulder and
  neck. Roll D10. On a roll of 1–3, your head is severed and your
  corpse tumbles to the ground. On a roll of 4–6, your secondary
  collar bone is crushed; on a roll of 7–10 your primary collar bone
  is crushed. If your collar bone is crushed, the results are
  identical to 53–60, except you suffer 4 Damage Points to Endurance.

\item[70–74] A crushing blow smashes your helmet and causes a
  concussion. Take 3 Damage Points from Endurance and suffer a
  reduction of 4 in both MD and AG lasting for 3 days.

\item[75–80] A massive chest wound accompanied by broken ribs and
  crushed tissues. Very ugly, this. Take 5 Damage Points immediately
  from Endurance. Reduce your MD and AG by 3 each until this wound
  heals (which should take about 4 months).  Increase your chance of
  infection by 10.

\item[81–84] A crushing blow smashes tissue and produces internal
  injuries. You suffer 2 Damage Points immediately to Endurance and 1
  per pulse thereafter to Fatigue (Endurance when Fatigue is
  exhausted) until unconscious or you receive the attention of a
  Healer of Rank 2 or above.

\item[85–87] A jarring blow to your primary shoulder inflicts 2 Damage
  Points immediately to Endurance. Roll D10; the result is the number
  of pulses the arm is useless. You immediately drop anything held in
  that hand.

\item[88–89] Similar to 85–87 except it is your secondary shoulder.

\item[90–92] Your right hip is smashed horribly. Take 5 Damage Points
  immediately to Endurance and fall prone. You will be unable to walk
  until the damage has healed (which should take about 6 months). Good
  fun.  When healed, you will still have a limp which will reduce your
  TMR by 1 and your AG by 2.

\item[93–94] The same as 90–92 except it is your left hip that is
  smashed.

\item[95–97] Your opponent’s weapon has come crashing down on your
  head and fractured your skull. You fall prone and are unconscious,
  and take 8 Damage Points to Endurance.  If you survive, you lose 2
  from AG, 2 from MD and 2 from Perception. It will take a year in bed
  to recover.

\item[98-100] Crushing blow to your pelvis breaks bone and tears
  tissue. Take 7 Damage Points immediately to Endurance and fall
  prone.  Make a WP check to avoid falling unconscious. If you
  survive, you will be unable to move for D10 months.

\end{Description}

\subsection{Notes}

The suggested recovery times are a guideline for the GM to use in
determining how long characters should be kept out of action. The
actions of a competent Healer may alter these times in some instances.

These Grievous Injuries are designed for combat between human-sized
opponents; any injuries sustained involving larger monsters should be
applied judiciously by the GM, taking into account size and mass
differences, etc.  In some situations, the GM may have to disallow the
Grievous Injury or change its effects.

\end{Chapter}
