\begin{Chapter}{Fright \& Awe Tables}

Rolls against these tables can made as the result of 
magic,  when  meeting  entities  who  are  extremely 
ugly or beautiful, and in other surprising situations. 

Modifiers 

All  modifiers  are  cumulative  lasting  24  hours  or 
until  8  hours  sleep,  but  each  division  only  counts 
once.  For  example,  a  target  gets  feared  3  times. 
The first is a hysterical which adds 10%; second is 
catatonic which adds 10%. Total so far is 20%. The 
third,  however,  is  another  hysterical  but  as  a  hys-
terical  component  is  already  included  in  the  20%, 
the  total  remains  at  20%.  A  minor  heart  attack 
within 24 hours would push the total to 30%. 

Familiarity 

In both the cases of extremely low and high beauty, 
the  effects  are  reduced  with  familiarity  with  the 
creature  causing  the  fear  or  awe.  Once  a  character 
has  successfully  made  their  Willpower  check,  or 
has  recovered  from  a  failed  check,  they  need  not 
check again (for an effect from the same creature) 
for  the  remainder  of  the  encounter.  If  the  same 
creature  is  encountered  again  another  Willpower 
check  must  be  made,  but  the  GM  may  add  1  or 
more  to  the  difficulty  factor,  thus making  it  easier 
to  succeed.  This  bonus  should  not  be  applied  to 
encounters with other similar creatures, only to the 

54.1 Fright Table (Ver 1.1) 

same ones (it is not true to say that when you have 
seen one Troll, you have seen them all). 

PB Fright Checks 

Whenever  characters  encounter  a  creature  whose 
Physical Beauty is less than 5, and whose descrip-
tion states a Fear causing ability, they must make a 
Willpower  check  to  determine  if  they  are  fright-
ened.  The  difficulty  factor  for  this  test  is  equal  to 
the  creature’s  PB  (use  a  factor  of  1  if  PB  is  0).  If 
this  test  is  failed,  the  character  must  then  roll  on 
the  Fright  Table  (see  §54.1)  and  apply  any  results 
before they take another action. They may attempt 
to  recover  every  pulse,  by  trying  to  succeed at the 
same  Willpower  check.  Until  that  time  they  will 
act as the Fright Table indicates. 

PB Awe Checks 

Whenever  characters  encounter  a  creature  whose 
Physical  Beauty  is  above  26,  and  whose  descrip-
tion states an Awe causing ability, they must make 
a Willpower check to determine if they are awed. 

If this Willpower check is failed the character must 
then  roll  on  the  Awe  Table  (see  §54.2)  and  apply 
any results before they take another action. Recov-
ery  from  awe  effects  is  as  indicated  on  the  Awe 
Table.  Awe  can  be  caused  by  a  creature  with  a 
Physical  Beauty  greater  than  26.  The  range  of  the 

54.2 Awe Table 

effect is line of sight to the creature. The character 
will  only  be  affected  if  the  creature  is  within their 
line  of  sight,  and  they  are  facing  towards  it.  Once 
seen, however, facing becomes irrelevant, i.e. if the 
Target turns away, so as to run, they do not lose the 
Awe  effect  simply  because  they  are  no  longer 
facing  the  source.  The  effect  will  remain  until  the 
Target is out of line of sight of the source. Should 
the character return into line of sight of the source, 
the  same  result  as  before  will  be  applied,  unless 
their Willpower has increased, in which case, they 
receive another Willpower check to resist the effect 
of the Awe. 

The  difficulty  factor  of  the  Willpower  check  is 
dependent on the creature’s PB: 

PB   Difficulty  

PB  Difficulty  

Factor 

Factor 

27–28  4.0 

3.5 

3.0 

2.5 

29 

30 

31 

 

32 

33 

34 

2.0 

1.5 

1.0 

35+  0.5 

< 20   Wary The target will not voluntarily approach the source of their 
fear. If they are not aware of the source they will be very cautious 
and seek to optimise safety. 

21–25   Berserk They immediately charge to attack the object of their 

26–75  

76–90  

rage. If the source is not apparent they will charge about noisily 
looking for it. Add +10 to Strike Chance and -10 to Defence. 
Panic They will attempt to maximise their safety in relation to the 
source of their fear. This usually involves fleeing as rapidly as 
possible, but could also include cowering in the centre of the 
party, curling up in a small ball, hiding under a bed, etc. While a 
state of panic prevails, some sanity is present and the target would 
not normally do anything suicidal (e.g. running over the edge of a 
cliff) but they might use abilities to increase their safety (e.g. 
flying away). If the target wishes to use an ability, (e.g. casting a 
spell) the GM should give a suitable negative modifier to their 
base chance (e.g. -20). 
Frozen They may take no action until snapped out of it (e.g. 
slapped on the face, attacked, etc). The target can attempt to break 
out of it themselves by making a 1 × WP check per pulse. On 
recovery, the target rolls again at -30 (with no other modifiers) to 
determine their next action. Add +10 to subsequent rolls on the 
fright table. 

91–95   Hysterical They stand and scream and may take no other action 
until snapped out of it (as for 76–90). On recovery, roll again at -
20 (with no other modifiers). Add +10 to subsequent rolls on the 
fright table. 

96-100   Catatonic Target becomes catatonic. Their hair turns white and 
they may take no other action until snapped out of it (as for 76–
90). On recovery, roll again at -20 (with no other modifiers). Add 
+10 to subsequent rolls on the fright table. 

101–110   Faints The target faints into unconsciousness and loses 5 Fatigue. 

At the end of each minute they roll 1 × WP in order to regain 
consciousness. Add +10 to subsequent rolls on the fright table. 

116+  

111-115   Collapses The target collapses into unconsciousness and loses all 
of their Fatigue. After (30 - Endurance) minutes, or being tended 
by a Healer, they will regain consciousness. All their Characteris-
tics and Ranks will be reduced by 2, and they will not be able to 
recover Fatigue, until they have had comfortable bed rest for (40 - 
Endurance - tending Healer Rank) hours. Add +10 to subsequent 
rolls on the fright table. 
Heart Attack The target suffers a heart attack and must receive 
the attention of a Healer of at least Rank 3 within Endurance 
pulses or they are dead. If they survive they will be on 0 Endur-
ance and 0 Fatigue, and will be unconsciousness for (30 - Endur-
ance) minutes. All their Characteristics and Ranks will be reduced 
by 5, and they will not be able to recover Fatigue or more than 
half their Endurance, until they have had comfortable bed rest for 
(60 - Endurance - tending Healer Rank) hours. Add +10 to subse-
quent rolls on the fright table. 

 

 

01–20   Awe Target is slightly awed. Will not voluntarily approach the 

source of the Awe, unless requested to do so by the source. If not 
aware of the source, will be slightly cautious. 

21–25   Enamoured Target is completely enamoured of the source. Will 

26–76  

do anything that the source requests, to the extent of attacking 
comrades and friends, but not to the extent of killing themselves if 
so requested. If the source is not apparent they will rush around 
noisily looking for it. 
Panic Target is panicked by the Awe. They will attempt to maxi-
mise their safety, as they perceive it. This may involve fleeing, 
cowering, hiding, pleading and whimpering. They will not usually 
further endanger themselves in their attempt to escape, by running 
off a cliff for example. Add +5 to subsequent Awe Table rolls. 

77–90   Humble Target is completely humbled and prostrates themselves 

before the source of the Awe. They may take no other actions 
until they are snapped out of their grovelling by an outside 
agency, or by rolling less than or equal to their Willpower on 
D100. They may attempt this roll every second Pulse after being 
affected. Add +10 to subsequent Awe Table rolls. 

91–95   Hysterical Target becomes hysterical and falls to the ground, 
weeping, laughing, singing and/or praying as appropriate until 
snapped out of it by an outside agency. They may attempt to 
break out of it by themselves, by rolling less than or equal to their 
Willpower on D100, at the end of each minute following the 
affect. On recovery, roll D100: 1–50 slightly awed (as for 1–20, 
this table), 51–55 enamoured (as for 21–25), 56–100 panic (as for 
26–76). Add +10 to subsequent Awe Table rolls. 

96–100   Catatonic Target collapses, becomes catatonic and may take no 
further action until snapped out of it by an outside agency. Upon 
recovery, roll D100: 1–26 slightly awed, 27–31 enamoured, 32–
95 panic, 96–100 hysterical. Add +15 to subsequent Awe Table 
rolls. 

101–106   Faints Target faints dead away and will remain unconscious for 

[D + 6] minutes. Add +15 to all subsequent Awe Table rolls. 

107–110   Mild Heart Attack Target suffer a mild heart seizure. The result 
is the same as for 101–106 except that the Target may not move 
about under their own power for the remainder of the day and 
suffers a decrease of 2 to PS, MD, AG, EN, and FT, until either 
they spend one month resting in bed, or their heart is repaired 
using the Healer ability “Repair Tissues and Organs”. Add +15 to 
subsequent Awe Table tolls. 
Severe Heart Attack Target suffers a severe heart attack and 
must have the attention of a Healer of at least Rank 2 within one 
minute (12 pulses) or they will die. Otherwise this result is the 
same as for 107–110, except that +20 is added to subsequent Awe 
Table rolls. 

111+  

Spell Effects 

1.  Wall  of  Bones  (Necromancy):  A  target  is  not 
affected by the wall until they touch it. If they fail 
to resist they are affected any time they are in line 
of sight of the wall but facing is not important (if a 
target  turns  to  flee  the  wall  then  they  do  not  lose 
the fear because they are no longer facing it). If the 
target  touches  the  same  wall  again  they  make  an-
other resistance check. Duration is that of the spell; 
once the wall is gone, so is the fear. 

2.  Fear  (Necromancy,  Wicca,  Celestial):  single 
target spell. 

If duration is immediate then after (Rank of spell / 
5)  pulses,  rounded  up,  the  target  may  make  a  3  × 
 

 

WP  check  each  pulse  to  recover  from  fear,  other-
wise the fear lasts the duration of the spell.  A fur-
ther  check  may  be  required  to  recover  from  the 
secondary  effects  of  76–90,  91–95  and  96–100. 
Once  recovered  from,  the  spell  has  no  further  ef-
fect.  The  range  of  the  spell  only  determines  the 
possible targets, so the target being more than this 
distance  from  the  Adept  will  not  affect  the  dura-
tion. 

3.  Mass  Fear  (Necromancy,  Wicca):  has  set  range 
and duration. Range is a sphere around the Adept. 
Targets  going  out  of  range  of  the  spell  are  no 
longer affected. Each time they re-enter the area of 
effect  of  the  spell  they  make  another  magic  resis-
tance (and roll on the Fright Table if they fail). 

A group of adventurers are in the middle of 
Example 
a  forest. Rolf the  Harrier  gets struck by a  Fear spell  from 
Nasty the Necro, and  fails to  resist.  A roll of 45  indicates 
panic. Rolf’s player thinks Rolf has 3 options: 

1. Run back to a clearing just left by the party. 

2. Feeling no direction to be safe, Rolf clings to Hu’ug the 
giant’s leg and trembles. 

3. Being a “If I can’t see them they can’t see me” believer, 
Rolf puts a sack over his head. 

Rolf’s  player  and  the  GM  decide  on  percentages  for  the 
options and one is determined. In this case Rolf flees. The 
spell  is  Rank  7  so  Rolf  gets  to  attempt  to  snap  out  of  the 
fear each pulse after the first two. Six pulses after the spell 
is  cast,  on  his  third  try,  Rolf  succeeds  his  1  ×  WP  check 
and may turn around and attack Nasty the Necro. 

\end{Chapter}
