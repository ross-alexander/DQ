\begin{Skill}[2.1]{ranger}{Ranger}

Rangers are trained to survive, and perhaps thrive, in wilderness.
They can feed themselves, shelter from the elements, choose the best
way to travel and identify natural dangers.  Rangers’ general training
is useful in any outdoors environment but they benefit further from
learning the specifics of particular environments.

\section{Benefits}

\subsection{Primary Environment}

A ranger knows far more about the environment with which they are most
familiar.  While in this primary environment the ranger’s base chances
and formulas should be calculated as if they were 2 Ranks higher.  A
ranger’s initial primary environment is that in which they learnt
the skill.  A ranger may later choose to change their primary
environment during ranking.

The ranger must train in the new environment, the ranking time is
increased by 1 week and the EP cost for this rank is increased by 50\%
(to maximum of +3000 EP).  To return to a previously learnt primary
environment the ranger must spend 500 EP and 2 weeks in the
environment.  After rank 10 a ranger may learn a new environment by
training for 4 weeks in the new environment and spending 3000 EP.

\subsection{Stealth Bonus}

While using any of the abilities in this skill a ranger gains a bonus
to stealth of + 3 / Rank.  No other skill bonuses to stealth may be
applied at the same time as this bonus.

\subsection{Finding Food}

\begin{Description}
\item[Foraging] A ranger knows how to find water, edible plants, and
  animals suitable for the pot. Foraging includes finding plants,
  setting snares, hunting small animals, fishing etc.  Snares should
  be left overnight (or even days) to be successful. A ranger does not
  need to make an attack roll but may automatically kill small
  animals that were caught during foraging.

  In an average area in one hour a ranger can find enough food to feed
  one person for a day (+ 30 minutes per extra person).  The volume of
  food available is dependent on fertility and season so the GM should
  adjust the time to suit the environment.  If a ranger wants to hunt
  larger animals they should use the Tracking ability to locate game
  and then use an appropriate hunting weapon to kill it (i.e.  ranged
  weapons, spear).  If they make a successful attack then their quarry
  is immediately killed.  If they miss the animal will flee.  If a
  ranger is hunting predators, extremely large animals or sentients
  they must use the combat rules to kill their quarry.  A ranger and
  mechanician may combine their abilities to build and conceal large
  traps, pit falls, etc at the GM’s discretion.

\item[Identify and Find Plants and Animals] A ranger can recognise
  common plants and animals.  They have a (Perception + 10 / Rank)\%
  chance of resolving whether a strange plant or animal is suitable
  for food.  If they roll 10\% or less than their success chance they
  may also notice other properties of the item (e.g. poisonous,
  valuable etc).

  A ranger can identify the types of entities living in an area from
  the traces they leave behind (tracks, game paths, grazing signs,
  prey remains etc).  This takes about 15 minutes and gives them an
  idea of the variety of animals in the area (e.g. the primary
  carnivore is a wolf pack; there is a large herd of red deer and a
  flock of pigeons).

  A ranger may search for a specific plant or animal (including herbs
  required in the First Aid ability), provided it is native to the
  region. The base chance is 2 × PC + 5 / Rank (- 0 if common, - 25 if
  uncommon and - 75 if rare). This roll should be made once per hour
  of searching.

\item[Tracking] A ranger can follow the tracks left by entities moving
  on the ground.  In calm weather, tracks normally last around 10 days
  but the clarity and duration of tracks will be enhanced by the
  number of entities, or soft ground, and reduced by hard ground,
  rough weather, or if the entity is trying to hide their tracks.  The
  base chance of following tracks is Perception (+ 5 / Rank) (+ 2 /
  entity in group) - (4 / Rank of quarry’s ability to hide tracks).
  If a ranger is following a fresh track they will be aware when they
  are close enough to be detected.  They may then use stealth to sneak
  up on their quarry and they will be able to get 25\% closer than a
  non-ranger before there is a possibility of being detected.

\end{Description}

\subsection{Camping}

\begin{Description}

\item[Preparing Food] A ranger knows how to get a fire going, gut and
  skin animals, and cook simple meals over an open fire.

\item[Campsites] A ranger knows where to set up camp so that they are
  sheltered from the elements, close to water, or other by criteria
  they may choose (e.g. hidden or defensible).

  A ranger can easily erect tents, they can add extra comfort to a
  campsite by setting up tarps to protect from wind or water, and they
  can take advantage of nearby resources to build a crude shelter.

\end{Description}

\subsection{Travelling}

\begin{Description}

\item[Orientation] A ranger has a sensitivity towards north. They are
  able to pinpoint true north to within (10 - Rank) degrees and from
  this they can work out the other compass directions.

\item[Map Reading] A ranger can read a simple map if they can relate
  their physical surroundings to the symbols on that map.  There are
  no standard symbols or keys so interpreting a new map is a challenge
  of the Ranger’s wits and experience.  A ranger may place themselves
  on a map if they can determine the direction of two marked
  landmarks.

\item[Route Finding] A ranger is rarely lost and can normally back
  track to a known point.  They learn to recognise landmarks from
  unfamiliar directions and estimate the time and effort required to
  travel through various terrain.  A ranger can pick a route through
  unknown terrain based on ease of travel, speed, stealth, or safety
  etc. The base chance of the ranger picking the best route for their
  purpose is 2 × Perception (+5 / rank)\%.  The roll should be made by
  the GM and if the ranger fails then the route travelled should be
  hard or longer or dangerous as appropriate.  This roll should only
  be made once per day.

  After a ranger has travelled through an area several times they do
  not need to use known routes but can freely take shortcuts or choose
  better routes.

\item[Distance Estimates] A ranger can estimate distance travelled
  overland to within (90 + Rank)\% accuracy.

\end{Description}

\subsection{Safety}

\begin{Description}
\item[Detect Hidden] In a natural setting a ranger may notice hidden
  entities, or recognise an ambush or trap before they walk into
  it.  The base chance is 3 × Perception (+ 5 / Rank) (-5 / Rank of
  person who did the hiding or set the ambush or trap).

\item[Hide Tracks] A ranger can obscure the tracks of 1 (+ 1 / Rank)
  entities moving in the same direction.  It takes 30 (1 / Rank)
  minutes to obscure 100 yards of track.  This time may be reduced if
  the ground is rocky or naturally hard.

\item[Hide Entities] A ranger can attempt to hide 1 (+ 1 / Rank)
  entities in natural cover. The ease of hiding someone is dependent
  on the available terrain. The GM should advise a modifier based on
  the terrain of 1 (e.g. flat open ground) to 10 (e.g. thick bushes or
  jungle). The base chance of hiding is (modifier × Rank) - 5. (NB
  this ability does not imply that the ranger can set up ambushes).

\item[First Aid] A ranger knows simple first aid to prevent minor
  accidents in the wilderness becoming severe. They know how to:
\begin{Itemize}
\item Stop external bleeding  
\item Splint broken bones  
\item Treat minor burns  
\item Recognise the effects of common natural poisons 
\end{Itemize}

They also know how to brew tisanes (herbal teas) which help reduce the
effects of headaches, nausea, fevers and food poisoning.  To make
tisanes the ranger requires fresh common herbs (which have been picked
within 24 hours of use).

The First Aid abilities cannot be used in combat.

\end{Description}

\section{Environments}

The environments a ranger may choose as their primary environment are
dominated by similarities of climate, terrain and fertility. These
environments cover lightly populated areas eg.  open farmland, moors,
but do not include towns, cities, etc.  Some environments overlap.

\begin{Description}

\item[Arctic] Includes tundra, steppes, permafrost and ice caps and
  other infertile lowlands in cold climates.  Fertility: Infertile,
  Seasons: standard, note that winter has no daylight \& summer has no
  night time.

\item[Caverns] Includes all caves, tunnels, natural caverns, and other
  substantial underground areas.  Fertility: Infertile, Seasons:
  always low season.

\item[Coastal] Includes land adjacent to saltwater, estuaries, coastal
  marshes etc. Fertility: Average or poor, Seasons: standard.

\item[Highlands] Includes hills and mountains, moors, high plateaus.
  Also includes evergreen forests on steep ground. These areas are
  fertile in summer but snow or ice covered and hostile in
  winter. Fertility: Poor, Seasons: standard.

\item[Jungle] Includes hot climate forests of any sort.  They are
  particularly characterised by heavy undergrowth and high rainfall.
  Fertility: Rich, Seasons: wet/dry.

\item[Plains] Includes grasslands, plains, pampas, savannah, prairie,
  veldt, and other more or less open and flat or rolling terrain. May
  include low hills where the land is open and not wooded.  Fertility:
  Poor, Seasons: standard.

\item[Rural] Generally mild climate cultivated terrain, lightly
  inhabited. Includes cultivated fields, grazing lands, vineyards,
  heaths, etc.  Fertility: Average, Seasons: standard.

\item[Waste] Includes all deserts, wastelands, salt flats, and other
  infertile lowlands in mild to hot climates.  Fertility: Infertile,
  Seasons: Reversed in hot regions as the most fertile period is
  autumn and the least fertile summer.

\item[Wetlands (freshwater)] Includes marshes \& swamps, and land
  adjacent to freshwater rivers, lakes \& ponds, etc.  Fertility:
  Rich, Seasons: standard.

\item[Woods] Includes mild climate deciduous and evergreen forests or
  large wooded areas with few sentient inhabitants, in mild to cold
  climates. Fertility: Average, Seasons: standard.

\end{Description}
\end{Skill}

\begin{table*}[t]
\section{Ranger Summary Chart}

\medskip

\begin{dqtblr}{colspec={|l|X|X|l|},
    vlines={green7},
    hline{1-2}={1.2pt}}
Ability 		& Base Chance		& 			& \\
Brew Tisanes		& 90 + 1 / Rank 	& Reduces effects of headaches, nausea, fevers and stomach upsets & \\
Choosing campsites 	& 90 + 1 / Rank		& 			& \\
Detect hidden \& traps	& 3 × PC + 5 / Rank (-5 / Rank of opposing ability) & incl. hidden entities, ambushes & \\
Distance Estimates	& 90 + 1 / Rank		&			& \\
Find specific plant/animal & 2 × PC + 5 / Rank (- 25 if uncommon / -75 if rare) & Roll per hour  & \\
First Aid 		& 90 + 1 / Rank		& Stop external bleeding, treat burns, splint bones, recognise poison & \\
Foraging		& 90 + 1 / Rank		& Modified by season and fertility & 60 mins + 30 per extra person  \\
Hide Entities		& (Modifier × rank) - 5 & Hides 1 + 1 / Rank entities. Modifier based on available cover & \\
Hide Tracks		& 90 + 1 / Rank 	& Obscures tracks of 1 + 1 / Rank entities & 30 - 1 / Rank minutes \\
Identify Local Inhabitants &  90 + 1 / Rank	& 			& 15 minutes \\
Map Reading		& 90 + 1 / Rank		&			& \\
Orientation		& 90 + 1 / Rank		&			& \\
Preparing Food		& 90 + 1 / Rank		&			& \\
Recognise plants/animals & PC + 10 / Rank	&			& \\
Route Finding		& 2 × PC + 5 / Rank	& Rolled by GM		& Roll per day \\
Tracking		& PC + 5 / Rank (+ 2 / entity) (- 4 / Rank opposing ranger) & Tracks last 7 - 10 days & \\
\end{dqtblr}

\end{table*}
