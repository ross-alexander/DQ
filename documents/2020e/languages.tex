\begin{Chapter}{Languages (Ver 2.1)}

The campaign has many languages.  Each sentient race usually has one
language intrinsic to itself, or more if that race is split into
various populations.  There is no universal language, but Common is
the first language of several nations.

\section{Restrictions}

A language may not be known above its maximum rank. Characters may not
speak a tongue for which they do not have the vocal apparatus.
Characters may not learn a language without instruction from a source
of at least the same rank as that being learnt.

\section{Structure}
Family Each language belongs to one particular Family of intrinsically
related tongues (see §39.6).

Group History, geography, and custom all transform languages.
Languages with a common history or interaction share the same language
Group (see §39.7). A language may belong to several Groups, and a
Group may link languages from different Families.

Learning a language is easier if one already knows a related or
similar language at a higher rank. The EP discount is:

• 20\% if in the same Family or Group,  

•  30\%  if  in  both  the  same  Group  and  the  same 
Family. 

\section{Benefits}

Languages vary in their complexity; a low maximum rank may indicates
less versatility, vocabulary, or foreignisms.

At Rank 0 in a language, you cannot speak it, but can usually sense
the general mood of plain statements: a threat, a greeting, etc.
Thereafter, with increasing rank, your competency and vocabulary
progressively increase, as compared to humans using a typical human
language to talk about everyday things in their village.

Rank  Effect (\& approximate Vocabulary). 

1  

2  

3  

4  

5  

6  

7  

8  

9  

Some of the simple, common words (2%). 

A few simple statements (5%). 

Common  phrases,  including  basic  direc-
tions; several tenses; effectively rank 0 in 
all other languages of that Group (20%). 

Common  idioms;  more  tenses;  can  give 
passable descriptions of events or people; 
effectively rank 0 in all other languages of 
the same Family (70%). 

Rarer  idioms;  most  tenses;  sufficient  to 
use most professional skills (90%). 

Normal,  every-day  fluency  \&  usage;  can 
give  clear  \&  accurate  descriptions  of 
events or people; effectively rank 1 in all 
other languages of that Group (100%). 

Courtly or professional speaker (120%+). 

Can  express  any  conceivable  thought; 
may cast college magic; effectively rank 1 
in all others of the same Family (200%+). 

Effectively  rank  2  in  all  other  languages 
of that Group (400%+). 

10  

Maximum  mastery  of 
(500%+). 

the language 

Note that some languages are very limited.  For example, many concepts
or emotions cannot be articulated in Troll.

\section{Literacy}

Literacy in a language is distinct from the skill of speaking.  It is
easily learnt if the written form is alphabetic.  Most cultures have a
large proportion of the population that is illiterate.

Not all languages have a written form.  It is not possible to attain
literacy in a language that does not have an established written form.
One may attempt to transcribe that language, adapting a known script,
but the “writing” produced is ineffectual for communicating with
others.

\subsection{Phonetic Reading \& Writing}

Most  Alusian  languages  are  written  using  a  pho-
netic  alphabet  —  a  set  of  signs  representing,  one-
to-one,  all  the  sounds  of  that  language.  Histori-
cally, a recently literate language usually re-uses an 
established  alphabet  with  minor  variations.  There-
fore there are many languages, but few alphabets. 

For each alphabet, the cost is 1000 EP and 4 weeks 
the first time you learn literacy using it; literacy in 
a subsequent language, using the same alphabet, is 
only 500 EP and 2 weeks. Sometimes, for different 
cultures,  one  language  is  written  in  different  pho-
netic  alphabets.  If  so,  you  must  pay  the  time  and 
EP for each one you learn. 

\subsection{Table of known Alusian alphabets}

b   Bedouin script (human, flowing, cursive). 

d   Drakonic. 

e  

i  

Elvish script. 

Island (used near the land-locked ocean). 

k   Kingdom (used near the Azurian Empire). 

m   Mer (suited to underwater use) 

n   Nagan (elaborate, but versatile).  

o   Ogham (human, rune-like). 

r   Dwarvish runes. 

w   Westron (usual Western human alphabet, also 
adopted by many newly literate societies). 

\subsection{Orthographic languages}

A  literate  language  not  using  a  truly  phonetic  al-
phabet  is  orthographic  (e.g.  it  uses  pictograms,  or 
an elaborate spelling structure). The written form is 
so complex that it must be learned as if it were, in 
effect,  another  language  of  the  same  language 
family  (e.g.  written  and  spoken  Erehleine  are 
treated as two separate members of the Eldar Fam-
ily).  Hence  one  often  speaks  and  writes  an  ortho-
graphic  language  at  different  ranks.  Orthographic 
languages are indicated in §39.6 by an asterisk (*). 
Each orthographic system is functionally unique to 
its particular language. 

\section{Special Rules}

Common  It is easy to learn Common. Knowledge 
of any other language at a higher rank gives a 50% 
EP discount. 

Accent Every speaker has an accent which reflects 
a  mixture  of  their  native  language  and  the  tutors 
from whom they learnt the language. At Rank 6 or 
higher, any speaker may gain a particular accent by 
spending 500 EP and 1–3 weeks studying or being 
tutored  (the  GM  decrees  how  much  time  is neces-
sary). 

Unpronounceable  Tongues  All  languages  of  the 
Dragon  Family  (except  Saurime)  require  unusual 
vocal  apparatus.  No  humanoid  race  may  normally 
speak  these  tongues.  However,  you  may  rank  the 
language at twice normal cost, to gain comprehen-
sion.  Alphabetic  literacy  in  an  unpronounceable 
language  costs  2000  EP  and  8  weeks.  If  you  do 
have  the  physiological  or  magical  ability  to  speak 
such  languages,  you  may  rank  them  without  pen-
alty. 

Immersion If character spends a number of weeks 
listening  to  a  particular  language  being  spoken 
daily  and  frequently  by  speakers  who  use  it  at  a 
rank  higher  than  the  character  knows  it,  the  GM 
may  allow  that  character  to  use  those  weeks  as 
ranking  time  for  that  language  in  addition  to  any 
other activity undertaken (e.g. going on adventure, 
other  training,  etc).  The  EP  must  still  be  paid.  A 
character  may  only  rank  one  language  by  immer-
sion at any one time. 

\subsection{New Languages}

When a new language is introduced into the campaign, the GM concerned
must determine the following:

1. Family and any language Groups. 

2.  Written  forms,  if  any.  Are  they  phonetic  or 
orthographic? If alphabetic, what alphabet is used? 

3. Its maximum rank. 

\section{Language Families}

The figure within [ ] represents the maximum rank that can be achieved
with the language; the letter(s) represent the phonetic alphabet(s)
used, and * identifies orthographic languages.  If no letter or
asterisk is given, the language does not have an established written
form.

Common Common [9i,k,w]. 

Western-Human  Alman[9o,w],  Brett[9o,e],  Des-
tinian[8w],  Ebolan[9w],  Folksprach[9w],  La-
lange[10w],  Raniterran[9e],  Reichspiel 
[9w], 
Saxony[9w]. 

Arabiq[9b], 

Central-Human 
Domani[9w], 
Draknbrger[9w],  Ellenic[10i],  Kipchak[8],  Kra-
vonian[9*], 
Pasifikan[8], 
Sanddweller[9e],  Sea-of-Grass[9],  Themiskryan 
[9i,*]. 

Panjari[9*], 

Eastern-Human  Five-Sisters  [10*],  Hindian  [9b], 
Lunar Empire[9*], Ruskan [9k]. 

Bestial  Dawon  [7],  Dimasa  [10b,n],  Gnoll  [7], 
Karbi  [9]  Rabari  [8b],  Sasquatch[3],  Sora  [6], 
Vanaran [9b]. 

39.7 Language Groups 
Archaic Eldaran, Quenchan, Tenochan. 

Austronesian Jhavanese, Madyrese, Mylae. 

Draconic  Culhuan,  Draconic,  Nagan,  Old-
Draconic, Wyvern. 

Dravidic  Drow, 
Sanddweller. 

Five-Sisters, 

Raniterran, 

Dwarvic Dwarvish, Gnomish, Halfling. 

Dwarvidic  Alman,  Brett,  Ebolan,  Reichspiel, 
Folksprach, Ruskan, Saxony. 

Ellenic Centaur, Ellenic. 

Elvic  Drow,  Eldaran,  Elvish,  Erehleine,  Terrano-
van-Drow. 

Elvidic Elvish, Lalange, Eloran.  

Gnomic Fossegrim, Gnomish.  

Herpetic Culhuan, Saurime. 

Hiin  Dawon,  Dimasa,  Doppleganger,  Gnoll,  Hin-
dian, Karbi, Rabari, Sora, Vanaran. 

Littoral Destinian, Ebolan. 

Bhasa Mylae[10i], Jhavanese[9i], Madyrese[8i]. 

Low Gigantic Hill-Giant, Ogre, Stone-Giant. 

Nomadic  Domani,  Draknbrger,  Kipchak,  Kra-
vonian, Sea-of-Grass. 

Orcal Goblin, Hobgoblin, Kobold, Ogre, Orcish.  

Panic Centaur, Dryad, Nymph, Satyr, Sylphine.  

Perfidic Fossegrim, Merfolk, Nixie, Pixie.  

Protonic Eldaran, Old-Draconic, Draconic.  

Rustic Brownie, Leprechaun. 

Titanic  Cloud-Giant,  Lunar-Empire,  Storm-Giant, 
Titan. 

 

 

Dragon  Old-High-Draconic[10d],  Culhuan[10*], 
Draconic[10d],  Nagan[10n],  Saurime[7d],  Wy-
vern[4]. 

Eldar  Drow[9e],  Eldaran[10d],  Eloran[9e,w], 
Elvish[10e],  Erehleine[10*],  Quenchan[10*],  Ter-
ranovan-Drow[9*], Tenochan[8*]. 

Faerie  Brownie[7],  Centaur[9i],  Dryad[6], 
Fossegrim[6], Leprechaun[6], Nixie[6], Nymph[7], 
Pixie[7], Satyr[7], Sylphine[6]. 

False-Fey 
Harpy[7], Medusa[6]. 

Doppelganger[8], 

Gargoyle[6], 

Earth-Dweller 
Goblin[8w], 
Halfling[9r],  Hobgoblin[8w],  Kobold[8],  Dwar-
vish[9r], Ogre[6w], Orcish[9w], Troll[4]. 

Gnomish[9r], 

Giant  Cloud[9w],  Fire[9w],  Frost[9w],  Hill[8w], 
Stone[8w], Storm[9w], Titan[10i]. 

Merfolk [8m]. 

Signing Silent-Tongue[6], Bandito [5]. 



\end{Chapter}
