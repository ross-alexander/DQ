\begin{Skill}[2.1]{language}{Language}

The campaign has many languages.  Each sentient race usually has one
language intrinsic to itself, or more if that race is split into
various populations.  There is no universal language, but Common is
the first language of several nations.

\section{Restrictions}

A language may not be known above its maximum rank. Characters may not
speak a tongue for which they do not have the vocal apparatus.
Characters may not learn a language without instruction from a source
of at least the same rank as that being learnt.

\section{Structure}
Family Each language belongs to one particular Family of intrinsically
related tongues (see §39.6).

Group History, geography, and custom all transform languages.
Languages with a common history or interaction share the same language
Group (see §39.7). A language may belong to several Groups, and a
Group may link languages from different Families.

Learning a language is easier if one already knows a related or
similar language at a higher rank. The EP discount is:

\begin{Itemize}

\item 20\% if in the same Family or Group,  

\item 30\% if in both the same Group and the same Family.

\end{Itemize}

\section{Benefits}

Languages vary in their complexity; a low maximum rank may indicates
less versatility, vocabulary, or foreignisms.

At Rank 0 in a language, you cannot speak it, but can usually sense
the general mood of plain statements: a threat, a greeting, etc.
Thereafter, with increasing rank, your competency and vocabulary
progressively increase, as compared to humans using a typical human
language to talk about everyday things in their village.

\begin{dqtblr}{colspec={lX}}
Rank	& Effect (\& approximate Vocabulary) \\
1	& Some of the simple, common words (2\%) \\
2	& A few simple statements (5\%) \\
3	& Common  phrases,  including  basic  directions; several tenses; effectively rank 0 in all other languages of that Group (20\%) \\
4	& Common  idioms;  more  tenses;  can  give passable descriptions of events or people; effectively rank 0 in all other languages of the same Family (70\%) \\
5	& Rarer  idioms;  most  tenses;  sufficient  to use most professional skills (90\%) \\
6	& Normal,  every-day  fluency  \&  usage;  can give  clear  \&  accurate  descriptions  of events or people; effectively rank 1 in all other languages of that Group (100\%) \\
7	& Courtly or professional speaker (120\%+) \\
8	& Can  express  any  conceivable  thought; may cast college magic; effectively rank 1 in all others of the same Family (200\%+) \\ 
9	& Effectively  rank  2  in  all  other  languages of that Group (400\%+) \\
10	& Maximum  mastery  of the language (500\%+) \\
\end{dqtblr}

Note that some languages are very limited.  For example, many concepts
or emotions cannot be articulated in Troll.

\section{Literacy}

Literacy in a language is distinct from the skill of speaking.  It is
easily learnt if the written form is alphabetic.  Most cultures have a
large proportion of the population that is illiterate.

Not all languages have a written form.  It is not possible to attain
literacy in a language that does not have an established written form.
One may attempt to transcribe that language, adapting a known script,
but the “writing” produced is ineffectual for communicating with
others.

\subsection{Phonetic Reading \& Writing}

Most Alusian languages are written using a phonetic alphabet --- a set
of signs representing, one-to-one, all the sounds of that language.
Historically, a recently literate language usually re-uses an
established alphabet with minor variations.  Therefore there are
many languages, but few alphabets.

For each alphabet, the cost is 1000 EP and 4 weeks the first time you
learn literacy using it; literacy in a subsequent language, using the
same alphabet, is only 500 EP and 2 weeks. Sometimes, for different
cultures, one language is written in different phonetic alphabets.
If so, you must pay the time and EP for each one you learn.

\subsection{Table of known Alusian alphabets}

\begin{dqtblr}{colspec={lX}}
b &  Bedouin script (human, flowing, cursive) \\
d &  Drakonic \\
e &  Elvish script \\
i &  Island (used near the land-locked ocean) \\
k &  Kingdom (used near the Azurian Empire) \\
m &  Mer (suited to underwater use) \\
n &  Nagan (elaborate, but versatile) \\
o &  Ogham (human, rune-like) \\
r &  Dwarvish runes \\
w &  Westron (usual Western human alphabet, also adopted by many newly literate societies) \\
\end{dqtblr}

\subsection{Orthographic languages}

A literate language not using a truly phonetic alphabet is
orthographic (e.g.  it uses pictograms, or an elaborate spelling
structure). The written form is so complex that it must be learned as
if it were, in effect, another language of the same language family
(e.g. written and spoken Erehleine are treated as two separate
members of the Eldar Family).  Hence one often speaks and writes an
orthographic language at different ranks.  Orthographic languages
are indicated in §39.6 by an asterisk (*).  Each orthographic system
is functionally unique to its particular language.

\section{Special Rules}

\begin{Description}

\item[Common] It is easy to learn Common.  Knowledge of any other
  language at a higher rank gives a 50\% EP discount.

\item[Accent] Every speaker has an accent which reflects a mixture of
  their native language and the tutors from whom they learnt the
  language. At Rank 6 or higher, any speaker may gain a particular
  accent by spending 500 EP and 1–3 weeks studying or being tutored
  (the GM decrees how much time is necessary).

\item[Unpronounceable Tongues] All languages of the Dragon Family
  (except Saurime) require unusual vocal apparatus.  No humanoid race
  may normally speak these tongues.  However, you may rank the
  language at twice normal cost, to gain comprehension.  Alphabetic
  literacy in an unpronounceable language costs 2000 EP and 8 weeks.
  If you do have the physiological or magical ability to speak such
  languages, you may rank them without penalty.

\item[Immersion] If character spends a number of weeks listening to a
  particular language being spoken daily and frequently by speakers
  who use it at a rank higher than the character knows it, the GM may
  allow that character to use those weeks as ranking time for that
  language in addition to any other activity undertaken (e.g. going on
  adventure, other training, etc).  The EP must still be paid.  A
  character may only rank one language by immersion at any one time.

\end{Description}

\subsection{New Languages}

When a new language is introduced into the campaign, the GM concerned
must determine the following:

\begin{Itemize}

\item Family and any language Groups. 

\item Written forms, if any.  Are they phonetic or orthographic? If
  alphabetic, what alphabet is used?

\item Its maximum rank. 

\end{Itemize}

\section{Language Families}

The figure within [ ] represents the maximum rank that can be achieved
with the language; the letter(s) represent the phonetic alphabet(s)
used, and * identifies orthographic languages.  If no letter or
asterisk is given, the language does not have an established written
form.

\begin{Description}
  
\item[Common] Common [9i,k,w]. 

\item[Western-Human] Alman[9o,w], Brett[9o,e], Destinian[8w],
  Ebolan[9w], Folksprach[9w], Lalange[10w], Raniterran[9e], Reichspiel
  [9w], Saxony[9w].

\item[Central-Human] Arabiq[9b], Domani[9w], Draknbrger[9w],
  Ellenic[10i], Kipchak[8], Kravonian[9*], Panjari[9*], Pasifikan[8],
  Sanddweller[9e], Sea-of-Grass[9], Themiskryan [9i,*].

\item[Eastern-Human] Five-Sisters [10*], Hindian [9b], Lunar
  Empire[9*], Ruskan [9k].

\item[Bestial] Dawon [7], Dimasa [10b,n], Gnoll [7], Karbi [9] Rabari
  [8b], Sasquatch[3], Sora [6], Vanaran [9b].

\item[Bhasa] Mylae[10i], Jhavanese[9i], Madyrese[8i]. 

\item[Dragon] Old-High-Draconic[10d], Culhuan[10*], Draconic[10d],
  Nagan[10n], Saurime[7d], Wyvern[4].

\item[Eldar] Drow[9e], Eldaran[10d], Eloran[9e,w], Elvish[10e],
  Erehleine[10*], Quenchan[10*], Terranovan-Drow[9*], Tenochan[8*].

\item[Faerie] Brownie[7], Centaur[9i], Dryad[6], Fossegrim[6],
Leprechaun[6], Nixie[6], Nymph[7], Pixie[7], Satyr[7], Sylphine[6].

\item[False-Fey] Harpy[7], Medusa[6].  Doppelganger[8], Gargoyle[6],

\item[Earth-Dweller] Gnomish[9r], Goblin[8w], Halfling[9r],
  Hobgoblin[8w], Kobold[8], Dwarvish[9r], Ogre[6w], Orcish[9w],
  Troll[4].

\item[Giant] Cloud[9w], Fire[9w], Frost[9w], Hill[8w], Stone[8w],
  Storm[9w], Titan[10i].

\item[Merfolk] [8m]. 

\item[Signing] Silent-Tongue[6], Bandito [5]. 
  
\end{Description}

\section{Language Groups}
\label{languages:groups}

\begin{Description}

\item[Archaic] Eldaran, Quenchan, Tenochan.

\item[Austronesian] Jhavanese, Madyrese, Mylae. 

\item[Draconic] Culhuan, Draconic, Nagan, Old-Draconic, Wyvern.

\item[Dravidic] Drow, Five-Sisters, Raniterran, Sanddweller.

\item[Dwarvic] Dwarvish, Gnomish, Halfling.

\item[Dwarvidic] Alman, Brett, Ebolan, Reichspiel, Folksprach, Ruskan,
  Saxony.

\item[Ellenic] Centaur, Ellenic.

\item[Elvic] Drow, Eldaran, Elvish, Erehleine, Terranovan-Drow.

\item[Elvidic] Elvish, Lalange, Eloran.  

\item[Gnomic] Fossegrim, Gnomish.  

\item[Herpetic] Culhuan, Saurime. 

\item[Hiin]  Dawon,  Dimasa,  Doppleganger,  Gnoll,  Hindian, Karbi, Rabari, Sora, Vanaran. 

\item[Littoral] Destinian, Ebolan. 

\item[Low] Gigantic Hill-Giant, Ogre, Stone-Giant. 

\item[Nomadic] Domani, Draknbrger, Kipchak, Kravonian, Sea-of-Grass.

\item[Orcal] Goblin, Hobgoblin, Kobold, Ogre, Orcish.  

\item[Panic] Centaur, Dryad, Nymph, Satyr, Sylphine.  

\item[Perfidic] Fossegrim, Merfolk, Nixie, Pixie.  

\item[Protonic] Eldaran, Old-Draconic, Draconic.  

\item[Rustic] Brownie, Leprechaun. 

\item[Titanic]  Cloud-Giant,  Lunar-Empire,  Storm-Giant, 
Titan. 

\end{Description}

\end{Skill}
