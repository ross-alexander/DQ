\begin{Chapter}{Navigator (Ver 1.1)}

The art of piloting a sea-going vessel and that of ascertaining one’s
location are inextricably linked.  Humanoids must venture across the
waters in awkward ships, and are unable to survive immersion in the
sea except for relatively short periods of time. Yet there are many
beings who dwell beneath the surface of the ocean, and it is
profitable for land-bound peoples to engage in commerce with
them. Adventurers, with the assistance of an Adept, will probably
choose to try to despoil some of the treasures of the deep.

A navigator can manage ships of increasing size as they become more
experienced. There is a limit to the size of ships constructed,
because of their relative fragility (sea creatures are wont to destroy
those vessels they consider overly large).  The navigator’s other
chief ability allows them to locate directions with instruments and
read maps.

\section{Benefits}

\subsubsection{A navigator can determine all compass directions if they can view the stars.}

If the night is cloudy, or during the day, the navigator’s chance of
correctly locating the compass direction is equal to (25 + 7 ×
Rank)\%.  If the roll is less than or equal to the success percentage,
the navigator has an exact reading on the compass directions.  If the
roll is greater than the success percentage, the reading is off by one
degree for each percentage point by which exceeds the success
percentage (the GM must decide in which direction the error is made).

\subsubsection{A navigator may always determine the compass direction of a landmark
relative to their position.}

A landmark is defined as any object which can be seen or to which a
being can precisely point.  A navigator may also judge the distance
between their position and a visible landmark. Their chance to
precisely gauge the distance is equal to (PC + 10 × Rank)\%.  When the
roll exceeds the success chance, the estimate is off by the percentage
difference between the roll and the chance to accurately judge,
randomly long or short.

\subsubsection{A navigator can read a map if they can relate their physical
surroundings to the symbols on that map.}

This skill allows a navigator to read a map, chart or rutter if they
can relate their physical surroundings to the symbols on that
document.  Even the best quality maps are not particularly accurate or
standardised. Interpreting each new map is a challenge of the
navigator’s wits and experience. If a charac- ter does not have a
map-reading skill, they may not read maps.

If a navigator tries to read a map which is of the area in which they
are presently located or is of an area with which they are quite
familiar, they clearly understand at least (2 × PC + 8 × Rank)\% of
the map.  Further, they are baffled by up to (2 × PC + 2 × Rank)\% of
the map. They may misinterpret the remainder of the map.  If a
navigator tries to read a map of an area with which they are not
familiar, they clearly understand only (PC + 4 × Rank)\% of what they
would have had they known the area.  If the map is inaccurate, it is
unlikely that the character will detect the flaw unless it was
relatively major.

\subsubsection{The navigator may place themselves on a map if they can determine the direction of two marked landmarks.}

\subsection{Map Creating}

The navigator may draw a map or chart or which shows the major
landmarks and features of the area in which they are presently located
or of an area with which they are quite familiar, or write a rutter
describing a route that they are travelling or are familiar with. At
least (2 × PC + 8 × Rank)\% of the map will be accurate, a further (2
× PC + 2 × Rank)\% will be confusing and unclear, and the rest will be
inaccurate and misleading.

\subsubsection{A navigator can competently pilot a ship of up to (25 + 25 × Rank) feet in length.}

A competent pilot of a ship has a negligible chance of damaging or
sinking a ship when faced with normal weather and sea conditions.
When a ship is not steered by a competent pilot, it is in very real
danger of experiencing an accident in choppy seas or during a storm.

A navigator can consistently maintain a ship’s speed at (50 + 5 ×
Rank)\% of its optimum speed.

If the ship is undercrewed, the optimum speed is calculated for the
ship with its current crew complement.

\subsubsection{A navigator can predict weather at sea with (PC + 5 × Rank)\% chance of accuracy.}

The GM rolls percentile dice; if the roll is equal to or less than the
success percentage, a navigator can correctly predict the weather for
the following (4 + 2 × Rank) hours. If the roll is greater than the
success percentage, the navigator’s version of the upcoming weather
becomes more and more inaccu- rate as the roll approaches 100.

\subsubsection{A navigator can sometimes recognise non-magical danger at sea before
subjecting the ship to it.}

A navigator’s success percentage to use their perceive danger ability
is (3 × Perception + 7 × Rank)\%.  If the GM’s roll is equal to or
less than half the success percentage (rounded down), the GM informs
the navigator character of the precise danger the ship is facing. If
the roll is between one- half and the full success percentage, the
navigator intuitively senses the direction and distance of the danger.
If the roll is greater than the success percentage, the navigator is
unaware of impending doom.

\end{Chapter}
