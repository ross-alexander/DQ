\begin{Chapter}{Courtier (Ver 2.1)}

A courtier may be an attendant at and frequenter of courts and
palaces, or merely a most civilised student of polished and refined
manners. Courtiers learn to survive in the Machiavellian political
situation prevalent in most courts and places of high estate, and to
be obsequious and intimidating by turns. They may learn musical or
creative skills to enhance their status, and may indulge in
manipulation and seduction.

\section{Restrictions}

A courtier pays 10\% more EP to increase their rank if:

\begin{Itemize}
  
\item their AG is less than 12,  

\item their PB is less than 15. 

\end{Itemize}

A courtier pays 10\% less EP to increase their rank if:

\begin{Itemize}

\item their AG is more than 22,  

\item their PB is more than 20.

\end{Itemize}

All modifiers are cumulative.

\section{Benefits}

A courtier gains 2 abilities at Rank 0, and 1 further ability per
Rank. All abilities are usually performed at the overall Rank of the
courtier.  However, a courtier may choose to specialise. If, upon
gaining a new Rank (or an additional ability without increasing in
rank), the courtier wishes to forego gaining a new ability, they may
specialise in one of the abilities that they already possess.  That
ability then operates at (courtier’s Rank + 1), maximum 10. A courtier
may specialise more than once with the same ability, gaining Rank + 2,
Rank + 3, etc.

Additional abilities may be gained without increasing in rank by the
expenditure of 1,000 Experience Points and 4 weeks of training per
ability.  These costs are discounted by 25\% if the courtier has
reached rank 8, or by 50\% if they have reached rank 10.

Individual Base Chances are not provided for the various courtier
skills; rather, there is a generic Base Chance of 3 × appropriate
characteristic (+ 5 / Rank), modified by the GM to reflect the
difficulty of the feat being attempted.

The abilities available to a courtier are:

\begin{Description}
\item[Bureaucracy] an understanding of organisations and hierarchies,
  how to get information, which wheels to oil, and which palms to
  grease.

\item[Carousing] the ability to socialise informally with persons of
  all social classes, without being seen as an outsider.  Also
  includes the ability to drink considerably less than most observers
  would think.

\item[Compose Music] the ability to create musical works, using
  instruments that the composer is familiar with.

\item[Dress sense] the knowledge of what to wear, how to wear it, and
  when.  This skill includes dressing formally, seductively, or
  ridiculously, as the occasion and culture require.  Also includes
  what cosmetics and scents to wear, what accessories, and even when
  to not wear things.

\item[Entertaining] the ability to organise events, ranging from
  intimate parties, to state dinners, grand fetes, and balls.  The
  courtier may supervise cater- ers, and menials, arrange the
  entertainments, and will know whom not to seat next to the Duke.

\item[Etiquette] the knowledge of what to do, how to do it, and when.
  This skill includes courtly graces, correct forms of address, and
  which fork to use for the fish.  Etiquette must be learnt separately
  for different cultures.

\item[Formal dance] a good grounding in formal courtly dances,
  particularly suitable for fetes and balls.

\item[Gaming] an understanding of the rules of such recreational
  pursuits as backgammon, chess, go, fox-and-geese, nine-mens-morris,
  and tafl, as well as various card and dice games.

\item[Hunting \& Hawking] a familiarity with the practice and styles
  of falconry, riding to hounds, and similar courtly sports.

\item[Intimidation] the ability to rule subordinates through terror,
  and knowing character flaws and weaknesses. Also includes a good
  general grounding in methods of personal manipulation, such as
  blackmail.

\item[Oratory] presenting a point of view or a set of information in a
  formal and forceful manner, to an audience.  Includes rhetoric and
  declamation, and also the ability to handle interjection and
  questioning.

\item[Play an instrument] the ability to play one musical instrument;
  the music taught will tend to be mostly formal and structured.  This
  ability may be taken several times with different instruments. A
  courtier can usually play similar instruments to the ones they have
  chosen at (Rank / 2).  A Singer is one who selects Voice as their
  instrument.

\item[Poetry] creating poetry, often of formal and highly complex
  structure.

\item[Seduction] see below. 

\item[Simulate emotions] the ability to keep careful check on the
  emotions being displayed, so as to deny observers information (such
  as when playing poker), or to give false information (feigned
  surprise, apparent pleasure).

\end{Description}

\section{Seduction}

Whilst seduction may be used to entice an entity who is compatible
with the seducer into a sexual relationship, it may also be used to
create a sense of friendship and trust, even with a being not sexually
compatible with the seducer.  The skill mostly consists of flattery
and gentle coaxing, and a seducer will greatly benefit from being
skilled at etiquette, dress sense, dance, playing music, or whatever
is appropriate to the type of seduction undertaken. Seduction is not a
rapid skill, requiring hours or even days to achieve the desired
result.

Often there is no skill check made since the GM will decide the
results of the seduction based on the character’s Rank and the way the
Player describes the attempted seduction. If a Base Chance is used, it
is seducer’s PB (+ 10 / Rank), modified by the GM to reflect the
difficulty of the seduction. If the attempt succeeds the seduction is
generally suc- cessful. If the attempt fails but is close to the Base
Chance the seduction may be attempted again, at a later time. A
particularly high roll indicates that the target is unimpressed or
repulsed by the seducer.  Player Characters are not bound by the
result of seduction attempted on them, but the GM should give them
strong hints as to how their character feels about the seducer.

\end{Chapter}
