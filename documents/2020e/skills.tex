\begin{Chapter}{Skills}

A character may acquire and refine skills during a campaign. They can
hone their talents in a series of interrelated non-magical and
quasi-magical abilities, which combine to form a single skill. A
character’s degree of talent is measured by their Rank in a
skill. They begin with the simplest abilities at the lowest Ranks, and
gain the more difficult ones as they progress through the Ranks. Their
percentage chance of successfully performing tasks associated with a
skill will increase as their Rank becomes higher.

The possession of a skill does not necessary imply any character
traits associated with that skill.

\section{Acquiring and Using Skills}

The rudiments of a skill are learned by dint of hard practice and
diligent study. A character must spend a good deal of time and effort
before they can use a skill at novice level (Rank 0).  The character’s
ability with a skill can improve only if they continue to work with it
during and between adventures.

Any skill may be acquired at Rank 0 at a variable cost of Experience
Points and 8 weeks of game time.

All eight weeks must fall within a period of six game months.  Time
spent on adventure may not count toward the necessary eight weeks.

The method by which a character learns a skill affects the Experience
Point cost to acquire that skill or to increase the character’s Rank.

If the character is taught by someone of greater Rank in the skill,
decrease any Experience Point cost by 10\%.  If the character learns
from a book, verbal descriptions, or practises with someone of equal
or lesser Rank in the skill, any Experience Point cost is
unmodified. If the character practices with no useful outside
assistance, any Experience Point cost is increased by 25\%. The
availability of qualified teachers, and the fees they charge the
character for their services, are left to the discretion of the GM.
Some skills have additional requirements (e.g.  literacy) before
learning some ranks.  Check each skill for details.

\subsubsection{A character may attempt to employ a skill any number of times during a
day.}

The use of a skill does not, in and of itself, prevent a character
from using the same or any other skill immediately afterwards.
However, a character might suffer adverse effects (for example, lose
Fatigue Points) while executing a skill, which would inhibit their
ability to act.

\subsubsection{The use of a skill is rarely automatic}

A character usually has a chance of failure when using a non-magical
skill.  Unless the ability is described as an exception to this rule,
the maximum chance to succeed with it is never greater than 90 (+
Rank)\%.  A character always fails to use an ability if the roll is
greater than the modified chance or 100 (regardless of Rank).

\subsubsection{Some of the abilities associated with the various skills are
quasi-magical.}

The following are the only quasi-magical abilities to be found in the
skills section: Alchemist, Astrologer, Healer, Herbalist, Ranger Bump
of North.

\subsection{Supervision of subordinates}

The possessor of a Skill, other than an Adventuring skill, is able to
supervise the work of subordinates in that Skill.  The supervisor may
instruct and supervise a number of subordinates equal to their Rank.
Subordinates must be practising the same Skill as their supervisor and
may themselves be supervising underlings, thus creating a “chain of
command”.  A subordinate may be replaced by a work-gang.  A work-gang
is a group of up to ten labourers working as a team.  Labourers may
not supervise others.  A character need not supervise their maximum
number of subordinates or labourers, and may work in proportion to
their unused supervision capacity.

\subsubsection{Example}

A character with Rank 6 in Artisan (Carpenter), may instruct up to 6
other Carpenters or 6 work-gangs (up to 60 labourers), or some
combination thereof.  If they were supervising 2 Carpenters and 1
work-gang, they would only be using half their supervision capacity,
and could themselves work about half of the time.

\subsection{Expert Knowledge}

The possessor of a skill, other than an Adventuring skill, also gains
an in-depth knowledge of the field associated with their skill.  This
is equivalent to having Knowledge in that skill

\section{Knowledge (area)}

This is a skill that can be taken many times — once for each area of
knowledge.  A character with this skill knows most of the common lore
and traditions concerning their chosen area. An area may include: a
particular city or territory, a culture, an historical period, or a
race, or species. In addition, an area of knowledge may be taken from
the Philosopher skill. If this is done, the area is equivalent in size
to a Sub-field, and any Subfields except Advanced, Experimental or
Ancient are available as areas of knowledge.

A character is limited to the knowledge available to their
culture. The knowledge held by the character may not be entirely
factual, and may contain certain popular misconceptions or
superstitions.  This skill mostly gives the character a much wider
general knowledge about their area, some history of it, and perhaps
some biographical knowledge of famous figures associated with it, both
historical and contemporary.  This skill is entirely one of knowledge,
and confers no special ability to perform a craft or trade.

Generally there is no success percentage; the GM simply gives far more
information regarding a certain topic to a character who has knowledge
of that area.  If there is doubt as to whether or not a character
should know something from their specific area, the Base Chances are:

\begin{dqtblr}{colspec={Xl}}
Rarity of Information	& Base Chance  \\
Common 			& WP + 70\%  \\
Uncommon 		& WP + 40\%  \\
Rare or Obscure		& WP + 10\% \\
\end{dqtblr}

These chances may be further modified by the GM to reflect the
individual rarity of the knowledge.  A character will not know the
theories behind the lore.

If a character learns an area of Knowledge that is also a Philosopher
Sub-field, and that character is, or becomes, a Philosopher, the area
of Knowledge may be used as the appropriate Sub-field.  See the EP
cost table note A (§55.2) for details on Ranking.

\end{Chapter}
