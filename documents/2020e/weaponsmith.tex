\begin{Chapter}{Weaponsmith (Ver 1.1)}

50.1 Restrictions 

The  skill  is  related  to  that  of  armourer,  and  a 
weaponsmith  who  is  a  more  skilled  armourer  ex-
pends  only  three-quarters  of  the  necessary  Experi-
ence  Points  to  acquire  or  improve  this  skill.  The 
reverse is also true. 

A weaponsmith’s progress in their skill is inhibited 
by  a  low  Manual  Dexterity,  and  aided  by  a  high 
Manual  Dexterity.  A  weaponsmith  has  an  in-
creased Experience Point cost of 5% for each point 
of  Manual  Dexterity  less  than  16.  A  weaponsmith 
decreases  their  Experience  point  cost  by  5%  for 
each point of Manual Dexterity  greater  than 20. A 
weaponsmith  will  have  great  difficulty  passing 
their  apprenticeship  if  their  Manual  Dexterity  is 
less than 10. 

50.2 Benefits 

A  weaponsmith  acquires  one  ability  every  two 
Ranks.  The  character  begins  with  one  of  the 
following abilities at Rank 

0.   All abilities can be performed skillfully. 
1.   Make and maintain swords including daggers. 
2.   Make and maintain hafted weapons. 
3.   Make and maintain thrown weapons. 
4.   Make and maintain pole weapons. 
5.   Make and maintain missile weapons. 
6.   Make and maintain entangling weapons. 
7.   Make and maintain experimental weapons. 
8.   Make and maintain siege engines. 
9.   Make and maintain shields. 
Additional abilities may be gained without increas-
ing in rank by the expenditure of 5,000 Experience 
Points  and  4  weeks  of  training  per  ability.  These 
costs  are  discounted  by  25%  if  the  weaponsmith 
has reached rank 8, or by 50% if they have reached 
rank 10. 

A  weaponsmith  can  build  increasingly  more 
effective weapons as their Rank increases. 

For every Rank that a weaponsmith achieves, they 
may  create  weapons  that  have  an  increased  base 
chance of 1%. For every Rank divisible by five that 
a weaponsmith achieves, they may create weapons 
that  cause  an  extra  point  of  damage.  These  two 
effects are not cumulative. 

For example, a Rank 8 weaponsmith may construct 
a weapon with a Base Chance increased by 3% and 
a  Damage  Modifier  increased  by  1,  or  a  weapon 
with a BC increased by 8% and no increase in DM. 

Note:  The  weapons  statistics  as  shown  in  the 
weapons  chart  are  manufactured  at  an  effective 
Rank  of  0 i.e.  they  are  the  mass-produced  variety. 
They  may  have  been  manufactured  by  a  weapon-
smith  of  greater  Rank  than  this, but  the  skill  level 
used was elementary. 

The  time  and  cost  required  for  a  weaponsmith 
to construct a weapon is dependent on the Rank 
that is used, and the type of weapon. 

A Weapon may be manufactured at any Rank up to 
the weaponsmith’s Rank. 

1.  The  time  required  is  (10  ×  (Effective  Rank  + 
DM)) hours, with a minimum of 10 hours. 

2.  The  cost  is  80%  of  the  Base  Cost  as  shown  in 
the  weapons  table  ×  (1  +  effective  Rank  +  DM 
increase) silver pennies. 

3.  For  every  rank  that  a Weaponsmith  has  beyond 
the effective rank of the of the weapon, they reduce 
the  time  required,  as  given  above,  by  5%.  For 
example, a Rank 8 weaponsmith churns out a Rank 
0 weapon in only 6 hours, or produces a +1 Dam-
age sword, no BC modifier, in 51 hours rather than 
60. 

50 WEAPONSMITH 

A weaponsmith is treated as a merchant of their 
weaponsmith  Rank  when  attempting  to  buy  or 
value weapons which are part of their abilities. 

If the equipment concerned is unfamiliar, then they 
operate  as  a  merchant  of  half  their  Rank  (rounded 
down). 

50.3 Costs 

A  weaponsmith,  with  the  exception  of  some 
missile weapons, can only perform  their skill in 
a properly maintained workshop. 

It  costs  2000  silver  pennies  to  construct  a  work-
shop and 500 silver pennies per year to maintain it 
with tools and materials. A basic tool kit will cost 
100  +  (100  ×  Rank)  silver  pennies.  A  workshop 
may  be  rented  at  a  cost  of  10  silver  pennies  per 
day. 

50.4 Silvering Weapons 
A  weaponsmith  may  incorporate  an  additional 
metal  during  the  manufacture  of  a  weapon.  This 
improves  the  appearance  of  the  weapon,  and  may 
provide  other  benefits  as  given  elsewhere  in  the 
rules. 

Cost is 80% of Base Cost as shown in the weapons 
table  ×  (metal  factor  +  effective  rank  +  DM  in-
crease).  The  metal  factor  is  1  for  cold  iron  weap-
ons,  10  for  silvered,  120  for  gilded,  and  180  for 
truesilvered. 

For example: a truesilvered rank 10 Hand \& a half 
with +2 DM costs (80% × 85) × (180 + 10 + 2) = 
13,056sp  or  a  gilded  Rank  6  Great  Axe  with  +1 
DM and +1% SC costs (80% × 30) × (120 + 6 + 1) 
= 3,048sp. 

\end{Chapter}
