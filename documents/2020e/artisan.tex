\begin{Chapter}{Artisan (Ver 2.0)}

Artisan is not a skill in itself. It is a heading under 
which  many  craft,  trade  and  service  skills  may  be 
grouped,  as  they  all  function  in  a  similar  manner 
within  the  rules.  Any,  or  indeed  all,  of  the  skills 
listed  below  may  be  learned,  but  each  is  learned 
and ranked as a separate skill. Knowledge of any of 
the  artisan  skills  confers  no  benefit  with  regard  to 
learning or ranking another. 

32.1 Artisan Skills 
The  most  common  skills  under  the  heading  of 
artisan are: 

Apiarist bee breeder, keeper, honey collector. 

Artist, Painter formal, perspective painting. 

Artist, Sculptor sculpture design and construction. 

Barber  /  coiffeur  simple  hairstyling  through  to 
massive  structures  of  hair,  wire  and  glue,  made  to 
resemble ships in full sail, castles, etc. 

Basketmaker basket weaving, making wicker. 

Blacksmith  iron  smelting  and  fashioning,  simple 
founding.  

Brewer  brewing  beer,  ale,  stout,  mead,  creating 
new recipes. 

Brickmaker/bricklayer  mixing  the  ingredients  for 
bricks,  using  brick  moulds,  checking  integrity, 
making  brick  constructions,  designing  paving 
stones. 

Butcher  killing,  butchering  and  preparation  of 
animals. 

Carpenter / cabinetmaker joints and wood-joining, 
woodworking, making basic wooden constructions, 
wooden furniture. 

Calligrapher  /  illuminator  fancy  script,  book 
illustration, drafting official documents. 

Cartographer  /  chart  maker  map  and  sea  chart 
making and copying. 

Cartwright  /  wheelwright  basic  wagon  and  cart 
design, wagon, cart and wheel construction. 

Carver  /  bone  /  etching  /  wood  carving,  engrav-
ing, woodcuts, plates for printing. 

Caster  /  pewterer  /  tinsmith  complex  and  intri-
cate casting by sand, mould or "lost wax" methods. 
Making and casting pewter and smelting and fash-
ioning of soft base metals. 

Chandler / lampmaker design and construction of 
lamps and lanterns. 

Charcoaler  making  charcoal  from  partially  burnt 
wood and bones. 

Cheesemaker  turning  milk  into  curds  and  whey, 
pressing curd to form cheese, creating new recipes. 

Clothmaker  /  fuller  /  weaver  material  collection, 
cleaning,  spinning,  weaving,  hammering  in  dirt  to 
soften, cloth design and creation. 

Cobbler  /  cordwainer  shoe  and  boot  design  and 
construction. 

Cook  /  baker  food  preparation  and  cooking,  bak-
ing, pastry making, creating new recipes. 

Cooper  design  and  construction  of  barrels  and 
casks. 

Dyer / inkmaker extraction of natural dyes, mixing 
of mineral compounds to make inks and dyes. 

Farmer/gardener  ploughing,  planting,  tending, 
harvesting, food crops or ornamental plants. 

Fisher fishing, basic net repair, fish identification. 

Glass-blower  glass  mixing,  blowing,  window 
construction, staining. 

Gold  /  silversmith  smelting  and  fashioning  of 
gold, silver, platinum and other precious metals. 

Hatter / milliner design and construction of men’s 
and women’s hats. 

Hunter / trapper use of gin, or other animal traps, 
skinning, animal collection. 

Husbander breeding, raising, tending of animals. 

Lapidary  gem  and  semi-precious  stone  cutting, 
polishing, finishing. 

Leatherworker  making  of  leather  into  garments 
and articles such as saddles. 

Locksmith  design  and  construction  of  simple 
locks. 

Lumberjack  tree  felling,  hewing,  sawing  for 
planks, replanting. 

Mason  stone  quarrying,  cutting,  finishing  and 
fitting. 

Miller milling grains into flour, millwheel use. 

Miner quarrying, prospecting, tunnelling, not mine 
design. Musical instrument maker design and con-
struction.  

Papermaker  plant  collection,  pulping,  screening, 
drying, grading, creating new recipes. 

Perfumer  extraction  of  scents,  perfume  recipe 
creation, perfume mixing. 

Plasterer mixing and application of plaster. 

Potter  clay  collection  and  mixing,  pottery  design 
and construction, firing, glazing. 

Printer  /  bookbinder  setting  printing  type  and 
plates, press operation, binding books. 

Roofer  /  thatcher  material  collection,  bundling, 
binding, attaching roofs of thatch, sod, or tile. 

Rope  /  netmaker  plant  collection,  unravelling, 
winding and braiding, net design and construction. 

Rug  /  carpetmaker  pattern  design,  material  col-
lection, weaving, and finishing. 

Sail/tentmaker  sail  construction,  sewing  tents, 
tarring and waterproofing. 

Sailor  operating  small  boats  and  crewing  ships. 
Basic sail repair and knots. 

Salter  salt  collection  from  inland  sources  or  by 
evaporating seawater. 

Shipwright  boat  and  ship  construction  from  stan-
dard designs, not creating new designs. 

Tailor/seamstress  cutting,  fitting,  designing  and 
constructing, men’s or women’s clothing. 

Tanner/hideworker/furrier  cleaning,  scraping, 
preserving,  grading,  leather  or  pelts,  making  of 
preserved pelts into garments, or garment trims.. 

Tattooist  tattoo  design,  pigment  mixing  and  im-
plementation. 

Taxidermist  pithing,  preserving,  stuffing  and 
mounting of animals and trophies. 

Tinker  basic  metal  implement  repair,  knife  grind-
ing. 

Toymaker  design  and  construction  of  puppets, 
dolls and simple mechanical toys. 

Undertaker  /  embalmer  funeral  preparation  and 
celebration,  body  preservation  and  reconstruction, 
mixing embalming fluids. 

Vintner manufacture of wine, sherry, port, brandy 
and other fermented or distilled beverages, creating 
new recipes. 

32.2 Benefits 
An  artisan  becomes  increasingly  more  effective  at 
their  skill  as  their  rank  increases.  An  artisan  may 
always  work  at an  effective  Rank  lower  than  their 
true  rank.  Standard  items,  as  shown  on  the  DQ 
Equipment List, are manufactured with an effective 
Rank of 0. The artisan creating the item may have 

108 

had  a  higher  rank than this,  but  the  skill  used  was 
elementary.  Generally,  work  produced  at  a  higher 
effective Rank will appear better, be more aestheti-
cally  pleasing,  be  more  durable,  taste  better,  or 
result in a higher yield, as appropriate. 

Applicable base 
chance  
Applicable character-
istic 
Difficulty modifier 

+1% / Rank applied.  

+ 1 / 5 full Ranks 
applied.  
-0.5 / 5 full Ranks 
applied. 
+5% / Rank applied. 

Yield  
These  effects  are  not  cumulative,  but  the  effective 
Rank  used  may  be  spread  between  these  aspects. 
The  benefits  of  the  improved  quality  will  only 
accrue if the skill (or item created) is used correctly 
and in appropriate circumstances. 

Example 
A  seamstress  with  Rank  8  in  her  skill 
creates a ball gown at an effective Rank of 8 for a courte-
san.  She  must  make  the  gown  out  of  very  high  quality 
cloth (of an appropriate type) and can either create a gown 
which  confers  +8%  on  reaction  rolls  or  one  that  gives  +1 
PB and +3% on reaction rolls, provided that the courtesan 
wears it  both correctly and  in a situation  for which  it  was 
designed. 

Example 
A carpenter with Rank 7  may build a  door 
(to  resist  the  gentle  ministrations  of  adventurers)  that 
either reduces their chance of kicking it down by 7% or is 
half a difficulty factor harder to kick in and further reduces 
their chances by 2%. 

A  fisher  with  Rank  10  may  catch  150% 
Example 
(100 + (Rank 10 × 5%)) of the normal amount of fish, in a 
day’s fishing. 

No  more  than  one  artisan  bonus  may  applied  to  a 
specific  Base  Chance  or  Difficulty  Modifier,  be 
gained  to  any  one  characteristic,  or  be  added  to  a 
Yield. If there is a conflict the better of the bonuses 
may be employed. 

Example  
If the courtesan in the example above were 
to wear her Rank 8 (+1 PB, +3%) ballgown in conjunction 
with a tiara she had made at Rank 5 that also grants +1 PB, 
she would still only gain +1 to PB. 

If  the  skill  (or  item  created)  is  used  incorrectly  or 
in  inappropriate  circumstances  then  no  bonus  will 
be gained and negative modifiers may apply. 

32.3 Time \& Cost 
The  time  and  cost  for  an  artisan  to  perform  their 
skill  is  dependent  on  the  effective  Rank  used  and 
the Base Time required for that skill. 

The  time  required  is:  (Base  Time  ×  ((effective 
Rank / 2) + 1)). 

Example 
If  the  base  time  to  make  a  Rank  0  ball 
gown is 1 week, then an Rank 7 one will take 1 week × ((7 
/  2)  +  1)  =  4.5  weeks.  The  Cost  is  (80%  of  Base  Cost  × 
(effective Rank + 1)) silver pennies. Note: This is the cost 
to the artisan, not the sale price. 

Exceptions 

Those Ranks used to gain extra yield do not count 
in  the  time  calculation.  Also,  half  of  any  Ranks 
possessed by the artisan above the rank being used 
may  be  subtracted  from  the  effective  Rank  in  the 
time calculation, to a minimum of the base time. 

32.4 Artisan as Merchant 
An  artisan  is  treated  as  a  merchant  of  half  their 
Rank  (rounded  down)  when  attempting  to  buy  or 
value equipment or materials with which someone 
with their skill would be familiar. 

32.5 Requirements 
An  artisan  will  usually  require  a  workshop,  or  at 
least  a  toolkit  to  perform  their  skill  properly.  The 
cost of tools and basic materials will vary, but will 
usually be (100 + (50 × Rank)) silver pennies. An 
artisan may not perform their skill at a higher rank 
than that of their workshop or tool kit. 

\end{Chapter}
