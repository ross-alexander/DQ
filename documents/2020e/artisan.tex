\begin{Chapter}{Artisan (Ver 2.0)}

Artisan is not a skill in itself. It is a heading under which many
craft, trade and service skills may be grouped, as they all function
in a similar manner within the rules.  Any, or indeed all, of the
skills listed below may be learned, but each is learned and ranked as
a separate skill. Knowledge of any of the artisan skills confers no
benefit with regard to learning or ranking another.

\section{Artisan Skills}

The  most  common  skills  under  the  heading  of 
artisan are: 

\begin{Description}

\item[Apiarist] bee breeder, keeper, honey collector. 

\item[Artist, Painter] formal, perspective painting. 

\item[Artist, Sculptor] sculpture design and construction. 

\item[Barber / coiffeur] simple hairstyling through to massive
  structures of hair, wire and glue, made to resemble ships in full
  sail, castles, etc.

\item[Basketmaker] basket weaving, making wicker. 

\item[Blacksmith] iron smelting and fashioning, simple founding.

\item[Brewer] brewing beer, ale, stout, mead, creating new recipes.

\item[Brickmaker / bricklayer] mixing the ingredients for bricks,
  using brick moulds, checking integrity, making brick constructions,
  designing paving stones.

\item[Butcher] killing, butchering and preparation of animals.

\item[Carpenter / cabinetmaker] joints and wood-joining, woodworking,
  making basic wooden constructions, wooden furniture.

\item[Calligrapher / illuminator] fancy script, book illustration,
  drafting official documents.

\item[Cartographer / chart] maker map and sea chart making and
  copying.

\item[Cartwright / wheelwright] basic wagon and cart design, wagon,
  cart and wheel construction.

\item[Carver / bone / etching / wood] carving, engrav- ing, woodcuts,
  plates for printing.

\item[Caster / pewterer / tinsmith] complex and intricate casting by
  sand, mould or "lost wax" methods.  Making and casting pewter and
  smelting and fashioning of soft base metals.

\item[Chandler / lampmaker] design and construction of 
lamps and lanterns. 

\item[Charcoaler] making charcoal from partially burnt wood and bones.

\item[Cheesemaker] turning milk into curds and whey, pressing curd to
  form cheese, creating new recipes.

\item[Clothmaker / fuller / weaver] material collection, cleaning,
  spinning, weaving, hammering in dirt to soften, cloth design and
  creation.

\item[Cobbler / cordwainer] shoe and boot design and construction.

\item[Cook / baker] food preparation and cooking, baking, pastry
  making, creating new recipes.

\item[Cooper] design and construction of barrels and casks.

\item[Dyer / inkmaker] extraction of natural dyes, mixing of mineral
  compounds to make inks and dyes.

\item[Farmer / gardener] ploughing, planting, tending, harvesting,
  food crops or ornamental plants.

\item[Fisher] fishing, basic net repair, fish identification. 

\item[Glass-blower] glass mixing, blowing, window construction,
  staining.

\item[Gold / silversmith] smelting and fashioning of gold, silver,
  platinum and other precious metals.

\item[Hatter / milliner] design and construction of men’s and women’s
  hats.

\item[Hunter / trapper] use of gin, or other animal traps, skinning,
  animal collection.

\item[Husbander] breeding, raising, tending of animals. 

\item[Lapidary] gem and semi-precious stone cutting, polishing,
  finishing.

\item[Leatherworker] making of leather into garments and articles such
  as saddles.

\item[Locksmith] design and construction of simple locks.

\item[Lumberjack] tree felling, hewing, sawing for planks, replanting.

\item[Mason] stone quarrying, cutting, finishing and fitting.

\item[Miller] milling grains into flour, millwheel use.

\item[Miner] quarrying, prospecting, tunnelling, not mine
  design. Musical instrument maker design and construction.

\item[Papermaker] plant collection, pulping, screening, drying,
  grading, creating new recipes.

\item[Perfumer] extraction of scents, perfume recipe creation, perfume
  mixing.

\item[Plasterer] mixing and application of plaster.

\item[Potter] clay collection and mixing, pottery design and
  construction, firing, glazing.

\item[Printer / bookbinder] setting printing type and plates, press
  operation, binding books.

\item[Roofer / thatcher] material collection, bundling, binding,
  attaching roofs of thatch, sod, or tile.

\item[Rope / netmaker] plant collection, unravelling, winding and
  braiding, net design and construction.

\item[Rug / carpetmaker] pattern design, material col- lection,
  weaving, and finishing.

\item[Sail / tentmaker] sail construction, sewing tents, tarring and
  waterproofing.

\item[Sailor] operating small boats and crewing ships.  Basic sail
  repair and knots.

\item[Salter] salt collection from inland sources or by evaporating
  seawater.

\item[Shipwright] boat and ship construction from standard designs,
  not creating new designs.

\item[Tailor / seamstress] cutting, fitting, designing and
  constructing, men’s or women’s clothing.

\item[Tanner / hideworker / furrier] cleaning, scraping, preserving,
  grading, leather or pelts, making of preserved pelts into garments,
  or garment trims.

\item[Tattooist] tattoo design, pigment mixing and implementation.

\item[Taxidermist] pithing, preserving, stuffing and mounting of
  animals and trophies.

\item[Tinker] basic metal implement repair, knife grinding.

\item[Toymaker] design and construction of puppets, dolls and simple
  mechanical toys.

\item[Undertaker / embalmer] funeral preparation and celebration, body
  preservation and reconstruction, mixing embalming fluids.

\item[Vintner] manufacture of wine, sherry, port, brandy and other
  fermented or distilled beverages, creating new recipes.

\end{Description}

\section{Benefits}

An artisan becomes increasingly more effective at their skill as their
rank increases.  An artisan may always work at an effective Rank lower
than their true rank.  Standard items, as shown on the DQ Equipment
List, are manufactured with an effective Rank of 0. The artisan
creating the item may have had a higher rank than this, but the skill
used was elementary.  Generally, work produced at a higher effective
Rank will appear better, be more aesthetically pleasing, be more
durable, taste better, or result in a higher yield, as appropriate.

\begin{tabularx}{\columnwidth}{XX}
Applicable base chance		& +1\% / Rank applied \\
Applicable characteristic	& + 1 / 5 full Ranks applied \\
Difficulty modifier		& -0.5 / 5 full Ranks applied \\
Yield				+5\% / Rank applied \\
\end{tabularx}

These effects are not cumulative, but the effective Rank used may be
spread between these aspects.  The benefits of the improved quality
will only accrue if the skill (or item created) is used correctly and
in appropriate circumstances.

\begin{Description}

\item[Example] A seamstress with Rank 8 in her skill creates a ball
  gown at an effective Rank of 8 for a courtesan.  She must make the
  gown out of very high quality cloth (of an appropriate type) and can
  either create a gown which confers +8\% on reaction rolls or one
  that gives +1 PB and +3\% on reaction rolls, provided that the
  courtesan wears it both correctly and in a situation for which it
  was designed.

\item[Example] A carpenter with Rank 7 may build a door (to resist the
  gentle ministrations of adventurers) that either reduces their
  chance of kicking it down by 7\% or is half a difficulty factor
  harder to kick in and further reduces their chances by 2\%.

\item[Example ]A fisher with Rank 10 may catch 150\% (100 + (Rank 10 ×
  5\%)) of the normal amount of fish, in a day’s fishing.

\end{Description}

No more than one artisan bonus may applied to a specific Base Chance
or Difficulty Modifier, be gained to any one characteristic, or be
added to a Yield. If there is a conflict the better of the bonuses may
be employed.

\begin{Description}

\item[Example] If the courtesan in the example above were to wear her
  Rank 8 (+1 PB, +3\%) ballgown in conjunction with a tiara she had
  made at Rank 5 that also grants +1 PB, she would still only gain +1
  to PB.

\end{Description}

If the skill (or item created) is used incorrectly or in inappropriate
circumstances then no bonus will be gained and negative modifiers may
apply.

\section{Time \& Cost}

The time and cost for an artisan to perform their skill is dependent
on the effective Rank used and the Base Time required for that skill.

The time required is: (Base Time × ((effective Rank / 2) + 1)).

\begin{Description}
\item[Example] If the base time to make a Rank 0 ball gown is 1 week,
  then an Rank 7 one will take 1 week × ((7 / 2) + 1) = 4.5 weeks.
  The Cost is (80\% of Base Cost × (effective Rank + 1)) silver
  pennies. Note: This is the cost to the artisan, not the sale price.
\end{Description}

\subsection{Exceptions}

Those Ranks used to gain extra yield do not count in the time
calculation.  Also, half of any Ranks possessed by the artisan above
the rank being used may be subtracted from the effective Rank in the
time calculation, to a minimum of the base time.

\section{Artisan as Merchant}

An artisan is treated as a merchant of half their Rank (rounded down)
when attempting to buy or value equipment or materials with which
someone with their skill would be familiar.

\section{Requirements}
An artisan will usually require a workshop, or at least a toolkit to
perform their skill properly.  The cost of tools and basic materials
will vary, but will usually be (100 + (50 × Rank)) silver pennies. An
artisan may not perform their skill at a higher rank than that of
their workshop or tool kit.

\end{Chapter}
