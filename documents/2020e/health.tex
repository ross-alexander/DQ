\begin{Chapter}{Health and Fitness}

A character’s Fatigue will vary depending upon the amount of food and
rest they get compared to their activities.

A character’s Endurance may be temporarily reduced by lack of
sustenance, extreme activities, damage, or illness.

\section{Eating and Drinking}

The amount of food and water required per day is dependent on many
factors.  These include the person (endurance, weight, build,
metabolic rate and race) and the level of activity they are involved
in (light, medium, hard or strenuous).

On average 1 lb of food and 2 pints of water per day is required.


\section{Starvation}

Starvation occurs when a character does not have at least 1 nourishing
meal a day.

If a character is starved they will have their Fatigue maximum and
Endurance temporarily reduced by 1 each day. This decrease will last
until the character starts receiving proper nourishment.

A starved character’s Fatigue maximum and Endurance will recover by
1 point each day, after the first, that they receive proper
nourishment


\section{Dehydration}

Dehydration occurs when a character does not have at least 2 pints of
water a day.  This amount will increase in high temperatures by 1 pint
per 10 degrees above 20.

If a character is dehydrated they will have their Fatigue maximum and
Endurance temporarily reduced by 5 each day.  If the character
receives part of their water requirement, the penalty is reduced.
For every 20\% (or fraction) less than the daily requirement they lose
1 from FT max and EN. This decrease will last until the character
starts receiving adequate quantities of water.

A dehydrated character’s Fatigue and Endurance maximums will recover
by 5 points each day, after the first, that they receive adequate
quantities of water.


\section{Tiredness and Rest}
\label{health:tireness}\label{health:rest}
Characters have a tendency to lose Fatigue points on adventure.  A
fatigued character must rest to recover Fatigue points.  Sleep, as
might be expected, is the best way to become refreshed, but food and
rest will also help.

The Fatigue point loss and recovery rates given in these rules assume
that the character is in good health and is well fed.  If the
character is not in good condition, the GM may adjust the effects of
activity, the effects of weight carried and the rate of recovery.

\subsection{Fatigue Loss}

A character can lose Fatigue points when they engage in any activity
more stressful than a leisurely walk.

There are four classes of activity which can fatigue a character:

\begin{Enumerate}

\item Light Exercise includes moderate to brisk walking, riding slowly
  or at a moderate pace on a docile mount, etc.

\item Medium Exercise includes jogging, riding on a cantering mount,
  light construction or precision work, etc.

\item Hard Exercise includes paced running, riding at a gallop, hard
  manual labour, etc.

\item Strenuous Exercise includes constant sprinting, breakneck
  riding, and generally those actions with which the character pushes
  their body to its practical limits.

\end{Enumerate}

It is possible for a character’s actions to be more taxing than
Strenuous Exercise, which requires superhuman exertion.  This Fatigue
loss from this activity will be determined by the GM.

A character’s degree of exertion is judged each hour.

The GM should indicate to players the level of exertion of their
activities (averaging where necessary).  If the GM gives consistent
guidelines the players will be able to keep an ongoing track of
fatigue loss.

\subsection{Encumbrance}

A character is limited in the weight they can bear, and may become
fatigued if they engage in exercise.

The Fatigue and Encumbrance Table (\S\ref{table:fatigue}) lists the
maximum weight a character may carry.

A player must determine the total weight their character is carrying
if the character is to engage in light or more stressful exercise for
a significant length of time during a day.

When an entity has a Physical Strength value greater than 40, the GM
divides that value by 40.  Multiply the quotient by the entry for 40,
and add the entry corresponding to the remainder to determine that
entity’s capabilities.

\subsection{Damage}

A character may lose Fatigue by being damaged.  This may be recovered
naturally or by being healed.

\subsection{Spell Casting}

A character may lose Fatigue by using magical abilities. This may be
recovered naturally but may not be healed.

\subsection{Calculating Current Fatigue}

The Fatigue status of a character only needs to be calculated before
they enter into combat, wish to perform magic or if they perform
fatiguing activities for long periods.  To calculate current Fatigue
use the Fatigue and Encumbrance Table (\S\ref{table:fatigue}):

\begin{Enumerate}

\item Cross-reference the character’s Physical Strength and the weight
  they are carrying.

\item Read down this column until it intersects with the row
  corresponding to the character’s rate of exercise.

\item Multiply the resulting number (Fatigue points lost per hour) by
  the number of hours at this exercise level.

\item Perform this calculation once for each time one (or more) of the
  three factors changes.

\item Add each separate subtotal to determine the total Fatigue points
  expended by the character so far.

\end{Enumerate}

\subsection{Exhaustion}

If a character’s Fatigue point total is reduced below zero, they are
exhausted. An exhausted character is limited in the activities they
may choose to do and the performance of their abilities is adversely
effected.  Their Fatigue is considered zero for the purposes of
combat or magic use.

A character may choose to exert themselves after their Fatigue points
are reduced to zero until they have expended a nominal one-half their
initial Fatigue points (round down). When they reach this limit they
will collapse unless they succeed a 1 × WP check every (2 × Endurance)
minutes.

An exhausted character must sleep for as much time as they were
performing any exercise while exhausted before they may recover any
Fatigue points.

If an exhausted character wishes to engage in Strenuous Exercise, they
must succeed a separate 1 × WP check.

\subsection{Exhaustion Modifier}

The character must subtract 1 / half hour (or fraction) of exhaustion
to any base chance.


\section{Fatigue Recovery}

A character may regain Fatigue points naturally by eating a hot meal
or resting.

A character may never have a Fatigue total greater than their Fatigue
Characteristic.

A character naturally recovers Fatigue points as follows:

\begin{dqtblr}{colspec={Xc},hline{1-2}={0.8pt}}
Activity		& FT / Hour \\
Eat Hot Meal		& 2 \\
Relaxation 		& 1 \\
Nap			& 2 \\
Sleep			& 3 \\
\end{dqtblr}

\begin{Enumerate}
  
\item A character may benefit from a hot meal no more than three times
  during a 24 hour period, and each meal must be separated by at least
  4 hours.

\item A character that does not get at least 6 hours of rest and/or
  sleep per day will have their Fatigue maximum temporarily reduced by
  1 FT / hour (or fraction) of sleep under 6 hours.  This may be
  recovered at the rate of 1 FT / 4 hours sleep.

\item If a character’s Endurance is less than 10, they recover
  one-half of a FT point less per hour or meal, and if their Endurance
  is less than 5, they recover one less FT point.  However, a
  character always recovers a minimum of one-half a FT point when
  resting.

\item If a character’s Endurance is from 21 to 30, they recover an
  additional one-half of a FT point per hour or meal. Each succeeding
  ten point Endurance bracket carries an additional one-half FT point
  bonus.

\item Fatigue loss from damage may also be recovered by magical
  healing (but not the Healer skill Heal Endurance).

\end{Enumerate}


\section{Damage and Illness}
\label{health:damage}\label{health:illness}
\subsection{Effects of Low Endurance}

Unconsciousness When an entity’s Endurance reaches 3 or less, they
must make a (current EN) × WP check or fall unconscious; this WP check
is repeated every minute or if their EN changes.

An entity on 0 Endurance is unconscious, but stable.  An entity with a
full Endurance of 5 or less does not make consciousness checks. They
remain conscious until they fall to 0 or less Endurance.

Below Zero Endurance An entity on negative Endurance will lose one
point of Fatigue (Endurance when no Fatigue remains) until the
bleeding is stanched by a Healer, or until dead.  They will continue
to take damage from any further blows, spells, grievous wounds which
are bleeders, burning, etc.

When an entity is below zero Endurance they are on the very brink of
death. It takes time and skill to tell the difference between this
state and death (e.g.  empathy, DA).  GMs should not let players take
advantage of out of character information when another player’s
character is below 0 Endurance.

Death When an entity’s Endurance falls below negative one-half their
full Endurance, they are dead.  Once dead, ongoing damage (e.g.
poison or bleeding) ceases but further damage may be inflicted on the
body.

\subsection{Endurance Recovery}

There are many causes of a character losing Endurance points.
However, once lost there are two primary methods of recovering them.

\subsection{Healers and Magical Healing}

Healers, herbalists, potions, medicines and some magics may aid the
recovery of Endurance.  The exact effects can be found under the
appropriate skill or magic.

\subsection{Natural Healing}

\subsubsection{Natural Healing of Grievous Injuries}

The rate at which Endurance Points recover naturally primarily depends
on how active the injured being is.

If an entity is resting they regain 1 Endurance point every three full
days.

This rate is reduced to 1 / 4 days if the entity:

\begin{Itemize}

\item takes any further EN damage  

\item uses more than half their FT  

\item does not receive adequate nourishment 

\end{Itemize}
  
If an entity is given ministrations from a physicker’s kit, their body
requires one less day to regain an Endurance Point.

Injuries which are not quantified as Endurance point losses or
grievous injuries (e.g.  hamstrung muscles) heal at the same rate as
they do in this world.

These healing rates are based on average Endurance value of 15.  The
GM may chose to increase the healing rate if an entity’s full
Endurance is very high or decrease it for a low Endurance entity.

\subsection{Potions \& Unconscious Patients}

An entity cannot drink a healing potion when they are unconscious or
below zero endurance but one can be massaged down their throat. The
chance of doing this is equal to the Manual Dexterity + Perception of
the person administering the potion, or if a healer, 90 + Healer
Rank. If successful then D10 per 10 points of the healing potion’s
curing (round down) will be received. If the person fails the roll,
the potion is wasted, but no harmful effects occur to the patient.

\subsection{Grievous Injuries}

Endurance loss resulting from specific grievous injuries may not be
healed separately from the underlying specific injury.  When the
specific injury is fully cured the related endurance is recovered
automatically.

Major injuries take a long time to heal and some will not heal
naturally but require a healer.  Here are guidelines for the healing
requirements of some common major injuries.

Broken bones will knit in 4 weeks for a simple fracture, or up to 10
weeks for a compound fracture.  A bone must be properly set before the
bone may knit together.

Internal injuries an entity will usually die from internal injuries.
If the patient is comfortable, unmoving, and kept alive by a healer or
physician, internal injuries will heal 1 Endurance point per week

Open wounds will heal at half the normal rate, provided that they are
kept free of infection. Open wounds will leave scars.

Removed body parts will not regrow naturally.  However, the remaining
wound will heal over at quarter the normal rate, provided it is kept
free from infection.

Magical healing of Specific Injuries Healers and certain magics may
heal specific injuries. The time taken and effects of these magics may
be found under the appropriate skill or magic.


\section{Infection}
\label{health:infection}
If a character is wounded there is the possibility that they have
become infected as a result of their wounds.

An Infection Check must be performed to determine whether they are
infected or not.

\subsection{Becoming Infected}

The chance of becoming infected depends on the entity’s health, the
type of injury, and the environment the entity is in.   Modifiers are
cumulative one is applied from each category:

There is a wound which is ... \\
\begin{dqtblr}{colspec={Xr}}
Dirty 			& +20\% \\ 
Heavily contaminated	& +50\% \\
\end{dqtblr}

The environment is ... \\
\begin{dqtblr}{colspec={Xr}}
Dry		& −5\% \\  
Humid		& +20\% \\
\end{dqtblr}

The average temperature is ... \\
\begin{dqtblr}{colspec={Xr}}
Below 0		& +20\% \\
1 -- 5		& +10\% \\
30 -- 40	& +10\% \\
Above 40	& +20\% \\
\end{dqtblr}

Some specific grievous injuries also increase the chance of infection.

\subsection{Effects of Infection}

An entity with an infection will be slowly poisoned by the infection.
The damage is [D - 5] Endurance per day, until the infection is cured.
An infected wound will not heal until the infection is cured.

\subsection{Curing Infection}

There are two ways to recover from infection.  The first is to tough
it out.  The second is to be healed by a healer.

\begin{Description}
  \item[Toughing it out] An infected character may make a 1 ×
    Endurance check every day to recover naturally.

\item[Healing] An infected character may be cured by the arts of a
  Healer or by magic. The rank at which this is possible, and the
  chance of success can be found under the appropriate skill or spell.
\end{Description}


\section{Conception}
\label{health:conception}
The natural conception chances for character races are:

\begin{dqtblr}{colspec={Xr}}
Dwarf		& 3\% \\
Elf		& 1\% \\
Halfling	& 4\% \\ 
Hill Giant	& 2\% \\
Human		& 6\% \\
Orc		& 10\% \\
Shapechanger	& 5\% \\
\end{dqtblr}

Checks against the relevant chance should be made no more often than
once per 48 hours of appropriate activity.
\end{Chapter}
