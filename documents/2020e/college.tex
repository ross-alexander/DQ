\begin{Chapter}{College Magic}

\section{Introduction}

This section includes those spells and rituals that are common to all
Colleges. These spells and rituals are still specific to a College so
an Adept can only learn them from another Adept of the same College.

\section{Counterspells}

Counterspells act to increase Magic Resistance and defeat the workings
of other magic.  Each college has two of these spells: a General
Knowledge Counterspell, and a Special Knowledge Counterspell.  These
are specific to the college — a Fire College Special Counterspell will
not affect the workings of any Earth College spell, nor would it
affect a General Knowledge spell of the Fire College.  Adepts learn
both Counterspells of their own college as part of their General
Knowledge.

\begin{spell}{Counterspell}
\range{25 feet + 25 / Rank}
\duration{(D10 + 5) minutes + 1 / Rank}
\multiple{\\
100 – General Knowledge Counterspell \\
200 – Special Knowledge Counterspell}
\basechance{40\%}
\resist{Passive}
\storage{Investment, Ward, Potion, Magical Trap}
\target{Entity, Object, Area}
\begin{effects}
There are several distinct uses for a Counterspell. They are: 

\begin{Itemize}
  
\item If cast upon an entity or object the target adds 30 (+ 3 / Rank)
  to their Magic Resistance when resisting the type of magic to which
  the Counterspell applies.

\item If cast upon an area the Counterspell affects a space 15 feet in
  diameter.  All targets within the area gain the magic resistance
  bonus detailed in \#1 above and additionally no one within the area
  may cast a spell of the type affected. A double or triple effect
  cast may increase the area of effect to 25 feet or 35 feet
  respectively.

\item If a Counterspell of the appropriate type is cast over an area
  under the effects of a Ward, then that part of the warded area is
  temporarily deactivated.  When the duration of the Counterspell
  ends, the Ward will become active again.

\item An Adept may use a Counterspell to dissipate a spell that they
  have cast.  They must direct the Counterspell at the specific spell
  effect that they wish to remove.  In the case of area effect spells
  it is sufficient to cast one Counterspell within the area — the
  entire area need not be covered.  One Counterspell will dissipate
  one spell.

\item Some spells may be removed by any Adept casting the appropriate
  Counterspell at them. Only spells that specifically state that they
  may be removed this way can be affected. One Counterspell will
  dissipate one spell.  The Adept must specify the name of the spell
  to be removed at the time of casting.

\end{Itemize}
  
A target may only be under the effects of the Counterspells of a
single College.  The target may also occupy an area under the effects
of another College’s Counterspells.  Thus the maximum number of
Counterspells that an entity or object may gain benefit from is four:
the General and Special Counterspells of one college cast upon them,
and the General and Special Counterspell of another college upon the
area they occupy.  Counterspells of other colleges cast upon them will
obey the normal rules for queuing.  A target may only benefit from one
Counterspell against a particular spell. If there is more than one
appropriate Counterspell protecting a target the highest ranked one
will have an effect.
\end{effects}
\end{spell}

Characters may learn Counterspells from colleges other than their own,
in which case they are considered Special Knowledge spells.  The
Counterspells of other colleges are practised at Rank 0 and may not be
ranked.


\section{General Knowledge Rituals}

\subsection{Ritual Spell Preparation}

For each hour spent in preparation, the Base Chance of a spell is
increased by 3 (up to a maximum of 30 if 10 full hours are spent in
preparation).  If, at any time during the preparation, the Adept’s
concentration is broken, the entire process must be restarted from
scratch or abandoned and any time previously spent in preparation is
lost. An Adept’s concentration is always broken if combat occurs
during the ritual. The Adept may engage in no other activity while
preparing the spell.  The spell must be cast immediately upon
completing the Ritual Preparation.  The Spell Preparation Ritual is
a General Knowledge Ritual.  An Adept cannot achieve Rank with this
Ritual.


\begin{ritual}{Purification}

\duration{4 hours + 4 / Rank }
\multiple{200 }
\basechance{MA + WP + 3\% / Rank }
\casttime{1 hour }
\begin{effects}
This purification ritual takes one hour and 
confers the following benefits:  
\begin{Itemize}
\item 0 (+ 1 / 5 full Ranks) MA.  

\item 0 (+ 1 / Rank) Magic Resistance. 
\end{Itemize}
The additional MA does not count towards any EP reduction (e.g.
ranking general spells or rituals).  This ritual cannot backfire.
\end{effects}
\end{ritual}


\section{Special Knowledge Rituals}

\begin{ritual}{The Ward Ritual}
\duration{Until triggered}
\multiple{400}
\basechance{MA + 3\% / Rank}
\resist{None}
\target{Volume}
\casttime{1 hour }
\material{None }
\actions{Concentration}
\concentration{Standard}
\begin{effects}
An entity may employ Ritual Magic to set a Ward over an area which
they occupy.

A Ward is a spell which is activated by the entry or exit of objects
or entities into the volume it occupies.  Whenever an entity wants to
create a Ward, they engage in one or more hours of Ritual Preparation
to create the Ward. At the end of the preparation, they check to see
if the Ward is set by making a Cast Check.  If the Cast Check is
successful, the Ward is set.  If the Check is not successful, no Ward
exists.

It is possible to backfire from an attempt to create a Ward.  In such
cases, the spell being incorporated into the Ward backfires
immediately. This is rolled for on the backfire table as though a
normal Cast Check had resulted in the backfire.

Once the Ward is cast, the entry or exit of any object or entity in
the area occupied by the Ward (determined by the range of the spell
incorporated into the Ward) may trigger the Ward. The area that the
Ward occupies and the range of the spell incorporated into the Ward
are identical.  This means that most spells (Range 15’ + 15 rank) have
a minimum sized ward of

30’ diameter (a sphere, centred on the caster, of 15’radius).  Also,
note that many spells have an indefinite range and hence cannot be
incorporated into a Ward.  This includes spells with ranges of
unlimited, self, or touch and spells that can only affect the
Adept. Once a Ward has been triggered, it ceases to exist. It takes
full effect on the entity(s) or object(s) that triggered it, but is
dissipated thereafter.  It takes full but normal effect on the
target(s) – there is no possibility for a double or triple effect (nor
for failure or backfire).  Note that a Ward set up for triggering by
an entity or object exiting the area cannot be a targeted spell, since
the target would no longer be within range.  The exiting method of
triggering is still useful for area of effect spells that don’t have a
specific target.  All Wards emanate from the exact spot occupied by
the individual who cast the Ward (important for determining range).
Note that there are a couple of spells that can have an effect beyond
their range.

Spells that are not suitable for incorporation into a Ward are those
which require concentration, or some other action by the Adept.  A
Ward always consists of only one spell.  More than one Ward may not be
set over a specific area. Any attempt to set a Ward on an area that
overlaps another Ward will fail. The Adept will only become aware of
this if they would otherwise have been successful.

Whenever creating a Ward, the Adept must also specify under what
conditions the Ward will be triggered.  They may decide not to limit
its effect, in which case the Ward will be triggered by anyone or
anything entering the area over which it is set, or they may limit it
to affecting specific individuals or anything in between.  Thus, an
Adept could set a Ward that would only be triggered by the entry or
exit of a troll.  If a multi-target spell is required to hit more than
one target from a Ward, then the trigger must include the number of
beings.  For example, a spell which affects three targets could be set
up to be triggered by three trolls. The instant that the third troll
entered the volume the Ward would be triggered, but prior to that any
number of lone trolls could have freely moved through the Ward.

A Ward, once it is successfully set, cannot be triggered until the
caster leaves the volume of the Ward.  Specifics of the triggering
mechanism must be something intrinsic to the object or entity (similar
to Detect Aura). Hence a Ward could be set up to be triggered by a
Rank 4 or higher assassin, but could not be constructed to trigger on
the assassin known as Mac the Knife.  In order to affect, or exclude,
specific entities, those entity’s Individual True Names may be
incorporated into the Ward, or a sufficiently detailed description so
as to identify the individual. If Individual True Names are
incorporated then there is no possible way to determine what those
names are, but a Divination would reveal the number of entities
specifically affected, or excluded.  A Ward cannot recognise a
specific object, but merely an instance of an object.  For example,
“my sword” could not be included in the triggering mechanism, but a
“magical sword with a yak-hide grip” could be.  A Ward cannot tell the
time so a Ward cannot include such phrases as “after × minutes” or “at
midnight”.  Nor does a Ward have any memory, so in cannot be set up to
be triggered by the third troll to pass.

Once a Ward is set, any entity or object which could trigger the Ward
and which enters the area occupied by the Ward is automatically
subject to whatever spell was woven into the Ward.  Only those spells
known by the caster of the Ward may be woven into the Ward and they
take effect exactly as if the caster of the Ward were present and
casting at the spot occupied by the entity when they set the Ward.
All entities or objects nearby which would normally be affected by the
spell are subject to its effects when it is cast as a result of the
Ward being triggered.  Note that for entities or objects to be
affected they must be within the volume that the Ward occupied, with
the exception of those spells which can affect beyond their range.
Targeted spells can only affect what is incorporated in the triggering
mechanism.

Wards are dispelled in one of two ways: either by a Namer casting the
appropriate Counterspell of the same College incorporated into the
Ward, or by being triggered by an entity or object.  The Adept who set
the Ward may always counterspell their own spells, and hence they can
dissipate their own wards by casting a counterspell into it. Wards
exist in perpetuity until dispelled
\end{effects}
\end{ritual}


\begin{ritual}{The Investment Ritual (Ver 1.2)}

\multiple{300}
\basechance{MA + 3\% / Rank}
\target{Object}
\casttime{Special }
\begin{effects}
This ritual allows an Adept to store a spell that they know in an
object or scroll.

\begin{Description}
\item[Creation of Invested Items] The object to be invested will often
  be in a form appropriate to the spell that it is to contain (e.g.
  Spell of Opening invested into a set of lock picks, or Spell of
  Enchanting Armour into a set of armour), and of a size appropriate
  to the rank and style of that spell.  Note that a staff engraved
  with the symbols of the Adept’s college is always considered to be
  appropriate. The item must weigh at least one ounce.

An Investment Ritual may not be performed on an object which still
carries charges of invested spells, or a shaped item, or anything made
of cold iron.  The Adept may invest any spell that they know at any
rank up to their rank in the spell.  The time taken to perform the
ritual is (Rank of Spell - Rank of Investment Ritual) days per item,
minimum 1 day.  In this time the Adept may invest up to Rank/2
(minimum 1) charges, or they may decrease the charges and save 1 day
per reduced charge (minimum of 1 day still applies). They may never
store more than Rank/2 charges in an invested item.

As a ritual which takes an extended period of time, the rules in §7.2
apply. The cost of materials used in creating an invested item is
[spell Rank (minimum 1) × charges × EM of spell / 2] silver pennies.
These ingredients are consumed progressively during the ritual, with
the last snatch of incense being burned as the success (or otherwise)
of the investment is determined.

The Adept may elect to spend more than this base cost to increase the
chance of the success of the ritual of investment.  For every 200
extra silver pennies spent on materials the base chance of success
with this ritual is increased by 1\%.

If the Investment ritual backfires, then it is as though the spell
being invested has backfired.

\item[Creation of Invested Scrolls] The Adept may instead opt to
  prepare a scroll (in a language in which they have a minimum of Rank
  8 literacy).  This takes one day per scroll, and costs only half the
  usual sum to create.  A scroll may only ever hold one charge, and
  weighs only two ounces — however a scroll case sufficient to protect
  it from the elements will weigh much more.

\item[Triggering] The Base Chance to successfully trigger an invested
  spell is the cast chance of the spell at the moment the adept
  completes the investment ritual, including all college bonuses, MA,
  magic, and environmental conditions.  Dice roll modifiers are
  applied at the moment of triggering, not investing.


The effects of a triggered spell are as if the caster were standing
there casting the spell; the triggering entity is not considered the
caster.

A spell contained in an invested item functions as any pulse cast
spell, with the usual chance of double and treble effect, and of
backfire. Any sentient entity may trigger an invested item, if it is
physically possible for them to do so, and if they have been taught
how to trigger it.  The Adept who created the item, and any Adept who
has divinated it, know how to trigger it.  Teaching someone how to
trigger an item takes 15 minutes.

Triggering always involves speech or a specific motion to target the
item, which may be perceived by a sufficiently alert observer.  An
invested item always takes a full five seconds to trigger.

Triggering a prepared scroll takes a full ten seconds and may only be
done by someone literate in the language in which the scroll is
written.  The scroll must be read aloud without interruption.  No
teaching is required to trigger a scroll.

Any item or scroll loses a charge when it is triggered regardless of
whether or not the triggering is successful.

\item[Limitations] A Namer casting the appropriate counterspell may
  drain an invested item of all magic; refer to Namer T-2 (§17.3) for
  details.
\end{Description}
\end{effects}
\end{ritual}
\end{Chapter}
