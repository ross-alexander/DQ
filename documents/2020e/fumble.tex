\begin{Chapter}{Fumble Tables}

When an attacker fumbles, they lose 10 from their 
Initiative  Value  until  the  end  of  the  next  pulse. 
Then they make a totally unmodified D100 roll.  If 
that roll is under their current Initiative Value, they 
suffer  no  further  penalty  for  their  slight  fumble;  if 
it  is  not under  their  current  Initiative  Value,  apply 
the corresponding result from the appropriate table 
below. See (§6.10). 

52.1 Consequences 
A  broken  weapon  is  useless until  repaired;  a  shat-
tered weapon is useless until reforged. Any combat 
spell on a broken or shattered weapon is dissipated. 

A  damaged  weapon  is  bent,  dented,  nicked,  or 
similarly  flawed.  You  may  still  use  the  damaged 
weapon but it does 1–2 points less damage and has 
0–20 penalty to its strike chance (GM decides these 
figures), until repaired. 

A damaged magical weapon must be repaired by a 
Weaponsmith  of  at  least  Rank  6.  A  damaged 
weaponsmithed  weapon  loses  all  its  weaponsmith 
bonuses  to  Strike  chance  and/or  Damage,  until 
repaired. 

Any  self-inflicted  damage  ignores  your  armour 
(including  magic);  it  usually  represents  bruising, 
minor  strains,  etc.  Naturally,  take  EN  damage  if 
you  have  no  FT  available.  If  a  specific  injury  is 
stated (e.g. pulled groin muscle, or broken bones), 
then healing requires a lot of time or the appropri-
ate minimum rank of Healer. 

52.2 Special cases 

An innately magical weapon  

• ignores asterisked results (*).  
• does not break.  
• may shatter or be damaged, but less often.  
• does not include any non-magical weapon merely 
under the effects of magic. 

Unarmed Combat  

•  “Shattered” = broken bone(s); lose 2 EN; useless 
in combat until healed.  
•  “Broken” = Seriously bruised; lose 2 FT; may be 
used  in  combat,  but  (until  healed)  each  successful 
blow does 2 damage points less to opponent and 2 
FT to yourself.  
•    “Damaged”  =  Ouch!  Lose  2  FT;  no  further  ef-
fect. 

Strike Chance over 100 

If  the  fumble  indicates  a  broken  or  damaged 
weapon, but your modified Strike Chance was over 
100, you have also struck your opponent; roll [Fa-
tigue] damage as normal. 

52.3 Bows and Crossbows only 

01–12  Bowstring snaps and lashes you; lose 2 EN. 

13   Bowstring snaps and lashes you in the eye; 
lose 2 EN; you are blinded in one eye for 3 
weeks or until cured by a Rank 7 Healer. A 
figure who is blind in one eye suffers the 
following subtractions: 1 from MD, 2 from 
PB, 4 from Perception and reduces their 
base chance with any missile or thrown 
weapon by 30. 

14–29  Bowstring snaps; no further penalty. 

30   Traditional Hunting accident. Clumsy re-

lease causes arrow / quarrel to fly towards a 
random “friendly” back in approximately 
the same direction as you were aiming: 
Strike Chance = Weapon BC + weapon 
bonuses + 30 - target’s defence. 

31–33  Clumsy release; bolt/arrow flies wide miss-

ing friends and foes. 

34–36  Brief twinge of pain in your arm or back; 

Lose 1 EN. 

37–39  as per 34-36, but lose 2 EN. 

51–52  Your vigorous swing causes a slight twinge. 

40–59  Dropped bolt or quarrel. 

60–99  Bowstring snaps; no further damage. 

Make 3 × EN or lose 2 FT. 

53   Make 3 × EN or pull a groin muscle, lose 2 

FT and have half Base TMR until healed. 

00   Bowstring snaps and lashes you; lose 2 EN. 

54–55  as per 51–52, but make 2 × EN. 

52.4 All Other Weapons 
The  following  results  are  generalised.  Therefore 
the  GM  is  free  to  ignore  or  downgrade  any  result 
which  is  inapplicable  to  a  specific  case.  Some 
outcomes are avoidable through a successful char-
acteristic check. 

01–09  Shattered weapon. 

10  

Shattered weapon; some slivers fly at you, 
potentially causing you a grievous injury — 
roll on the Grievous Injury Table (§51), but 
ignore any result over 13. 

11–12  Shattered weapon, flying splinters; you and 

13  

your opponent(s) lose 1 EN each. 
Playing the Roman fool? You just did your-
self an Endurance blow; fortunately you 
rolled minimum damage (but don’t forget 
the extra damage from poison, magic, etc). 

14–16  Your wild swing possibly connects with 

someone other than your intended target or 
yourself – immediately make a strike check 
at your new victim, the nearest being in 
range other than you or your intended target. 
Hope you weren’t mounted. 

17–18  Lose 1 EN. Feels as if you pulled some-

thing. 

56  

as per 53, but make 2 × EN. 

57–58  as per 51-52, but make 1 × EN. 

59  

as per 53, but make 1 × EN. 

60   No effect unless you used a A or B-class 
melee weapon against an opponent with a 
non-magical shield. In which case, you have 
spectacularly wedged you weapon into their 
shield. Make 1 × PS to immediately wrench 
your weapon out, or it will be torn from 
your grasp in the fracas. Don’t worry if you 
fail — perhaps their shield is now useless? 

61–62  Your melee weapon is stuck, caught, or 

entangled in your opponent’s armour or gear 
(and you didn’t even hurt them). Make 3 × 
PS to immediately disengage your weapon, 
or it will be yanked from your grasp in the 
fracas. 

63   You palpably hit a tree, rock, wall-hanging, 
furniture, or some other adjacent “scenery”. 
Make 3 × PS to immediately disen-
gage/extract your weapon. You may try 
again, as a future action; but perhaps you 
should prepare a new weapon instead. 

64–65  as per 61-62, but make 2 × PS. 

66  

as per 63, but make 2 × PS. 

19   Lose 2 EN. You really pulled something. 

67–68  as per 61-62, but make 1 × PS. 

20   Oops! You’ve flung your weapon in a high 
parabolic arc. Normally a flung or dropped 
weapon falls without hurting anyone — 
however, in this case, it falls on a random 
target, possibly even you, and maybe hurts 
them: Strike Chance = [Weapon’s BC] + 
[magical / weaponsmith bonuses] - [random 
target’s defence]. 

21–26  Butterfingers! Make 3 × MD to avoid your 
weapon flying 2–3 hexes in a random direc-
tion. 

27–28  Klutz! Make 3 × MD to avoid dropping your 

weapon in your hex. 

29   Whoops! You’ve caught your weapon in 

your own armour or gear. You may choose 
to automatically free it in the next pulse, in 
which case you may not attack or cast magic 
until after the end of the next pulse. Or else 
you may wish to prepare another weapon in 
your next action. 

30   Overly enthusiastic lunge. GM moves you to 
an unoccupied forward hex (make 3 × AG to 
choose your new facing) — but if no empty 
hex is available, you just tried to close on an 
opponent, who gets a free chance to keep 
you out of close. If you did close, you don’t 
have to drop any non-close weapon, but it 
may not be used to attack effectively. 

31–32  Poor balance; make 3 × AG or no offensive 

action until after the end of the next pulse. 
Stumble; make 3 × AG or fall prone. 

33  

34–35  as per 31–32, but make 2 × AG. 

36  

as per 33, but make 2 × AG. 

37–38  as per 31–32, but make 1 × AG. 

39  

as per 33, but make 1 × AG. 

40   Broken weapon. 

41–49  Damaged weapon. 

50   Momentary dizziness; make 3 × EN or you 
may not attack or cast magic until after the 
end of next pulse. 

69  

as per 63, but make 1 × PS. 

70*   Shattered weapon, if it is not at least Rank 1 

71–
73*  

weaponsmithed. 
Your weapon breaks unless you roll under 
its weaponsmith rank on D10. Indeed it 
shattered if you failed the roll by 5 or more. 

74–77  Twinge of pain. Take (D10 - rank in 

weapon) FT damage. 

78–79  as per 74–77, but also you may not attack or 

cast magic for remainder of the pulse. 

80   Your weapon flies from your grasp. You 

may choose to drop whatever is in your 
other hand; in which case, make 3 × MD to 
catch the weapon in that other hand. 

81–82  Butterfingers! Make 3 × MD to avoid your 
weapon flying 2–3 hexes in a random direc-
tion. 

83   Klutz! Make 3 × MD to avoid dropping your 

weapon in your hex. 

84–85  as per 81-82, but make 2 × MD. 

86  

as per 83, but make 2 × MD. 

87–88  as per 81–82, but make 1 × MD. 

89  

as per 83, but make 1 × MD. 

90   Broken weapon. 

91–99  Damaged weapon. 

00   Your bizarre but highly spectacular fumble 

is mistaken for an obscure martial technique. 
All engaged melee opponents hastily elect to 
neither attack or cast magic as their next 
action. If you have another action before 
they actually perform their next action, you 
may choose to run away (retreat up to full 
TMR) as your action without the need for a 
Withdrawal manoeuvre — you are no longer 
engaged with those particular opponents. 

\end{Chapter}
