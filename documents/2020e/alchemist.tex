\begin{Skill}[1.1]{alchemist}{Alchemist}

Almost all natural chemicals can be combined into a variety of useful
mixtures by expert hands.  The potions which will be in most demand by
characters will be those that affect the bodily functions of
humanoids. The effects of these potions range from stimulation and
depression of emotions to deadly poisons.  In a sense, alchemy is a
“poor man’s magic”; it is more cost-efficient in affecting the actions
of beings than the use of mana, albeit not as easily applied to the
victim.

There are five main areas of study within alchemy.  The first is that
of chemical analysis, the ability to determine the effects of
ingestion or application of a given liquid substance.  The others are:
standard chemicals, medicines and antidotes, poisons (including
venoms) and potions.  The creation of a potion requires the aid of an
Adept or a Healer.

As a character gains experience in the field of alchemy, they will
increase the efficacy of the mixtures they produce.  The character
will also decrease the cost of goods (to manufacture).

\section{Restrictions}

An alchemist must know how to read and write in one language if they
wish to advance beyond Rank 0.

\section{Benefits}
\label{alchemist:benefits}
\precis{An alchemist gains the ability to analyse chemicals at Rank 0.}

An alchemist may identify a liquid by its type (e.g.  medicine,
poison).  If the liquid is not a common one, the alchemist must spend
(110 - 10 × Rank) minutes using the proper equipment to analyse the
liquid’s type.

If a liquid to be analysed is particularly well-known to the alchemist
(e.g.  water or wine), they will recognise it almost immediately.  If
an alchemist wishes to determine the exact nature of a not readily
identifiable substance, the GM rolls D100.  If the roll is equal to or
less than (Perception + 8 × Rank), the alchemist is told the common
name of the substance in question (e.g. hemlock, quicksilver).  If the
roll is greater than the success percentage, the GM either informs the
alchemist that they are not sure or gives an incorrect answer.  The
greater the roll, the more likely the GM is to give false information.

\precis{An alchemist can injure themselves while working with dangerous chemicals.}

Whenever an alchemist uses or analyses a potentially dangerous liquid,
there is a chance that they will come in contact with some of the
substance.  The GM incorporates the accident chance into any other
alchemy-related percentile roll; should there not be one, they roll
D100.  The chance of no accident is (70 + 2 × Rank + Manual
Dexterity)\%.  If the roll is within the span of numbers for accident,
the alchemist suffers from the chemical.  A roll of 100 always causes
an accident.

\emph{Example}

An alchemist character with a Manual Dexterity of 17 and of
Rank 3 would have a 7\% chance of failure. Any roll from 94 to 100
will cause the alchemist to have an accident.

The GM will determine the exact effects upon the unfortunate
character.  The minimum damage will be from formaldehyde type
chemicals, which will cause about 1 Damage Point and causes blisters.
The maximum damage from a non-magical liquid will be from something on
the order of non-dilute hydrochloric acid, which will cause about 12
Damage Points per pulse, and possibly permanent bone and tissue
damage.  The effects of certain chemicals are described in the
following sections. Unless either the GM or the player have a fair
knowledge of chemistry, the alchemist should restrict themselves to
common liquids.

If the alchemist is dabbling with dangerous chemicals without using
the proper equipment (see \S\ref{alchemist:costs}), double the chance
of accident. If an alchemist is working in their lab they may prevent
damage due to chemicals after the first pulse (unless they are
incapacitated during the first pulse) by pouring the appropriate
counter-agent upon the affected area.

If a combination of chemicals forms a gas or a solid, the character’s
Agility value is substituted for their Manual Dexterity when rolling
for accident.

\precis{An alchemist can mix standard chemicals at Rank 3, and may add one
additional ability to their repertoire at Ranks 5, 7 and 9.}

An alchemist chooses their additional ability from the following:
medicines and antidotes, poisons (including venoms) and potions.

\precis{The ability to mix standard chemicals allows the alchemist to produce
mixtures which can prove useful on expeditions.}

An alchemist may produce well-known chemical combinations (e.g. oil
and vinegar, water and anything) at any Rank. The standard chemicals
ability allows the alchemist to perform most distillations and
extractions, and mix the simplest of compounds.

For example, an alchemist can produce Greek Fire and methane with the
standard chemicals ability.  The components for 12 ounces of Greek
Fire (enough to fill a grenado) cost 600 Silver Pennies.  Enough
methane to fill a grenado can be manufactured at a cost of 300 Silver
Pennies. If a creature is directly hit by a grenado filled with Greek
Fire, that creature will suffer [D + 7] Damage Points per Pulse until
the flames are extinguished (the virtue of Greek Fire as a weapon is
that it sticks to the target).  A partial hit will cause [D − 3]
Damage Points per Pulse; if a shield is interposed between target and
grenado, the shield catches fire, though the intended target suffers
no more than 2 Damage Points. A methane grenado creates a ball of fire
in the hex in which it detonates and the adjacent six hexes.  Any
creature in one of those hexes will suffer [D - 3] Damage Points, but
will be able to avoid further damage by exiting the fire hexes
(methane is not a persistent inflammable).

Whenever an alchemist wishes to manufacture standard chemicals, they
must spend [D + 7] hours in a laboratory and pay for the components.
The quantity mixed does not affect the time required, but an alchemist
is limited to the manufacture of one end product during a given
laboratory session.

An alchemist can produce standard chemicals for the use of local
businessmen (e.g. embalming fluid for the undertaker), and earn
between 50 and 75 Silver Pennies per full week of labour. Alternately,
they may produce chemicals which are likely to be put to illegal uses
(e.g.  a corrosive for iron) or manufacture addictives (e.g.  cocaine,
heroin).  The alchemist must discover an outlet to sell such
chemicals, and the return on the goods is up to the GM’s discretion.

The cost for a standard chemical will range from 1 Silver Penny for a
quart of flammable oil to 2000 Silver Pennies for a fluid ounce of
non-dilute hydrochloric acid.  The GM should scale the costs of other
chemicals appropriately.

\precis{Medicines and antidotes are used to cure a being suffering from either
disease, fever or poison.}

An alchemist may manufacture three types of medicine:

\begin{Itemize}

\item bactericide (remedy for disease)  

\item antipyretic (remedy for fever)  

\item salve (remedy for skin inflammation) 

\end{Itemize}

A bactericide or antipyretic must be ingested, while one dose of salve
can cover up to two square feet of skin. Salves will cure minor skin
inflammations, irritations (eg sunburn, rashes \& insect bites) and
may cure severe burns.

Whenever a being uses a medicine to counteract an affliction from
which they are suffering, the GM rolls percentile dice.  If the roll
is equal to or less than (8 × Alchemist’s Rank + User’s Endurance),
the user is completely cured. If the roll is above the success
percentage, the user subtracts 10 from their next dice-roll to see if
they naturally recover from their infection (see
\S\ref{health:infection}).  The failure of one medicine to work has no
effect upon any subsequent medicines used by a being.

A medicine costs (150 − 10 × Rank) Silver Pennies.  An alchemist can
produce up to three doses per day.

When an alchemist manufactures an antidote, they must specify the type
of poison they are negating.  Natural poisons are classified by their
source.  Thus, a snake antidote will cure all poison from snakes, and
so on. Synthetic poisons (those manufactured by alchemists) are cured
by an antidote from an alchemist of equal or higher Rank than the
alchemist who created the poison.  When a being ingests the proper
antidote, the poison in their system will no longer affect them.

An antidote costs (250 − 15 × Rank) Silver Pennies.  An alchemist can
produce up to three doses per day.

\precis{Poisons cause damage when introduced into the 
blood stream of a being.}

Poisons come from two sources: those which occur in nature (venoms
from animals and plants) and those which are created in a laboratory
(synthetic poisons).  An alchemist may distill venoms and synthesise
poisons.  A venom is distilled from either the poison sacs of a
poisonous animal (the most common being a snake), or from certain
plants.  An alchemist may distill [D − 1] doses of poison from poison
sacs.  The amount they may distill from plants depends on the type of
plant (GM’s discretion).  An alchemist requires (11 − Rank) hours to
distill one dose of venom from either source.  The cost of a poison
plant or sac is (750 + 150 × average damage per Pulse) Silver Pennies,
and there is no cost for the distillation process.

Venoms come in two forms: nerve agents and blood agents.  Nerve agents
work quickly (doing damage every Pulse) while blood agents (such as
arsenic) work over a long period of time.  The effects of slow acting
(blood agent) poisons function in the same manner as infections except
there is no roll for cure. The damage a being will suffer from a dose
of nerve agent venom is equal to the damage it would suffer from the
venom of the source animal or plant.

An alchemist may also manufacture synthetic poisons (both venoms and
paralysants) in their laboratory. A synthetic venom will do [D + Rank
− 5] damage points per Pulse and costs (1000 − 75 × Rank) Silver
Pennies to manufacture. If a synthetic paralysant is used to affect a
being, the Willpower Check of the victim is (4 × Willpower + 20 − 5 ×
Rank).  A synthetic paralysant costs 1750 − (60 × Rank) Silver Pennies
to manufacture. An alchemist can produce up to three doses of
synthetic poison per day.

\precis{Potions are created by an alchemist with the aid of either an Adept or
a Healer.}

Potions are designed to create a specific effect when imbibed by a
being.  They are manufactured in one-use doses and the entire dose
must be swallowed for the effect.

Magical potions are created by the concerted efforts of an Adept and
the alchemist (who may bethe same person).  Any spell or talent which
the Adept knows and which is designed to affect only the Adept or some
facet of their own person may be imbued into a potion. It takes two
whole days of continuous combined effort to create the potion.  It is
successfully created if at the end of the time the player rolls less
than (10 × Alchemist’s Rank) + Adept’s Rank with the spell or
talent). A roll above this indicates the potion is useless and the
process must be repeated with new ingredients.  The effect of a
successful potion for the imbiber is as if the Adept had already made
a successful Cast Check and the spell had taken effect.  The workings
of magical potions are immediate.  The cost to manufacture a magical
potion is equal to [(Experience Multiple of spell or talent × 20) -
  (Alchemist’s Rank × 10)].

An alchemist and a healer working together may create a healing potion
(again, they may be the same person).  The potions possible and their
Base Value are:

\begin{dqtblr}{colspec={Xr}}
Base Healer Ability			& Value  \\
Cure Disease				& 600 \\
Cure Fever				& 600 \\
(Graft) Skin Salve			& 650 \\
Neutralise Poison (specify type)	& 700 \\
Cure Endurance Points			& 1500 \\ 
Prolong Life				& 2500 \\
\end{dqtblr}

The time required to produce the potion is the same as a magical one,
and the equation to see if the process was successful is (10 ×
Alchemist’s Rank + 3 × Healer’s Rank).  If successfully created, the
potion will act on the imbiber as if a healer of the healer’s Rank was
attempting to heal them (any success rolls must still be attempted).
The cost to manufacture a healing potion is (Base Value − 50 ×
Alchemist’s Rank) Silver Pennies.

The duration of a potioned talent, once imbibed, is 1 hour × Rank of
Talent (minimum 1).

\section{Costs}
\label{alchemist:costs}
An alchemist will be able to better perform their skill when using the
proper equipment or when working in a laboratory.

It costs 2500 Silver Pennies to construct a lab, and 1000 Silver
Pennies per year to maintain it.  An alchemist can only manufacture
medicines, antidotes, poisons, or potions or distill venoms in a lab.
A laboratory may be rented at a cost of 15 Silver Pennies per day.

The chance of an alchemist correctly analysing a chemical (see
\S\ref{alchemist:benefits}) is increased by 10 when they perform the
analysis in a laboratory.

The GM and an alchemist player should scale costs and effects of
improved alchemical support material to the above rules.

An alchemist must purchase the components necessary to manufacture
each product.

The costs for poisons and potions are given with their rules.  All
costs given are for one creation attempt; if that attempt fails, new
ingredients must be purchased.

\end{Skill}
