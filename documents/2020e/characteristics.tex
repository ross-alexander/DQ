\begin{Chapter}{Explanation of Characteristics}

This section is an explanation of a character’s characteristics and
how they are used in the game.

All characteristics are calculated when the character is generated but
Adventurers in a world of magic can expect them to change from time to
time.  A “temporary” change indicates an increase or decrease of
limited duration to the value of a characteristic; a “permanent”
change indicates an increase or decrease of indefinite duration to the
value of a characteristic.

The first six characteristics are the primary characteristics.  These
can be increased temporarily by magic or permanently by training
(expenditure of experience points), and can be decreased temporarily
by magic or injury, or permanently by injury to the character.  These
primary characteristics can never be trained more than 5 above their
starting value, and never above racial maximum, except by unusual
magical means.

All other characteristics are secondary characteristics.  The manner
in which a secondary characteristic can be changed will be covered in
the appropri- ate explanation.

Generally, a high characteristic value indicates a character’s ability
to perform a certain task well, while a low value indicates a relative
lack of such ability.  A characteristic’s effect is almost always
translated into numerical terms for the purposes of resolving action
during play. Adventurers generally have higher characteristics than
normal people in the world, that is what makes them heroes after all.

\subsection{Effects of Characteristics}

A character develops specific skills during the game, and their
characteristics influence their base chances with these skills.
However there are also many feasible tasks that a character may wish
to perform without having a specific skill to do so.  The GM then uses
the most appropriate character- istic to generate a base chance to
perform that task.

\subsubsection{Difficulty Factors (Characteristic Multipliers)}

When a player declares that their character will attempt a task which
the GM acknowledges as dependent upon a particular characteristic, the
GM assigns the task a difficulty factor.  This difficulty factor will
be a number from 1/2 through to 5.

The greater the difficulty factor value, the easier a task will be to
perform.

The player multiplies the difficulty factor by the appropriate
characteristic, arriving at the percent- age chance of the character
performing the task.  The maximum base chance is (70 + characteristic
+ difficulty factor)\%.  The player then rolls D100, and if the roll
is less than or equal to the percentage then the character has
successfully performed the task.  If the roll is greater than the
percentage, the character has failed.  If the roll fails by at least
the value of the characteristic or exceeds the maximum base chance,
the character has failed miserably and may have injured themselves.
The GM may wish to determine the extent of the injury by how much the
roll exceeds the percentage plus the characteris- tic.


\section{Strength (PS)}

Physical Strength is a measure of a character’s muscle coordination
and strength.  The Physical Strength characteristic represents the
brute force a character can exert from the thews of their arms, the
thrusting power of their leg muscles, and their lift and weight
capacity.

\subsection{Specific Influences}


\begin{Itemize}

\item Effects of weight carried  

\item Minimum PS requirement for weapons

\item Damage 

\end{Itemize}

\subsection{Generic Uses}

\begin{Itemize}

\item Breaking objects

\item Lifting heavy or awkward objects 

\end{Itemize}

\begin{example}
Consider the sturdiness of the object and the implement being used to
break it for the former, and consider the weight and bulk of the
object plus the purchase afforded the character for the latter.
\end{example}


\section{Manual Dexterity (MD)}

Manual Dexterity is a measure of a character’s control with their
hands.  The Manual Dexterity characteristic represents the character’s
hand-to-eye co-ordination, the speed at which the character can
perform a complex task with their hands, and the ability to manipulate
their hands.

\subsection{Specific Influences}

\begin{Itemize}

\item Minimum MD requirement for weapons

\item Thievery  

\item Strike Chance in Combat 

\end{Itemize}

\subsection{Generic Uses}

\begin{Itemize}

\item Handling dangerous substances

\item Fine handicrafts \& other delicate tasks.

\end{Itemize}

\begin{example}
Consider the delicacy of the task when a character seeks the careful
manipulation or removal of an object.
\end{example}


\section{Agility (AG)}

Agility is a measure of a character’s ability to manoeuvre their whole
body and their speed of movement.  The Agility characteristic
represents the character’s litheness of body, the speed at which the
character can run, and their ability to dodge with or contort their
body.

\subsection{Specific Influences}

\begin{Itemize}

\item Tactical Movement Rate

\item Speed in combat

\item Defence

\item Most physical skills 

\end{Itemize}

\subsection{Generic Uses}

\begin{Itemize}

\item Manoeuvring 

\end{Itemize}

\begin{example}
Consider speed, distance, and complexity of the manoeuvre, as well as
the nature of any obstacles or features they are using.
\end{example}


\section{Magical Aptitude (MA)}

Magic Aptitude is a measure of a character’s ability to harness and
direct magical energies.  The Magic Aptitude characteristic represents
the character’s control over the flow of mana (the stuff of magic),
and their ability to remember spells and rituals.

\subsection{Specific Influences}

\begin{Itemize}

\item Magic Colleges have a minimum MA requirement

\item Cost of training magic  

\item Base chances of magical skills 

\end{Itemize}

\subsection{Generic Uses}

\begin{Itemize}

\item Noticing arcane mana effects 

\end{Itemize}


\section{Willpower (WP)}

Willpower is a measure of a character’s self control of mind and body,
especially in stressful situations.  The Willpower characteristic
represents a charac- ter’s ability to concentrate, their ability to
resist the imposition of another’s will upon their own, and the degree
to which their will can be used to counter their instincts (when, for
instance, the character might be attempting an action which could be
suicidal).

\subsection{Specific Influences}

\begin{Itemize}

\item Magic resistance  

\item Fear resistance  

\item Concentration checks to perform magic  

\item Recovering from being stunned 

\end{Itemize}

\subsection{Generic Uses}

\begin{Itemize}

\item Resisting suffering  

\item Persevering with boring or dangerous tasks

\end{Itemize}


\section{Endurance (EN)}

Endurance is a measure of the punishment a character’s body can absorb
before the character becomes unconscious, sustains mortal wounds, or
dies.  The Endurance characteristic represents the character’s
capacity to withstand wounds, their resistance to disease and
infection and their rate of recovery from same, and directly affects
their ability to over-exert themselves.

\subsection{Specific Influences}

\begin{Itemize}

\item Starting Fatigue  

\item Damage capacity  

\item Stunning from damage 

\end{Itemize}

\subsection{Generic Uses}

\begin{Itemize}

\item Resisting poison, infection \& disease

\end{Itemize}

\section{Fatigue (FT)}

Fatigue is a measure of a character’s physical and mental fitness.
The Fatigue characteristic represents the degree to which the
character can exert themselves before becoming exhausted, the number
of minor cuts and bruises they can take before their abilities are
affected, and the mental energy that can be used to cast spells.  This
characteristic directly reflects a character’s current level of
tiredness as it is reduced temporarily with any strenuous activity and
restored to normal with rest.  Fatigue may be permanently increased by
training up to 5 points or to racial maximum.

\subsection{Specific Influences}

\begin{Itemize}

\item Sustained activity  

\item Minor damage capacity  

\item Spell casting energy 

\end{Itemize}

\subsection{Generic Uses}

\begin{Itemize}

\item Ignoring cold  

\item Coping with missing meals or sleep 

\end{Itemize}


\section{Physical Beauty (PB)}

Physical Beauty is a measure of a character’s exterior attractiveness
(or repulsiveness) as perceived by the humanoid races.  Physical
Beauty is a characteristic representing a character’s appearance
compared to the aesthetic standards of the main sentient races.  It is
in no way a reflection of a character’s personality.  Specific
reactions to PB are also influenced by the observer’s race and gender.
The Physical Beauty values for monsters describe how that monster
appears to a character, and not to another monster of the same
race. Physical Beauty can be increased or decreased temporarily by
magic, and decreased permanently by disfigurement. It cannot be
increased by training.

\subsection{Specific Influences}

\begin{Itemize}

\item Reaction rolls 

\end{Itemize}

\subsection{Generic Uses}

\begin{Itemize}

\item Influencing NPCs 

\end{Itemize}


\section{Perception (PC)}

Perception is a measure of a character’s intuition developed as a
result of their experience. The Perception characteristic represents
the character’s ability to note peculiarities in a given situation,
their ability to deduce a person’s habits or customs from scant
information, and their general knowledge of the world.

The Perception value can be increased or decreased temporarily, and
can be increased permanently through training up to racial maximum.
Magic, certain natural or alchemical preparations, and the character’s
condition can cause a temporary increase or decrease in the Perception
value.

\subsection{Specific Influences}

\begin{Itemize}

\item Detecting ambushes or traps

\item Detecting hidden things

\item Initiative 

\end{Itemize}

\subsection{Generic Uses}

\begin{Itemize}

\item Picking up information from conversation or observation

\item Peripheral vision

\item Noticing things out of the ordinary

\item Remembering vague information

\item Making connections between new clues and previous knowledge

\end{Itemize}


\section{Tactical Movement Rate (TMR)}

The Tactical Movement Rate is the fastest speed a character can move
in combat. A character’s Tactical Movement Rate (TMR) characteristic
is based on their Agility and influenced by any weight carried or
restricting clothing. It may be temporarily modified by magic or
injury, but cannot be trained.

\subsection{Specific Influences}

\begin{Itemize}

\item Distance moved in combat 

\end{Itemize}

\subsection{Generic Uses}

\begin{Itemize}

\item comparative speeds 

\end{Itemize}

\end{Chapter}
