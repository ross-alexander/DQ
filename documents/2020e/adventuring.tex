\begin{Skill}[1.2]{adventuring}{Adventuring Skills}

These skills may be ranked as with any other skill.  The only
differences are that all characters start with swimming, climbing,
stealth and horsemanship at Rank 0, and if the skill is used
conspicuously during an adventure it can be ranked once without the
need for training time, but there must be a tutor with a similar skill
who is present to advise the character on the technique they should
employ.

\section{Climbing}

This skill allows a character to climb anything from walls to
mountains without the aid of specialised equipment, if this is at all
possible.  The Base Chance to use this skill is (4 × MD + 8 × Rank -
[structure height in feet / 10])\%.  A character using this skill
should make a roll at approximately 20’ intervals, but if the climb is
especially difficult, every 10’.  Note that the GM may modify the
formula in certain instances.

A climber suffers ([Height of fall (in feet) / 10] Squared) Endurance
Points when they fall.

Various items of equipment may be used to improve a character’s chance
of climbing as follows:
\begin{Itemize}
  
\item Climbing Claws add 15\% to BC but have no use for rock climbing
  where hands are more use.

\item Rope allows the user to climb the structure making only one roll
  but are only useful where ropes may be practically used.

\end{Itemize}

\section{Horsemanship}

A character will use horsemanship to direct animals which they ride.
A character may use their horsemanship with any animal or monster
which they would ordinarily ride (such as horses, donkeys, camels,
elephants, etc.). Enchanted or Fantastical monsters do not necessarily
fall into this category, and the GM must make rulings governing these
situations.

The character’s player will roll percentile dice whenever their
horsemanship is called into play. A character’s horsemanship is equal
to [(modified AG + WP) / 2 + Rank × 8], round down.

The type of mount a character is riding will modify 
their horsemanship as follows: 

\begin{dqtblr}{colspec={XrXr}}
Donkey		& -10	& Palfrey	& +15 \\
Mustang †	& -12	& Warhorse †	& -5 \\
Quarterhorse	& -10	& Camel		& -10 \\
Dire Wolf	& -10	& Mule		& -8 \\
Draft Horse	& -5	& Pony		& +10 \\
Elephant	& -10	& 		& \\
\end{dqtblr}

†Rating unless trained by rider; in that case, 0. 

The GM should also take into account the familiarity the character has
with the individual animal type and apply modifiers thereby (i.e. the
first time a character finds themselves atop a camel should be worth
at least an additional - 15).

A character’s horsemanship is called into play whenever they wish
their mount to perform an unusual or difficult action.  Any mount can
be directed into moving at a walking pace or even a brisk trot; an
unusual or difficult action would be to break into a gallop or charge,
jump an obstacle, etc.  During combat, horsemanship is called into
play during every Pulse to a) keep the mount controlled, b) regain
control if it is lost, and c) direct the mount to take any specific
Action.  Remember only a Warhorse can be directed to enter into Close
Combat by its rider, and all other mounts will only attack if directly
assaulted.

A successful roll will result in the mount obeying the directions of
the rider.  A roll above the modified percentage but less than the
modified percentage plus the rider’s WP indicates the mount either
does nothing or continues to do whatever it was doing.  A roll above
both of these indicates the mount will either disobey the rider, buck,
attempt to throw the rider, or some other unpleasant result.

The actual occurrence must be decided by the GM and should become
worse the farther the roll is above the modified percentage.

If the GM judges the rider has totally lost control of their mount,
the rider may take no other action until they have regained control
(presuming they manage to stay mounted).

Using horsemanship while in combat may be done in combination with any
other Action.  A trained rider receives certain abilities as they rise
in Rank:

\begin{Description}

\item[Rank 3] May use two-handed weapons
  
\item [Rank 5]May fire a missile weapon or cast a spell while moving

\item [Rank 7] May use two one-handed weapons at once

\end{Description}

\section{Flying}

Flying is the skill of performing aerial manoeuvres using magical
flying.  As a rule aerial combat is difficult. Flying is an
adventuring skill.

A character may always take off, fly, or land in an appropriate manner
and reasonable conditions, and under such circumstances no roll is
necessary. Note that landing appropriately is not precise.  The
success chance to perform a complex aerial manoeuvre with precision is
(3 × AG + 10 × Rank).  This base chance may be modified by the
following:

\begin{dqtblr}{colspec={Xl}}
Environmental conditions	& 0 to -50 \\
Type of flight used		& +10 to -50 \\
Speed				& 0 to -\% MPH \\
\end{dqtblr}

Flying into an obstacle causes up to [D + (relative speed in miles per
  hour / 10) squared] endurance damage. The nature of the obstacle may
reduce the damage.  Specific Grievous injuries (normally C class) may
also be incurred.  See Climbing (§29.1) for falling (as opposed to
flying) damage.

As a rule of thumb, an airborne clothed humanoid who falls through the
air drops 350ft in the first pulse, 650ft in the second, and 1000ft in
each subsequent pulse.

Note that a speed of one mile per hour is equal to 30 yards per minute
in the chase sequence and 1.5 hexes per pulse in combat.

A trained magical flier receives certain combat abilities as they rise
in rank.

\begin{Description}

\item[Rank 3]  May use two-handed weapons 

\item [Rank 5] May fire a missile weapon or cast a spell while moving

\item[Rank 7] May use two one-handed weapons at once

\end{Description}

\section{Stealth}

A character can use stealth to move as soundlessly and unobtrusively
as possible.

A character may use their stealth ability only if they have adequate
cover (i.e. space in which to conceal or obscure themselves) in the
area they wish to traverse, they are appropriately clad (e.g.  not in
plate armour or luminescent clothing), and they are not currently
under observation by the entities from whom they are attempting to
conceal their presence.

The GM will roll percentile dice to determine if a character is able
to use their stealth ability successfully.  The GM only makes such a
check if there is a reasonable possibility that the character could be
detected.  The GM makes one check each time the character attempts one
continuous action, or each time an unexpected change of condition has
a significant effect upon the character’s chance of remaining hidden
(e.g.  one of the entities under surveillance heads for a room which
happens to be through the doorway in which the character is
hidden). The GM may modify the success percentage.

A character’s base chance of using their stealth ability is (3 ×
Agility + 5 × Rank + Thief Rank + 2 × Spy Rank + 2 × Assassin Rank)\%.
The greatest Perception value of the entities who may be able to
discover the character using the stealth ability is subtracted if
those entities are unaware of the character’s presence, or three
times that Perception value if they are.

\section{Swimming}

This skill is required in order to perform any actions in the water.
All player characters start off with Rank 0.  This, under good
conditions, will allow the character to tread water in order to stay
afloat. The higher the rank, the more the character will be able to do
until they are at the stage where they can swim like a fish and
survive even in adverse conditions.

\subsection{Base Chance}

The base chance for swimming is PS + AG + EN + 8 × Rank and is
modified by the following (all adjustments cumulative):

\begin{dqtblr}{colspec={Xl}}
Wearing no or little clothing	& +10 \\
Encumbered (per pound)		& -1 \\
Water Temperature		& +5 to -25 \\
Water Conditions		& +10 to -25 \\
May not swim freely		& -10 to -50 \\
\end{dqtblr}

Other modifiers may be applied by GM as appropriate.  An unsuccessful
skill roll does not imply drowning (yet) but the character could be in
serious trouble.  If they are trying to float and the roll is failed
then they need to make another successful skill roll in order to stay
afloat.  Two failed skill rolls implies they are underwater, holding
their breath, without preparation.

If an Adept is attempting to cast then they can do so, within the
restrictions of their College, if breathing water or if they make a
successful skill roll.  A concentration check (3 × WP) may also be
required in adverse conditions.

\subsection{Breath Holding}

The base time a character can hold their breath is (current EN / 3 +
swimming Rank / 2) pulses rounded up.  This time is doubled if a Pass
Action is used in the previous pulse to prepare.

\subsection{Drowning}

Once that time is expired then the character must make a 5 × WP check
in order to continue holding their breath.  At the end of subsequent
pulses, the WP factor is reduced by 1 until the roll fails.

At that point the character starts drowning, taking physical damage at
a rate of D10 EN per pulse until death or rescue.  A drowning
character needs to make a 2 × (WP + swimming rank) check before being
able to perform useful activity as above.

\subsection{Sight and Communication}

The character can see PC hexes in clear water. This is halved in lakes
and rivers because of algae and silt.

Communication is by sign language, or a range of one hex if speaking.

\subsection{Movement Rates}

Swimming TMR = (Land TMR + Rank) / 3. Walking on the bottom (if
weighted) = Land TMR / 3.  Swimming is generally a hard or strenuous
activity unless the entity concerned is an aquatic.

Characters that are encumbered by non-buoyant materials descend at the
following rates:

\begin{dqtblr}{colspec={Xl}}
Unencumbered to 5 lbs	& 0 ft per pulse \\
5–10 lbs encumbrance	& 1 ft per pulse \\
10–15 lbs		& 2 ft per pulse \\
15–20 lbs		& 3 ft per pulse \\
20–25 lbs		& 4 ft per pulse \\
25+			& 5 ft per pulse \\
\end{dqtblr}

Unencumbered characters floating to the surface (e.g. if unconscious)
do so at 1 ft per pulse.

\end{Skill}
