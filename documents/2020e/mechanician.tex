\begin{Chapter}{Mechanician (Ver 2.2)}

Mechanicians are a blend of engineer and builder who possess both
design knowledge and crafting ability so that they may plan and
personally manu- facture devices.  Even without modern power sources
and techniques, mechanicians can still build quite sophisticated
devices using systems such as springs, hydraulics and wind-based motor
systems to drive well greased moving parts.  Mechanicians may also be
called on to devise locks and traps to foil the efforts of thieves.
They often practice a particular trade and are called locksmiths,
shipwrights, architects, etc.  A skilled mechanician may master
several such professions.  Mechanicians often build overly large and
complex devices that are frequently non-functional and occasionally
dangerous. Their profession is usually considered more of an art than
a science.

\section{Restrictions}

A character must be literate in at least one language at Rank 6 or
above to acquire the mechani-cian skill.

MD affects a mechanician’s Experience costs.  A mechanician pays 10\%
extra EP if their MD is less than 15 and pays 10\% less if their MD is
more than 22.

A mechanician must pay money for the upkeep of a studio or workshop,
tools, work-in-progress, and possibly guild fees.

The more complex, dangerous and experimental a mechanician’s project
is, the more likely that it is to be temperamental, expensive to
upkeep and prone to breakdowns.

\section{Benefits}

\begin{Description}
  
\item[Drafting] A mechanician may draft and use plans accurately.  A
  mechanician may draw freehand sketches and may draft, read and use
  plans and diagrams, provided that they relate to an ability with
  which the mechanician is familiar and that the mechanician is
  literate in the language used.

\item[Supervision] Many projects will require the assistance of
  artisans and labourers, as well as other mechanicians.  Mechanicians
  gain the ability to supervise subordinates who are practising either
  the mechanician skill or an artisan skill necessary to the
  mechanician’s project.

\item[Artisan] discount Many of the mechanician abilities give the
  character a grounding in an artisan skill.  A character may rank
  artisan skills that are listed under abilities they have learned, at
  half of the normal experience cost and time (round up), up to the
  same Rank as mechanician. Artisan skills are shown in the
  mechanician ability listings as [craft].

\item[Combinations] A mechanician may combine known abilities.  A
  mechanician may combine any or all of their areas of expertise in
  the design and execution of a project.  A mechanician may also
  combine their skills with other crafters to produce items. The GM
  must decide, based on the abilities possessed by the mechanician and
  other assistants, whether or not they may design and build a certain
  project.

\end{Description}


\subsection{Example}

A mechanician who knew bridge building, stoneworking, earthworks and
hydro engineering, could design and build an aqueduct that spanned a
gorge.  A mechanician who knew chronometrical engineering, fine
materials and spell containment, might design and build a “magical
trap” that has a time delay in the trigger mechanism.  A mechanician,
an armourer and a weaponsmith could combine abilities to build a suit
of plate armour with retractable blades at various locations.  A
mechanician wishing to build a waterwheel-powered mill would need
architecture, complex mechanics, stoneworking, and woodworking.  A
mechanician who knew boat building and animal and textile products
could design and build a sail-powered coracle made of leather, but not
a wooden dinghy.

\section{Abilities}

All mechanicians have certain rudimentary abilities.  At Rank 0 a
mechanician gains an in-depth knowledge of basic mechanics (including
levers, wedges, simple gears and pulleys, balances and use of ropes)
and basic foundations (including simple earthworks, digging and
shoring pits, piled stone walls and brick making and laying).  [Brick
  maker / layer].

After Rank 0 a mechanician acquires one new ability per Rank.
Additional abilities may be gained without increasing in rank by the
expenditure of 2,500 Experience Points and 4 weeks of training per
ability.  These costs are discounted by 25\% if the mechanician has
reached rank 8, or by 50\% if they have reached rank 10.

Some mechanician abilities give abstract comprehension of the theory,
design and construction techniques involved in crafting different
projects.  Others offer an understanding of materials along with a
basic practical knowledge and ability in crafting those substances.
Each ability lists the particular crafts or substances with which
knowl- edge is gained. Special abilities are fully explained in later
sections of this skill.

Mechanician knowledge is of a more practical and less esoteric nature
than that gained through equivalent philosopher fields and may be
complemented by the acquisition of philosophic knowledge.

The abilities available are: 

\begin{Description}
  
\item[Animal] and textile products includes material such as horn,
  furs and leather, natural fibres and other non-wooden plant
  products, heavy cloth and ropes.  Does not include venoms and
  alchemical extracts.  [Rope / Netmaker], [Sail / Tentmaker],
  [Leatherworker], [Tanner / Hideworker / Furrier].

\item[Architecture] unfortified buildings of any size.

\item[Bridges]  includes  suspension,  span,  swing,  humpback and floating bridges. 

\item[Carriages] wagons, carriages and coaches.  [Cartwright /
  Wheelwright].

\item[Chronometers] clocks, time-pieces and other timing devices.
  Complex locks special ability.  (See below). [Locksmith]. Complex
  mechanics includes stresses, valves, pumps, power transmission
  (complex gears, compound pulleys, pistons, hydraulics, etc.)  and
  power generation (springs, wind, water, etc.).

\item[Earthworks] complex earthworks, foundations and landscaping.
  Civil engineering (including road, ramp and town square building, as
  well as town planning).  Earthworks will be required to build most
  large structures. [Lumberjack].

\item[Fine] materials fine and delicate materials, wirepulling and
  small component manufacture.  [Gold/Silversmith].

\item[Fortifications] defensive military works. Includes a basic
  knowledge of siege warfare.

\item[Glassworking] glass mixing, blowing, window construction and
  staining. [Glass-blower].

\item[Hydro-mechanics] devices (pumps, pistons, valves, waterscrews,
  etc.), canals, sealocks, drainage, irrigation, sewage systems and
  plumbing.

\item[Metalworking] the forging and casting of base metals.
  [Blacksmith], [Caster / Pewterer / Tinsmith].

\item[Mines] mine design \& construction, pneumatic devices (air
  pumps, fans, ventilators, etc.), knowledge of air shafts,
  ventilation and basic geology.  [Miner].

\item[Optics] optical devices (telescopes, magnifying glasses,
  spectacles, mirrors, etc.), knowledge of light, optics, and lens
  making, grinding and finishing.

\item[Prosthetics] articulated artificial limbs. 

\item[Traps] special ability. (See §40.5). 

\item[Ships] new designs for boats and ships.  [Shipwright].

\item[Siege] engines offensive military machines.  Includes a basic
  knowledge of siege warfare.

\item[Spell containment] special ability. (See §40.6).

\item[Stoneworking] quarrying, cutting, finishing and
  fitting. [Mason].

\item[Woodworking] carpentry, joints and wood-joining.  Also making
  basic wooden constructions. [ter / Cabinetmaker].

\item[Experimental] engineering this ability may be learnt any number
  of times with different experimental areas. It may first be learnt
  when acquiring Rank 8.  Experimental engineering areas may include:
  aeronautics, steam, geo-thermal, gases, explosions, perpetual
  motion, vacuum, sub-marine and advanced versions of any other
  mechanician ability they already possess.

  \end{Description}

\section{Complex Locks}

\begin{Description}

\item[Rank] A Complex Lock is considered to have a Rank, which is the
  Effective Rank that the mechanician used in the construction of the
  lock.  The Rank of a lock may be less than or equal to the Rank of
  the mechanician constructing it.

\item[Time \& cost] The time to construct a Complex Lock is (11 + Lock
  Rank - mechanician Rank) hours.

The cost is (25 × Lock Rank [minimum 10]) sp. 

A mechanician may always open one of their own Locks in (12 -
mechanician Rank) minutes.

\end{Description}

\section{Trap Construction}

\begin{Description}

\item[Rank] A trap is considered to have a Rank, which is the
  Effective Rank that the mechanician used in the construction of the
  trap.  The Rank of a trap may be less than or equal to the Rank of
  the mechanician constructing it.

\item[Time \& cost] The time and cost to create a trap will vary
  greatly, depending on the complexity, size and nature of the trap.

The most commonly encountered type of mechanician trap is the
precision trap.  This is the type of small needle or blade trap that
may be set into or adjacent to locks or other precision devices.

A lock or similar device may have up to Rank / 3 (round up) traps on
or adjacent to it.

The time to build each trap is (11 + Trap Rank - mechanician Rank)
hours.

The cost is (125 × Trap Rank) sp, minimum of 50, plus the cost of
poisons, alchemical materials.

A triggered trap may be reset by any Mechanician whose Rank is at
least half that of the trap.  This will take (11 - mechanician Rank)
hours.  A trap may need refuelling.

A mechanician may disable or re-enable one of their own traps in (12 -
mechanician Rank) minutes.

\item[Triggering] The precise actions that will trigger a trap must be
  specified at the time that the trap is constructed.  Traps on a lock
  or other precision device are automatically triggered if the device
  is operated in the pre-specified manner and the traps have not been
  removed or disabled.

\item[Damage] A trap may be built that causes physical damage or
  explosively discharges its contents in a cone up to (Trap Rank + 1)
  feet wide and (10 + Trap Rank) long, or activates the mechanical
  trigger of a “spell container”.

  A precision trap that causes physical damage may cause up to [D10 +
    Trap Rank] Damage Points.  It may also be poisoned, coated with
  acid, etc., so as to cause additional damage.

\end{Description}

\section{Spell Containment}

A mechanician may use this ability to create a spell container, or
magical trap, with a mechanical trigger.

\begin{Description}

\item[Construction] A mechanician with the “spell containment” and
  “fine materials” abilities can build a mechanical device into which
  a spell can be stored.  The device is usually referred to as a
  magical trap, or a spell container.  The device is made out of
  silver, truesilver or starsilver.  Often the device is built inside
  or incorporating other materials with which the mechanician is
  familiar.

\item[Spell] Storage A single charge of a suitable spell may be stored
  in the trap or container by an Adept successfully casting a spell
  into the device after performing Ritual Spell Preparation.  A double
  or triple effect stores an enhanced spell as specified by the Adept.
  A failure has no effect.  A backfire affects the Adept as normal,
  and also results in the device being damaged so that 20\% more time
  and materials are required before another try at storing a spell may
  be attempted.  The spell to be stored must include a Storage type of
  “Magical Trap”.

\item[Triggering] The precise actions that will trigger the device
  must be specified at the time that the device is constructed.  When
  these actions are performed the spell is released.  The spell stored
  must either affect only the entity or object that triggered the
  release of the spell, or affect an area in relation to the
  device. All variable spell effects, such as direc- tion and volume
  affected, must be defined at the time of storage.  Once the spell
  has been triggered the device is useless, although metal equal to
  10\% of the cost may be recovered.  If the spell is dissipated,
  then 20\% of the cost may be recovered.

\item[Time to construct] 25 - (2 × mechanician Rank) hours.

\item[Time to store spell] (Spell Rank - mechanician Rank) hours
  (minimum 1, maximum 10).

\item[Cost] (Spell Rank (minimum 1) × Spell EM) + 100 sp.

\item[Minimum Weight] (16 - mechanician Rank) ounces.

\end{Description}
  
\end{Chapter}
