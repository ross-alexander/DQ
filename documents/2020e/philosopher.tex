\begin{Skill}[2.0]{philosopher}{Philosopher}

Philosophers become familiar with the general characteristics of their
world, within the limits of the knowledge available to their culture,
discarding many popular misconceptions. They acquire extensive
knowledge on a wide range of subjects, and are, in many ways, the
encyclopaedias and expert opinions of the medieval world.
Philosophers are also well versed in using the unusual and obscure
indexing methods employed in medieval libraries, and so may research
and answer enquiries that they do not immediately know the answers to.

\section{Library}
Any place with 50 or more books may be considred a library, for the
purposes of study.  Libraries are rated for the number of days that a
philosopher may study in them to answer any particular question. This
rating is usually equal to Books divided by 50. Once a philosopher has
exhausted the possibilities of a library they must either find another
and continue their study or attempt to answer the question anyway. A
day of study is 10 hours, and is full-time work. Some libraries with
specific collections may be rated higher for some Realms than
others. GMs should bear in mind that the books in some libraries will
be predominately in particular languages, and that if the philosopher
is not literate in those languages, the library may be of reduced
usefulness.

\section{Requirements}

\subsubsection{Language restriction}

A character may not become a philosopher unless they possess at least
one language at Rank 8, and are literate in that language.

\subsubsection{Books}

A philosopher must possess (or have frequent access to) at least Rank
times 10 books, written in languages that they are literate in.

\section{Structure}
\label{philosopher:structure}

The philosopher skill is designed as a tree-like structure, with
several separate Realms of knowledge, each of which has its own
Fields, which in turn, have Sub-fields.

\subsection{Realms}

These are the largest and least detailed divisions of knowledge. There
are 5 Realms of knowledge: the Social World, the Material World, the
Magical World, the Animal World, and the Plant World.

\subsection{Fields}

Realms are divided into large blocks of knowledge, called Fields.  GMs
should not need to add new Fields to the Realms, but may do so if they
wish.

\subsection{Sub-fields}

Small, and quite specific divisions of a Field, these are not limited
to only those suggested below.  A philosopher may learn almost any
sub-division of a Field as a Sub-field, with the GM as the final
arbiter.  The most common Sub-fields concern a particular race or
area within a Field.

\section{Language Benefits}

Philosophers gain a reduction in the EP costs to learn languages, in
addition to any other reductions available to the character.

Provided that the Rank of language being learnt is not greater than
their Rank of philosopher, philosophers may learn to speak the
language or to read an orthographic language at a 10\% discount.  If
the philosopher has chosen the Field of Linguistics, the discount is
20\% instead, increasing to 30\% if the philosopher has chosen the
appropriate Language Group subfield (see
\S\ref{philosopher:structure}).

\section{Knowledge Benefits}

\subsection{Realms}

At Ranks 0, 4, 7, and 10, the philosopher may learn a Realm of
knowledge.  Each Realm provides a thorough grounding in the basics
associated with it (see the individual Realms for more details).  If a
philosopher wishes to forego learning a Realm, they receive an extra 8
Sub-fields (which may be traded for Fields as below).

\subsection{Fields \& Sub-fields}

At each Rank above 0, the philosopher receives a number of Sub-fields.
They receive: at Ranks 1 to 4, 3 Sub-fields; at Ranks 5 to 7, 5
Sub-fields; and at Ranks 8 to 10, 7 Sub-fields.

3 Sub-fields may be traded for 1 Field.  Any part of the Subfield
allotment may be retained and used in conjunction with the allotment
received for further Ranks.  Once a philosopher has achieved Rank 10
they may not Rank their Skill further, but may acquire new areas of
knowledge.  A new Realm costs 8 weeks and 4000ep, a new Field 3 weeks
and 1500ep, and a new Sub-field 1 week and 500ep.

\subsection{Field Restrictions}

A philosopher may not learn a Sub-field if they have not already
learned the Field that it is part of.  They may not learn a Field if
they have not already learned the Realm that it is part of.

\subsection{Overlaps \& Connections}

In some cases it is possible to reach the same Sub-fields by different
routes.  These duplicated Fields may be treated as identical and no
benefit accrues from having the same Subfield more than once.

\section{Research Benefits}

Philosophers may attempt to answer questions put to them.  These
questions may be posed by themselves, or by other characters.  If the
philosopher does not already know the answer, their chance of success
depends on the difficulty of the question and the relevant Realms and
Fields of the philosopher.

\subsection{Difficulty}

Questions that may be answered by a philosopher fall into one of seven
categories: Automatic, Very Easy, Easy, Standard, Hard, Very Hard, and
Impossible.  The first step in determining the difficulty of answering
the question is for the GM to determine which Realm(s) the question
pertains to, and the level of difficulty of the question.

A Standard question is one of average difficulty, relative to a given
Realm, as determined by the GM.  They usually deal with a reasonably
large sub-set of the knowledge of the Realm.  If the philosopher
possesses the Realm to which the question pertains, but has no more
in-depth knowledge applicable to the question, the difficulty is as
set by the GM.  If the philosopher has a Field within that Realm that
the GM determines is relevant to the question, the difficulty
decreases by one step.  If the philosopher has a Sub-field within that
Field, and the GM determines that it is relevant to the question, the
difficulty decreases by another step.  If a philosopher does not even
possess the Realm of the question, it becomes two steps harder.  A
philosopher will immediately know the answer to an Automatic
question. A philosopher may not answer an Impossible question.

\subsection{Answers}

The accuracy of the answer that a philosopher can offer is dependent
on Rank and the difficulty of the question. To increase their
accuracy, a philosopher may also undertake a course of study.  For
each study period (the length of which is determined by difficulty),
+1\% is added to the philosopher’s Base Chance. A philosopher may, at
any time, attempt to answer the question.  The base Accuracy, Rank
bonus, and length of study period is shown on the Answer Table
(\S\ref{philosopher:answertable}).

Even though philosophers keep notes during their course of study, an
extended interruption may prove a setback. If a philosopher ceases a
course of study but resumes it within Rank weeks there are no adverse
effects. If the interruption is longer than this, then half of the
percentage amount that they had achieved from study is lost.

\subsection{Final Result}

If the question is of a yes/no nature, the Accuracy is the Base Chance
that the philosopher will arrive at the correct answer. If the
question is more open, the Accuracy is the amount of relevant
information that the philosopher will come up with.  It is also
possible that some questions (as determined by the GM) are simply
unanswerable.  If this is the case, the Accuracy becomes the Base
Chance that the philosopher will become aware of this fact.

\section{Realms \& Fields}

Each of the five Realms is listed below, along with its associated
Fields.  Some Fields are followed by a list of suggested Sub-fields.

\subsection{The Social World}

Standard Sub-fields include: Area, Race, History.  The Fields of this
Realm are:

\begin{Itemize}
\item Art \& Music — Style  

\item Ethnology  

\item Heraldry \& Genealogy — Tinctures, Furs  

\item History — Ancient  

\item Legends \& Folklore  

\item Linguistics — Language Group  

\item Philosophy \& Ethics  

\item Politics \& Customs  

\item Theology \& Mythology 
\end{Itemize}
  
\subsection{The Material World}

Standard Sub-fields include: Area, Race, History, Advanced. The Fields
of this Realm are:

\begin{Itemize}
\item Alchemy — Experimental  

\item Architecture — Experimental, Ancient  

\item Astronomy  

\item Cartography  

\item Engineering — Experimental  

\item Geography  

\item Geology \& Mineralogy — Group of Minerals  

\item Mathematics  

\item Metallurgy — Experimental  

\item Oceanography 
\end{Itemize}

\subsection{The Magical World}

Standard Sub-fields include: Area, History.  The Fields of this Realm
are:

\begin{Itemize}
\item Artefacts \& Magical Items — Shaper, Legends  

\item (Any College) — Politics, Famous People  

\item Demi-Powers — Groups, Races  

\item Deities — Pantheon, Religion  

\item Dragons — Type, Genealogy, Behaviour  

\item Elements — Any element or amalgam  

\item Fantastical Beings — Any group  

\item History \& Theory — College Divisions, Backfires

\item Naming — Structure  

\item Magical Animals — Type  

\item Magical Plants — Type  

\item Mana Zones — Places of Power  

\item Other Planes — Plane  

\item The Powers — Pacts, Invocations, Agency, Factions

\item Undead — Lesser, Greater 
\end{Itemize}

\subsection{The Animal World}

Standard Sub-fields include: Area, Type.  The Fields of this Realm
are:

\begin{Itemize}
\item Amphibians  

\item Aquatics  

\item Avians  

\item Insects \& Spiders  

\item Land Animals  

\item Magical Animals 
\end{Itemize}

\subsection{The Plant World}

Standard Sub-fields include: Area, Type.  The Fields of this Realm
are:
\begin{Itemize}
\item Aquatic Plants  

\item Flowers  

\item Grasses \& Cereals  

\item Herbs  

\item Magical Plants  

\item Root Plants  

\item Shrubs \& Bushes  

\item Trees 
\end{Itemize}

\section{Answer Table}
\label{philosopher:answertable}
\smallskip

\begin{dqtblr}{colspec={XllX}}
Difficulty	& Accuracy	& Per rank	& Period \\
Very Easy	& 90\%		& +1\%		& 1 minute \\
Easy		& 70\%		& +2\%		& 5 minutes \\
Standard	& 40\%		& +3\%		& 15 minutes \\
Hard		& 20\%		& +3\%		& 30 minutes \\
Very Hard	& 0\%		& +3\%		& 1 hour \\
\end{dqtblr}

\end{Skill}

