\begin{Tables}{Miscellaneous Tables}

\section{Fatigue, Encumbrance and Movement Charts}

\begin{tabularx}{\linewidth}{Xlllllllll}
PS		& \multicolumn{8}{l}{Weight of Load (lbs)} & Max \\ \hline
3-5		& 0	& 0	& 5	& 14	& 21	& 30	& 37	& 45	& 50 \\
6-8		& 0	& 5	& 12	& 17	& 25	& 40	& 55	& 67	& 75 \\
9-12		& 5	& 12	& 17	& 25	& 40	& 60	& 75	& 90	& 100 \\
13-17		& 12	& 17	& 25	& 50	& 60	& 80	& 95	& 112	& 125 \\
18-20		& 17	& 25	& 35	& 50	& 75	& 105	& 125	& 140	& 150 \\
21-23		& 25	& 40	& 55	& 70	& 100	& 140	& 165	& 185	& 200 \\
24-27		& 35	& 50	& 65	& 85	& 120	& 160	& 185	& 202	& 225 \\
28-32		& 45	& 65	& 85	& 105	& 140	& 180	& 205	& 230	& 250 \\
33-36		& 55	& 80	& 110	& 140	& 180	& 220	& 245	& 262	& 275 \\
37-40		& 65	& 85	& 135	& 170	& 207	& 247	& 280	& 307	& 325 \\ \hline
 & \multicolumn{8}{l}{Fatigue loss from Exercise} \\ \hline
Light 		& 0	& 0	& 0	& 1/2	& 1/2	& 1	& 2	& 3	& 5 \\
Medium 		& 0	& 0	& 1/2	& 1/2	& 1	& 1	& 3	& 4	& 6 \\
Hard 		& 1/2	& 1/2	& 1	& 1	& 2	& 3	& 5	& 6	& 8 \\
Strenuous 	& 2	& 2	& 3	& 3	& 4	& 5	& 6	& 7	& 9 \\ \hline
 & \multicolumn{8}{l}{Agility Loss in Combat} \\ \hline
Loss 		& 0	& 1	& 2	& 3	& 5	& 7	& 9	& 10	& 12 \\
\end{tabularx}

\begin{Description}
\item[Weight of Load (lbs)] The mximum weight, in pounds, that a
  character can carry (excluding clothing worn), to fall into that
  category. Note: A mount can carry weight for a character while they
  are riding.

\item[Max] The maximum load, in pounds, that a character can carry for
  a sustained period of time.

\item[Fatigue loss from Exercise] Tiredness Fatigue loss per hour of
  encumbered exercise, see §4.4.

\item[Agility Points Lost] The temporary Agility Point loss suffered by a
character toting the given weight in combat.  Use the procedure in
rule §4.4 to use this chart.
\end{Description}

\begin{multicols}{2}

\section{Tactical Movement Rate}

\begin{tabularx}{\columnwidth}{Xl}
Modified Agility  & TMR \\
< 1	& 0 \\
1 – 2	& 1 \\
3 – 4	& 2 \\
5 – 8	& 3 \\
9 – 12	& 4 \\
13 – 17	& 5 \\
18 – 21	& 6 \\
22 – 25	& 7 \\
26 – 27	& 8 \\
> 27	& † \\
\end{tabularx}

†  TMR  =  9  +  1  for  every  two  points  of  AG  over 
28, e.g. AG 32 gives 11 TMR 


\section{Overland Movement Rate} 

\begin{tabularx}{\columnwidth}{lllll}
\multicolumn{5}{c}{Rate of Exercise} \\
Terrain		& Light		& Medium	& Heavy		& Strenuous \\
Cavern		& 5/-		& 10/-		& 15/-		& 20/-	\\
Field		& 15/15		& 25/25		& 30/40*	& 35/50* \\
Marsh		& -/-		& 5/5		& 10/10*	& 15/15* \\
Plain		& 15/15		& 25/25		& 30/40*	& 40/50* \\
Rough		& 10/5		& 15/10		& 20/15*	& 25/- \\
Waste		& 10/5		& 15/10		& 20/10*	& -/- \\
Woods		& 10/5		& 15/10		& 20/15*	& 25/- \\
\end{tabularx}

The number before the slash indicates movement in miles per day on
foot; the number following the slash indicates mounted movement
(assuming horses). Rates for other animal types must be adjusted by
the GM.  The day assumes a total of 8 hours marching. Effects of
adverse weather must be adjudicated by the GM.  Any paths or roads
negative the effect of other terrain, and the Plain movement rates are
used.

(-): Movement type impossible at this exercise rate. 

* In these exercise rate categories, horses’ maximum rates will
deteriorate 33\% per day. They can travel at these rates for
approximately 4 consecutive days and then they will die.
\end{multicols}

\end{Tables}

