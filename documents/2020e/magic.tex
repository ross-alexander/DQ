\begin{Chapter}{Magic}

There are thirteen sections in Magic: 

Introduction to Magic 
How Magic Works 
How to Cast Spells 
Cast Check Modifiers 
Spell Effects 
Restrictions on Magic 
Backfires 

7.1 
7.2 
7.3 
7.4 
7.5 
7.6 
7.7 
7.8  Magic Resistance 
7.9 
7.10  The Colleges of Magic 
7.11  Magic Descriptions 
7.12  Spell Descriptions 
7.13  Storage and Entrapment of Magic 

Incorporating Magic into Combat 

7.1 Introduction to Magic 
Magic represents the effects of the unknown forces 
that shape and control the worlds. Those who have 
talent  and  knowledge  can  tap  these  energies 
(known  as  mana)  and  shape  them  to  their  own 
ends. These people are known as Mages. They are 
usually  either  revered  or  reviled  by  the  normal 
population. 

There  are  three  types  of  Magic:  Talent,  Spell,  and 
Ritual.  Talent  Magic  operates  more  or  less  imme-
diately,  while  Spell  and  Ritual  Magic  require 
preparation  before  taking  effect.  Spells  may  be 
prepared  in  seconds  or  minutes,  but  Rituals  take 
hours (and sometimes many weeks) to perform. 

There are a number of separate Colleges of Magic. 
Each represents a specific type of magic, and each 
has  a  list  of  Spells,  Rituals  and  Talents  available 
only to  Adepts of that College. Most of the magic 
detailed within these rules is College magic. 

Definitions: 

only  have  one  concentration  spell  in  effect  at  any 
time. 

Consecrated  Ground  Any  ground  that  has  been 
consecrated  to  the  “Powers  of  Light”  affects  the 
Magic Resistance of all within it. There is no Col-
lege  specifically  dedicated  to the  Powers  of  Light, 
because  they  are,  in  effect,  opposed  to  the  use  of 
magic.  Most  temples  and  monasteries  and  some 
graveyards  will  be  consecrated  ground.  Barrows, 
pagan temples and the dwellings of magical beings 
can  never  be  consecrated  ground.  Undead  and 
Necromancers  suffer  special  penalties  on  conse-
crated ground. 

Counterspell  A  type  of  spell  which  helps  to  pro-
tect  individuals  and  areas  against  the  effects  of  a 
particular College of Magic. 

Fatigue Cost The amount of energy, in the form of 
Fatigue, that a Mage must expend in order to cast a 
spell. 

General Knowledge All Colleges of Magic have a 
body of Spells, Talents and Rituals which are clas-
sified  as  General  Knowledge.  These  magics  are 
taught  to  all  Adepts  of  the  College  during  their 
initial training. 

High Mana An area that is rich in mana is referred 
to  as  a  high  mana  area.  Such  areas  are  rare,  and 
include  locations  where  human  sacrifice  is  prac-
tised  or  where  the  inter-planar  boundaries  are 
weak,  and  mana  leaks  through.  Often  mountain 
tops or clearings in jungles will contain such areas. 
They  are  likely  to  be  well  guarded  by  beasts  and 
individuals  attracted  by  the  mana,  including  a 
larger  than  usual  proportion  of  fantastical  beasts. 
Magic is easier to perform in these areas. 

Active Resistance A special type of  Magic Resis-
tance,  where  the  entity  can  choose  to  concentrate 
their  attention  on  resisting  a  magical  effect,  and 
thus  reduce  its Cast  Chance by  their Magic  Resis-
tance. Only some magic is actively resistible. 

Low Mana An area with depleted mana is known 
as  a  low  mana  area.  Most  densely  populated  or 
civilised  parts  of  the  world  are  Low  Mana,  as  are 
some  battle-scarred  areas.  Magic  is  harder  to  cast 
in low mana areas. 

Adept A member of a College of Magic is known 
as an Adept. 

Backfire If a spell or ritual is particularly incompe-
tently  cast,  unpredictable  and  often  dangerous 
effects  can  occur.  This  is  colloquially  known  as  a 
backfire. 

Branches of Magic There are 3 branches of Magic: the Thaumaturgies,
the theoretical branch of Magic including the Bardic, E \& E, Mind,
Nam- ing, Illusion and Binding Colleges; the Elementals, the
naturalistic branch of Magic that includes the Earth, Water, Fire,
Air, Ice and Celestial Colleges; and the Entities, the old, “dark”
branch of Magic that includes the Necromantic, Rune, Summoning and
Witchcraft Colleges.  The Thaumaturgies and Entities are opposed to
each other.

Cast Chance The modified Base Chance of effectively casting a spell or
performing a ritual.

Cast  Check  The  game  mechanic  whereby  a 
Mage’s  player  determines  the  result  of  an  at-
tempted spell or ritual. 

Cold Iron All solid metals that are primarily com-
posed  of  iron  ore  are  termed  Cold  Iron.  This  in-
cludes both Iron and Steel. Such metals in a liquid 
state are not “cold”. Cold Iron inhibits the ability of 
Mages to use mana. 

College  Most  magic  is  divided  up  into  numerous 
Colleges,  each  of  which  specialise  in  a  type  of 
magic  (e.g.  Fire,  Necromancy).  A  Mage  who  has 
joined a particular College is known as an Adept of 
that  College,  and  may  not  belong  to  another  Col-
lege  without  first  forsaking  all  knowledge  of  their 
previous College. 

Concentration If a spell has a concentration com-
ponent in its duration, then the Adept must concen-
trate  in  order  to  maintain  the  spell.  A  Mage  may 

Mage  Any  sentient  being  who  can  manipulate 
mana to produce (often fantastic) results (excluding 
racial Talents). A Mage must have a Magical Apti-
tude characteristic. 

Magic  Resistance  All  targets  with  a  Willpower 
value  have  the  capacity  to  resist  some  magics 
directed  against  them.  This  ability  is  their  Magic 
Resistance,  and  is  a  function  of  their  Willpower. 
Not all magic is resistible. 

Magical  Animates  Anything  that  has  been  ani-
mated, except undead, is a magical animate. Some 
Magical  animates  gain  a  magic  resistance.  Only 
those  animates  that  have  a  Magical  Aptitude  or 
Willpower  gain  a  Magic  Resistance.  Those  ani-
mates that have neither MA nor WP have no resis-
tance to magic, and in addition, may be affected by 
spells  that  affect  Entities  and  those  that  affect 
objects. 

Mana The type of energy that is used in all magic. 
A  Mage  must  draw  upon  mana  to  perform  any 
magic.  If there is no mana present, a Mage cannot 
perform any magic. 

Object  An  item  wholly  composed  of  never  living 
or  formerly  living  matter,  or  some  combination 
thereof.  Objects  do  not  have  a  Magic  Resistance 
except when they are Possessions or Magical Ani-
mates. 

Passive  Resistance  This  is  the  default  Magic  Re-
sistance  made  by  all  targets  with  willpower  and 
operates  automatically  against  all  spells  that  may 
be  passively  resisted.  It  is  possible  to  stop  pas-
sively resisting temporarily. 

Place  of  Power  Certain  places  aid  the  practise  of 
magic.  The  most  well  known  places  are  Earth 
places of power, but they exist for all the Elemental 
and Entity Colleges (excluding Rune). Such places 
are rare, and often co-exist with High Mana areas. 

21 

7 MAGIC 

Possessions  Possessions  are  those  objects  held, 
carried or otherwise within the personal area of an 
Entity. They are affected by those spells that affect 
the  Entity,  and  are  entitled  to  the  Entity’s  Magic 
Resistance. 

Resistance  Check  The  game  mechanic  which 
determines  whether  a  resisting  entity  is  fully  af-
fected by a magical effect. 

Ritual  Magic  Complex  procedures  and  techniques 
that  require  the  Mage  to  spend  large  amounts  of 
preparation  time  (and  often  ingredients)  to  com-
plete successfully. 

Special Knowledge All Colleges of  Magic have a 
body  of  complex  or  specialised  spells  and  rituals 
which  are  not  taught  to  mere  apprentices,  but 
which  are  gained  with  time  and  effort  after  the 
Adepts prove themselves worthy. These magics are 
termed Special Knowledge. 

Spell  Magic  Codified  magical  formulae  that  take 
anywhere  from  a  few  seconds  to  a  minute  to  per-
form,  require  energy  from  the  Mage,  and  which 
result in specific alterations to Natural Law. 

Talent Magic Magical  abilities that require  mana, 
but  no  energy  and  minimal  time  from  the  user. 
Many species have racial Talents. 

7.2 How Magic Works 
There  are  three  types  of  Magic:  Talent,  Spell  and 
Ritual Magic. 

Talent  Magic  is  broken  into  Racial  and  College 
Talents.  Talents  are  common  to  all  members  of  a 
Race  or  College  of  which they  are  a  characteristic 
part and may never be learned or forgotten, though 
they  often  may  be  “ranked”.  Talents  require  no 
preparation,  take  a  maximum of  5  seconds to  util-
ise,  and  require  no  expenditure  of  energy.  All 
Talents can be classified as either active or passive. 
Passive Talents are always in effect. Active Talents 
require  a  Pass  Action  to  utilise,  and  often  require 
rolls  to  see  if  they  succeed.  Racial  Talents  are 
described in Character Generation. College Talents 
are discussed in the individual Colleges. 

Spell  Magic  constitutes  the  great  majority  of  the 
magic  utilised  by  Mages.  Unless  otherwise  stated, 
all  magic mentioned in these  rules  is  Spell  Magic. 
All Spell Magic has the following characteristics in 
common: 

Each  individual  Spell  has  a  defined  range,  dura-
tion, base chance and effect. Spells must usually be 
prepared by the Mage through a process of incanta-
tion to  draw  mana to  activate  the  Spell.  Spells  are 
unstable in their workings, and if cast ineptly, may 
fail  entirely  or  have  unexpected  effects  on  the 
vicinity. The casting of a Spell drains energy from 
the caster in the form of tiredness Fatigue. 

Ritual  Magic  requires  the  expenditure  of  large 
blocks  of  time  (usually  hours)  and  usually  certain 
conditions  must  be  fulfilled  while  performing  the 
Ritual.  Ritual  Magic  occasionally  requires  a  large 
number of special tools and substances and may be 
restricted  to  particular  times  or  places.  Magical 
effects  from Ritual  Magic tend  to be  more  power-
ful,  prolonged  or  delayed  than  those  of  Spells. 
Most  rituals  require  a  Cast  Check  to  determine 
whether  the  ritual  was  successful.  If  not  otherwise 
stated in the specific ritual description, a ritual may 
backfire (roll greater than Base Change + 30) with 
similar  consequences  to  a  spell.  Rituals  may  also 
cause a multiple effect similar to spells. 

Material 

Some  spells  and  rituals  require  material  compo-
nents.  These  materials  must  be  present  to  perform 
the magic.  If the spell or ritual also has a Material 
Cost  then  unless  stated  otherwise  in  the  descrip-
tion, these materials are consumed during the cast-
ing of the magic regardless of the success or failure 
of the casting. 

7 MAGIC 

Extended Rituals 

Some  rituals  require  a  far  greater  time  to  perform 
than  the  standard  one  hour,  possibly  requiring 
weeks  or  even  months.  During  these  rituals  the 
Adept  is  not  involved  in  constant  concentration. 
The Adept may eat, sleep (8 hours a day) and per-
form  other  activities  requiring  less  than  2  hours  a 
day  while  engaged  in  a  lengthy  ritual.  During  the 
extended  ritual  the  Adept  can  utilise  only  stored 
magic, and that inherent in the ritual. 

7.3 How to Cast Spells 
Casting a Spell is a complex process. 

Preparing  Firstly,  the  spell  must  be  prepared  and 
mana  gathered  for  the  spell.  This  does  not  require 
any fatigue, and normally carries no risk. However, 
it  does  involve  gesticulations  and  conversation-
level  speech,  which  will  be  obvious  to  observers. 
The spell may be prepared in 5 seconds, 1 minute, 
or multiples of an hour (using Ritual Spell prepara-
tion). The length of time taken to prepare the spell 
is proportional to the resulting safety of the Mage. 
The length of time spent preparing a spell must be 
decided upon in advance.  Preparing a spell is sub-
ject to the restrictions mentioned in §7.6. 

Casting  Once  prepared,  the  spell  is  Cast  by  an 
expenditure  of  Energy  in  the  form  of  tiredness 
Fatigue,  used  to  shape  and  direct  the  magic.  This 
takes  5  seconds.  Once  cast,  a  spell  will  either  im-
pact  upon its  target  or  fail.  If  the  spell  impacts  on 
its target, it may  be  partially  or  wholly  resisted.  If 
the  spell  fails,  it  may  backfire  (see  §7.7).  If  the 
spell  is  cast  particularly  competently,  it  may  be 
especially effective. 

Casting Mechanics 

Preparation  The  Mage’s  player  announces  the 
spell and length of preparation (either 5 seconds, 1 
minute, or a number of hours). They may break off 
their  preparation  at  any  time  and  abort  the  casting 
process.  A  spell  must  be  used  immediately  upon 
being  prepared  or  it  is  dissipated  and  the  prepara-
tion  must  be  restarted  before  it  can  be  Cast.  Only 
one  spell  can  be  prepared  at  any  one  time.  At  the 
end  of  the  preparation,  the  Mage  is  aware  of  the 
state  of  the  surrounding  mana.  During  combat, 
Preparing is a Pass Action. 

Casting The Mage’s player announces the spell, its 
target,  and any  additional  options  desired  (such  as 
lowering  the  Rank  of  some  attributes).  During 
combat, Casting is a Fire Action. 

Cast  Check  The  player  then  modifies  the  Base 
Chance  of  the  spell  due  to  current  conditions  to 
produce  a  Cast  Chance  as  a  percentage.  This  Cast 
Chance  is  then  compared  with  a  D100  roll.  If  the 
die roll is less than or equal to the Cast Chance, the 
spell  works.  If  the  die  roll  is  less  than  5%  of  the 
Cast  Chance,  the  spell  succeeds  with  a  “triple 
effect”.  If  the  die  roll  is  between  6%  and  15%  of 
the Cast Chance, the spell succeeds with a “double 
effect”.  If  the  die  roll  is  more  than 30  higher  than 
the Cast Chance (with 5 second preparation), or 40 
higher  than the  Cast  Chance  (with  longer  prepara-
tion),  the  spell  has  not  only  failed,  but  Backfired, 
as per §7.7. 

Fatigue  A  Mage  may  not  cast  a  spell  unless  they 
have sufficient Fatigue to pay the expected Fatigue 
cost. At the end of the preparation, the state of the 
Mana  (none,  Low,  normal  or  High)  is  known,  and 
the  Mage  may  abandon  the  attempt  at  that  point, 
before  losing  the  Fatigue.  Whether  the  spell  suc-
ceeds,  fails  or  backfires,  the  Mage  must  pay  the 
Fatigue cost. It usually costs 1 Fatigue point to cast 
a  General  Knowledge  Spell  or  2  Fatigue  points  to 
cast  a  Special  Knowledge  Spell.  In  a  High  Mana 
area,  these  Fatigue  costs  are  reduced  by  one.  In  a 
Low Mana area, the Fatigue Cost is doubled. 

Success  If  the  Cast  Check  is  a  double  or  triple 
effect, the element to be doubled or tripled must be 
decided  before  anything  else  is  resolved.  If  the 
Cast Check is a success or better, the target(s) may 
resist the spell, if it is passively resistible (see §7.8 
for  details).  This  will  reduce  or  nullify  the  effects 

of  the  spell,  as  defined  in  the  individual  spell  de-
scription. See §7.5 to resolve the spell effects. Note 
that some backfires will result in partial success. 

Failure  Nothing  occurs  (except  Fatigue  loss).  If 
the  result  is  a backfire,  consult  the  Backfire  Table 
§53. 

7.4 Cast Check Modifiers 
In addition to the individual College  modifiers, all 
Mages  receive  the  following  modifiers  whenever 
engaging in Spell casting: 

Each point the Mage’s MA is greater than 15 

Each point the Mage’s MA is less than 15 
Each Rank the Mage has with the spell they 
are casting 
Each hour the Mage engages in Ritual Spell 
preparation 

 
+1 
-1 
+3 

+3 

7.5 Spell Effects 
Spells  which  are  successfully  cast  on  valid  targets 
immediately  take  effect  on  those  targets  (unless 
explicitly stated within the spell description). If the 
spell  is  Passively  Resistible,  as  stated  in  the  indi-
vidual  spell,  all  targets  with  a  Willpower  value 
may  resist  at  their  current  Magic  Resistance.  A 
successful  resistance  may  reduce  the  spell  effect 
for the target, or even nullify the effects altogether. 
In  some  cases,  the  duration,  damage,  or  other  as-
pect,  may  be  random,  and  will  need  to  be  deter-
mined.  If  the  Cast  Check  is  a  double  or  triple  ef-
fect,  the  element  to  be  doubled  or  tripled  must  be 
decided before anything else is resolved. 

All  magic  works  in  quanta.  Any  attribute  of  any 
magic  may  be  performed  at  any  Rank  up  to  the 
Mage’s  maximum  level  of  skill,  but  only  at  a 
whole  number  of  Ranks.  It  is  assumed  that  all 
attributes  are  being  cast  at  maximum,  unless  it  is 
otherwise stated before the Cast Check is made. 

Damage 

Damage due to magic ignores armour and does not 
cause  stunning,  unless  a  strike  check  against  the 
target’s  defence  is  required  as  part  of  the  spell. 
Damage is done to Fatigue, and to Endurance once 
Fatigue is exhausted. A single Damage Check will 
not  “wrap”  from  Fatigue  to  Endurance,  unless  the 
total  damage  exceeds  the  target’s  combined  full 
Fatigue and Endurance. 

Doubles and Triples 

There are three characteristics of a spell which can 
be  increased  by  the  Mage  as  a  result  of  a  spell 
causing  a  double  or  triple  effect:  Range,  Duration 
and Damage. 

Whenever  a  spell  is  cast  for  double  effect,  the 
Mage has the option to double one of Range, Dura-
tion or Damage. Some spells may not have one or 
more of these attributes. Such attributes may not be 
affected. 

Whenever a spell is cast for triple effect, the Mage 
has  the  option  of  either  tripling  one  of  Range, 
Duration  or  Damage;  doubling  any  two  of  these 
attributes;  or  decreasing  the  target’s  Magic  Resis-
tance by 20. 

A  Mage  may  attempt  to  cast  a  spell  at  a  target 
which is not within range in the hope of achieving 
a double or triple effect. 

7.6 Restrictions on Magic 
Mages are restricted as to where and when they can 
employ magic. General restrictions that apply to all 
Mages are covered in this section. Specific restric-
tions applying to Adepts of particular Colleges are 
covered in the opening sections of those Colleges. 

Cold Iron 

A  Mage  may  never  prepare  or  cast  a  spell  or  en-
gage in Ritual Magic while in physical contact with 
Cold  Iron.  They  may  exercise  any  Racial  Talent 
Magic,  but  no  learnt  Talent  Magic.  Wearing  ar-
mour  made  of  Cold  Iron,  or  holding  weapons  or 
tools  made  of  Cold  Iron  is  regarded  as  being  in 
physical  contact.  GM  discretion  covers  all  other 

22 

cases.  Several  ounces  of  Cold  Iron  is  required  to 
cut off the mana flow. 

There are several possible means of circumventing 
the effects of cold iron:  

•  The  Mage  may  drop  all  iron  items  prior  to  per-
forming any magic. Note that donning and doffing 
armour is very time consuming.  

•  The  Mage  may  employ  weapons,  tools  and  ar-
mour that are not metallic (e.g. quarterstaff, leather 
armour).  Weapons  that  are  normally  metallic  can 
be made out of wood, bone or stone, but their Base 
Chance is reduced by 10, their Damage reduced by 
2 and their Fumble chance increases (see §6.10). A 
similar  loss  of  effectiveness  will  be  experienced 
with other tools that are normally iron.  

•  The  Mage  may  use  metallic  items  that  have  no 
iron  content,  such  as  copper,  tin  and  bronze.  Such 
items  cost  the  same  as  equivalent  iron  items,  but 
they  are  less  effective:  weapons  do  one  less  point 
of  damage,  and  fumble  on  a 98  to 00  roll  (instead 
of  just  a 00);  and  armour provides  two  less points 
of protection.  

•  The  effects  of  cold  iron  can  be  neutralised  by 
combining  it  with  precious  metal  (silver  or  true-
silver).  Such  items  are  as  effective  as  Iron  items. 
Silvered  items  costs  at  least  10  times  the  standard 
price  and  truesilvered  items  180  times  the  cost.  A 
Mage’s Cast Chances are reduced by 10 if carrying 
silvered items. Wearing Cold Iron does not protect 
from the effects of magic. 

Confinement 

A Mage must have the freedom to make the neces-
sary gestures and sounds in order to cast a spell or 
perform  a  ritual.  Mute,  bound,  paralysed,  uncon-
scious, stunned, prone or restrained Mages may not 
use Spell or Ritual Magic, though Talent Magic is 
usually  possible.  In  addition,  an  Adept  must  have 
at least one hand free and be able to  speak clearly 
to prepare and cast a spell. 

Proper Procedure 

A  Mage  may  never  employ  a  type  of  Magic, 
whether a Spell, Ritual or Talent, which they have 
not  learnt.  The  Mage  also  must  have  whatever 
equipment or working materials are specified in the 
Spell or Ritual description. 

Concentration 

A Mage may not cast a spell or perform a ritual if 
their  concentration  is  broken.  This  usually  occurs 
by being engaged in Melee or Close Combat. Other 
types of attack or distraction may also suffice. If an 
event  is  deemed  distracting,  the  Mage’s  player 
must  roll  a  4  times  Willpower  check  to  maintain 
concentration, or the spell or ritual is disrupted. 

The concentration required to control spells already 
cast,  or  the  concentration  required  to  control  an 
entity,  will  not  be  broken  by  entering  combat  or 
being  attacked.  It  will  only  be  broken  if  they  are 
stunned, knocked out or killed. 

Queuing 

Spells  that  have  the  same  effects  are  not  cumula-
tive.  If  a  spell  is  cast  on  a  target  that  is  already 
under  the  effect  of  a  spell  which  has  the  same 
effect,  then  the  spell  “queues”.  This  means  that, 
although  the  spell  is  in  effect  on  the  target,  it  has 
no effect until the earlier spell is gone. 

Spells with any overlapping effects are affected by 
this  rule.  Note  that  it  is  the  spells’  effects  that  are 
important,  not  the  spell  itself.  If  the  spells  both 
affect  one  attribute  (e.g.  defence),  they  queue.  If 
two spells affect different attributes (e.g. PS \& FT), 
they  stack.  This  rule  also  applies  to  items.  Unless 
stated  otherwise,  any  item  that  contains  a  magical 
effect  that  can  be  caused  by  a  spell,  cannot  be 
affected by that spell (e.g. a weapon with a magical 
bonus to hit or damage may not stack with Weapon 
of Flames). 

The duration of the second, queuing, spell is meas-
ured from casting, but it only takes effect when the 
first spell wears off. 

7.7 Backfires 
Particularly  inept  Spell  casts  or  Ritual  perform-
ances  may  cause  backfires.  If  the  Mage’s  Cast 
Check  fails  by  more  than  30  for  a  5  second  Spell 
preparation  or  Ritual  performance,  or  40  for  a 
longer Spell preparation, the Magic backfires. The 
Magic will always backfire on a natural roll of 100, 
unless  the  Cast  Chance  is  over  100%,  in  which 
case the Magic fails. The GM rolls on the Backfire 
Table  (§53)  and  applies  the  result.  Effects  include 
extra  Fatigue  loss,  partial  or  awry  magical  effects, 
and curses and afflictions on the caster. 

Backfire Interpretation 

The  effects  are  to  be  interpreted  as  widely  as  de-
sired  by  the  GM.  All  curses  and  afflictions  are 
resistible,  and partial  or  awry  effects  may  be  also, 
depending  on  the  magic.  All  backfire  effects  are 
cumulative.  It  may  be  impossible  to  apply  a  spe-
cific  backfire  effect  in  certain  situations.  This  is 
generally described by “No apparent effect”, along 
with most subtle afflictions. The effects of a back-
fire should be kept secret for as long as possible. 

In  most  cases,  specific  reductions  in  numerical 
ratings are given when a Mage is cursed as a result 
of  a  backfire.  However,  ancillary  effects  of  the 
curse  must be  determined  within  the  guidelines  of 
the curse description. 

7.8 Magic Resistance 
An entity who is the target of a spell may resist the 
spell if it is resistible. There are two types of resis-
tance:  Active  Resistance  reduces  the  Cast  Chance 
of a spell; Passive Resistance avoids or reduces the 
effects of a spell. Magic Resistance is used for both 
Active  and  Passive  Resistance.  It  is  equal  to  an 
entity’s Willpower, modified as follows: 

-5 

+5 

Target and caster are of the same 
Branch of Magic 
Target and caster are of opposed 
Branches of Magic 
Target is sentient but not a Mage 
Target is in a consecrated area 
Target has on, or is in the area of, the 
appropriate counterspell 
Spell had triple effect and the caster 
chose to affect Magic Resistance 
Branches  of  Magic  are  covered  in  §7.10.  Sentient 
entities  are  those  with  an  MA  of  0  or  greater. 
Counterspells  are  covered  in  §10.2.  Triple  Effects 
are covered in §7.5. Consecration is not covered in 
these rules. 

+20 
+50 
+30 + 3 
per Rank 
-20 

Purification,  certain  other  spells,  and  many  items 
also affect Magic Resistance. 

If  an  entity  successfully  resists  a  targeted  spell, 
then  they  will  be  aware  that  they  have  resisted 
some Magic. While the spell did not directly affect 
the entity, their aura is marked sufficiently that the 
last spell to impact is determinable. Note that only 
spells  which  target  possessions  or  entities  are  no-
ticeable  in  this  way.  Spells  which  affect  an  area 
cannot be detected when a character resists them. 

Objects cannot normally resist Magic. However, if 
the  object  is  an  entity’s  possession,  and  the  spell 
can be passively resisted, then the entity may apply 
their normal resistance. 

Passive Resistance 

When a spell that is passively resistible impacts on 
an  entity,  the  entity  may  attempt  to  resist  the  ef-
fects  of  the  spell.  This  is  known  as Passive  Resis-
tance. The player must roll D100. If  the die roll is 
less than or equal to the entity’s Magic Resistance, 
the spell’s effects are reduced or nullified. Passive 
Resistance  is  an  automatic  bodily  function  which 
occurs regardless of whether an entity is conscious 
or  not.  If  an  entity  re-encounters  an  area  effect 
magic,  they  must  re-resist,  whether  they  success-
fully  resisted  last  time  or  not.  At  the  start  of  a 
pulse,  an  entity  may  choose  to  not  resist.  For  the 
remainder  of  the  pulse,  the  entity  may  not  pas-
sively resist any spells, unless they become stunned 

or  unconscious.  While  choosing  to  not  resist,  an 
entity may only perform a pass action. 

Active Resistance 

An  entity  can  choose  to  actively  resist  another 
entity. When an entity attempts to cast or trigger a 
spell  which  is  actively  resistible,  then  the  highest 
Magic  Resistance  of  any  target  who  is  actively 
resisting  the  entity  is  subtracted  from  the  Cast 
Chance. Note that Active Resistance is only effec-
tive  if  the  entity  who  is  actively  resisting  is  a  tar-
get, or in the area of effect of the spell. In combat, 
Active Resistance is a Pass Action. An entity must 
concentrate on the caster in order to actively resist 
them.  Anything  that  can  disrupt  Spell  preparation 
can  also  disrupt  Active  Resistance.  Active  Resis-
tance  against a  spell  that  is  not  able  to  be  actively 
resisted, or against a spell  which is not targeted at 
the entity, has no effect. 

7.9 Incorporating Magic into Combat 
Spell  Magic  may  be  employed,  usually  with  5 
second preparation. A Mage must perform a magi-
cal pass action to prepare and a cast action to cast a 
spell. See §6.4 and §6.6 for what can be done dur-
ing pass and cast actions in combat. 

Talent Magic may be used during combat. Passive 
talents  operate  normally  and  do  not  require  any 
actions to employ. Active talents (e.g. Detect Aura) 
require  a  pulse  to  implement.  An  entity  may  ac-
tively resist a spell during combat by implementing 
a  pass  action.  They  may  lower  passive  resistance 
during any pass action. Triggering an invested item 
takes  one  full  pulse  irrespective  of  the  number  of 
actions that can be performed. 

Ritual Magic is difficult to employ during combat. 
During  each  pulse  that  noisy  or  dangerous  events 
take place, the Mage may need to make a concen-
tration check (as per §7.6). 

7.10 The Colleges of Magic 
Most Magic is divided into 16 Colleges represent-
ing  specific  types  of  magic.  A  Mage  may  only 
employ  the  powers  and  spells  of  one  College.  If  a 
Mage  belongs  to  a  College,  they  are  known  as  an 
Adept. 

Branches of Magic 

The  Colleges  are  divided  into  three  Branches  of 
Magic, as follows: 

The Thaumaturgies: 

The College of Bardic Magics 
The College of Binding and Animating 
The College of Ensorcelments and Enchantments 
The College of Illusions 
The College of Sorceries of the Mind The College 
of Naming Incantations  

The Elementals: 

The College of Air Magics 
The College of Celestial Magics 
The College of Earth Magics 
The College of Fire Magics 
The College of Ice Magics 
The College of Water Magics 

The Entities: 

The College of Greater Summonings 
The College of Necromantic Conjurations 
The College of Rune Magics 
The College of Witchcraft 
An  Adept’s  Magic  Resistance  is  affected  by  their 
Branch of Magic. Their resistance against spells of 
the  same  branch  as  their  own  is  increased  by  5, 
while their resistance against magic of the opposed 
branch is reduced by 5. Thaumaturgies and Entities 
are  opposed  Branches.  The  Elemental  Branch  is 
not opposed to any group of Magics. 

Thaum 
Same 

 
Thaum 
Elemental  Neutral 
Entity 

Elemental  Entity 
Neutral 
Same 

Opposed 
Neutral  
Same 

Opposed  Neutral 

23 

7 MAGIC 

Restrictions and Modifications 

Each  College  of  Magic  has  its  own  individual 
minimum  Magical  Aptitude  requirement.  This 
must  be  met  at  the  time  any  entity  becomes  an 
Adept  of  the  college.  Many  Colleges  have  restric-
tions  on  casting  magic  further  to  those  in  §7.6. 
These are specified in the first sub-section of each 
College. 

Most Colleges  are  subject  to  certain modifications 
to  Cast  Chances  in  addition  to  §7.4.  These  are 
specified  in  the  second  sub-section  of  each  Col-
lege. 

Learning College Magic 

An  Adept  is  assumed  to  have  mastered  all  of  the 
General  Knowledge  magic  of  their  College  upon 
the completion of their training. This mastery is at 
Rank 0. Special Knowledge magic is not taught to 
novices,  and  can  only  be  acquired  by  expending 
time  (and  usually  money)  to  learn  it  to  Rank  0. 
Most  Special  Knowledge  magic  is  available  at the 
Guild,  at  fixed  prices.  All  General  and  Special 
Knowledge  magic  may  be  ranked  to  Rank  20  by 
the expenditure of time and experience. 

An  Adept  must  have  ranked  a  spell  or  ritual  to 
Rank  6  before  they  can  teach  it.  They  must  have 
ranked  all  talents,  and  general  knowledge  spells 
and rituals to Rank 6 before they can teach a nov-
ice their College. 

In general, an Adept may never use spells, talents, 
or rituals of more than one College of Magic at one 
time (except Counterspells). An Adept may change 
College  but  loses  all  General  and  Special  Knowl-
edge  magics  from  their  old  College,  and  must 
spend 6  months  (and  6,500  EP)  learning  the  ways 
of the new College (including characters learning a 
college  for  the  first  time).  Once  an  Adept  has  re-
nounced a College, they may never return to it. 

Knowledge Limitations 

An  Adept  may  only  employ  Talents,  Spells  and 
Rituals that they know. They may know any num-
ber  of  talents,  but  may  not  know  more  spells  and 
rituals  below  Rank  6  than  their  Magical  Aptitude. 
They may know an unlimited number of Spells and 
Rituals of Rank 6 or higher. All  General and Spe-
cial  Knowledge  Spells  and  Rituals  of  the  Adept’s 
College,  plus  whatever  non-Colleged  magics 
known,  apply  to  this  limit  (except  Ritual  Spell 
Preparation,  and  as  specified  in  the  Namer  Col-
lege).  This  includes  Counterspells  of  other  Col-
leges. 

An  entity  may  not  become  an  Adept  of  a  College 
of Magic unless they have the Magical Aptitude to 
account  for  mastery  of  the  General  Knowledge 
spells and Rituals of that College.  This is enumer-
ated  in  the  restrictions  of  the  College.  An  Adept 
may not learn another spell or ritual if they already 
know as many spells and rituals (below Rank 6) as 
they have points in MA. 

If, as a result of a decrease of Magical Aptitude or 
spell ranks, the Adept knows more spells and ritu-
als below Rank 6 than their Magical Aptitude, then 
they will permanently lose sufficient knowledge to 
satisfy  this  rule,  losing  the  lowest  ranked  magics 
first. They will still remain Adepts of their College, 
while sentient. 

7.11 Magic Descriptions 
The  description  of  all  the  College  Magics  work 
under  certain  conventions.  The  more  important  of 
them follow: 

Sub-Sections 

Each College description is broken up into  

• An Introduction  

• Restrictions  

• Cast Chance Modifications  

• Talents (coded T-\#)  

• General Knowledge Spells (coded G-\#)  

• General Knowledge Rituals (coded Q-\#)  

7 MAGIC 

• Special Knowledge Spells (coded S-\#)  

• Special Knowledge Rituals (coded R-\#) 

7.12 Spell Descriptions

The description of each spell lists its specific effects, range,
duration, and other attributes.  Each spell is fully described under
the College to which it belongs. The following information is
included:

Rank Modifications Often range, duration and other effects will be
given as “x + y / Rank”. This means that the characteristic is equal
to x, with an additional y for each Rank attained in the magic.
Unless otherwise noted, the unit of measurement is the same for x \&
y. If an increase of y is noted for each n Ranks, then partial
multiples of n do not count unless specifically stated.

Range The maximum radius (in feet) within which 
the  Mage  can  make  the  spell  take  effect.  This  is 
always  the  distance  from  the  Adept.  It  can  be  a 
linear measurement between Adept and target, or a 
radius  of  effect.  Unless  explicitly  stated,  magical 
effects  will  not  occur  beyond  the  range  of  the 
magic.  In  combat,  measurements  are  taken  from 
the middle of a hex, and rounded upwards. 

Duration  The  maximum  length  of  time  that  the 
spell remains in effect. Spells with a concentration 
component  will  stop  as  soon  as  concentration  is 
broken.  Spells  that  do  not  require  concentration 
will persist regardless of the suffering of the Mage, 
even unto death. If a spell is cast in the middle of a 
pulse, that pulse counts towards its duration. 

Experience  Multiple  The  multiple  used  in  con-
junction with the rank to be achieved to determine 
the  experience  cost  of  increasing  a  Mage’s  Rank 
with a particular spell. 

Base  Chance  The  base  percentage  chance  of  suc-
ceeding  in  casting  a  spell.  This  may  be  equal  to 
some multiple of a characteristic of the Adept. The 
characteristic  is  taken  at  its  current  value,  multi-
plied  appropriately,  and  then  other  modifiers  are 
added. 

Resistance  The  Magic  Resistances  (Passive  and 
Active)  which may be applied against the spell by 
its targets. 

Storage  The  valid  ways  that  the  spell  may  be 
stored, for example, investment, potion and ward. 

Target Spells and Rituals are targeted at either an 
Area,  Object  or  Entity.  Some  spells  can be  cast  at 
more than one target type. If a multi-target or area 
effect  spell  is  actively  resisted,  the  highest  Active 
Resistance  of those  to  be affected  is applied.  Area 
effect  spells  are  resisted  each  time  that  an  entity 
encounters  them.  Animated  objects  count  as  ob-
jects and entities. 

Effects  The  general  purpose  and  consequences  of 
the  spell.  Includes  potential  damage,  effects  of 
resistance,  special  effects,  and  any  exceptions  to 
the normal workings of magic. 

Difficulty  Factor  A  difficulty  factor  will  some-
times  be  given  to  avoid  a  spell.  This  is  always  a 
number  by  which  the  stated  characteristic  of  the 
target is multiplied, before modifiers are added. 

Interpretations 

Most of the magic in DQ is designed to be flexible 
in  application,  and  up  to  the  interpretation  of  the 
GM  within  the  guidelines  laid  down  by  the  Gods. 
The  effects  and  procedures  are  meant  to  apply  to 
humanoid  entities  of  human  size.  An  Incinerate 
Spell that would fry a human would do little more 
than discomfort a Dragon. To close every loophole 
and explain every application would be impossible. 
Therefore,  these  matters  of  interpretation  have 
been left to your GM, in the context of their game 
and the atmosphere that they are trying to promote. 

7.13 Storage and Entrapment of Magic 
There  are  various  methods  of  storing  magical  ef-
fects.  Each  has  different  properties  and  can  store 
different types of spells. 

Potion  Spells,  Talents  and  some  quasi-magical 
skills  (e.g.  Healer)  can  be  potioned  by  an  Alche-
mist. For Talents, the imbiber receives the usage of 
the  talent  for  a  duration  dependent  on  the  Rank 
(see  §30).  For  skills,  see  the  skill  involved.  For 
spells,  imbibing  a  potioned  spell  is  equivalent  to 
being  the  Adept  and  casting  the  spell  on  oneself. 
This  is  normally  the  only  way  for  self-only  spells 
to  be  stored.  Target  area  and  target  object  spells 
cannot  be  potioned,  but  spells  that  affect  entities 

may (possibly) be potionable, for example, Necro-
sis  could  be  potioned,  but  drinking  the  potion 
would only cause the imbiber to have their internal 
organs  ruptured.  Potions  always  work,  but  they 
cannot double or triple effect. 

Investment  Spells  effects  can  often  be  stored  in 
items  and  at  a  later  stage,  be  triggered.  When  a 
spell is triggered, it is as if the Adept was there and 
had just cast the spell (except the spell characteris-
tics such as base chance, range, etc, are fixed at the 
time of investment). The entity triggering the spell 
gets to choose the target(s) of the spell at the time 
of triggering and maintain concentration spells. 

Ward A ward is a way of storing a spell within an 
object,  area  or  volume  so  that  when  some  simple 
condition  is  met,  the  ward  is  triggered,  and  the 
entities  or  objects  that  fulfilled  the  condition  be-
comes  the  target  for  the  spell.  When  a  ward  is 
triggered, it is always successful, but cannot cause 
a multiple effect. For a spell to be wardable, it must 
have a range or area component (the range may not 
be touch, nor self). Wards cannot maintain concen-
tration spells. 

Magical  Trap  Magical  effects  can  be  stored  in 
mechanician  traps.  Unlike  wards,  traps  have  to  be 
physically  triggered.  The  spell  effect,  unless  an 
area  effect,  will  only  be  targeted  on  the  triggerer. 
Magical traps can only store spells that have a non-
self  range.  The  spell  in  a  trap  can  only  target  an 
area  or  the  triggerer.  Spells  in  traps  will  always 
work, but cannot multiple effect. 

Shaped  Magic  Magical  items  beyond  those  con-
taining simple invested spells are known as shaped 
items. Shaped items come in two flavours, charged 
and  permanent.  Charged  items  have  a  number  of 
charges  which  diminishes  with  use.  Certain  items 
are  said  to  have  “bound  charges”,  which  means 
they behave as an invested item except they can be 
recharged.  Charged  items  which  cannot  be  re-
charged  lose  their  magical  status  once  all  charges 
have  been  expended.  Permanent  shaped  items  can 
come  in  any  shape  or  form  and  can  defy  many  of 
the usual “rules” of the magical universe. 

\end{Chapter}
