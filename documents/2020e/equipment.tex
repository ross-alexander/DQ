\begin{Chapter}{Equipment and Money}

\section{Purchase of goods and items}

The GM will be guided in determining the price (in Silver Pennies) of
the various goods produced by craftsmen by the Price List (see Players
Handbook and Tables \S\ref{table:weapons}, \S\ref{table:accessories}
and \S\ref{table:shields}).  The three factors which determine the
price of finished goods are the quality of the material used, the
hours spent in construction, and the estimated Rank of the artisan (if
one person produces the goods) or of the overseer (if the effort is a
team project).  However, if a character wishes to purchase a
custom-made or rare item, then they will have to negotiate with the
artisan (represented by the GM), and may defray costs by providing
some of the scarcer components themselves.  The barter system is
acceptable when dealing in costly or rare items.

The value of a coin is determined by its weight and the metal of which
it is made.

\begin{dqtblr}{colspec={Xll},hline{1-2}={0.8pt}}
Name			& Weight	& Value \\
Copper farthing (cf)	& 1/5 oz	& \\
Silver penny (sp)	& 1/20 oz	& 4 cf \\
Gold shilling (gs)	& 1/20 oz	& 12 sp \\ 
Truesilver guinea (tg)	& 1/10 oz	& 21 gs \\
\end{dqtblr}

Other common coins include the halfpenny, threepence, and sixpence.
The values and weights of these coins correspond to those of the
Silver Penny.

\section{Encumbrance Modifies Agility}

The weight borne by a character may temporarily reduce the character’s
Agility.

To calculate modified Agility use the Fatigue and Encumbrance Table
(\S\ref{table:fatigue}) and:

\begin{Enumerate}

\item Cross reference the character’s Physical Strength and the weight
  they are carrying. Clothing (other than armour) the character is
  wearing does not count towards this weight.

\item Read down this column until it intersects with the row which
  reads “Agility Loss.”

\item Deduct the resulting number from the character’s Agility to give
  Modified Agility.

\item Re-calculate this number if there is a change in the weight they
  bear.

\end{Enumerate}

The character’s Modified Agility is used as a basis for determining
their current TMR.  A character is considered to have a minimum
Agility of 1 for all other game functions.

\end{Chapter}
