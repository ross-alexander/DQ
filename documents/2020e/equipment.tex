\begin{Chapter}{Equipment and Money}

5.1 Purchase of goods and items 
The GM will be guided in determining the price (in 
Silver  Pennies)  of  the  various  goods  produced  by 
craftsmen by the Price List (see Players Handbook 
and  Tables  §56.1,  §56.2  and  §56.3).  The  three 
factors which determine the price of finished goods 
are the quality of the material used, the hours spent 
in  construction,  and  the  estimated  Rank  of  the 
artisan (if one person produces the goods) or of the 
overseer (if the effort is a team project). However, 
if a character wishes to purchase a custom-made or 
rare item, then they will have to negotiate with the 
artisan  (represented  by  the  GM),  and  may  defray 
costs by providing some of the scarcer components 
themselves.  The  barter  system  is  acceptable  when 
dealing in costly or rare items. 

The value of a coin is determined by its weight and 
the metal of which it is made. 

Name 

Weight  Value 

 

1/5 oz 
Copper farthing (cf) 
1/20 oz  4 cf 
Silver penny (sp) 
Gold shilling (gs) 
1/20 oz  12 sp 
Truesilver guinea (tg)  1/10 oz  21 gs 
Other  common  coins include  the  halfpenny,  three-
pence,  and  sixpence.  The  values  and  weights  of 
these  coins  correspond  to  those  of  the  Silver 
Penny. 

5.2 Encumbrance Modifies Agility 
The  weight  borne  by  a  character  may  temporarily 
reduce the character’s Agility. 

the 

1.  Cross-reference 
character’s  Physical 
Strength and the weight they are carrying. Clothing 
(other  than  armour)  the  character  is  wearing  does 
not count towards this weight. 

2.  Read  down  this  column  until  it  intersects  with 
the row which reads “Agility Loss.” 

3.  Deduct  the  resulting  number  from  the  charac-
ter’s Agility to give Modified Agility. 

4. Re-calculate  this  number  if  there  is  a  change  in 
the weight they bear. 

The character’s Modified Agility is used as a basis 
for  determining  their  current  TMR.  A  character  is 
considered to have a minimum Agility of 1 for all 
other game functions. 

To  calculate  modified  Agility  use  the  Fatigue  and 
Encumbrance Table (§58.1) and: 

\end{Chapter}
