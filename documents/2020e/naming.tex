\begin{College}[2.0]{naming}{Naming Incatations}{NA}

The College of Naming Incantations is concerned with the essential
truths and underlying realities that make up the world and with the
knowledge of auras and true names.  Naming Incantations is one of the
two oldest Colleges of Magic and just as the Entities branch grew out
of runic magic so too do the Thaumaturgical Colleges have their roots
in Naming.  Adepts of this College are commonly called Namers.

The Naming College holds that magic is a form of deception, a
manipulation of reality, whereby Mages use mana to impose their will
on the world.  The College’s abilities include divining the true
nature of things and enforcing those truths by protecting against and
preventing magic.  Living beings express their true nature and
intrinsic essence in their auras and names, and Namers study these in
order to understand, protect, restore, and gain control over them. It
is said that in the ancient days Namers were capable of commanding the
sea, the wind, and the rocks by their names — but if this is true then
that knowledge has long been lost.  Namers still learn the names of
the plants that grow in the earth, but they have little influence over
them.

Many Namers also learn the healing arts, and perhaps this is linked to
a desire to restore beings to their true state.

Given its abilities in neutralising magic, and the low Magical
Aptitude requirement, the College attracts considerable interest from
individuals engaged in the arts of war. Many Adepts use it as a means
to protect themselves against hostile magics, while they operate in a
more physical manner.

\subsection{Traditional Colours}

No particular colour has a strong association with the College, as
astrologically magic is of all colours, and of none.

\subsection{Traditional Symbols}

Members of this College sometimes wear small symbols made of iron,
(insufficient to cause them any inconvenience), symbolising their
ability to neutralise magic. Circles or spheres are very common,
harkening perhaps to circles of protection.

\section{Restrictions}

Adepts of the College of Naming Incantations may practise their arts
without restriction.  Some abilities may require that the Namer know a
particular Generic or Individual True Name, or have learned a
particular Counterspell.

The MA requirement for this College is 1. 


\section{Base Chance Modifiers}

The following numbers are added to the Base Chance of performing any
talent, spell or ritual of the College of Naming Incantations:

\begin{dqtblr}{colspec={Xr}}
Never before encountered target’s generic type		& −15\% \\
Has not learned target’s Generic True Name		& −10\% \\
Each Rank achieved with target’s Generic True Name	& +1\% \\
Each Rank achieved with target’s Individual True Name 	& +2\% \\
\end{dqtblr}

All modifiers are cumulative. 

In addition, each Rank achieved with the target’s Individual True Name
reduces the target’s Magic Resistance by 1\%.

\section{Benefits}

\subsection{Language}

Due to their knowledge of True Names, Namers may Rank any Language in
the Protonic Language Group as if they already know another language
in that group at Rank 5. (See Languages \S\ref{languages:groups}).

\subsection{Ranking Names}
\label{namer:ranking}

\begin{Itemize}
\item Learning a Generic True Name takes a day, while learning an
  Individual True Name takes one week (see \S\ref{ranking:names}).

\item Further ranking of both Generic and Individual True Names takes
  1 week × Rank to be achieved.
\end{Itemize}

For full details on the use of Counterspells see \S\ref{college:counterspells}.

\begin{Itemize}
\item Ranking Names costs no EP.  
\item The maximum Rank for True Names is 20.  
\item Ranking of Names may be done in combination with ranking either
  a magical or non-magical ability.
\item Namers may Rank one Name in addition to other forms of ranking.
\end{Itemize}


\section{Talents}
\label{naming:talents}
\begin{talent}[T-1]{Detect Aura}

\multiple{75}
\basechance{2 × PC ( + 5 / Rank)}
\resist{May only be actively resisted}
\target{Entity, Object, Area, Volume}
\begin{effects}
The Base Chance is reduced by 1\% for every foot after the first five
from the Adept.  See the Detect Aura Talent
(\S\ref{talent:detectaura}) for the results of this talent.  In
addition to other information gained, the Namer also receives the
target’s Generic True Name, if any.
\end{effects}
\end{talent}

\begin{talent}[T-2]{Expel Magic}

\multiple{75}
\resist{May only be passively resisted}
\target{Object, Area, Volume}
\begin{effects}
This talent allows the Namer to dissipate a magical spell stored in a
Ward, Magical Trap, Potion, or Invested Item. In order to use this
talent, the Namer must specify the name of the spell to be affected
and cast the appropriate Counterspell on the target with the specific
intent of dissipating the stored magic.  The chance of the stored
magic resisting destruction is 50\% [( + 3 / Rank of the stored magic)
  (3 / Rank of this Talent)].  If successful all of the magic of the
same type stored within the target is destroyed.  The appropriate
Counterspell is the one that affects the magic stored not the storing
magic.  For example, a Ward of Enchanted Sleep would require the use
of an E \& E General Knowledge Counterspell. Possessions gain a single
Resistance Check but use the better of the chance above, or their
wielder’s MR.
\end{effects}
\end{talent}

\begin{talent}[T-3]{Quick Cast}

\begin{effects}
Namers may cast any Counterspell that they know without preparing it
first.
\end{effects}
\end{talent}

\section{General Knowledge spells}

The entire general spell knowledge of the Namer college consists of
the ability to cast Counterspells.  A Namer may cast a Rank 0
Counterspell against any College of magic with which they are
familiar.  Counterspells at Rank 0 do not count towards the Namers MA
limit for spells and rituals.

Familiarity with all of the commonly encountered Colleges will be
taught to a Namer during their apprenticeship, and beginning Namers
will have the ability to cast all of the Counterspells of the standard
Colleges at Rank 0. If a Namer encounters Colleged magic of a form
with which they are not familiar they may familiarize themselves with
the College by one of the following methods:
\begin{Itemize}
\item By using the Ritual of Divination on an Adept of that College.
\item By Divinating a magical effect produced by that College,
  provided that it is still in effect.
\item By spending a day’s study with a Namer who is already familiar
  with the College.
\end{Itemize}
Once they have done this they will be able to cast Rank 0
Counterspells against that college.

Unlike other Adepts, Namers may gain Rank with Counterspells that are
not of their College. Namers rank all Counterspells as General
Knowledge spells of the Namer College.  Once a Namer begins to rank a
Counterspell it will count towards the Namers MA limit for spells and
rituals as normal.


\section{General Knowledge Rituals}


\begin{ritual}[Q-1]{Dissipation}


\target{Entity, Object, Area, Volume}
\basechance{As per Counterspell + Ritual preparation}
\casttime{1+ hours, maximum 10}
Actions: Concentration 
Concentration Check: Standard 
\begin{effects}
By engaging in Ritual Spell Preparation a Namer may use a Counterspell
to dissipate the effects of a spell. The Namer must perform at least
one hour of Ritual Spell Preparation at the end of which they must
cast the appropriate Counterspell, and specify the name of the spell
to be dissipated.  Only spells (not rituals) may be dissipated using
this technique.  It is not possible to achieve Rank with this ability
since it is not an independent ritual, but rather a specialized use of
Ritual Spell Preparation.
\end{effects}
\end{ritual}

\section{Special Knowledge Spells}

\begin{spell}[S-1]{Bane}

\range{10 feet + 10 / Rank}
\duration{30 seconds + 5 / Rank}
\multiple{300}
\basechance{20\%}
\resist{Passive}
\storage{Investment, Ward, Magical Trap}
\target{Area}
\begin{effects}
This spell strengthens reality and stabilizes the mana in an area 15
feet in diameter (+ 10 / 5 Ranks) such that all magical Cast Chances
are reduced within the area by 5\% (+ 3 / Rank).  This will affect
spells and rituals, and talents with base chances.  The spell has no
affect on stored magics (such as invested items), shaped items, or
magic without base chances.
\end{effects}
\end{spell}

\begin{spell}[S-2]{Banishment}

\range{Self}
\duration{10 seconds + 10 / Rank}
\multiple{200}
\basechance{20\%}
\resist{No}
\storage{Potion}
\target{Self}
\begin{effects}
Through use of this spell, a Namer may banish a summoned being back to
its own plane.  While the banishment spell is in effect, the
appropriate Counterspell cast by the Namer at a summoned entity will
cause the entity to return to its own plane, unless it resists. The
Counterspell must match the type of spell or ritual used in the
summoning of the creature. In general the spells/rituals affected are
the elemental summonings (Summon Fire Elemental, etc.), all Greater
Summonings, Dark/Light Sphere conjuration and Fire college Efreeti and
Salamander summoning.  The Call Patron ability of Agents is not
classified as a summoning spell and is not affected.  The being to be
banished may actively and passively resist the Counterspell.
\end{effects}
\end{spell}

\begin{spell}[S-3]{Compel Obedience}

\range{15 feet + 5 / Rank}
\duration{Concentration:  max.  10  minutes  +  10  / Rank}
\multiple{400}
\basechance{20\%}
\resist{Active, Passive}
\target{Entity}
\storage{None}
\begin{effects}
The Adept may cast this spell over 1 (+ 1 / 4 Ranks) targets whose
Generic True names are known to them.  Those targets who fail to
resist may be commanded by the Adept to perform actions that are
both within their physical capabilities and in their true natures.
Commands are given in the Namer tongue and will be understood by all
entities. Commands must be short and simple, such as: “Stop!”, “Wait
here”, “Follow me”, “Hide under the table”. Entities can only be
compelled to perform actions that they might perform naturally.  For
example, brigands who were involved in a combat might be compelled to
“Flee!”, but if those same brigands felt they were winning the fight,
they would heed no such compulsion but could perhaps be directed to a
different target. If the spell is cast at targets with different GTNs
the Namer must use the lowest applicable base chance modifier.

If the Adept chooses to pronounce a target’s Individual True Name as
part of the spell then only one entity may be affected but the Namer
is vested with much greater control over that entity, even against its
nature.  It is possible for the target to defy the Namer, but there
are serious consequences for disobedience. Should the target decline
to obey any command of the caster that is not obviously suicidal, they
must make a Willpower check of [1 × WP − (1\% per Rank that the Namer
has with the target’s ITN)]. This check does not break the spell.
Should the target fail their check, they will feel great pain and
immediately take damage equal to half of the Namer’s Rank with this
spell. This damage cannot be resisted.
\end{effects}
\end{spell}

\begin{spell}[S-4]{Disjunction}

\range{10 feet + 10 / Rank}
\duration{1 minute + 1 / Rank}
\multiple{300}
\basechance{30\%}
\resist{Passive}
\storage{Investment, Ward, Magical Trap}
\target{Object, Area}
\begin{effects}
This spell prevents stored magics within an object or area from coming
into effect. Magics that are affected by this spell include Wards,
Invested items, Potions, Magical Traps, and permanent magics that need
to be triggered.  If a potion under the effects of a Disjunction is
consumed, the potion will take effect after the spell effect ceases,
provided it is still inside an entity.  Other items will simply be
unable to be triggered, and no charges will be lost.
\end{effects}
\end{spell}

\begin{spell}[S-5]{Dispel Magic}

\range{Self}
\duration{5 seconds + 5 / 4 Ranks}
\multiple{400}
\basechance{5\%}
\resist{No}
\storage{Potion}
\target{Self}
\begin{effects}
While the Dispel Magic is in effect, the appropriate Counterspell cast
at a target may dissipate magic. The Counterspell must match the type
of spell to be dissipated, and the Adept must specify the name of the
spell that they wish to remove.  If the Counterspell is successfully
cast, the chance of the magic being dispelled is 50\% [(+ 3 / Rank
  with Dispel Magic) (3 /Rank of the target magic)].  This spell
cannot remove the effects of rituals, or remove curses.

\end{effects}
\end{spell}

\begin{spell}[S-6]{Forbidding}

\range{10 feet + 10 / Rank}
\duration{10 minutes + 10 / Rank}
\multiple{250}
\basechance{30\%}
\resist{Passive}
\storage{Investment, Ward, Magical Trap}
\target{Area}
\begin{effects}
This spell creates a thin, invisible wall, 10 feet high and 20 feet
long. The Adept may increase either height or length by 1 foot per
Rank.  This barrier obeys all of the usual rules for insubstantial
walls.  A single Generic or Individual true name is crafted into the
forbidding. To those entities whose names are contained therein, or if
seen by means of Witchsight or similar, the wall appears bluish and
crackling with magical energy.  If a Generic True Name is in the
forbidding, then to those named who fail to resist upon initial
contact the forbidding is completely solid to them and they are unable
to pass through it.  If they resist, the barrier is insubstantial, as
it is those who are not named by it. If an Individual True name is
placed in the forbidding then in addition to the Generic effects, the
entity must resist each contact with the barrier or suffer [D - 4] + 1
/ Rank damage, even if they are able to pass through the wall because
they initially resisted.
\end{effects}
\end{spell}

\begin{spell}[S-7]{Mana Sense}

\range{Self}
\duration{5 minutes + 5 / Rank}
\multiple{200}
\basechance{20\%}
\resist{No}
\storage{Potion}
\target{Self}
\begin{effects}
This spell allows the adept to “sense” the mana flows within 10 feet
(+ 10 / Rank). If a spell is cast or magic triggered within range, the
Adept will see it flying off towards its target. Similarly, if the
target of any spell is within range, the Adept will see the magic
impact.  If the Adept chooses magical Pass actions of Concentration
with this spell in effect, they will be able to see Adepts drawing
mana, and be able to see if a target resists a spell or not.  While
concentrating the Adept will have a (2 × PC) chance of being able to
distinguish the College of the magic they can see, the name of the
spell, and whether the spell is low, medium, high or very high in
rank.
\end{effects}
\end{spell}

\begin{spell}[S-8]{Scry Shield}

\range{10 feet + 5 / Rank}
\duration{10 minutes + 10 / Rank or Special}
\multiple{300}
\basechance{20\%}
\resist{No}
\storage{Investment, Ward, Magical Trap}
\target{Volume}
\begin{effects}
This spell protects an area from scrying by Wizard Eyes, Crystals and
Waters of Vision, Bard’s Ear and similar divinatory magics of a Rank
equal to or less than the rank of the Scry Shield. It does not prevent
normal vision, infravision, Witchsight and similar spells.  A Scry
Shield is a shell over the protected volume, so once the area is
penetrated by any means, e.g. on foot or by flying, spells cast inside
the protected volume work normally. At Rank 20 this spell alarms the
Adept that an attempt to divine into the volume by magical means has
taken place, provided that the Adept is within the volume at the time.
This spell may be cast as a ritual if the Adept so chooses. In this
form casting takes 10 hours and the duration is increased to 4 weeks
(+ 1 / Rank).
\end{effects}
\end{spell}

\begin{spell}[S-9]{Spell Barrier}

\range{10 feet + 5 / Rank}
\duration{1 minute + 1 / Rank}
\multiple{300}
\basechance{30\%}
\resist{No}
\storage{Investment, Ward, Magical Trap}
\target{Volume}
\begin{effects}
The Adept creates a thin, glowing, translucent wall which blocks the
passage of magic.  The barrier is either 10 feet high and 20 feet long
as a wall, or 10 feet high and 5 feet in radius as a ring.  The Adept
may increase any dimension --- other than thickness --- by 1 foot per
Rank. This barrier obeys all of the usual rules for insubstantial
walls.  Any magic cast in such a way that a direct line drawn from the
caster to their target passes through the wall (from either side) has
a 40\% [(+ 3 / Rank with this spell) (3 / Rank of the target magic)]
chance of having its energies dissipated.  If a spell passes through
more than one Spell Barrier only a single roll for dissipation should
be made, with the highest dissipation chance being used.
\end{effects}
\end{spell}

\begin{spell}[S-10]{True Seeing}

\range{10 feet + 5 / 2 Ranks}
\duration{30 seconds + 10 / Rank}
\multiple{300}
\basechance{25\%}
\resist{No}
\storage{Investment, Ward, Magical Trap}
\target{Area}
\begin{effects}
All entities within the area of effect have a chance of detecting
things in the area that have been magically altered to appear other
than they truly are.  True Seeing may reveal the visual component of
illusions, and entities or objects that are invisible, insubstantial,
or have been magically rearranged or transformed (such as by
curses). The chance of an observer detecting such alterations or
concealments will depend upon the rank of this spell, the rank of the
concealing or transforming magic, and the perception of the
observer. For odd magical abilities without a Rank the GM should
substitute some appropriate alternative (such as [MA − 10]).

True Seeing is of lesser or equal Rank: Slight imperfections may be
revealed, (e.g. invisible figures shimmer a little, the colour of an
illusion may appear a bit off, etc.), and there is a (1 × PC) chance
of an observer scrutinizing the area noting this.

True Seeing is of higher Rank: More major imperfections may be noticed
(e.g. invisible figures have a slight will-o’-the-wisp glow, toads
that are really Princes may have tiny gold crowns).  The detection
chance rises to (2 × PC).

True Seeing is 10+ Ranks higher: The imperfections in concealing and
transforming magics become quite obvious, (e.g.  invisible figures
appear ghostly, illusions may appear painted or translucent,
etc.). Detection is automatic.
\end{effects}
\end{spell}

\section{Special Knowledge Rituals}

\begin{ritual}[R-1]{Divination}

\range{5 feet + 1 / Rank}
\duration{Immediate}
\multiple{250}
\basechance{40\% + 10 / Rank}
\casttime{1 hour or 3 hours}
\resist{No}
\target{Entity, Object, Area}
\material{None}
\actions{Concentration}
\concentration{Standard}
\begin{effects}
There is no possibility of backfire from this ritual.  By use of this
ritual a Namer may determine if an individual, object, or area is
currently, or has been recently, under the effects of magic a spell by
employing the Ritual of Magic Divination.

If the ritual is successful, the nature of all magic in effect (exact
names and Colleges) is revealed to the Namer.  If the magic is of
non-college origin general effects are revealed.  In the case of magic
that is no longer in effect, for each 5\% under the Cast Chance that
the Namer rolled, magic that expired an extra week ago is revealed.
For example if a Namer rolled 12\% under their Cast Chance magic that
expired up to two weeks ago would be revealed — in addition to all
magic currently in effect.

If the Namer wishes they may perform an Ancient Divination.  The Base
Chance of the ritual is reduced to 40\% (+ 2 / Rank), and the Cast
Time increased to 3 hours.  If successful the Namer will learn the
exact nature of all enchantments, magical mechanisms, triggering
conditions, curses, side-effects, etc., placed upon an entity or
object even if they are of non-college origin.  If an object has an
Individual True Name the Ancient Divination will reveal its existence,
though not the actual name.
\end{effects}
\end{ritual}

\begin{ritual}[R-2]{Expulsion}

\range{5 feet + 1 / Rank}
\duration{Immediate}
\multiple{300}
\basechance{10\% + 5 / Rank}
\casttime{1 hour} 
\resist{Active, Passive}
\target{Entity}
\material{None}
\actions{Concentration}
\concentration{Standard}
\begin{effects}
This Ritual will return one entity to its plane of origin, regardless
of how it got to the current plane. Upon completion of the ritual, if
the target fails to resist, they will be immediately returned to the
point from which they left their plane of origin.
\end{effects}
\end{ritual}

\begin{ritual}[R-3]{Interregnum}

\range{10 feet}
\duration{Special}
\multiple{250}
\basechance{MA + 4\% / Rank}
\casttime{2 hours}
\resist{Active, Passive}
\target{Entity or Object}
\material{None}
\actions{Concentration}
\concentration{Standard}
\begin{effects}
The targeted entity or object has all magical effects currently upon
them, that are of lesser or equal rank to the Interregnum, suspended.
Whilst suspended their durations will not reduce, but the magic will
have no effect.  The duration of the Interregnum may be chosen by the
Adept at the time of casting, from a minimum of 1 day to a maximum of:


\begin{dqtblr}{colspec={m{3em}X}}
Rank	& Duration (maximum) \\
0--10	& 1 day (+ 1 / Rank) \\
11--15	& 1 month \\
16--19	& 3 months \\
20	& 1 year \\
\end{dqtblr}
\end{effects}
\end{ritual}

\begin{ritual}[R-4]{Remove Curse}

\begin{effects}
Namers have a greater ability to remove curses than do the Adepts of
other Colleges.  Namers learn the normal Remove Curse ritual (see
\S\ref{ritual:removecurse}) as R4 of this College, but gain a bonus to
Base Chance of + 2 / Rank for Minor curses and + 1 / Rank for Major
curses (including Death Curses).
\end{effects}
\end{ritual}

\begin{ritual}[R-5]{Sealing}

\range{20 feet +20 / Rank}
\duration{1 day + 1 / Rank}
\multiple{300}
\basechance{20\% + 4 / Rank}
\casttime{1 hour}
\resist{None}
\target{Area}
\material{Chalk, paint, blood, cornmeal or other symbol making materials}
\actions{Chanting and inscribing symbols}
\concentration{Standard}
\begin{effects}
This Ritual seals an area against entities from a single, specific,
named plane. The name of the plane must be known to the Adept, and the
name of the plane that the Adept is currently occupying cannot be
used.  No entity whose plane of origin has been sealed against can
voluntarily enter the sealed area. They will stop at the boundary and
refuse to go any further.  Any entity taken into the area against
their will (or without their knowledge, e.g.  unconscious) will
attempt to leave the area as quickly as possible.  If an attempt is
made to summon an entity from the named plane into the area the
summoning will fail.  Any entities from the named plane who are inside
the area when the sealing is created are unaffected, but should they
leave the area they will be unable to re-enter it.
\end{effects}
\end{ritual}

\begin{ritual}[R-6]{True Form}
\range{5 feet}
\duration{Immediate}
\multiple{300}
\basechance{20\% + 3 / Rank}
\casttime{3 hours}
\resist{Active}
\target{Entity, Object}
\material{None}
\actions{Concentration}
\concentration{Standard}
\begin{effects}
By means of this ritual the Adept may force a target that has been
magically altered, cursed, or rearranged into a form other than their
natural one to assume their true form and nature. It will not remove
effects that could occur naturally.  For example, the ritual would
restore the form of a human that had been cursed into the shape of a
toad, and would return to flesh a human turned to stone but would do
nothing to remove a curse of weeping sores or restore a lost limb.
\end{effects}
\end{ritual}

\begin{ritual}[R-7]{True Speaking}

\range{10 feet}
\duration{30 minutes}
\multiple{300}
\basechance{40\% + 3 / Rank}
\casttime{Special / 1 hour}
\resist{Active, Passive}
\target{Entity}
\material{None}
\actions{Asking questions}
\concentration{Standard} 
\begin{effects}
By means of this ritual the Adept may attempt to force an entity who
is present to speak the truth.  The Adept must prepare for 30 minutes
after which they may question the entity for the remaining 30 minutes
of the 1 hour ritual.  The effects of the ritual do not last beyond
the hour.  The target need not answer or speak at all, but if they
fail to resist and they choose to answer the Adept’s questions, they
must, to the best of their knowledge, speak no falsehoods.  They need
not volunteer information.  The GM rolls for the success of this
ritual and need not inform the Adept’s player of the result.
\end{effects}
\end{ritual}
\end{College}
