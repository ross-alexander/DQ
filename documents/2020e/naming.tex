\begin{Chapter}{The College of Naming Incantations (Ver 2.0)}

The College of Naming Incantations is concerned with the essential
truths and underlying realities that make up the world and with the
knowledge of auras and true names.  Naming Incantations is one of the
two oldest Colleges of Magic and just as the Entities branch grew out
of runic magic so too do the Thaumaturgical Colleges have their roots
in Naming.  Adepts of this College are commonly called Namers.

The Naming College holds that magic is a form of deception, a
manipulation of reality, whereby Mages use mana to impose their will
on the world.  The College’s abilities include divining the true
nature of things and enforcing those truths by protecting against and
preventing magic.  Living beings express their true nature and
intrinsic essence in their auras and names, and Namers study these in
order to understand, protect, restore, and gain control over them. It
is said that in the ancient days Namers were capable of commanding the
sea, the wind, and the rocks by their names — but if this is true then
that knowledge has long been lost.  Namers still learn the names of
the plants that grow in the earth, but they have little influence over
them.

Many Namers also learn the healing arts, and perhaps this is linked to
a desire to restore beings to their true state.

Given its abilities in neutralising magic, and the low Magical
Aptitude requirement, the College attracts considerable interest from
individuals engaged in the arts of war. Many Adepts use it as a means
to protect themselves against hostile magics, while they operate in a
more physical manner.

\subsection{Traditional Colours}

No particular colour has a strong association with the College, as
astrologically magic is of all colours, and of none.

\subsection{Traditional Symbols}

Members of this College sometimes wear small symbols made of iron,
(insufficient to cause them any inconvenience), symbolising their
ability to neutralise magic. Circles or spheres are very common,
harkening perhaps to circles of protection.

\section{Restrictions}

Adepts of the College of Naming Incantations may practise their arts
without restriction.  Some abilities may require that the Namer know a
particular Generic or Individual True Name, or have learned a
particular Counterspell.

The MA requirement for this College is 1. 


\section{Base Chance Modifiers}

The following numbers are added to the Base Chance of performing any
talent, spell or ritual of the College of Naming Incantations:

Never before encountered target’s generic 
type 
Has not learned target’s Generic True 
Name 
Each Rank achieved with target’s Generic 
True Name 
Each Rank achieved with target’s Individ-
ual True Name 
All modifiers are cumulative. 

-
15\%  
-
10\% 
+1\% 

+2\% 

In addition, each Rank achieved with the target’s Individual True Name
reduces the target’s Magic Resistance by 1\%.

\section{Benefits}

\subsection{Language}

Due  to  their  knowledge  of  True  Names,  Namers 
may Rank any Language in the Protonic Language 
Group as if they already know another language in 
that group at Rank 5. (See Languages §39.7). 

\subsection{Ranking Names}

• Learning a Generic True Name takes a day, while 
learning  an  Individual  True  Name  takes  one  week 
(see §3.9)  

•  Further  ranking  of  both  Generic  and  Individual 
True Names takes 1 week × Rank to be achieved.  

For  full  details  on  the  use  of  Counterspells  see 
§10.2. 

• Ranking Names costs no EP.  

• The maximum Rank for True Names is 20.  

•  Ranking  of  Names  may  be  done  in  combination 
with ranking either a magical or non-magical abil-
ity.  

• Namers may Rank one Name in addition to other 
forms of ranking. 


\section{Talents}

Detect Aura (T-1) 

Experience Multiple: 75 
Base Chance: 2 × PC ( + 5 / Rank) 
Resist: May only be actively resisted 
Target: Entity, Object, Area, Volume 
Effects:  The  Base  Chance  is  reduced  by  1\%  for 
every  foot  after  the  first  five  from  the  Adept.  See 
the Detect Aura Talent (§9.1) for the results of this 
talent.  In  addition  to  other  information  gained,  the 
Namer  also  receives  the  target’s  Generic  True 
Name, if any. 

Expel Magic (T-2) 

Experience Multiple: 75 
Resist: May only be passively resisted 
Target: Object, Area, Volume 
Effects: This talent allows the Namer to dissipate a 
magical  spell  stored  in  a  Ward,  Magical  Trap, 
Potion, or Invested Item. In order to use this talent, 
the Namer must specify the name of the spell to be 
affected  and  cast  the  appropriate  Counterspell  on 
the target with the specific intent of dissipating the 
stored  magic.  The  chance  of  the  stored  magic  re-
sisting  destruction  is  50\%  [(  +  3  /  Rank  of  the 
stored magic) (3 / Rank of this Talent)]. If success-
ful all of the magic of the same type stored within 
the  target  is  destroyed.  The  appropriate  Counter-
spell is the one that affects the magic stored not the 
storing  magic.  For  example,  a  Ward  of  Enchanted 
Sleep  would  require  the  use  of  an  E  \&  E  General 
Knowledge Counterspell. Possessions gain a single 
Resistance  Check  but  use  the  better  of  the  chance 
above, or their wielder’s MR. 

Quick Cast (T-3) 

Effects:  Namers  may  cast  any  Counterspell  that 
they know without preparing it first. 


\section{General Knowledge spells}

The  entire  general  spell  knowledge  of  the  Namer 
college consists of the ability to cast Counterspells. 
A  Namer  may  cast  a  Rank  0  Counterspell  against 
any College of magic with which they are familiar. 
Counterspells  at  Rank  0  do  not  count  towards  the 
Namers MA limit for spells and rituals. 

Familiarity  with  all  of  the  commonly  encountered 
Colleges  will  be  taught  to  a  Namer  during  their 
apprenticeship,  and  beginning  Namers  will  have 
the  ability  to  cast  all  of  the  Counterspells  of  the 
standard Colleges at Rank 0. If a Namer encounters 
Colleged magic of a form with which they are not 
familiar  they  may  familiarize  themselves  with  the 
College by one of the following methods:  

• By using the Ritual of Divination on an Adept of 
that College.  

•  By  Divinating  a  magical  effect  produced  by  that 
College, provided that it is still in effect.  

•  By  spending  a day’s  study  with  a Namer  who  is 
already familiar with the College. 

Once they have done this they  will be able to cast 
Rank 0 Counterspells against that college. 

Unlike other Adepts, Namers may gain Rank with Counterspells that are
not of their College. Namers rank all Counterspells as General
Knowledge spells of the Namer College.  Once a Namer begins to rank a
Counterspell it will count towards the Namers MA limit for spells and
rituals as normal.



\section{General Knowledge Rituals}


Dissipation (Q-1) 


Target: Entity, Object, Area, Volume 
Base Chance: As per Counterspell + Ritual prepa-
ration 
Cast Time: 1+ hours, maximum 10 
Actions: Concentration 
Concentration Check: Standard 
Effects:  By  engaging  in  Ritual  Spell  Preparation  a 
Namer  may  use  a  Counterspell  to  dissipate  the 
effects of a spell. The Namer must perform at least 
one  hour  of  Ritual  Spell  Preparation  at  the  end  of 
which they must cast the appropriate Counterspell, 
and specify the name of the spell to  be dissipated. 
Only  spells  (not  rituals)  may  be  dissipated  using 
this  technique.  It  is  not  possible  to  achieve  Rank 
with  this  ability  since  it  is  not  an  independent  rit-
ual,  but  rather  a  specialized  use  of  Ritual  Spell 
Preparation. 


\section{Special Knowledge Spells}

Bane (S-1) 

Range: 10 feet + 10 / Rank 
Duration: 30 seconds + 5 / Rank 
Experience Multiple: 300 
Base Chance: 20\% 
Resist: Passive 
Storage: Investment, Ward, Magical Trap 
Target: Area 
Effects: This spell strengthens reality and stabilizes 
the  mana  in  an  area  15  feet  in  diameter  (+  10  /  5 
Ranks)  such  that  all  magical  Cast  Chances  are 
reduced  within  the  area  by  5\%  (+  3  /  Rank).  This 
will  affect  spells  and  rituals, and  talents  with  base 
chances.  The  spell  has  no  affect  on  stored  magics 
(such  as  invested  items),  shaped  items,  or  magic 
without base chances. 

Banishment (S-2) 

Range: Self 
Duration: 10 seconds + 10 / Rank 
Experience Multiple: 200 
Base Chance: 20\% 
Resist: No 
Storage: Potion 
Target: Self 
Effects:  Through  use  of  this  spell,  a  Namer  may 
banish  a  summoned  being  back  to  its  own  plane. 
While  the  banishment  spell  is in  effect,  the  appro-
priate  Counterspell  cast  by  the  Namer  at  a  sum-
moned  entity  will  cause  the  entity  to  return  to  its 
own plane, unless it resists. The Counterspell must 
match  the  type  of  spell  or  ritual  used  in  the  sum-
moning of the creature. In general the spells/rituals 
affected  are  the  elemental  summonings  (Summon 
Fire  Elemental,  etc.),  all  Greater  Summonings, 
Dark/Light  Sphere  conjuration  and  Fire  college 
Efreeti  and  Salamander  summoning.  The  Call 
Patron ability of Agents is not classified as a sum-
moning  spell  and  is  not  affected.  The  being  to  be 
banished  may  actively  and  passively  resist  the 
Counterspell. 

Compel Obedience (S-3) 

Range: 15 feet + 5 / Rank 
Duration:  Concentration:  max.  10  minutes  +  10  / 
Rank 
Experience Multiple: 400 
Base Chance: 20\% 
Resist: Active, Passive 
Target: Entity 
Storage: None 
Effects: The Adept may cast this spell over 1 (+ 1 / 
4  Ranks)  targets  whose  Generic  True  names  are 
known  to  them.  Those  targets  who  fail  to  resist 
may  be  commanded  by  the  Adept  to  perform  ac-
tions that are both within their physical capabilities 
and  in  their  true  natures.  Commands  are  given  in 
the  Namer  tongue  and  will  be  understood  by  all 
entities. Commands must be short and simple, such 
as:  “Stop!”,  “Wait  here”,  “Follow  me”,  “Hide 
under the table”. Entities can only be compelled to 
perform actions that  they  might  perform naturally. 
For  example,  brigands  who  were  involved  in  a 
combat might be compelled to “Flee!”, but if those 
same  brigands  felt  they  were  winning  the  fight, 
they  would  heed  no  such  compulsion  but  could 
perhaps be directed to a different target. If the spell 
is  cast  at  targets  with  different  GTNs  the  Namer 
must  use  the  lowest  applicable  base  chance  modi-
fier. 

If  the  Adept  chooses  to  pronounce  a  target’s  Indi-
vidual True Name as part of the spell then only one 
entity may be affected but the Namer is vested with 
much greater control over that entity, even against 
its  nature.  It  is  possible  for  the  target  to  defy  the 
Namer,  but  there  are  serious  consequences  for 
disobedience. Should the target decline to obey any 
command  of the  caster  that  is  not  obviously  suici-
dal, they must make a Willpower check of [1 × WP 
- (1\% per Rank that the Namer has with the target’s 
ITN)]. This check does not break the spell. Should 
the target fail their check, they will feel great pain 
and  immediately  take  damage  equal  to  half  of  the 
Namer’s Rank with this spell. This damage cannot 
be resisted. 

Disjunction (S-4) 

Range: 10 feet + 10 / Rank 
Duration: 1 minute + 1 / Rank 
Experience Multiple: 300 
Base Chance: 30\% 
Resist: Passive 
Storage: Investment, Ward, Magical Trap 
Target: Object, Area 
Effects: This spell prevents stored magics within an 
object or area from coming into effect. Magics that 
are  affected  by  this  spell  include  Wards,  Invested 
items,  Potions,  Magical  Traps,  and  permanent 
magics that need to be triggered.  If a potion under 
the effects of a Disjunction is consumed, the potion 
will  take  effect  after  the  spell  effect  ceases,  pro-
vided  it  is  still  inside  an  entity.  Other  items  will 
simply  be  unable  to  be  triggered,  and  no  charges 
will be lost. 

Dispel Magic (S-5) 

Range: Self 
Duration: 5 seconds + 5 / 4 Ranks 
Experience Multiple: 400 
Base Chance: 5\% 
Resist: No 
Storage: Potion 
Target: Self 
Effects:  While  the  Dispel  Magic  is  in  effect,  the 
appropriate Counterspell cast at a target may dissi-
pate magic. The Counterspell must match the  type 
of spell to be dissipated, and the Adept must spec-
ify the name of the spell that they wish to remove. 
If the Counterspell is successfully cast, the chance 
of  the  magic  being  dispelled  is  50\%  [(+  3  /  Rank 
with Dispel Magic) (3 /Rank of the target magic)]. 
This  spell  cannot  remove  the  effects  of  rituals,  or 
remove curses. 

Forbidding (S-6) 

Range: 10 feet + 10 / Rank 
Duration: 10 minutes + 10 / Rank 
Experience Multiple: 250 
Base Chance: 30\% 
Resist: Passive 
Storage: Investment, Ward, Magical Trap 
Target: Area 
Effects: This spell creates a thin, invisible wall, 10 
feet high and 20 feet long. The Adept may increase 
either  height  or  length  by  1  foot  per  Rank.  This 
barrier obeys all of the usual rules for insubstantial 
walls.  A  single  Generic  or  Individual  true name  is 
crafted into the forbidding. To those entities whose 
names are contained therein, or if seen by means of 

Witchsight  or  similar,  the  wall  appears  bluish  and 
crackling  with  magical  energy.  If  a  Generic  True 
Name  is  in  the  forbidding,  then  to  those  named 
who fail to resist upon initial contact the forbidding 
is completely solid to them and they  are unable to 
pass through  it.  If  they  resist,  the  barrier  is  insub-
stantial, as it is those who are not named by it. If an 
Individual  True  name  is  placed  in  the  forbidding 
then  in  addition  to  the  Generic  effects,  the  entity 
must  resist  each  contact  with  the  barrier  or  suffer 
[D - 4] + 1 / Rank damage, even if they are able to 
pass  through  the  wall  because  they  initially  re-
sisted. 

Mana Sense (S-7) 

Range: Self 
Duration: 5 minutes + 5 / Rank 
Experience Multiple: 200 
Base Chance: 20\% 
Resist: No 
Storage: Potion 
Target: Self 
Effects:  This  spell  allows  the  adept  to  “sense”  the 
mana flows within 10 feet (+ 10 / Rank). If a spell 
is  cast  or  magic  triggered  within  range,  the  Adept 
will see it flying off towards its target. Similarly, if 
the  target  of  any  spell  is  within  range,  the  Adept 
will  see  the  magic  impact.  If  the  Adept  chooses 
magical  Pass  actions  of  Concentration  with  this 
spell  in  effect,  they  will  be  able  to  see  Adepts 
drawing mana, and be able to see if a target resists 
a  spell  or  not.  While  concentrating  the  Adept  will 
have a (2 × PC) chance of being able to distinguish 
the College of the magic they can see, the name of 
the  spell,  and  whether  the  spell  is  low,  medium, 
high or very high in rank. 

Scry Shield (S-8) 

Range: 10 feet + 5 / Rank 
Duration: 10 minutes + 10 / Rank or Special 
Experience Multiple: 300 
Base Chance: 20\% 
Resist: No 
Storage: Investment, Ward, Magical Trap 
Target: Volume 
Effects: This spell protects an area from scrying by 
Wizard  Eyes,  Crystals  and  Waters  of  Vision, 
Bard’s Ear and similar divinatory magics of a Rank 
equal to or less than the rank of the Scry Shield. It 
does not prevent normal vision, infravision, Witch-
sight  and  similar  spells.  A  Scry  Shield  is  a  shell 
over  the  protected  volume,  so  once  the  area  is 
penetrated by any means, e.g. on foot or by flying, 
spells  cast  inside  the  protected  volume  work  nor-
mally. At Rank 20 this spell alarms the Adept that 
an  attempt  to  divine  into  the  volume  by  magical 
means  has  taken  place,  provided  that  the  Adept  is 
within  the  volume  at  the  time.  This  spell  may  be 
cast as a ritual if the Adept so chooses. In this form 
casting takes 10 hours and the duration is increased 
to 4 weeks (+ 1 / Rank). 

Spell Barrier (S-9) 

Range: 10 feet + 5 / Rank 
Duration: 1 minute + 1 / Rank 
Experience Multiple: 300 
Base Chance: 30\% 
Resist: No 
Storage: Investment, Ward, Magical Trap 
Target: Volume 
Effects: The Adept creates a thin, glowing, translu-
cent  wall  which blocks  the  passage  of  magic.  The 
barrier  is  either  10  feet  high and 20 feet  long  as  a 
wall, or 10 feet high and 5 feet in radius as a ring. 
The  Adept  may  increase  any  dimension  —  other 
than thickness — by 1 foot per Rank. This barrier 
obeys all  of the usual rules for insubstantial walls. 
Any  magic  cast  in  such  a  way  that  a  direct  line 
drawn from the caster to their target passes through 
the wall (from either side) has a 40\% [(+ 3 / Rank 
with  this  spell)  (3  /  Rank  of  the  target  magic)] 
chance  of  having  its  energies  dissipated.  If  a  spell 
passes through more than one Spell  Barrier only a 
single roll for dissipation should be made, with the 
highest dissipation chance being used. 

True Seeing (S-10) 

Range: 10 feet + 5 / 2 Ranks 
Duration: 30 seconds + 10 / Rank 
Experience Multiple: 300 
Base Chance: 25\% 
Resist: No 
Storage: Investment, Ward, Magical Trap 
Target: Area 
Effects: All entities within the area of effect have a 
chance  of  detecting  things  in  the  area  that  have 
been  magically  altered  to  appear  other  than  they 
truly  are.  True  Seeing  may  reveal  the  visual  com-
ponent  of  illusions, and  entities  or  objects  that  are 
invisible,  insubstantial,  or  have  been  magically 
rearranged or transformed (such as by curses). The 
chance of an observer detecting such alterations or 
concealments  will  depend  upon  the  rank  of  this 
spell,  the  rank  of  the  concealing  or  transforming 
magic, and the perception of the observer. For odd 
magical  abilities  without  a  Rank  the  GM  should 
substitute  some  appropriate  alternative  (such  as 
[MA - 10]). 

True  Seeing  is  of  lesser  or  equal  Rank:  Slight 
imperfections  may  be  revealed,  (e.g.  invisible 
figures  shimmer  a  little,  the  colour  of  an  illusion 
may appear a bit off, etc.), and there is a (1 × PC) 
chance  of  an  observer  scrutinizing  the  area  noting 
this. 

True  Seeing  is  of  higher  Rank:  More  major  im-
perfections  may  be  noticed.  (e.g.  invisible  figures 
have  a  slight  will-o’-the-wisp  glow,  toads  that  are 
really  Princes  may  have  tiny  gold  crowns,  ).  The 
detection chance rises to (2 × PC). 

True Seeing is 10+ Ranks higher: The imperfec-
tions  in  concealing  and  transforming  magics  be-
come  quite  obvious,  (e.g.  invisible  figures  appear 
ghostly,  illusions  may  appear  painted  or  translu-
cent, etc.). Detection is automatic. 


\section{Special Knowledge Rituals}

Divination (R-1) 

Range: 5 feet + 1 / Rank 
Duration: Immediate 
Experience Multiple: 250 
Base Chance: 40\% + 10 / Rank 
Cast Time: 1 hour or 3 hours 
Resist: No 
Target: Entity, Object, Area 
Material: None 
Actions: Concentration 
Concentration Check: Standard 
Effects:  There  is  no  possibility  of  backfire  from 
this  ritual.  By  use  of  this  ritual  a  Namer  may  de-
termine if an individual, object, or area is currently, 
or  has  been  recently,  under  the  effects  of  magic  a 
spell by employing the Ritual of Magic Divination. 
If the ritual is successful, the nature of all magic in 
effect (exact names and Colleges) is revealed to the 
Namer.  If  the  magic  is  of  non-college  origin  gen-
eral  effects  are  revealed.  In  the  case  of  magic  that 
is  no  longer  in  effect,  for  each  5\%  under  the  Cast 
Chance  that  the  Namer  rolled,  magic  that  expired 
an  extra  week  ago  is  revealed.  For  example  if  a 
Namer  rolled  12\%  under  their  Cast  Chance  magic 
that  expired  up  to  two  weeks  ago  would  be  re-
vealed  —  in  addition  to  all  magic  currently  in 
effect. 

If the Namer wishes they may perform an Ancient 
Divination.  The  Base  Chance  of  the  ritual  is  re-
duced  to  40\%  (+  2  /  Rank),  and  the  Cast  Time 
increased  to  3  hours.  If  successful  the  Namer  will 
learn the exact nature of all enchantments, magical 
mechanisms,  triggering  conditions,  curses,  side-
effects, etc., placed upon an entity or object even if 
they  are  of  non-college  origin.  If  an  object  has  an 
Individual  True  Name  the  Ancient  Divination  will 
reveal its existence, though not the actual name. 

Expulsion (R-2) 

Range: 5 feet + 1 / Rank 
Duration: Immediate 
Experience Multiple: 300 
Base Chance: 10\% + 5 / Rank 

Cast Time: 1 hour 
Resist: Active, Passive 
Target: Entity 
Material: None 
Actions: Concentration 
Concentration Check: Standard 
Effects:  This  Ritual  will  return  one  entity  to  its 
plane  of  origin,  regardless  of  how  it  got  to  the 
current plane. Upon completion of the ritual, if the 
target  fails  to  resist,  they  will  be  immediately  re-
turned to the point from which they left their plane 
of origin. 

Interregnum (R-3) 

Range: 10 feet 
Duration: Special 
Experience Multiple: 250 
Base Chance: MA + 4\% / Rank 
Cast Time: 2 hours 
Resist: Active, Passive 
Target: Entity or Object 
Material: None 
Actions: Concentration 
Concentration Check: Standard 
Effects: The targeted entity or object has all magi-
cal effects currently upon them, that are of lesser or 
equal  rank  to  the  Interregnum,  suspended.  Whilst 
suspended  their  durations  will  not  reduce,  but  the 
magic  will  have  no  effect.  The  duration  of  the 
Interregnum  may  be  chosen  by  the  Adept  at  the 
time  of  casting,  from  a  minimum  of  1  day  to  a 
maximum of: 

Rank  Duration (maximum) 

0–10 
11–15 
16–19 
20 

1 day (+ 1 / Rank) 
1 month 
3 months 
1 year 

Remove Curse (R-4) 

Effects: Namers have a greater ability to remove curses than do the
Adepts of other Colleges. Namers learn the normal Remove Curse ritual
(see §11.3) as R4 of this College, but gain a bonus to Base Chance of
+ 2 / Rank for Minor curses and + 1 / Rank for Major curses (including
Death Curses).

Sealing (R-5) 

Range: 20 feet +20 / Rank 
Duration: 1 day + 1 / Rank 
Experience Multiple: 300 
Base Chance: 20\% + 4 / Rank 
Cast Time: 1 hour 
Resist: None 
Target: Area 
Material:  Chalk,  paint,  blood,  cornmeal  or  other 
symbol making materials 
Actions: Chanting and inscribing symbols 
Concentration Check: Standard 
Effects:  This  Ritual  seals  an  area  against  entities 
from a single, specific, named plane. The name of 
the  plane  must  be  known  to  the  Adept,  and  the 
name of the plane that the Adept is currently occu-
pying  cannot  be  used.  No  entity  whose  plane  of 
origin has been sealed against can voluntarily enter 
the sealed area. They will stop at the boundary and 
refuse  to  go  any  further.  Any  entity  taken into  the 
area against their will (or without their knowledge, 
e.g.  unconscious)  will  attempt  to  leave  the  area  as 
quickly  as  possible.  If  an  attempt  is made  to  sum-
mon  an  entity  from  the  named  plane  into  the  area 
the  summoning  will  fail.  Any  entities  from  the 
named  plane  who  are  inside  the  area  when  the 
sealing  is  created  are  unaffected,  but  should  they 
leave the area they will be unable to re-enter it. 

True Form (R-6) 

Range: 5 feet 
Duration: Immediate 
Experience Multiple: 300 
Base Chance: 20\% + 3 / Rank 
Cast Time: 3 hours 
Resist: Active 
Target: Entity, Object 
Material: None 
Actions: Concentration 
Concentration Check: Standard 
Effects:  By  means  of  this  ritual  the  Adept  may 
force  a  target  that  has  been  magically  altered, 
cursed,  or  rearranged  into  a  form  other  than  their 
natural one to assume their true form and nature. It 
will  not  remove  effects  that  could  occur  naturally. 
For example, the ritual would restore the form of a 
human  that  had  been  cursed  into  the  shape  of  a 
toad,  and  would  return  to  flesh  a  human  turned  to 
stone  but  would  do  nothing  to  remove  a  curse  of 
weeping sores or restore a lost limb. 

True Speaking (R-7) 

Range: 10 feet 
Duration: 30 minutes 
Experience Multiple: 300 
Base Chance: 40\% + 3 / Rank 
Cast Time: Special / 1 hour 
Resist: Active, Passive 
Target: Entity 
Material: None 
Actions: Asking questions 
Concentration Check: Standard 
Effects:  By  means  of  this  ritual  the  Adept  may 
attempt  to  force  an  entity  who  is  present  to  speak 
the  truth.  The  Adept  must  prepare  for  30  minutes 
after  which  they  may  question  the  entity  for  the 
remaining  30  minutes  of  the  1  hour  ritual.  The 
effects  of  the  ritual  do  not  last  beyond  the  hour. 
The  target  need  not  answer  or  speak  at  all,  but  if 
they  fail  to  resist  and  they  choose  to  answer  the 
Adept’s  questions,  they  must,  to  the  best  of  their 
knowledge,  speak  no  falsehoods.  They  need  not 
volunteer  information.  The  GM  rolls  for  the  suc-
cess of this ritual and need not inform the Adept’s 
player of the result. 

\end{Chapter}
