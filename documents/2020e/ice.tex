\begin{Chapter}{The College of Ice Magics (Ver 1.5)}

The College of Ice Magics is concerned with the shaping of ice and
snow, freezing and the manipulation of cold.

Ice Mages have been most valuable to those fragile 
communities living in cold, arid and often danger-
ous  places,  where  food  is  hard  to  grow  and 
neighbours  can  be  separated  by  days  of  travel. 
Many  of  the  Ice  Mages’  Spells  and  Rituals  are 
designed either to enhance living in these inhospi-
table  areas  or  to  aid  hunting  and  defending  them-
selves from arctic predators. Ice Mages are seldom 
seen  in  warm  climates,  and  have  been  unpopular 
with  Philosophers  of  magic  who  cannot  agree 
whether  to  place  Cold  as  an  element  in  its  own 
right.  The  College  of  Ice  Magics has  been  likened 
to  the  Colleges  of  Water,  due  to  its  connections 
with  Ice and Snow,  Fire, as an antithesis, Air, due 
to  its  weather-like  effects  and  Mind,  due  to  the 
peculiar  mental  fortitude  its  Adepts  have  demon-
strated while living in cold climes. 

Philosophy 

Ice Mages tend to be of a solitary and quiet nature, 
often  having  lived  their  lives  isolated  in  areas  of 
sparse  population  density.  Many  are  hunters  or 
only  part-time  Mages.  They  generally  have  fair 
relations with most Colleges, but relations with the 
Fire and Water Colleges are distinctly icy. 

Traditional Colours 

Ice  Mages  are  not  known  to  dress  flamboyantly. 
Whites and Greys are popular colours when dress-
ing  in  warmer  climes,  perhaps  with  a  hint  of  pale 
blue.  However  in  their  own  element  they  usually 
wear undyed leathers and furs. 

Notes 

All the Ice and Snow generated by the spells of this 
college is real unless otherwise specified. It is in no 
way magical, and does not have a duration. Exter-
nal effects (e.g. heat) will begin to affect it imme-
diately upon its generation. 

Ice  Magics  is  considered  “opposite”  to  Fire 
Magics.  Many  Ice  spells  and  effects  are  uniquely 
vulnerable to the spells of the Fire College, and to 
Fire. Additionally Ice Magics have some spells and 
effects that are especially effective versus Fire and 
Fire  creatures.  This  in  no  way  affects  the  “oppo-
site”  relationship  that  the  Colleges  of  Water  and 
Fire Magics have to each other. Water Magics and 
Ice  Magics  are  not  considered  related,  rather  they 
are generally regarded as antagonistic rivals. 


\section{Restrictions}

Adepts  of  the  College  of  Ice  Magics  may  practise 
their arts without restriction. 

The MA requirement for this College is 13. 


\section{Base Chance Modifiers}

All  base  chances  of  Ice  Mages  are  affected  by 
temperature. Only one of the temperature modifiers 
may  be  applied  at  one  time.  Consult  the  Weather 
Scale Table for Weather Gauge details. 

Temp 

Gauge  Temp °C  Mod. 

0-3 
4-5 
6-8 
9-10 
11 + 

<= - 5 
<= 5 
6-24 
>= 25 
>= 35 

Very Cold 
Cold 
Average 
Hot 
Very Hot 
If the Adept is standing on  Ice or Snow in an area 
where  it  is  in  abundance  they  gain  an  additional 
+5\%. 

+ 10 
+ 5 
+ 0 
- 5 
- 10 


\section{Talents}

\begin{talent}[T-1]{Cold Affinity }

\multiple{100 }
\begin{effects}
 This  talent allows  the  Adept  to  ignore  the 
deleterious  effects  of  low  body  temperature.  The 
adept  is  treated  as  if having a Resist Cold  spell  of 
equal Rank to the Rank in this talent in effect at all 
times. Should the Adept also be under the effect of 

a Resist Cold spell, the higher of the two ranks is in 
effect. 

\end{effects}
\end{talent}

\begin{talent}[T-2]{Endure Hardship }

\multiple{150 }
\begin{effects}
 This  talent  allows  the  Adept  to  function 
capably  even  in  harsh  and  forbidding  environ-
ments.  The  Adept  may  go  without  food  (but  not 
water!) for 1 (+1 / Rank) days  every three months 
with no ill effects. These days may be taken singly 
or consecutively, and the Adept need not consume 
extra  food  later  to  make  up  for  this  time  spent 
fasting.  The  Adept  may  additionally  increase  the 
Base Chance of any concentration checks made in 
hostile  environments,  when 
the  concentration checks are reduced below 4 × WP due to environmental effects, by 1 × WP / 5 full Ranks the Adept 
has  in  this  talent,  up  to  a  maximum  of  4  ×  WP. 
Finally,  the  Adept  may  reduce  high  fatigue  rates 
due  to  environmental  and  weather  effects,  other 
than those relating to heat or fire, by a 1 row shift 
per 10 full ranks the Adept has in this talent on the 
Rate  of  Exercise  Chart  (see  §58.1,  under  the  Fa-
tigue  and  Encumbrance  Chart),  down  to  a  mini-
mum  rate  of  medium  fatigue,  or  light  fatigue  at 
rank  20.  For  example,  at  Rank  10  a  Strenuous 
climb up a mountain may be treated as if it is only 
a Hard climb. 


\end{effects}
\end{talent}

\section{General Knowledge Spells}

\begin{spell}[G-1]{Extinguish Fires }

\range{15 feet + 15 / Rank }
\duration{Immediate }
\multiple{100 }
\basechance{50\% }
\resist{None }
\storage{Investment, Ward, Magical Trap }
\target{Volume }
\begin{effects}
 When  successfully  cast,  this  spell  will 
extinguish all fire in a 10 foot (+ 10 / Rank) radius 
sphere,  by  smothering  them  with  ice  crystals.  All 
the  volume  affected  must  be  within  range  of  the 
spell. If the range is doubled or tripled the volume 
may  likewise  be  increased.  Magical  fires  are  not 
affected.  This  spell  is  identical  to the  Fire  College 
“Extinguish  Fires”,  except 
is 
achieved  by  physical  means  (this  is  merely  a  cos-
metic difference). 

that  the  effect 

\end{effects}
\end{spell}

\begin{spell}[G-2]{Freeze }

\range{5 feet + 5 / Rank }
\duration{1 day + 1 / Rank }
\multiple{50 }
\basechance{40\% }
\resist{None }
\storage{Investment }
\target{Object }
\begin{effects}
The Adept may freeze  one object of up to 
5  lb.  (+  5  /  Rank).  This  freezing  will  protect  the 
object  from  decay  while  the  duration  lasts.  While 
frozen the object will be as cold as ice to the touch 
and will drip slightly from condensation. When the 
duration  has  expired  the  object  will  defrost  at  the 
rate of 1 minute per pound of weight. 

\end{effects}
\end{spell}

\begin{spell}[G-3]{Ice Creation }

\range{15 feet + 10 / Rank }
\duration{Immediate }
\multiple{100 }
\basechance{25\% }
\resist{None }
\storage{Investment, Ward, Magical Trap }
\target{Area }
\begin{effects}
This spell creates a film of ice 1 inch thick 
in a single square of dimensions 1 (+ 1 / Rank) foot 
each side or a single cube of ice of dimensions 6 (+ 
6 / Rank) inches cubed. The ice must be created on 
the  ground  and  not  on  top  of  an  entity.  It  is  non-
magical and will persist until melted, etc. 

\end{effects}
\end{spell}

\begin{spell}[G-4]{Ice Traversal }

\range{10 feet + 10 / Rank }
\duration{20 minutes + 20 / Rank }
\multiple{125 }
\basechance{30\% }
\resist{None }
\storage{Investment }
\target{Entity }
\begin{effects}
 This  spell  enables  1  target  (+  1  /  4  full 
Ranks)  to  travel  over  ice  and/or  snow  without 
slipping  or  sinking  in,  as  if  it  were  normal  earth 
and/or rock. For example, this would enable climb-
ers  to  climb  icy  slopes.  Quadrupeds  are  treated  as 
two  targets  for  the  purposes  of  this  spell.  In  addi-
tion,  if  the  terrain  travelled  on  is  flat  ice,  each 
target’s TMR is increased by 1 (+ 1 / 3 full Ranks) 
while  on  the  ice.  See  Travel  on  Ice  (22.8),  for 
additional detail. 

\end{effects}
\end{spell}

\begin{spell}[G-5]{Refrigeration }

\range{25 feet + 5 / Rank }
\duration{1 hour + 1 / Rank }
\multiple{50 }
\basechance{35\% }
\resist{None }
\storage{Investment }
\target{Volume }
\begin{effects}
The caster may cause the ambient tempera-
ture of one 15 × 15 × 15 foot cube to lower by 2°C 
/ Rank.  

\end{effects}
\end{spell}

\begin{spell}[G-6]{Resist Cold }

\range{Touch }
\duration{1 hour + 1 / Rank }
\multiple{100 }
\basechance{40\% }
\resist{None }
\storage{Investment, Ward, Potion }
\target{Entity }
\begin{effects}
 This  spell  protects  the  target  from  the 
effects  of  cold  temperature  by  increasing  the 
Gauge by 1 (+ 1 / 4 full Ranks) up to a maximum 
of Gauge 7 (Comfortable). It will totally protect the 
target  from  the  effects  of  Hypothermia  at  Rank 
11+.  In  addition,  the  target  suffers  1  (+  1  /  4  or 
fraction Ranks) less damage due to magical or non-
magical  cold  based  attacks.  This  spell  is  identical 
to  the  special  knowledge  Air  college  spell  of  the 
same name. 

\end{effects}
\end{spell}

\begin{spell}[G-7]{Snow Shovel }

\range{Self }
\duration{Concentration: maximum of 15 minutes }
+ 15 / Rank 
\multiple{125 }
\basechance{20\% }
\resist{None }
\storage{Potion }
\target{Self }
\begin{effects}
This spell enables the Adept to clear a path 
along  snow  or  ice  obstructed  ground  and/or  to 
tunnel  through  snow  and  ice.  Any  snow  or  ice  up 
to 2 feet in front of the Adept undergoes a change 
in density to dry snow, and is pushed to either side, 
leaving  a  gap  2  feet  wider,  and  higher  (if  applica-
ble),  than  the  size  of  the  Adept.  No  more  than  1 
hex (+ 1 / 2 full Ranks),  may be cleared per pulse 
in  this  manner.  This  effect  moves  with  the  Adept. 
The  Adept  may  lean  in  order  to  direct the path  up 
or  down.  Note  that  the  walls  and  roof  of  a  tunnel 
through  snow  or  ice  are  merely  packed  snow  and 
do  not  confer  any  particular  structural  support  or 
stability. See Travel on or through Snow (22.8) for 
additional detail. 

\end{effects}
\end{spell}

\begin{spell}[G-8]{Water to Ice }

\range{10 feet + 10 / Rank }
\duration{Immediate }
\multiple{100 }
\basechance{15\% }
\resist{None }
\storage{Investment, Ward, Magical Trap }
\target{Volume of Water }
\begin{effects}
 The  Adept  can  freeze  up  to  10  (+  10  / 
Rank)  cubic  feet  of  existing  water  based  liquids 
into solid ice, or into snow of a density chosen by 
the Adept. All the water to be transformed must be 
within  the  Adept’s  range  at  the  time  of  casting. 
This  spell  may  not  be  cast  on  or  near  entities  or 
their possessions. 


\end{effects}
\end{spell}

\section{General Knowledge Rituals}

\begin{ritual}[Q-1]{Create Igloo }

\duration{1 hour + 1 / Rank }
\multiple{100 }
\basechance{MA + 4\% / Rank }
\begin{effects}
 The  Adept  must  spend  one  hour  in  ritual 
construction of a miniature dome made out of snow 
or ice cubes. At the end of this time the Adept must 
make  a  successful  cast  check.  If  successful,  the 
dome swells in size to become an igloo of internal 
size 5 (+ 1 / Rank) feet radius and 2 feet thick. This 
ritual  cannot  backfire.  The  Igloo  has  a  single  en-
trance  which  is  chosen  by  the  Adept  to  be  up  to 
half  its  internal  height  in  both  height  and  length. 
The  inside  temperature  of  the  igloo  always  counts 
as  very  cold  (-10  degrees)  and  the  following  en-
chantments  apply  to  objects  or  entities  while  they 
remain inside the igloo:  

•  All  entities  and  creatures  are  treated  as  having  a 
resist cold spell upon them of equal rank to that of 
the Adepts Rank in this ritual  

• All organic objects are preserved from decay 

In  addition  the  igloo  counts  as  bound  snow  while 
the duration is in effect and will not melt or break 
due  to  non-magical  forces  (although  magical  at-
tacks  affect  it  as  normal).  Once  the  duration  runs 
out  the  igloo  reverts  to  a  normal  (non-magical) 
igloo  and  will  thereafter  melt,  collapse  etc.  as 
normal due to external conditions. 


\end{effects}
\end{ritual}

\begin{ritual}[Q-2]{Bind Ice and Snow }

\duration{ Concentration:  Maximum  1  hour  +  1  / }
Rank 
\multiple{750 }
\basechance{MA + 4\% / Rank }
\begin{effects}
 The  Adept  may  bind  all  of  the  ice  and 
snow  within  a  5  (+  5  /  Rank)  feet  radius  circle  of 
the  Adept.  The  results  of  this  ritual  are  similar  to 
those for the binding of other elements. The Adept 
gains  control  of  all  of  the  facets  of  the  element. 
The  Adept  may  move  or  shape  the  ice  and  snow, 
change  its  consistency  and  instil  intelligence  in  it 
as desired. Finally, the Adept may sacrifice a point 
of MA (this may be bought back with EP) in order 
to  make  a  part  or  all  of  the  bound  ice  and  snow 
permanently bound. In this instance, the bound ice 
and  snow  is  non-intelligent  but  magical,  and  is 
enduring. Almost no magical or physical force will 
affect  it  (e.g.  it  resists  Wizard’s  Eye  and  Telepa-
thy),  with  the  exception  of  magical  heat  and  fire, 
against  which  it  has  100  times  the  resistance  of 
ordinary ice and snow, and if any part of it remains 
it will (slowly) regenerate. 






\end{effects}
\end{ritual}

\section{Special Knowledge Spells}

\begin{spell}[S-1]{Armour of Ice }

\range{Touch }
\duration{30 minutes + 30 / Rank }
\multiple{250 }
\basechance{20\% }
\resist{Active, Passive }
\storage{Investment, Ward, Magical Trap }
\target{Entity }
\begin{effects}
 The  target  of  this  spell  is  covered  by  a 
magical Armour of Ice, which provides 5 points of 
Armour Protection (+ 1 / 4 full Ranks) which fluc-
tuates  according  to  the  current  temperature:  the 
Armour  gains  +2  points  of  Armour  Protection 
when  the  temperature  is  very  cold,  +1  when  it’s 
cold,  -1  when  it’s  hot,  and  -2  when  it’s  very  hot 
(see 22.2). The armour has a weight rating of 5 (see 
56.3) and subtracts 2 from AG and 20 from stealth. 
Ice  Armour  may  not  be  cast  on  entities  wearing 
armour.  This  spell  will  stack  with  other  defensive 
spells.  The  Ice  Armour  is  vulnerable  to  Fire,  and 
has  an  ablative  effect;  it  will  absorb  up  to  half  of 
any  fire  damage  taken,  but  for  every  10  points  of 
damage taken (before halving) one point of protec-
tion is removed from the armour, and if the protec-
tion  is  reduced  to  zero  the  spell  is  immediately 
dissipated. 

\end{effects}
\end{spell}

\begin{spell}[S-2]{Icy Transformation }

\range{10 feet + 10 / Rank }
\duration{10 minutes + 10 / Rank }
\multiple{300 }
\basechance{25\% }
\resist{Special }
\storage{Investment, Ward, Magical Trap }
\target{One (metal or mineral) Object }
\begin{effects}
 This  spell  turns  one  metal  or  mineral  ob-
ject of up to 5 lbs (+ 5 / Rank) entirely into ice (a 
wall of iron would have to be entirely turned to ice 
but a single brick in a wall could be transformed). 
The  object  is  then  transparent  and  vulnerable  to 
damage,  heat  etc.  At  the  end  of  the  duration,  the 
object will revert back, but any damage will not be 
repaired. 

\end{effects}
\end{spell}

\begin{spell}[S-3]{Freezing Wind }

\range{5 feet + 5 / Rank }
\duration{5 seconds + 5 / Rank }
\multiple{225 }
\basechance{30\% }
\resist{Passive }
\storage{Investment, Ward, Magical Trap }
\target{Volume }
\begin{effects}
 This  spell  causes  Arctic  conditions  to 
prevail  in  a  10  (+  1  /  Rank)  feet  cube.  Any  entity 
within this cube which fails to resist will suffer [D 
-  4]  (+  1  /  2  full  Ranks)  points  of  magical  cold 
damage  per  pulse.  Creatures  of  fire  (efreet,  sala-
mander or elemental) take half damage even if they 
resist. 

\end{effects}
\end{spell}

\begin{spell}[S-4]{Frostbite }

\range{50 feet + 25 / Rank }
\duration{Immediate }
\multiple{200 }
\basechance{20\% }
\resist{Special }
\storage{Investment }
\target{Special }
\begin{effects}
 This  spell  may  either  be  used  to  kill  (5  × 
[Rank])\%  of  all  non-sentient  plants  within  the 
Adepts  range,  or  the  Adept  may  target  and  kill  1 
individual  plant  (+  1  /  Rank).  There  is  no  visible 
effect, but death due to frostbite occurs at the time 
of  casting  and  no  amount  of  non-magical  effort 
will  revive  plants  affected  by  this  spell.  Sentient 
plants  take  1  (+  1  /  Rank)  points  of  damage  (in-
stead  of  dying),  and  may  passively  resist  for  no 
damage. Other plants that are magical or especially 
resistant  to  cold  may  also  be  entitled  to  a  passive 
resistance roll versus the spell’s effects. 

\end{effects}
\end{spell}

\begin{spell}[S-5]{Frozen Doom }

\range{10 feet + 10 / Rank }
\duration{Immediate }
\multiple{500 }
\basechance{5\% }
\resist{Active, Passive }
\storage{Investment, Ward, Magical Trap }
\target{Entity }
\begin{effects}
This spell uses magical cold to freeze solid 
the blood of one target entity, which may be up to 
1 hex (+ 1 / 2 full Ranks) in size, killing it instantly 
if they fail to resist. 

\end{effects}
\end{spell}

\begin{spell}[S-6]{Hibernation }

\range{5 feet }
\duration{Special }
\multiple{250 }
\basechance{20\% }
\resist{Active, Passive }
\storage{Investment, Ward, Magical Trap }
\target{Entity }
\begin{effects}
 The  target  of  this  spell  is  placed  in  sus-
pended  animation  for  up  to  ([Rank  ×  Rank])  days 
as  specified  by  the  Adept.  At  Rank  20  there  is  no 
maximum and the Adept may choose any duration. 
All  bodily  functions  including  ageing,  are  sus-
pended  for  the  duration  of  the  spell  and  the  target 
feels  cool  to  the  touch.  The  target  is  immune  to 
cold  and  suffocation,  and  takes  no  more  damage 
from  existing  injuries  while  this  spell  is  in  effect. 

Additional  injuries  will  still  affect  the  target,  but 
any  damage  that  would  occur  due  to  bleeding, 
poison etc. is ignored. When the spell duration runs 
out,  or  the  spell  is  dispelled,  the  target  awakens 
with  physical  strength  reduced  by  1  /  full  week 
hibernated  and  immediately  begins  to  suffer  from 
any  existing  injuries  and  conditions  (poison,  dis-
ease,  shock,  bleeding  and  suchlike).  Physical 
strength  may  be  regained  at  a  rate  of  1  point  per 
day, and is not reduced below 1. Entities that natu-
rally hibernate suffer -20 to their Magic Resistance 
vs. this spell. 

\end{effects}
\end{spell}

\begin{spell}[S-7]{Ice Bolt }

\range{20 feet + 10 / Rank }
\duration{Immediate }
\multiple{300 }
\basechance{35\% }
\resist{None }
\storage{Investment, Ward, Magical Trap }
\target{Entity, Object }
\begin{effects}
 The  Adept  creates  a  2  foot  long,  2  inch 
diameter non-magical bolt of ice which is projected 
at a target with a Strike chance of 30\% (+ 3 / Rank) 
+  MD,  modified  by  range  as  if  it  were  a  heavy 
crossbow.  The  ice  bolt  strikes  as  an  A  class 
weapon doing [D + 4] (+ 1 / Rank) damage which 
can stun and inflict specific grievous injuries. Once 
the spell is cast the target gets no magic resistance, 
but  the  target’s  defence  is  subtracted  from  the 
chance  to  hit,  since  the  effect  of  the  spell  is  to 
create  a  physical  bolt.  A  double  or  triple  effect 
cannot  affect  damage  but  may  add  +10\%  and 
+20\% to the Strike Chance respectively. 

\end{effects}
\end{spell}

\begin{spell}[S-8]{Ice Construction }

\range{15 feet + 5 / Rank }
\duration{10 minutes + 10 / Rank }
\multiple{225 }
\basechance{15\% }
\resist{None }
\storage{Investment, Ward, Magical Trap }
\target{Volume }
\begin{effects}
 The  Adept  may  conjure  30  (+  30  /  Rank) 
cubic  feet  ice  in  up  to  1  +  Rank  shapes  of  the 
Adept’s choice. The shapes always appear entirely 
within  the  range  of  the  Adept  and  may  not  appear 
above  or  inside  (partially  or  wholly)  any  entity. 
Each  shape  must  appear  on  the  ground  in  a  stable 
fashion (not about to topple over) and must have a 
minimum thickness of  6  inches  in  any  part. When 
the  spell  duration  expires,  the  ice  disappears  (re-
turns to whence it came). 

\end{effects}
\end{spell}

\begin{spell}[S-9]{Iceberg }

\range{10 feet + 10 / Rank }
\duration{ Concentration:  Maximum  30  minutes  + }
30 / Rank 
\multiple{150 }
\basechance{30\% }
\resist{None }
\storage{Investment, Ward, Magical Trap }
\target{Area of Water }
\begin{effects}
The  Adept creates a polyhedral iceberg of 
dimensions 10 (+ 2 / Rank) feet cubed. It may only 
be  successfully  created  in  an  existing  volume  of 
liquid  sufficient  to  hold  it  without  it  touching  the 
bottom. 8/9 of the Iceberg will be submerged. The 
Iceberg will be flat topped and may stably support 
up  to  1  hex  of  entities  (+  1  /  Rank)  (Additional 
entities will cause it to roll in the water). While the 
Adept is in contact with the Iceberg the Adept may 
move  the  Iceberg  at  a  speed  of  5  (+  1  /  Rank  ) 
miles  per  hour.  The  Iceberg  spell  also  has  a calm-
ing effect on the water around it, reducing the size 
of all waves up to Rank feet away by Rank feet. 

\end{effects}
\end{spell}

\begin{spell}[S-10]{Ice Pack }

\range{10 feet +5 / Rank }
\duration{Immediate }
\multiple{150 }
\basechance{15\% }
\resist{None }
\storage{Investment, Potion }
\target{Entity }

Effects:  This  spell  immediately  halts  any  damage 
being  taken  due  to  shock  or  blood  loss  and  the 
target gains 1 / Rank to their chance of recovering 
from stun or unconsciousness (if applicable). It will 
restore  a  (live)  character  on  negative  endurance  to 
zero,  although  any  further  damage  will  start  the 
process of shock and bleeding as usual. In addition, 
if the target is suffering from the adverse effects of 
a  fright  roll  or  similar  emotional  effect, they  get  a 
second roll (in some cases a second resistance roll) 
to  recover  from  (“or  snap  out  of”)  it  with  +1\%  / 
Rank  added  to  their  dice  roll.  The  cause  of  the 
fright  or  shock  may  have  been  magical  or  other-
wise.  This  spell  may  be  recast  on  each  target  as 
often as desired. This spell will not work on regen-
erating entities (including those under the effects of 
a trollskin spell). 

\end{spell}

\begin{spell}[S-11]{Ice Projectiles }

\range{20 feet + 5 / Rank }
\duration{Immediate }
\multiple{300 }
\basechance{30\% }
\resist{Passive }
\storage{Investment, Ward, Magical Trap }
\target{1 Entity + 1 / Rank }
\begin{effects}
 Each  entity  targeted  by  this  spell  must 
resist or suffer [D - 4] (+ 1 / Rank) points of magi-
cal damage due to being pierced by flying A class 
shards of ice (armour does absorb damage but there 
is  no  AG  loss).  If  this  spell  is  doubled  the  adept 
may  not  double  damage  but  may  choose  to  have 
the  ice  projectiles  ignore  armour  instead.  If  this 
spell is tripled, the adept may roll for a possible A 
class  specific  grievous  injury  for  each  target  that 
failed  to  resist,  in  addition  to  ignoring  armour  as 
above  (and  still  doing  fatigue  damage  except  as 
part  of  the  grievous  result),  or  reduce  the  targets’ 
resistances as usual. 

\end{effects}
\end{spell}

\begin{spell}[S-12]{Ray of Cold }

\range{30 feet + 15 / Rank }
\duration{Immediate }
\multiple{300 }
\basechance{30\% }
\resist{Passive }
\storage{Investment, Ward, Magical Trap }
\target{Entity, Object }
\begin{effects}
This spell projects a blast of intense magi-
cal  cold  at  the  target.  The  ray  of  cold  will  impact 
either  on  the  target  or  on  the  first  obstruction 
blocking  the  path  from  the  Adept  to  the  target. 
Anything  struck  by  the  ray  must  either  resist  or 
suffer [D + 1] (+ 1 / Rank) damage (resist for half 
damage). 

\end{effects}
\end{spell}

\begin{spell}[S-13]{Snowball }

\range{10 feet +10 / Rank }
\duration{Concentration: Maximum of 10 minutes }
+ 10 / Rank 
\multiple{200 }
\basechance{20\% }
\resist{None }
\storage{Investment, Ward, Magical Trap }
\target{Adept, Special (chosen each pulse) }
\begin{effects}
 This  spell  causes  the  ground  within  5’  of 
the Adept to be instantly covered  with 5 inches of 
conjured snow (returns to whence it came when the 
spell duration expires). Snowballs then form them-
selves  out  of  the  snow  and  launch themselves  at  a 
target  or  targets  of  the  Adepts  choice.  Up  to  1 
target  (+  1  /  4  full  Ranks)  can  be  pelted  with  a 
flurry  of  snowballs  each  pulse.  A  different  set  of 
targets  may  be  chosen  by  the  Adept  each  pulse. 
Each target must resist or suffer a reduction of -1\% 
(1  /  2  full  Ranks)  to  their  Base  Chance  of  doing 
anything  while  being  snowballed  and  in  addition 
must  make  a  4  ×  WP  roll  (3  ×  WP  if  Adept  is 
above Rank 10 in this spell) to perform any action 
that  involves  concentrating  (e.g.  casting  a  spell). 
Each  target  need  only  resist  this  spell  once.  If  the 
Adept  leaves  the  5’  diameter  circle  of  snow  while 
this  spell  is  in  effect,  the  spell  is  automatically 
dissipated. 

\end{effects}
\end{spell}

\begin{spell}[S-14]{Snowfall }

\range{10 feet + 10 / Rank }
\duration{10 minutes + 5 / Rank }
\multiple{200 }
\basechance{40\% }
\resist{None }
\storage{Investment, Ward }
\target{Area }
\begin{effects}
 This  spell  causes  snow  to  begin  gently 
falling  in  an  area  of  between  5  feet  minimum and 
40 feet (+ 5 / Rank) maximum diameter (chosen by 
the  adept).  The  snow  will  form  in  the  air  at  be-
tween  10  and  40  feet  above  the  ground,  and  will 
gently  float  to  the  ground.  The  area  within  this 
spell, if above zero degrees in temperature, will be 
magically  cooled  to  zero  degrees  for  the  spell 
duration  (nb.  This  will  not  count  as  a  positive 
modifier to Ice mage base chances as it is magical, 
although it may reduce negative modifiers to zero). 
For  each  5  minutes  that  this  spell  is  in  effect,  one 
inch of powder will cover the ground. At the end of 
the spell the snow will remain, but the temperature 
will be restored to normal (and the snow may begin 
melting). Note that if this spell is, for example, cast 
inside a house with a 20 foot ceiling, as the snow is 
formed in the air between 10 and 40 feet up, only a 
third  of  the  snow  will  fall  in  the  room.  The  other 
two  thirds  will  appear  above  the  20  foot  line  and 
will fall on the roof,  with the exception of a small 
amount that appears in the attic. 

\end{effects}
\end{spell}

\begin{spell}[S-15]{Wall of Ice }

\range{20 feet + 10 / Rank }
\duration{10 minutes + 10 / Rank }
\multiple{200 }
\basechance{30\% }
\resist{None }
\storage{Investment, Ward }
\target{Area }
\begin{effects}
This spell conjures a wall of ice that is 10’ 
high,  20’  long  and  2’  thick,  or  a  pillar  15’  high 
with  a  3’  diameter.  The  adept  may  choose  to  in-
crease the height of the wall by 1’, the width by 2’ 
or the thickness by 6", or the height or diameter of 
the pillar by 1’, for each Rank the Adept possesses 
in the spell. The wall may be uniformly curved up 
to a half circle. The wall may not be created on top 
of an entity, and is subject to the usual restrictions 
on  physical  walls.  The  wall  is  translucent  but  not 
transparent. When the spell duration expires the ice 
returns to whence it came. 

\end{effects}
\end{spell}

\begin{spell}[S-16]{Weapon of Cold }

\range{10 feet + 5 / Rank }
\duration{5 minutes + 1 / Rank }
\multiple{250 }
\basechance{30\% }
\resist{None }
\storage{Investment }
\target{Weapon }
\begin{effects}
 The  weapon  over  which  the  spell  is  cast 
becomes infernally cold without harm to either the 
weapon or the user of it. The base chance of hitting 
with the weapon is increased by 1 (+ 1 / Rank) and 
the  damage  done  by  the  weapon  is  increased  by  1 
(+  1  /  3  full  Ranks).  This  amount  of  damage  is 
tripled  if  the  object  of  the  attack  is  a  creature  of 
fire. 

\end{effects}
\end{spell}

\begin{spell}[S-17]{Winter Garden }

\range{20 feet + 10 / Rank }
\duration{2 weeks + 2 / Rank }
\multiple{100 }
\basechance{35\% }
\resist{None }
\storage{Investment, Ward }
\target{Plant }
\begin{effects}
 This  spell  bestows  resistance  to  cold  on 
one  plant  (+  1  /  Rank)  or  25  square  feet  (+  25  / 
Rank) in one patch of the same species of plant of 
the  Adepts  choice.  Plants  with  this  resistance  are 
immune  to the  Frostbite  spell  and  will  thrive  even 
in permafrost and arctic temperatures. 


\end{effects}
\end{spell}

\section{Special Knowledge Rituals}

\begin{ritual}[R-1]{Snow Simulacrum }

\duration{ Concentration:  Maximum  1  hour  +  1  / }
Rank 
\multiple{300 }
\basechance{MA + 3\% / Rank }
\resist{Special }
\begin{effects}
 The  Adept  spends  an  hour  forming  a  hu-
man or animal figure, no larger than 1 hex (+ 1 / 3 
full  Ranks)  in  size,  out  of  snow,  which  must  be 
already present.  

The animated sculpture will have the same charac-
teristics  as  the  sculptured  entity,  except  that  all 
characteristics are  reduced  by  25\%. This  will  gen-
erally  cause  abilities  to  be  lessened  by  a  like 
amount  (e.g.  Flying  speed  is  reduced  by  25\%  (if 
the  entity  can  fly)  due  to  Agility  being  reduced, 
Damage  from  attacks  is  reduced  by  25\%  due  to 
Strength  reduction,  Base  Chance  to  hit  is  reduced 
because of MD reduction, etc.).  

Magical abilities are not copied in the copied form, 
although quasi-magical abilities may be. The simu-
lacrum  will  not  bear  a  close  resemblance  to  any 
other  figure  unless  the  figure  chosen  to  be  dupli-
cated  is  present  during  the  ritual.  In  the  latter  in-
stance  the  figure  being  duplicated  may  choose  to 
actively resist the ritual; otherwise the ritual is non-
resistible.  The  simulacrum has  only  animal  intelli-
gence,  and  the  Adept  may  give  it  simple  instruc-
tions  or  actively  concentrate  to  control  its  move-
ments (requires the Adept to perform pass actions).  

The simulacrum will have vague inclinations relat-
ing  to  its  original  (borrowed)  form,  and  although 
no  longer  comprised  of  snow  (it  takes  on  a  fleshy 
appearance,  or  whatever  is  appropriate  for  the 
original entity) will have an adverse reaction to the 
presence  of  heat  and  flame  and  will  take  an  addi-
tional  2  points  from  heat  and  flame  attacks.  The 
simulacrum normally resembles a generic example 
of  its  borrowed  form  except  that  it  has  a  snowy 
sheen  to  it  and  is  cool  to  the  touch.  Clothes,  pos-
sessions  and  suchlike  are  not  duplicated  by  this 
ritual.  

When the Adepts concentration ceases, the simula-
crum  will  collapse  into  a  pile  of  ordinary  non-
magical snow. 


\end{effects}
\end{ritual}

\begin{ritual}[R-2]{Summoning and Controlling Ice Elemental }

\duration{ Concentration:  Maximum  1  hour  +  1  / }
Rank 
\multiple{250 }
\basechance{MA + 5\% / Rank }
\begin{effects}
 The  Adept  summons  an  Ice  Elemental 
from its home plane (conjectured to be the elemen-
tal  plane  of  Cold),  which  will  appear  within  20 
feet.  This  ritual  takes  2  hours,  and  may  only  be 
performed  in  temperatures  of  0  degrees  or  less.  If 
the  ritual  backfires  the  Elemental  arrives  uncon-
trolled  and  will  attack  the  Adept  (and  others 
nearby);  however,  if  the  ritual  is  successful  the 
Elemental is controlled and must obey the Adept’s 
every  whim.  The  Elemental  remains  until  dis-
pelled,  which  the  Adept  may  do  by  successfully 
casting a Special Knowledge counterspell at it. 1\% 
is added to the base chance of success on this ritual 
for each point of Willpower the Adept has over 15. 
Ice  Elementals  are  similar  to  other  Elementals  in 
that  they  do  not  normally  exist  on  this  plane,  but 
are summoned by Ice magics. They will always be 
hostile to their summoner and will attack if uncon-
trolled.  Ice  Elementals  are  impervious  to  attacks 
with  non-magical  weapons.  Magic  does  affect 
them. Ice Elementals are vulnerable to fire and can 
be  damaged  by  attacks  involving  this  “opposite 
element”.  Ice  Elementals  have  a  combined  endur-
ance and fatigue of 15 (+ 6 / Rank), which must be 
divided in such a way that they fall into the ranges 
indicated  below.  Any  reference  to  Rank  below 
refers to the Adept’s Rank in this ritual. 

Ice Elementals have the following characteristics: 

Habitat Other Planes  

Frequency Very Rare  

Number 1 

Description Ice Elementals appear as lean crystal-
line  humanoids  with  frosty  hair  and  silvery  blue 
eyes.  They  are  half  as  tall,  in  feet,  as  their  Endur-
ance. 

Talents,  Skills  and  Magic  Ice  Elementals  can 
disappear into Ice with only a 10\% chance of being 
detected. They can freeze water within line of sight 
at a rate of one cubic foot per pulse for every point 
of  Physical  strength  available  to  the  Elemental 
(entities  within  the  area  get  a  3  ×  AG  check  to 
avoid  being  caught  and  trapped  while  Ice  forms 
around them), and they can cast a  Ice construction 
and  Wall  of  Ice  at  a  Rank  equal  to  their  sum-
moner’s  Rank  +  4  in  this  ritual.  These  are  talents 
and cost no fatigue. They may expend 2 fatigue to 
fire  an  Ice  bolt  (as  per  the  Spell)  striking  as  an  A 
class weapon with a base chance equal to the Ele-
mental’s  combined  maximum  Endurance  and  Fa-
tigue,  and  doing  [D  +  4]  (+  1  /  Rank)  points  of 
damage.  Since  this  is  a  physical  attack  formed 
from  the  Ice  Elemental,  the  Ice  bolt  remains  after 
firing. 

\end{effects}
\end{ritual}

\begin{ritual}[R-3]{Ritual of Winter }

\duration{ Concentration:  Maximum  1  hour  +  1  / Rank }
\multiple{350 }
\basechance{MA + 3\% / Rank }
\begin{effects}
The Adept may change one or more of the 
three  components  which  make  up  the  weather  by 
performing a ritual dance. The three components of 
weather are: 

1. Precipitation, Degree 

2. Temperature, Gauge 

3. Wind, Force 

The  GM  should  consult  the  weather  table  and 
advise the player of the current level of these three 
components before the Adept starts performing the 
ritual.  The  Adept  may  change  the  current  compo-
nents  by  1  (+  1  /  2  full  Ranks).  All  the  changes 
may  be  in any  direction  on  the  table  with  the  pro-
viso  that  the  Adept  may  never  raise  the  tempera-
ture,  and must  lower  it  by  at  least  one  degree.  All 
weather within 2 miles (+ 2 / Rank) of the Adept is 
affected by the ritual. This ritual cannot backfire. 

PS: 20 + 5 / Rank  WP: 14 - 18 
EN: 5 - 50 
MD: 20 - 25 
FT: 10 - 85 
AG: 15 - 20 
PC: 15 - 20 
MA: None 
PB: 8 - 10 
TMR: 4 + 1 / 5 Ranks 
NA: Skin absorbs 5 DP 
Weapons  Ice  Elementals  strike  their  opponents 
with  open  hands  and  pierce  them  with  their  long 
icy fingers. They can attack twice in the same pulse 
without penalty doing [D + 3] (+ 1 / Rank) damage 
per strike. 

\end{effects}
\end{ritual}

\section{The Element of Ice}

Ice 

Ice weighs 47.2 lb. per cubic foot and is translucent 
(but not transparent). 

Movement  Rates  Running  200  +  10  ×  Sum-
moner’s Rank 

Breaking through Ice 

This applies particularly to the spells  Ice Creation, 
Ice Construction and Wall of  Ice, but may also be 
used  in  natural  settings  (for  example  a  single  inch 
of ice may cover a lake). 

Ice is deemed to have 10 points of fatigue per inch, 
120 points per foot. It takes only half damage from 
being  hit  by  B  class  weapons,  full  damage  from 
axes,  fire  and  water  based  attacks,  and  double 
damage from picks (note that for fire based attacks 
bonuses  against  cold  creatures  apply,  for  those 
spells  that  have  them).  Other  weapons  only  do  a 
single  point.  Endurance  blows  do  double  damage. 
The  exception  is  when  a  blow  by  any  weapon 
exceeds  the  remaining  fatigue  of  the  ice;  in  this 
case  all  damage  is  applied  (the  ice  breaks).  Suc-
cessful  elimination  of  the  ice’s  fatigue  makes  a 
human  sized  hole  in  the  ice, big  enough  for  a  one 
hex  entity  to  go  through.  Multiple  entities  may 
attempt  to  break  through  the  same  area  of  ice, 
within  reason.  Smaller  holes  may  be  made  but  for 
the  purposes  of  simplicity  are  no  quicker  to  make 
(when  digging  far  into  the  ice  at  least  a  human 
sized hole must be made in order to keep working 
anyway). 

Travel on Ice 

Bipeds  must  travel  at  half  TMR  (round  up)  while 
on  ice  or  make  a  2  ×  AG  roll  each  pulse  or  go 
prone.  Quadrupeds  may  make  a  4  ×  AG  roll  to 
travel at full speed, or go prone. Subtract 1 × AG if 
the  ice  is  wet.  Entities  going  prone may  also  slide 
up to  half  their  TMR  along  the  ice  in  the  pulse  in 
which  they  fell,  depending  on  how  much  of  their 
movement was in one direction (GM’s discretion). 

Travel on or through Snow 

Travel through snow usually causes entities to lose 
1/4  of  their  TMR  per  foot  of  powder,  down  to  a 
minimum  TMR  of  1  (unless  the  powder  is  higher 
than  the  entity  is  tall).  In  addition  there  is  an  en-
cumbrance  shift  of  one  column  on  the  encum-
brance  table  for  each  1/4  TMR  slowed.  Note  the 
reference  to  powder  —  if  the  snow  is  denser  than 
freshly  fallen  powder  the  effects  may  be  less. 
These figures are based on a human sized entity; a 
giant (for example) would only suffer the effects of 
4  foot  of  powder  (TMR  reduced  to  1)  when  up  to 
its chest; that is, in about 13 feet for a Stone Giant 
(normally 20 feet tall). Quadrupeds tend to manage 
snowy  conditions  well,  and  should be  treated  as  if 
they were standing erect for the purposes of height 
calculation. Hobbits, and other creatures with large 
feet,  may  have  these  penalties  halved  at  the  GM’s 
discretion.  GM’s  may  also  allow  items  such  as 
snow  shoes;  a  suggestion  is  a  straight  halving  of 
TMR  but  the  entity  does  not  sink  into  the  snow. 
Note that such items are not known or common in 
warmer climes! 

\end{Chapter}
