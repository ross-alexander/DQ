\begin{Skill}[1.1]{beastmaster}{Beast Master}

A loyal animal or monster is likely to serve its master far better
than adventuring comrades ever will.  A beast master is one who trains
these creatures to obedience.  They take a wild animal and, from an
adversarial relationship, develop a rapport with it.  They train young
animals from birth until they heed their every command.  A beast
master will, in almost all cases, become very fond of animals.  They
will defend them against wanton cruelty and slaughter, and will treat
their personal charges as family.

A beast master will encounter three kinds of animals: the easily
domesticated (such as the horse), the naturally wild (such as the
pegasus) and the intelligent or rebellious creature (such as the
unicorn).  The latter can never be steadfastly loyal to the beast
master; such creatures always have at least a subconscious desire to
escape.  A beast master can be a slaver if they specialise in training
humanoids.

\section{Restrictions}

A beast master must have at least 15 Willpower. 

A beast master will normally use their skill to train or domesticate
animals for their own use. Animals which spend their lives with a
beast master and are trained by them will be loyal to their master and
serve and protect them as much as possible.  If necessary, an animal
can be trained to temporarily serve another master (if one week of
mutual training is undergone), but the animal will always obey the
original master before any new one.

If a beast master’s Rank is 5 or greater, they may train animals for
other people.  The being who is acquiring the trained creature must
spend (12 - Rank) weeks before it will accept them as its new master,
during which the beast master must be present at least one day per
week. The creature will heed the beast master’s commands before those
of its new owner for as many years as the beast master’s Rank at the
time the creature’s ownership is transferred.

A beast master of any Rank may domesticate, rather than train,
animals.  Such animals can be commanded by any other person, but will
tend to wander off or revert to their wild state if not supervised,
tied up, or stabled.  Note that horses and dogs, the most common
domestic animals, are governed by this rule.

\section{Benefits}

\subsubsection{A beast master acquires the ability to train one type of animal or
monster at Ranks 0, 5 and 10.}

A beast master may acquire the ability to train additional types of
creatures without increasing in rank by spending 5,000 Experience
Points and 4 weeks training per type. These costs are discounted by
25\% if the beast master has reached rank 8, or by 50\% if they have
reached rank 10.

A type consists of all creatures listed within one subsection of the
beastiary (e.g.  avians).  A beast master may choose, instead, all
creatures subsumed under a single animal family (e.g. canines).

A beast master must spend (12 - Rank) months to train an animal or
monster, or a like number of weeks to domesticate one.

\begin{dqtblr}{colspec={Xl}}
Creature to be trained is			& Time \\
Easily domesticated				& × 0.5 \\
Naturally wild					& × 1.0 \\
Intelligent or rebellious			& × 3.0 \\
Raised by beast master from adolescence 	& × 0.5 \\
Domesticated by another beast master		& × 1.0 \\
Caught in wilderness				& × 1.5 \\
\end{dqtblr}

The unmodified number of months required is multiplied by all
applicable modifiers. The time to train a monster or animal is always
dependent on the beast master’s Rank when they begin the process.  Any
increases in Rank during the training or domestication period have no
effect on the time required.

\subsection{Loyalty Checks}

A trained animal or monster must make a loyalty check whenever it
recognises that its master is endangering it, or whenever its master
commands an action that runs counter to its instincts.  Whenever a
loyalty check is required, the GM rolls percentile dice.  The base
chance is 2 × beast master’s WP + (4 × Rank if the creature is
intelligent or rebellious, 6 × Rank if the creature is naturally wild,
and 8 × Rank if the creature is easily domesticated).  If the owner is
not a beast master, use their WP and the Rank of the beast master when
they trained the creature. If the roll is less than or equal to this
success percentage, the trained creature will do as its master
commands.  If the roll is greater than the success percentage, the
creature’s reactions will range from balking to fleeing to turning
on its master, as the roll increases (GM’s discretion).

A domesticated creature must make a loyalty check if the circumstances
described above arise.  The GM rolls D100. If the resulting number is
less than or equal to current master’s WP + beast master’s Rank, the
domesticated creature will perform the action.  If the roll is greater
than the success percentage, but less than or equal to twice that
percentage, the creature will balk. If the roll is greater than two
times the success percentage, but less than three times that
percentage, the creature will take flight. If the roll is greater than
three times the success percentage, the creature will turn on its
master.  A roll of 100 always indicates that a domesticated creature
turns on its master. A roll of 96 through 99 indicates that the
creature takes flight if the success percentage is 47 or greater.

A beast master who intimidates their animals adds one to their Rank
when calculating training or domestication time, but the GM adds 10 to
any loyalty check dice-roll for one of their animals.

\subsubsection{A beast master may train or domesticate as many creatures as their
Rank at one time.}

All creatures being trained or domesticated concurrently must be of
the same type.

\section{Cost}

A beast master must pay 100 Silver Pennies per creature trained and 25
Silver Pennies per creature domesticated.

They may halve the cost for upkeep of creatures if they build a
stable.  A horse-sized stable costs (500 + 150 × Stalls) Silver
Pennies to construct, and costs (Stalls) Silver Pennies for repairs
after the first year.

\end{Skill}

