\begin{Chapter}{Character Generation}

There are six sections in Character Generation: 
\begin{description}
\item[\ref{characteristic points}] Characteristic Points
\item[\ref{race}] Race
\item[\ref{description}] Description
\item[\ref{aspects}]  Aspects
\item[\ref{heritage}] Heritage
\item[\ref{starting}] Starting Abilities \& Possessions
\end{description}

Sections 1.1 -- 1.5 may be done in any order.  Each section is
designed so that a player may choose from a range of options or
randomly generate their character. Section 1.6 should be done last.

\section{Characteristic Points}
\label{characteristic points}

A character has six primary statistics which are generated by
allocating points from a total, and four secondary statistics which
are either derived from the primary statistics or are generated
randomly.  The higher the number, the better the characteristic.

\subsection{Generating Characteristic Points}

The player may choose to allocate the primary statistics from a total
of 90 points or may roll 2D10 once against the following table.  If
they choose to roll the result must stand.

\begin{dqtblr}{colspec={ccX},hline{1-2}={0.4mm}}
Die Roll  & Points Total \\
2	& 81 \\
3	& 82 \\ 
4	& 83 \\
5	& 84 \\
6	& 85 \\
7	& 86 \\
8	& 87 \\
9	& 88 \\
10	& 89 \\
11	& 90 & (default choice) \\
12	& 91 \\
13	& 92 \\
14	& 93 \\
15	& 94 \\
16	& 95 \\
17	& 96 \\
18	& 97 \\
19	& 98 \\
20	& 99 \\
\end{dqtblr}

\vskip 0pt plus 10mm

\subsection{Assigning Characteristic Points}

This total of points needs to be spent on the following
characteristics: Physical Strength, Manual Dexterity, Agility, Magical
Aptitude, Willpower \& Endurance.  These characteristics may change
during the game, and may be raised up to 5 points through training,
though not past the character’s racial maximum.

The human range for each of these characteristics is 5 -- 25; this
range is adjusted for non-humans (see the Characteristic Modifier
tables for the non-human races).  These ranges represent the minimum
and maximum capabilities of the races. The player should assign the
points and then make any adjustment for race.

Prior to assigning the characteristic points, the player should give
some thought to what kind of character they wish to have and what
weapons, spells, and/or skills are desired for the newly created
individual. Some weapons require a great deal of Physical Strength or
Manual Dexterity, and the player should be sure to assign enough
points in those areas to use the weapons of their choice.  All magical
colleges require a minimum Magic Aptitude to join and the player
should be aware of these restrictions.  Most skills do not have any
special requirements, but many give bonuses for exceeding a minimum
value in certain characteristics.

When the player has chosen the values for the character, they must
record them on a Character Sheet.  The total value of the six primary
characteristics (before racial modifiers) must equal the amount
received in the Generating Characteristic Points section; thus, a
player cannot “save” Characteristic Points and assign them to
characteristicsat a later date.  The value of each of the six primary
characteristics must be recorded before any secondary characteristics
are generated.

\subsection{Generating Secondary Characteristics}

Fatigue, Physical Beauty, Perception and Tactical Movement Rate are
secondary characteristics.  They may be modified if the character is
non-human (see the Characteristic Modifier tables for the non-human
races).

\subsubsection{Fatigue}

The value of a character’s Fatigue is a direct function of their
Endurance.  The player enters the Fatigue value corresponding to the
character’s Endurance value after their Endurance has been modified
for race.

\begin{dqtblr}{colspec={XX},hline{1-2}={0.4mm}}
Endurance	& Fatigue \\
3 or 4		& 16 \\
5 to 7		& 17 \\
8 to 10		& 18 \\
11 to 13	& 19 \\
14 to 16	& 20 \\
17 to 19	& 21 \\
20 to 22	& 22 \\
23 to 25	& 23 \\
26 to 27	& 24 \\
\end{dqtblr}

Endurance and Fatigue values in bold type can be achieved only by
members of certain non-human races.

From this point on, a change in a character’s Endurance value will not
affect their Fatigue value and vice-versa.  Fatigue may be raised by
up to 5 points, though not past the character’s racial maximum.

\subsection{Physical Beauty}

The value of the Physical Beauty characteristic is generated randomly
by rolling 4D5 + 3. This characteristic can never be increased by
training.

\subsection{Perception}

A character’s perception value begins at 5.  This may be trained up to
racial maximum.

\subsection{Tactical Movement Rate}

A character’s Tactical Movement Rate (TMR) is a direct function of
their Agility.  It is based on the character’s Agility value and is
recalculated when Agility is modified by encumbrance and armour
penalties; see the TMR table (\S\ref{table:tmr}) for values.

\vskip 0pt plus 10mm

\section{Race}
\label{race}

A player must choose the race of their character.

The majority of people in Alusia are human, but the player may choose
one of the common non- human races: dwarf, elf, halfling, or orc.

If the player wishes their character to be a giant or shapechanger
they must roll D100.  They may roll once per race and if the roll is
lower than the race chance \% they must take that race. If they fail
then the character must be of one of the common races.  If the player
is attempting to be a shapechanger they must decide what type of
shapechanger they want prior to rolling (i.e. wolf, tiger, bear or
boar).

\begin{dqtblr}{colspec={Xc},hline{1-2}={0.4mm}}
Race		& Chance (\%) \\
Hill Giant	& 06 \\
Shapechanger	& 04 \\
\end{dqtblr}

A player may wish to play one of the very rare sentient races.  To do
so they must get the agreement of both the generating GM and a member
of the character tribunal.  They will decide which of the common races
has the appropriate racial modifiers.  For example Erelheine
characters are generated using the Elf option.

Humans learn faster than non-humans. Learning is represented in game
by spending Experience Points (EP).  Divide any experience points a
character gains by the “racial modifier” and then spend the result
normally.

\begin{dqtblr}{colspec={Xc},hline{1-2}={0.4mm}}
Race		& Modifier \\
Dwarf		& 1.1 \\
Elf		& 1.2 \\
Halfling	& 1.1 \\
Hill Giant	& 1.5 \\
Human		& 1.0 \\
Orc		& 1.1 \\
Shapechanger	& 1.4 \\
\end{dqtblr}

For every 25,000 Experience Points (EP) the character has spent
towards the 'racial modifier' it is lowered by .1 (but not below 1.0)
or after 20 adventures (when PC reaches racial max), the EM becomes 1
(whichever happens first).  E.g. once a giant has lost 25,000 EP to
their race, their 'racial modifier' is lowered to 1.4.  Once they
spend an additional 25,000 EP on their new 'racial modifier' of 1.4 it
would become 1.3.

\begin{dqtblr}{colspec={lXX},hline{1-2}={0.4mm}}
RM	& Amt earned that cost 25k	& Amt spent that cost 25k  \\
1.5	& 75,000			& 50,000 \\
1.4	& 87,500			& 62,500 \\
1.3	& 108,333			& 83,333 \\
1.2	& 150,000			& 125,000 \\
1.1	& 275,000			& 250,000 \\
\end{dqtblr}

Each race has a description of a stereotypical member of the race and
any special abilities and characteristic modifiers that apply to a
character of that race.

\subsection{Dwarf}

A dwarf is a short, bearded humanoid, usually taciturn who frequents
mountainous areas.

\begin{description}
\item[Description] Pride and attention to detail are important to
  dwarves.  They form strong community ties, and are distrustful of
  strangers, especially those of other races. Their strongest
  antipathies are towards orcs and elves.  Although dwarves are greedy
  by nature, they are essentially honest and stand by their word.
  Dwarves covet precious stones and metals, and appreciate fine,
  detailed workmanship.  Dwarven warriors favour the axe as weapon.
\end{description}

\subsubsection{Special Abilities}

\begin{enumerate}

\item Dwarves’ close vision is exceptionally sharp, but many have poor
  distance vision.  They can see in the dark as a human does at dusk.
  Their effective range of vision in the dark is 50 feet under the
  open sky, 100 feet inside manmade structures, and 150 feet inside
  caves and tunnels.

\item Dwarves can assess the value of and deal in gems and metals as
  if they are a Merchant of Rank 5.  If a dwarf character progresses
  in the Merchant skill, their ability to assess the value of gems and
  metals is five greater than their current Rank, to a maximum of ten.

\item If a dwarf character is a Ranger specialising in mountains or
  caverns, they pay half the EP cost necessary to advance ranks.

\item A dwarf’s capacity for alcohol is twice that of a human.

\end{enumerate}

\smallskip

\begin{dqtblr}{colspec={Xr},hline{1-2}={0.4mm}}
Characteristic			& Modifier \\
Physical Strength		& \pl 2 \\
Agility				& \mi 2 \\
Endurance			& \pl 2 \\
Magical Aptitude		& \mi 2 \\
Willpower			& \pl 2 \\
Perception			& \pl 1 \\
Physical Beauty			& \mi 2 \\
Tactical Movement Rate		& \mi 1 \\
Starting Age:			& 20 + \\
Average Life Span (years):	& 125 -- 150 \\
\end{dqtblr}

\subsection{Elf}

An elf is a slim agile humanoid, who frequents wooded areas.

\begin{description}
\item[Description] Elves are virtually immortal and generally take the
  long term view.  They are insular, indifferent to others and tend to
  be traditional.  Elves are great respecters of nature and learning.
  Their Elders are repositories of great wisdom while elvish youth are
  enthusiastic merry makers.  Elven warriors favour bow weapons and
  disdain metal armour.  Members of other races generally find elves
  attractive.
\end{description}

\subsubsection{Special Abilities}

\begin{enumerate}
\item Elves have superior vision especially over long distances or in
  poor lighting.  An elf can see in the dark as a human does on a
  cloudy day. Their effective range of vision in the dark is 150 feet
  under the open sky, and 75 feet elsewhere.

\item If an elf character is a ranger specialising in woods, they pay
  one-half the EP to advance ranks.

\item An elf receives a racial Talent which functions in all respects
  as the Witchcraft Witchsight Talent.

\item An elf makes little or no noise while walking and adds 10\% to
  their chance to perform any activity requiring stealth.

\item If an elf character takes the healer skill, the elf pays
  three-quarters the EP to advance ranks, though they cannot resurrect
  the dead.

\item An elf is impervious to the special abilities of the lesser
  undead.

\item If an elf character takes the courtier skill, the elf pays
  one-half the EP to advance ranks.
\end{enumerate}

\smallskip

\begin{dqtblr}{colspec={Xr},hline{1-2}={0.4mm}}
Characteristic			& Modifier \\
Physical Strength		& \mi 1 \\
Agility				& \pl 1 \\
Endurance			& \mi 1 \\ 
Magical Aptitude		& \pl 1 \\
Willpower			& \pl 1 \\
Perception			& \pl 1 \\
Physical Beauty			& \pl 2 \\
Starting Age:			& 30 -- 300 + \\
Average Life Spa (years):	& ~ 10,000 \\
\end{dqtblr}

\subsection{Halfling}

A halfling is a short, cheerful humanoid, who will be an active
participant in village life.

\begin{description}
\item[Description] Halflings appreciate the good life more than most;
  a successful halfling will arrange a schedule of much sleep, good
  food, and relaxed study or conversation.  Halflings are shy around
  other races, preferring to merge into the background.  Amongst
  themselves they are a friendly folk who form into small communities
  where everyone knows everyone else’s business.  While Halflings take
  their social responsibilities seriously, they are renowned for their
  practical jokes and light fingers.  Halflings are noted for their
  tough, hairy feet and usually go barefoot.  Halflings avoid the
  rigours of military life but when forced to defend themselves they
  favour small weapons.
\end{description}

\subsubsection{Special Abilities}

\begin{enumerate}
\item A halfling has infravision, which allows them to see faint red
  shapes where living beings are located in the dark. Their range of
  vision is 100 feet.

\item A halfling adds 20\% to their chance to perform any activity
  requiring stealth.

\item If a halfling takes the thief skill, they pay half the EP cost
  to advance ranks.

\item A halfling may drop jewellery down active volcanoes without
  anyone thinking the worse of them.
\end{enumerate}

\smallskip

\begin{dqtblr}{colspec={Xr},hline{1-2}={0.4mm}}
Characteristic			& Modifier \\
Physical Strength		& \mi 3 \\
Manual Dexterity		& \pl 3 \\
Agility				& \pl 1 \\
Endurance			& \mi 2 \\
Magical Aptitude		& \mi 1 \\
Willpower			& \pl 1 \\
Physical Beauty			& \mi 1 \\
Starting Age:			& 21 + \\
Average Life Span (years)	& 80 -- 90 \\
\end{dqtblr}

\subsection{Hill Giant}

A hill giant is a huge, coarse featured humanoid, who has no patience
for laborious learning.

\begin{description}
\item[Description] Giants are lusty types, preferring nothing better
  than to go through life brawling, drinking, and wenching.  They tend
  to gather together in a clan arrangement, building huge halls (or
  steadings) in out-of-the-way locations.  They are not overly
  intelligent, and resent humans and elves particularly.  Giants enjoy
  riddling and bartering.  Giant warriors favour simple weapons scaled
  to their size.
\end{description}

\subsubsection{Special Abilities}

\begin{enumerate}
\item A giant has infravision, which allows them to see faint red
  shapes where living beings are located in the dark. Their range of
  vision is 250 feet.

\item A giant’s magic resistance is increased by 10\%. 

\item Whenever a giant attempts minor magic, the GM should increase
  the difficulty factor by one, making it easier.

\item Giants may use the giant weapons listed in the Weapons Table
  (\S\ref{table:weapons}).

\end{enumerate}

\smallskip

\begin{dqtblr}{colspec={Xr},hline{1-2}={0.4mm}}
Characteristic			& Modifier \\
Physical Strength		& \pl 7 \\
Manual Dexterity		& \mi 1 \\
Agility				& \mi 2 \\
Endurance			& \pl 8 \\
Magical Aptitude		& \mi 1 \\
Willpower			& \mi 1 \\
Fatigue				& \pl 1 \\
Physical Beauty			& \mi 1 \\
Tactical Movement Rate		& \pl 3 \\
Natural Armour			& \pl 1 \\
Starting Age:			& 26 \pl \\
Average Life Span (years)	& 500 \\
\end{dqtblr}

\subsection{Human}

Humans are by far the most common race on Alusia, frequenting most
areas and climes.

\begin{description}
\item[Description] Humans have a great diversity of cultures,
  languages and sub-racial traits, such as hair and eye colour or skin
  tone.  Human behaviour is an odd mix.  They can be superstitious and
  distrustful of the unknown, but they are also insatiably curious and
  look for new knowledge.  Many also seek personal fame and fortune as
  most human social structures are less rigid than those of non-humans
  and a person’s birth need not permanently define their place in
  society.  This odd combination of attributes has led them to become
  great explorers and sailors, and they will venture boldly into
  unexplored areas in search of knowledge and wealth.  Humans build
  great cities and are far more welcoming of other races than most.
  Outside of their own culture they are social chameleons, adept at
  adapting their behaviour to match local customs.
\end{description}

\subsubsection{Special Abilities}

\begin{enumerate}

\item Humans can ingratiate themselves with strangers more readily
  than other races.  A human character has +10 to any reaction roll in
  an encounter with sentient creatures.

\end{enumerate}

\smallskip

\begin{dqtblr}{colspec={Xr},hline{1-2}={0.4mm}}
Characteristic				& Modifier \\
Starting Age				& 16 \pl \\
Average Life Span (varies widely with wealth and culture) & 40 -- 90 \\
\end{dqtblr}

\subsection{Orc}

An Orc is a stoop-shouldered, surly humanoid and a pack member by
nature.

\begin{description}
\item[Description] Orcs are a cruel, violent folk, liking nothing
  better than to loot and pillage.  Individuals test themselves
  against their peers, bullying anything weaker but cowering away from
  anything stronger.  A strong individual will form a pack around
  them, and the pack leader’s word is law.  Orcs enjoy the sensual
  pleasures of life, and reduce their already short life spans through
  hard living.  They have a robust digestion and will eat foods that
  others turn their nose up at.  Orc warriors favour the great axe and
  glaive.  Orcs are considered unattractive by other humanoid races.
\end{description}

\subsubsection{Special Abilities}

\begin{enumerate}
  
\item An orc’s eyes are highly light-sensitive.  They must decrease
  their chance of hitting a target with Ranged Combat by 10\% in
  daylight.

\item An orc has infravision, which allows them to see faint red
  shapes where living beings are located in the dark. Their range of
  vision is 150 feet.

\item Orcs are either back-stabbing scum or brutal bully-boys.  An orc
  may take one of either Assassin Skill or Warrior Skill and pay
  three-quarters the EP to advance in Ranks.

\item An orc’s seed is highly fertile.  The orc and hybrid orc
  population increase mitigates against the high orc fatality rate.

\end{enumerate}

\smallskip

\begin{dqtblr}{colspec={Xr},hline{1-2}={0.4mm}}
Characteristic 			& Modifier \\
Physical Strength		& \pl 2 \\
Endurance			& \pl 1 \\
Magical Aptitude		& \mi 2 \\
Willpower			& \mi 2 \\
Fatigue				& \pl 2 \\
Physical Beauty			& \mi 4 \\
Natural Armour			& \pl 1 \\
Starting Age			& 12 \pl \\
Average Life Span (years)	& 40 -- 45 \\
\end{dqtblr}

\subsection{Shapechanger}


Shapechangers are a hidden race amongst humans, with the ability to
change into the form of a particular animal.

\begin{description}
\item[Description] Shapechangers are identical in appearance to humans
  when not in animal form.  They are somewhat bestial in nature,
  adopting traits one might expect from an anthropomorphised wolf,
  tiger, bear or boar.  There exists a love/hate relationship between
  humans and shapechangers: shapechangers possess some degree of
  animal magnetism, but, if discovered, can expect severe treatment at
  the hands of humans.  Shapechangers are, on the whole, bitter
  towards humans, and are not above using humans to their advantage.
  There are very few ways to tell a shapechanger from a human (e.g.
  they will be discomforted by wolfbane) and these vary by
  shapechanger type.  Shapechangers are a ruthless lot.
\end{description}

\subsubsection{Special Abilities}

\begin{enumerate}

\item A shapechanger can change from human to animal form (or
  vice-versa) in 10 seconds during daytime and 5 seconds during the
  night-time.

\item A shapechanger possesses a dual nature.  While in animal form,
  human inhibitions will be muted; while in human form, animal
  instincts will be dulled.

\item A shapechanger cannot be harmed while in animal form, unless
  struck by a silvered weapon, magic or by a being with a Physical
  Strength greater than 25.  Five Damage Points are automatically
  absorbed in the latter case.

\item A shapechanger will regenerate 1 Endurance Point every 60
  seconds while in animal form.

\item The player must devise a set of characteristics for their animal
  form. Take the difference between the average for each
  characteristic in animal and human form, and modify the human
  characteristics appropriately.

\item A shapechanger is automatically lunar aspected.

\item A shapechanger can remain in animal form for a quarter of the
  night times the quarters of the moon showing (i.e. at full moon they
  may remain in animal form all night).  During the day a shapechanger
  can remain in animal form for one hour times the quarter of the
  moon.  A shapechanger can make one set of transformations times the
  quarter of the of the moon per day (i.e. dawn to next dawn).

\item If a shapechanger is in animal form during the day, there is a
  1\% cumulative chance for each 5 minutes they remain in animal form
  that they will never be able to change back into human form.
  Similarly, if the shapechanger exceeds the time limits given above,
  there is a 1\% cumulative chance (per 5 minutes) of their not being
  able to return to human form.

\item A shapechanger will be inconvenienced by those wards which can
  be used against were-creatures.

\item A shapechanger’s magic resistance is increased by 5\%.

\item If a shapechanger takes the courtier skill they pay
  three-quarters the Experience Points necessary to advance ranks.

\end{enumerate}

\smallskip

\begin{dqtblr}{colspec={Xr},hline{1-2}={0.4mm}}
Characteristic			& Modifier \\
Physical Beauty			& \pl 1 \\
Starting Age			& 16 \pl  \\
Average Life Span (years)	& 55 -- 65 \\
\end{dqtblr}

A separate set of characteristics must be generated for the animal
form (see Ability 5 above).

\section{Description}
\label{description}

This section covers height, weight, gender, primary hand, and general
description.

\subsection{Height and Weight}

A player should choose their character’s height and weight.  The
character’s height and weight should be chosen according to the
player’s idea of the character, with due regard to the character’s
primary characteristics, race and background.

The following charts give a range of heights and weights within which
90\% of adventurers fall, and the average values within that range.
Please modify your chosen height and weight according to gender and
racial adjustments as below.

\subsubsection{Normal Base}

\begin{dqtblr}{colspec={XXX},hline{1-2}={1.0pt}}
Height		& Weight	& Range \\
5'3''		& 130		& 100 -- 170 \\
5’6"		& 140		& 110 -- 185 \\
5’9"		& 150		& 120 -- 200 \\
6’0"		& 165		& 130 -- 220 \\
6’3"		& 180		& 145 -- 240 \\
\end{dqtblr}

\bigskip

\begin{dqtblr}{colspec={Xll},hline{1-2}={1.0pt}}
Adjustments	& Height (in)	& Weight \\
Human Male	& \pl 0''	& 100\% \\
Human Female	& \mi 4''	& 80\% \\
Orc Male	& \mi 4''	& 110\% \\
Orc Female	& \mi 6''	& 100\% \\
Elf Male	& \pl 5''	& 80\% \\
Elf Female	& \pl 2''	& 65\% \\
\end{dqtblr}

\bigskip

\subsubsection{Short Folk Base}

\begin{dqtblr}{colspec={XXX},hline{1-2}={0.4mm}}
Height		& Weight	& Range \\
3’9"		& 85		& 65 -- 110 \\
4’0"		& 95		& 75 -- 125 \\
4’3"		& 105		& 85 -- 140 \\
4’6"		& 115		& 95 -- 155 \\
4’9"		& 125		& 105 -- 170 \\
\end{dqtblr}

\bigskip

\begin{dqtblr}{colspec={Xll},hline{1-2}={0.4mm}}
Adjustments		& Height	& Weight \\
Dwarf Male		& \pl 0"	& 100\% \\
Dwarf Female		& \mi 2"	& 90\% \\
Halfing Male		& \mi 12"	& 65\% \\
Halfing Female		& \mi 3"	& 60\% \\
\end{dqtblr}

\bigskip

\subsubsection{Hill Giants Base}

\begin{dqtblr}{colspec={XXX},hline{1-2}={0.4mm}}
Height		& Weight	& Range \\
8’4"		& 370		& 295 -- 490 \\
8’8"		& 420		& 335 -- 555 \\
9’0"		& 470		& 375 -- 625 \\
9’4"		& 525		& 420 -- 700 \\
9’8"		& 580		& 465 -- 780 \\
\end{dqtblr}

\bigskip

\begin{dqtblr}{colspec={Xll},hline{1-2}={0.4mm}}
Adjustments	& Height	& Weight \\
Giant Male	& \pl 0"	& 100\% \\
Giant Female	& \mi 4"	& 90\% \\
\end{dqtblr}

\bigskip

\subsection{Gender}

A player may choose whether their character is male or female.  It is
recommended the character be the same gender as the player, as playing
the opposite gender convincingly is difficult.

Optionally, some characteristics may be adjusted for a female
character.  This would also modify her appropriate racial maximums.

\subsubsection{Female Characteristic Modifier}

\begin{dqtblr}{colspec={Xr},hline{1-2}={0.4mm}}
Characteristic		& Modifier \\
Physical Strength	& \mi 2 \\
Manual Dexterity	& \pl 1 \\
Endurance		& \pl 1 \\
\end{dqtblr}

\subsection{Primary Hand}

A player must determine whether their character’s Primary Hand is
their right or their left.  This determination affects which hand a
weapon is held during combat, and any penalties assigned for attacking
with a weapon in a secondary hand.

They may choose either right or left, or roll randomly. If they choose
to roll, the result must stand.

The player rolls D5 and D10.  If the D10 result is greater, the
character’s right hand is primary. If the D5 result is higher, their
left hand is primary. If the two results are equal, the character is
ambidextrous.

\subsection{Description}

The player will sometimes need to describe their character and should
therefore think about the character’s physical appearance based on the
generated characteristics.  They should choose hair, eye and skin
colour (based on race and family background).

\section{Aspects}
\label{aspects}

The timing of a character’s birth orients them towards one of several
astrological influences, or aspects.  A character will benefit during
the time their aspect is powerful, and will suffer when the opposite
aspect is powerful.

The times of high noon and midnight are extremely important when
applying the effects of aspects.  The GM should allow characters to
perform actions at precisely those instants, though the passage of
time must be properly monitored.

\subsection{Generating an Aspect}

The player may choose an Aspect as if they had rolled any number up to
80, or roll D100 once against the following table.  If they choose to
roll on the table, any roll over 80 may be re-rolled.

If the character is joining one of the elemental colleges the player
may choose any aspect between 1 and 80 that is neutral to their
college, or they may roll.

\begin{inline}
\begin{dqtblr}{colspec={m{4em}X},hline{Z}={black,0.0pt},hline{1-2}={0.4mm}}
Die		& Aspect  \\
01--05	& Winter Air  \\
06--10	& Winter Water  \\
11--15	& Winter Earth  \\
16--20	& Winter Fire  \\
\end{dqtblr}

\begin{dqtblr}{colspec={m{4em}X},hline{Z}={black,0.0pt}}
21--25	& Spring Air  \\
26--30	& Spring Water  \\
31--35	& Spring Earth  \\
36--40	& Spring Fire  \\
\end{dqtblr}

\begin{dqtblr}{colspec={m{4em}X},hline{Z}={black,0.0pt}}
41--45	& Summer Air  \\
46--50	& Summer Water  \\
51--55	& Summer Earth  \\
56--60	& Summer Fire  \\
\end{dqtblr}

\begin{dqtblr}{colspec={m{4em}X},hline{Z}={black,0.0pt}}
61--65	& Autumn Air  \\
66--70	& Autumn Water  \\
71--75	& Autumn Earth  \\
76--80	& Autumn Fire  \\
\end{dqtblr}

\begin{dqtblr}{colspec={m{4em}X}}
81--85	& Solar  \\
86--90	& Lunar  \\
91--95	& Life  \\
96--00	& Death  \\
\end{dqtblr}
\end{inline}

\subsection{Effects of Aspects}

Apart from elemental aspects, all modifiers apply to percentile rolls,
not base chances.

\subsection{Elemental Aspects}

Characters gain a bonus of 1\% on the Base Chance of performing any
magic of the same College as their elemental aspect, and a penalty of
-1\% on the opposed College.  Air opposes Earth and Fire opposes
Water.  Ice and Celestial magic is not affected.

\subsection{Seasonal Aspects}

A character is affected by their seasonal aspect during their aspect’s
season and the opposite season.  The following table lists the
seasonal aspect effects and when they apply.

\begin{dqtblr}{colspec={Xr},hline{1-2}={0.4mm}}
Time							& Effect \\
Midnight, Aspect’s Season				& \mi 10 \\
Midnight, Equinox or Solstice of Aspect’s Season	& \mi 25 \\
Midnight, Opposite Season				& \pl 10 \\
Midnight, Equinox or Solstice of Opposite Season	& \pl 25 \\
\end{dqtblr}

The effect is applied for 30 seconds before and after midnight.

\subsection{Solar and Lunar Aspects}

A character of solar or lunar aspect is affected by their aspect at
high noon and midnight. The following table lists the Solar aspect
effects, and when to apply them.

\begin{dqtblr}{colspec={Xr},hline{1-2}={0.4mm}}
Time				& Effect \\
Noon				& \mi 5 \\
Midnight			& \pl 5 \\
Noon, Summer Solstice		& \mi 25 \\
Midnight, Winter Solstice	& \pl 25 \\
\end{dqtblr}

Lunar aspected characters gain opposite bonuses and penalties for the
same times.  The effect is applied for 10 seconds before and after
high noon or midnight. If the sky is cloudy, the effect may be reduced
to a minimum of +/1 and 5.

\subsection{Life and Death Aspects}

Life and Death aspected people are affected by the creation and
destruction of life force.

The following table lists the Death aspect effects, and when to apply
them.

\begin{dqtblr}{colspec={Xlll},hline{1-2}={0.4mm}}
Event				& Range		& Aspect	& Effect \\
Birth of mammal  		& 100’		& \mi 5		& Death \\
Birth of humanoid  		& 250’		& \pl 10	& Death \\
Birth of close relative†	& 500’		& \pl 25	& Death \\
Death of mammal  		& 50’		& \mi 5		& Death \\
Death of humanoid  		& 125’		& \mi 10	& Death \\
Death of close relative†	& 250’		& \mi 25	& Death \\
\end{dqtblr}

†A close relative is no more distant than a second cousin.

Life aspected characters gain opposite bonuses and penalties for the
same times.  Deaths are non-cumulative (only one can be in effect at a
given time), though births are cumulative.  A stillbirth does not
affect a life or death aspected character. A resurrection is treated
as a birth.

A death event is applied for as many seconds as the effect range in
feet.  A life event is applied for 3 times as long.

A female life aspected character will suffer no pain after giving
birth, and will be as healthy and active as she was before she became
pregnant.

\subsection{Light and Dark Aspects}

All living creatures have an additional celestial Light or Dark
Aspect.  This is fully explained in an addendum to the College of
Celestial Magics (\S\ref{celestial:aspect}).


\section{Heritage}
\label{heritage}

This section is relevant to humans, primarily from the Western Kingdom
and Cazarla, and should be adapted for other races or regions.

\subsection{Social Status}

The “social status” is that of the character’s parent(s), usually the
father. The table does not represent the population, merely the
proportion of backgrounds from which accepted Adventurer’s Guild
applicants originally come. Most social classes are present in a
variety of environments (city, town/village, rural, court,
castle/stronghold, maritime).  A player may choose any social category
in the 01--80 range for the character’s background or roll; however,
any such dice roll must be accepted.  In general, the higher the
number rolled, the higher the social status within each band. A roll
of 40--90 optionally may indicate a respectably retired ex-adventurer.

\begin{dqtblr}{colspec={m{3em}X},hline{1-2}={0.4mm}}
Die	& Social Status \\
01--14	& Trash / Criminal \\
15--20	& Bonded \\
21--29	& Skilled retainer \\
30--44	& Goodman \\
45--54	& Master \\
55--70	& Military \\
71--84	& Gentry \\
85--94	& Lesser Noble \\
95--98	& Merchant-prince \\ 
99--00	& Greater Noble \\
\end{dqtblr}


\subsection{Explanation of Classes}

\begin{description}

\item[Trash/Criminal] No legitimate employment. 

  \begin{example}
    Thug, body-snatcher, bandit, pirate, beggar.
  \end{example}

\item[Bonded] There is no slavery in the Baronies. This is the next
  best thing: enforced servitude to one master for a long period [up
    to life], through birth, contract, or debt.
  
  \begin{example}
    Serf, villain, unskilled or semi-skilled servant, labourer,
    indentured apprentice or journeyman in a craft or trade guild,
    dependent artisan contracted to a master, lay member of an
    accepted religious community, ordinary soldier or sailor.
  \end{example}

\item[Skilled Retainer] Voluntarily employed person, physically and
  legally capable of seeking a position elsewhere.  Owns the tools of
  the trade and has other, limited possessions. Usually works under
  the direction of a goodman or master. Occasionally an itinerant
  artisan of low status.

  \begin{example}
    Clerk, court musician, religious acolyte, freeborn shepherd or
    farm hand, merchant’s assistant, family chaplain, tinker, fisher.
  \end{example}

\item[{Goodman [Goodwife, Goody]}] Head of a household:
  more possessions and commitments than a mere retainer, comparatively
  independent.  Usually leases or owns a smallholding (if in the
  countryside) or a few rooms (if in a town).  Much contact with
  social peers and superiors.  Often employs skilled retainers.
  Includes itinerant professionals and artisans of high status.

    \begin{example} 
      Miller, pilot, established artisan, minor trader, innkeeper,
      accredited witch, priest in an accepted temple, shop owner, poor
      freeman-farmer, forester, gamekeeper, itinerant or privately
      employed alchemist, healer, magician or blacksmith.
    \end{example}
    
\item[{Master [Mistress, Mother]}] Like a goodman, but with
  a larger establishment, more employees, more commitments to
  subordinates and equals.  Tied to one place as direct contact with,
  and obligations to, social superiors and Guilds may make impolitic
  any relocation or other changes in social conduct, despite
  theoretical liberties and rights.

  \begin{example} 
    Guild master of a smaller craft/trading guild, or councillor in a
    more powerful one, wealthy freeman farmer, professional
    (alchemist, healer etc) trading publicly, with own shop and
    apprentices, Alphonse the famous chef, a Ducal Kapellmeister,
    high-priest of an accepted temple, captain-owner of a trade ship,
    mayor of a medium town.
  \end{example}
  
\item[Military] A socially sanctioned, trained fighter or skilled
  ancillary.  This includes sergeants and lowborn lesser officers
  (lieutenants, etc); high-ranking officers are ex officio gentry.
  
  \begin{example} 
    Town guardsman, skilled scout or military spy, army blacksmith,
    (legal) mercenary captain.
  \end{example}

\item[Gentry] By birth or service entitled to a coat of arms:
  significant social or military duties.  There are often many social
  gradations of gentry not comprehensible to persons outside that
  class.  Often possesses an estate or “independent means” but is not
  of lordly rank; such persons may, technically, be employed (but
  usually to a lord, or in service to their country).  May have
  difficulty ensuring all children have an acceptable start in society
  (especially in larger families).

  \begin{example}
    Knight, country squire, beneficed parson, port-reeve, courtier of
    significance, respected \& influential magician, judicial officer
    of a town or district, tax farmer, non-noble army or navy officer
    (generally Captain \& up), cadet member of a noble family.
  \end{example}

\item[Lesser Noble] Of lordly rank. Similar to the gentry, but
  definitely a cut above.  Normally owes feudal service to, or
  through, a greater noble.

  \begin{example}
    Non-independent Baron, Lord Admiral of a small navy, General,
    ordinary Abbot or other Head of an established, accepted,
    religious house, former gentry ennobled for extraordinary or
    personal services to a great noble or royalty.
  \end{example}
  
\item[Merchant-Prince] Extremely wealthy city-based merchant, head of
  an extended trade/family.  Controls a nationally significant
  trade-empire and / or monopoly.  Has significant power in the local
  guilds.  Extensive resources (especially in his/her home city), with
  contacts and enemies in several countries.  Capable of ordering
  actions deemed criminal in less influential personages. On a roll of
  98, the family head is the character’s parent; on 95--97, the head
  is a little more distant (perhaps uncle or cousin).

  \begin{example}
    Owner of a trade-fleet, trader with a national monopoly on a
    commodity (e.g.  silk, wine), Guild master of a powerful guild.
  \end{example}

\item[Greater Noble] Ruler of a minor country, or head of one of the
  “Great” families in a larger country.  Will have several estates and
  titles.  Usually has subordinates of lordly rank.  Children may have
  courtesy titles.

  \begin{example}
    An independent Baron, Marshall of a Duke’s or independent Count’s
    armies, Bishop, Abbot of a mother-abbey, Marshall or vicar-general
    of a powerful order, Count within a duchy, Lord Admiral of a
    maritime nation.
  \end{example}

\end{description}

Greater Noble and Merchant-Prince families impact seriously on the
campaign; the generating GM may need time to consult with other GMs
before the character’s background is finalised. Characters who wish to
retain an acknowledged, good social-standing may have to devote time
and money to maintain their position by indulging in appropriate
behaviour --- noblesse oblige.

\subsection{Birth Order}

Players should now choose their birth order, or roll on the following
table. Note that it is unlikely that an heir will go adventuring (at
least not without active encouragement from the next-in-line).

\begin{dqtblr}{colspec={lX},hline{1-2}={0.4mm}}
Die	& Birth Order \\
1	& 1st or 2nd \\  
2--3	& 3rd \\
4	& 4th \\
5	& 5th \\
6	& 6th \\  
7--8	& 7th \\
9	& 8th or later \\
0	& bastard \\
\end{dqtblr}

\subsection{Disinheritance}

Beginning characters never start with an estate, magic possessions, or
other “real” wealth.  For game reasons, characters seldom inherit
while actively adventuring.  Most classes will happily pass over an
adventurer in favour of more deserving and capable stay-at-home
siblings.  If the heir or heiress cannot be passed over (e.g. a noble
estate) and the player does not wish to retire the character, a
trusted kinsman or tenant must be appointed as trustee or warder, to
administer and enjoy the inheritance until it is reclaimed.

A noble or wealthy parent may disown adventuring children either
through disfavour, or for mutual protection.  A beginning character
doesn’t want to be set upon by family enemies, and no parent wants the
social stigma of refusing to pay a ransom.  The guild fully supports
such characters adventuring under an alias, just as it also supports
gifted adventurers who fled legal restraints in order to join the
guild (e.g. a runaway serf turned mage).  Both classes do have the
obligation not to expose their fellow adventurers to unnecessary risks
arising from their backgrounds.

In most cases, achievement begets amnesty.  A serf who has spent a
year and a day in a town becomes a freeman; a now wealthy prodigal is
welcomed back into the family fold.


\section{Starting Abilities and Possessions}
\label{starting}

This section covers abilities and possessions a character has prior to
starting life as an adventurer.  None of the experience points awarded
in this section are adjusted by any racial experience modifiers but
the player must use their character’s race and heritage as a guideline
to the allocation or choices they make. Except where noted, the normal
acquisition and ranking rules apply to the spending of experience
points.  This section must be started after all other sections are
completed, and each sub-section must be completed in order.

\subsection{Language Skills}

Every character knows their native language, the Alusian trade
language (Common) and possibly another language. A Guild member will
be literate in at least one language and literacy is required to learn
magic.

The player should get the GM’s assistance to determine what their
character’s native language is and then choose one of the following
options for their starting language skills:

\begin{description}

\item[Option A] Rank 8 and literate in either native language or
  common, Rank 6 in the other of native language or common, Rank 4 in
  any other common language.

\item[Option B] Rank 8 and literate in either native language or
  common, Rank 7 and literate in the other of native language or
  common, Rank 1 in any other language.

\item[Option C] Rank 9 and literate in either native language or
  common, Rank 6 and literate in the other of native language or
  common.

\end{description}
  
\subsection{Adventuring Skills}

A character starts with Rank 0 in the Adventuring skills of
Horsemanship, Climbing, Swimming and Stealth.  The player now receives
1250 experience points that may be spent on improving these skills.
Any experience points left over are lost. They also gain Rank 0
Flying, but may not raise it at any stage during Character Generation.

The possible combinations are: 

\begin{dqtblr}{colspec={m{16em}X},hline{1-2}={0.4mm}}
Horsemanship, Climbing and Swimming (in any order)	& Stealth \\
4, 0, 0							& 0 \\
3, 0, 0							& 1 \\
3, 2, 1							& 0 \\
2, 2, 0							& 1 \\
2, 2, 2							& 0 \\
\end{dqtblr}

\subsection{Mage or Non Mage?}

The player must decide whether the character will be a magic user or
not. (This choice can be made at any time during character
generation).

\subsubsection{Mage}

If the character is to be a magic user then the player must choose a
college of magic for the character to belong to.  Remember that there
is a minimum Magical Aptitude requirement for each college.


\begin{dqtblr}{colspec={Xc},hline{1-2}={0.4mm}}
College			& MA \\
Naming Incantations	& 1 \\
Mind			& 11 \\
Fire			& 12 \\
Air			& 13 \\
Ice			& 13 \\
Illusion		& 13 \\
Celestial		& 14 \\
Earth			& 15 \\
Bardic			& 16 \\
E \& E			& 16 \\
Necromancy		& 16 \\
Binding \& Animating	& 17 \\
Water			& 18 \\
Witchcraft		& 18 \\
\end{dqtblr}

The character now receives all of the general knowledge abilities of
their college including talents, general knowledge spells, general
knowledge rituals, both counterspells, the purification ritual and
ritual spell preparation.

The player should list these on their character sheet.

\subsubsection{Non Mage}

If a player decides that their character will not be a magic user then
they receive 6500 experience points to be spent in the following
order:

\begin{enumerate}
\item 2500 must be spent on either 1 point of Fatigue 
or 3 points of Perception. 

\item The character must acquire one new skill at rank 2, and may
  acquire a second new skill at no more than rank 1. The Warrior skill
  may not be chosen at this time.

\item The character must acquire one weapon at rank 2, and may acquire
  up to two weapons at no more than rank 1.

\item The player may save up to 500 points to spend later.  The player
  must spend any remaining points on any of:

  \begin{itemize}
    \item 1 rank in any known adventuring skills  
    \item more ranks in any known languages
    \item more perception. 
  \end{itemize}

\end{enumerate}

Any remaining points (other than the permissible 500) are lost.

\subsection{Background Experience}

A character now chooses any one Artisan skill at Rank 0.  This
reflects knowledge gained through childhood and must be appropriate to
their family background.

They also receive 2500 Experience Points which, together with any left
over from the non-mage generation, can be spent freely.

At this time the character may acquire any one new skill at Rank 0 for
the cost of only 100 EP (rather than the usual cost).

If there is any EP remaining it may be saved for spending later in the
game.

\subsection{Background Possessions}

The character will have two sets of clothing of a quality appropriate
to their family background.  They also have goods up to the value of
500 sp which may be chosen from the Basic Price List in the Players
Guide.  Up to 50 sp may be saved as cash.

\subsection{Modified Agility and Manual Dexterity}

The player should calculate any agility modifiers from the weight of
their possessions and any armour penalties; see the Encumbrance Table
(\S\ref{table:fatigue}) for values.  They should then calculate their
modified TMR from this value, see the table (\S\ref{table:tmr}) for
values.  If the character uses a shield, they should modify their
Manual Dexterity as well.

\subsection{Finishing the Character}

The player must choose a name for their character. 

They should enter every piece of relevant data onto their Character
Sheet, and calculate base chances and other variables. The generating
GM will check it, and then sign \& date it as complete.

\end{Chapter}
