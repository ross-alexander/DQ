\begin{Chapter}{Character Generation}

There are six sections in Character Generation: 
\begin{Itemize}
\item 1.1  Characteristic Points 
\item 1.2   Race 
\item 1.3   Description 
\item \ref{aspects}  Aspects
\item 1.5   Heritage 
\item 1.6  Starting Abilities \& Possessions 
\end{Itemize}

Sections 1.1 – 1.5 may be done in any order. Each section is designed
so that a player may choose from a range of options or randomly
generate their character. Section 1.6 should be done last.

\section{Characteristic Points}

A character has 6 primary statistics which are generated by allocating
points from a total, and 4 secondary statistics which are either
derived from the primary statistics or are generated randomly.  The
higher the number, the better the characteristic.

\subsection{Generating Characteristic Points}

The player may choose to allocate the primary statistics from a total
of 90 points or may roll 2D10 once against the following table.  If
they choose to roll the result must stand.

Die Roll  Points Total 

2 
3 
4 
5 
6 
7 
8 
9 
10 
11 
12 
13 
14 
15 
16 
17 
18 
19 
20 

81 
82 
83 
84 
85 
86 
87 
88 
89 
90 (default choice) 
91 
92 
93 
94 
95 
96 
97 
98 
99 

\subsection{Assigning Characteristic Points}

This total of points needs to be spent on the follow- ing
characteristics: Physical Strength, Manual Dexterity, Agility, Magical
Aptitude, Willpower \& Endurance.  These characteristics may change
during the game, and may be raised up to 5 points through training,
though not past the character’s racial maximum.

The human range for each of these characteristics is 5 – 25; this
range is adjusted for non-humans (see the Characteristic Modifier
tables for the non- human races). These ranges represent the minimum
and maximum capabilities of the races. The player should assign the
points and then make any ad- justment for race.

Prior to assigning the characteristic points, the player should give
some thought to what kind of character they wish to have and what
weapons, spells, and/or skills are desired for the newly cre- ated
individual. Some weapons require a great deal of Physical Strength or
Manual Dexterity, and the player should be sure to assign enough
points in those areas to use the weapons of their choice. All magical
colleges require a minimum Magic Apti- tude to join and the player
should be aware of these restrictions.  Most skills do not have any
special requirements, but many give bonuses for exceeding a minimum
value in certain characteristics.

When the player has chosen the values for the character, they must
record them on a Character Sheet. The total value of the six primary
character- istics (before racial modifiers) must equal the amount
received in the Generating Characteristic Points section; thus, a
player cannot “save” Characteristic Points and assign them to
characteristicsat a later date. The value of each of the six primary
characteristics must be recorded before any secon- dary
characteristics are generated.

\subsection{Generating Secondary Characteristics}

Fatigue, Physical Beauty, Perception and Tactical Movement Rate are
secondary characteristics.  They may be modified if the character is
non-human (see the Characteristic Modifier tables for the non-human
races).

\subsubsection{Fatigue}

The value of a character’s Fatigue is a direct function of their
Endurance.  The player enters the Fa- tigue value corresponding to the
character’s Endurance value after their Endurance has been modified
for race.

Endurance  Fatigue 

16 
17 
18 
19 
20 
21 
22 
23 
24 

3 or 4 
5 to 7 
8 to 10 
11 to 13 
14 to 16 
17 to 19 
20 to 22 
23 to 25 
26 to 27 
Endurance  and  Fatigue  values  in  bold  type  can  be 
achieved  only  by  members  of  certain  non-human 
races. 

From  this  point  on,  a  change  in  a  character’s  En-
durance  value  will  not  affect  their  Fatigue  value 
and  vice-versa.  Fatigue  may  be  raised  by  up  to  5 
points,  though  not  past  the  character’s  racial 
maximum. 

Physical Beauty 

The  value  of  the  Physical  Beauty  characteristic  is 
generated randomly by rolling 4D5 + 3. This char-
acteristic can never be increased by training. 

Perception 

A  character’s  perception  value  begins  at  5.  This 
may be trained up to racial maximum. 

Tactical Movement Rate 

A character’s Tactical Movement Rate (TMR) is a 
direct  function  of  their  Agility.  It  is  based  on  the 
character’s  Agility  value  and  is  recalculated  when 
Agility  is  modified  by  encumbrance  and  armour 
penalties; see the TMR table (§58.2) for values. 

\subsection{Race}

A player must choose the race of their character.

The majority of people in Alusia are human, but the player may choose
one of the common non- human races: dwarf, elf, halfling, or orc.

If the player wishes their character to be a giant or shapechanger
they must roll D100.  They may roll once per race and if the roll is
lower than the race chance \% they must take that race. If they fail
then the character must be of one of the common races.  If the player
is attempting to be a shapechanger they must decide what type of
shapechanger they want prior to rolling (i.e. wolf, tiger, bear or
boar).

Race 

Chance (%) 

06 
04 

Hill Giant 
Shapechanger 
A  player  may  wish  to  play  one  of  the  very  rare 
sentient  races.  To  do  so  they  must  get  the  agree-
ment of both the generating GM and a member of 
the  character  tribunal.  They  will  decide  which  of 
the common races has the appropriate racial modi-
fiers.  For  example  Erelheine  characters  are  gener-
ated using the Elf option. 

Humans learn faster than non-humans. Learning is 
represented in game by spending Experience Points 
(EP).  Divide  any  experience  points  a  character 
gains  by  the  “racial  modifier”  and  then  spend  the 
result normally. 

Race 

Modifier 

1.1 
1.2 
1.1 
1.5 
1.0 
1.1 
1.4 

Dwarf 
Elf 
Halfling 
Hill Giant 
Human 
Orc 
Shapechanger 
 
For every 25,000 Experience Points (EP) the char-
acter  has  spent  towards  the  'racial  modifier'  it  is 
lowered  by  .1  (but  not  below  1.0)  or  after  20  ad-
ventures  (when  PC  reaches  racial  max),  the  EM 
becomes  1  (whichever  happens  first).  E.g.  once  a 
giant  has  lost  25,000  EP to  their  race,  their  'racial 
modifier'  is  lowered  to  1.4.  Once  they  spend  an 
additional 25,000 EP on their new 'racial modifier' 
of 1.4 it would become 1.3. 

RM   Amt  earned  
that cost 25k  

spent  
Amt 
that cost 25k  

75,000  
87,500  
108,333  
150,000 
275,000  

1.5  
1.4  
1.3  
1.2  
1.1  
 

50,000  
62,500  
83,333  
125,000  
250,000 

Each  race  has  a  description  of  a  stereotypical 
member of the race and any special abilities and 
characteristic  modifiers  that  apply  to  a  charac-
ter of that race. 

Dwarf 

A  dwarf  is  a  short,  bearded  humanoid,  usually 
taciturn who frequents mountainous areas. 

Description:  Pride  and  attention  to  detail  are  im-
portant  to  dwarves.  They  form  strong  community 
ties,  and  are  distrustful  of  strangers,  especially 
those of other races. Their strongest antipathies are 
towards  orcs  and  elves.  Although  dwarves  are 
greedy  by  nature,  they  are  essentially  honest  and 
stand by their word. Dwarves covet precious stones 
and metals, and appreciate fine, detailed workman-
ship. Dwarven warriors favour the axe as weapon. 

Special Abilities 

1. Dwarves’ close vision is exceptionally sharp, but 
many  have  poor  distance  vision.  They  can  see  in 
the  dark  as  a  human  does  at  dusk.  Their  effective 
range of vision in the dark is 50 feet under the open 
sky,  100  feet  inside  manmade  structures,  and  150 
feet inside caves and tunnels. 

2.  Dwarves  can  assess  the  value  of  and  deal  in 
gems and metals as if they are a Merchant of Rank 
5.  If  a  dwarf  character  progresses  in  the  Merchant 
skill,  their  ability  to  assess  the  value  of  gems  and 
metals  is  five  greater  than their  current Rank,  to a 
maximum of ten. 

3.  If  a  dwarf  character  is  a  Ranger  specialising  in 
mountains  or  caverns,  they  pay  half  the  EP  cost 
necessary to advance ranks. 

4. A dwarf’s capacity for alcohol is twice that of a 
human. 

Characteristic 

Modifier 

Physical Strength 
Agility 
Endurance 
Magical Aptitude 
Willpower 
Perception 
Physical Beauty 
Tactical Movement Rate 
Starting Age: 
Average Life Span: 

+ 2 
- 2 
+ 2 
- 2 
+ 2 
+ 1 
- 2 
- 1 
20 +  
125 – 150 years 

Elf 

An  elf  is  a  slim  agile  humanoid,  who  frequents 
wooded areas. 

Description: Elves are virtually immortal and gen-
erally  take  the  long  term  view.  They  are  insular, 
indifferent  to  others  and  tend  to  be  traditional. 
Elves  are  great  respecters  of  nature  and  learning. 
Their Elders are repositories of great wisdom while 
elvish  youth  are  enthusiastic  merry  makers.  Elven 
warriors  favour  bow  weapons  and  disdain  metal 
armour.  Members  of  other  races  generally  find 
elves attractive. 

Special Abilities 

1.  Elves  have  superior  vision  especially  over  long 
distances or in poor lighting. An elf can see in the 
dark as a human does on a cloudy day. Their effec-
tive  range  of  vision  in  the  dark  is  150  feet  under 
the open sky, and 75 feet elsewhere. 

2.  If  an  elf  character  is  a  ranger  specialising  in 
woods, they pay one-half the EP to advance ranks. 

3.  An  elf  receives  a  racial  Talent  which  functions 
in all respects as the Witchcraft Witchsight Talent. 

4.  An  elf  makes  little  or  no  noise  while  walking 
and adds 10% to their chance to perform any activ-
ity requiring stealth. 

5.  If  an  elf  character  takes  the  healer  skill,  the  elf 
pays  three-quarters  the  EP  to  advance  ranks, 
though they cannot resurrect the dead. 

6.  An  elf  is  impervious  to  the  special  abilities  of 
the lesser undead. 

7. If an elf character takes the courtier skill, the elf 
pays one-half the EP to advance ranks. 

Characteristic 

Modifier 

- 1 
Physical Strength 
+ 1 
Agility 
- 1 
Endurance 
+ 1 
Magical Aptitude 
+ 1 
Willpower 
+ 1 
Perception 
+ 2 
Physical Beauty 
Tactical Movement Rate  + 1 
Starting Age 
Average Life Span 

30 – 300 +  
Circa 10,000 years 

Halfling 

A  halfling  is  a  short,  cheerful  humanoid,  who 
will be an active participant in village life. 

Description:  Halflings  appreciate  the  good  life 
more than most; a successful halfling will arrange a 
schedule  of  much  sleep,  good  food,  and  relaxed 
study  or  conversation.  Halflings  are  shy  around 
other  races,  preferring  to  merge  into  the  back-
ground.  Amongst  themselves  they  are  a  friendly 
folk  who form into small communities  where  eve-
ryone  knows  everyone  else’s  business.  While 
Halflings  take  their  social  responsibilities  seri-
ously,  they  are  renowned  for  their  practical  jokes 
and  light  fingers.  Halflings  are  noted  for  their 
tough, hairy feet and usually go barefoot. Halflings 
avoid the rigours of military life but when forced to 
defend themselves they favour small weapons. 

Special Abilities 

1. A halfling has infravision, which allows them to 
see faint red shapes where living beings are located 
in the dark. Their range of vision is 100 feet. 

2. A halfling adds 20% to their chance to perform 
any activity requiring stealth. 

3.  If  a  halfling  takes  the  thief  skill,  they  pay  half 
the EP cost to advance ranks. 

4.  A  halfling  may  drop  jewellery  down  active 
volcanoes  without  anyone  thinking  the  worse  of 
them. 

Characteristic 

Modifier 

Physical Strength 
Manual Dexterity 
Agility 
Endurance 
Magical Aptitude 
Willpower 
Physical Beauty 
Starting Age 
Average Life Span 

- 3 
+ 3 
+ 1 
- 2 
- 1 
+ 1 
- 1 
21 + 
80 – 90 years 

Hill Giant 

A hill giant is a huge, coarse featured humanoid, 
who has no patience for laborious learning. 

Description:  Giants  are  lusty  types,  preferring 
nothing  better  than  to  go  through  life  brawling, 
drinking,  and  wenching.  They  tend  to  gather  to-
gether  in  a  clan  arrangement,  building  huge  halls 
(or steadings) in out-of-the-way locations. They are 
not overly intelligent, and resent humans and elves 
particularly.  Giants  enjoy  riddling  and  bartering. 
Giant  warriors  favour  simple  weapons  scaled  to 
their size. 

Special Abilities 

1.  A  giant  has  infravision,  which  allows  them  to 
see faint red shapes where living beings are located 
in the dark. Their range of vision is 250 feet. 

2. A giant’s magic resistance is increased by 10%. 

3. Whenever a giant attempts minor magic, the GM 
should increase the difficulty factor by one, making 
it easier. 

4.  Giants  may  use  the  giant  weapons  listed  in  the 
Weapons Table (§56.1 ). 

Characteristic  

Modifier 

+ 7 
Physical Strength  
- 1 
Manual Dexterity  
- 2 
Agility  
+ 8 
Endurance  
- 1 
Magical Aptitude  
- 1 
Willpower  
+ 1 
Fatigue  
Physical Beauty  
- 1 
Tactical Movement Rate   + 3 
+ 1 
Natural Armour  
26 +  
Starting Age  
Average Life Span  
500 years 

Human 

Humans  are  by  far  the  most  common  race  on 
Alusia, frequenting most areas and climes. 

Description:  Humans  have  a  great  diversity  of 
cultures,  languages  and  sub-racial  traits,  such  as 
hair and eye colour or skin tone. Human behaviour 
is  an  odd  mix.  They  can  be  superstitious  and  dis-
trustful of the unknown, but they are also insatiably 
curious  and  look  for  new  knowledge.  Many  also 
seek  personal  fame  and  fortune  as  most  human 
social  structures  are  less  rigid  than  those  of  non-
humans and a person’s birth need not permanently 
define their place in society. This odd combination 
of  attributes  has  led  them to  become  great  explor-
ers  and  sailors,  and  they  will  venture  boldly  into 
unexplored  areas  in  search  of  knowledge  and 
wealth. Humans build great cities and are far more 
welcoming  of  other  races  than  most.  Outside  of 
their own culture they are social chameleons, adept 
at adapting their behaviour to match local customs. 

Special Abilities 

1. Humans can ingratiate themselves with strangers 
more  readily  than  other  races.  A  human  character 
has  +10  to  any  reaction  roll  in  an  encounter  with 
sentient creatures. 

Characteristic  

Starting Age  
Average Life Span (varies 
widely with wealth and culture) 

Modifier 

16 + 
40 – 90 years 

Orc 

An  Orc  is  a  stoop-shouldered,  surly  humanoid 
and a pack member by nature. 

Description:  Orcs  are  a  cruel,  violent  folk,  liking 
nothing  better  than to  loot  and  pillage.  Individuals 
test  themselves  against  their  peers,  bullying  any-
thing  weaker  but  cowering  away  from  anything 
stronger.  A  strong  individual  will  form  a  pack 
around  them,  and  the  pack  leader’s  word  is  law. 
Orcs enjoy the sensual pleasures of life, and reduce 
their  already  short  life  spans  through  hard  living. 
They  have  a  robust  digestion  and  will  eat  foods 
that  others  turn  their  nose  up  at.  Orc  warriors  fa-

6 

vour the great axe and glaive. Orcs are considered 
unattractive by other humanoid races. 

Special Abilities 

1.  An  orc’s  eyes  are  highly  light-sensitive.  They 
must  decrease  their chance  of  hitting  a  target  with 
Ranged Combat by 10\% in daylight. 

2. An orc has infravision, which allows them to see 
faint red shapes where  living beings are located in 
the dark. Their range of vision is 150 feet. 

3.  Orcs  are  either  back-stabbing  scum  or  brutal 
bully-boys. An orc may take one of either Assassin 
Skill or Warrior Skill and pay three-quarters the EP 
to advance in Ranks. 

4.  An orc’s seed is highly fertile.  The orc and hybrid orc
population increase mitigates against the high orc fatality rate.

Characteristic  

Modifier 

Physical Strength  
Endurance  
Magical Aptitude  
Willpower  
Fatigue  
Physical Beauty  
Natural Armour  
Starting Age  
Average Life Span 

Shapechanger 

+ 2 
+ 1 
- 2 
- 2 
+ 2 
- 4 
+ 1 
12 +  
40 – 45 years 

Shapechangers  are  a  hidden  race  amongst  hu-
mans, with the ability to change into the form of 
a particular animal. 

Description:  Shapechangers  are  identical  in  ap-
pearance to humans when not in animal form. They 
are somewhat bestial in nature, adopting traits one 
might  expect  from  an  anthropomorphised  wolf, 
tiger,  bear  or  boar.  There  exists  a  love/hate  rela-
tionship  between  humans  and  shapechangers: 
shapechangers  possess  some  degree  of  animal 
magnetism,  but,  if  discovered,  can  expect  severe 
treatment  at  the  hands  of  humans.  Shapechangers 
are,  on  the  whole,  bitter  towards  humans,  and  are 
not  above  using  humans  to  their  advantage.  There 
are  very  few  ways  to  tell  a  shapechanger  from  a 
human  (e.g.  they  will  be  discomforted  by  wolf-
bane)  and  these  vary  by  shapechanger  type. 
Shapechangers are a ruthless lot. 

Special Abilities 

1.  A  shapechanger  can  change  from  human  to 
animal  form  (or  vice-versa)  in  10  seconds  during 
daytime and 5 seconds during the night-time. 

2.  A  shapechanger  possesses  a  dual  nature.  While 
in  animal  form,  human  inhibitions  will  be  muted; 
while  in  human  form,  animal  instincts  will  be 
dulled. 

3.  A  shapechanger  cannot  be  harmed  while  in 
animal  form,  unless  struck  by  a  silvered  weapon, 
magic  or  by  a  being  with  a  Physical  Strength 
greater than 25. Five Damage Points are automati-
cally absorbed in the latter case. 

4.  A  shapechanger  will  regenerate  1  Endurance 
Point every 60 seconds while in animal form. 

5.  The  player  must  devise  a  set  of  characteristics 
for their animal form. Take the difference between 
the  average  for  each  characteristic  in  animal  and 
human form, and modify the human characteristics 
appropriately. 

6. A shapechanger is automatically lunar aspected. 

7. A shapechanger can remain in animal form for a 
quarter of the night times the quarters of the moon 
showing  (i.e.  at  full  moon  they  may  remain  in 
animal  form  all  night).  During 
the  day  a 
shapechanger  can  remain  in  animal  form  for  one 
hour 
the  moon.  A 
shapechanger can make one set of transformations 
times  the  quarter  of  the  of  the  moon  per  day  (i.e. 
dawn to next dawn). 

the  quarter  of 

times 

8.  If  a  shapechanger  is  in  animal  form  during  the 
day,  there  is  a  1\%  cumulative  chance  for  each  5 
minutes  they  remain  in  animal  form that  they  will 
never  be  able  to  change  back  into  human  form. 
Similarly,  if  the  shapechanger  exceeds  the  time 
limits  given  above,  there  is  a  1\%  cumulative 
chance  (per  5  minutes)  of  their  not  being  able  to 
return to human form. 

9. A shapechanger will be inconvenienced by those 
wards which can be used against were-creatures. 

10. A shapechanger’s magic resistance is increased 
by 5\%. 

11.  If  a  shapechanger  takes  the  courtier  skill  they 
pay three-quarters the Experience Points necessary 
to advance ranks. 

Characteristic  

Modifier 

+ 1 
16 +  

Physical Beauty  
Starting Age  
Average Life Span   55 – 65 years 
A separate set of characteristics must be generated 
for the animal form (see Ability 5 above). 

\section{Description}

This section covers height, weight, gender, primary hand, and general
description.

\subsection{Height and Weight}

A player should choose their character’s height and weight.  The
character’s height and weight should be chosen according to the
player’s idea of the character, with due regard to the character’s
pri- mary characteristics, race and background.

The following charts give a range of heights and weights within which
90\% of adventurers fall, and the average values within that range.
Please modify your chosen height and weight according to gender and
racial adjustments as below.

Normal Base  

Height  

Weight   Range 

100–170 
5’3"  
110–185 
5’6"  
120–200 
5’9"  
130–220 
6’0"  
6’3"  
145–240 
Adjustments   Height   Weight 

130  
140  
150  
165  
180  

Human Male  
Human Female  
Orc Male  
Orc Female  
Elf Male  
Elf Female  
Short Folk Base 

+0"  
-4"  
-4"  
-6"  
+5"  
+2"  

100% 
80% 
110% 
100% 
80% 
65% 

Height  

Weight   Range 

3’9"  
4’0"  
4’3"  
4’6"  
4’9"  
Adjustments  

65-110 
85  
75-125 
95  
85-140 
105  
95-155 
115  
105-170 
125  
Height   Weight 

Dwarf Male  
Dwarf Female  
Halfling Male  
Halfling Female  
Hill Giants Base 

+0"  
-2"  
-12"  
-13"  

100\% 
90\% 
65\% 
60\%  

Height  

Weight   Range 

295–490 
8’4"  
335–555 
8’8"  
375–625 
9’0"  
420–700 
9’4"  
9’8"  
465–780 
Adjustments   Height   Weight 

370  
420  
470  
525  
580  

Giant Male  
Giant Female  

+0"  
-4"  

100\% 
90\% 

Gender 

A  player  may  choose  whether  their  character  is 
male or female. It is recommended the character be 
the same gender as the player, as playing the oppo-
site gender convincingly is difficult. 

Optionally,  some  characteristics  may  be  adjusted 
for a female character. This would also modify her 
appropriate racial maximums. 

Female Characteristic Modifier 

Characteristic   Modifier 

Physical Strength  
-2 
Manual Dexterity   +1 
Endurance  
+1 

Primary Hand 

A  player  must  determine  whether  their  character’s 
Primary  Hand  is  their  right  or  their  left.  This  de-
termination  affects  which  hand  a  weapon  is  held 
during  combat,  and  any  penalties  assigned  for 
attacking with a weapon in a secondary hand. 

They  may  choose  either  right  or  left,  or  roll  ran-
domly. If they choose to roll, the result must stand. 

The  player  rolls  D5  and  D10.  If  the  D10  result  is 
greater, the character’s right hand is primary. If the 
D5 result is higher, their left hand is primary. If the 
two  results  are  equal,  the  character  is  ambidex-
trous. 

Description 

The  player  will  sometimes  need  to  describe  their 
character  and  should  therefore  think  about  the 
character’s  physical  appearance  based  on  the  gen-
erated characteristics. They should choose hair, eye 
and  skin  colour  (based  on  race  and  family  back-
ground). 

\section{Aspects}
\label{aspects}

The timing of a character’s birth orients them towards one of several
astrological influences, or aspects.  A character will benefit during
the time their aspect is powerful, and will suffer when the opposite
aspect is powerful.

The times of high noon and midnight are extremely 
important  when  applying  the  effects  of  aspects. 
The  GM  should  allow  characters  to  perform  ac-
tions at precisely those instants, though the passage 
of time must be properly monitored. 

Generating an Aspect 

The  player  may  choose  an  Aspect  as  if  they  had 
rolled  any  number  up  to  80,  or  roll  D100  once 
against  the  following  table.  If  they  choose  to  roll 
on the table, any roll over 80 may be re-rolled. 

If  the  character  is  joining  one  of  the  elemental 
colleges the player may choose any aspect between 
1 and 80 that is neutral to their college, or they may 
roll. 

Die  

Aspect 

01–05   Winter Air 
06–10   Winter Water 
11–15   Winter Earth 
16–20   Winter Fire 
21–25   Spring Air 
26–30   Spring Water 
31–35   Spring Earth 
36–40   Spring Fire 
41–45   Summer Air 
46–50   Summer Water 
51–55   Summer Earth 
56–60   Summer Fire 
61–65   Autumn Air 
66–70   Autumn Water 
71–75   Autumn Earth 
76–80   Autumn Fire 
81–85   Solar 
86–90   Lunar 
91–95   Life 
96-00   Death 

Effects of Aspects 

Apart  from  elemental  aspects,  all  modifiers  apply 
to percentile rolls, not base chances. 

Elemental Aspects 

Characters gain a bonus of 1% on the Base Chance 
of  performing  any  magic  of  the  same  College  as 
their elemental aspect, and a penalty of -1% on the 
opposed  College.  Air  opposes  Earth  and  Fire  op-
poses  Water.  Ice  and  Celestial  magic  is  not  af-
fected. 

Seasonal Aspects 

A  character  is  affected  by  their  seasonal  aspect 
during  their  aspect’s  season  and  the  opposite  sea-
son.  The  following  table  lists  the  seasonal  aspect 
effects and when they apply. 

Time  

Effect 

-10 
-25 

Midnight, Aspect’s Season  
Midnight, Equinox or Solstice of As-
pect’s Season  
Midnight, Opposite Season  
Midnight, Equinox or Solstice of Oppo-
site Season  
The  effect  is  applied  for  30  seconds  before  and 
after midnight. 

+10 
+25 

Solar and Lunar Aspects 

A  character  of  solar  or  lunar  aspect  is  affected  by 
their aspect at high noon and midnight. The follow-
ing table lists the Solar aspect effects, and when to 
apply them. 

Time  

Effect 

-5 
Noon  
+5 
Midnight  
-25 
Noon, Summer Solstice  
Midnight, Winter Solstice   +25 
Lunar  aspected  characters  gain  opposite  bonuses 
and  penalties  for  the  same  times.  The  effect  is 
applied  for  10  seconds  before  and  after  high  noon 
or midnight. If the sky is cloudy, the effect may be 
reduced to a minimum of +/1 and 5. 

Life and Death Aspects 

Life and Death aspected people are affected by the 
creation and destruction of life force. 

The  following  table  lists  the  Death  aspect  effects, 
and when to apply them. 

Event  

Range  Aspect  Effect 

+5 
+10 
+25 

100’ 
250’ 
500’ 

Death  
Death  
Death  

Birth of mammal  
Birth of humanoid  
Birth of close rela-
tive†  
Death of mammal  
Death of humanoid  
Death of close rela-
tive†  
†A  close  relative  is  no  more  distant  than  a  second 
cousin. 

Death  
Death  
Death  

50’  
125’ 
250’ 

-5 
-10 
-25 

Life aspected characters gain opposite bonuses and 
penalties  for  the  same  times.  Deaths  are  non-
cumulative  (only  one  can  be  in  effect  at  a  given 
time),  though  births  are  cumulative.  A  stillbirth 
does not affect a life or death aspected character. A 
resurrection is treated as a birth. 

A death event is applied for as many seconds as the 
effect  range  in  feet.  A  life  event  is  applied  for  3 
times as long. 

A female life aspected character will suffer no pain 
after giving birth, and will be as healthy and active 
as she was before she became pregnant. 

Light and Dark Aspects 

All  living  creatures  have  an  additional  celestial 
Light or Dark Aspect. This is fully explained in an 
addendum  to  the  College  of  Celestial  Magics 
(§19.8). 

\section{Heritage}

This section is relevant to humans, primarily from the Western Kingdom
and Cazarla, and should be adapted for other races or regions.

Social Status 

The  “social  status”  is  that  of  the  character’s  par-
ent(s), usually the father. The table does not repre-
sent the population, merely the proportion of back-
grounds  from  which  accepted  Adventurer’s  Guild 
applicants originally come. Most social classes are 
present 
in  a  variety  of  environments  (city, 
town/village,  rural,  court,  castle/stronghold,  mari-
time). A player may choose any social category in 
the  01-80  range  for  the  character’s  background  or 
roll; however, any such dice-roll must be accepted. 
In general, the higher the number rolled, the higher 
the social status within each band. A roll of 40-90 
optionally  may  indicate  a  respectably  retired  ex-
adventurer. 

Die  

Social Status 

01–14   Trash / Criminal 
15–20   Bonded 
21–29   Skilled retainer 
30–44   Goodman 
45–54   Master 
55–70   Military 
71–84   Gentry 
85–94   Lesser Noble 
95–98   Merchant-prince 
99–00   Greater Noble 

Explanation of Classes 

Trash/Criminal No legitimate employment. 

Example 

Thug, body-snatcher, bandit, pirate, beggar. 

Bonded There is no slavery in the Baronies. This is 
the next best thing: enforced servitude to one mas-
ter  for  a  long  period  [up  to  life],  through  birth, 
contract, or debt. 

Example 
Serf,  villain,  unskilled  or  semi-skilled 
servant,  labourer,  indentured  apprentice  or  journeyman  in 
a  craft  or  trade  guild,  dependent  artisan  contracted  to  a 
master,  lay  member  of  an  accepted  religious  community, 
ordinary soldier or sailor. 

Skilled  Retainer  Voluntarily  employed  person, 
physically and legally capable of seeking a position 
elsewhere.  Owns  the  tools  of  the  trade  and  has 
other, limited possessions. Usually works under the 
direction of a goodman or master. Occasionally an 
itinerant artisan of low status. 

Clerk,  court  musician,  religious  acolyte, 
Example 
freeborn  shepherd  or  farm  hand,  merchant’s  assistant, 
family chaplain, tinker, fisher. 

Goodman  [Goodwife,  Goody]  Head  of  a  house-
hold:  more  possessions  and  commitments  than  a 
mere  retainer,  comparatively  independent.  Usually 
leases  or  owns  a  smallholding  (if  in  the  country-
side)  or  a  few  rooms  (if  in  a  town).  Much contact 
with  social  peers  and  superiors.  Often  employs 
skilled  retainers.  Includes  itinerant  professionals 
and artisans of high status. 

Example 
Miller,  pilot,  established  artisan,  minor 
trader,  innkeeper,  accredited  witch,  priest  in  an  accepted 
temple, shop owner, poor freeman-farmer, forester, game-
keeper,  itinerant  or  privately  employed  alchemist,  healer, 
magician or blacksmith. 

Master: [Mistress, Mother] Like a  goodman, but 
with a larger establishment, more employees, more 
commitments  to  subordinates  and  equals.  Tied  to 
one place as direct contact with, and obligations to, 
social  superiors  and  Guilds  may  make  impolitic 
any  relocation  or  other  changes  in  social  conduct, 
despite theoretical liberties and rights. 

Example 
Guild  master  of  a  smaller  craft/trading 
guild,  or  councillor  in  a  more  powerful  one,  wealthy 
freeman  farmer,  professional  (alchemist,  healer  etc)  trad-
ing publicly, with own shop and apprentices, Alphonse the 
famous  chef,  a  Ducal  Kapellmeister,  high-priest  of  an 
accepted temple, captain-owner of a trade ship, mayor of a 
medium town. 

Military  A  socially  sanctioned,  trained  fighter  or 
skilled  ancillary.  This  includes  sergeants  and  low-
born lesser officers (lieutenants, etc); high-ranking 
officers are ex officio gentry. 

Example 
spy, army blacksmith, (legal) mercenary captain. 

Town  guardsman,  skilled  scout  or  military 

Gentry  By  birth  or  service  entitled  to  a  coat  of 
arms:  significant  social  or  military  duties.  There 
are  often  many  social  gradations  of  gentry  not 
comprehensible to persons outside that class. Often 
possesses  an  estate  or  “independent  means”  but  is 
not  of  lordly  rank;  such  persons  may,  technically, 
be employed (but usually to a lord, or in service to 
their  country).  May  have  difficulty  ensuring  all 
children  have  an  acceptable  start  in  society  (espe-
cially in larger families). 

Knight,  country  squire,  beneficed  parson, 
Example 
port-reeve, courtier of significance, respected \& influential 
magician, judicial officer of a town or district, tax farmer, 
non-noble army or  navy officer (generally  Captain \&  up), 
cadet member of a noble family. 

Lesser Noble Of lordly rank. Similar to the gentry, 
but  definitely  a  cut  above.  Normally  owes  feudal 
service to, or through, a greater noble. 

Example 
Non-independent Baron, Lord Admiral of a 
small  navy,  General,  ordinary  Abbot  or  other  Head  of  an 
established,  accepted,  religious  house,  former  gentry 
ennobled  for  extraordinary  or  personal  services  to  a  great 
noble or royalty. 

Merchant-Prince  Extremely  wealthy  city-based 
merchant,  head  of  an  extended  trade/family.  Con-
trols  a  nationally  significant  trade-empire  and  /  or 
monopoly.  Has  significant  power  in  the  local 
guilds.  Extensive  resources  (especially  in  his/her 
home  city),  with  contacts  and  enemies  in  several 
countries.  Capable  of  ordering  actions  deemed 
criminal in less influential personages. On a roll of 
98,  the  family  head  is  the  character’s  parent;  on 
95–97,  the  head  is  a  little  more  distant  (perhaps 
uncle or cousin). 

Example  
Owner  of  a  trade-fleet,  trader  with  a  na-
tional  monopoly  on  a  commodity  (e.g.  silk,  wine),  Guild 
master of a powerful guild. 

Greater  Noble  Ruler  of  a  minor  country,  or  head 
of  one  of  the  “Great”  families  in  a  larger  country. 
Will  have  several  estates  and  titles.  Usually  has 
subordinates  of  lordly  rank.  Children  may  have 
courtesy titles. 

An 

Example 
independent  Baron,  Marshall  of  a 
Duke’s or independent Count’s armies, Bishop, Abbot of a 
mother-abbey,  Marshall  or  vicar-general  of  a  powerful 
order, Count within a duchy,  Lord  Admiral  of a  maritime 
nation. 

Greater  Noble  and  Merchant-Prince  families  im-
pact seriously on the campaign; the generating GM 
may  need  time  to  consult  with  other  GMs  before 
the  character’s  background is  finalised. Characters 
who  wish to  retain an  acknowledged,  good  social-
standing  may  have  to  devote  time  and  money  to 
maintain their position by indulging in appropriate 
behaviour - noblesse oblige. 

Birth Order 

Players should now choose their birth order, or roll 
on the following table. Note that it is unlikely that 
an  heir  will  go  adventuring  (at  least  not  without 
active encouragement from the next-in-line). 

Die   Birth Order 

1st or 2nd 

1  
2–3   3rd 
4th 
4  
5th 
5  
6th 
6  
7–8   7th 
9  
0  

8th or later 
bastard 

Disinheritance 

Beginning  characters  never  start  with  an  estate, 
magic  possessions,  or  other  “real”  wealth.  For 
game  reasons,  characters  seldom  inherit  while 
actively  adventuring.  Most  classes  will  happily 
pass over  an adventurer  in  favour  of more  deserv-
ing and capable stay-at-home siblings. If the heir or 
heiress cannot  be  passed  over  (e.g.  a  noble  estate) 
and the player does not wish to retire the character, 
a  trusted  kinsman  or  tenant  must  be  appointed  as 
trustee  or  warder,  to  administer  and  enjoy  the  in-
heritance until it is reclaimed. 

A noble or wealthy parent may disown adventuring 
children  either  through  disfavour,  or  for  mutual 
protection.  A  beginning  character  doesn’t  want  to 
be  set  upon  by  family  enemies,  and  no  parent 
wants  the  social  stigma  of  refusing  to  pay  a  ran-
som.  The  guild  fully  supports  such  characters 
adventuring  under  an  alias, just  as it also  supports 
gifted adventurers who fled legal restraints in order 
to join the guild (e.g. a runaway serf turned mage). 
Both  classes  do  have  the  obligation  not  to  expose 

8 

their  fellow  adventurers  to  unnecessary  risks  aris-
ing from their backgrounds. 

In most cases, achievement begets amnesty. A serf 
who has spent a year and a day in a town becomes 
a  freeman;  a  now  wealthy  prodigal  is  welcomed 
back into the family fold. 

1.6 Starting Abilities and Possessions 
This  section  covers  abilities  and  possessions  a 
character has prior to starting life as an adventurer. 
None  of  the  experience  points  awarded  in  this 
section are adjusted by any racial experience modi-
fiers  but  the  player  must  use their character’s  race 
and  heritage  as  a  guideline  to  the  allocation  or 
choices they make. Except where noted, the normal 
acquisition and ranking rules apply to the spending 
of  experience  points.  This  section  must  be  started 
after  all  other  sections  are  completed,  and  each 
sub-section must be completed in order. 

Language Skills 

Every  character  knows  their  native  language,  the 
Alusian  trade  language  (Common)  and  possibly 
another language. A Guild member will be literate 
in at  least  one  language  and  literacy  is  required  to 
learn magic. 

The  player  should  get  the  GM’s  assistance  to  de-
termine  what  their  character’s  native  language  is 
and  then  choose  one  of  the  following  options  for 
their starting language skills: 

•  Option  A  Rank  8  and  literate  in  either  native 
language or common, Rank 6 in the other of native 
language or common, Rank 4 in any other common 
language. 

•  Option  B  Rank  8  and  literate  in  either  native 
language  or  common,  Rank  7  and  literate  in  the 
other of native language or common, Rank 1 in any 
other language. 

•  Option  C  Rank  9  and  literate  in  either  native 
language  or  common,  Rank  6  and  literate  in  the 
other of native language or common. 

Adventuring Skills 

A  character  starts  with Rank 0  in the  Adventuring 
skills  of  Horsemanship,  Climbing,  Swimming  and 
Stealth.  The  player  now  receives  1250  experience 
points that may be spent on improving these skills. 
Any experience points left over are lost. They also 
gain  Rank  0  Flying,  but  may  not  raise  it  at  any 
stage during Character Generation. 

The possible combinations are: 

Horsemanship, 
Climbing and 
Swimming  
(in any order) 

4, 0, 0 
3, 0, 0 
3, 2, 1 
2, 2, 0 
2, 2, 2 

Stealth 

0 
1 
0 
1 
0 

Mage or Non Mage? 

The  player  must  decide  whether  the character  will 
be a magic user or not. (This choice can be made at 
any time during character generation). 

Mage 

If  the  character  is  to  be  a  magic  user  then  the 
player  must  choose  a  college  of  magic  for  the 
character  to  belong  to.  Remember  that  there  is  a 
minimum  Magical  Aptitude  requirement  for  each 
college. 

College  

MA 

Naming Incantations  
Mind  
Fire  
Air  
Ice  
Illusion  
Celestial  
Earth  
Bardic  

1 
11 
12 
13 
13 
13 
14 
15 
16 

16 
E \& E  
Necromancy  
16 
Binding \& Animating   17 
18 
Water  
Witchcraft  
18 
The  character  now  receives  all  of  the  general 
knowledge  abilities  of  their  college  including  tal-
ents, general knowledge spells, general knowledge 
rituals,  both  counterspells,  the  purification  ritual 
and ritual spell preparation. 

The  player  should  list  these  on  their  character 
sheet. 

Non Mage 

If a player decides that their character will not be a 
magic  user  then  they  receive  6500  experience 
points to be spent in the following order: 

1. 2500 must be spent on either 1 point of Fatigue 
or 3 points of Perception. 

2. The character must acquire one new skill at rank 
2, and  may  acquire  a  second new  skill  at  no  more 
than rank 1. The Warrior skill may not be chosen at 
this time. 

3.  The  character  must acquire  one  weapon  at  rank 
2, and may acquire up to two weapons at no more 
than rank 1. 

4.  The  player  may  save  up  to  500  points  to  spend 
later.  The  player  must  spend  any  remaining points 
on any of:  

• 1 rank in any known adventuring skills  

• more ranks in any known languages  

• more perception. 

Any  remaining  points  (other  than  the  permissible 
500) are lost. 

Background Experience 

A  character  now  chooses  any  one  Artisan  skill  at 
Rank  0.  This  reflects  knowledge  gained  through 
childhood  and  must  be  appropriate  to  their  family 
background. 

They  also  receive  2500  Experience  Points  which, 
together  with  any  left  over  from  the  non-mage 
generation, can be spent freely. 

At this time the character may acquire any one new 
skill at Rank 0 for the cost of only 100 EP (rather 
than the usual cost). 

If  there  is  any  EP  remaining  it  may  be  saved  for 
spending later in the game. 

Background Possessions 

The  character  will  have  two  sets  of  clothing  of  a 
quality  appropriate  to  their  family  background. 
They  also  have  goods  up  to  the  value  of  500  sp 
which may be chosen from the Basic Price  List in 
the  Players  Guide.  Up  to  50  sp  may  be  saved  as 
cash. 

Modified Agility and Manual Dexterity 

The  player  should  calculate  any  agility  modifiers 
from  the  weight  of  their  possessions  and  any  ar-
mour penalties; see the Encumbrance Table (§58.1) 
for  values.  They  should  then  calculate  their  modi-
fied TMR from this value, see the table (§58.2) for 
values.  If  the  character  uses  a  shield,  they  should 
modify their Manual Dexterity as well. 

Finishing the Character 

The player must choose a name for their character. 

They should enter every piece of relevant data onto 
their  Character  Sheet,  and  calculate  base  chances 
and other variables. The generating GM will check 
it, and then sign \& date it as complete. 

\end{Chapter}
