\begin{College}[1.3]{celestial}{Celestial Magics}{CE}

The College of Celestial Magics is concerned with the practice of
those magic arts having to do with the elements of light and dark and
their contrasts.  There are four distinct divisions of the College of
Celestial Magics, each of which is concerned with a different
combination of light and dark.

\begin{tabular}{ll}
Solar Mages	& Light \\
Star Mages	& Light within Dark \\
Shadow Weaving	& Dark within Light \\
Dark Mages	& Dark  \\
\end{tabular}

Solar \& Star Mages use the element of Light; Shadow Weavers and Dark
Mages use the element of Darkness.

All members of the college must be associated with only one division
of this College, and may only change divisions by forsaking their
present division, and relearning the new division as if it were a
different college.

Most entities are aligned with either Light or Dark, and Celestial
Magics will often only affect entities of the opposed aspect.
Entities’ fear of their opposing element gives this College special
power.  Whether an entity is Light aspected, Dark aspected, or has
neither aspect, is determined per the rules given in Light and Dark
Aspect (§19.8).

\subsection{Restrictions}

Adepts of the College of Celestial Magics may not practise their arts
in an area where their element is not present.

A Magical Aptitude of 14 is required to learn this College. Note also
that certain spells of this college may only be learnt by specific
divisions (as listed after the Spell Number).

\subsection{Base Chance Modifiers}

The Base Chance of performing any talent, spell, or ritual of the
College of Celestial Magics is modified by the following numbers,
based on the division to which the adept belongs.

\subsubsection{Aspect Modifiers}

Solar Mage with a Solar Aspect 
Solar Mage with a Lunar Aspect 
Dark Mage with a Solar Aspect 
Dark Mage with a Lunar Aspect 

+1\%  
-1\%  
-1\%  
+1\% 

\subsubsection{Lighting Condition Modifiers}

An Adept of Celestial Magics is affected greatly by the lighting
conditions in their vicinity.  The bonuses and penalties gained in
this section apply only to non-magical forms of Light and Dark.
Magical forms of the elements may at best neutralise any penalties
suffered due to the natural elements.  For the purposes of these
modifiers, the vicinity is deemed to be any bounded area around the
Adept (such as a room) or, if the Adept is in the open, the area
within 30’ of the Adept.

Shadow Weavers must be within a shadow that has a defined edge within
the vicinity (the shadow must be large enough to contain the Adept,
and cannot be generated from the Adept’s possessions), and Star Mages
must be in direct light from point sources (eg. casting a shadow),
otherwise the lighting condition modifier is -25\\%.

The lighting modifiers are in the Celestial Lighting Modifiers table
(§19.9).

\section{Talents}

\begin{talent}[T-1]{Speak to Creatures of Light/Darkness}

\range{10 feet + 10 / Rank }
\duration{Immediate }
\multiple{50 }
\resist{None }
\begin{effects}
 This  talent  allows  the  adept  to  communi-
cate  in  a  limited  fashion  with  non-sentient  crea-
tures.  A  Solar  Mage  or  Star  mage  may  communi-
cate  with  those  creatures  who  are  light  aspected, 
whereas  a  Shadow  Weaver  or  Dark  Mage  may 
communicate  with  those  creatures  who  are  dark 
aspected.  The talent is limited to a range of 10’ (+ 
10’  /  Rank),  and  the  communication  is  equivalent 
to a language skill of 1 (+ 1 for every 5 full Ranks). 
The  communication  is  a  combination  of  spoken 
and  sign  language.  If  either  vision  or  sound is  not 
possible  then  the  talent  operates  at  half  its  actual 
Rank  (round  down).  Moreover,  if  neither  of  these 
senses are available then the talent cannot function 
at all. 
\end{effects}
\end{talent}

\begin{talent}[T-2]{Night Vision}

\range{50 feet + 10 / Rank }
\duration{Always active }
\multiple{100 }
\begin{effects}
This  talent  allows  the  adept  to  see  in  the 
dark with vision similar to that of a cat. Everything 
will  appear  monochromatic  (i.e.  shades  of  grey) 
and  it  is  difficult  to  accurately  estimate  distance. 
The higher the Rank, the less of a problem this will 
be. Because the vision is monochromatic it cannot 
be  used  to  do  a  Detect  Aura.  Note  that  some 
amount of light must be present before any sort of 
vision is possible. 
\end{effects}
\end{talent}

\begin{talent}[T-3]{Detect Aura }

\range{Special }
\duration{Immediate }
\multiple{75 }
\basechance{Perception + 5\% / Rank }
\resist{Active }
\begin{effects}
 The  effects  of  this  talent  are  described  in 
§9.1. 
\end{effects}
\end{talent}


\section{General Knowledge Spells}

\begin{spell}[G-1]{Blending }

\range{15 feet + 1 / Rank }
\duration{1 hour + 1 / Rank }
\multiple{50 }
\basechance{60\% }
\resist{None }
\storage{Investment, Potion, Ward }
\target{Entity }
\begin{effects}
Once this spell is cast, the target must remain still in order for it
to have effect.  While remaining still the target is not able to be
seen by non-magical means (i.e.  as for invisibility).  If the target
moves, the spell ceases to work. However, if the target becomes still
again during the duration of the spell, it will resume its effect.
The duration of the spell refers to the time since casting, not the
time that the spell is actually in effect (i.e.  while the target is
still).

Keeping still will require (as a minimum) a 4 × WP 
check  every  hour.  The  target  may  be  required  to 
make  additional  willpower  checks  at  the  GM’s 
discretion. 

The spell only has effect while the target is “still”. 
This means that the target is unable to move any of 
its external surfaces, with limited exceptions. Mov-
ing  an  external  surface  encompasses  such  actions 
as  moving  a  limb,  or  opening  and  closing  the 
mouth.  Blinking  and  normal  bodily  movement 
caused  by  normal  breathing  will  not  constitute 
moving  for  the  purposes  of  this  spell.  The  follow-
ing  actions  will  automatically  cause  the  spell  to 
cease  working:  talking,  spell  casting,  triggering 
(subject  to  any  revision  of  the  Investment  ritual) 
readying  a  weapon,  altering  facing  in  a  hex,  mov-
ing from the hex, using a silent language, or indeed 
any  Action  other  than  a  Pass  action  (and  Pass  ac-
tions  being  restricted  as  outlined).  Note  that  it  is 
not  relevant  if  an  observer  would  see  the  move-
ment  for  the  spell  to  cease  working  (e.g.  moving 
hands  behind  back,  or  talking  behind  hand  will 
both cause the spell to cease having an effect). 

\end{effects}
\end{spell}

\begin{spell}[G-2]{Light }

\range{15 feet + 15 / Rank }
\duration{15 minutes × [D - 5] × [Rank] }
\multiple{75 }
\basechance{50\% }
\resist{None }
\storage{Investment, Ward }
\target{Area }
\begin{effects}
The Adept creates a volume in which non-
magical  darkness  is  partially  suppressed.  The  vol-
ume  will  be  1000  (+  500  /  Rank)  cubic  feet,  and 
may  be  in  any  one  contiguous  area  the  Adept  de-
sires,  provided  that  no  dimension  is  smaller  than 
one  foot.  The  entire  volume  must  be  visible  and 
within  range  at  time  of  casting,  and  may  not  be 
moved.  For  visibility  purposes,  the  Spell  will  in-
crease Lighting levels within the volume to 60\% + 
2\%  /  Rank.  Rank  20  Light  may  not  be  seen 
through.  It  will  not  aid  in  providing  bonuses  for 
casting purposes, though it will neutralise penalties 
due to natural darkness, to a maximum of 5\% + 1\% 
/  Rank.  The  volume  counts  as direct  light  for  Star 
\&  Shadow  Mages.  If  the  lighting  conditions  are 
higher  than  that  provided  by  the  spell,  no  effect 
will  be  apparent.  Note  that  because  darkness  is 
being suppressed, no light is generated, so any area 
outside  the  volume  will  not  be  lit.  This  spell  can 
engender silhouettes, though not create shadows. If 
it is not possible to see into a lit volume, then ob-
jects within the volume are not visible. Any of this 
volume  may  be  overridden  by  a  higher  ranked 
Spell  of  Darkness,  or  neutralised  (back  to  original 
conditions) by an equal rank. 

\end{effects}
\end{spell}

\begin{spell}[G-3]{Darkness }

\range{15 feet + 15 / Rank }
\duration{15 minutes × [D - 5] × [Rank] }
\multiple{75 }
\basechance{50\% }
\resist{None }
\storage{Investment, Ward }
\target{Area }
\begin{effects}
The Adept creates a volume in which non-
magical  light  is  partially  suppressed.  The  volume 
will be 1000 (+ 500 / Rank) cubic feet, and may be 
in any one contiguous area the Adept desires, pro-
vided  that  no  dimension  is  smaller  than  one  foot. 
The entire volume must be visible and within range 
at  time  of  casting,  and  may  not  be  moved.  For 
visibility  purposes,  the  Spell  will  increase  Dark-
ness levels within the volume to 60\% + 2\% / Rank. 
Rank 20 Darkness may not be seen through. It will 
not  aid  in  providing  bonuses  for  casting  purposes, 
though  it  will  neutralise  penalties  due  to  natural 
light, to a maximum of 5\% + 1\% / Rank. The vol-
ume  counts  as  direct  shadow  for  Star  \&  Shadow 
Mages.  If  the  lighting  conditions  are  lower  than 
that provided by the spell, no effect will be appar-
ent.  Note  that  because  light  is  only  being  sup-
pressed, it may still pass through, and no shadows 
are generated outside the volume. If it is possible to 
see  through  a  Darkness,  everything  beyond  it  is 
normally  visible.  This  spell  can  engender  silhou-
ettes of lit objects against the darkness, though not 
create  shadows.  Any  of  this  volume  may  be  over-
ridden  by  a  higher  ranked  Spell  of  Light,  or  neu-
tralised  (back  to  original  conditions)  by  an  equal 
rank. 
\end{effects}
\end{spell}

\begin{spell}[G-4]{Shadow Form / Coruscade}

\range{15 feet + 1 / Rank }
\duration{30 minutes + 30 / Rank }
\multiple{150 }
\basechance{10\% }
\resist{None }
\storage{Investment, Ward, Potion }
\target{Entity }
\begin{effects}
 The  target  of  this  spell  is  enveloped  in  a 
confusing  pattern  of  either  shadows  (for  Dark  and 
Shadow)  or  coruscating  light  (for  Solar  and  Star), 
which  increases  their  defence  versus  physical 
Melee  or  Ranged  attacks  by  2  (+  2  /  Rank).  In 
Close  combat  only  1  (+  1  /  Rank)  is  gained.  No 
form  of  magical  vision  will  aid  in  avoiding  the 
defence bonus produced as a result of this spell, but 
any  attack  made  without  using  the  sense  of  sight 
(e.g. by a blind entity, a trample attack) will not be 
affected. It is usually quite apparent when an entity 
is under the effect of this spell. 
\end{effects}
\end{spell}


\begin{spell}[G-5]{Wall of Starlight }

\range{15 feet + 15 / Rank }
\duration{10 minutes + 10 / Rank }
\multiple{150 }
\basechance{15\% }
\resist{Passive }
\storage{Investment, Ward }
\target{Area }
\begin{effects}
Creates a 10’ high, 1’ thick, 20’ long wall 
of  light,  or  a  10’  high,  l’  thick,  5’  internal  radius 
ring  of  light,  or  a  15’  high,  5’  diameter  pillar  of 
light. 

The  adept  may  increase  any  dimension  by  1’  / 
Rank. The wall cannot be cast so as to include any 
entity  within  it,  other  than  the  Adept.  Any  entity 
who is dark aspected must passively resist or suffer 
[D - 5] ( + 1 / Rank) damage each time they come 
within  contact  with  the  wall  (not  per  pulse).  Any 
entity  damaged by  the  wall  must  roll  on  the  fright 
table.  The  wall  created  banishes  darkness  from 
within its bounds in the same manner as a Spell of 
Light  of  the  same  Rank.  The  entirety  of  one  edge 
must be affixed to a surface. This means that a wall 
can  be  created  with  a  smaller  dimension  than 
would otherwise be possible. For  example, casting 
a wall of light on a stepping stone that is 3’ square 
will  result  in  a  wall  which  is  only  3’  long.  Any 
edge  may  be  affixed  but,  for  the  purposes  of  this 
spell,  this  does  not  include  either  face.  For  exam-
ple, a wall could not be placed flat on a large open 
surface.  The  surface  or  surfaces  that  the  Wall  of 
Light  is  affixed  to  do  not  need  to  be  flat,  but  the 
length  of  the  wall  is  measured  from  the  deepest 
depression  on  the  surface  that  the  wall  fills.  For 
example,  a  circularly  concave  wall  of  5’  radius, 
with  a  rank  0  wall  affixed  to  it  will  end  in  a  flit 
edge  15’  beyond  the  end  of  the  curvature.  The 
entire anchoring edge must be visible to the Adept. 
The  wall  itself  cannot  be  moved.  Should  an  entire 
cross-section  of  the  last  remaining  anchoring  edge 
be  removed  then  the  wall  will  immediately  dissi-
pate. 

Example 
An  Adept  casts  a  wall  1’  off  the  ground, 
attached to a door. As soon as the door is opened the wall 
is  dissipated.  If,  however,  the  adept  had  cast  it  so  that  it 
overlapped the door frame, and it projected slightly above 
the top of the door, then it would not have been dissipated 
because  no  entire  cross-section  of  the  anchoring  edge  has 
been removed. In this case somebody with a sharp imple-
ment  (and  quite  a  bit  of  patience)  could  scratch  away  at 
the stone wall until they had created a groove through the 
entire  cross-section  of  the  Wall  of  Starlight  in  order  to 
make it dissipate. 

Note  that this  spell  will  not  be  affected  by  a  Spell 
of Darkness except to reduce its lighting effect. 

Solar and Star mages get a reduction to the Experi-
ence  Multiple  of  50  (to  100)  and  +5\%  to  base 
chance. 

\end{effects}
\end{spell}

\begin{spell}[G-6]{Wall of Darkness }

\range{15 feet + 15 / Rank }
\duration{10 minutes + 10 / Rank }
\multiple{150 }
\basechance{15\% }
\resist{Passive }
\storage{Investment, Ward }
\target{Area }
\begin{effects}
 This  spell  works  similarly  to  the  Wall  of 
Starlight Spell, except that light aspected creatures 
are affected by it, and it banishes light in the same 
manner as a Spell of Darkness of the same Rank. 

Shadow  and  Dark  mages  get  a  reduction  to  the 
Experience  Multiple  of  50  (to  100)  and  +5\%  to 
base chance. 
\end{effects}
\end{spell}

\begin{spell}[G-7]{Witchsight }

\range{15 feet + 15 / Rank }
\duration{30 minutes + 30 / Rank }
\multiple{150 }
\basechance{15\% }
\resist{None }
\storage{Investment, Ward, Potion }
\target{Entity }
\begin{effects}
 The  Adept  may  see  objects  or  entities 
which are invisible and they appear to have a slight 

blue  sheen  around  them.  If  the  invisibility  effect 
(excluding  Walking  Unseen)  is  of  a  higher  Rank 
than the Witchsight, the object or entity may not be 
clearly  identified  or  directly  magically  targeted. 
The  Adept  may  also  see  in  the  dark  as  a  Human 
does  on  a  cloudy  day,  with  an  effective  range  of 
vision of 150 feet under the open sky, and 75 feet 
elsewhere. 
\end{effects}
\end{spell}

\begin{spell}[G-8]{Walking Unseen }

\range{1 foot + 1 / Rank }
\duration{1 hour + 1 / Rank }
\multiple{100 }
\basechance{50\% }
\resist{None }
\storage{Investment, Potion, Ward }
\target{Entity }
\begin{effects}
 The  target  of  this  spell  may  move  unno-
ticed,  not  invisible.  This  means  that  it  will  not 
transmit light. As a consequence the target will cast 
a  shadow  (which  may  or  may  not  be  noticed  de-
pending on the lighting etc — even if noticed may 
not  be  connected  to  the  target)  and  have  a  reflec-
tion  in  a  mirror  (or  any  reflective  surface).  How-
ever  the  target  may  not be noticed  even  if  another 
entity  is  looking  directly  at  him/her.  It  should  be 
noted  that  a  crystal  of  vision  or  similar  would 
count  as  looking  directly  at  the  target,  not  as  a 
reflection.  An  entity  will  get  a  perception check if 
the target becomes invasive on that entity’s senses 
(e.g. standing in a frontal adjacent hex, or standing 
behind  the  entity  with  the  target’s  hands  over 
his/her eyes). Although the target is not invisible, it 
may  be  detected  using  any  magical  means  for 
detecting invisible entities (e.g. witchsight). 

If the target of the spell, or the target’s possessions, 
are  touched  by  another  entity,  or  that  entity’s  pos-
sessions, then the spell is broken. The target of the 
spell  may  not  break  it  voluntarily  (other  than  by, 
for example, touching another entity). Once broken 
the spell must be recast. 
\end{effects}
\end{spell}

\begin{spell}[G-9 Solar]{Resistance to Light}

\range{Self }
\duration{10 minutes + 10 / Rank }
\multiple{200 }
\basechance{15\% }
\resist{None }
\storage{Potion }
\target{Self }
\begin{effects}
 While  under  the  effects  of  this  spell,  an 
adept gains 2\% (+ 2 / Rank) to the chance of resist-
ing  magical,  light-based  attacks.  This  includes 
Flash  of  Light,  Wall  of  Starlight,  Bolt  of  Starfire, 
Web of Light, Solar Flare and Whitefire. The target 
will  also  become  fully  protected  from  damage 
caused  by  non-magical  light  (e.g.  sunburn,  snow-
blindness),  with  the  exception  that  it  will  not  pro-
tect  Greater  Undead  from  sunlight.  In  addition,  it 
allows vision in a Rank 20 Light Spell. Only Solar 
Mages may learn this spell. 
\end{effects}
\end{spell}

\begin{spell}[G-9 Star]{Illumination}

\range{15 feet }
\duration{10 minutes + 10 / Rank }
\multiple{200 }
\basechance{15\% }
\resist{None }
\storage{Investment, Ward }
\target{Object or area }
\begin{effects}
 This  spell  causes  a  1  inch  circle  on  any 
non-living surface to radiate light. The intensity of 
light  is  determined  by  Rank:  at  Ranks  0–5  it  is 
merely a glow; at Ranks 6–10 it is equivalent to the 
light of a candle; at Ranks 11–15 it is equivalent to 
the light of a torch; and at 16–20 it is equivalent to 
that  of  a  lantern.  Only  Star  Mages  may  learn  this 
spell.  It  will not aid in providing bonuses for cast-
ing purposes. 
\end{effects}
\end{spell}

\begin{spell}[G-9 Shadow]{Charismatic Aura}

\range{Self }
\duration{10 minutes + 10 / Rank }
\multiple{200 }
\basechance{15\% }
\resist{None }
\storage{Potion }
\target{Self }
\begin{effects}
This spell allows the adept to use shadows 
advantageously  to  influence  reaction  rolls.  The 
adept  can  choose  one  of  three  different  effects  at 
the  time  of  casting.  These  are:  to  appear  imposing 
or threatening, to appear alluring or seductive, or to 
appear  helpless  and  in  need  of  protection.  When 
used  in  appropriate  circumstances  these  effects 
modify  reaction  rolls  by  5\%  (+  1  /  Rank).  For 
example,  when dealing with an Orc Chief the first 
of the effects would probably be most beneficial. It 
is  very  difficult  to  perceive  that  the  spell  is  in  ef-
fect. Only Shadow Weavers may learn this spell. 
\end{effects}
\end{spell}


\begin{spell}[G-9 Dark]{Strength of Darkness}

\range{15 feet }
\duration{10 minutes + 10 / Rank }
\multiple{200 }
\basechance{15\% }
\resist{None }
\storage{Investment, Ward, Potion }
\target{Entity }
\begin{effects}
The target’s Physical Strength is increased 
by  1  (+  1  for  every  2  (or  fraction)  Ranks)  for  the 
duration  of  the  spell.  This  spell  may  only  be  cast 
by Dark Mages when they are in an area of at least 
60\% Darkness. 
\end{effects}
\end{spell}


\section{General Knowledge Rituals}

\begin{ritual}[Q-1]{Reading the Night Sky }

\multiple{150 }
\basechance{MA + 4\% / Rank }
\casttime{1 hour }
\begin{effects}
 The  Adept  may  read  something  of  the 
future  by  performing  this  ritual.  The  ritual  may 
only  be  performed  from  a  vantage  point  with  a 
clear view  of the sky (not indoors or in a hollow), 
and  it  must  be  a  clear  night.  The  GM  rolls  for 
success  or  failure.  The  GM  provides  the  answers 
writ  in  the  stars  in  the  form  of  generalised  state-
ments.  If  a  successful  roll  occurs  the  statements 
should  be  generally  accurate.  If  a  failure  occurs 
then  nothing  is  read.  If  a  backfire  occurs  then  the 
statements should be misleading. 
\end{effects}
\end{ritual}


\begin{ritual}[Q-2]{Summoning and Binding Creatures of Light / Darkness}

\multiple{200 }
\basechance{20\% + 4\% / Rank }
\casttime{1 hour }
\begin{effects}
The adept may summon and bind 1 (+1 for 
every  5  or  fraction  Ranks)  non-sentient  creature 
whose aspect is the same as the aspect of the divi-
sion  of  the  college  to  which  the  Adept  belongs. 
Any creature summoned must be native to the area. 
If the ritual succeeds, the creature will arrive bound 
to  the  Adept.  In  this  state  the  creature  will  try  to 
protect and aid the adept to the utmost of its ability 
(but it does not automatically know what the adept 
wishes  it  to  do).  If  the  ritual  backfires  then  the 
creature  will  arrive  and  immediately  attack  the 
adept.  The  creature  will  arrive  after  (20  -  D10  - 
Rank)  minutes  (minimum  of  0).  Bound  creatures 
will continue to serve the Adept as long as passive 
concentration  is  maintained  (the  Adept  stays  con-
scious  and  does  not  attempt  any  other  spell  that 
requires concentration). If the Adept is stunned, a 3 
×  Willpower  (  +  2  /  Rank)  Willpower  check  is 
required  to  maintain  concentration.  If  the  concen-
tration  is  broken  the  creature  will  immediately 
attack  the  Adept.  The  Adept  may  at  any  time  re-
lease  any  of  the  creatures,  in  which case  the  crea-
ture  concerned  will  immediately  flee  from  the 
Adept’s presence.  

Creatures  that  may  be  summoned  using  this  ritual 
are 
those  appropriately  aspected,  non-sentient 
beings from the following categories: 66.2 Felines, 
66.4 Small Land Mammals, 67.1 Common Avians, 
69.1  Lizards  and  Kindred  (except  Hydras),  69.2 
Snakes, 69.3 Insects and Spiders, and 72 Creatures 
of  Night  and  Shadow.  Note:  Weres  can  only  be 
affected  by  this  ritual  while  they  are  in their  beast 
form. 
\end{effects}
\end{ritual}


\section{Special Knowledge Spells}

\begin{spell}[S-1]{Healing }

\range{Touch }
\duration{Immediate }
\multiple{200 }
\basechance{40\% }
\resist{None }
\storage{Investment, Potion }
\target{Entity }
\begin{effects}
Cures 1 point of Endurance or Fatigue ( + 
1  /  every  2  or  fraction  Ranks).  This  spell  will  not 
cure  specific  Grievous  Injuries  and  the  extra  en-
durance  points  associated  with  any  Grievous  In-
jury, nor will it “cure” tiredness fatigue (including 
that  lost  due  to  spell  casting).  However,  this  spell 
can cure as if the curing was attempted by a healer 
of  Rank  equivalent  to  the  Rank  of  the  spell  /  5 
(round down). 
\end{effects}
\end{spell}

\begin{spell}[S-2]{Creating Light/Dark Sword}
\range{15 feet + 5 / Rank }
\duration{5 minutes + 1 / Rank }
\multiple{250 }
\basechance{30\% }
\resist{None }
\storage{Investment }
\target{Object }
\begin{effects}
The Adept may cause any sword (as listed 
on  the  weapons  chart)  within  range  to  become  a 
sword of the Adept’s element (i.e. light for Solar or 
Star  Mages,  and  Dark  for  Shadow  Weavers  or 
Dark  Mages).  The  sword  will  then  have  its  Strike 
Chance increased by 1\% (+ 1 / Rank) and its Dam-
age increased by 1 (+ 1 / every 3 or fraction Ranks) 
whenever it is used against a creature of the oppo-
site  element  (i.e.  dark  aspected  for  a  Light  Sword 
and  light  aspected  for  a  Dark  Sword).  Light 
Swords  sparkle  with  small  white  sparks,  and  dark 
swords appear blacker than black. 
\end{effects}
\end{spell}

\begin{spell}[S-3]{Bolt of Starfire }

\range{40 feet + 15 / Rank }
\duration{Immediate }
\multiple{200 }
\basechance{35\% }
\resist{Active, Passive }
\storage{Investment, Ward, Magical Trap }
\target{Entity, object or volume }
\begin{effects}
 The  Adept  casts  a  bolt  of  starfire  towards 
any  target  within  range.  The  target  may  be  a  vol-
ume  of  air.  The  first  entity  or  object  that  the  bolt 
hits along its flight path suffers [D - 4] (+ 1 / Rank) 
damage  unless  the  target  successfully  resists.  If 
fired at figures in Close Combat the bolt will hit a 
random target (based on their relative sizes). 
\end{effects}
\end{spell}

\begin{spell}[S-4]{Meteorite Shower }

\range{60 feet + 15 / Rank }
\duration{Delayed effect }
\multiple{200 }
\basechance{10\% }
\resist{Passive }
\storage{Investment, Ward, Magical Trap }
\target{Area }
\begin{effects}
 The  adept  calls  down  from  the  heavens  a 
meteorite  shower  which  peppers  a  given  area  that 
was entirely within the Adept’s range at the time of 
casting.  The  meteorite  shower  is  targeted  to  hit  a 
specific  hex  and  takes  2  minutes  (-10  seconds  / 
Rank) to arrive (minimum of the end of the follow-
ing  pulse).  Any  entities  within  a  vertical  column 
that  is  25  feet  in  diameter  (centred  on  the  target 
hex), with a height equal to the spell’s range) must 
resist  or  suffer  [D  -  4]  (+1  /  Rank)  damage.  The 
Adept may counterspell this spell at any time prior 
to  the  meteorite  shower  arriving  by  casting  the 
appropriate  counterspell  at  the  targeted  hex.  The 
targeted  hex  will  have  a  magical  aura  until  the 
meteorite  shower  arrives  (or  is  counterspelled  by 
the  Adept).  This  spell  will  have  no  effect  if  it  is 
targeted  on  a  hex  which  is  already  a  target  of  this 
spell.  Note  that  a  solid  surface  (such  as  10’  of 
earth)  will  prevent  the  meteorite  shower  from 
reaching its target hex. 
\end{effects}
\end{spell}

\begin{spell}[S-5]{Star / Shadow Wings}

\range{10 feet + 10 / Rank }
\duration{30 minutes + 30 / Rank }
\multiple{250 }
\basechance{25\% }
\resist{None }
\storage{Investment, Ward }
\target{Sentient Entity }
\begin{effects}
 The  target  of  this  spell  receives  wings 
comprising  of  the  element  of  the  Adept  (i.e.  Light 
if  a  Solar  or  Star  Mage,  and  Dark  if  a  Shadow 
Weaver or Dark Mage). These wings will carry the 
target,  and  anything  that  the  target  can  carry,  at  a 
speed  of  30  (+1  /  Rank)  miles  per  hour.  (NB  :  1 
mph  approximately  1.5’  /  sec  =  1.5  hexes  /  pulse. 
During  a  takeoff  or  landing  half  that  distance  will 
be  travelled).  The  wings  have  a  wingspan  of  30’ 
and are insubstantial. If the wings come in contact 
with  an  object  they  will  cease  to  work  until  they 
can once more spread unhindered (e.g. 30’ of open 
ground  is  usually  necessary  in  order  to  start  using 
them).  Note  that  normal  precipitation  (i.e.  rain, 
mist,  snow  and  hail)  will  not  cause  the  wings  to 
cease  functioning.  The  Wings  will  not  become 
invisible  or  unseen  if the  wearer  does.  Star Wings 
are clearly visible at night and barely visible during 
the  day,  and  Shadow  Wings  are  clearly  visible 
during  the  day  and  barely  visible  at  night.  During 
the last 5 seconds of the duration of the wings they 
will  automatically  try  to  land.  There  is  no  earlier 
warning of the end of duration of the  wings. Only 
sentient creatures can control the wings. 

Since  shadow  wings  are  made  of  shadow  they  are 
clearly  insubstantial  and  hence  can  be  worn  in 
confined  spaces.  However,  for  them  to  be  able  to 
be  used,  they  require  to  be  properly  and  fully  ex-
tended, that is shaped, and will fly at full speed or 
not at all. It requires 1 pulse to start and 1 pulse to 
stop.  They  will  only  fly  a  humanoid character  and 
characters  of  human  size.  That  is  taken  to  mean 
characters  of  3  hexes  or  less.  They  will  carry  a 
character plus the character’s normal encumbrance. 
\end{effects}
\end{spell}

\begin{spell}[S-6]{Web of Light / Darkness}

\range{30 feet + 15 / Rank }
\duration{ Concentration  (Maximum  of  15  minute + 15 / Rank)}
\multiple{250 }
\basechance{25\% }
\resist{Passive }
\storage{Investment, Ward, Magical Trap }
\target{Area }
\begin{effects}
A five foot wide web of the element of the 
Adept  is  projected  from  the  finger  tips  to  a  target 
hex,  object  or  entity.  Any  entities  not  aspected  to 
the  element  of  the  web,  and  all  objects,  are  en-
snared  by  the  web.  The  web  may  only  ensnare  a 
number  of  entities  equal  to  the  Adepts  rank,  so  it 
will  stop  at  the  hex  at  which  this  limit  is  reached 
(or  at  maximum  range).  Entities  ensnared  in  the 
web  suffer  [D  -  2]  (+1  /  Rank)  damage  (halved  if 
they  successfully  resist)  each  pulse  that  they  re-
main in the web after the first. The damage is done 
at the end of each pulse. Any ensnared entity must 
roll 1 × Physical Strength (2 × if they successfully 
resisted)  in  order  to  move  themselves  to  an  adja-
cent  hex  (which  may  be  free  of  the  web),  or  to 
perform an action within the web. A similar check 
is  required  for  any  entity  (regardless  of  aspect) 
attempting to remove an object from the web. If an 
entity  receives  aid  in  removing  themselves  from 
the  web,  the  PS  of  the  aiding  character  may  be 
combined  with  their  own.  Five  or  more  points 
damage from a single blow from a B-class weapon 
will  destroy  the  entire  web.  Treat  the  web’s  de-
fence  as  being  equal  to  its  Rank.  Entities  of  the 
same aspect as the element of the web may ignore 
its  effects,  but  consequently  may  not  affect  the 
web.  They  may  aid  other  character  in  getting  free 
of  the  web.  Any  dropped  object  will  become  en-
snared by  the  web,  as  will  any  entity  not  aspected 
to  the  element  of  the  web  who  comes  into  contact 
with it (up to the limit of the web). 

\end{effects}
\end{spell}

\begin{spell}[S-7]{Fear }

\range{15 feet + 15 / Rank }
\duration{15 seconds + 15 / Rank }
\multiple{350 }
\basechance{20\% }
\resist{Active, Passive }
\storage{Investment, Ward, Magical Trap }
\target{Entity }
\begin{effects}
The target of this spell is seized by uncon-
trollable  fear  and  must  roll  on  the  Fright  table.  At 
the time of casting, the Adept may choose to mod-
ify the Fright Table roll up or down by an amount 
up  to  the  rank  of  the  spell.  On  a  double  or  triple 
effect  this  modification  may  be  doubled  or  tripled 
respectively. 

\end{effects}
\end{spell}

\begin{spell}[S-8]{Increased Gravity }

\range{60 feet + 15 / Rank }
\duration{Concentration: maximum 1 minute + 1 / }
Rank 
\multiple{450 }
\basechance{2\% }
\resist{Active, Passive }
\storage{Investment }
\target{Entity }
\begin{effects}
 The  spell  causes  a  target  of  the  Adept’s 
choice which is within range to suffer the effects of 
an  increase  in  gravity  unless  they  successfully 
resist.  This  increase  in  gravity  subtracts  2  (+2  / 
Rank) from the target’s strength and 1 for every 2 
Ranks  (or  fraction)  from  the  target’s  TMR.  The 
target  must  roll  under  3  ×  strength  each  pulse  or 
become immediately prone. Once prone, a roll of 1 
×  modified  strength  is  required  in  order  to  stand 
up.  If  the  target’s  strength  is  reduced  to  less  than 
zero,  the  target  suffers  the  negative  amount  as 
damage each pulse and must roll under Willpower 
+ current Endurance in order to remain conscious. 
If  the  target  and  Adept  become  separated  by  a 
distance greater than the range of the spell then the 
spell  immediately  ceases  to  work.  If  the  target  is 
under (or comes under) the effect of a flying spell, 
the following applies:  

•  If  the  rank  of  the  flying  spell  is  greater  than  the 
rank  of  the  Increased  Gravity  then  the  target  may 
be  able  to  fly.  However,  twice  the  rank  of  the  In-
creased  Gravity  is  subtracted  from  the  rank  of  the 
flying spell for purposes of determining speed and 
lift of the flying spell. This may make it a negative 
modifier which may reduce the speed to 0 or  less, 
in which case the target may not fly but may stand.  

• If the rank of the flying spell is less than or equal 
to  the  rank  of  the  Increasing  Gravity,  then  the 
target  may  not  fly.  However,  half  the  rank  of  the 
flying  spell  is  subtracted  from  the  rank  of  the  In-
creased  Gravity  for  purposes  of  determining  the 
strength and TMR reductions. 

\end{effects}
\end{spell}

\begin{spell}[S-9]{Whitefire }

\range{30 feet + 15 / Rank }
\duration{Immediate }
\multiple{500 }
\basechance{1\% }
\resist{Active, Passive }
\storage{Investment, Ward, Magical Trap }
\target{Entity }
\begin{effects}
 The  target  of  this  spell  must  resist  or  be 
instantaneously subjected to the heat of the interior 
of  a  star,  causing  death.  The  target’s  body  is  a 
blackened husk, their skin needs to be regenerated 
and their chance of resurrection is reduced by 2\% / 
Rank.  If  the  target’s  Willpower  is  greater  than  or 
equal to the cast chance then the target will not be 
affected. Protection from magical fire will not help 
against this spell. 
\end{effects}
\end{spell}

\begin{spell}[S-10 solar]{Solar Flare}

\range{75 feet + 15 / Rank }
\duration{Delayed effect }
\multiple{500 }
\basechance{5\% }
\resist{Passive }
\storage{Investment, Ward, Magical Trap }
\target{Area }
\begin{effects}
 The  Adept  calls  down  an  incandescent 
lance of sunlight which blasts an area of 1 hex (+ 2 
hexes  /  4  full  ranks)  in  diameter.  All  non-sentient 
flora  is  immediately  withered  and  charred.  Any 
entities  within  the  area  must  resist  or  suffer  [D  + 
10]  damage  (+  1  /  2  or  fraction  ranks).  The  flare 
takes 60 seconds (-5 / Rank) to arrive (minimum of 
5 seconds). The flare  will always arrive at the end 
of  a  pulse  and  during  that  pulse  the  area  will  be 
brightly illuminated (that is 99\% Light). This spell 
may only be cast when the sun is in the sky. Only 
Solar Mages may learn this spell. 
\end{effects}
\end{spell}

\begin{spell}[S-10 Star]{Falling Star}

\range{75 feet + 15 / Rank }
\duration{Delayed effect }
\multiple{500 }
\basechance{1\% }
\resist{Passive }
\storage{Investment, Ward, Magical Trap }
\target{Area }
\begin{effects}
 The  Adept  calls  from  the  sky  a  meteor 
which crashes into a given area that was within the 
Adept’s range at the time of casting. The meteor is 
targeted  to  hit  a  specific  hex  and  takes  5  minutes 
(20  seconds  /  Rank)  to  arrive  (minimum  of  5  sec-
onds). The meteor  will always arrive at the end of 
the pulse and will be preceded during that pulse by 
a  high-pitched  whistle  in  the  general  area.  Any 
entities within the target hex suffer [D + 12] (+ 4 / 
Rank)  damage.  Entities  within  adjacent  hexes 
suffer  [D  +  2]  (+  1  /  Rank)  damage.  If  an  entity 
successfully  resists  it  suffers  only  half  damage 
(round  up).  The  Adept  may  counterspell  this  spell 
at any time prior to the meteor arriving by casting 
the  appropriate  counterspell  at  the  targeted  hex. 
The targeted hex will have a magical aura until the 
meteor arrives (or is counterspelled by the Adept). 
This  spell  will  have  no  effect  if  it is targeted  on  a 
hex  which  is  already  a  target  of  this  spell.  Only 
Star Mages may cast this spell. 
\end{effects}
\end{spell}

\begin{spell}[S-10 Dark]{Blackfire}

\range{30 feet + 5 / Rank }
\duration{Immediate }
\multiple{350 }
\basechance{5\% }
\resist{Passive }
\storage{Investment, Ward, Magical Trap }
\target{Area }
\begin{effects}
 From  the  Adept’s  fingertips  erupts  a  col-
umn  of  black  flames  which  travel  to  the  extent  of 
the  spell’s  range,  and  is  5  feet  wide.  The  Adept 
may  increase  the  width  by  one  foot  per  rank.  All 
entities  occupying  hexes  through  which  the  fire 
passes must resist or suffer [D - 2] ( + 1 per Rank) 
damage.  Entities  who  are  damaged  by  this  spell 
have their base chance of infection increased by 20 
(+ 2 / Rank). Any entities wholly within the area of 
the  fire  must  also  roll  under  1  ×  Willpower  (2  × 
Willpower  if  they  successfully  resisted)  or  suffer 
the  effects  of  a  roll  on  the  fright  table.  Protection 
from magical  fires  will  not  help  against  this  spell. 
Only Dark Mages may learn this spell. 
\end{effects}
\end{spell}

\begin{spell}[S-10 Shadow]{Shadow Walking}

\range{Self }
\duration{Immediate }
\multiple{550 }
\basechance{1\% }
\resist{None }
\storage{Ward, Potion }
\target{Self }
\begin{effects}
 The  Adept  may  instantly  teleport  from 
within one shadow to another shadow. The destina-
tion  must  be  within  sight  or  must  have  been  care-
fully  memorised  beforehand.  The  destination  may 
be up to 5 miles (+ 1 / Rank) distant. Only Shadow 
Weavers may cast this spell. 


\end{effects}
\end{spell}

\section{Special Knowledge Rituals}

\begin{ritual}[R-1]{Conjuring and Controlling Light / Dark Sphere}

\duration{Concentration: Maximum 5 minutes + 5 / Rank }
\multiple{450 }
\basechance{1\% + 3\% / Rank }
\casttime{1 hour }
\begin{effects}
 The  Adept  may  summon  a  12”  (+  1”  / 
Rank) diameter sphere comprised of the element of 
their division of the college. Solar and Star Mages 
summon  a  Light  Sphere  which  is  as  bright  as  the 
sun  and  coruscates  with  sparks  of  light.  Dark 
Mages and Shadow Weavers create a Dark Sphere 
which is inky black and seems to suck light into it. 
If  the  ritual  is  successful  the  sphere  will  appear 
within 15’ of the Adept under the Adept’s control. 
Active  concentration  is  required  to  move  the 
sphere,  which  may  move  at  up  to  6  (+  1  /  4  full 
Ranks) TMR. Once the Adept stops concentrating, 
or  the  duration  of  5  minutes  (+  5  /  Rank)  is 
reached,  the  sphere  will  return  to  its  own  dimen-
sion.  If  the  Adept  fails  to  summon  the  sphere, 
nothing  happens.  If  a  backfire  results  (the  Cast 
Check is more than 30 above the Cast Chance), the 
sphere  appears,  but  is  not  under  control  so  will 
move  randomly  about  at  maximum  TMR.  Any-
thing  that  comes  into  contact  with  a  Dark  Sphere 
must resist or be immediately sucked into oblivion. 
Anything  that  comes  into  contact  with  a  Light 
Sphere must resist or be immediately reduced to a 
pile  of  ashes.  An  entity  that  resists  simply  suffers 
D10  damage  and  is  thrown  to  the  ground  by  the 
shock  of  contact.  It  is  believed  that  if  a  Light 
Sphere  comes  in  contact  with  a  Dark  Sphere  a 
cataclysmic  explosion  results  (however  there  are 
no known witnesses to such an event). 
\end{effects}
\end{ritual}


\section{Light and Dark Aspect}

Most  creatures  are  either  Light  or  Dark  aspected, 
depending on whether they are nocturnal or diurnal 
(active during the day). There is no direct connec-
tion  between  the  possession  of  a  Light  or  Dark 
aspect  and  the  self-styled  “Powers  of  Light  and 
Darkness”.  Entities  of  Light  aspect  are  not  necessarily  “good”  nor  entities  of  Dark  aspect  “evil”. 
The aspect refers only to the Entity’s position with 
regard to the Elements of Light and Dark. It is also 
differs  from  an  entity’s  astrological  Aspect  (see 
§1.4), but may be influenced by it. 

To determine if an Entity is aspected with either of 
the elements of Light or Dark, follow these rules:  

•  If  the  entity  is  a  Celestial  Mage  then  this  deter-
mines their aspect: Solar and Star Mages are Light 
aspected;  Shadow  and  Dark  Mages  are  Dark  as-
pected.  

• If the entity is not a Celestial Mage, and is Lunar 
aspected,  then  they  are  Dark  aspected.  Note  that 
Shapechangers are Lunar aspected.  

• If the entity is not a Celestial Mage, and is Solar 
aspected, they are Light aspected.  

• If none of the above applies, then an entity’s race 
or type may mean that they are aspected with either 
Light  or  Dark.  Races  or  creatures  that  are  noctur-
nal,  crepuscular  (active  at  twilight)  or  live  pre-
dominantly  underground  are  Dark  aspected  (Alu-
sian  examples:  Dwarves,  Orcs,  most  cats,  bats, 
wolves).  Races  and  creatures  that  are  diurnal,  and 
who  do  not  live  underground  are  Light  aspected. 
Golems, Elementals, and Beings of Mana or Spirit 
(or their manifestations) are neither Light nor Dark 
aspected. 
\end{College}

\begin{table*}
\section{Celestial Lighting Modifier Table}

\begin{tabularx}{\linewidth}{llllllXX}
Light	& Darkness	& Solar	& Dark	& Shadow	& Star	& Natural Lighting		& Artificial Lighting \\
0\%	& 100\%		& -	& +25	& -		& -	& Pitch Blackness		& Magical Effect – no vision works \\
1\%	& 99\%		& -25	& +25	& -10		& +5	& Midnight in a storm		& Underground, no lights \\
5\%	& 95\%		& -20	& +20	& 0		& +15	& Overcast night		& Single Candle Underground \\
10\%	& 90\%		& -15	& +15	& +10		& +25	& New Moon, Moonless night	& 1 Torch Underground, Window less room in day \\
20\%	& 80\%		& -10	& +10	& +20		& +20	& Night with crescent Moon + stars	& 1 Lantern Underground	\\
30\%	& 70\%		& -5	& +5	& +25		& +15	& Night in a town		& Campfire at night, Shuttered room in day \\
40\%	& 60\%		& -5	& +5	& +15		& +10	& Night of Full Moon		& Torch-lit Underground	\\
50\%	& 50\%		& 0	& 0	& +10		& +5	& Twilight, Major Storm		& Inside on overcast day \\
60\%	& 40\%		& +5	& -5	& +5		& 0	& Bright day in a forest, Solid rain	& Lamp-lit Interior \\
70\%	& 30\%		& +10	& -10	& 0		& -5	& Overcast, Mist, Light rain	& Brightly lit Underground \\
80\%	& 20\%		& +15	& -15	& -5		& -10	& Autumn Morning, Light cloud	& Brightly lit Interior	\\
90\%	& 10\%		& +20	& -20	& -10		& -15	& Bright sunny afternoon	& - \\
95\%	& 5\%		& +20	& -20	& -15		& -20	& Noon				& - \\
99\%	& 1\%		& +25	& -25	& -20		& -25	& Noon in a desert		& - \\
100\%	& 0\%		& +25	& -	& -		& -	& –				& Magical Effect – no vision works \\
\end{tabularx}

- Adept cannot cast under these conditions. 
\end{table*}
