\begin{Skill}[1.1]{merchant}{Merchant}

Since adventurers are highly talented individuals who often risk their
lives, and a person is usually compensated for the value of the work
they do, the player characters will fare better than most
economically.  A merchant character, blessed with the ability to earn
even more Silver Pennies, has the best of all worlds.  Their business
acumen enables them to command a stiff price for those goods they
vend, and to acquire that which they covet at bargain rates.  The
merchant is not often fooled in monetary matters, for they can be an
expert in evaluating the worth of rare and costly goods.

The economies of most DragonQuest worlds do not promote the growth of
capitalism.  Basically, the nobility has a vested interest in all
rural lands, which comprise the vast majority of human-settled areas.
An ambitious, dynamic merchant could perhaps own the entirety of a
large town, but it is quite likely that a jealous duke or prince would
twist justice to break the merchant’s power. Therefore, it behooves a
merchant to cultivate powerful allies when their holdings burgeon.

\section{Restrictions}

A merchant must be able to read and write in at least three languages
at Rank 6 in order to use their assaying ability.

\section{Benefits}

\precis{The merchant’s ability to buy and sell a particular item is dependent
upon its type.}

Any item will be classified as one of three types: common, uncommon,
and rare or costly.  Items listed in the Players’ Handbook are of the
common type. Jewellery set with semiprecious stones, spices from
another continent, and fine paintings are examples of the uncommon
type.  Rare and costly items include magic-invested objects, diamonds,
roc’s eggs, giant slaves, etc. The GM must classify each item with
which a merchant wishes to deal.

\precis{A merchant can purchase items at a cost cheaper than the asking price.}

If the result is odd, the quote is below the actual asking price; if
even, it is above.

\begin{dqtblr}{colspec={XX}}
Item Type	& Discount to Merchant  \\
Common		& [5 × Rank]\% \\
Uncommon	& [2 × Rank]\% \\ 
Costly or Rare	& [1 × Rank]\% \\
\end{dqtblr}

If the GM is actively playing the role of the seller, or another
player is the seller, the merchant must do their own haggling.  There
will also be those items which the vendor cannot afford to sell at the
usual discount to the merchant.  The GM should use their discretion
here.

\precis{A merchant may mark up the price of an uncommon or rare item.}

A merchant can gain (1.5 × Rank)\% above the value of an uncommon item
they are selling.  They can gain (0.5 × Rank)\% above the value of a
costly or rare item they are selling.

\precis{A merchant can assay an item to determine its exact worth.}

The player characters will generally receive a fair quote on the price
of basic goods, but must accept the word of the being with whom they
are dealing when conducting a transaction involving uncommon, rare or
costly items.  The odds of the player characters being bilked increase
as they venture forth from their native land(s). However, if a
merchant is amongst them, they can assay the value of any item after
(11 - Rank) minutes.

The success percentage for assaying a common item is equal to the
merchant’s (Perception + 12 × Rank)\%, to assay an uncommon item equal
to (Perception + 9 × Rank)\%, and to assay a rare or costly item equal
to (Perception + 6 × Rank)\%.  If the GM’s roll is equal to or less
than the success percentage the merchant is told the exact value of
the item in question.  If the roll is greater than the success
percentage, the GM’s quote increasingly diverges from reality as the
result approaches 100.

A merchant may use their skill to affect transactions involving up to
(250 + 50 × Rank Squared) Silver Pennies per month, or a single
transaction of any amount.

The merchant must buy and sell at the asking price for any
transactions over their monthly limit.

A merchant can specialise in a specific category of item assaying for
every three full ranks.

The merchant chooses their speciality from the following list (and any
the GM should add):

\begin{Enumerate}

\item Ancient Writings 
\item Antiques 
\item Archaeological Finds 
\item Art 
\item Books 
\item Gems 
\item Jewellery 
\item Land
\item Magic Items 
\item Monster and Animal Products (e.g. furs, eggs) 
\item Precious Metals 
\item Slaves 

\end{Enumerate}

When a merchant assays an item of a category in which they specialise,
they add (2 × Rank)\% to their success percentages.  It is possible
for a merchant to attain a 100\% chance of accurately pricing a
speciality item (exception to 90\% + Rank limit).

If a merchant wishes to add additional specialities without increasing
in rank, they must expend 4,000 Experience Points and 4 weeks of
training per speciality.  These costs are discounted by 25\% if the
merchant has reached rank 8, or by 50\% if they have reached rank 10.

\end{Skill}
