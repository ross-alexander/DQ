\begin{Chapter}{The College of Rune Magics (Ver 2.2)}

\textbf{College in Playtest}

The Rune College is currently in play test, and this is the test
version in use at time of publication.  Significant changes from this
version are expected to happen at irregular intervals. Check
http://www.dq-nz.org/dqwiki/index.php?title=Rune for the latest
version.

All characters that join this College do so under the understanding
that it may be withdrawn or radically changed.  Contact a member of
the Character tribunal for advice before taking this College.

The College of Rune Magics is concerned with the use of special
symbols of power to shape mana into desired forms.  A Rune is a
graphic symbol representing some actual, elemental, or mystical force.
In rare cases, additional Runes may be developed or discovered which
employ parts of existing Runes. However, much of the power of the
Runes derives from their constant usage over many centuries, and most
useful Runes will be known to all Adepts of this College (or at least
be readily available to them with very little research). It is
believed that the origins of Runes come from the original written
script of the dragons. As the dragons investigated the world they
attempted to codify this knowledge as written symbols.  Ages later
early mortals discovered fragments of these writing.  From these
discoveries, prehistoric shamans developed primitive magic to give
them simple power over the world around them.


\section{Totem Animals}

As shamans, primitive Rune mages often chose totem animals to aid and
guide them. This binding of their spirit to that of their totem enable
them to sense when their totem animal is in the vicinity and their
totem will never make an unprovoked attack on the Adept.  Using the
Spell of Summon Totem Helper the Adept may gain assistance from the
Totem spirits.


\section{Rune Wands and Staves}

A wand is defined as a length of wood or bone one foot long. It cannot
be used in combat. It has negligible weight.  A staff is defined as a
quarterstaff in terms of weight, length and damage.

\subsection{Materials}

\begin{tabularx}{\columnwidth}{llX}
Material	& Area		& Bonus (+5\% to) \\
Willow		& Healing	& healing spells  \\
Poplar		& Divination	& runes of sight \\
Bone/Ivory	& Control	& control spells \\
Pine		& Creation	& rune wall, weapon \\
Elm		& Warning	& purification, warding \\
Beech		& Spirit	& spirit spells (-5\% to others) \\
Oak		& Strength	& stores extra Ft \\
Ash		& Destruction	& elemental \\
Blackthorn	& Curse		& curses \\
Redwood		& Travelling	& sending, visitation etc. \\
\end{tabularx}

\section{Restrictions}

Adepts of the College of Rune Magics may use their talent magic
without restriction.  Many spells require inscribing the appropriate
Rune on a surface or item to be enchanted. This location is indicated
(as ‘Rune:’) in the spell’s description, and full details are given
under the spell’s ‘Effects:’.

In order to write the Rune, the Adept may use any substance that will
mark the surface of the object to be enchanted.  Any tool may be used
to carve a Rune into a substance, so long as the tool is hard enough
to do the job and it is not composed of Cold Iron.

The MA requirement for this college is 14. 


\section{Ritual Casting}

Some spells may be ritually cast.  The spell is cast as a ritual,
taking at least one hour.  The adept spends the same amount of fatigue
as they would if the spell was cast normally.


\section{Base Chance Modifiers}

The Base Chance of performing a talent, spell, or ritual of the
College of Rune Magics is modified by the addition of the following
numbers:
\begin{tabularx}{\columnwidth}{Xl}
Adept takes a minute to inscribe a Rune on a surface & +5 \\
Adept uses their own blood to inscribe a Rune on a surface (1 pt tiredness FT, minimum 1 minute) & +5 \\
Adept employs Ritual Spell Preparation or Casting (maximum 10 hours) & +5  + (5 / hr) \\
Adept uses fresh Dragon’s blood to inscribe a Rune  & +50 \\
Adept uses a wand or staff  & as material table \\
\end{tabularx}

All modifiers are cumulative. 

\section{Talents}

\begin{talent}[T-1]{Interpret Runes and Symbols}

\range{5 foot + 1 / Rank}
\multiple{150}
\basechance{MA + PC + 3\% / Rank}
\begin{effects}
This talent allows the adept to divine the meaning of any symbols,
maps or writings etc which are in range and can be clearly seen.  This
will supply vague definitions about the piece of information.  It may
be only attempted once per piece of information (GMs discretion).  If
a double effect is rolled, the adept may ask 1 question about the
information. If a triple effect is rolled, the adept may ask 2
questions.

If the symbol is magical then Adept will discern its general
effect. If a double effect is rolled, the adept can ask for 1 of the
attributes of the spell (e.g.  Rank, specific name, etc.).  If a
triple effect is rolled, 2 attributes may be discovered.
\end{effects}
\end{talent}

\begin{talent}[T-2]{Spirit Vision}

\range{50 feet + 10 / Rank}
\multiple{200}
\begin{effects}
The Adept may attempt to see into the spirit world. They can see
spirits, such as the souls of the dead (which normally remain close to
their bodies for 3 days before travelling to the lands of the dead),
those travelling outside their bodies (e.g.  via the Spell of
Visitation or the Herbalist Potion), incorporeal or insubstantial
undead etc.  (e.g.  a vampire in the form of a cloud of mist as an
undead spirit), insubstantial Fae (e.g. dryads, sylphs), summoned
spirits (e.g.  whispering wind, speak with dead), as though they were
normally visible.

Although the Adept cannot normally see the spirit of a living being
(inside their body) they may, at the GM’s discretion, gain some
inkling into a characters soul should they have attracted any spirit
followers.
\end{effects}
\end{talent}

\section{General Knowledge Spells}

\begin{spell}[G-1]{Control Entity}

\range{Touch}
\duration{Special}
\multiple{500}
\basechance{10\%}
\resist{Passive}
\target{Entity}
\rune{Entity} 
\begin{effects}
This spell requires the blood of either the target or the Adept to be
used to paint a Rune of Compulsion onto the forehead of the target.
If target fails to resist, then they are compelled by the Adept.

The compulsion does not in any way affect the mindset or opinion of
the target, but they are forced to obey an direct command given to
them.  Should the target be opposed to the Adept, then they will
interpret any command in the narrowest and least useful manner
possible. This spell has a duration of 1 hour (+ 1 / Rank), unless it
is ritually cast, when it lasts for 1 day (+ 1 / Rank).
\end{effects}
\end{spell}

\begin{spell}[G-2]{Darkness Rune}

\range{5 feet + 1 / Rank}
\duration{15 minutes + 15 / Rank}
\multiple{100}
\basechance{40\%}
\resist{None}
\target{Point}
\rune{Object/Runestaff}
\begin{effects}
The Adept creates a volume which elemental darkness fills like fog.
The volume is a sphere with a radius equal to the spells range,
centred on the Rune drawn by the caster and may not be moved unless
the Darkness Rune is in- scribed on the Adept’s Runestaff.

The darkness created will be 60\% + 2 / Rank). At Ranks 0--4 the
darkness is like evening twilight, at Ranks 5--9 it is like moonlit
night, at Ranks 10--14 it is a starlit night, at Ranks 15--19 like
pitch dark room and at Rank 20 (100\% dark) no vision is possible.
Although infravision works off heat and elvish and dwarvish visions
work in total darkness, it is still not possible to see at all at rank
20. This is elemental darkness and will cast shadows.  However it does
not give Celestials bonuses (but may give penalties).
\end{effects}
\end{spell}

\begin{spell}[G-3]{Lesser Healing Rune}

\range{Touch}
\duration{Immediate}
\multiple{200}
\basechance{35\%}
\resist{None}
\target{Living entity}
\rune{Living Entity}
\begin{effects}
The Adept paints Runes of Healing over the body of the target.  The
spell takes at least a minute to cast and heals 1 + (Rank / 2) points
of damage.
\end{effects}
\end{spell}

\begin{spell}[G-4]{Light Rune}

\range{5 feet + 1 / Rank}
\duration{15 minutes + 15 / Rank}
\multiple{100}
\basechance{40\%}
\resist{None}
\target{Point}
\rune{Object / Runestaff} 
\begin{effects}
The Adept draws the Light Rune (or has it inscribed on their staff).
The Rune will emit light as a point source and cannot be moved unless
inscribed on a Runestaff.  The light within the specified range will
be 60\% + 2 / Rank. At Ranks 0--4 this light is equivalent to a small
lamp and will clearly illuminate the immediate hex, Ranks 5--9 the
light is like that of a camp fire and will clearly illuminate the
surrounding megahex, Ranks 10--14 the light is like a large bonfire
and will brightly illuminate a radius of 15 feet, Ranks 15--19 the
light like a searing forge and brightly illuminates a radius of 20
feet and Rank 20 the light as if the sun on a bright day and will be
blinding within a 25 feet radius.

This light is elemental light and is a point source so extends beyond
the specified range (at naturally reducing levels). This will create
shadows but does not give Celestials bonuses (but may give penalties).
At Rank 10 and beyond the actual point source is over a foot in
diameter and creates shadows without a defined edge.
\end{effects}
\end{spell}

\begin{spell}[G-5]{Liquid Purification}

\range{Touch}
\duration{Immediate}
\multiple{100}
\basechance{30\%}
\resist{May not be resisted}
\target{Liquid}
\rune{Runestaff / Container}
\begin{effects}
The Adept may turn any aqueous substance into potable water by
touching the substance with their Runestaff which has the Purification
Rune incised into it. The Adept may purify 1 (+ 1 / Rank) gallon by
volume with this spell.  This spell may be used to neutralise poison
in solution. Note: This spell is not intended for use in combat and
will not work on anything with magic resistance. If the Rune is drawn
on a vessel of maximum capacity 1 (+ 1 / Rank) quart then any liquid
within the vessel is purified.  At Rank 11 or above, the Rune may
cause the vessel to shatter if it contains poison.

This spell may be cast reversed to pollute a liquid.
\end{effects}
\end{spell}

\begin{spell}[G-6]{Pyrogenesis}

\range{5 feet + 1 / Rank}
\duration{Immediate}
\multiple{75}
\basechance{40\%}
\resist{None}
\target{Object or area}
\rune{Point / Object / Runestaff}
\begin{effects}
A Fire Rune is drawn and all eligible things (small flammable objects,
or entities no larger than a mouse) within range burst into flame.
The flames are fuelled by the object or entity, and may be
extinguished normally.

If the Rune is on an object, then only the object will ignite. If the
Adept has the Rune on their staff, then they may target a hex up to 5
feet (+ 5 / Rank) away.
\end{effects}
\end{spell}

\begin{spell}[G-7]{Smite}

\range{Touch}
\duration{1 hour + 1 / Rank}
\multiple{200}
\basechance{35\%}
\resist{Passive}
\target{Entity}
\rune{Entity}
\begin{effects}
The Adept must paint a Smite Rune on the target.  This Rune is then
activated by the Smite Spell.  Should the target make a successful
strike, then the opponent must make a magic resistance against the
Smite Spell.  If the opponent fails to resist, they suffers [D + 1] (+
1 / Rank) damage. If the spell is Rank 10 or above, should the
opponent fail to resist then they are thrown prone.  At Rank 20 should
the opponent fail to resist, they are also stunned.

Once a successful strike has been made the spell ceases to be in
effect. The strike should be considered as being performed with a
magical weapon.

If the spell is ritually cast then target may make 1 (+ 1 / 6 Ranks)
successful strikes before the spell ceases to be in effect.
\end{effects}
\end{spell}

\section{General Knowledge Rituals}

\begin{ritual}[Q-1]{Fashioning Runestaff}

\multiple{300}
\basechance{30\% + 3\% / Rank}
\casttime{1 week}
\rune{Staff or Wand}
\begin{effects}
The Adept may employ this ritual to create a Runestaff or Runewand out
of any of the materials listed for Rune Wands (§26.2). The implement
is fashioned by inscribing Runes into the material’s surface, which
describe its use, name, and history.  Once it has been fashioned and
consecrated in this ritual, it remains fully effective unless and
until it is broken or otherwise destroyed.  A rune mage may only have
one rune staff or Runewand at any time.

All materials used in an unsuccessful ritual (or a ritual that
backfires) are destroyed or ruined. If the ritual is successful, the
Adept may use the Runestaff or Runewand thereafter to cast spells and
perform rituals that require the use of a Runestaff.

The adept may also inscribe runes upon the Runestaff or Runewand that
aid in the casting of certain spells. The adept may inscribe 1 rune
into a wand or up to 1 + (Rank / 3) runes into a staff.

In addition, the Adept may store a maximum of 2 Fatigue Points in the
Runestaff at Rank 0, and an additional 1 Fatigue Point for every Rank
they have with the ritual of Fashioning Runestaff at the time the
Runestaff is fashioned.  (This amount is doubled if the staff is made
of Oak, or halved if the item is a wand.  An oak wand holds the
standard amount.)  Fatigue is stored may be used by the Adept to cast
spells at any time that they are holding the Rune staff while making a
Cast Check. The Staff will be restored to full fatigue at midnight on
the night of a full moon.
\end{effects}
\end{ritual}

\begin{ritual}[Q-2]{Runes of Sight}

\multiple{300}
\basechance{40\% + 3\% / Rank}
\casttime{1 hour}
\rune{Self/Area, Object, Entity} 
\begin{effects}
The Adept may gain insight into the future by drawing the Runes of
Sight (Runes which represent the cosmic balance). There is no
possibility of backfire from this ritual.  The performance of this
ritual allows the Adept to exercise one of the following functions
during its course:
\begin{Description}
\item[Limited Precognition] The Adept draws a Rune of Sight on
  themselves. This ritual produces the same results as for the Spell
  of Limited Precognition of the Mind College.

\item[Divining Enchantment] The Adept draws Runes of Sight around the
  target to attempt to determine if an entity or object is currently,
  or had been recently, under the effects of a spell.  The object or
  entity must be present for the entire duration of the ritual, and be
  within 5 feet (+ 1 / Rank). The ritual may not be resisted.  The
  Base Chance is reduced by 5 for every week or part thereof since the
  spell that is being divined was cast.  Permanent magic (e.g.
  invested items still with charges) or spells currently in effect
  carry no modifier.  The Adept gains knowledge of those spells that
  fall within their cast chance.

  If the Adept can divine the spell, its exact name and college are
  revealed.  If the spell is noncolleged in origin, its general
  effects are revealed.  Only one of these two options may be
  performed at each casting of the Ritual.
\end{Description}
\end{effects}
\end{ritual}

\begin{ritual}[Q-3]{Sending}

\range{10 miles + 5 / Rank}
\multiple{250}
\basechance{30\% + 5\% / Rank}
\casttime{5 hours}
\resist{The ritual can be only passively resisted}
\rune{Self}
\begin{effects}
The Adept must paint their forehead with a Sending Rune before
retiring to sleep.  They will then require a five hour period of sleep
with no disturbances sufficient to wake them or the ritual will
fail. The target of the spell is likewise required to be asleep for
five undisturbed hours or the ritual will not work. The time asleep
counts as resting for Fatigue recovery purposes. During the time
asleep, the Adept will be in communication with one entity of their
choice that they have seen and studied sufficiently (as per College of
Ensorcelments and Enchantments Spell of Location for “seen and
studied”). Alternatively, the Adept may employ the target’s Individual
True Name if it is known.

If the Cast Check is successful and the target fails to resist then it
will answer all questions asked of it in a yes / no fashion.  This
ritual does not allow communication with entities on other planes of
existence. Upon completion of the ritual the Adept may receive the
answers to Rank questions.
\end{effects}
\end{ritual}

\begin{ritual}[Q-4]{Warding with Runes}

\range{70 feet}
\duration{1 week + 1 / Rank}
\multiple{200}
\basechance{30\% + 5\% / Rank}
\casttime{2  hours  (-10  minutes  /  rank)  minimum 10 minutes}
\resist{None}
\target{Area}
\rune{Around area}
\begin{effects}
The Adept must draw Rank Rune of Warding symbols in a roughly circular
configuration around the area to be warded (the Adept must remain
inside the area while the ritual is being prepared). At the end of the
ritual, if it is successful, a Rune Ward exists that will help to
protect those inside it from magic.

No magical item (amulet, weapon, etc.)  can enter the warded area
unless it is a possession, though items already inside the warded area
can be taken out.

Any magical creature, spirit or Adept attempting to enter the warded
area must make a Passive Resistance (-2 / Rank of ritual) check, or it
will be unable to enter the area.  In addition, an entity which is
wholly or partially of another plane (such as demons, devils, imps,
hellhounds) decreases its Magic Resistance by 3 / Rank when it
attempts to enter the warded area.

If the ward is breached then one of the Runes supporting the ward
momentarily glows and then disappears. When the last Rune disappears
then the ward dissipates.

In addition, so long as it is in effect, all targeted spells cast into
(not out of) the warded area have a 30\% + 2 / Rank of ward chance of
being dissipated harmlessly when striking the warded area.

Backfire from this ritual results in D10 damage to the Adept’s
Endurance.

All entities which were in area of the ward for the duration of its
casting of the ward are not subject to it.
\end{effects}
\end{ritual}


\section{Special Knowledge Spells}

\begin{spell}[S-1]{Banishment}

\range{Touch with Runestaff}
\duration{Immediate}
\multiple{250}
\basechance{30\%}
\resist{Passive}
\target{Entity}
\rune{none} 
\begin{effects}
The Adept may banish any one entity back to its own plane of origin.
In order to do so the Adept must touch the target entity with their
Runestaff at the moment the spell is completed.  If successful, the
spell results in the entity immediately returning to its own plane
unless the entity successfully resists.  The touch is automatic unless
the target is actively avoiding being touched, in which case the
target must make a successful strike at the moment of casting.  The
spell must be prepared normally.  The target returns to a random spot,
in an appropriate medium, on its own plane.  The exact whereabouts is
GM’s discretion, however, entities banished at approximately the same
time will appear in approximately the same area.
\end{effects}
\end{spell}

\begin{spell}[S-2]{Control Corpse}

\range{Touch}
\duration{1 hour + 1 / Rank}
\multiple{300}
\basechance{15\%}
\resist{None}
\target{Corpse}
\rune{Corpse}
\begin{effects}
The Adept inscribes the Animate Rune on the target corpse (can be
either sentient or non-sentient but must be formally living).  With a
successful cast check the Adept will animate the corpse into a zombie
under their control. The zombie will work at (4 / Rank)\% of their
living physical ability. The zombie is completely mindless and
requires at least passive concentration for the Adept to function. The
maximum size of the entity is 1 hex + (1 / 5 ranks).
\end{effects}
\end{spell}

\begin{spell}[S-3]{Converse with Spirits}

\range{10 feet + 5 / Rank}
\duration{5 minutes + 5 / Rank}
\multiple{200}
\basechance{30\%}
\resist{None}
\target{Self}
\rune{Self}
\begin{effects}
The Adept inscribes the Converse Rune and a Rune representing the
target spirit on their face.  Should the Adept successfully cast this
spell they will be able to “converse” with a single spirit which is
within range and falls within the Rune of representation. For example,
the Adept could use a rune representing lesser undead, and then any
ghost or other lesser undead could answer, or use a number of runes
to represent say Girden Bloodaxe, a fallen dwarven warrior.  Then, if
the spirit of Girden is within range then only he would answer.

This spell does not compel any spirit to answer any question and if
they do answer then it does not compel them to speak the truth.
\end{effects}
\end{spell}

\begin{spell}[S-4]{Creating Rune Weapon}

\range{Touch}
\duration{5 minutes + 1 / Rank}
\multiple{200}
\basechance{20\%}
\resist{None}
\target{Weapon }
Rune: Entity 
\begin{effects}
The Adept may create a magically poisoned weapon by inscribing a Rune
of Acid on a weapon and activating it with the Rune Weapon spell.  If
at least one point of effective damage is inflicted on a target, they
will take [D - 5] (+ 1 / 3 or fraction ranks) damage per pulse for D10
pulses.  The target can only have such affect in effect at any one
time, i.e. acid from different strikes is not cumulative.  The acid is
considered magical in origin and will affect creatures not normally
affected by such things.  The normal rules for using poisoned
weapons apply but the Adept is immune to their own weapon spell.

The Adept may choose instead to draw a Weapon rune in the air and
create a magical weapon of their choosing.  The weapon will be
insubstantial and magical in natural and will hit everything,
including creatures of a spiritual or spectral nature, but otherwise
will be completely normal.
\end{effects}
\end{spell}

\begin{spell}[S-5]{Greater Heart Rune}

\range{Touch}
\duration{1 day + 1 / Rank}
\multiple{300}
\basechance{25\%}
\resist{Active, Passive}
\target{Living Entity}
\rune{Circle}
\begin{effects}
The Adept expends 5 FT and takes 10 minutes to paint a Rune of Healing
on the skin over the heart, of at least 4 inches in diameter.  The
rune will heal the target 3 (+ 1 / 2 Ranks) Endurance, immediately, or
when the target next takes endurance damage.  The rune can be washed
of easily with water.

The Adept can make the rune semi-permanent by tattooing the Rune onto
the target. This reduces the target’s Endurance by 3 points until the
rune is used or the spell ends.
\end{effects}
\end{spell}

\begin{spell}[S-6]{Rune Curse}

\range{5 feet + 5 / Rank}
\duration{Special}
\multiple{200}
\basechance{15\%}
\resist{Active, Passive}
\target{Entity or Object }
Rune: Target/Runestaff 
\begin{effects}
The Adept must first have the Curse Rune inscribed on their Runestaff
for this spell to work at range, otherwise they can carve the Curse
Rune into the victim (taking a minute).  The duration of the curse is
based on the cast time.

\begin{tabularx}{\columnwidth}{ll}
Cast time	& Duration  \\
Pulse		& Rank minutes  \\
Minute		& Rank hours \\
Hour		& Rank days \\
Day		& Permanent \\
\end{tabularx}

The Adept curses any one target with a particular unpleasantness as
listed below. If the effects of the curse are doubled or tripled, the
Adept may inflict 2 or 3 different results.  If a Ritual of Remove
Curse is employed, the Rune Curse is considered a Minor Curse.  Ritual
of Remove Curse must be used on each separate curse.  Identical Rune
Curse effects are not cumulative. The Adept may always choose to
inflict a curse of lesser Rank than their actual Rank. The Curses that
the Adept may inflict are dependent on the Rank of the spell:

\begin{Description}
\item[0--4] The victim will suffer hallucinations that will reduce
  their Perception by 5 in addition to any specific effects. The GM
  and the Adept must work out the exact nature of the hallucination at
  the time that the curse is made. Hallucinations should, how- ever,
  be of a minor, generalised nature, seeing coloured lights in the
  distance, hearing sounds like the clanking of weaponry, smelling
  meat cooking from time to time, and so forth.

\item[5--9] The victim will suffer from terrible migraines and must
  make a concentration check for every complex action (such as casting
  or using a skill, but not standard combat).

\item[10--13] The victim will suffer from limited Amnesia. Any complex
  activity (using a weapon, casting a spell etc) will require a Magic
  resistance check.  Should the victim fail they will be unable to
  remember how to perform that action will not be able to remember it
  again for a period of (Rank × Cast time).  The victim has not
  forgotten anything but simply temporarily can’t remember how to do
  something.

\item[14--16] The victim is afflicted with Creeping Senility and will
  lose 1+ (Rank / 5] points of MA immediately and a similar amount
    every day afterward.

\item[17--19] The Adept may afflict the target with extreme paranoia
  and nightmares.  The target will recover only one fatigue point per
  hour from taking a nap, and only 2 per hour from sleeping. In
  addition, the target will feel hag-ridden and imagine themselves
  pursued by phantasms. They will, until the curse is removed, become
  more and more estranged from reality, distrustful of friends and
  companions, and obsessed with the idea of destroying their enemies
  (who they think are “all around”). If the curse is not removed
  within D10 × [target’s Willpower 2 × Rank] days, the target will
  completely lose touch with reality.  They will then plot to destroy
  their friends in the belief that they are “out to get them” and will
  exhibit other bizarre behaviour.  They will be cured of the advanced
  stage of this affliction only by having the curse removed and then
  spending a number of days equal to the Adept’s Rank × D10 in rest
  and recuperation.

\item[20] Total Amnesia 
\end{Description}
\end{effects}
\end{spell}

\begin{spell}[S-7]{Rune Lock}

\range{Touch}
\duration{1 hour + 1 / Rank}
\multiple{200}
\basechance{30\%}
\resist{None}
\target{Portal}
\rune{Portal}
\begin{effects}
This spell may be cast over any portal (door or window) inscribed with
the Lock Rune that can normally be opened or closed and is in sight.
It effectively locks the portal with an unpickable lock.  The spell
can be dispelled by anyone casting the Rune College Special
Counterspell or Spell of Opening of at least equal Rank.
\end{effects}
\end{spell}

\begin{spell}[S-8]{Rune of Truth}

\range{Touch}
\duration{10 minutes + 10 / Rank}
\multiple{300}
\basechance{30\%}
\resist{None}
\target{Entity}
\rune{Entity} 
\begin{effects}
Prior to casting this spell the Adept must draw a Truth Rune on the
forehead (or equivalent) of the target.  The Rune of Truth causes the
target to be unable to speak a falsehood for the duration of the
spell.  The target must not knowingly say anything false, but may
refuse to answer a question put to them.

In addition, the bearer of the Truth Rune may attempt to see the true
nature of all things with a (PC + 2 × Rank) chance of noticing
deceptions, such as illusions, invisibility, shape or skin changing,
traps, and any other deception the GM sees fit.  Only one attempt may
be made per object.

The Truth Rune does not necessarily help the target see through the
deception, for example, an Illusory Wall will still be opaque, but the
target will know it is an illusion.
\end{effects}
\end{spell}

\begin{spell}[S-9]{Rune of Willow Healing}

\range{15 feet + 15 / Rank}
\duration{Special}
\multiple{450}
\basechance{35\%}
\resist{Passive}
\target{Any living creature}
\rune{Over heart of target}
\begin{effects}
The Adept first paints The Rune of Healing over the heart of the
target. At any time within twice rank hours the adept activates the
Rune within range. The rune will heal the target 3 endurance damage
per pulse for (Rank + 2) pulses then fades.
\end{effects}
\end{spell}

\begin{spell}[S-10]{Rune Shield}

\range{Touch}
\duration{1 hour + 1 / Rank}
\multiple{250}
\basechance{40\%}
\resist{None}
\target{Entity}
\rune{Entity}
\begin{effects}
The Adept must inscribe a Rune of Protection onto the target.  The
magic will create a shield of protection around the target, giving 5\%
+ Rank to defence and absorbing Rank / 4 points of physical damage.
Any Grievous Blow to the target will disrupt the shield but, in that
case, the specific grievous injury will not be applied to the target.
\end{effects}
\end{spell}

\begin{spell}[S-11]{Rune Wall}

\range{None}
\duration{30 minutes + 30 / Rank}
\multiple{250}
\basechance{20\%}
\resist{Passive}
\target{Area}
\rune{Point}
\begin{effects}
The Adept may, by drawing a Rune of Protection, not necessarily on an
object, create a 20 (+ 2 / Rank) feet radius, transparent, shimmering
wall of force 1 inch thick, centred on the Rune.

The wall can be of any orientation and need not be anchored.  It will
expand around solid objects but will not pass through them and will
not form touching an entity (the spell will fail immediately if it
comes in contact with an entity while forming).

Any entity who comes into contact with the wall must resist or be
thrown back prone and will suffer [D - 2] (+ 1 / Rank) damage.

If the portal is destroyed by brute force (or by magical means) then
the spell will dissipate. It will take rank × rank points of damage to
destroy the Rune locked portal.

The Adept may open any portal they have locked without dissipating the
lock.
\end{effects}
\end{spell}

\begin{spell}[S-12]{Sacrifice}

\range{Touch}
\duration{10 seconds + 10 / Rank}
\multiple{650}
\basechance{5\%}
\resist{Passive}
\target{Any living creature}
\rune{Self}
\begin{effects}
The Adept first draws the Death Rune across their forehead and then
activates it.  They must then touch their victim (successful unarmed
strike) and release the spell.  If the victim fails to resist the
Adept immediately gains all current Fatigue and Endurance from the
victim.  Up to Rank points each of this can be used to heal damage
and restore fatigue respectively.  If the victim has zero or less
current fatigue and Endurance nothing is gained.

In addition, if the Adept then spends a hour making a meal of their
victim they can temporarily increase their following characteristics
to up to (5 / Rank) \% of that the victim.

\begin{tabularx}{\columnwidth}{Xl}
Organ / Body part	& Stat  \\
Brain			& WP \\
Heart			& EN \\
Arms			& PS \\
Legs			& AG \\ 
Hands			& MD \\
\end{tabularx}

Eating the genitals will give a Rank\% increase to the Adept’s
virility.  See Conception (§4.8) for conception chances.

If the victim is skinned then the Adept may “wear” skinned.  This will
cause superficial physical changes such as snout nose, hoofed feet,
clawed hands, hairy skin.

All transformations last 10 (+ 10 / Rank) minutes.  This spell does
not work on plants.
\end{effects}
\end{spell}

\begin{spell}[S-13]{Summon Totem Helper}

\range{Unlimited}
\duration{10 minutes + 10 / Rank}
\multiple{200}
\basechance{30\%}
\resist{None}
\target{Spirit}
\rune{Self}
\begin{effects}
If successful a Totem spirit will arrive in D10 pulses to aid the
Adept. This aid may include asking the spirit to summon a totem animal
to the Adept, give basic geographical knowledge and do simple scouting
tasks.  The spirit will automatically warn the Adept of any immediate
danger to the Adept that they see (with a base chance of Adept’s PC +
2 / Rank).  A Totem spirit cannot be summoned again until (24 - Rank)
hours have passed.
\end{effects}
\end{spell}

\begin{spell}[S-14]{Torment}

\range{15 feet + 15 / Rank}
\duration{Immediate}
\multiple{250}
\basechance{15\%}
\resist{Active, Passive}
\target{Entity}
\rune{Entity}
\begin{effects}
The Adept can, by pointing their Runestaff, inscribed with the Pain
Rune, at one entity, cause that entity extreme pain. Entities who fail
to resist may only take a Pass action every second pulse until they
recover.  Entities who successfully resist reduce all Strike Chances
by 30, and take twice as long to perform any action until they
recover.  Note that Mind Mages gain a bonus to resist this spell equal
to 2 × Rank with their Talent of Resisting Pain.

Each pulse that the Adept continues to point the Runestaff at the
entity (requiring a pass action) it suffers [Rank / 4] point of damage
and may not attempt to recover from the spell.

\begin{tabularx}{\columnwidth}{lX}
Rank	& Difficulty \\
0--4	& 4 × WP \\
5--9	& 3 × WP \\
10--14	& 2 × WP \\
15–19	& 1 × WP \\
20	& 0.5 × WP \\
\end{tabularx}
\end{effects}
\end{spell}

\begin{spell}[S-15]{Trapping Spirit}

\range{10 feet + 5 / Rank}
\duration{1 minute + 1 / Rank}
\multiple{250}
\basechance{5\%}
\resist{Active, Passive}
\target{Spirit}
\rune{Circle}
\begin{effects}
The Adept must draw a circle or at least one foot radius with the
Runes of Protection around its circumference and then draw a pentacle
using fresh blood and inscribe the Runes of Binding and Representation
(see Converse with Spirits above on specifying the target spirit)
within. If the named spirit is within range, they will be drawn into
the pentacle where they are trapped for the duration of the spell.

If the spell is ritually cast then the duration increases to 10 (+ 10
/ Rank) minutes.
\end{effects}
\end{spell}

\begin{spell}[S-16]{Visitation}

\range{1 mile + 1 / Rank}
\duration{ Concentration:  maximum  1  hour  +  1  / Rank}
\multiple{300}
\basechance{15\%}
\resist{None}
\target{Entity}
\rune{Self}
\begin{effects}
The Adept must draw the Runes of Farseeing on themselves while
performing the spell.  If successful, the Adept is able to send a
ghost-like image of themselves instantly to a previously drawn Rune of
Location, within range.  They are present in that location in all ways
except bodily (i.e. the Adept may communicate and use all their senses
while the image is there, but may not be harmed by any attack).  The
image may move no more than 10 feet (+ 10 / rank) from the specific
Rune of Location, and may materialise anywhere within that area.  The
Adept may not cast any spells or rituals. When the visitation time has
expired (or anytime prior that the Adept wishes), the image quickly
fades and travels back to the Adept.
\end{effects}
\end{spell}

\begin{spell}[S-17]{Warning Stones}

\range{Touch}
\duration{1 hour + 1 / Rank}
\multiple{100}
\basechance{20\%}
\resist{None}
\target{Self}
\rune{Stone}
\begin{effects}
The Adept draws the Rune of Warning and at least one of the Runes of
Body and/or Mind on a stone (which must weigh at least 1/2 lb. per
Rune).  The Adept may then leave the stone somewhere and will
instantly know if an entity comes within 5 feet (+ 1 / Rank) of the
stone. The stone will detect living and/or sentient entities depending
on which Runes it is inscribed with.  The Adept may use as many
warning stones as they wish, but will be unable to tell which of their
stones has detected an entity.
\end{effects}
\end{spell}

\section{Special Knowledge Rituals}

\begin{ritual}[R-1]{Binding Elements}

\duration{2 hours + 2 / Rank}
\multiple{500}
\basechance{MA + 3 / Rank}
\casttime{30 minutes}
\rune{Runestaff}
\begin{effects}
The Adept may gain control of any element by using this ritual.  They
must have had the Binding Rune and the Rune representing the element
to be bound inscribed on their Runestaff and they must touch the
element with their Runestaff at the conclusion of the ritual.  The
Adept may bind 500 pounds of earth (+ 500 / Rank), 500 gallons of
water (+ 500 / Rank), 1000 cubic feet of air (+ 500 / Rank), or all
fire within a 10 foot radius (+ 15 feet / Rank).  They may do anything
with the element except form an elemental.  This ritual may not be
used over an area occupied by an elemental and cannot be used in any
way to control an elemental.
\end{effects}
\end{ritual}

\begin{ritual}[R-2]{Binding Spirits}

\duration{Permanent}
\basechance{MA + 3 / Rank}
\casttime{4 hours}
\resist{May not be resisted}
\rune{Skull}
\begin{effects}
To perform this ritual the Adept must possess the skull of the spirit
they wish to bind, and the spirit must be present (e.g. within 100
hours of death, the body has been preserved by a Healer, the spirit is
held in a spirit trap, the spirit has remained on plane as a greater
undead or ghost). The Adept must spend four hours cleaning the skull
(boiling off any remaining flesh etc.)  and etching it with Runes to
bind the spirit.  Note that although they may not resist this ritual
the spirit may, should they be able, attempt to disrupt the ritual or
slay the Adept.  If the Adept has the victim’s heart, they may burn
this during the ritual to gain an extra +20 on Base Chance.

Upon successful completion of this ritual the spirit is bound to the
skull and may not leave unless and until the skull is destroyed.
Although a bound greater undead would be able to drain anyone who
touched the skull, in general the bound spirit will be unable to
affect the material world.  The Adept can use the Spell of Converse
with Spirits to question the spirit, and may gain useful answers /
advice should the spirit have any expertise in the area. Although the
spirit cannot lie it may refuse to answer and can mislead by omission
or neglecting to correct false assumptions and the like.  A backfire
result destroys the skull and the Adept’s Endurance value is reduced
by [D-5] (minimum 1) points which may only be recovered by the
expenditure of Experience Points.  The Adept will be unable to attempt
to bind that spirit again.  NOTE: Unless the spirit has some reason to
wish to remain as an adviser it is unlikely to be happy about being
kept trapped on this plane.
\end{effects}
\end{ritual}

\begin{ritual}[R-3]{Casting the Runes}

\multiple{500}
\basechance{5\% + 5\% / Rank}
\casttime{1 hour}
\rune{Paper}
\begin{effects}
The Adept must prepare a piece of paper or vellum on which are written
the Runes of Doom.  At the end of the ritual, the Adept chooses a
creature (see below) from the Seventh Plane to be the executor of the
doom and also writes this name on the paper. The name must be capable
of being read.  The Adept’s player must actually write this
information down, since it will only come into play in the future.
Once the ritual is prepared, the Adept then passes the sheet of paper
on to the victim whose name is written on the paper.  The victim must
voluntarily accept the paper (though they need not know what is on
it).  Once they accept it, a creature named on the paper will turn up
in [20 + D10 - Rank] days and hunt them down and kill them.  Even if
the creature is destroyed another will return within a similar time.

Only by passing the paper on to another entity who voluntarily accepts
it can the doom be transferred.  If the paper is destroyed, the doom
can never be transferred. The Doom may be lifted by the Adept by
ritually casting a Rune Special Counterspell on the target or a remove
curse may be performed.  The curse is considered a Major Curse with an
MA of (MA of Adept + 2 × Rank of ritual) and can be removed by Ritual
of Remove Curse.

This ritual requires the expenditure of one point of Endurance
(permanently) regardless of success.  If the ritual backfires, the
named creature will immediately turn up and attempt to kill the Adept,
but will not return once destroyed.

\begin{tabularx}{\columnwidth}{lX}
Rank	& Creature \\
0--4	& Imp \\
5--9	& Half devil \\
10--14	& Devil \\
15--19	& Succubus or Incubus \\
20	& Named Demon \\
\end{tabularx}
\end{effects}
\end{ritual}

\begin{ritual}[R-4]{Creeping Doom}

\multiple{450}
\basechance{20\% + 4 / Rank}
\casttime{1 hour}
\resist{Special}
\rune{Bones}
\begin{effects}
The Adept creates 13 Runes of Destruction by carving the appropriate
maledictions into human bones.  They then perform a ritual over them
and bury the sticks beneath the dwelling of someone they wish to
curse.  It is best if the victim’s name is carved in the bones as
well, otherwise others in the house may become ill instead. For each
month that the bones remain in or under the victim’s dwelling, they
must make a Resistance Check, the Base Chance for which is composed of
the victim’s Endurance multiplied by the Difficulty Rating of the
resistance.

\begin{tabularx}{\columnwidth}{lX}
Rank	& Difficulty \\
0--5	& 4 × EN \\
6--10	& 3 × EN \\
11--15	& 2.5 × EN \\
16--18	& 2 × EN \\
19--20	& 1.5 × EN \\
\end{tabularx}

If the victim fails to resist, they suffer a wasting disease and lose
[D-3] Endurance points for the purposes of future resistance (only).
If they fail to resist for three straight months, they die.

Generally, the victim of these maledictions does not know exactly what
is wrong with them. Should they discover the bones, they may remove
the curse by removing the bones from the house.  Other means of ending
a curse do not normally suffice, although the sufferer would show
immediate improvement upon leaving the house and sleeping elsewhere
for a few weeks.  There is no chance of this ritual backfiring.
\end{effects}
\end{ritual}

\begin{ritual}[R-5]{Rune Healing}

\range{Touch}
\duration{Immediate}
\multiple{300}
\basechance{2 × MA + 3 / Rank}
\resist{None}
\casttime{30 minutes}
\target{Living entity}
\rune{Living Entity}
\begin{effects}
The Adept paints Runes of Healing over the body of the target. The
rune will heal the target 3 + 3 / Rank.  In addition, all afflictions
which can be cured by a (Rank / 3) healer will be fixed, with
exception of preserve dead, which cannot be done as the target is not
living.
\end{effects}
\end{ritual}

\begin{ritual}[R-6]{Rune Portal}

\range{Special}
\duration{Special}
\multiple{400}
\basechance{MA + 5 / Rank}
\casttime{30 minutes}
\rune{Stone circle/circle}
\begin{effects}
Rune portals allow a Rune Mage to transport themselves and Rank other
entities with them to any other portal which the Rune Mage has
visited, has sufficient knowledge (uniquely distinguishable) about,
or the source portal’s “linked destination”.  The Adept can create two
types of Rune portals.
\begin{Description}
\item[Permanent] A permanent portal is constructed by the Adept
  inscribing Runes of Translocation on large stones, placing them on a
  flat surface to create a circle of a size, in hexes, at least equal
  to that of the number of entities the portal can transport.
  Permanent portals take a day per hex to construct and cannot be
  destroyed unless all the stones forming the portal are smashed. At
  the time the portal is created, a “link” destination can be imbued
  in it.  The destination must be a permanent portal the Adept has
  previously visited.
\item[Temporary] The portal is constructed by the Adept painting Runes
  of Translocation onto a surface is a circular fashion (taking half
  an hour).  The circle must be of a size, in hexes, at least equal to
  that of the number of entities the portal can transport.  Temporary
  portals last 1 week (+ 1 / Rank), unless a Rune Special Counterspell
  is cast into the area, in which case, it will immediately dissipate.
\end{Description}
The ritual to create the portal takes half an hour to perform and has
a Base Chance of MA + 5 / Rank.  If the ritual fails then nothing
happens and ritual can be performed again without additional work but
a backfire will ruin the entire ritual and a new portal will have to
be constructed.

To utilise a Rune portal, the Adept performs a half hour ritual.  The
Adept may transport a maximum of themselves and Rank others (multihex
creatures count as size, in hexes, entities).  Base Chance to
transport is MA + (5 / Rank) - (1 / 5 miles). Of the ritual is
successful, the Adept must spend 1 FT per entity transported.  A
backfire will only result the expenditure of the Fatigue.
Transportation is in- stantaneous.

All entities passing through a Rune portal lose all their Fatigue, in
the form of tiredness fatigue.
\end{effects}
\end{ritual}

\begin{ritual}[R-7]{Transformation}

\duration{Special}
\multiple{500}
\basechance{MA + 3\% / Rank}
\resist{Passive only}
\casttime{1 hour}
\rune{Circle}
\begin{effects}
By the performance of this ritual the Adept merges a living sentient
entity with a non-sentient animal.  Both entities must be living
(though they need not be conscious), and must remain within a circle
of runes for the entirety of the performance of the ritual.  Upon
completion of the ritual the animal will shrivel and wither away to
dust, while the sentient entity’s body will writhe and transform into
that of the animal. Both entities may choose to resist and should
either be successful the ritual will fail. The sentient will remain
trapped in the form of the animal (having the animals physical
characteristics but retaining their own Magical Aptitude, Willpower,
and Perception) until subject to a Ritual of Remove Curse (this ritual
counts as a Major Curse).

They will only be able to perform those skills and abilities which the
GM deems feasible in animal form, and will be unable to perform Spells
or Rituals.

At Rank 10, should the Adept have a sample of blood, hair, or nail
clippings etc.  from an entity they wish to target as prey, they may
include this in the Ritual.  In this case, instead of the above, the
animal and sentient’s bodies will writhe together and merge to form an
anthropomorphised version of the animal with each characteristic being
the higher of the two.  The hybrid is imbued with an irresistible
hunger for the targeted entity.  They will always know in which
direction their target is, and will be unable to perform any action
except hunting for and eating their prey. They will not sleep or eat
(except for their target) and will die in 8 hours (+8 / Rank).  The
sentient will be aware of their actions, but will have no control.  At
Rank 20 the Adept may use this ritual to make a chimera from any two
non-fantastical living creatures. The exact effects are up to the GM.
Some obvious examples are a minotaur from a human and a bull, a
gryphon from a eagle and a horse, etc.  Should both creatures be
sentient it will retain both heads.
\end{effects}
\end{ritual}

\end{Chapter}
