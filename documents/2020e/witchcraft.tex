\begin{Chapter}{The College of Witchcraft (Ver 1.1)}

The College of Witchcraft is concerned with natural magics, the
rhythms of the world, and espe- cially with blessings and curses.
Practitioners of the College of Witchcraft are commonly known as
Witches if female, Warlocks if male, or collectively as Wicca. This
College is without doubt the most primitive and least formal in its
approach to magic.  The Wicca generally feel themselves to be more in
tune with the world than the various Elemental Colleges, who dedicate
themselves to only part of the whole, and certainly more than the
Thaumaturgists who practise a sterile and scientific magic.  The
widespread use of Witchcraft predates the present renaissance of magic
on Alusia and Wicca are much more often found practising their trade
in small towns and villages than in cities.  Novice Wicca will usually
have been apprenticed to a local Witch or Warlock rather than having
attended any form of Magical Academy.

The College’s magic touches on alchemy, herbalism and astrology, and
many Wicca possess some of these skills.  The other Colleges often
treat Wicca with a degree of contempt as they view this dabbling with
“quasi-magic” to be less than worthwhile.  This is not to say that the
Wicca are without power, as experienced Adepts of Witchcraft have
available to them powerful magics, fully the equal of any other
College. In these destructive and powerful magics there lies danger
however, and some Wicca become so obsessed with the “darker” side of
natural magic that they begin to follow the Dark Path of magic and
make agreements with the Powers of Darkness so as to further their
material goals. These Adepts are often known as “Black Mages” and are
greatly feared.  Many “Black Mages” view the other members of the
College as weak or foolish for failing to exploit the powers they
possess.  At the other extreme, some Wicca follow a path of pacifism,
and eschew the curses and destructive side of Witchcraft.  These
Adepts are sometimes referred to as “White Mages” or “White Witches”,
in contrast to their darker brethren.  Most Wicca view both of these
extremes as unfortunate deviances from the Col- lege’s holistic path.
Many of the Agents of the Powers are Wicca as the College’s general
and undivided outlook does not conflict often with a Power’s
interests.

Being highly in tune with magic as a whole, Wicca are more sensitive
to changes in the “mana flow” than other Colleges.  The Wicca’s magic
is somewhat affected by the presence of large amounts of spirit such
as from proximity to many entities, or from a lessening of the “mana
flow”, such as on the certain “mana poor” days of the year, which have
universally become known as “High Holidays” of the Elohim, the spirit
Powers. Conversely, a Wicca’s powers increase when away from sources
of spirit and on certain “magic rich” days, sometimes referred to as
“Faerie days”.  Finally, due to the ancient nature of the College and
the equally ancient association between the elements of light and
spirit, a Wicca’s powers are slightly decreased during the hours of
daylight, whether the Wicca is in light or not, and are slightly
enhanced during the hours of darkness.

Traditional Colours 

Wicca usually wear clothing in the colours of nature itself, in much
the same way as the Elemental mages, but often in combination that the
Elementals do not use.  Blues and greens reminiscent of sky and sea
are worn with the light browns of the earth, and with the warm golds
and oranges of autumn or fire.  “Black Mages” traditionally cloak
themselves in midnight robes sometimes embroidered with pentagrams and
other Daemon associated symbols, whilst “White Mages” often wear robes
of bleached or unbleached wool or linen.

Traditional Symbols 

Animals are most commonly associated with the Wicca as many Adepts of
this College keep animal familiars with cats, ravens, toads and owls
being the most usual.  There is only one symbol that is often
connected with the Wicca, the Great Wheel of Being, representing light
and darkness, earth and air, water and fire, spirit and mana.  This
symbol sometimes appears as an eight pointed star or eight arrows
radiating from a central point, and at other times as two circles
passing through each other at right angles, or even simply, as two
intersecting circles. Black Mages often use and wear Daemonic symbols,
but very seldom use the Great Wheel.  White Wicca seldom use any
symbols, but when they do, the Great Wheel is the most common.


\section{Restrictions}

Adepts  of  the  College  of  Witchcraft  may  practice 
their arts without restriction. 

The MA requirement for this College is 18. 


\section{Base Chance Modifiers}

The Base Chance of performing any talent, spell or 
ritual of this College is modified by the addition of 
the following numbers: 

-5 
+5 
-5 

From sunrise to sunset 
From sunset to sunrise 
In large town or city (pop. greater than 
2000) 
In small town or village (pop. 500 to 2000)  
In hamlet or rural (pop. less than 500)  
Mana poor day 
Mana rich (Faerie) day 
In a high mana area 
In a low mana area 

All modifiers are cumulative.  Those modifiers pertaining to sunrise
and sunset are applied depending on the hour of day, and not on
whether the Wicca is standing in light or darkness.  A Wiccan
underground with no light during the day still receives the negative
modifier.  Modifiers pertaining to population, refer to the population
of sentient beings, within 1 mile of the Wicca.

+0 
+5 
-5 
 +5 
+5 
-5 


\section{Talents}

\begin{talent}[T-1]{Farsensing }

\range{15 feet + 15 / Rank }
\duration{Active concentration }
\multiple{150 }
\target{Familiar }
\begin{effects}
 The  Adept  can,  by  remaining  stationary 
and  actively  concentrating  for  the  duration  of  the 
talent’s  workings,  see,  hear,  taste,  smell  and  feel 
the  same  things  as  their  familiar,  provided  that 
their familiar is within 15 feet (+ 15 feet / Rank) of 
their  position.  This  talent  allows  no  special  communication  with the  familiar,  merely  the  ability  to 
utilise  their  senses.  The  Adept  must  have  already 
acquired  a  familiar  through  the  use  of  Finding 
Familiar Ritual (Q-1) for this talent to be effective. 

It takes about 10 seconds (- 1 / Rank) for the Adept to tune in to the
familiar’s senses. If the familiar is killed while the Adept is using
this talent the magical backlash is harsher, due to the tighter link,
and the amount of magical damage incurred is increased by 5 points,
see Q-1.

\end{effects}
\end{talent}

\begin{talent}[T-2]{Special Alchemy }

\begin{effects}
 The  Adept  gains  certain  knowledge  of 
Alchemy.  The  specific  benefits  accruing  to  the 
Adept are: 

Distilling  Venoms  The  ability  to  distill  venoms 
from  such  plants  as  belladonna.  The  Adept  func-
tions  as  a  Rank  1  Alchemist  for  this  purpose.  See 
the Alchemist Skill. 

Distilling  Toad’s  Sweat  The  ability  to  distill  a 
dose  of  a  potion  of  Toad  Sweat  that  will  remove 
blemishes, warts, corns, pimples, etc., at the rate of 
1  disfigurement  (wart,  corn,  etc.)  (+ 1  / Rank)  per 
dose.  The  Experience  Multiple  for  this  potion  is 
100  and  its  cost  is  50  Silver  Pennies.  The  Base 
Chance  of  effectively  preparing  it  is  60%  (+  2  / 
Rank). 

Making  Amulets  The  ability  to  make  the  follow-
ing amulets: 

Amethyst Wards bad dreams and assists the wearer 
in achieving a restful sleep.  Increases the wearer’s 
Fatigue  recovery  during  sleep  periods  by  10% 
(round down). Cost: 3000 SP. 

Aquilegius  The  wearer  subtracts  10  from  all  rolls 
on the Fear Table. Cost: 2400 SP. 

Beryl  Increases  the  wearer’s  ability  to  detect  traps 
and ambushes by 5. Cost: 4000 SP. 

Betony Decreases the Base Chance of infection by 
15. Cost: 2200 SP. 

Bloodstone  Prevents  miscarriage  and  decreases 
Base Chance of infection by 20. Cost: 3000 SP. 

Carbuncle Decreases damage done by poison by 2 
points of damage per pulse or day. Cost: 9600 SP. 

Chalcedony  No  undead  will  willingly  approach 
closer  than  10  feet  to  the  wearer  in  most  cases. 
Cost: 4800 SP. 

Diamonds  Increases  all  of  the  wearer’s  Strike 
Chances by 2. Cost: 8000 SP. 

Elder Flowers Makes the wearer proof against the 
Evil Eye. Cost: 400 SP. 

Hypericum  Increases  the  wearer’s  Magical  Resis-
tance  by  10  to  any  magical  act  performed  by  a 
Demon or Daemonic being. Cost: 800 SP. 

Iron  No  Demon  or  Daemonic  being will  willingly 
approach closer than 10 feet to the wearer in most 
cases. Cost: 4000 SP. 

Jade  No  undead  will  willingly  approach  closer 
than 30 feet to the wearer in most cases. Cost: 4000 
SP. 

Jet  No  Demon  or  Daemonic  being  will  willingly 
approach closer than 50 feet to the wearer in most 
cases. Cost: 4800 SP. 

Luck  Made  from  tiger’s  or  alligator’s  teeth.  It 
increases the wearer’s Magical Resistance by 3 and 
adds 2 to the wearer’s defence. Cost: 2400 SP. 

Note  that  the  “cost”  is  the  cost  of  material  neces-
sary  to  manufacture  the  amulet.  Each  amulet  re-
quires  3  days  to  manufacture  once  the  necessary 
materials  have  been  gathered  or  purchased.  Amu-
lets  are  usually  sold  at  (cost  +  25\%).  The  time 
taken  to  prepare  an  Amulet  is  full-time  work,  and 
no  training  may  be  undertaken  at  the  same  time. 
Those Amulets that prevent the “willing approach” 
of  certain  creatures  create  a  “circle  of  protection” 
around the wearer. The creatures protected against 
will not willingly cross the circle’s boundary, but if 
forced across it, for instance by the approach of the 
wearer,  are  no  longer  inconvenienced  by  the  pro-
tection. 

Love Philtre The ability to distill from a variety of 
substances  a  Love  Philtre  which  will  cause  the 
imbiber  to  fall  in  love  with  the  first  entity  upon 
whom he or she sets eyes after drinking it (regard-
less of species or sex). The Base Chance to prepare 
the Philtre is 30\% (+ 3 / Rank) and the Experience 
Multiple  is  200.  The  cost  of  the  materials  will 
average  600  silver  pieces.  The  effects  of  the  sub-
stance  will  last  for  1  week  (+  1  week  /  Rank), 
unless  dispelled  by  the  casting  of  the  General 
Knowledge  Counterspell  of  the  College  of  Witch-
craft  by  the  creator  of  the  Love  Philtre,  or  by  the 
successful use of the Curse Removal Ritual. In the 
latter case, the curse is treated as Minor. 

Fertility  /  Infertility  Potion  The  ability  to  distill 
from a variety of substances a Potion of Fertility or 
Infertility that increases or decreases the chances of 
conception by 5\% (+ 5 / Rank). It has a 30\% (+ 3 / 
Rank)  chance  of  working  and  may  be  passively 

27 COLLEGE OF WITCHCRAFT 

resisted by the imbiber. The effects of the Potion of 
Fertility  last  1  day  (+1  per  3  or  fraction  Ranks) 
whilst  that  of  the  Potion  of  Infertility  last  1  week 
(+1  per  3  or  fraction  Ranks),  unless  dispelled  by 
the casting of the General Knowledge Counterspell 
of  the  College  of  Witchcraft  by  the  creator  of  the 
Potion or a Ritual of Remove Curse is employed. If 
the latter option is taken, the curse is considered a 
Minor  Curse.  The  Experience  Multiple  for  this 
potion is 200 and its cost is 100 Silver Pennies. See 
Conception (§4.8) for conception chances. 

\end{effects}
\end{talent}

\begin{talent}[T-3]{Witchsight }

\multiple{200 }
\begin{effects}
 The  Adept  may  see  objects  or  entities 
which are invisible and they appear to have a slight 
blue  sheen  around  them.  If  the  invisibility  effect 
(excluding  Walking  Unseen)  is  of  a  higher  Rank 
than the Witchsight, the object or entity may not be 
clearly  identified  or  directly  magically  targeted. 
The  Adept  may  also  see  in  the  dark  as  a  Human 
does  on  a  cloudy  day,  with  an  effective  range  of 
vision of 150 feet under the open sky, and 75 feet 
elsewhere. 


\end{effects}
\end{talent}

\section{General Knowledge Spells}

\begin{spell}[G-1]{Damnum Minatum }

\range{15 feet + 15 / Rank }
\duration{Special }
\multiple{200 }
\basechance{40\% }
\resist{Active, Passive }
\storage{Investment, Ward, Magical Trap }
\target{Entity }
\begin{effects}
 The  Adept  curses  any  one  target  within 
range  with  a  particular  unpleasantness  as  listed 
below. Some of the effects are identical to backfire 
results;  such  effects  are  cross  referenced  to  the 
backfire  table  (§53).  If  the  effects  of  the  curse  are 
doubled  or  tripled,  the  Adept  may  inflict  2  or  3 
different  results.  The  curse  is  permanent  until  a 
General Knowledge Counterspell of the College of 
Witchcraft is cast over the afflicted entity, a Ritual 
of  Remove  Curse  is  employed,  the  duration  ex-
pires,  or  the  effect  is  cured  by  a  healer  of  the  ap-
propriate Rank. Curses that have a duration, or may 
be cured by a Healer are indicated in their descrip-
tions. If a Ritual of Remove Curse is employed, the 
Damnum Minatum is considered a Minor Curse. A 
separate  Counterspell  or  Ritual  of  Remove  Curse 
must  be  used  on  each  separate  curse.  Identical 
Damnum  Minatum  effects  are  not  cumulative. 
Note that the Adept may always choose to inflict a 
curse  of  lesser  Rank  than  their  actual  Rank.  The 
Curses that the Adept may inflict are dependent on 
the Rank of the spell: 

Rank Curse 

0–3 Boils 1 (+ l / Rank); Warts 1 (+1 / Rank). 
4–6 Clumsiness (-l AG); Maladroitness (-l MD). 
7–9 Weakness (-2 PS); Poor health (-3 EN). 
10–11 Cowardice (-3 WP \& +5 Fright/Awe rolls); 
Lose Smell \& Taste (B:73). 
12–13 Deafness (B:67); Lose Tactile Sense (B:75); 
Virulent Skin Disease (B:79-80). 
14–15 Insomnia (B:77); Wasting Disease (B:81); 
Periodic Hallucinations (B:88). 
16–17 Periodic Muscle Spasms (B:82-83); Asthma 
(B:93); Migraines (B:86-87). 
18–19 Creeping Senility (B:94-95); Struck Mute 
(B:71); Arthritis (B:89-90); Enfeeblement (B:91-
92). 
20 Blindness (B:63); 
Amnesia [Roll D10]: 
 1–2 Partial — Magic (B:96) 
 3–5 Partial — Skills (B:97) 
 6–7 Partial — Recent (B:98) 
 8–10 Total (B:99) 

\end{effects}
\end{spell}

\begin{spell}[G-2]{Darkness }

\range{15 feet + 15 / Rank }
\duration{15 minutes + 15 / Rank }
\multiple{100 }
\basechance{60\% }
\resist{None }
\storage{Investment, Ward }

\target{Volume }
\begin{effects}
The Adept creates a volume in which non-
magical  light  is  partially  suppressed.  The  volume 
will be 1000 (+ 500 / Rank) cubic feet, and may be 
in any one contiguous area the Adept desires, pro-
vided  that  no  dimension  is  smaller  than  one  foot. 
The entire volume must be visible and within range 
at  time  of  casting,  and  may  not  be  moved.  For 
visibility  purposes,  the  Spell  will  increase  Dark-
ness levels within the volume to 60\% + 2\% / Rank. 
Rank 20 Darkness may not be seen through. It will 
not  aid  in  providing  bonuses  for  casting  purposes, 
though  it  will  neutralise  penalties  due  to  natural 
light, to a maximum of 5\% + 1\% / Rank. The vol-
ume  counts  as  direct  shadow  for  Star  \&  Shadow 
Mages.  If  the  lighting  conditions  are  lower  than 
that provided by the spell, no effect  will be appar-
ent.  Note  that  because  light  is  only  being  sup-
pressed, it may still pass through, and no shadows 
are generated outside the volume. If it is possible to 
see  through  a  Darkness,  everything  beyond  it  is 
normally  visible.  This  spell  can  engender  silhou-
ettes of lit objects against the darkness, though not 
create  shadows.  Any  of  this  volume  may  be  over-
ridden  by  a  higher  ranked  Spell  of  Light,  or  neu-
tralised  (back  to  original  conditions)  by  an  equal 
rank. 

\end{effects}
\end{spell}

\begin{spell}[G-3]{Decay }

\range{15 feet + 15 / Rank }
\duration{Immediate }
\multiple{100 }
\basechance{50\% }
\resist{None }
\storage{Investment, Ward, Magical Trap }
\target{Object }
\begin{effects}
 The  Adept  may  cause  an  amount  of  food, 
produce or beverage to quickly age, moulder, spoil 
and  rot.  Upon casting  the  spell  the  targeted  matter 
will  decay,  causing  parasitic  fungi  to  spring  forth, 
and  an  odour  of  decay  to  prevail.  The  spell  may 
ruin up to 1 cubic foot of food and drink (+ 1 cubic 
foot / Rank). If a double or triple effect is achieved 
the amount of food that is spoiled may be doubled 
or trebled. Once affected by the spell the food and 
drink will thereafter be inedible. 

\end{effects}
\end{spell}

\begin{spell}[G-4]{Fear }

\range{15 feet + 15 / Rank }
\duration{Immediate }
\multiple{350 }
\basechance{20\% }
\resist{Active, Passive }
\storage{Investment, Ward, Magical Trap }
\target{Entity }
\begin{effects}
 The  Adept  instills  in  the  target  an  uncon-
trollable fear. Unless the target successfully resists 
they  must  roll  on  the  Fright  Table  (§54.1).  If  a 
double effect is achieved, the Adept may choose to 
modify  the  Fright  Table  roll  up  or  down  by  an 
amount  equal  to  the  rank  of  the  spell.  If  a  triple 
effect is achieved the Adept may modify the Fright 
Table  roll  by  twice  the  rank  of  the  spell.  See  the 
Fright Table for the exact results of the Fear. 

\end{effects}
\end{spell}

\begin{spell}[G-5]{Harming Entity }

\range{15 feet + 15 / Rank }
\duration{10 second + 10 / Rank }
\multiple{200 }
\basechance{20\% }
\resist{Active, Passive }
\storage{Investment, Ward, Magical Trap }
\target{Entity }
\begin{effects}
 Unless  successfully  resisted,  the  Adept 
causes  the  target  intense  pain  for  the  duration  of 
the  spell.  The  target  must  check  to  see  if  their 
concentration is broken and must subtract 10 (+ 3 / 
Rank)  from  their  Strike  Chances  whilst  suffering 
the  pain.  The  difficulty  multiplier  for  the  Concen-
tration  Check  is  dependent  on  the  Rank  of  the 
spell: 

Rank  Multiplier 

0–5 
6–10 
11–15 

3.0 
2.5 
2.0 

98 

1.0 
0.5 

16–19 
20 
No  actual  damage  is  inflicted  as  a  result  of  this 
spell.  Note  that  Mind  Mages  will  be  somewhat 
unaffected  by  this  spell,  and  may  halve  the  reduc-
tion to their Strike Chances. 

\end{effects}
\end{spell}

\begin{spell}[G-6]{Hypnotism }

\range{15 feet }
\duration{Concentration: maximum 5 minutes + 5 }
/ Rank 
\multiple{200 }
\basechance{40\% }
\resist{Active, Passive }
\storage{Investment }
\target{Entity }
\begin{effects}
The Adept may lull an entity that is within 
range into a trance-like state in which they will be 
subject  to  suggestion.  The  spell  may  not  be  cast 
over  a  totally  hostile  entity.  Once  the  subject  has 
been hypnotised, the Adept may make suggestions 
(provided that they can communicate verbally with 
the  subject)  that  will  be  readily  accepted  unless 
they  directly  conflict  with  the  subject’s  best  inter-
ests. The subject will remain suggestible so long as 
the  Adept  maintains  concentration  and  the  subject 
remains  in  range.  The  subject  will  continue  to 
implement implanted suggestions for 3 hours (+ 3 / 
Rank)  after  the  suggestions have  been  made,  even 
when no longer hypnotised. The subject will never 
have  any  idea  where  the  suggestion  that  it  is  im-
plementing came from. 

\end{effects}
\end{spell}

\begin{spell}[G-7]{Igniting Flammables }

\range{15 feet + 15 / Rank }
\duration{Immediate }
\multiple{150 }
\basechance{20\% }
\resist{Passive }
\storage{Investment, Ward, Magical Trap }
\target{Object }
\begin{effects}
 The  Adept  may  call  forth  fire  and  cause 
flammable material (cloth, paper wood, and similar 
items, but not flesh) to burst into flames. The mate-
rial  will  thereafter  burn  normally  and  the  flames 
may be extinguished by normal means. 

\end{effects}
\end{spell}

\begin{spell}[G-8]{Mind Cloak }

\range{Self }
\duration{1 hour + 2 / Rank }
\multiple{250 }
\basechance{30\% }
\resist{None }
\storage{Potion }
\target{Self }
\begin{effects}
 The  Adept  creates  a  cloak  around  their 
own mind so that their thoughts cannot be detected 
or  read.  This  spell  does  not  prevent  the  Adept’s 
presence or emotions from being detected, but their 
mind  will  simply  not  appear  to  be  there  when  an 
attempt is made to “read” it. 
\end{effects}
\end{spell}

\begin{spell}[G-9]{Protection Against Were-Creatures}

\range{15 feet }
\duration{30 minutes + 30 / Rank }
\multiple{200 }
\basechance{20\% }
\resist{None }
\storage{Investment, Ward, Magical Trap }
\target{Area }
\begin{effects}
 The  spell  creates  an  invisible  Circle  of 
Protection  with  a  radius  of  15  feet  (+  1  /  Rank) 
which  will  not  willingly  be  crossed  by  any  Were-
creature or Shapechanger in beast form unless they 
successfully  resist  the  circle’s  effects  upon  first 
encountering  it.  Even  if  the  Circle’s  effects  are 
resisted,  the  Were-creature  or  Shapechanger  will 
be  discomforted  while  within  the  Circle,  and  will 
have  their  Strike  Chances  reduced  by  10.  If  the 
circle  is  seen  with  the  use  of  Witchsight,  it  will 
appear as a glowing red circle, similar to a ring of 
fire. 

\end{effects}
\end{spell}

\begin{spell}[G-10]{Storm Calling }

\range{Works at any range }
\duration{60 minutes + 30 / Rank }
\multiple{200 }

Base Chance: 40\% 
\resist{None }
\storage{Investment, Magical Trap }
\target{Special }
\begin{effects}
 The  Adept  may  summon  any  storm  front 
which may exist anywhere in sight. Upon reaching 
the spot occupied by the Adept at the time of cast-
ing,  the  storm  front  will  slow  and  finally  cease 
moving  and  begin  a  downpour  (snow,  rain,  hail, 
sleet, or whatever else that the GM feels the clouds 
may  contain).  Generally  a  storm front  can be  seen 
for 20 to 30 miles. If no front can be seen the spell 
may  still  be  cast,  but  the  Base  Chance  is  reduced 
by 20. The storm front will take [D × 3 - 1 / Rank] 
minutes  to  arrive.  Once  the  duration  has  expired, 
the  weather  will  gradually  return to  normal  over  a 
similar amount of time. 

\end{effects}
\end{spell}

\begin{spell}[G-11]{Summoning Enchanted Creatures }

\range{5 miles + 1 / Rank }
\duration{Immediate }
\multiple{200 }
\basechance{20\% }
\resist{None }
\storage{Investment, Magical Trap }
\target{Entity }
\begin{effects}
 The  Adept  may  summon  1  enchanted 
fantastical  creature  (+  1  per  5  or  fraction  Ranks). 
Only  creatures  that  are  native  to  the  area  may  be 
summoned.  It  will  take  them  D10  minutes  (15 
seconds / Rank), minimum 1 minute, to arrive and 
they  will  be  uncontrolled  when  they  do  arrive.  If 
more  than  1  creature  is  summoned,  all  must  be  of 
the same type. 

\end{effects}
\end{spell}

\begin{spell}[G-12]{Walking Unseen }

\range{1 foot + 1 / Rank }
\duration{1 hour + 1 / Rank }
\multiple{100 }
\basechance{60\% }
\resist{None }
\storage{Potion, Investment, Ward, Magical Trap }
\target{Entity }
\begin{effects}
 The  target  of  this  spell  may  move  unno-
ticed,  not  invisible.  This  means  that  it  will  not 
transmit light. As a consequence the target will cast 
a  shadow,  which  may  or  may  not  be  noticed,  de-
pending  on  the  lighting  conditions,  etc,  and  will 
have  a  reflection  in  a  mirror  or  other  reflective 
surface.  However,  the  target  may  not  be  noticed 
even  if  another  entity  is  looking  directly  at  them. 
An entity will get a Perception check to notice the 
target if the target becomes invasive on the entity’s 
senses (e.g. standing next to the entity and putting 
their  hands  over  the  entity’s  eyes).  Note  that  a 
Crystal  of  Vision  or  similar  means  of  viewing  is 
considered  direct  viewing  and  is  affected  by  this 
spell.  If  the  target,  or  the  target’s  possessions,  are 
touched  by  another  entity,  or  an  entity’s  posses-
sions,  then  the  spell  is  broken.  Although  not  truly 
invisible,  the  target  may  be  detected  by  using 
magical  means  to  detect  invisible  entities  (e.g. 
Witchsight). 

\end{effects}
\end{spell}

\begin{spell}[G-13]{Wind Whistle }

\range{Self }
\duration{D10 hours }
\multiple{100 }
\basechance{40\% }
\resist{None }
\storage{Investment, Potion }
\target{Self }
\begin{effects}
The Adept is able to create a wind over an 
open  space  of  up  to  100  feet  (+  100  /  Rank)  in 
diameter, centred upon themselves. Outside of this 
area the wind will fade back to the prevailing wind 
over half again the distance. The wind will build up 
over  [D  -  2]  minutes  and  the  Adept  must  choose 
before  that  time  which  direction  the  wind  will 
blow.  The  speed  of  the  wind  is  determined  by  a 
D100 roll as follows: 

Dice 

Velocity 

01–10 
11–25 
26–50 

35 mph 
15 mph 
10 mph 

25 mph 
35 mph 
45 mph 

51–75 
76–90 
91–100 
The  Adept  may  add  or  subtract  a number  equal to 
the  Rank  of  the  spell  from  the  dice  roll  used  to 
determine  velocity.  This  need  not  be  done  until 
after the dice have been rolled and the result ascer-
tained.  If  a  double  or  triple  effect  is  achieved  the 
Adept  may  add  or  subtract  double  or  treble  the 
Rank  of  the  spell.  If  the  resulting  wind  is  over  30 
mph  missile  fire  will  be  affected,  reducing  Base 
Chances  by  the  (wind  speed  /  2)  but  extending 
ranges  by  a  similar  number  of  hexes  if  firing  with 
the  wind,  or  reducing  them  respectively  if  firing 
into the wind. 


\end{effects}
\end{spell}

\section{General Knowledge Rituals}

\begin{ritual}[Q-1]{Finding Familiar }

\duration{Special }
\multiple{250 }
\basechance{40\% + 4\% / Rank }
\resist{None }
\target{Animal }
\casttime{1 hour }
\material{A piece of food acceptable to the type of }
animal being summoned 
Actions: Concentration 
Concentration Check: Standard 
\begin{effects}
The Adept may attempt to summon a small 
animal that will serve them as a familiar. The type 
of  animal  is  chosen  by  the  Adept  and  may  be  any 
natural,  unenchanted,  small  animal  such  as  a  cat, 
dog, bat, rat, toad, weasel, falcon, owl, goat, mon-
key,  trout,  etc,  and  must  be  native  to  the  area  in 
which  the  summoning  is  performed.  If  the  sum-
moning is successful, an animal of the chosen sort 
will  arrive  at  the  Adept’s  location  in  (25  -  Rank) 
minutes.  The  ritual  allows  the  Adept  to  communi-
cate  with  the  animal  when  it  first  arrives.  The 
Adept must promise to feed and protect the animal. 
The  GM  should  roll  a  reaction  check  for  the  ani-
mal.  If the result is Enraged, the animal attacks, if 
Belligerent  it  leaves  immediately.  If  neither  of 
these  results  are  achieved,  the  animal  agrees  to 
serve  the  Adept  as  a  familiar.  Regardless  of  the 
result, the Ritual confers no further ability to com-
municate with the animal. If the Adept fails to feed 
the familiar on a regular basis, or mistreats it in any 
way, the familiar may run away, and a new famil-
iar  must  be  found.  The  familiar  will  serve  the 
Adept  to  the  best  of  its  ability,  warning  them  of 
danger,  and  so  forth.  If  the  Adept  is  unable  to 
communicate  with  the  familiar  magically  it  will 
attempt  to  warn  them  by  tugging  at  their  cloak, 
whimpering,  or  whatever,  as  appropriate.  If  the 
familiar is killed, the Adept suffers [D + 5] points 
of damage in the form of a magical backlash. This 
damage  may  not  be  resisted.  An  Adept  may  only 
have one familiar at any one time. A familiar is not 
an  enchanted  creature.  The  range  of  the  summon-
ing caused by this ritual is 1 mile (+1 / Rank). 

\end{effects}
\end{ritual}

\begin{ritual}[Q-2]{Tarot Reading }

\duration{Immediate }
\multiple{500 }
\basechance{Special + 3\% / Rank }
\resist{Special }
\target{Special }
\casttime{30 minutes }
\material{78 card Tarot deck }
Actions: Laying out \& reading Tarot cards 
Concentration Check: None 
\begin{effects}
 The  Adept  may  read  the  tarot  to  gain  in-
sight  and  information.  The  Tarot  may  be  used  in 
one  of  four  ways,  and  only  one  of  these  four  op-
tions may be chosen per reading. Once one of these 
options has  been  successfully  implemented, a new 
reading  must  be  begun  in  order  to  implement  an-
other.  There  is  no  Backfire  except  as  specifically 
noted. The four options are: 

Divining Aspects The Adept may use the Tarot to 
attempt to divine the Aspect or Aspects of an entity 
that is present for the entire ritual and within 5 feet 
(+1  /  Rank).  The  entity  may  actively  but  not  pas-
sively  resist  the  reading.  The  Base  Chance  of  the 
reading is 40\% and if successful, the Tarot will tell 
the  Adept  the  entity’s  basic  Aspect  (autumn  air, 
lunar, death, etc), and whether the entity is light or 
dark  aspected.  Failure  will  result  in  no  sensible 
answer and Backfire in an incorrect reading. 

Divining  Enchantment  The  Adept  may  use  the 
Tarot to attempt to determine if an entity or object 
is currently, or had been recently, under the effects 
of a spell. The object or entity must be present for 
the  entire  duration  of  the  ritual,  and  be  within  5 
feet  (+  1  /  Rank).  The  ritual  may  not  be  resisted. 
The  Base  Chance  of  the  ritual  being  successful  is 
45\%.  The  Base  Chance  is  reduced  by  5  for  every 
week  or  part  thereof  since  the  spell  that  is  being 
divined  was  cast.  Permanent  magic  (e.g.  invested 
items still with charges) or spells currently in effect 
carry  no  modifier.  The  Adept  gains  knowledge  of 
those spells that fall within their cast chance. 

If  the  Adept  can  divine  the  spell,  its  exact  name 
and  college  are  revealed.  If  the  spell  is  non-
colleged in origin, its general effects are revealed. 

Divining the Future The Adept may use the Tarot 
to  attempt  to  learn  something  about  future  events. 
The  Adept  must  decide  on  a  question  to  be  posed 
or  a  general  course  of  action  being  considered 
before  attempting  this  divination.  The  GM  may 
make  the  reading  as  simple  or  as  complex  as  they 
desire,  but  in  all  cases  the  information  gained 
should be vague. 

The  Base  Chance  of  successfully  Divining  the 
Future  is  20\%.  If  the  Adept  fails  this  option,  the 
reading  will  be  gibberish  and  obviously  a  failure, 
but  if  a  Backfire  occurs,  a  sensible  but  otherwise 
false, reading will be gained. 

Questioning  the  Dead  The  Adept  may  attempt  to 
communicate  with  the  spirit  of  a  deceased  entity 
provided that they occupy the place that the entity 
died  or  was  buried.  The  Adept  may  only  attempt 
this  if  they  are  aware  that  the  place  they  occupy 
was  the  site  of  the  entity’s  death  or  burial.  The 
Base Chance of the spirit responding to the Adept’s 
questioning is 10\%. If the spirit responds the Adept 
may  ask  it  questions  and  interpret  its  answers  by 
turning  over  cards.  Only  questions  that  can  be 
answered  with  yes  or  no  should  be  asked,  and  the 
spirit’s answer is indicated by the orientation of the 
card turned. The dead can only provide knowledge 
of  events  that  transpired  while  they  were  alive. 
Once  the  dead  initially  respond  they  will  continue 
to  answer  all  questions  until  dismissed,  or  the 
entire deck has been used. 


\end{effects}
\end{ritual}

\section{Special Knowledge Spells}

\begin{spell}[S-1]{Blessing Crops }

\range{Sight }
\duration{1 year + 1 / Rank }
\multiple{125 }
\basechance{40\% }
\resist{None }
\storage{Investment }
\target{Area }
\begin{effects}
The spell increases the richness of the soil 
of 1 acre (+ 1 acre / Rank). For the duration of the 
spell  everything  grown  in  that  soil  will  be  proof 
against locusts, droughts, flooding, frost, and other 
natural  disasters.  This  spell  will  also  dissipate  the 
effects  of  a  Spell  of  Blighting  Crops  which  has 
previously been cast on the target area of this spell.
\end{effects}
\end{spell}

\begin{spell}[S-2]{Blessing/Curse on Unborn Child}

\range{Sight }
\duration{Until birth of target’s child }
\multiple{200 }
\basechance{20\% }
\resist{Active, Passive }
\storage{Investment, Magical Trap, Potion }
\target{Entity }
\begin{effects}
 The  Adept  may  mar  or  bless  any  unborn 
child  whose  mother  is  in  sight  while  she  is  preg-
nant. The Adept may increase or decrease any one 
characteristic  of  the  child  by  1  (+  1  per  3  or  frac-
tion  Ranks).  This  spell  may  only  be  cast  on  the 
same  unborn  child  more  than  once  if  it  is  cast  by 
different Adepts, and is used on different character-
istics.  The  spell  may  raise  characteristics  above 
normal racial maximums. If cast so as to curse, it is 
a Major Curse and may only be removed before the 
child is born. Note that if this spell is made into a 
potion,  the  target  of  the  spell  is  the  imbiber.  The 
imbiber may only passively resist the effects of the 
potion’s magic. 

\end{effects}
\end{spell}

\begin{spell}[S-3]{Blessing Livestock }

\range{Sight }
\duration{1 month + 1 / Rank }
\multiple{150 }
\basechance{45\% }
\resist{None }
\storage{Investment }
\target{Livestock }
\begin{effects}
 The  spell  may  be  cast  on  up  to  5  (+  1  / 
Rank)  livestock  that  are  within  sight.  These  ani-
mals  will  then  be  resistant  to  natural  disorders, 
such  as  rabies,  dysentery,  worms,  and  hoof  and 
mouth  disease  for  the  duration  of  the  spell.  This 
spell  will  also  dissipate  the  effects  of  a  Spell  of 
Pestilence  which  has  previously  been  cast  on  the 
targets of the spell. 

\end{effects}
\end{spell}

\begin{spell}[S-4]{Blighting Crops }

\range{Sight }
\duration{1 year + 1 / Rank }
\multiple{125 }
\basechance{45\% }
\resist{None }
\storage{Investment }
\target{Area }
\begin{effects}
The spell causes 1 acre + 1 / Rank of land 
within  sight  to  become  sour  and  lose  fertility. 
There is a 20\% (+ 1 / Rank ) chance of future crops 
failing while this spell is in effect. Those years that 
the  crops  do  not  fail,  they  will  be  stunted  and  ap-
proximately  half  a  normal  yield  will  be  obtained. 
This  spell  is  a  minor  curse.  This  spell  will  also 
dissipate  the  effects  of  a  Spell  of  Blessing  Crops 
which  has  previously  been  cast  on  the  target  area 
of this spell. 

\end{effects}
\end{spell}

\begin{spell}[S-5]{Cat Vision }

\range{15 feet + 15 / Rank }
\duration{1 hour + 1 / Rank }
\multiple{100 }
\basechance{60\% }
\resist{None }
\storage{Investment, Ward, Potion }
\target{Entity }
\begin{effects}
 The  Adept  causes  the  target  to  develop 
vision  similar  to  that  of  a  cat.  Everything  will 
appear  monochromatic  (i.e.  shades  of  grey)  and  it 
is  difficult  to  accurately  estimate  distance.  The 
higher the Rank, the less of a problem this will be. 
Some  amount  of  light  must  be  present  for  this 
vision to operate. The range of the vision is 50 feet 
(+ 10 / Rank). 

\end{effects}
\end{spell}

\begin{spell}[S-6]{Controlling Animals }

\range{15 feet + 15 / Rank }
\duration{ Concentration:  maximum  1  hour  +  1  / }
Rank 
\multiple{100 }
\basechance{20\% }
\resist{Passive }
\storage{Investment }
\target{Animal }
\begin{effects}
 The  Adept  controls  the  actions  of  one 
normal  and  unenchanted  animal,  bird  or  aquatic, 
that  does  not  successfully  resist.  The  creature  will 
serve  the  Adept  as  long  as  they  maintain  their 
concentration.  If  the  Adept  chooses  to  release  the 
animal or has their concentration broken, the crea-
ture  may  attack  them  or  flee.  The  chance  to  cast 
this  spell  is  reduced  by  5  if  the  Adept  cannot 
Communicate  with  the  creature.  If  the  Adept  can-
not  make  eye  contact,  the  Base  Chance  is  also 
reduced by 5. 

\end{effects}
\end{spell}

\begin{spell}[S-7]{Converse With Animals }

\range{Self }
\duration{1 hour + 3 / Rank }
\multiple{50 }

\basechance{60\% }
\resist{None }
\storage{Investment, Potion }
\target{Self }
\begin{effects}
 The  Adept  may  communicate  with  any 
natural  and  unenchanted,  animal,  bird,  or  aquatic. 
Whether this communication is verbal or symbolic, 
and  to  what  extent  the  communication  may  be 
carried is left up to the GM’s discretion. The Adept 
must specify at the time of casting what particular 
type  of  animal,  bird  or  aquatic  is  to  be  conversed 
with. The spell must be re-cast to speak to another 
type of animal, bird, or aquatic. 

\end{effects}
\end{spell}

\begin{spell}[S-8]{Creating Plague }

\range{15 feet }
\duration{1 day + 1 / Rank }
\multiple{200 }
\basechance{20\% }
\resist{Active, Passive }
\storage{Investment, Ward, Magical Trap, Potion }
\target{Entity }
\begin{effects}
The spell infects any one target with any of 
the following diseases: 

Rank  Disease 

Measles 
Consumption 

0–5 
6–10 
11–15  Typhoid 
16–18  Bubonic Plague 
19–20  Pneumonic Plague 
The  target  will  not  die  of  the  disease,  but  will  be-
come  habitually  ill  and  all  who  come  in  contact 
with  them  (except  the  Adept  who  cast  the  spell) 
may contract a potentially fatal dose of the disease. 
In effect, the target becomes a carrier. This spell is 
a major curse. Note that if this spell is made into a 
potion,  the  target  of  the  spell  is  the  imbiber.  The 
imbiber may only passively resist the effects of the 
potion’s magic. 

\end{effects}
\end{spell}

\begin{spell}[S-9]{Creating Restorative }

\range{Touch }
\duration{Immediate }
\multiple{200 }
\basechance{30\% }
\resist{None }
\storage{Potion }
\target{Water }
\begin{effects}
 The  spell  creates  out  of  drinkable  water  a 
potion  which,  when  imbibed,  subtracts  2  from 
Endurance and repairs 

4 lost Fatigue. The amount subtracted from Endur-
ance  is  increased  by  1  and  the  amount  of  Fatigue 
repaired is increased by 2 per Rank. The fatigue so 
restored  may  have  been  lost  through  damage  or 
tiredness,  including  spell  casting.  The  potion  will 
only  restore  lost  Fatigue.  This  spell  can  be  pre-
pared in two ways: 

1.  The  Adept  can  turn  water  into  a  restorative 
potion that will last 2 minutes (+2 / rank). 

2.  The  Adept  may  spend  an  hour  and  burn  oils 
costing  500sp  to  make  a  potion  with  the  same 
effects that will last indefinitely. 

The effects of drinking the potion may be resisted. 
The Endurance damage caused by this potion may 
be healed by normal means. 

\end{effects}
\end{spell}

\begin{spell}[S-10]{Damnum Magnatum }

\range{20 feet + 15 / Rank }
\duration{Special }
\multiple{600 }
\basechance{5\% }
\resist{Active, Passive }
\storage{Investment, Ward, Magical Trap }
\target{Entity }
\begin{effects}
The Damnum Magnatum is a Major Curse 
and may take one of three forms, as chosen by the 
Adept.  The  Damnum  Magnatum  may  normally 
only  be  removed  by  the  use  of  a  Remove  Curse 
Ritual,  by  a  Counterspell  cast  by  the  Adept  that 
laid  the  curse,  or  by  the  death  of  the  target.  This 
spell may not be dissipated. 

100 

Affliction The Adept may choose to torment or kill 
their  target.  If  the  effects  of  the  Affliction  are  in-
tended  to  be  deadly,  the  target  may  not  die  as  a 
result  of  the  curse  before  (24  -  Rank)  hours  have 
passed.  The  Adept’s  player  states  what  the  Afflic-
tion is to do, and then the exact effects and results 
must  be  decided  by  the  GM.  In  addition  to  the 
normal ways of lifting a curse, afflictions may have 
durations or conditions worded into them, in which 
case the curse is lifted when the duration expires or 
the  condition  is  met.  Players  should  note  that  Af-
flictions  are  particularly  capricious,  and  can  never 
be  relied  upon  to  operate  in  precisely  the  same 
manner twice. Some sample Afflictions are: 

1. Target begins to age at 10 years per day. Target 
may  die  of  old  age.  Once  the  curse  is  lifted  the 
target  will  age  backwards  to  their  correct  age,  at 
the same rate. 

2.  Target  contracts  a  deadly  disease  (including 
open  running  sores)  that  may  not  be  cured  by  the 
arts of a Healer. 

3.  Target  is  transformed  into a  frog  or  other  small 
creature (but retain their own mind). Condition: the 
Curse  may  be  lifted  by  the  kiss  of  a  member  of 
royalty of the opposite gender. 

4.  Target  is  cursed  with  Lycanthropy  (random 
species). 

5.  Target  will  fall  into  a  century  long  sleep  (see 
Hibernation, College of Ice Magics S-6). 

Ill Luck Add two times the Rank of the Adept with 
this  spell  to  any  percentile  dice  roll  involving  the 
target’s  use  of  their  abilities.  This  may  never  be 
applied favourably. Note that this is an addition to 
the dice roll, not a subtraction from Base Chances. 

Doom A Doom is a pronouncement, by the Adept, 
upon an event that will occur in the target’s future, 
such as: “You will die by the hand of a loved one.” 
The statement, which must be indefinite, will come 
true in not less than (24 - Rank) weeks. The Doom 
remains  until  it  is  fulfilled,  and  may  not  be  re-
moved  by  a  Remove  Curse  Ritual,  or  even  by  the 
death  of  the  target,  unless  that  death  fulfils  the 
Doom.  The  target  is  immediately  aware  of  the 
nature of the Doom, and its wording. A Doom may 
be  modified,  so  as  to  decrease  the  severity,  make 
the  time  factor  longer,  etc.,  by  the  casting  of  a 
modified  Doom  on  the  same  target,  by  an  Adept 
with Rank in this spell at least equal to the Rank at 
which  the  original  Doom  was  cast.  The  exact  ef-
fects of the Doom must be decided by the GM, and 
players  should  note  that  two  Dooms,  even  if 
worded the same, need not have precisely the same 
effects. 

\end{effects}
\end{spell}

\begin{spell}[S-11]{Earth Tremor }

\range{15 feet + 15 / Rank }
\duration{5 seconds + 5 / Rank }
\multiple{350 }
\basechance{20\% }
\resist{None }
\storage{Investment, Ward, Magical Trap }
\target{Area }
\begin{effects}
 By  the  use  of  this  spell  the  Adept  causes 
the  very  earth  to  pitch  and  roll  uncontrollably  as 
though  in  a  tremendous  earthquake.  The  area  that 
may be affected is a one hex area of ground (+ 1 / 
Rank). Any Entities within the Area  must roll less 
than or equal to 1 × AG to retain their footing. 

Those  who  fail  to  remain  standing  fall  prone  im-
mediately  and may  not  rise  for  the  duration  of the 
tremor. Objects within the Area will tend to topple 
and roll around. If the spell is cast under part of, or 
all of, a building, wall,  or other such construction, 
significant  structural  damage  will  occur,  probably 
causing partial or total collapse. 

\end{effects}
\end{spell}

\begin{spell}[S-12]{Hex }

\range{15 feet + 15 / Rank }
\duration{1 day + 1 / Rank }
\multiple{300 }
\basechance{20\% }
\resist{Passive }

Storage: Investment, Ward, Magical Trap 
\target{Entity }
\begin{effects}
 By  use  of  this  spell,  the  Adept  curses  the 
target with ill-fortune. Unless the target resists, all 
their  Base  Chances,  Strike  Chances,  and  their 
Magic  Resistance  are  reduced  by  the  Rank  of  the 
spell (1 if unranked). This spell is a minor curse. 

\end{effects}
\end{spell}

\begin{spell}[S-13]{Hellfire }

\range{10 feet + 5 / Rank }
\duration{Immediate }
\multiple{650 }
\basechance{5\% }
\resist{Active, Passive }
\storage{Investment, Ward, Magical Trap }
\target{Entity }
\begin{effects}
This sulphurous fire attacks 1 human-sized 
target for every 3 (or fraction) Ranks. The target’s 
Magical  Resistance  is  reduced  by  5  (+  1  /  Rank). 
The  spell  does  D10  (  +  2  /  Rank)  damage  to  each 
target.  If  a  target  successfully  resists,  they  suffer 
only  half  damage  (round  up).  Double  damage  add 
an  additional  1  /  Rank  damage  and  triple  damage 
adds an additional 2 / Rank damage. 

\end{effects}
\end{spell}

\begin{spell}[S-14]{Instilling Flight }

\range{Touch }
\duration{ Concentration:  maximum  30  minutes  + }
30 / Rank 
\multiple{350 }
\basechance{20\% }
\resist{None }
\storage{None }
\target{Object }
\begin{effects}
 This  spell  enables  the  Adept  to  instil  a 
possession  of  up  to  5  lbs  (+5  /  Rank)  with  the 
power  of  flight.  The  spell  will  dissipate  if  the  ob-
ject  stops  being  a  possession  of  the  Adept,  the 
Adept  loses  concentration,  or  if  the  object  is  bro-
ken.  The  Adept  may  cause  the  object  to  fly  at  20 
miles  per  hour  (+  2  /  Rank).  It  will  take  off  and 
accelerate  up  to  full  speed,  or  halt  and  land,  in  a 
single pulse. The object may support 150 lbs (+ 50 
/  Rank)  in  addition  to  its  own  weight.  Naturally 
flexible or fragile items gain sufficient strength and 
rigidity  to  support  the  load.  Any  object  or  entity 
that falls from the flying object will  move off in a 
random  direction.  If  the  object  is  about  to  crash 
into  a  surface,  it  will  attempt  to  land,  although 
some  surfaces  may  be  inappropriate  for  this  (lava, 
sheer walls, etc.). 

\end{effects}
\end{spell}

\begin{spell}[S-15]{Mass Fear }

\range{10 feet + 15 / Rank }
\duration{30 seconds + 10 / Rank }
\multiple{400 }
\basechance{10\% }
\resist{Passive }
\storage{Investment, Ward, Magical Trap }
\target{Area }
\begin{effects}
 The  spell  instills  in  all  entities  within 
range,  other  than  the  Adept,  and  those  who  suc-
cessfully  resist,  an  unreasoning  and uncontrollable 
fear.  All  entities  that  fail to  resist  must  roll  on  the 
Fright Table (see §54.1). 

\end{effects}
\end{spell}

\begin{spell}[S-16]{Pestilence }

\range{Sight }
\duration{1 month + 1 / Rank }
\multiple{150 }
\basechance{45\% }
\resist{Special }
\storage{Investment }
\target{Livestock }
\begin{effects}
 The  spell  may  be  cast  on  up  to  5  (+  1  / 
Rank) livestock that are within sight. All livestock 
so  cursed  that  do  not  resist  (individually)  are  in-
fected (see §4.7). Any new stock which comes into 
contact with the infected stock while the curse is in 
effect  must  also  resist  (individually)  or  become 
infected.  This  spell  is  a  minor  curse  on  each  indi-
vidual. This spell will also dissipate the effects of a 
Spell  of  Blessing  Livestock  which  has  previously 
been cast on the targets of this spell. 

\end{effects}
\end{spell}

\begin{spell}[S-17]{Skin Change }

\range{Touch }

\duration{Until dispelled by the appropriate coun-}
terspell 
\multiple{350 }
\basechance{30\% }
\resist{Passive }
\storage{Special }
\target{Entity }
\begin{effects}
 The  Adept  may  enchant  an  animal  pelt  or 
skin so that anyone  who touches, or  is touched by 
the  “inside”  will  turn  into  the  type  of  animal  to 
whom  the  pelt  originally  belonged,  but  will  retain 
their  own  mind  and  memories.  The  spell  is,  in 
effect,  stored  in  the  pelt  or  skin,  and  may  be  re-
tained unused for an length of time dependent upon 
the Rank of the spell: 

Rank  Duration 

1 week (+ 1 / Rank) 
2 weeks (+ 2 / Rank) 
1 month (+ 1 / Rank) 
permanent until used 

0–6 
7–12 
13–19 
20 
The wearer of the pelt may only resume their own 
form  by  having  the  Special  Knowledge  Counter-
spell  of  the  College  of  Witchcraft  cast  over  them. 
The  pelt  is  destroyed  by  the  process  of  returning 
the wearer to their original form. This spell may be 
used  to  enchant  the  pelts  or  skins  of  sentient  enti-
ties, but if used for that purpose its Base Chance is 
reduced  by  20\%.  Note  that  if  a  backfire  result  is 
achieved when casting this spell, the pelt or skin is 
destroyed in addition to any other backfire effects. 
The  Skin  Change  spell  itself  may  be  stored  by 
Investment  (so  that  it  may  later  be  triggered  on  a 
pelt) but not by other means. The “inside” of a pelt 
is the side that was closest to the animal’s body. 

The  skin  must  be  applied  to  the  target  for  at  least 
30  seconds  before  they  will  change,  where  upon 
the target gets to resist its effects. 

\end{effects}
\end{spell}

\begin{spell}[S-18]{Virility }

\range{5 feet + 5 / Rank }
\duration{2 hours + 1 / Rank }
\multiple{100 }
\basechance{30\% }
\resist{None }
\storage{Investment, Ward, Magical Trap, Potion }
\target{Male Entity }
\begin{effects}
 The  spell  is  cast  over  a  male  entity  and 
greatly  increases  the  target’s  virility.  In  addition 
the  chance  of  the  target’s  female  partner  conceiv-
ing is increased by 5 (+ 5 / Rank). Note that if this 
spell is made into a potion, the target of the spell is 
the imbiber. 

\end{effects}
\end{spell}

\begin{spell}[S-19]{Wall of Thorns }

\range{15 feet + 15 / Rank }
\duration{30 minutes + 30 / Rank }
\multiple{150 }
\basechance{20\% }
\resist{None }
\storage{Investment, Ward, Magical Trap }
\target{Area }
\begin{effects}
 The  Adept  causes  a  wall  of  tangled  briars 
and  thorns  to  spring  forth  from  the  ground.  The 
wall  is  10  feet  high  ×  20  feet  long  ×  2  feet  thick. 
The  Adept  may  increase  the  length  or  height by  1 
foot per Rank. The wall may not be cast on top of 
an entity. A human-sized hole may be made in the 
wall  by  causing  20  points  of  damage  in  one  area. 
This  damage  need  not  come  from  a  single  attack. 
All  “A”  class  weapons  and  most  “C”  class  will 
have  little  effect  on  the  tough  springy  vines.  An 
entity forcibly pressed against the wall, or attempt-
ing to force their way through it will suffer [D + 2] 
damage  for  every  Pulse  they  are  so  engaged.  This 
damage  is  entirely  physical,  and  armour  may  pro-
tect against it. An entity can normally force its way 
through  the  wall  in  6  Pulses.  The  wall  is  very 
dense, and may not be seen through. 


\end{effects}
\end{spell}

\section{Special Knowledge Rituals}

\begin{ritual}[R-1]{Controlling Weather }

\duration{8 hours × Rank (minimum 1) }
\multiple{300 }
\basechance{30\% + 3\% / Rank }
\resist{None }
\target{Area }
\casttime{1 hour }
\material{None }
Actions: Dance 
Concentration Check: None 
\begin{effects}
The Adept may change one or more of the 
three  components  that  make  up  the  weather  by 
performing a ritual dance. The three components of 
weather are: Precipitation / Cloud Cover, Tempera-
ture,  and  Wind.  The  GM  should  consult  the 
Weather Table and advise the player of the current 
level  of  each  of  the  three  components  before  they 
start  dancing.  The  Adept  may  change  one  compo-
nent by Rank / 2 (round down), or two components 
by Rank / 3 (round down) levels each, or all three 
components by Rank / 4 (round down) levels each. 
The  changes  are  independent  and  may  be  in  any 
direction.  The  weather  will  change  gradually  over 
30 minutes (l / Rank) per level shifted, and all three 
components  will  change  simultaneously.  The  area 
of the effect is circular and the diameter is 2 miles / 
Rank (minimum 2). This ritual counts as Strenuous 
activity and the Adept will lose Fatigue. This ritual 
may not Backfire. 

\end{effects}
\end{ritual}

\begin{ritual}[R-2]{Creeping Doom }

\duration{Special }
\multiple{450 }
\basechance{20\% + 4\% / Rank }
\resist{Special }
\target{Entity }
\casttime{2 hour }
\material{13 bones }
Actions: Carving bones 
Concentration Check: None 
\begin{effects}
The Adept creates 13 Rune-bones by carv-
ing  the  appropriate  maledictions  into  bones  from 
an entity of the same race as the target. The Adept 
then  buries  the  bones  beneath  the  dwelling  of  the 
entity  that  they  wish  to  curse.  It  is  best  if  the  vic-
tim’s  name  is  carved  in  the  bones  as  well.  If  the 
intended victim’s name is not carved on the bones, 
and there are 1 or more other entities inhabiting the 
dwelling, there is a 20\% (- 1 per Rank) chance that 
the  curse  will  settle  on  someone  other  than  the 
intended  victim.  For  each  month  that  the  bones 
remain in or under the victim’s dwelling, they must 
make  a  Resistance  Check,  the  Base  Chance  for 
which  is  composed  of  the  victim’s  Endurance 
multiplied by the Difficulty Multiplier of the resis-
tance. The Difficulty Ratings are: 

Rank  Multiplier 

4.0 
3.0 
2.5 
2.0 
1.5 

0–5 
6–10 
11–15 
16–18 
19–20 
If  the  victim  fails  to  resist,  they  suffer  a  wasting 
disease and lose [D - 4] Endurance for purposes of 
future  resistance  (only).  If  they  fail  to  resist  for 
three straight months they die. 
\end{effects}
\end{ritual}

\begin{ritual}[R-3]{Dead Man’s Candle}

\duration{Special }
Experience  Multiple:  None
Base  Chance:  Automatic 
\resist{None }
\target{Materials }
\casttime{Variable }
\material{As detailed }
Actions: As detailed 
Concentration Check: None 
\begin{effects}
By means of this ritual the Adept creates an horrific and evil candle.
Only the darkest of the Wicca would ever perform this Ritual.  The
Adept makes a Dead Man’s candle by severing the left hand of a
convicted murderer who has been hung.  The hand must be severed during
a full moon and wrapped in a burial shroud. It must then be dried in
the sun until desiccated. The Adept must render the fats and oils from
the body of a stillborn baby, so that the hand can be coated with them
and a candle made.  The wick of the candle is then made from the hair
of the same murderer.  The Adept says words of darkest power over the
candle.  Thereafter, it may be lit as part of any spell or ritual of
this College and will increase the chance that the spell or ritual is
successful by 20, provided that the ritual is being performed with
malign intent.  This Ritual may not be Ranked, and it always works, if
it is correctly performed.  A Dead Man’s Candle will burn for about 10
hours before it is no longer usable, and may be extinguished and relit
an indefinite number of times.
\end{effects}
\end{ritual}

\begin{ritual}[R-4]{Hand of Glory }

\duration{Permanent }
Experience  Multiple:  None
Base  Chance:  Auto-
matic 
\resist{None }
\target{Severed hand }
\casttime{Variable }
\material{Murderer’s hand }
Actions: As detailed 
Concentration Check: None 
\begin{effects}
This gruesome ritual creates an amulet of great and malign potency.
Many Wicca consider this ritual to be evil and it would certainly
never be studied or performed by a “White Mage”.  To successfully
perform the ritual, the Adept must sever the right hand of a convicted
murderer who has been hung.  The hand may only be severed during the
new moon and must be wrapped in a winding sheet. It must then be dried
in the sun and the blood entirely removed.  When the desiccated hand
is worn as an amulet, thereafter, it will subtract 10 from the Cast
Check of creating any Plague, Blight, or Curse.  This Ritual may not
be Ranked, and it always works if it is correctly performed.
\end{effects}
\end{ritual}

\begin{ritual}[R-5]{Summoning Animals }

\duration{Immediate }
\multiple{150 }
\basechance{MA + 5\% / Rank }
\resist{None }
\target{Animals }
\casttime{1 hour }
\material{None }
Actions: Concentration 
Concentration Check: Standard 
\begin{effects}
The Adept may summon a number of small animals equal to the Rank of
the ritual (1 if unranked).  The animals that the Adept attempts to
summon must be native to the area.  The animals are not controlled in
any way when they arrive.
\end{effects}
\end{ritual}

\end{Chapter}
