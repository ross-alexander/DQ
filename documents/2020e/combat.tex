\begin{Chapter}{Combat (Ver 1.1)}

There are nineteen sections in Combat: 

6.1   Definitions \\
6.2   Preparation for Combat \\
6.3   Combat Sequence \\
6.4   Engaged Actions \\
6.5   Close Combat Actions \\
6.6   Unengaged Actions 
6.7   Free Acts \\
6.8   Action Restrictions \\
6.9   Attacking \\
6.10   Resolving Attempted Attacks \\
6.11   Damage \\
6.12   Effects of Damage \\
6.13   Weapons \\
6.14   Unarmed Combat \\
6.15   Multi-hex Figures \\
6.16   Mounted Combat \\
6.17   Aerial Combat \\
6.18   Aquatic Combat \\
6.19   Magical Combat

Individual Combat is an inevitable and sometimes necessary occurrence,
and characters should be aware of its dangers.  Fighting is a deadly
process and should be avoided if at all possible. Heroes are made by
defeating the dragon, but more graves are dug than heroes made.  The
DragonQuest combat system reflects these dangers and emphasises skills
and smarts over brawn and brutality.

When combat has begun, the players should place the figures
representing their characters on the Tactical Display, with the GM
determining their final position. The hostile figures are placed by
the GM, and the Combat Sequence begins.

Combat time on the Tactical Display is divided into five second Pulses
during which all figures may attempt to take an action, depending on
their position relative to hostile figures.  The order in which these
actions take place is determined by the figures’ engaged or unengaged
Initiative values.  Attacks are resolved by comparing the attacker’s
Modified Strike Chance with a percentile roll.  A successful hit does
D10 damage, plus any bonus for weapon type and skill.

When all figures on the Tactical Display are dead, unconscious,
incapacitated, or friendly to each other, the combat is finished.
Combat should never last longer than necessary to resolve the
situation; returning to normal interactive roleplaying will speed
overall play.

\section{Definitions}

\begin{Description}
\item[Action] The movement and combat activity a figure may attempt
  during one pulse.

\item[Attacker] The figure performing the Action currently being
  resolved.

\item[Attack] Zone Any hex into which a figure may attack in Melee or
  at Range.

\item[Base Chance] The base percentage chance of hitting with a
  weapon, as listed in the Weapon Chart (§56.1).

\item[Blocked Hex] A hex which contains enough solid material to block
  any attack.  A Blocked Hex is never part of an Attack Zone.

\item[Cast] A Magical Action, used to Cast magic.

\item[in Close] A figure in the same hex as a hostile figure is in
  Close with the opponent.

\item[Damage Check] A roll on a D10 to determine the amount of damage
  done after a successful hit. This roll is modified by the weapon,
  the Rank or Physical Strength of the wielder, circumstances and
  magic.

\item[Damage Points] The number of points of damage done as a result
  of a damage check.

\item[Defence] The amount that a target may subtract from an
  Attacker’s Strike Chance, determined by Modified Agility, a shield,
  magic and conditions.

\item[Effective Damage] Any Damage Points (to either Fatigue or
  Endurance) that are actually inflicted on the figure hit after
  allowing for absorption due to armour or magic.

\item[Engaged] A figure who is in the Melee Zone of an opponent, or is
  in Close, is engaged.

\item[Tactical Display] The area to which a combat is confined,
  assumed to be covered with a grid of hexes.

\item[Tactical Movement Rate (TMR)] The maximum number of hexes that a
  figure may move in a single action, determined by Modified Agility
  and Race.

\item[Engagement] A group of adjacent figures, all of who are engaged
  with each other.

\item[Target] The figure on the receiving end of any Attacker’s
  action.

\item[Facing] A figure must be unambiguously oriented towards one hex
  side.  This determines their Front and Rear Hexes, and Attack Zone.
  They may change facing during any Action unless specifically
  prohibited.

\item[Figure] Any entity or combatant occupying the Tactical Display.

\item[Grapple] An attack in Close Combat.

\item[Grievous Injury] An injury that results in specific damage to a
  body part.

\item[Hex] A 5 foot diameter hexagonal area, with sufficient room for
  a figure to fight in Melee.

\item[Initiative] Engaged Initiative determines the order of
  individual actions within an Engagement. Unengaged Initiative
  determines the order in which entire sides of Unengaged figures act.

\item[Line of Fire] A straight line drawn from the centre of an
  Attacker’s hex to a target’s hex that is in the Attacker’s Ranged
  Zone.

\item[Melee Zone] The front hexes of any conscious, unstunned,
  standing or kneeling figure armed with a prepared weapon rated for
  Melee combat.

\item[Modified Agility] A figure’s Agility after it has been modified
  due to weight carried (see §58.1), armour worn (see §56.3) and
  circumstance.

\item[Modified Manual Dexterity] A figure’s Manual Dexterity after it
  has been modified due to the type of shield carried (see §56.2) and
  circumstance.

\item[Modified Strike Chance] The percentile chance to successfully
  hit a target after the target’s Defence and all Strike Chance
  Modifiers (see §57.1) have been taken into account.

\item[Obscured Hex] A hex which a figure cannot see into, but through
  which objects may pass.

\item[Pass Action] Any preparatory Action that does not directly
  affect another figure and is not otherwise covered by specific
  Actions.

\item[Preparing a Spell] A type of Magical Pass Action.

\item[Prepared Item] Any item (weapon, shield, flask, etc.)  that a
  figure has in their hands and may immediately use.

\item[Pulse] A five second period of game time that regulates Actions
  on the Tactical Display.

\item[Ranged Zone] The hexes radiating out from a figure’s front hexes
  into which that figure may see and fire a missile weapon.

\item[Sheltered Hex] A hex which contains enough solid material for a
  target to shelter behind such that approximately half of their body
  is protected from any attack.

\item[Strike] Any Action that attempts to hit a figure.

\item[Strike Chance] The standard percentage chance of hitting with a
  weapon; it is a combination of Base Chance, Manual Dexterity, Rank
  and magic.

\item[Strike Check] The percentile roll against an Attacker’s Modified
  Strike Chance to see if a Strike was successful.

\item[Stun] A figure who takes sufficient Effective Damage in a single
  blow is Stunned, and may not attempt to perform any action except
  Recover from Stun (see §6.8).

\item[Unengaged] A figure who is not adjacent to an opponent, or who
  is not in an opponent’s melee zone and chooses to be unengaged.

\item[Weapon] Any item used to Strike a figure. 

\end{Description}
  
\section{Preparation for Combat}

\subsection{Paperwork}

Character Sheets for all combatants should be prepared before the
combat.  These contain information that will be used continuously
during combat, such as Strike Chances, Initiatives, Movement Rates,
etc. A player is responsible for ensuring the completeness and
accuracy of the information on their Character Sheet, while the GM
should prepare this information for all NPCs in advance.  Any damage
or other losses in combat should be recorded as the combat
proceeds. Percentile dice and copies of all charts and tables should
be available for ready usage.  Lead figures or counters for all
combatants should also be available, as these add to the clarity and
excitement of a combat.

\subsection{Surprise}

One side in a combat may gain a free pulse of activity if it surprises
the other.  If one side in a combat is unaware of either the
opposition’s intent or their location, they are surprised unless the
figure with the highest effective Perception succeeds in making a
Perception Check. This Perception check is modified by both Ranger
Detect Ambush and the Sense Danger Talent.

Prior to placing any figures on the Tactical Display, the GM should
determine whether surprise exists.  If one side is surprised, they
should place their figures in a way that represents their lack of
readiness.  If no surprise exists, the players may place their figures
as they choose, then the GM places the opposing figures, with the GM
having the final say on all placements.  As a guideline, if there is
surprise, the distance between the two parties should not exceed 8
hexes, while if there is no surprise, the distance should not be less
than 8 hexes.

If surprise exists, the party with the advantage may have a free pulse
of activity without the surprised party being able to respond;
otherwise, the normal Combat Sequence starts.

\subsection{Fatigue}

After placing all the figures on the Tactical Display the GM must
assign any Fatigue losses the figures may have incurred as a result of
their actions prior to combat.  For player characters, this should
have been recorded as they slept, rested, travelled, cast magic or
attempted other fatiguing actions. For all NPCs, the GM should make a
quick estimate relating all presumed activity they may have undertaken
up to the start of combat.

\section{Combat Sequence}

The order in which all actions are attempted in a pulse is called the
Combat Sequence.  This sequence should be adhered to at all times as
this will greatly speed play. Each pulse, events occur in the
following order:

\begin{Enumerate}
\item Unengaged Initiative is determined for each side.

\item If any figures are engaged or in Close, these figures are
  grouped into Engagements, and each Engagement is dealt with
  separately.

\item In each Engagement, engaged Initiative is determined, and then
  the figures act in order of Initiative (highest to lowest), each
  performing one Ac- tion from the Engaged Actions list.

\item The winner of the unengaged Initiative now resolves the Actions
  (selected from the Unengaged Actions List) of all their unengaged
  figures, in any order they choose.  All their actions must be
  resolved before any figures on the opposing side may act.

\item Remaining unengaged figures may act as in Step 4.

\item End of Pulse activity occurs.  This may include an additional
  stun recovery attempt for figures that were Stunned during the Pulse
  and any housekeeping.

\end{Enumerate}

\subsection{Exceptions to the Combat Sequence}

\begin{Description}

\item[Engaged Figures becoming Unengaged] If a figure becomes
  unengaged before their engaged Action is resolved, they must act on
  their unengaged Initiative.  If they become unengaged after they
  have had their engaged Action, they do not gain an extra Action.

\item[Unengaged Figures becoming Engaged] If a figure becomes engaged
  before their unengaged Action is resolved, they must select an
  engaged Action on their unengaged Initiative.

\item[Optionally Engaged Figures] A figure who is adjacent to a
  hostile figure, but is not in any opponent’s Melee Zone, may choose
  whether to be treated as engaged or unengaged, and acts on the
  appropriate Initiative.

\item[Extraordinarily Agile Figures] A figure who has a modified
  Agility of 26 or more may perform two Actions on their
  initiative. Their choice of Actions is restricted. See §6.8.

\item[Stunned Figures] If a figure is Stunned before their Action,
  they may attempt to Recover from Stun as their Action.  This takes
  place when specified for Stunned figures in the Initiative Section
  below. If a figure is Stunned during the Pulse they may attempt to
  Recover from Stun at the end of the Pulse in which they were
  Stunned, regardless of whether they acted or not.

\end{Description}

\subsection{Initiative}

\begin{Description}
  
\item[Engaged Initiative] A figure’s engaged Initiative Value is their
  modified Agility + Perception + their Rank with prepared weapon +
  Warrior bonus.  If the figure has no prepared weapon, they may use
  their Unarmed Rank.  Any ties indicate simultaneous combat.  In each
  Engagement, figures with faster initiative may choose to act after
  figures with lower initiative, but all engaged Actions must be
  resolved before any unengaged Actions.  If an engaged figure is not
  in any opponent’s Melee Zone, they may act first in the engagement.
  If a figure is Stunned they act last in their Engagement.

\item[Unengaged Initiative] A side’s unengaged Initiative is their
  leader’s Perception + D10. If the leader is a Military Scientist,
  they gain a bonus to this roll. Any ties are re-rolled. The leader
  may choose to have their entire party act after a slower side.  This
  decision is made before any engaged actions are resolved. If a
  side’s leader is engaged, Stunned or otherwise incapacitated at this
  point, another character may assume this role. All figures who are
  under a single Leader take their Actions in the same Initiative, in
  any order that they find convenient.  A Stunned figure always acts
  last in their side.

\end{Description}

\subsection{Action Timing}

When a figure chooses an Action, they are assumed to be performing
that Action until they start a new action or are stunned. However, any
Action which requires a dice roll to resolve is completed when that
dice roll is made; the figure is assumed to be engaged in follow-up
manoeuvres until their next action.  After successfully recovering
from Stun, a figure is assumed to have just finished a Pass Action
until their next Action, for all purposes, and any previously prepared
shield or weapons are once again prepared.

\subsection{GM Conventions}

These are only conventions; the GM may modify these conventions to
suit their own style.

\begin{Description}
  
\item[Announcements of Intent] When combat occurs on the Tactical
  Display, there should be no lapses of time between player
  announcements of character intentions and resolution of them.  When
  it is a character’s turn to take action, the player must announce
  within 5 seconds what their character will do, or the character will
  take a Pass Action.  The GM should restrict themselves to a similar
  guideline for their NPCs. A player may change the action they
  announced for their character to a Pass Action (only) up to five
  seconds after they announce it.

\item[Discussions during Combat] If the players wish to discuss
  tactics amongst themselves, they must do so quietly while the GM is
  determining the result of a particular action. Anything said by one
  character to another during combat may be overheard.  A Military
  Scientist character may allow a party a Time-out during which they
  will not be overheard.

\item[Rule Clarification] Any player may, at the GM’s discretion,
  suspend the passage of time by requesting a clarification of a
  relevant point by the GM.  They may also appeal a decision made by
  the GM that they feel to be arbitrarily or improperly resolved.  The
  player has as much time as the GM may grant.  The GM may modify or
  reverse their decision, or stand behind it.  The GM’s word is always
  law.

\end{Description}

\section{Engaged Actions}

Being engaged imposes certain limitations on the actions that a figure
may attempt.  The primary restriction is that an engaged figure may
not move out of the Melee Zone of an opponent except in special
circumstances.  If an engaged figure is in Close Combat, their range
of Actions is further restricted.

The order in which the Actions of engaged figures is resolved is
determined by each figure’s engaged Initiative.

\begin{Description}

\item[Melee Attack] An engaged figure may move one hex and change
  facing, and then attempt a Melee Attack. They may not move after
  they have attacked.

\item[Close \& Grapple] An engaged figure may attempt to Close \&
  Grapple. If the Attacker is within the Melee Zone of the target,
  they may try to Repulse the Attacker’s attempt to Close by rolling
  less than or equal to their prepared Melee weapon Rank on a D10.
  Multihex figures may not be Repulsed, but the defending figure may
  avoid the attack by succeeding in a 1 × AG check.

  A successful Repulse means that the target has interposed their
  weapon between themselves and the Attacker, and the action has
  failed.  If the Re- pulse is unsuccessful, the Attacker may go into
  Close and make a normal Grapple attack.

\item[Evade] An engaged figure may move one hex and change facing
  while executing an Evade. If a Melee Attack is attempted on a figure
  who is Evading with a Ranked weapon, they may be able to Parry the
  attack.  An Evading figure receives a bonus to Defence versus Melee
  and Ranged Attacks.

\item[Offensive Withdraw] An engaged figure may make a Melee Attack
  with a -20 penalty to their Strike Chance.  They may then move one
  hex and change facing.  They may freely leave the Melee Zone of any
  opponent, but may not move into that opponent’s rear hex.

\item[Defensive Withdraw] An engaged figure may solely defend,
  increasing their defence by 20. They may then move one hex and
  change facing.  They may freely leave the Melee Zone of any
  opponent, but may not move into that opponent’s rear hex.

\item[Flee] An engaged figure who does not have an opponent in their
  Melee Zone may Flee.  This allows them the same options as an
  unengaged Move.  Any opponent able to Melee attack the figure
  automatically receives Initiative.

\item[Pass] An enaged figure may move one hex and change facing while
  performing a Pass Action.

\item[Cast] An engaged figure may change facing but not move while
  attempting to Cast a Spell. Casting is a Magical Action.  Like all
  other actions, Casting is resolved on the figure’s initiative.

\item[Throw]An engaged figure may change facing but not move while
  attempting to Throw a weapon.  They may only Throw into their Ranged
  Zone where they have a Line of Fire.

\item[Recover from Stun] An engaged figure who is stunned may attempt
  to recover from Stun.  They may not make any movement or change
  facing.

\end{Description}

\subsection{Leaving Melee Combat}

A figure engaged in Melee Combat may only leave Melee Combat by (i)
executing a Close \& Grapple, (ii) executing a Withdraw or Flee, or
(iii) by stunning or otherwise incapacitating all opponents who have
the figure in their Melee Zones.  Other Ac- tions may never take the
figure out of any hostile Melee Zone.

\section{Close Combat Actions}

All figures in Close Combat are treated as engaged.  However, while in
Close, only close-rated weapons may be employed. All other weapons or
items must be dropped immediately. Figures in Close Combat are treated
as prone, and thus have no Melee Zone or facing.

If an engaged figure is in Close Combat, their Action is limited to
one of the following:

\begin{Description}

\item[Grapple] A figure engaged in Close Combat may neither move nor
  change facing while attempting a Grapple.  A Grapple is an attack
  with any close-rated weapon (including Unarmed).

\item[Withdraw from Close] A figure may attempt to Withdraw from Close
  Combat. If they are successful, they may move one hex, but are still
  treated as prone.  A figure may Withdraw from Close if a D10 roll
  plus any positive difference in total Physical Strength between the
  friendly and hostile figures in the hex is at least 10.

\item[Pass] A figure engaged in Close Combat may neither move nor
  change facing while taking a Pass Action.  They may not attempt a
  Magical Pass Action or Multi-Pulse Action.  Some other Pass Actions
  will be impossible, as they are effectively prone.

\item[Recover from Stun] A figure engaged in Close Combat who is
  stunned may attempt to recover from Stun as their Close Combat
  Action.

\end{Description}

\subsection{Leaving Close Combat}

A figure engaged in Close Combat may leave Close Combat by (i)
executing a Withdraw from Close, or (ii) by stunning or otherwise
incapacitating all opponents who are in Close with the figure.

\section{Unengaged Actions}

An unengaged figure is one that is not engaged. 

\begin{Description}

\item[Move] An unengaged figure may move any number of hexes up to
  their TMR. During movement, a figure may change facing as desired.

\item[Step \& Melee Attack] An unengaged figure may move one hex and
  change facing, and then attempt to Melee Attack.  They may not move
  after they have attacked.

\item[Charge] An unengaged figure may move up to 1/2 TMR and attempt
  to Melee Attack with a non-pole weapon.  At the end of the figure’s
  movement, if there is a hostile figure in the Attacker’s Melee Zone,
  they may make a Melee Attack with a -15 penalty to Strike Chance.
  The figure may not change facing after the Melee Attack.

\item[Charge with Pole Weapon or Shield] An unengaged figure may move
  up to TMR and attempt to Melee Attack with a pole weapon or Shield.
  At the end of the figure’s movement, if there is a hostile figure in
  the Attacker’s Melee Zone, they may make a Melee Attack with a +20
  bonus to Strike Chance.  The figure must move at least 2 hexes, and
  may not change facing after the Melee Attack.  This action may not
  be attempted with a Tower Shield or a Main Gauche.

\item[Charge \& Close] An unengaged figure may move up to 1/2 TMR and
  attempt to Close.  If the figure passes through the Melee Zone of
  the target, the target may try to Repulse the figure in the same way
  as for a Close \& Grapple.

  If the Repulse is successful, the target has interposed their
  weapon between themselves and the Attacker.  If the Attacker cannot
  or will not stop entering Close, the target automatically inflicts a
  potential Specific Grievous Injury on the Attacker.

  If the Repulse is unsuccessful, the Attacker may go into Close, and
  may attempt a normal Grapple action or a Trample attack.

\item[Evade] An unengaged figure may move up to 1/2 TMR and change
  facing while executing an Evade.  If a Melee attack is attempted on
  a figure who is Evading with a Ranked weapon, they may be able to
  Parry the attack.  An Evading figure receives a bonus to Defence
  versus Melee and Ranged Attacks.

\item[Retreat] An unengaged figure may Retreat, increasing their
  defence by 20.  They may move up to 2 hexes backwards and change
  facing.

\item[Pass]An unengaged figure may move two hexes and change facing
  while performing a Pass Action.

\item[Cast] An unengaged figure may not move while attempting to Cast
  a Spell, but may change facing. Like all other actions, Casting is
  resolved on the figure’s initiative. Casting is a Magical Action.

\item[Throw] An unengaged figure may move up to 2 hexes and change
  facing while attempting to Throw a weapon.  They may only Throw into
  their Ranged Zone where they have a Line of Fire.

\item[Fire] An unengaged figure may not move while attempting to Fire
  a missile weapon, but may change facing.  Once a Crossbow is
  prepared and loaded, a figure may carry it around and fire whenever
  they wish. In this instance, the figure may move up to 2 hexes
  either before or after firing.  All missile weapons need to be
  Loaded before they may be Fired.  The figure may only Fire into
  their Ranged Zone where they have a Line of Fire.

\item[Recover from Stun] An unengaged figure who is stunned may
  attempt to recover from Stun.  They may not make any movement or
  change facing.

\end{Description}

\section{Free Acts}

In addition to their normal Action, a figure may always drop whatever
they have in their hands and, if not performing a Magical Action, they
may say a short phrase during their Action.

\section{Action Restrictions}

Movement may be restricted by terrain or other conditions. Figures
with a modified Agility of 8 or less may have their movement reduced
when performing other Actions, while those with a modified Agility of
22 or more may gain extra movement or Actions. Figures who become
stunned or otherwise incapacitated may not attempt normal Actions,
except that stunned figures may attempt to recover from Stun as their
Action.  The type of Action a figure may attempt is restricted by
their position on the Tactical Display, and their visibility. The use
of magic in combat is subject to restrictions, and may in some cases
be impossible.

\subsection{Movement}

Any complicated turning manoeuvre may result in a reduction in the
figure’s movement allowance for that pulse.  A reduction of 1 hex is
suggested for each 180◦ turn. At the end of the Action, the figure
must be unambiguously oriented towards one hex side.

A figure’s movement allowance assumes a flat surface with little or no
hindrance to movement.  Some terrain is not conductive to quick
traversal, and the figure should suffer a reduction to movement in
such conditions.  A figure should normally be able to move at least 1
hex per pulse, no matter what the terrain.

If a figure enters the Melee Zone of any hostile figure, they become
engaged, and must stop movement, though they may change facing.  If
the figure is performing a Charge \& Close, they may attempt to enter
Close, but the opponent’s hex counts as a hex of movement.

If a figure wishes to jump during their movement, they should have
their movement allowance reduced, and the figure must make an Agility
Check to land cleanly.

Often two or more friendly figures will wish to pass through a hex at
the same time, or need to squeeze past each other in the same hex.
The GM should judge whether circumstances permit this, and if so,
whether the figures are hindered.  If neither figure is endeavouring
to do more than move through the hex, there will usually be little
problem, but more dangerous manoeuvres may reduce movement or require
Agility Checks.

A figure may move backwards at half their movement rate, and crawl at
1/4 rate.

\subsection{Pass Actions}

A Pass Action is any generic non-attacking action a figure may attempt
which is not otherwise covered by specific Actions.  Typical Pass
Actions include: preparing an item or weapon, putting an item away,
picking up a dropped item, mounting or dismounting a steed, loading a
missile weapon, drinking from a flask, dropping to one knee or prone,
rising up, etc.

Pass Actions also include Multi-Pulse Actions and the following
Magical Pass Actions:

\begin{Itemize}
  
\item Prepare Spell (see §7.3).

\item Actively Resist (see §7.8).  

\item Concentrate (see §7.6). 

\end{Itemize}

\subsubsection{Typical}

\emph{Actions:} The following list is intended as a guide for the GM
to be able to judge how many pulses an attempted Action will take to
perform.  Note that some Actions that figures in combat wish to
attempt will take far more than one pulse.

\begin{tabularx}{\columnwidth}{Xp{16mm}}
Search for trap in specific place		& 2 \\
Attempt to remove trap				& see §47.2 \\
Quick search of 10’ × 10’ for disguised objects	& 3 \\
Sound Wall					& 1 \\
Pick Lock					& see §47.2 \\
Force Lock					& 3 \\
Spike Door					& 2 \\
Light Torch					& 3 \\
Light Lantern					& 5 \\
\end{tabularx}

\begin{tabularx}{\columnwidth}{Xp{16mm}}
Changing Armour:			& \\ 
Helm						& 1 \\
Leather						& 6 \\
Scale						& 24 \\
Chain						& 12 \\
Plate						& 60 \\
Using a backpack:				& \\
Put on / Take off				& 1 \\
Remove item					& 2 \\
Store item					& 1 \\
Dismount / Mount Horse				& 1 \\
Drink 1/2 pint flask				& 1 \\
Load missile weapon:				& \\
Crossbow					& 2 \\
Crossbow using cranequin			& 3 \\
Other						& note L §56.1 \\
\end{tabularx}


\subsection{Figures with Low Agility}

Figures with modified Agility of 8 or less are allowed one less hex of
movement when executing any of the following Actions: engaged Melee
Attack, engaged Evade, Retreat, Pass, Fire Crossbow, Throw.

\subsection{Figures with High Agility}

Figures with modified Agility of 22 through 25 are allowed one extra
hex of movement when executing any of the following Actions: Melee
Attack, engaged Evade, Withdraw, Retreat, Pass, Fire Crossbow, Throw.

\subsection{Figures with Extraordinary Agility}

Figures with a modified Agility of 26 or more may perform an
additional defensive withdraw, retreat or non-magical pass action
except when Stunned.  The actions are resolved consecutively, in
either order.  The figure’s total movement may not exceed their TMR.

\subsection{Stunned Figures}

A figure who becomes Stunned may only take Recover from Stun as their
Action.  A figure who was Stunned during the Pulse gets an additional
attempt to Recover from Stun at the end of that Pulse.  They may still
take free Acts.  The Base Chance of Recovering from Stun is 2 × WP +
current FT.  A stunned figure has no Melee Zone.

\subsection{Position of Opposing Figures}

The type of Action a figure may attempt is restricted by the position
of the nearest opposing figure.  If a figure is in the same hex as a
hostile figure, they are in Close, and may only select an Action from
§6.5. If a figure is in a hostile figure’s Melee Zone and is not in
Close, they are engaged, and may only select an Action from §6.4.  If
a figure is not in a hostile figure’s Melee Zone, but has a target in
their Melee Zone, they may select an Action from either §6.4 or §6.6,
depending on how they wish to be treated.  Otherwise, a figure is
unengaged, and must select an Action from §6.6.

\subsection{Visibility}

If a figure is attempting to perform a Melee or Ranged Attack on a
hostile figure who occupies a hex that is obscured (due to smoke,
magic, etc.), they may be affected by visibility modifiers (see
§57.1).  If they are attempting to cast a spell that requires
targeting, they must make a Perception Check.  If the figure is
totally obscured, they are treated as invisible for Strike Chances,
and may not normally be targeted by a target: Entity spell.  The GM
must determine if a figure is affected by an Area or Line of Fire
spell, or Ranged Attack.

\subsection{Disturbing Magical Actions}

If an Adept is performing a Magical Action, and is attacked, they must
make a Concentration Check (see §7.6) or their Action will fail.  If
the Adept is stunned or has sufficient cold iron lodged within them,
their Action will automatically fail.  An Adept may not cast while
prone.

\subsection{Action Summary}

The Action Summary (§57.3) lists all valid Actions and their
Restrictions.

\section{Attacking}

The order of all attacking Actions is determined by the Initiative
procedure as detailed in §6.3. Combat involving engaged figures is
always resolved before any combat involving unengaged figures.  An
attacker’s weapon is always assumed to be held in their primary hand
unless stated otherwise.  Empty bare hands are always considered a
prepared weapon.

A hostile figure may be attacked by Ranged, Melee or Close combat
while on the Tactical Display.  Special types of attacks are allowed,
and these include Multi-hex Strikes, Multiple Weapon Strikes, and
attempting to Trip, Entangle, Restrain, Knockout, Shield Rush or
Disarm.

\subsection{Ranged Attacks}

A figure may attempt to attack a hostile figure in their Ranged Zone
via ranged combat by executing a Fire or Throw Action.  The figure
declares their target, determines and applies any Ranged Combat
modifiers (see §57.1), and executes a Strike Check.

To Fire a missile weapon, the figure must be armed with a prepared and
loaded missile weapon.  To Throw a weapon, the figure must be armed
with a prepared weapon rated for ranged combat.  The figure must have
a Line of Fire to the target. If the Line of Fire contains an obscured
hex, the figure may not Aim, and treats the target as if invisible.

Whenever the weapon enters a hex occupies by a figure or object (other
than a solid obstacle that the missile must hit), there is a chance
(as determined by the GM) that the weapon will hit the figure or
object instead of continuing its flight. This must be resolved for
each figure occupying any hex along the Line of Fire until the weapon
hits something or loses momentum and falls to the ground.

A figure cannot check a Line of Fire without executing an Aim, Fire or
Throw action, whether or not the weapon is actually loosed.

\begin{Description}

\item[Snapshooting] A figure with a prepared Short Bow, Long Bow,
  Composite Bow, Giant Bow or Sling, with which they are at least Rank
  3, may prepare an arrow or bullet and Fire in the same Action. The
  Strike Chance is reduced by -15. Snapshooting is a Fire Action.

\item[Aiming] A figure with a prepared and loaded mis- sile weapon may
  choose to take a Pass Action to Aim the missile weapon at a
  particular target. If the figure then Fires at that target in their
  next Action, their Strike Chance is increased by +20, and in
  addition, the chances of causing Endurance or Specific Grievous
  damage are increased to 20\% and 10\% of the modified Strike Chance,
  respectively.

\end{Description}

\subsection{Melee Attacks}

A figure may attempt to Melee Attack any hostile figure who occupies
at least one hex of their Melee Zone.  The figure declares their
target, determines and applies any Melee Combat modifiers (see §57.1),
and executes a Strike Check with a prepared melee-rated weapon. The
attacker may move adjacent to the target during that pulse.

The normal Melee attack is intended to do as much damage to the target
as possible, but other forms of specialised attack exist.

\subsection{Close Combat Attacks}

A figure may attempt to attack any figure who occupies the same hex
via Close Combat by exe cuting a Grapple Action.  The figure declares
their target, determines and applies any Close Combat modifiers (see
§57.1), and executes a Strike Check with a prepared close-rated
weapon.  The attacker may move into the target’s hex during the pulse;
this is known as closing.

\subsection{Attacking into Combat}

A Ranged attack on a figure in Melee combat is resolved normally,
bearing in mind the Line of Sight restrictions.  A Ranged or Melee
attack on a figure in Close combat suffers a penalty of -10.  If the
attack misses, an additional attack with the same penalty must be
resolved against each remaining figure in that hex (in a random
order). If a multi-hex creature is in close with single-hex creatures,
it may be targeted normally.

\subsection{Special Attacks}

A figure may attempt to attack using any one of the following special
attacks.

\begin{Description}
  
\item[Multiple Strike] A figure who is armed with two prepared weapons
  (one in each hand) may attempt a Multiple Strike.  The two weapons
  need not be targeted against the same opponent, but must be of the
  same type (Ranged, Melee or Close).  The Strike Chance of the
  Primary weapon is reduced by -10, while the Strike Chance of the
  Secondary weapon is reduced by -30.  Ambidextrous figures suffer a
  -10 penalty with each attack. A figure may not move while making a
  Multiple Strike.

\item[Multi-hex Strike] A figure who has a prepared two-handed B-class
  weapon, with which they are at least Rank 4, may strike up to three
  figures in adjacent hexes in their Melee Zone.  Their Strike Chance
  is reduced by -20 on each attack. A figure may not move while making
  a Multi-hex Strike.

\item[Trip] A figure with a prepared Quarterstaff, Spear, Halberd,
  Poleaxe or Glaive may attempt to trip an opponent in their Melee
  Zone. The Base Chance is reduced to 40\%, and the damage to D10.  If
  the attack is successful, the target must make a 3 × AG Check or
  fall prone.  This attack may not be attempted on a target
  significantly larger than the attacker. A figure may not move before
  attempting a Trip.

\item[Entangle] A figure with a prepared Net, Whip, Lasso or Bola may
  attempt to Entangle their opponent during any attack.  If the attack
  is successful, the target must make a 3 × AG Check or fall prone.
  The target must disentangle themselves before rising, requiring 2
  Pass Actions.

\item[Restrain] A figure may attempt to restrain an op- ponent by
  pinning them to the ground.  The Base Chance is three times the
  difference in total PS \& AG between the attacker(s) and their
  opponent. No damage is done.  A restrained figure is treated as
  incapacitated, and remains restrained until the restraint is broken
  by an attack from outside the hex that does effective damage to a
  restrainer.  A Restrain may only be attempted in Close Combat.

\item[Knockout] A figure with any prepared Melee rated weapon
  excluding entangling weapons, Lances and Pikes, may attempt to knock
  out their opponent.  The attack is successful if the Strike Check
  would normally result in an Endurance blow (see §6.11).  No damage
  is done, but the target is unconscious for [D + 5] minutes.  This
  attack may not be attempted on a target significantly larger than
  the attacker.  A figure may not move while attempting a Knockout.

\item[Shield Rush] A figure with a prepared shield (other than a Main
  Gauche or Tower Shield) may attempt to Shield Rush their
  opponent. If the attack is successful, the target must make a 3 × AG
  Check or fall prone.  This attack may not be attempted on a target
  significantly larger than the attacker.  A figure must move at least
  one hex before attempting a Shield Rush.

\item[Disarm] A figure may attempt to Disarm an opponent with any
  prepared Melee or Close rated weapon. The Strike Chance of the
  attack is reduced by -20. If the attack is successful, one point of
  EN is inflicted, and the target must roll under (MD + Rank) or drop
  a weapon or item of the attacker’s choice.  If the item is being
  held in two hands, the check is (2 × MD + Rank). A figure may not
  move while attempting a Disarm.

\end{Description}

\section{Resolving Attempted Attacks}

Every weapon and attack form has a Base Chance.  The Base Chance with
all modifiers applied is the Modified Strike Chance.  The attacker
performs a Strike Check by rolling D100; if the result is less than or
equal to the Modified Strike Chance, the attack has been successful;
above and the attack has missed. Particularly poor rolls may result in
the weapon being broken or dropped. Once a successful hit has been
made, a Damage Check occurs.

If the target is Evading, the attacker has a reduced strike chance
and, if they miss, they may be Disarmed or Riposted.

\subsection{Strike Chance}

When attacking with any Ranked weapon, the Strike Chance is (Weapon
Base Chance) + (Mod.  Manual Dexterity) + (4 × Rank).  When attacking
with an unranked weapon, the Strike Chance is equal to the Base
Chance.  Wild creatures using natural attack forms such as teeth,
claws, etc., always add their Manual Dexterity + (4 × Rank).

\subsection{Modified Strike Chance}

An attacker’s Modified Strike Chance is equal to their Strike Chance
plus any modifications for attack type and conditions, minus the
target’s current defence. If the attacker rolls less than or equal to
the Modified Strike Chance, a successful hit has occurred, and a
Damage Check is made (see §6.11).

Attack condition modifiers are detailed in (§57.1 Strike Chance
Modifiers).

\subsection{Evading}

If a figure evades, their Defence against Melee attacks increases by
10 + 4 / Rank of their prepared Melee weapon, and their Defence
against Missile attacks increases by 20.

If a figure is Evading, and an opponent in their Melee Zone misses an
attack at them by 30 or more, they may choose to try to Parry the
attack.

The target rolls D10, adds the Rank of the prepared weapon they are
Evading with, and subtracts the Rank of the attacker’s weapon. If this
result is 3 or less, the attack has been successfully Parried, but the
target has been thrown off balance, and their next action must be a
pass action.  If the modified result is 4 through 7, the target may
Disarm the attacker (see Disarm). If the modified result is 8 or
above, the attack has been Parried and the target may execute a free
Melee Attack on their attacker as well as a Disarm. This is called a
Riposte.

A Riposte cannot itself be Parried, and may occur as many times in the
pulse as the evading target was Melee Attacked. An unarmed figure may
Parry if they are ranked in Unarmed Combat.

\subsection{Defence}

A figure’s defence is subtracted from an attacker’s Strike Chance.
Defence is equal to modified Agility, plus defence afforded by a
prepared shield, defensive manoeuvres and magic.

Defensive advantages due to terrain conditions and visibility
modifiers are covered in §57.1 Strike Chance Modifiers. A figure has
no defence except for that provided by magic if they are stunned or
incapacitated.

A prepared shield provides defence against all Melee and Ranged
attacks that pass through a figure’s front hexes, if they have the
Shield skill.  At Rank 0 and each additional Rank, the defence bonus
(2\% to 6\%, see §56.2 Shields) is added to defence. No bonus is given
for an unranked shield.  A figure may not attack with their shield or
count their shield as a prepared weapon for Evading while retaining
the shield defence bonus.  A prepared Main Gauche also provides some
defence; however defence is only applied against Melee attacks, and no
defence is gained at Rank 0.

\subsection{Fumbles}

An unmodified Strike Check of 00 indicates that the attacker has
fumbled; they lose 10 from their Initiative Value until the end of the
next pulse. This chance of fumbling is increased if the weapon is made
of a material other than cold iron, as listed below, unless it is
magical, or a Bow or Crossbow.

\begin{tabularx}{\columnwidth}{Xl}
any silver or truesilver alloy of iron		& 1\% \\
any other hard metal alloy (e.g. bronze)	& 2\% \\ 
viable weapons made of other materials		& 3\% (or more)* \\
\end{tabularx}

* the actual figure should be specified by the GM at the time of the
weapon’s creation.


When an attacker fumbles, they make a totally unmodified D100 roll.
If that roll is under their current Initiative Value, they suffer no
further penalty for their slight fumble; if it not under their current
Initiative Value, apply the corresponding result from §52.3 or §52.4
(the Fumble tables).

\section{Damage}

A successful Strike Check usually results in a Damage Check being
performed. Each attack has a damage modifier that is applied to a D10
roll, and the result is the number of damage points inflicted by the
attack (minimum damage 1). There are three types of physical damage
possible from a successful strike, depending on how successful the
Strike Check was: Fatigue Damage, Endurance Damage, and Specific
Grievous Injuries.

\subsection{Fatigue Damage}

Physical Damage affecting Fatigue may be absorbed by armour.  Each
type of armour has a Protection Rating (as listed in §56.3 Armour
Chart), which is subtracted from the Fatigue damage inflicted.  When
a figure’s Fatigue reaches 0, any subsequent attacks affecting Fatigue
are subtracted from Endurance instead. A figure normally cannot lose
both Fatigue and Endurance from one Strike Check.

\subsection{Endurance Damage}

A Strike Check of 15\% or less of the Modified Strike Chance results
in damage directly affecting Endurance, and which is never absorbed by
armour.

\subsection{Specific Grievous Injuries}

In addition to Endurance damage, a Specific Grievous Injury may occur
if the Strike Check is 5\% or less of the Modified Strike Chance.  If
a potential Specific Grievous Injury occurs, the attacker rolls D100
and consults the Grievous Injury Table (§51). If the roll falls within
the range specified for the weapon class, a Specific Grievous Injury
has occurred, and the effects of the resulting injury are applied in
combination with any Endur- ance damage inflicted.

A figure who suffers a Grievous Injury while wearing armour has the
Protection Rating of their armour reduced by two until repaired.
Optionally, a figure who is also carrying a shield may choose to have
the shield cloven instead.  A cloven shield is useless.

\subsection{Magical Damage}

All magical damage affects Fatigue unless otherwise states in the
spell description. Spell damage is assumed to be nonphysical, and thus
unaffected by armour, unless the spell explicitly states that it is
affected by armour.  Magical damage that is not affected by armour
never stuns.  Breath weapons are treated as magical damage, but are
Passively Resistible for half damage.

\subsection{Additional Damage}

The damage inflicted with a particular weapon may be increased due to
exceptional Physical Strength or Rank.  Only one of these two
modifiers may be applied at any time.

If a figure chooses to over-strength a weapon, they may inflict an
additional point of damage for every 5 full points of Physical
Strength they have over the minimum required to use the weapon.
Thrown or Missile weapons may not be over-strengthed.  See §6.14 for
Unarmed Combat.

If a figure chooses to apply skill to inflict extra damage, they may
inflict an additional point of damage for every full 4 Ranks they have
in the weapon.  This affects Close, Melee Thrown, and Missile weapons.

\section{Effects of Damage}

\subsection{Missile Lodgement}

When a figure takes effective Endurance Damage from an A-class Missile
or Thrown weapon, the weapon has lodged itself in their body, and
reduces the figure’s Agility by 3 (5 if a pole weapon). The Agility
loss for multi-hex creatures will be reduced in proportion to their
size.  The weapon remains lodged until a Pass Action is taken to
remove it. A barbed arrow lodges if it inflicts any effective damage,
and the figure will take D-4 Fatigue damage when the barbed arrow is
removed unless it is removed by a Healer. Barbed arrows have a Strike
Chance penalty of -25.

\subsection{Stunning}

Whenever a figure suffers effective damage greater than one-third
their full Endurance, they become stunned.

\begin{Itemize}

\item They stop performing any existing Action.  

\item They have no Melee Zone, but remain Engaged as long as they are
  in the Melee Zone of an oppo- nent.

\item Their Initiative changes (see §6.3)  

\item They have no defence except that provided by magic.

\item Any shield or weapon (including unarmed) becomes unprepared.

\item Their only Action which they may attempt is recover from Stun.

\item At the end of the Pulse in which they were stunned, a figure may
  attempt to Recover from Stun.

\item The Base Chance to recover from Stun in 2 × WP + current
  Fatigue.

\item They may not move on the Tactical Display, or change facing.
  They may still perform Free Acts(see §6.7).

\end{Itemize}

\subsection{Massive Damage}

If a figure with positive Fatigue suffers effective Fatigue damage
greater than their combined full Fatigue and Endurance, they lose all
their Fatigue and are reduced to -1 Endurance.  If they suffer more
than their combined full Fatigue and 1.5 × Endurance, they are dead.

\subsection{Unconsciousness}

When a figure’s Endurance reaches 3 or less, they must make a (current
EN) × WP check or fall unconscious; this WP check is repeated every
minute.  A figure on 0 Endurance is unconscious, but stable.  A figure
on negative Endurance will lose one point of Fatigue (Endurance when
no Fatigue remains) until the bleeding is stanched by a Healer, or
until dead. A creature with a full Endurance of 5 or less does not
make consciousness checks.  They remain conscious until they fall to 0
or less Endurance.

\subsection{Death}

When a figure’s Endurance falls below negative one-half their full
Endurance, the figure is dead.  Once dead, further damage may be
inflicted, but no more damage will be inflicted from poison or
bleeding.

\subsection{Infection}

Whenever a figure has had Physical Damage inflicted (or some
particularly nasty form of magical attack), they may have become
Infected.  There is normally a 10\% chance of any wound becoming
infected.  This is increased by (20 + Endurance Damage)\% if any
Endurance damage was inflicted.  Bite, claw and talon attacks, hostile
environmental conditions and poor treatment may further increase the
chance. See §4.7 for more information.

\section{Weapons}

Any instrument used to inflict damage on a figure is called a weapon.
Weapons may include the figure’s hands, feet, teeth, etc. All normal
weapons are listed on the Weapons Chart along with their
characteristics.  The only limits to the number of weapons a character
may have in their possession are the weight and bulk of those weapons;
the GM should disallow any odd or unlikely method of carrying weapons.

\subsection{Normal Weapons}

The Weapons Chart (§56.1) lists all the normal weapons and their
characteristics.

Weight The weight of the weapon in pounds (excluding scabbards,
etc.).

\begin{Description}

\item[Physical Strength] The minimum Physical Strength a figure needs
  to wield the weapon properly; a figure without the required PS does
  1 less point of damage for each point of PS they are below the
  minimum.  A figure may never achieve Rank in a weapon they do not
  have the PS to wield.

\item[Manual Dexterity] The minimum modified Manual Dexterity a figure
  needs to manipulate the weapon properly; a figure without the
  required MD has the Base Chance of the weapon lowered by 5 for every
  point they are below the minimum.  A figure may never achieve Rank
  in a weapon they do not have the MD to manipulate.

\item[Range] The distance (in hexes) which the weapon may be Fired or
  Thrown.

\item[Class] The type of damage done by the weapon: A-class for
  thrusting damage, B-class for slashing damage, and C-class for
  crushing damage.  This is used for determining Specific Grievous
  Injuries.

\item[Use] The range(s) of attack the weapon may be used at: R for
  ranged combat, M for Melee combat, C for Close combat. A weapon may
  not be used at an inappropriate range.

\item[Cost] The standard cost (in Silver Pennies) to buy a typical
  example of the weapon.

\item[Maximum] Rank The highest Rank attainable with the weapon.

\end{Description}
  
\subsection{Unusual Weapons}

A figure may attempt to strike bare-handed (see Unarmed Combat), but
only if one hand is free.  A figure may attempt to use an item not
normally used as a weapon at the GM’s discretion, who assigns Base
Chances, damage modifiers, and so forth.  Makeshift weapons will
generally be no better than a Crude Club.

\subsection{Envenomed Weapons}

If the GM permits, figures may carry and use A-class \& B-class
weapons coated with poison.  At least one point of effective damage
must be done for the poison to affect the target.

When anyone except an Assassin handles an envenomed weapon (§33.2),
they must make a 3 × MD check every time they handle the weapon.  This
includes coating the weapon, preparing or unpreparing the weapon, and
attacking. An envenomed weapon will usually remain effective for 6
hours or until at least one point of effective damage has been
inflicted.

\section{Unarmed Combat}

Any figure may attempt to attack a hostile figure by using their
natural weapons.  For many creatures, this is the only way they may
attack. Unless otherwise specified, all figures receive one Unarmed
attack per pulse without penalty.  Some creatures may be able to
attack more than once (see Bestiary). A figure may achieve Rank with
natural weapons just as they may with any weapon.

The Base Chance for a humanoid to strike with their primary hand is
their modified Agility × 2 plus Physical Strength over 15. The damage
modifier is -4 (+ 1 for every 3 full points of Physical Strength over
15).

Figures with Rank 3 or more Unarmed may kick rather than striking with
their hands, enabling them to attack with their hands full. They may
attempt to Trip with their feet; the normal Unarmed Base Chance and
damage apply.  They may also use a kick as their secondary weapon for
a Multiple Strike Attack.

\section{Multi-hex Figures}

Many figures will occupy more than one hex on the tactical display.
Their size necessitates alterations in the resolution of movement and
combat.

Multi-hex figures have three types of hexes surrounding them: Front,
Rear and Flank.  The exact configuration of Front, Rear and Flank
hexes varies with the size of the figure.  Front and Rear hexes
function in the same way for them as for any other figure.  Figures in
Flank hexes gain a bonus to strike (see §57.1), and are not in the
Multi-hex figure’s Melee Zone, but do not gain the advantages of a
Rear attack.

A multi-hex figure may move in any way so that its head enters any
Front hex, and may move up to its full TMR in this fashion.  At the
end of its move, the figure must be unambiguously oriented towards one
hex vertex.  A reduction of 1 hex is suggested for each 120◦ turn.

A multi-hex figure may freely pivot or move into any hex occupied by a
1-hex figure.  The smaller figure is knocked prone automatically and
the figure may then attempt to trample with a Base Chance of 40\%,
doing (D10 + size of the monster in hexes) damage.  Trampling is
C-class damage.  Subsequent attacks on the prone figure use the
Trample Base Chance and damage in the Bestiary.  A multi-hex figure in
close with smaller figures does not automatically fall prone.

\section{Mounted Combat}

In mounted combat, the TMR of the figure (mount and rider combined) is
that of the mount; the rider may not move at all.  A rider and mount
will occupy the hexes that the mount would normally occupy (as
specified in the Bestiary).

Controlling a mount during combat is dependent on the rider’s
Horsemanship skill.  An inexperi- enced horseman will have an
incredibly difficult time even controlling their mount in a chaotic
melee; it would be better for them to dismount and fight on foot.

\subsection{Action Restrictions}

Almost any action the figure is capable of while standing on the
ground may be performed while mounted.  They may not (1) use a
two-handed weapon, (2) fire a missile weapon or throw a weapon while
moving, (3) use more than one weapon at a time.  These restrictions
are lifted depending on the Horsemanship Rank on the Rider (see §29.2
Horsemanship).  A figure may always use a shield and a one-handed
weapon while mounted.

On a normal mount, the rider will not be able to attack figures
directly in front of them except with a spear (or similar long hafted
weapon) or any Ranged weapon.  A mounted figure may not attempt a
Shield Bash (except against other mounted figures).  However, they may
attempt a Mounted Charge.  A rider may freely mount or dismount when
the mount is stationary, by taking a Pass Action; the difficulty of
dismounting when moving is determined by the GM.

\subsection{Charge}

A Charge on a mount is executed in the same manner as a Charge on foot
except the amount of movement prior to the attack may be greater and
the Charge must be in a straight line (no facing changes allowed).

In addition to the normal charging options, an unengaged mounted
figure may attempt a Mounted Charge.  This requires the mount to move
at least 1/2 TMR without changing facing.  At the end of the figure’s
movement, they may make a Melee Attack with a +20 bonus to Strike
Chance.  If the figure overstrengths the weapon, the Mount’s TMR may
be added to the rider’s Physical Strength.  If using a Lance, the
Mount’s Physical Strength may be used for the purposes of
over-strengthing (§6.11 Additional Damage).

Any act of turning the mount or stopping it after the Charge will
require a Horsemanship Check (see §29.2).  The pulse following any
mounted Charge, the momentum will take the mount past the target to
its full TMR.  Any attempt to turn or stop the mount will occur after
that movement is terminated.  A failed check will result in the mount
continuing on its way.

\section{Aerial Combat}

Whenever an avian (or any other flying entity) is airborne, the
figure’s height above the ground may have to be noted.

\subsection{Combat Ranges}

Hostile figures are regarded as being in adjacent hexes if the Range
between them is less than 10 feet. Hostile figures are in Close Combat
if they are in the same hex and the height difference is 3 feet or
less.  For Ranged \& Magical Combat, the range of weapons \& spells
may be calculated using the following formula:

A2 + B2 = C2 (Pythagorean) where A is the horizontal distance between
the two characters, B is the difference in their altitude, and C is
the range between the figures.

\subsection{Close Combat}

An airborne figure will be pulled from the air and become prone if
their combined PS + AG is less than that of their ground-based
opponent.  Otherwise the airborne figure will remain in flight.  The
ground-based figure may be lifted from the ground if the airborne
figure has sufficient Physical Strength and leverage.

An airborne figure may benefit from making a charge attack by diving
on the target.

\subsection{Casting}

If an Adept is flying and the Adept is in all other ways eligible to
cast a spell (has their hands free, is not out of Fatigue, etc.)  they
may move up to 1/2 (rounded down) of their TMR and attempt to cast the
spell prior to, during or after their movement.  This also applies to
all flying magic-using mon- sters and Adepts with flying mounts.

\section{Aquatic Combat}

Aquatic Combat may take place between figures at different
depths. Refer to the aerial combat section for guidelines.

\subsection{Defence }

\begin{Itemize}
\item Defence caused by natural agility is halved for non-aquatics.

\item Non-magical defence is always halved.  

\item  Magical defences are unaffected. 

\end{Itemize}

\subsection{Weapons}

If the character is on a solid surface then the following applies:

\begin{Itemize}
  
\item A class weapons are unaffected  

\item B \& C class weapons have their non-magical base 
chances and damage halved. 

\end{Itemize}

If the character is floating:  

\begin{Itemize}

\item A class weapons have their non-magical base chances and damage
  halved.  Exceptions are tridents, javelins, spears.

\item B \& C class weapons cannot be used. Exceptions are nets and
  garottes.

\end{Itemize}

Magical bonuses are unaffected. 

Close combat is unaffected but the GM can rule that certain actions
are impossible.

Bows and crossbows must be waterproofed.  The effective range of a
thrown or missile weapon is divided by 10.

No shield rushes are possible with a standard shield because of water
resistance.

Evading defence bonuses are 10\% + 2\% / Rank for prepared B \& C
class weapons.  A class weapons are unaffected.

\section{Magical Combat}

See §7.9 Incorporating Magic into Combat for a summary.

\end{Chapter}
