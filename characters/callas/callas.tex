\documentclass{article}

\usepackage{fontspec}
\usepackage{graphicx}
\usepackage[margin=20mm]{geometry}
\RequirePackage[skip=4pt plus 4pt, indent=0mm]{parskip}
\RequirePackage{tabularray}

\setmainfont{TeX Gyre Pagella}[Scale=0.95,Ligatures=TeX]


\title{Callas Verdicini}
\author{Ross Alexander}
\date{30/6/96 AP}

\begin{document}

\maketitle

\begin{center}
\begin{tabular}{rl}
Name		& \emph{Callas Verdicini} \\
Race		& \emph{Human} \\
Age		& \emph{24} \\
Date of birth	& \emph{October 27 1971 AP} \\
Aspect		& \emph{Autumn Stars Earth} \\
Status		& \emph{Bandit} \\
Birth		& \emph{Ligitimate 4th of 5} \\
College		& \emph{Earth (Druidic)} \\
\end{tabular}
\end{center}

\bigskip

\begin{center}
  \leavevmode
  \includegraphics{callas-01.png}
\end{center}

\bigskip

\begin{center}
\begin{tblr}{colspec={ccccccc}} \hline
\makebox[1.2cm]{PS} & 
\makebox[1.2cm]{MD} & 
\makebox[1.2cm]{AG} & 
\makebox[1.2cm]{MA} & 
\makebox[1.2cm]{WP} & 
\makebox[1.2cm]{EN} & 
\makebox[1.2cm]{FT} \\ \hline
11 & 14 & 19 & 19 & 19 & 22 & 26 \\ \hline \hline
PC  & PB  & HT & WT & HAND & & \\ \hline
24  & 22  & 5'11''& 140 lb & Right & & \\ \hline
\end{tblr}
\end{center}

\section{Description}

\subsection{Physical Apperance}

Callas is a medium height human with waist length blonde hair.  She
has pale blue eyes and lobeless ears.  She is thin and not physically
built at all.  Her age looks about twenty two years old and gives the
impression of having quick movement.  She has ivory skin with a
longish, chiselled face, almost sharp featured.  The face is quite
striking and in general she looks very beautiful (to other humans
anyway).

\subsection{Clothing}

She wears primarily browns and greens with definite woodland hues.
All of her clothes are of very good cloth and bespoke tailored.  For
footware the norm is half height brushed leather boots.  Normally
wears black tights and for the more adventureous party members black
silk under clothes.  Her shirts are long (down to her thighs) and made
of black satin over which she often wears a green or brown dress.  The
dresses are almost always long but not tight around the legs.

\subsection{Equipment and Adornments}

She normally wears two rings on her left hand and a carved bone
pendent.  She always carries a dagger in a belt sheath and when
adventuring often a short sword.  She has a small satchel whoes strap
normally hangs from the shoulder across the body.  It can be clipped
to her belt and the strap removed.  It normally contains healing
potions, herbs any other magical items.

Apart from the above most items are carried in a 40 lb back pack.  It
contains general adventuring equipment like sleeping blankets,
utensils, canteen and alcohol (in the form of aged scotch).  When
there is room and weight permits she also has a 50 foot rope, a tarp
and an oil latern (with oil).


\section{Times Past}

It seems like centuries since I was a young girl living on Alusia
but in reality only been twenty or so years have passed there since I
first started along the this path.  The hills, mountains is too strong
a term, between Ranke, Ormond and Mordeaux are still the deeply wooded
areas of my youth, and hills between Ranke and Ormond are continue to
wild enough still to make banditry a viable living.   However both are
church states and period sweeps my militia of the area meant there was
constant risk so the band could never stay in one place.

My parents had been forced to move south from Flugelheim when I was
just a stripling after my father could no longer take the bitterly
cold northern winters.  I can't recall the name of the village where
my father used to hire out mules to local to transport goods and my
mother used to knit woolens.  It journey certainly took its toll on my
mother, with six children then, but somehow we all made it.  But the
authorities of Ranke were not welcoming to migrant peasants, and so
when bandits captured the mules my father decided it was our fate to
join them.  The hills had many small camps and bands of people, some
outlaws, others just refugees.  We joined one of the larger groups
and my father joined in the raiding to make a poor living.

While most adults seem to always be fretting and worrying, as a child
it was a wonderfully free existance.  The weather was mild enough to
be out all year around and with a father away for much of the time and
a mother busy looking after younger siblings one could spend days
playing in the wilds.  There was always something about to see or
strange people to talk too.

My mother still showed some of the beauty she once possessed in
quantity but the many years of hardship and toil had lined her face
and greyed her hair.  As I child I was beautiful despite the constant
dirt and stains and as a consequence I was doted on.  My father was
always talking about marrying me off into some rich family and
believed that I should be spared of hard work and the dangers of the
raiding parties, least I have an accident and become disfigured.
Instead I was encouraged to learn to write, which I enjoyed, and
persue crafts we though were appropiate to a woman of high station,
which I didn't.  It is always difficult being marked out as a child
and certainly I was treated differently then those around me.  I
found myself being cut off them the other children, even my brothers
and sister to some extent, and I'm sure they resented my special
treatment as much as I did.

I still remember clearly the day my world changed.  I was just
reaching my womanhood and the long dry summer had baked the lands
hard.  With the usual forest harvest decimated and band had been
forced to venture into the low lands of Ormond to pillage what they
could.  I never found out what actually happened. All I can remember
was seeing my oldest brother telling my mother father and Marco had
been killed.  Not that he or any other of the others were in much
better shape. The Urialites came after all with a vengence, pursuing
us ever deeping into the highest and remotest parts. My rudimentary
skill with healing was never enough to do more than enable the injured
to keep the bleeding at bay, and I still recall swearing to myself
that I would never allow myself to be in a position where all your can
is lie down and wait for merciful death.

It was only went the briefest of autumns passed and the fell winter
forced the pursuing templars to withdraw.  The winter also forced the
survivors to head back towards the low country and the western coast
but couldn't face the bitterness that had cast itself over my family.
We had journeyed into an area well away from any passes and well away
from anywhere I had been before.  The forest life had made me hardy
and one bright winter's day I decided to leave the valleys I had been
travelling though and climb the highest peak I could find. All day I
climbed but the daylight was gone as I broke through the treeline.
However that night was a fullmoon, and it the winter soltice was still
to come, so I knew my luck would be in.  Midnight was near when I
reached the summit and my epithany.  There at the absolute peak was a
pool, no more than twenty feet wide, which glowed silver in the
moonlight.  It was unexpected but at the time it failed to make an
impact on my troubled life.  All I remember feeling was that I had
made it all the way to the summit on my own, without the anybody else
helping me, no father to pick me when I fell or mother to dry my tears
when I cut myself on the sharp rocks.  As dawn broke from the east the
first rays of light seemed to wash away the dream of becoming some
rich noble's dutiful wife.  There was only myself and the uncaring
world around me and from then on a felt sundered from human
attachment.  I rejoined the band but from then on I lived a seperate
life, paying my way in the world though my limited magic and healing
abilities.  I remember being treated almost like an outsider, much
like the occasional old woman who had been driven out of her home by
religious zealots. The men were afraid of my magic and the women
jelous of my beauty, still sure I was just biding my time until my
prince handsome would rescue me from destitution.

I was over twenty, and feeling increasingly isolated and trapped, when
news came that a barony to the south had declared itself a duchy.
This would have normally never been of interest to those that live
outside the law and ordered society except that a group of adventurers
had setup a guild in Seagate, the main city of Carzala, and that these
adventurers had grown in reputation to the point where most could tell
you a story about them, however apocryphal.  Few ever had a good word
to say about them although there seemed to no evidence that they were
actually harmful.  The main source of discontent seemed to be the
guild's membership policy, with the continual phase "They'll that any
old scum off the streets.  Don't know their right place, they don't."
The worse the rumours spoke, of accepting witches or dark mages, and
orcs or giants, the more convinced I became this guild was my path to
personel salvation.  That spring I set out south, and reached Seagate
at summer solstice.

\subsection{Life as an Adventurer}

Certainly the first few weeks at the Guild were extremely
difficult.  Whatever ideas I had that I was somehow a woman of the
world were destoyed in the first minute, with orcs constantly learing,
noisy dwarves, arrogant elves and ever disappearing and reappearing
hobbits.  The Guild contained numerous magical colleges and
professions, along with trainers in a vast numbers of weapons and
styles. All my saving went to the membership fee but after half a year
I had finally mastered the basics of my powers and was ready to earn a
living as a hired adventurer.



\end{document}
